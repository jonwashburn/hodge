\documentclass[11pt]{article}

\usepackage[margin=1in]{geometry}
\usepackage{amsmath,amssymb}

\newcommand{\Mass}{\mathrm{Mass}}

\title{\bfseries Response: where SYR / mass convergence is claimed in the manuscript}
\author{}
\date{\today}

\begin{document}
\maketitle

\section*{Answer to Amir Rahnama}

In the current manuscript \texttt{hodge-SAVE-dec-12-handoff.tex}:

\begin{enumerate}
\item \textbf{The constant $c_0$ (mass target) is defined and used explicitly.}

It is introduced in Theorem \texttt{thm:realization-from-almost} as
\[
c_0 \ :=\ \langle A,[\psi]\rangle
\ =\ \int_X m\,\gamma\wedge \psi,
\qquad A=\mathrm{PD}(m[\gamma]),\ \ \psi=\omega^{n-p}/(n-p)!.
\]
This is the cohomology--homology pairing lower bound for any cycle in class $A$.

\item \textbf{SYR is defined explicitly.}

Definition \texttt{def:syr} defines Stationary Young--measure Realizability (SYR) for a cone-valued representative $\beta$ by requiring the existence of a sequence of (stationary) integral cycles whose tangent-plane Young measures converge to a field with barycenter $\widehat\beta(x)$, and whose masses converge to the calibration pairing (i.e.\ to $c_0$).

\item \textbf{Yes: the manuscript claims (and labels) that SYR holds for the cone-valued representatives used in the proof.}

The explicit global claim is Theorem \texttt{thm:automatic-syr} (``Automatic SYR for cone-valued forms''):
for every smooth closed cone-valued $(p,p)$-form $\beta$ representing a rational Hodge class $[\gamma]$, there exist integral cycles $T_k$ with
\[
\partial T_k=0,\qquad [T_k]=\mathrm{PD}(m[\gamma])\ \text{(for one fixed $m$ independent of $k$)},
\]
such that
\[
\Mass(T_k)\to m\int_X \beta\wedge\psi \ (=c_0),
\]
and the tangent-plane Young measures of $T_k$ converge a.e.\ to a measurable field $\nu_x$ supported on complex $(n-p)$-planes with barycenter $\int \xi_P\,d\nu_x(P)=\widehat\beta(x)$.
\end{enumerate}

\medskip
\noindent
\textbf{Isolation of the mass-convergence estimate (referee-facing).}
Following your suggestion, the manuscript now isolates the global ``almost-calibration'' estimate for the \emph{constructed} glued sequence as Proposition \texttt{prop:almost-calibration} (in Step 5):
\[
0\ \le\ \Mass(T_\varepsilon)-\langle T_\varepsilon,\psi\rangle\ \le\ 2\,\Mass(U_\varepsilon)\ \to\ 0,
\]
and since $\langle T_\varepsilon,\psi\rangle=c_0$ by the fixed homology class and $d\psi=0$, this gives $\Mass(T_\varepsilon)\to c_0$ as a single clean quantitative argument.

\medskip
\noindent
\textbf{What is not claimed:} the manuscript does \emph{not} claim a general principle of the form
``stationary $\Rightarrow$ almost-calibrated.''  Instead, the mass convergence to $c_0$ is proved for the specific constructed sequences (and recorded in \texttt{thm:automatic-syr}), and then Theorem \texttt{thm:realization-from-almost} upgrades ``almost-calibrated'' sequences to a calibrated limit current.

\end{document}


