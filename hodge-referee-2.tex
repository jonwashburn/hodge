\documentclass[11pt]{article}

% ---------------- Packages ----------------
\usepackage[margin=1in]{geometry}
\usepackage{amsmath,amssymb,amsthm,amsfonts}
\usepackage{enumitem}
\usepackage{hyperref}

% ---------------- Macros ----------------

\newcommand{\eps}{\varepsilon}
\newcommand{\dd}{\,\mathrm{d}}
\newcommand{\supp}{\operatorname{supp}}

\newcommand{\FF}{\mathcal{F}} % flat norm
\newcommand{\dist}{\operatorname{dist}}
\newcommand{\pd}{\partial}


% ---------------- Packages ----------------
\usepackage{amsmath,amssymb,amsthm,amsfonts,mathrsfs}
\usepackage{fullpage}
\usepackage{enumitem}

% ---------------- Macros (minimal; extend as needed) ----------------
\newcommand{\Q}{\mathbb{Q}}
\newcommand{\R}{\mathbb{R}}
\newcommand{\C}{\mathbb{C}}
\newcommand{\Z}{\mathbb{Z}}
\newcommand{\N}{\mathbb{N}}
\newcommand{\Mass}{\mathbf{M}}
\newcommand{\F}{\mathcal{F}}
\newcommand{\PD}{\mathrm{PD}}

\title{Referee Note (High-Level Audit): ``Unconditional Hodge'' Manuscript}
\author{(Reader's summary of main claims, proof skeleton, and referee-critical gaps)}
\date{24-Dec-25}


% ---------------- Document ----------------
\begin{document}
	
	\begin{center}
		{\Large \textbf{Referee Summary: Key Issues, Claimed Advances, and Closing Strategy}}\\[3pt]
		
	\end{center}
	
	\section*{Claim}
	Here is a broad view of my referee on the new approach Hodge Conjecture problem and some ideas to reduce the unconditional argument to a small set of key steps.

	What is genuinely new in this new approach comparing with the old manuscript is a sharper identification of \emph{where} the logical gaps sit (especially the
	template-to-holomorphic verification needed for global matching/gluing).
	
	\section*{Assumptions}
	\begin{itemize}[leftmargin=18pt]
		\item ``New facts'' means: a new lemma/proposition with complete quantitative proof, or a strict strengthening with proof.
		\item This is a \emph{structural} referee report: it separates what is proved from what is only sketched or deferred.
	\end{itemize}
	
	\section*{Proof sketch (of the manuscript's intended pipeline)}
	The manuscript's intended logical pipeline (as described in the file) is:
	\[
	\text{(Coercivity/cone rep.)} \Rightarrow
	\text{(microstructure templates)} \Rightarrow
	\text{(holomorphic realization of slivers)} \]
	\[ \Rightarrow
	\text{(global gluing / boundary cancellation)} \Rightarrow
	\text{(stationary/positive current)} \]
	\[\Rightarrow\
	\text{(analytic cycle)} \Rightarrow
	\text{(algebraic cycle)}.
	\]
	The referee-critical issue is whether the two ``bridges'' are fully proved:
	\begin{enumerate}[leftmargin=18pt]
		\item \textbf{Template-to-holomorphic bridge:} the holomorphic realizations must preserve the geometric/mass hypotheses
		needed for the combinatorial matching estimates.
		\item \textbf{Local-to-global gluing bridge:} the bookkeeping/mismatch estimates must force global boundary smallness
		in the sense required by the gluing theorem.
	\end{enumerate}
	
	\section*{Ten key items (results/issues/idea/innovation/proof status)}
	Each item below is written as: \emph{(i) manuscript claim/role, (ii) proof status in the document, (iii) referee issue,
		(iv) what would close it}. This list is designed to be \textbf{nonrepetitive}.
	
	\begin{enumerate}[leftmargin=18pt, label=\textbf{(\arabic*)}]
		
		\item \textbf{Core bottleneck trio (global strategy compression).}\\
\emph{Current state:} The manuscript concentrates the unconditional strategy into three main results: Theorem 8.46 (global gluing), Proposition 8.96 (holomorphic corner--exit ``slivers''), and Lemma 8.84 (uniform holomorphic control on each cell).\\
\emph{Issue:} This reduction only works if each of these three results is proved with \emph{uniform} hypotheses and \emph{quantitative} constants (independent of the cell index and of the auxiliary parameters used in the construction).\\
\emph{Suggested fix:} For each of the three results, state a precise input--output theorem with explicit constants and parameter dependence, and then show in the proof that all constants are controlled uniformly in the regime needed downstream.
		
		\item \textbf{Theorem 8.46 (global gluing / LICD elimination point).}\\
\emph{Current state:} This is the step where the local cellwise construction is supposed to produce a single global object by cancelling (or controlling) boundaries across shared faces. The manuscript itself treats this as the main bottleneck.\\
\emph{Issue:} The present argument does not yet give a complete, quantitative proof that the sum of the local pieces has globally small boundary (in a clearly specified norm), i.e.\ that the ``unmatched'' boundary contributions across faces cancel or can be filled with controlled mass.\\
\emph{Suggested fix:} Add a proved face-by-face cancellation estimate: for each shared face, quantify the boundary mismatch and then sum these mismatches to obtain a global bound (mass / flat norm, whichever is used later). Conclude with a clean statement of what boundary smallness is obtained and which constants it depends on.
		
		\item \textbf{Proposition 8.79 (combinatorial mismatch bookkeeping) is coherent but conditional.}\\
\emph{Current state:} As a combinatorial statement, Proposition 8.79 appears consistent \emph{provided} the geometric axioms (G1)--(G2) hold uniformly for the objects being glued.\\
\emph{Issue:} The manuscript does not yet prove (G1)--(G2) for the \emph{holomorphic} slivers it constructs (with uniform constants). Without that verification, Proposition 8.79 cannot be used in the gluing step.\\
\emph{Suggested fix:} Insert a separate lemma/proposition that checks (G1)--(G2) for the constructed holomorphic slivers, including the exact constants and scale separations required by Proposition 8.79, and make explicit any regularity assumptions that are being used.
		
		\item \textbf{The decisive gap: flat templates vs holomorphic realizations.}\\
\emph{Current state:} The manuscript aims to: (i) build flat ``corner--exit'' templates, (ii) realize them by holomorphic complete intersections, and (iii) transfer the required face-incidence and localization properties from the templates to the holomorphic objects.\\
\emph{Issue:} A small-slope/$C^{1}$ graph approximation by itself does \emph{not} automatically preserve the strong incidence properties needed later (e.g.\ which faces are hit, and the localization of boundary traces near the intended vertex-stars). This is exactly where the logical chain can fail.\\
\emph{Suggested fix:} Prove a stability theorem: under an explicit smallness condition (e.g.\ $C^{1}$-closeness on a cube at a fixed scale), the holomorphic realization has (a) the same list of hit faces as the template and (b) boundary trace supported in the intended vertex-star neighborhoods. The statement should include the required scale separation and quantitative constants.
		
		\item \textbf{Proposition 8.95 (holomorphic realization of separated planes) is a sketch with a global-control gap.}\\
\emph{Current state:} Proposition 8.95 is presented as a sketch, and it relies on earlier analytic lemmas (including Lemmas 8.15, 8.16, 8.93). The goal is to construct disjoint holomorphic complete intersections which, on the whole cube $Q$, are single $C^{1}$ graphs over well-separated translated planes and have near-planar mass.\\
\emph{Issue:} The missing link is an explicit argument that upgrades \emph{local} jet/derivative control to a \emph{global} single-sheet $C^{1}$ graph representation on all of $Q$, with uniform bounds (and hence disjointness and mass control).\\
\emph{Suggested fix:} Add a quantitative propagation/extension lemma (e.g.\ via Cauchy estimates, elliptic estimates, and uniform Bergman-scale bounds) that takes the local jet bounds assumed/proved in the cited lemmas and produces a global $C^{1}$ graph bound on $Q$. Then explicitly derive disjointness and the stated mass comparison from that bound.
		
		\item \textbf{Proposition 8.96 (corner--exit holomorphic slivers) inherits the weaknesses of 8.95.}\\
\emph{Current state:} Proposition 8.96 is also labeled as a sketch, and it is intended to follow from Proposition 8.95 plus an ``inheritance'' statement transferring face-exit and mass comparisons from a template to a holomorphic graph.\\
\emph{Issue:} Even if one can show that the intended faces are hit, one still needs quantitative control ruling out \emph{accidental} additional face hits and proving the required boundary trace localization (the strong form of (G1)), and then one must re-check (G2) with the same constants used in the combinatorial step.\\
\emph{Suggested fix:} Strengthen the inheritance statement to a proved proposition with two explicit outputs: (i) a no-accidental-hits criterion and (ii) a localized-trace estimate. Then verify (G1)--(G2) for the constructed corner--exit slivers using those outputs, with the exact constants required later.
		
		\item \textbf{Remark 8.100 / ``microstructure gluing estimate established'' is not acceptable without the missing lemma.}\\
\emph{Current state:} At this point the manuscript indicates that the microstructure/gluing estimate is essentially complete.\\
\emph{Issue:} The argument still depends on an unproved verification step (checking (G1)--(G2) for the actually constructed holomorphic slivers). A remark cannot substitute for that verification when it is needed as a formal input to the global gluing theorem.\\
\emph{Suggested fix:} Replace the remark by a numbered lemma/proposition that explicitly verifies (G1)--(G2) for the constructed corner--exit slivers (or whatever family is used), including a complete proof and the uniform constants used downstream.
		
		\item \textbf{SYR stage is ``classical after gluing'' but only if positivity/type is proved.}\\
\emph{Current state:} After gluing, the manuscript intends to apply standard results in complex geometry/GMT to pass from a stationary positive current to an analytic cycle, and then to an algebraic cycle (e.g.\ via results in the spirit of Siu, King, Harvey--Lawson, and Chow).\\
\emph{Issue:} The key technical point is not yet checked: the manuscript must show that its cone/calibration condition implies the \emph{exact} notion of positivity of type $(p,p)$ (often ``strong'' positivity) required by the holomorphic-chain/analytic-cycle theorems it cites. Without a precise statement and a verified implication, this step remains conditional.\\
\emph{Suggested fix:} (1) State precisely the positivity$\Rightarrow$analytic-cycle theorem being invoked, including all hypotheses (closedness, integrality, local finiteness, and the specific positivity notion). (2) Prove a lemma of the form ``(CPM/cone alignment) $\Rightarrow$ strong positivity of type $(p,p)$'' and explicitly check that all other hypotheses of the cited theorem are satisfied for the glued limit object.
		
		\item \textbf{Flat-norm / filling step must be parameter-free (no hidden regime).}\\
\emph{Current state:} Once a global boundary smallness estimate is available, the manuscript treats the passage to a closed limit (via a filling/correction current with controlled mass) as standard GMT.\\
\emph{Issue:} Any restrictions on parameters (dimension, codimension, or a range of $p$) must be stated and verified. If the filling estimate only holds in a restricted regime, then the proof is not unconditional in the stated generality.\\
\emph{Suggested fix:} Write the exact filling/flat-norm statement used (including the dependence of constants), cite an appropriate source, and then show that the hypotheses hold in the manuscript's setting \emph{without} additional hidden assumptions. In particular, spell out clearly whether any dimension-dependent restriction enters the argument, and if so, how it is removed.
		
		\item \textbf{Writing/structure: make the dependency chain explicit once, then reference it.}\\
\emph{Current state:} The manuscript's main conceptual idea is the pipeline ``microstructure templates $\rightarrow$ holomorphic realizations $\rightarrow$ global gluing $\rightarrow$ analytic/algebraic cycle.''\\
\emph{Issue:} The same dependency chain is repeated in several places, and it becomes difficult to see exactly which inputs are assumed at each step and where each conclusion is first proved.\\
\emph{Suggested fix:} Add one clearly labeled ``Dependency Theorem'' (or schematic implication diagram) that lists the minimal hypotheses and the resulting conclusion. Then replace later repetitions by short forward references to that theorem.
		
	\end{enumerate}
	
	\section*{Missing steps (what a referee would still require)}
	\begin{enumerate}[leftmargin=18pt]
		\item A proved lemma verifying (G1)--(G2) for the \emph{holomorphic} corner-exit slivers (not just the flat templates).
		\item A quantitative global-graph lemma closing the Prop 8.95 sketch: jet control $\Rightarrow$ single-sheet $C^1$ graph on all of $Q$.
		\item A complete global face-cancellation argument closing Theorem 8.46(iv) from the mismatch estimates.
		\item A precise positivity/type implication lemma enabling invocation of the holomorphic-chain theorem, with exact hypotheses.
	\end{enumerate}
	
	\section*{Counterexample search (failure modes to test)}
	\begin{itemize}[leftmargin=18pt]
		\item \textbf{Accidental face hits:} a holomorphic sliver intersects a non-designated face away from the anchor vertex, violating (G1).
		\item \textbf{Multiplicity/orientation mismatch:} adjacent cubes produce traces that do not cancel due to mismatched multiplicities.
		\item \textbf{Heavy-tail mismatch:} ``few'' mismatched indices but carrying disproportionate boundary mass, breaking the $O(h)$ edit estimate.
		\item \textbf{Sheet splitting:} local graphing does not prevent multi-sheet behavior elsewhere in $Q$ without a global argument.
		\item \textbf{Positivity gap:} cone alignment is weaker than strong positivity of type $(p,p)$, invalidating the holomorphic-chain invocation.
	\end{itemize}
	
	\section*{Required references (must be stated precisely in the paper)}
	\begin{itemize}[leftmargin=18pt]
		\item A \textbf{holomorphic chain theorem} of the form ``closed + (strongly) positive rectifiable $(p,p)$-current $\Rightarrow$ holomorphic chain''
		(e.g., King / Harvey--Shiffman-type statements), with exact hypotheses.
		\item \textbf{Chow-type theorem} converting analytic subvarieties in projective manifolds to algebraic cycles.
		\item \textbf{Flat norm / isoperimetric filling / deformation theorems} for currents used in the boundary-correction step.
		\item Any \textbf{Bergman-scale / jet-to-global} analytic estimates used to propagate local holomorphic jet control to uniform control on a cube.
	\end{itemize}
	
	
	
	
	
	
	
	
	
	
	
	
	
	
	
	
	
	
	
	
	

		
		\section*{Claim}
		\begin{enumerate}[leftmargin=2em]
			\item \textbf{Main claim (as stated in the manuscript).}
			For every smooth projective K\"ahler manifold $X$ of complex dimension $n$ and every rational Hodge class
			$\gamma \in H^{2p}(X,\Q)\cap H^{p,p}(X)$, the class $\gamma$ is algebraic (i.e.\ in the $\Q$-span of algebraic cycle classes).
			\item \textbf{Structural intermediate claim.}
			A closed cone-valued $(p,p)$-form representative $\beta$ of $\gamma$ implies an SYR-realizing sequence of integral cycles
			$T_k$ with $\partial T_k=0$, $[T_k]=\PD(m\gamma)$, and a tangent-plane Young-measure barycenter matching $\widehat\beta$;
			then (via a calibrated/positivity theorem) the limit is a positive sum of complex analytic subvarieties, hence algebraic.
		\end{enumerate}
		
		\section*{Assumptions}
		\begin{enumerate}[leftmargin=2em]
			\item $X$ is smooth projective and K\"ahler; a fixed K\"ahler form $\omega$ is chosen.
			\item The argument uses a ``cone of calibrated planes'' and a defect functional $\mathrm{Def}(\cdot)$ measuring distance to that cone.
			\item The global construction (microstructure / template / activation) requires quantitative hypotheses of the form:
			\begin{enumerate}[leftmargin=2em,label=(A\arabic*)]
				\item Many pieces per cell (piece count $\gtrsim h^{-1}$ where mass is non-negligible),
				\item Slow variation of piece counts between adjacent cells,
				\item Local holomorphic realizability of template ``slivers'' with mass matching to a target budget,
				\item Face-level coherence / ``edit regime'' ensuring mismatch is an $O(h)$ fraction on interfaces.
			\end{enumerate}
			\item The signed decomposition step assumes one can write any rational Hodge class
			$\gamma=\gamma^+-\gamma^-$ with $\gamma^\pm$ still rational Hodge classes and $\gamma^-$ a positive rational multiple of $[\omega^p]$.
		\end{enumerate}
		
		\section*{Proof sketch (as the paper appears to intend)}
		\begin{enumerate}[leftmargin=2em]
			\item \textbf{Analytic cone forcing:} establish a calibration--coercivity inequality
			(energy gap controls distance-to-cone defect), and deduce existence of cone-valued representatives
			in the effective case.
			\item \textbf{Local holomorphic multi-sheet construction:} in a Bergman-scale chart, realize the local cone-data
			by disjoint holomorphic complete intersections (``slivers'') that are graphs over calibrated planes
			and with controlled mass error.
			\item \textbf{Cohomology quantization / rounding:} discretize local budgets into integer piece counts and ensure the resulting
			assembled current has homology class $\PD(m\gamma)$ for a fixed integer $m$.
			\item \textbf{Microstructure gluing:} control the interface boundary mismatch in flat norm by a quantitative
			face-edit estimate plus a global weighted sum bound; fill $\partial T^{\mathrm{raw}}$ by a small-mass current $U$
			to obtain a genuine cycle.
			\item \textbf{SYR conclusion:} extract a subsequential limit with tangent-plane Young measure barycenter matching $\widehat\beta$.
			\item \textbf{Promotion to algebraic:} apply a calibrated/positivity theorem (Harvey--Lawson/Siu-type) to conclude the limit
			is a positive sum of complex analytic subvarieties; hence $\gamma$ is algebraic.
			\item \textbf{Unconditional reduction:} for general $\gamma$, use $\gamma=\gamma^+-\gamma^-$ with $\gamma^\pm$ effective, and apply
			the effective case to $\gamma^\pm$.
		\end{enumerate}
		
		\section*{Missing steps (referee-critical items; up to 10)}
		\begin{enumerate}[leftmargin=2em]
			\item \textbf{Global ``microstructure matching'' is not proved at theorem level (MM/edit regime gate).}
			The argument requires a uniform face-by-face mismatch bound (unmatched boundary is an $O(h)$ fraction) to invoke the interface
			flat-norm estimate. As a referee, I need an explicit derivation of MM from the paper's activation / checkerboard / prefix machinery,
			with constants uniform in $h$ and independent of the local charts.
			
			\item \textbf{Absolute-scale / ``no vanishing sliver mass'' lemma (prevents tiny weights).}
			If the template footprint scale can degenerate (even with ``uniform fatness'' in a \emph{shape} sense), then individual pieces can carry
			arbitrarily small mass, and the weighted sum estimates can fail to imply $\F(\partial T^{\mathrm{raw}})=o(m)$.
			A clean lemma must force a lower bound of the form $\Mass(\text{each active piece in a cell})\gtrsim h^{2n-2p}$ (or equivalent)
			whenever it is declared active.
			
			\item \textbf{Exponent/parameter regime barrier in the weighted-sum decay step.}
			The proof route ``$\sum m_i^{(k-1)/k}$ small $\Rightarrow$ $o(m)$'' typically needs a dimension inequality on $k=2n-2p$
			(or a replacement argument that avoids Hölder/packing losses). If the manuscript claims full unconditionality for all $(n,p)$,
			this barrier must be removed or an alternative bound must be given.
			
			\item \textbf{Slow variation is used as an input but must be derived from the rounding/quantization mechanism.}
			The slow-variation inequality for neighboring cell counts is not a cosmetic condition: it is what drives prefix coherence across faces.
			Referee requirement: a precise lemma with hypotheses \emph{already proved earlier} (not circular) and an explicit dependence on
			the Lipschitz bounds of the target density.
			
			\item \textbf{Cohomology quantization: integrality and ``fixed $m$'' must be justified cleanly.}
			Several steps require that the constructed cycles represent $\PD(m\gamma)$ for a single integer $m$ independent of the mesh $h$ (or $k$).
			This needs a transparent argument that rounding/activation does not drift in homology and that error terms are exact boundaries.
			
			\item \textbf{Flat-norm filling $\partial T^{\mathrm{raw}}$ must preserve the ``good'' structure needed later.}
			Filling by Federer--Fleming gives existence of $U$ with $\Mass(U)\lesssim \F(\partial T^{\mathrm{raw}})$, but one must check:
			does $T^{\mathrm{raw}}-\partial U$ remain calibrated/positive enough for the Harvey--Lawson/Siu promotion step, or is positivity
			only recovered in the limit? This must be made explicit.
			
			\item \textbf{Harvey--Lawson/Siu promotion step needs an exact theorem statement and hypotheses.}
			To conclude ``$\psi$-calibrated integral cycle $\Rightarrow$ complex analytic cycle'' (and then algebraic),
			one must cite a precise theorem and verify its assumptions (rectifiability, closedness, positivity/type, integrality, etc.).
			As written, this is often the single most scrutinized bridge.
			
			\item \textbf{Signed decomposition lemma: rationality/integrality issues.}
			Writing $\gamma=\gamma^+-\gamma^-$ with $\gamma^-=c[\omega^p]$ and $c\in\Q_{>0}$ is not automatic unless $[\omega]$ (or a substitute ample class)
			is itself rational/integral and the chosen constant can be taken rational. Referee requirement: make this explicit and avoid using an
			irrational spectral bound directly.
			
			\item \textbf{Local holomorphic sheet construction: dependencies and disjointness.}
			Claims of the form ``construct disjoint holomorphic complete intersections, each a single $C^1$ graph over prescribed planes, with mass matching''
			require (i) a specific approximation theorem (Donaldson/Auroux-type), (ii) a quantitative transversality/disjointness argument, and
			(iii) uniform control on Jacobians on the relevant scale. I am currently \emph{uncertain} whether the paper supplies all three at the
			level referees expect, without hidden smallness assumptions.
			
			\item \textbf{Analytic calibration--coercivity inequality: constants and curvature dependence.}
			The inequality that ``energy gap controls cone defect'' can silently depend on curvature/Weitzenb\"ock terms or on positivity assumptions.
			Referee requirement: state the inequality with all assumptions (metric bounds, K\"ahler identities used, elliptic estimates, etc.)
			and cite a standard source or provide a full proof.
		\end{enumerate}
		
		\section*{Counterexample search (what to stress-test)}
		\begin{enumerate}[leftmargin=2em]
			\item \textbf{Vanishing-footprint failure mode:} construct a sequence of ``fat''-shaped but shrinking footprints inside each cell so that each
			piece mass tends to $0$, yet counts remain large; check whether the manuscript's hypotheses actually exclude this.
			\item \textbf{Interface mismatch accumulation:} build a checkerboard assignment where per-face unmatched mass is only controlled in expectation
			but not uniformly; see whether $\F(\partial T^{\mathrm{raw}})$ can stay $\gtrsim m$.
			\item \textbf{Parameter barrier:} test the weighted sum decay in the borderline codimension-2 case $p=2$ in small complex dimension (e.g.\ $n=3,4$),
			where exponent conditions are tight; verify whether the stated decay remains valid.
			\item \textbf{Signed decomposition rationality:} pick $\gamma$ and a K\"ahler form $\omega$ not representing a rational class; the decomposition
			$\gamma^- = c[\omega^p]$ with $c\in\Q$ may fail unless $\omega$ is chosen differently.
			\item \textbf{Promotion-to-analytic stress test:} verify whether the limit current can be merely semi-calibrated or have diffuse tangent-plane
			Young measure rather than an a.e.\ single complex plane; then Harvey--Lawson/Siu conclusions may not apply directly.
		\end{enumerate}
		
		\section*{Required references (must be cited precisely in the manuscript)}
		\begin{enumerate}[leftmargin=2em]
			\item H.\ Federer, \emph{Geometric Measure Theory} (Federer--Fleming compactness; deformation and isoperimetric filling; flat norm).
			\item H.\ B.\ Lawson, Jr.\ \& R.\ Harvey, \emph{Calibrated Geometries} (and follow-ups): calibrated currents and regularity/structure results.
			\item Y.-T.\ Siu: decomposition/structure theorems for positive closed currents; analyticity of Lelong level sets.
			\item J.-P.\ Demailly: regularization of positive currents; analytic methods in algebraic geometry.
			\item S.\ K.\ Donaldson and D.\ Auroux: approximately holomorphic techniques / quantitative transversality producing symplectic or holomorphic-type
			submanifolds with controlled geometry (as used by the local sheet/sliver construction).
			\item Standard GMT/varifold references (Allard; Simon) for Young measures / varifold limits and stationarity implications as invoked.
		\end{enumerate}
		
	
	
	
	
	
\end{document}