\documentclass[11pt,a4paper]{article}
\usepackage[utf8]{inputenc}
\usepackage[T1]{fontenc}
\usepackage{geometry}
\geometry{a4paper, margin=0.75in}
\usepackage{amsmath, amssymb}
\usepackage{hyperref}
\hypersetup{colorlinks=true, linkcolor=blue, urlcolor=blue}
\usepackage{booktabs}
\usepackage{longtable}
\usepackage{array}
\usepackage{xcolor}
\usepackage{rotating}
\definecolor{lightgray}{RGB}{245,245,245}
\definecolor{priorityred}{RGB}{200,50,50}
\definecolor{priorityorange}{RGB}{200,120,50}

\title{\textbf{Hodge Lean Proof: Axiom Completion Roadmap}\\[0.4em]\large What must be proved vs.\ what can remain assumed}
\author{Generated from \texttt{DependencyCheck.lean}}
\date{\today}

\begin{document}
\maketitle

\fcolorbox{black}{lightgray}{\parbox{\dimexpr\textwidth-2\fboxsep-2\fboxrule}{%
\textbf{Purpose.} This document lists the \textbf{current axiom dependencies} of \texttt{hodge\_conjecture'} and classifies them into:
\begin{itemize}
\item \textbf{Must complete} (strategy-critical: likely to contain the conjecture's hard content if left as axioms),
\item \textbf{Can leave (for now)} (classical pillars + interface glue), if your goal is a solid proof \emph{modulo named classical theorems}.
\end{itemize}
\textbf{Note:} This is not an unconditional proof unless \emph{all} axioms are ultimately discharged.
}}

\section{Current axiom list (mechanical)}

Lean reports the following axioms for \texttt{hodge\_conjecture'} (currently 38):

\begin{verbatim}
'hodge_conjecture'' depends on axioms: [
  FundamentalClassSet_isClosed, IsAlgebraicSet, IsAlgebraicSet_empty,
  IsAlgebraicSet_union, calibration_inequality, exists_volume_form_of_submodule_axiom,
  flat_limit_of_cycles_is_cycle, hard_lefschetz_inverse_form,
  harvey_lawson_fundamental_class, harvey_lawson_represents, harvey_lawson_theorem,
  instAddCommGroupDeRhamCohomologyClass, instModuleRealDeRhamCohomologyClass,
  isClosed_omegaPow_scaled, isIntegral_zero_current, isSmoothAlternating_add,
  isSmoothAlternating_neg, isSmoothAlternating_smul, isSmoothAlternating_sub,
  isSmoothAlternating_zero, lefschetz_lift_signed_cycle, limit_is_calibrated,
  microstructureSequence_are_cycles, microstructureSequence_defect_bound,
  microstructureSequence_flat_limit_exists, ofForm_smul_real, ofForm_sub,
  omega_pow_isClosed, omega_pow_represents_multiple, propext, serre_gaga,
  signed_decomposition, simpleCalibratedForm_is_smooth, smoothExtDeriv_add,
  smoothExtDeriv_smul, wirtinger_comass_bound, Classical.choice, Quot.sound]
\end{verbatim}

\section{Axioms you still need to complete (recommended)}

If your goal is a \emph{solid} proof relative to this strategy (i.e.\ not assuming the core bridge from rational Hodge class to algebraic cycle), these are the first axioms to target.

\subsection{P0 (strategy-critical; highest priority)}

\renewcommand{\arraystretch}{1.4}
\small
\begin{longtable}{>{\ttfamily\small}p{3.8cm} >{\ttfamily\footnotesize}p{4cm} p{4.5cm} p{4.5cm}}
\toprule
\textnormal{\textbf{Axiom}} & \textnormal{\textbf{Declared at}} & \textbf{Why it must be completed} & \textbf{What completion means} \\
\midrule
\endhead

signed\_decomposition &
Hodge/Kahler/\newline SignedDecomp.lean:61 &
This is where rationality is turned into a decomposition used to build algebraic cycles. If axiomatized, it can encode most of the conjecture's content. &
Prove as theorem (or replace by a genuinely standard theorem known \emph{not} to imply Hodge). \\
\midrule

microstructureSequence\_are\_cycles &
Hodge/Kahler/\newline Microstructure.lean:228 &
Part of the microstructure pipeline; asserts the constructed approximants are genuine cycles. If axiomatized, it hides the geometric construction. &
Define the construction and prove $\partial = 0$ for each approximant. \\
\midrule

microstructureSequence\_defect\_bound &
Hodge/Kahler/\newline Microstructure.lean:234 &
Controls calibration defect; needed to pass to calibrated limits. If axiomatized, it hides the key analytic estimate. &
Prove the defect estimate from concrete norms/currents. \\
\midrule

microstructureSequence\_flat\_limit\_exists &
Hodge/Kahler/\newline Microstructure.lean:269 &
Provides the convergent subsequence / limit current. If axiomatized, it assumes the compactness/extraction needed by the strategy. &
Prove via a formal Federer--Fleming compactness theorem for your current model. \\
\midrule

harvey\_lawson\_fundamental\_class &
Hodge/Kahler/\newline Main.lean:94 &
Cohomology-level bridge equating the fundamental class of the HL/GAGA output to the target class. This is the exact representation step. &
Prove the de Rham class identification from the definition of cycle/fundamental class. \\
\midrule

lefschetz\_lift\_signed\_cycle &
Hodge/Kahler/\newline Main.lean:150 &
Cycle-level lifting used in the Hard Lefschetz reduction ($p > n/2$). If axiomatized, it assumes compatibility of cycle classes with Lefschetz operator/hyperplane intersection. &
Prove via intersection-with-hyperplane compatibility of cycle class maps. \\

\bottomrule
\end{longtable}
\normalsize

\subsection{P1 (pipeline integrity; classical GMT facts but still assumed here)}

These are standard in GMT once currents/flat topology are fully defined, but are still axioms in this repo. Complete them if you want the analytic limit behavior internal to Lean.

\small
\begin{longtable}{>{\ttfamily\small}p{3.8cm} >{\ttfamily\footnotesize}p{4cm} p{4.5cm} p{4.5cm}}
\toprule
\textnormal{\textbf{Axiom}} & \textnormal{\textbf{Declared at}} & \textbf{Why it matters} & \textbf{What completion means} \\
\midrule
\endhead

limit\_is\_calibrated &
Hodge/Analytic/\newline Calibration.lean:93 &
Needed to ensure the flat limit current is calibrated, so Harvey--Lawson applies. &
Prove from lower semicontinuity of mass + calibration inequality in a concrete current model. \\
\midrule

flat\_limit\_of\_cycles\_is\_cycle &
Hodge/Classical/\newline HarveyLawson.lean:186 &
Needed to ensure the flat limit remains a cycle ($\partial = 0$), another HL hypothesis. &
Prove continuity of boundary in flat norm for your integral current model. \\

\bottomrule
\end{longtable}
\normalsize

\section{Axioms you can leave (if you accept classical pillars)}

If you are comfortable treating major named theorems as axioms, the following can remain assumed while you focus on removing the strategy-critical bridge axioms above.

\subsection{Major classical pillars (deep but standard)}

\small
\begin{longtable}{>{\ttfamily\small}p{4cm} >{\ttfamily\footnotesize}p{4cm} p{8.5cm}}
\toprule
\textnormal{\textbf{Axiom}} & \textnormal{\textbf{Declared at}} & \textbf{Reason it is reasonable to leave (for now)} \\
\midrule
\endhead

hard\_lefschetz\_inverse\_form &
Hodge/Classical/\newline Lefschetz.lean:48 &
Hard Lefschetz / Hodge theory infrastructure; large formalization project. \\
\midrule

serre\_gaga &
Hodge/Classical/\newline GAGA.lean:100 &
GAGA/Chow direction (analytic $\Rightarrow$ algebraic on projective varieties); large AG formalization. \\
\midrule

harvey\_lawson\_theorem &
Hodge/Classical/\newline HarveyLawson.lean:166 &
Harvey--Lawson structure theorem for calibrated currents; deep GMT/complex-analytic theorem. \\
\midrule

harvey\_lawson\_represents &
Hodge/Classical/\newline HarveyLawson.lean:170 &
Companion representation statement for HL conclusion. \\
\midrule

omega\_pow\_represents\_multiple &
Hodge/Kahler/\newline Main.lean:143 &
$\omega^p$ represented by algebraic cycle (complete intersections/hyperplane sections); classical AG fact. \\

\bottomrule
\end{longtable}
\normalsize

\subsection{Interface / glue axioms (engineering layer)}

These provide algebraic/smoothness/linearity properties for the abstract APIs used in the formalization. They are typically discharged only after choosing fully concrete definitions.

\small
\begin{longtable}{>{\ttfamily\small}p{5cm} >{\ttfamily\footnotesize}p{5cm} p{6.5cm}}
\toprule
\textnormal{\textbf{Axiom}} & \textnormal{\textbf{Declared at}} & \textbf{Reason it can be left} \\
\midrule
\endhead

IsAlgebraicSet & Hodge/Classical/GAGA.lean:33 & Interface law; not strategy-critical. \\
IsAlgebraicSet\_empty & Hodge/Classical/GAGA.lean:55 & Interface law; not strategy-critical. \\
IsAlgebraicSet\_union & Hodge/Classical/GAGA.lean:67 & Interface law; not strategy-critical. \\
FundamentalClassSet\_isClosed & Hodge/Classical/GAGA.lean:174 & Interface law; not strategy-critical. \\
omega\_pow\_isClosed & Hodge/Kahler/TypeDecomposition.lean:152 & Interface law; not strategy-critical. \\
isClosed\_omegaPow\_scaled & Hodge/Kahler/TypeDecomposition.lean:160 & Interface law; not strategy-critical. \\
wirtinger\_comass\_bound & Hodge/Analytic/Calibration.lean:36 & Interface law; not strategy-critical. \\
calibration\_inequality & Hodge/Analytic/Calibration.lean:55 & Interface law; not strategy-critical. \\
exists\_volume\_form\_of\_submodule\_axiom & Hodge/Analytic/Grassmannian.lean:70 & Interface law; not strategy-critical. \\
simpleCalibratedForm\_is\_smooth & Hodge/Analytic/Grassmannian.lean:96 & Interface law; not strategy-critical. \\
isIntegral\_zero\_current & Hodge/Analytic/IntegralCurrents.lean:40 & Interface law; not strategy-critical. \\
smoothExtDeriv\_add & Hodge/Basic.lean:246 & Interface law; not strategy-critical. \\
smoothExtDeriv\_smul & Hodge/Basic.lean:252 & Interface law; not strategy-critical. \\
ofForm\_sub & Hodge/Basic.lean:1004 & Interface law; not strategy-critical. \\
ofForm\_smul\_real & Hodge/Basic.lean:1021 & Interface law; not strategy-critical. \\
isSmoothAlternating\_zero & Hodge/Basic.lean:66 & Interface law; not strategy-critical. \\
isSmoothAlternating\_add & Hodge/Basic.lean:69 & Interface law; not strategy-critical. \\
isSmoothAlternating\_neg & Hodge/Basic.lean:72 & Interface law; not strategy-critical. \\
isSmoothAlternating\_sub & Hodge/Basic.lean:78 & Interface law; not strategy-critical. \\
isSmoothAlternating\_smul & Hodge/Basic.lean:75 & Interface law; not strategy-critical. \\
instAddCommGroupDeRhamCohomologyClass & Hodge/Basic.lean:605 & Algebraic structure axiom. \\
instModuleRealDeRhamCohomologyClass & Hodge/Basic.lean:621 & Algebraic structure axiom. \\

\bottomrule
\end{longtable}
\normalsize

\subsection{Lean foundations}

\small
\begin{longtable}{>{\ttfamily\small}p{4cm} p{3cm} p{9.5cm}}
\toprule
\textnormal{\textbf{Axiom}} & \textbf{Declared at} & \textbf{Reason it can be left} \\
\midrule
\endhead

Classical.choice & (Lean core) & Standard classical logic; removing it is a separate (constructive) project. \\
propext & (Lean core) & Standard extensionality principle in Lean/Mathlib classical developments. \\
Quot.sound & (Lean core) & Core quotient principle used by Lean; not a mathematical assumption. \\

\bottomrule
\end{longtable}
\normalsize

\section{Recommended completion order}

\begin{enumerate}
\item \textbf{Discharge P0 axioms} (strategy-critical) so the proof does not assume the core bridge.
\item \textbf{Discharge P1 axioms} if you want the analytic limit behavior internal to Lean.
\item \textbf{Optionally}, begin a long-term project to formalize the classical pillars (Hard Lefschetz, GAGA, Harvey--Lawson).
\end{enumerate}

\end{document}
