\documentclass[11pt,a4paper]{article}
\usepackage[utf8]{inputenc}
\usepackage[T1]{fontenc}
\usepackage{geometry}
\geometry{a4paper, margin=1in}
\usepackage{amsmath, amssymb}
\usepackage{hyperref}
\hypersetup{colorlinks=true, linkcolor=blue, urlcolor=blue}
\usepackage{booktabs}
\usepackage{longtable}
\usepackage{xcolor}
\definecolor{lightgray}{RGB}{245,245,245}
\title{\textbf{Hodge Lean Proof: Axiom Completion Roadmap}\\[0.4em]\large What must be proved vs what can remain assumed}
\author{Generated from \texttt{DependencyCheck.lean}}
\date{\today}
\begin{document}
\maketitle
\fcolorbox{black}{lightgray}{\parbox{\dimexpr\textwidth-2\fboxsep-2\fboxrule}{%
\textbf{Purpose.} This document lists the \textbf{current axiom dependencies of} \texttt{hodge\_conjecture'} and classifies them into:
\begin{itemize}
\item \textbf{Must complete} (strategy-critical: likely to contain the conjecture's hard content if left as axioms),
\item \textbf{Can leave (for now)} (classical pillars + interface glue), if your goal is a solid proof \emph{modulo named classical theorems}. 
\end{itemize}
\textbf{Note:} This is not an unconditional proof unless \emph{all} axioms are ultimately discharged.
}}
\section{Current axiom list (mechanical)}
Lean reports the following axioms for \texttt{hodge\_conjecture'} (currently 38):
\begin{verbatim}
'hodge_conjecture'' depends on axioms: [
 FundamentalClassSet_isClosed,
 IsAlgebraicSet,
 IsAlgebraicSet_empty,
 IsAlgebraicSet_union,
 calibration_inequality,
 exists_volume_form_of_submodule_axiom,
 flat_limit_of_cycles_is_cycle,
 hard_lefschetz_inverse_form,
 harvey_lawson_fundamental_class,
 harvey_lawson_represents,
 harvey_lawson_theorem,
 instAddCommGroupDeRhamCohomologyClass,
 instModuleRealDeRhamCohomologyClass,
 isClosed_omegaPow_scaled,
 isIntegral_zero_current,
 isSmoothAlternating_add,
 isSmoothAlternating_neg,
 isSmoothAlternating_smul,
 isSmoothAlternating_sub,
 isSmoothAlternating_zero,
 lefschetz_lift_signed_cycle,
 limit_is_calibrated,
 microstructureSequence_are_cycles,
 microstructureSequence_defect_bound,
 microstructureSequence_flat_limit_exists,
 ofForm_smul_real,
 ofForm_sub,
 omega_pow_isClosed,
 omega_pow_represents_multiple,
 propext,
 serre_gaga,
 signed_decomposition,
 simpleCalibratedForm_is_smooth,
 smoothExtDeriv_add,
 smoothExtDeriv_smul,
 wirtinger_comass_bound,
 Classical.choice,
 Quot.sound]
\end{verbatim}
\section{Axioms you still need to complete (recommended)}
If your goal is a \emph{solid} proof relative to this strategy (i.e. not assuming the core bridge from rational Hodge class to algebraic cycle), these are the first axioms to target.
\subsection{P0 (strategy-critical; highest priority)}
\begin{longtable}{p{4.2cm}p{3.4cm}p{5.1cm}p{3.3cm}}
\toprule\textbf{Axiom} & \textbf{Declared at} & \textbf{Why it must be completed} & \textbf{What completion means}\\\midrule\endhead
\texttt{signed\_decomposition} & \texttt{Hodge/Kahler/SignedDecomp.lean:61} & This is where rationality is turned into a decomposition used to build algebraic cycles. If axiomatized, it can encode most of the conjecture's content. & Prove as theorem (or replace by a genuinely standard theorem known \textbackslash\{\}emph\{not\} to imply Hodge). \\

\texttt{microstructureSequence\_are\_cycles} & \texttt{Hodge/Kahler/Microstructure.lean:228} & Part of the microstructure pipeline; asserts the constructed approximants are genuine cycles. If axiomatized, it hides the geometric construction. & Define the construction and prove boundary=0 for each approximant. \\

\texttt{microstructureSequence\_defect\_bound} & \texttt{Hodge/Kahler/Microstructure.lean:234} & Controls calibration defect; needed to pass to calibrated limits. If axiomatized, it hides the key analytic estimate. & Prove the defect estimate from concrete norms/currents. \\

\texttt{microstructureSequence\_flat\_limit\_exists} & \texttt{Hodge/Kahler/Microstructure.lean:269} & Provides the convergent subsequence / limit current. If axiomatized, it assumes the compactness/extraction needed by the strategy. & Prove via a formal Federer--Fleming compactness theorem for your current model. \\

\texttt{harvey\_lawson\_fundamental\_class} & \texttt{Hodge/Kahler/Main.lean:94} & Cohomology-level bridge equating the fundamental class of the HL/GAGA output to the target class. This is the exact representation step. & Prove the de Rham class identification from the definition of cycle/fundamental class. \\

\texttt{lefschetz\_lift\_signed\_cycle} & \texttt{Hodge/Kahler/Main.lean:150} & Cycle-level lifting used in the Hard Lefschetz reduction (p>n/2). If axiomatized, it assumes compatibility of cycle classes with Lefschetz operator/hyperplane intersection. & Prove via intersection-with-hyperplane compatibility of cycle class maps. \\

\bottomrule\end{longtable}
\subsection{P1 (pipeline integrity; classical GMT facts but still assumed here)}
These are standard in GMT once currents/flat topology are fully defined, but are still axioms in this repo. Complete them if you want the analytic limit behavior internal to Lean.
\begin{longtable}{p{4.2cm}p{3.4cm}p{5.1cm}p{3.3cm}}
\toprule\textbf{Axiom} & \textbf{Declared at} & \textbf{Why it matters} & \textbf{What completion means}\\\midrule\endhead
\texttt{limit\_is\_calibrated} & \texttt{Hodge/Analytic/Calibration.lean:93} & Needed to ensure the flat limit current is calibrated, so Harvey--Lawson applies. & Prove from lower semicontinuity of mass + calibration inequality in a concrete current model. \\

\texttt{flat\_limit\_of\_cycles\_is\_cycle} & \texttt{Hodge/Classical/HarveyLawson.lean:186} & Needed to ensure the flat limit remains a cycle (boundary=0), another HL hypothesis. & Prove continuity of boundary in flat norm for your integral current model. \\

\bottomrule\end{longtable}
\section{Axioms you can leave (if you accept classical pillars)}
If you are comfortable treating major named theorems as axioms, the following can remain assumed while you focus on removing the strategy-critical bridge axioms above.
\subsection{Major classical pillars (deep but standard)}
\begin{longtable}{p{4.2cm}p{3.4cm}p{8.6cm}}
\toprule\textbf{Axiom} & \textbf{Declared at} & \textbf{Reason it is reasonable to leave (for now)}\\\midrule\endhead
\texttt{hard\_lefschetz\_inverse\_form} & \texttt{Hodge/Classical/Lefschetz.lean:48} & Hard Lefschetz / Hodge theory infrastructure; large formalization project. \\

\texttt{serre\_gaga} & \texttt{Hodge/Classical/GAGA.lean:100} & GAGA/Chow direction (analytic $\textbackslash\{\}Rightarrow$ algebraic on projective varieties); large AG formalization. \\

\texttt{harvey\_lawson\_theorem} & \texttt{Hodge/Classical/HarveyLawson.lean:166} & Harvey--Lawson structure theorem for calibrated currents; deep GMT/complex-analytic theorem. \\

\texttt{harvey\_lawson\_represents} & \texttt{Hodge/Classical/HarveyLawson.lean:170} & Companion representation statement for HL conclusion. \\

\texttt{omega\_pow\_represents\_multiple} & \texttt{Hodge/Kahler/Main.lean:143} & $\textbackslash\{\}omega\textasciicircum{}p$ represented by algebraic cycle (complete intersections/hyperplane sections); classical AG fact. \\

\bottomrule\end{longtable}
\subsection{Interface / glue axioms (engineering layer)}
These provide algebraic/smoothness/linearity properties for the abstract APIs used in the formalization. They are typically discharged only after choosing fully concrete definitions.
\begin{longtable}{p{4.2cm}p{3.4cm}p{8.6cm}}
\toprule\textbf{Axiom} & \textbf{Declared at} & \textbf{Reason it can be left}\\\midrule\endhead
\texttt{IsAlgebraicSet} & \texttt{Hodge/Classical/GAGA.lean:33} & Defines the minimal interface laws needed by the proof; not expected to be strategy-critical. \\

\texttt{IsAlgebraicSet\_empty} & \texttt{Hodge/Classical/GAGA.lean:55} & Defines the minimal interface laws needed by the proof; not expected to be strategy-critical. \\

\texttt{IsAlgebraicSet\_union} & \texttt{Hodge/Classical/GAGA.lean:67} & Defines the minimal interface laws needed by the proof; not expected to be strategy-critical. \\

\texttt{FundamentalClassSet\_isClosed} & \texttt{Hodge/Classical/GAGA.lean:174} & Defines the minimal interface laws needed by the proof; not expected to be strategy-critical. \\

\texttt{omega\_pow\_isClosed} & \texttt{Hodge/Kahler/TypeDecomposition.lean:152} & Defines the minimal interface laws needed by the proof; not expected to be strategy-critical. \\

\texttt{isClosed\_omegaPow\_scaled} & \texttt{Hodge/Kahler/TypeDecomposition.lean:160} & Defines the minimal interface laws needed by the proof; not expected to be strategy-critical. \\

\texttt{wirtinger\_comass\_bound} & \texttt{Hodge/Analytic/Calibration.lean:36} & Defines the minimal interface laws needed by the proof; not expected to be strategy-critical. \\

\texttt{calibration\_inequality} & \texttt{Hodge/Analytic/Calibration.lean:55} & Defines the minimal interface laws needed by the proof; not expected to be strategy-critical. \\

\texttt{exists\_volume\_form\_of\_submodule\_axiom} & \texttt{Hodge/Analytic/Grassmannian.lean:70} & Defines the minimal interface laws needed by the proof; not expected to be strategy-critical. \\

\texttt{simpleCalibratedForm\_is\_smooth} & \texttt{Hodge/Analytic/Grassmannian.lean:96} & Defines the minimal interface laws needed by the proof; not expected to be strategy-critical. \\

\texttt{isIntegral\_zero\_current} & \texttt{Hodge/Analytic/IntegralCurrents.lean:40} & Defines the minimal interface laws needed by the proof; not expected to be strategy-critical. \\

\texttt{smoothExtDeriv\_add} & \texttt{Hodge/Basic.lean:246} & Defines the minimal interface laws needed by the proof; not expected to be strategy-critical. \\

\texttt{smoothExtDeriv\_smul} & \texttt{Hodge/Basic.lean:252} & Defines the minimal interface laws needed by the proof; not expected to be strategy-critical. \\

\texttt{ofForm\_sub} & \texttt{Hodge/Basic.lean:1004} & Defines the minimal interface laws needed by the proof; not expected to be strategy-critical. \\

\texttt{ofForm\_smul\_real} & \texttt{Hodge/Basic.lean:1021} & Defines the minimal interface laws needed by the proof; not expected to be strategy-critical. \\

\texttt{isSmoothAlternating\_zero} & \texttt{Hodge/Basic.lean:66} & Defines the minimal interface laws needed by the proof; not expected to be strategy-critical. \\

\texttt{isSmoothAlternating\_add} & \texttt{Hodge/Basic.lean:69} & Defines the minimal interface laws needed by the proof; not expected to be strategy-critical. \\

\texttt{isSmoothAlternating\_neg} & \texttt{Hodge/Basic.lean:72} & Defines the minimal interface laws needed by the proof; not expected to be strategy-critical. \\

\texttt{isSmoothAlternating\_sub} & \texttt{Hodge/Basic.lean:78} & Defines the minimal interface laws needed by the proof; not expected to be strategy-critical. \\

\texttt{isSmoothAlternating\_smul} & \texttt{Hodge/Basic.lean:75} & Defines the minimal interface laws needed by the proof; not expected to be strategy-critical. \\

\texttt{instAddCommGroupDeRhamCohomologyClass} & \texttt{Hodge/Basic.lean:605} & Defines the minimal interface laws needed by the proof; not expected to be strategy-critical. \\

\texttt{instModuleRealDeRhamCohomologyClass} & \texttt{Hodge/Basic.lean:621} & Defines the minimal interface laws needed by the proof; not expected to be strategy-critical. \\

\bottomrule\end{longtable}
\subsection{Lean foundations}
\begin{longtable}{p{4.2cm}p{3.4cm}p{8.6cm}}
\toprule\textbf{Axiom} & \textbf{Declared at} & \textbf{Reason it can be left}\\\midrule\endhead
\texttt{Classical.choice} & (Lean core) & Standard classical logic; removing it is a separate (constructive) project. \\

\texttt{propext} & (Lean core) & Standard extensionality principle in Lean/Mathlib classical developments. \\

\texttt{Quot.sound} & (Lean core) & Core quotient principle used by Lean; not a mathematical assumption. \\

\bottomrule\end{longtable}
\section{Recommended completion order}
\begin{enumerate}
\item Discharge P0 axioms (strategy-critical) so the proof does not assume the core bridge.
\item Discharge P1 axioms if you want the analytic limit behavior internal to Lean.
\item Optionally, begin a long-term project to formalize the classical pillars (Hard Lefschetz, GAGA, Harvey--Lawson).
\end{enumerate}
\end{document}