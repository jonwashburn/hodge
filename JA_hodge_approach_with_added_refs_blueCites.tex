\pdfoutput=1
\documentclass[11pt]{article}

% ---------- Packages ----------
\usepackage[utf8]{inputenc}
\usepackage[T1]{fontenc}

\usepackage{amsmath, amssymb, amsfonts, amsthm}
\usepackage{mathtools}
\usepackage{mathrsfs}
\usepackage{bm}
\usepackage{geometry}
\usepackage{graphicx}
\usepackage[protrusion=true,expansion=false]{microtype}
\geometry{margin=1in}

% Hyperref should generally be loaded last
\usepackage[dvipsnames]{xcolor}
\newcommand{\REV}[1]{\textcolor{blue}{#1}}
\newcommand{\REVMZ}[1]{\textcolor{teal}{#1}}  % Milan Zlatanović feedback round

\usepackage[hypertexnames=false,hidelinks]{hyperref}
\usepackage{xurl}

% ==========================================================
% Theorem Environments
% ==========================================================
\numberwithin{equation}{section}  % (1.1), (1.2), etc

\theoremstyle{plain}
\newtheorem{theorem}{Theorem}[section]
\newtheorem{conjecture}[theorem]{Conjecture}
\newtheorem{lemma}[theorem]{Lemma}
\newtheorem{proposition}[theorem]{Proposition}
\newtheorem{corollary}[theorem]{Corollary}
\newtheorem{hypothesis}[theorem]{Hypothesis}

\theoremstyle{definition}
\newtheorem{definition}[theorem]{Definition}
\newtheorem{example}[theorem]{Example}

\theoremstyle{remark}
\newtheorem{remark}[theorem]{Remark}

% ==========================================================
% Macros / Notation
% ==========================================================

\DeclareMathOperator{\spt}{spt}
\DeclareMathOperator{\Lip}{Lip}

% Basic sets
\newcommand{\R}{\mathbb{R}}
\newcommand{\C}{\mathbb{C}}
\newcommand{\Z}{\mathbb{Z}}
\newcommand{\Q}{\mathbb{Q}}
\newcommand{\N}{\mathbb{N}}

\newcommand{\RR}{\mathbb{R}}
\newcommand{\CC}{\mathbb{C}}
\newcommand{\ZZ}{\mathbb{Z}}
\newcommand{\QQ}{\mathbb{Q}}

\newcommand{\CP}{\mathbb{CP}}
\newcommand{\PP}{\mathbb{P}}

% Small notation
\newcommand{\eps}{\varepsilon}
\newcommand{\mhol}{m_{\mathrm{hol}}} % holomorphic/Bergman tensor-power parameter
\newcommand{\ome}{\omega}
\newcommand{\del}{\partial}

\newcommand{\dd}{\mathrm{d}}
\newcommand{\dr}{\mathrm{d}}
\newcommand{\vol}{\mathrm{vol}}
\newcommand{\dvol}{\mathrm{dvol}}    % volume form symbol, e.g. \dvol_\omega

% Script letters
\newcommand{\calH}{\mathcal{H}}
\newcommand{\calO}{\mathcal{O}}
\newcommand{\calC}{\mathcal{C}}
\newcommand{\calK}{\mathcal{K}}
\newcommand{\calU}{\mathcal{U}}
\newcommand{\calV}{\mathcal{V}}
\newcommand{\calB}{\mathcal{B}}
\newcommand{\calG}{\mathcal{G}}

% Blackboard bold misc
\newcommand{\bP}{\mathbb{P}}
\newcommand{\bE}{\mathbb{E}}
\newcommand{\bB}{\mathbb{B}}

% Inner product and norm
\newcommand{\inner}[2]{\left\langle #1, #2 \right\rangle}
\newcommand{\norm}[1]{\left\lVert #1 \right\rVert}

% Linear-algebraic operators
\newcommand{\Id}{\mathrm{Id}}
\newcommand{\tr}{\mathrm{tr}}
\newcommand{\HS}{\mathrm{HS}}         % Hilbert--Schmidt label for norms
\newcommand{\proj}{\mathrm{proj}}     % orthogonal projection

\DeclareMathOperator{\End}{End}
\DeclareMathOperator{\Herm}{Herm}
\DeclareMathOperator{\diag}{diag}
\DeclareMathOperator{\Vol}{Vol}
\DeclareMathOperator{\Mass}{Mass}
\DeclareMathOperator{\M}{M}
\DeclareMathOperator{\Span}{span}
\DeclareMathOperator{\diam}{diam}

% Geometry / Grassmannians
\newcommand{\Gr}{\mathrm{Gr}}
\newcommand{\Kah}{\mathrm{K\ddot{a}hler}}

\newcommand{\net}{\mathrm{net}}
\newcommand{\dist}{\mathrm{dist}}

% Harmonic / primitive notation
\newcommand{\harm}{\mathrm{harm}}
\newcommand{\gharm}{\gamma_{\harm}}
\newcommand{\prim}{\mathrm{prim}}

% --- Calibration defect & cone distance ---
\newcommand{\Def}{\mathrm{Def}}
\newcommand{\cone}{\mathrm{cone}}

\newcommand{\Defcone}{\Def_{\cone}}          % global calibrated cone defect
\newcommand{\distcone}{\dist_{\cone}}        % pointwise distance to calibrated cone

% --- K\"ahler calibration form ---
\newcommand{\varphiK}{\varphi}               % symbolic calibration name
\newcommand{\calib}{\omega^{p}/p!}           % actual calibration definition
\newcommand{\calibform}{\frac{\omega^{p}}{p!}} % same, but as a proper fraction

% --- Calibrated Grassmannian (K\"ahler case) ---
% We will write \Gp(x) for the calibrated Grassmannian at x
\newcommand{\Gp}{G_p}

% --- Parallel calibration notation (Section 11) ---
\newcommand{\distPhi}{\dist_{\Phi}}
\newcommand{\DefPhi}{\Def_{\Phi}}
\newcommand{\Clin}{C_{\mathrm{lin}}}         % C_lin(\Phi) used as \Clin(\Phi)

% Define colors





% ==========================================================
% ==========================================================
% Title & Author Info
% ==========================================================

\title{\bfseries Calibration--Coercivity and the Hodge Conjecture:\\
	A Quantitative Analytic Approach}

\author{
    Jonathan Washburn\thanks{Recognition Science, Recognition Physics Institute,
        Austin, Texas, USA. Email: \texttt{jon@recognitionphysics.org}.}
    \and
    Amir Rahnamai Barghi\thanks{Recognition Science, Recognition Physics Institute,
    Austin, Texas, USA. Corresponding author.
        Email: \texttt{arahnamab@gmail.com}.}
}

\date{\today}
\emergencystretch=2em
\setlength{\parindent}{1.2em}
\setlength{\parskip}{0pt}
\setlength{\emergencystretch}{2em}
\begin{document}
	\maketitle

\begin{abstract}
\REVMZ{Let $X$ be a smooth complex projective manifold of complex dimension $n$. 
We introduce a calibration-theoretic approach to the Hodge conjecture that reformulates the existence of rational Hodge classes as a problem of constructing sequences of integral currents with vanishing calibration defect.

Given a rational Hodge class $\gamma \in H^{2p}(X,\Q) \cap H^{p,p}(X)$, the approach reduces the conjecture to a realization statement for smooth closed strongly positive $(p,p)$-forms. Choosing $N \gg 1$, we decompose $\gamma = \gamma^+ - \gamma^-$, where $\gamma^+ := \gamma + N[\omega^p]$ admits a smooth closed strongly positive representative, while $\gamma^- := N[\omega^p]$ is algebraic, being represented by a complete intersection.

For such a cone-positive class $\gamma^+$ with representative $\beta$, we construct, for a fixed integer $m$, a sequence of integral cycles $T_k$ in the class $\mathrm{PD}(m[\gamma^+])$ whose calibration defects tend to zero and whose masses converge to the corresponding cohomological lower bound. In particular, any calibrated limit of this sequence yields an effective integral cycle associated with $m[\gamma^+]$.

The algebraic realization of $\gamma$ is then obtained by subtracting the complete-intersection contribution and reducing to the range $p \le n/2$ via the Hard Lefschetz theorem. The final passage from analytic to algebraic cycles relies on Chow's theorem and GAGA \cite{Hartshorne77,GH78,Serre56,Chow49}.}
\end{abstract}
\noindent\textbf{2020 Mathematics Subject Classification.}
\noindent Primary 14C30.
\noindent Secondary 14C25.
\\ \newline 

\noindent\textbf{Purpose.} The manuscript uses several global parameters repeatedly.  This short dictionary
records the intended meaning when a symbol is used without an immediate local definition.  If a later
statement explicitly redefines a symbol, that local definition takes precedence.

\section{Notation}\label{sec:notation}
This section lists symbols and standing conventions used throughout; it introduces no new concepts. Local definitions in later sections take precedence.

\medskip
\noindent\textbf{Geometric data.}
\begin{itemize}
\item $X$: smooth complex \emph{projective} manifold, $\dim_\C X=n$.
\item $L\to X$: ample line bundle (polarization); $\omega\in c_1(L)$ is a fixed K\"ahler form representing the Chern class.
\item $p$: codimension parameter; the Hodge class lives in degree $2p$.
\item $\psi$: the K\"ahler (Wirtinger) calibration $\psi:=*\varphi=\omega^{n-p}/(n-p)!$ of type $(n-p,n-p)$, calibrating complex $(n-p)$--planes (see the paragraph ``Let $\varphi=\omega^p/p!$ and let $\psi:=*\varphi$ in the main text).
\end{itemize}


\noindent\textbf{Holomorphic sections / jets.}
\begin{itemize}
\item We reserve $m\in\mathbb{N}$ for the fixed cohomology multiplier in $\mathrm{PD}(m[\gamma])$
(Definition~\ref{def:syr}).  For holomorphic/Bergman/H\"ormander constructions we use the tensor-power parameter
$\mhol\in\mathbb{N}$ (so the intrinsic analytic length scale is $\asymp \mhol^{-1/2}$).
\item $H^0(X,L^{\otimes \mhol})$: complex vector space of global holomorphic sections of the $\mhol$th tensor power.
\item $J_x^k(L^{\otimes \mhol})$: the $k$--jet space at $x$, i.e.\ germs of holomorphic sections of $L^{\otimes \mhol}$
modulo those vanishing to order $k+1$ at $x$. Equivalently,
\[
J_x^k(L^{\otimes \mhol})\ \cong\ \mathcal{O}_X(L^{\otimes \mhol})_x\,/\,\mathfrak{m}_x^{k+1}.
\]
\end{itemize}

\noindent\textbf{Cohomology / cycles.}
\begin{itemize}
\item $\gamma\in H^{2p}(X,\Q)$: a rational Hodge class (viewed in de~Rham or singular cohomology as needed).
\item $\mathrm{PD}(\gamma)$: Poincar\'e dual homology class.
\item $\mathcal{F}(\cdot)$: Federer--Fleming flat norm on integral currents (Definition~\ref{def:flat-norm}).
\item $\partial$: boundary operator on currents.
\end{itemize}



\begin{definition}[Mass of an integral current]\label{def:mass}
\REVMZ{(See Federer \cite[4.1.7]{Fed69}.)}
Fix an integer $\ell\ge 0$.
For an integral $\ell$--current $T$ on $X$, its \emph{mass} is the standard mass norm
\[
\Mass(T)\;:=\;\sup\bigl\{\, T(\eta)\;:\;\eta\in\Omega_c^\ell(X),\ \|\eta\|_\infty\le 1 \,\bigr\},
\]
where $\|\eta\|_\infty$ denotes the comass norm on $\ell$--forms (for any fixed background Riemannian metric on $X$).
Equivalently, if $T$ is represented by a countably $\ell$--rectifiable set $M$ \REVMZ{\cite[4.1.28]{Fed69}}
with integer multiplicity $\theta$ and orientation $\tau$, then
\[
\Mass(T)\;=\;\int_M |\theta|\, d\mathcal{H}^\ell.
\]
\end{definition}


\begin{definition}[Flat norm on integral currents]\label{def:flat-norm}
\REVMZ{(See Federer \cite[4.1.12]{Fed69}.)}
Fix an integer $\ell\ge 0$.
For an integral $\ell$--current $T$ on $X$, the \emph{flat norm} is
\[
\mathcal{F}(T)\ :=\ \inf\Bigl\{\Mass(R)+\Mass(Q)\ :\ T=R+\partial Q,\ 
R\ \text{integral $\ell$--current},\ Q\ \text{integral $(\ell+1)$--current}\Bigr\}.
\]
In particular, if $T$ is an integral $\ell$--cycle, then in any decomposition $T=R+\partial Q$ with $R,Q$ integral
one has $\partial R=0$ automatically.
\end{definition}




\noindent\textbf{Scale parameters (used in the gluing/template constructions).}
\begin{itemize}
\item $m\in\mathbb{N}$: fixed cohomology multiplier in the target class $\mathrm{PD}(m[\gamma])$ (Definition~\ref{def:syr});
$m$ is fixed once and for all.
\item $\mhol\in\mathbb{N}$: holomorphic/Bergman tensor-power parameter for $L^{\otimes \mhol}$ in the local
H\"ormander--Bergman realization; intrinsic analytic scale is $\asymp \mhol^{-1/2}$.
\item $h>0$: mesh size of the cubulation into $2n$--cubes of side length $h$.
\item $s\in(0,h)$: corner--exit translation scale (template footprint scale).  In a cube $Q$ and direction label $j$,
$A_{Q,j}(s)$ denotes the common $\psi$--mass of one template piece; typically $A_{Q,j}(s)\asymp s^{2n-2p}$.
\item $\varepsilon\in(0,1)$: small geometric tolerance (angle/slope/separation).  When passing from an affine corner--exit
template to a holomorphic corner--exit sheet inside a cube of size $h$, we impose the compatibility condition
{$\varepsilon\,h\le s/100$} so the realized sheet inherits the same designated-face footprint
(cf.\ Lemma~\ref{lem:corner-simplex-hits-designated-faces} and Corollary~\ref{cor:holomorphic-corner-exit-inherits}).
\item $\delta\in(0,1)$: bookkeeping tolerance for integer rounding in per--cube mass matching; the rounding loss per cube is
bounded by $\sum_j A_{Q,j}(s)$.
\item $\varrho=\varrho(h)\in(0,1]$: transverse patch radius used in borderline constructions; in the case $p=n/2$ we impose
$\varrho=o(\varepsilon)$ as $h\to 0$.
\end{itemize}



\noindent\textbf{Template-net constants (depend on $(h,\varepsilon_h)$ unless explicitly made uniform).}
\begin{itemize}
\item $\alpha_*(h)$, $\alpha^*(h)$: lower/upper coefficient bounds for the linear template system.
\item $A_*(h)$: bound controlling the size of the coefficient vectors (a conditioning constant).
\item $\Lambda(h)$: Lipschitz/variation constant for the templates as labels vary.
\item $c_0$: fixed universal constant appearing in corner-exit/realization inequalities.
\end{itemize}

\noindent\textbf{Calibration defect.}
\[
\Def_{\mathrm{cal}}(T):=\Mass(T)-\langle T,\psi\rangle,
\]
so ``almost-calibrated'' means $\Def_{\mathrm{cal}}(T)$ is small compared to the relevant scaling.




	\section{Introduction}

We formulate the Hodge problem for a fixed rational $(p,p)$ class on a smooth complex projective manifold and outline the argument proving algebraicity.
The core ingredient is a \emph{realization/microstructure} scheme producing integral cycles whose calibration defect tends to $0$ and whose masses converge to the cohomological lower bound.
Any calibrated limit is a positive combination of complex subvarieties \REVMZ{\cite{King71,HL82}}, hence algebraic on projective manifolds \REVMZ{\cite{Chow49,Serre56}}.
A signed decomposition reduces the general case to the cone--positive case, and Hard Lefschetz \REVMZ{\cite{Voisin02,GH78}} treats the full $p$--range.

\REVMZ{We retain the term ``calibration--coercivity'' for historical continuity. \textbf{Note:} The explicit calibration--coercivity estimate (Section~\ref{sec:cal-coercivity}) provides optional background and is not logically required for the main proof.}



\vspace{0.3cm}

\subsection*{Problem}

Let $X$ be a smooth projective complex variety of complex dimension $n$,
equipped with a K\"ahler form $\omega$ in the Chern class of a fixed ample line bundle $L\to X$ (i.e.\ $[\omega]=c_1(L)$).  Fix an integer $1 \leq p \leq n$ and a
rational Hodge class
\[
\gamma \;\in\; H^{2p}(X,\Q) \cap H^{p,p}(X).
\]
The Hodge problem asks whether there exists an algebraic cycle $Z$ of
codimension $p$ whose cohomology class satisfies
\[
[Z] = \gamma \in H^{2p}(X,\Q).
\]
Equivalently, the problem is to decide whether every rational $(p,p)$ class on a
smooth complex projective manifold admits an algebraic cycle representative.
This is the classical Hodge conjecture for the class $\gamma$.

\REV{See, for example, \cite{Voisin02,Lewis99} for background and survey material on the Hodge conjecture; for foundational perspectives on algebraic cycles, see \cite{Kleiman68,Grothendieck69,Fulton98}.}

\subsection*{Route via calibration and energy}

Set the K\"ahler calibration \REVMZ{(see Harvey--Lawson \cite{HL82} for calibrated geometry)}
\[
\varphi := \frac{\omega^{p}}{p!}.
\]
For any smooth closed $2p$--form $\alpha$ representing the class $[\gamma]$, define
its Dirichlet energy
\[
E(\alpha) := \int_{X} \|\alpha\|^{2}\, d\mathrm{vol}_{\omega}.
\]
Let $\gamma_{\harm}$ denote the $\omega$--harmonic representative of $[\gamma]$.

To measure the pointwise misalignment of $\alpha$ from the \emph{strongly positive} calibrated cone
$K_{p}(x)$ associated to $\varphi$, define the pointwise cone distance
\[
\dist_{\cone}(\alpha_{x})
:=
\inf_{\beta_x\in K_p(x)}\|\alpha_x-\beta_x\|.
\]
The global cone defect is then
\[
\Def_{\cone}(\alpha)
:=
\int_{X} \dist_{\cone}(\alpha_{x})^{2}\, d\mathrm{vol}_{\omega}.
\]

This functional quantifies, in an $L^{2}$ sense, how far a closed
representative $\alpha$ lies from the K\"ahler calibrated cone.  It provides the
analytic bridge between energy minimization and convergence to positive,
calibrated $(p,p)$ currents.

\subsection*{Main quantitative theorem (calibration--coercivity, explicit)}

\begin{remark}
\label{rem:spine-theorem-interpretation}[Calibration--coercivity (preview)]\label{thm:cal-coercivity-intro}
\REVMZ{\textbf{Optional background:}} the rigorous statement and proof are given in Theorem~\ref{thm:cal-coercivity}.
Assume the $\omega$--harmonic representative satisfies $\gamma_{\harm}(x)\in K_p(x)$ for all $x\in X$.
Then for every smooth closed $2p$--form $\alpha \in [\gamma]$,
\[
E(\alpha) - E(\gamma_{\harm})\ \ge\ \Def_{\cone}(\alpha).
\]
\end{remark}


This inequality asserts that the Dirichlet energy gap above the harmonic
representative uniformly controls the global calibration defect of $\alpha$, and
thus links energy minimization quantitatively to geometric alignment with the
K\"ahler calibrated cone.

\subsection*{Consequences for Hodge: cone--positive classes}

For \emph{cone--positive} classes $\gamma$---those admitting a smooth closed cone-valued
representative $\beta$ with $\beta(x) \in K_p(x)$---the microstructure/gluing theorem
recorded in \REVMZ{Proposition~\ref{prop:glue-gap}} produces fixed-class integral
cycles $T_k$ with $\Mass(T_k)\to c_0$ (equivalently, $\Mass(T_k)-\langle T_k,\psi\rangle\to 0$).
By \REVMZ{Theorem~\ref{thm:realization-from-almost}}, a subsequential limit is a $\psi$--calibrated integral current; Harvey--Lawson \REVMZ{\cite{HL82}} then identifies it as a positive sum of complex analytic subvarieties, hence algebraic on projective $X$ by Chow/GAGA \REVMZ{\cite{Chow49,Serre56}}.

\subsection*{Consequences for Hodge: general classes via signed decomposition}

For a general rational Hodge class $\gamma$, the harmonic representative
$\gamma_{\mathrm{harm}}$ need not be cone-valued.  The key observation is that
every such $\gamma$ admits a \emph{signed decomposition}
\[
\gamma = \gamma^{+} - \gamma^{-},
\]
where both $\gamma^{+}$ and $\gamma^{-}$ are cone--positive (in the smooth cone sense).  Specifically:
\begin{itemize}
\item $\gamma^{-} := N[\omega^{p}]$ is already algebraic (represented by
complete intersections of hyperplane sections \REVMZ{\cite{GH78,Hartshorne77}}).
\item $\gamma^{+} := \gamma + N[\omega^{p}]$ becomes cone-valued for $N$
sufficiently large, since the K\"ahler form $\omega^{p}$ is strictly positive
in the calibrated cone.
\end{itemize}

Applying the cone--positive machinery to $\gamma^{+}$ yields an algebraic
cycle $Z^{+}$.  Combined with the algebraic cycle $Z^{-}$ representing
$\gamma^{-}$, we obtain
\[
\gamma = [Z^{+}] - [Z^{-}],
\]
proving that $\gamma$ is algebraic.  The signed decomposition is an complete reduction:
it reduces the general case to proving algebraicity for cone--positive classes via the
realization/microstructure step.

\subsection*{Contributions and proof outline}

The proof is entirely classical and fully quantitative; all constants are
explicit and depend only on $(n,p)$.  In particular:

\begin{itemize}
	\item An $\varepsilon$--net on the calibrated Grassmannian with
	$\varepsilon = \tfrac{1}{10}$ satisfies the explicit covering bound
	\[
	N(n,p,\varepsilon) \le 30^{\,2p(n-p)}.
	\]
	
	\item A cone-to-net distortion factor $K$ may be recorded for comparison with the
	ray/net framework, though the cone-based argument does not require it.
	
	\item A uniform pointwise linear-algebra constant controls the distance to the
	calibrated net in terms of the off-type $(p\pm1,p\mp1)$ components and the
	primitive part of the $(p,p)$ component:
	\[
	C_{0}(n,p) = 2.
	\]
\end{itemize}


These components are included only as optional quantitative background (nets and Hermitian linear algebra).
The main realization/SYR chain does not use them.


\medskip\noindent\textbf{Proof outline.}

The proof has three conceptual steps.

\paragraph{1. Reduction to $p\le n/2$ and to cone--positive classes.}
By Hard Lefschetz \REV{(see, e.g., \cite{Voisin02,GH78,Wells})} (\REVMZ{Remark~\ref{rem:lefschetz-reduction}}), it suffices to treat the range $p\le n/2$.
For a general rational Hodge class $\gamma\in H^{2p}(X,\Q)\cap H^{p,p}(X)$, a signed decomposition
$\gamma=\gamma^+-\gamma^-$ with $\gamma^- = N[\omega^p]$ and $\gamma^+=\gamma+N[\omega^p]$
reduces the problem to showing that \emph{cone--positive} classes (those admitting smooth closed cone-valued representatives) are algebraic.

\paragraph{2. Realization (SYR) for a cone-valued representative.}
Fix a cone--positive class $\gamma^+$ with a smooth closed cone-valued representative $\beta$.
\REVMZ{Section~\ref{sec:realization}} constructs, for a fixed integer $m$, a sequence of integral cycles $T_k$
in the class $\mathrm{PD}(m[\gamma^+])$ such that $\Mass(T_k)-\langle T_k,\psi\rangle\to 0$ (hence $\Mass(T_k)\to m\int_X\beta\wedge\psi$), culminating in the SYR summary theorem (\REVMZ{Theorem~\ref{thm:automatic-syr}}).
The key technical point is the microstructure/gluing estimate $\mathcal F(\partial T^{\mathrm{raw}})=o(m)$ (\REVMZ{Proposition~\ref{prop:glue-gap}}),
which is achieved by holomorphic corner-exit slivers and weighted flat-norm summation on a mesh.

\paragraph{3. Calibrated limit and algebraicity.}
Almost-calibration implies that any flat/varifold limit of the $T_k$ is $\psi$--calibrated \REVMZ{\cite{HL82}}.
By Harvey--Lawson \REVMZ{\cite{King71,HL82}}, the limit is integration along a positive sum of complex analytic subvarieties, hence algebraic on projective $X$ by Chow/GAGA \REVMZ{\cite{Chow49,Serre56}}.
Thus $\gamma^+$ is algebraic; together with algebraicity of $\gamma^-$, this yields algebraicity of $\gamma=\gamma^+-\gamma^-$.

\smallskip\noindent\textbf{Remark on ``coercivity''.} Section~\ref{sec:cal-coercivity} records a coercivity inequality in the special CPM--bridge regime where the harmonic representative is cone-valued;
\REVMZ{this observation provides optional context and is not required for the main proof.}


\subsection*{Scope and remarks}


The analytic estimates are uniform in $(n,p)$.
However, the \emph{microstructure/gluing} scaling regime used to conclude the decisive estimate
$\mathcal F(\partial T^{\mathrm{raw}})=o(m)$ is proved in the range $p\le n/2$
(see \REVMZ{Remark~\ref{rem:weighted-scaling}}).
This is sufficient for the full Hodge statement because, in the projective setting, Hard Lefschetz reduces the Hodge conjecture to $p\le n/2$
(\REVMZ{Remark~\ref{rem:lefschetz-reduction}}), and the case $p>n/2$ is recovered by intersecting with hyperplanes.

On K\"ahler manifolds not assumed projective, the construction yields analytic cycles; algebraicity then requires projectivity of $X$.

All constants are explicit and uniform in $(X,\omega)$.
While some constants (e.g.\ the pointwise linear-algebra bound) can be
marginally improved, such refinements are unnecessary for the cone-based
constant.

The bound $N \le 30^{\,2p(n-p)}$ for the covering number of the calibrated
Grassmannian is convenient but not optimal; any standard packing estimate would
suffice.

\subsection*{Conventions}

All norms and inner products are induced by the K\"ahler metric.  Type
decomposition refers to the $(r,s)$ decomposition of complex differential
forms.  The Lefschetz decomposition into primitive and non-primitive components
is orthogonal with respect to $\omega$.  Weak convergence is taken in the sense
of currents.  Energies and $L^{2}$ norms are over $\R$, while cohomology is
taken over $\Q$ when rationality is required.

\subsection*{Organization}


Sections~2--6 record geometric/analytic background (K\"ahler preliminaries, calibrated Grassmannian geometry, and auxiliary linear algebra on nets and Hermitian models).
Section~\ref{sec:cal-coercivity} records an optional coercivity observation in the CPM--bridge regime (where the harmonic representative is cone-valued).
\REVMZ{Section~\ref{sec:realization}} is the heart of the manuscript: it proves the projective tangential approximation and the microstructure/gluing theorem needed to realize smooth cone-valued forms by holomorphic pieces with vanishing flat-norm boundary (after correction by integral fillings), culminating in the SYR summary theorem (\REVMZ{Theorem~\ref{thm:automatic-syr}}).
Finally, the signed decomposition lemma reduces an arbitrary rational Hodge class to the cone--positive case, and the main theorem follows.


\subsection*{Proof structure}

The overall strategy has three main components:
\begin{enumerate}
\item \textbf{Signed decomposition:} Any $\gamma$ equals $\gamma^{+} - \gamma^{-}$
with $\gamma^{\pm}$ cone--positive.  Here $\gamma^{-} = N[\omega^{p}]$ is already
algebraic.
\item \textbf{Cone--positive $\Rightarrow$ algebraic:} For cone--positive classes,

the realization/SYR construction produces almost-calibrated integral cycles and a calibrated limit current (\REVMZ{Theorem~\ref{thm:automatic-syr}}), which is
algebraic by Harvey--Lawson and Chow/GAGA.

\item \textbf{Conclusion:}
$\gamma = [Z^{+}] - [Z^{-}]$ is algebraic.
\end{enumerate}



\begin{center}
\fbox{\begin{minipage}{0.94\linewidth}
\small
\textbf{Main closure chain (used for \REVMZ{Theorem~\ref{thm:main-hodge}}).}
\begin{enumerate}
\item \textbf{Hard Lefschetz reduction} (\REVMZ{Remark~\ref{rem:lefschetz-reduction}}): reduces the Hodge problem to the range $p\le n/2$.
\item \textbf{Signed decomposition} (\REVMZ{Lemma~\ref{lem:signed-decomp}}): $\gamma=\gamma^+-\gamma^-$ with $\gamma^- = N[\omega^p]$ and $\gamma^+$ cone--positive.
\item \textbf{Algebraicity of $\gamma^-$} (\REVMZ{Lemma~\ref{lem:gamma-minus-alg}}): $[\omega^p]$ is represented by complete intersections, hence $\gamma^-$ is algebraic.
\item \textbf{Microstructure/gluing estimate} (\REVMZ{Proposition~\ref{prop:glue-gap}}): $\mathcal F(\partial T^{\mathrm{raw}})=o(m)$ for the constructed sheet-sum on a mesh (in the range $p\le n/2$; see \REVMZ{Remark~\ref{rem:weighted-scaling}}).
\item \textbf{Mass convergence / almost-calibration} (\REVMZ{Proposition~\ref{prop:almost-calibration}}): for the corrected cycles $T_\epsilon=S-U_\epsilon$ one has
$\Mass(T_\epsilon)-\langle T_\epsilon,\psi\rangle\to 0$ and hence $\Mass(T_\epsilon)\to c_0$ with $c_0=\langle \mathrm{PD}(m[\gamma^+]),[\psi]\rangle$.
\item \textbf{Automatic SYR} (\REVMZ{Theorem~\ref{thm:automatic-syr}}): starting from a smooth closed cone-valued representative $\beta$ of $\gamma^+$, the construction yields fixed-class integral cycles with vanishing calibration defect (hence $\Mass(T_k)\to c_0$).
\item \textbf{Calibrated limit and algebraicity}:
\REVMZ{Theorem~\ref{thm:realization-from-almost}} gives a $\psi$--calibrated integral limit current; Harvey--Lawson identifies it with a positive sum of complex analytic subvarieties, which are algebraic on projective $X$ by \REVMZ{Remark~\ref{rem:chow-gaga}}.
\end{enumerate}

\smallskip
\REVMZ{\textbf{Optional background sections (not required for the main proof):}}
\REVMZ{Sections~\ref{sec:energy-gap}--\ref{sec:linear-algebra} (Hermitian/PSD linear algebra and $\varepsilon$-nets) and Section~\ref{sec:cal-coercivity} (calibration-coercivity) provide context but are not logically required for the main realization/SYR chain.}

\REVMZ{\textbf{Guide to remarks in this paper:} Remarks serve three roles: (1)~\emph{Intuition/context}---providing geometric motivation or historical background; (2)~\emph{Technical clarifications}---explaining parameter choices or proof-branch status; (3)~\emph{Roadmap}---indicating what is used where. Remarks labeled ``optional'' or ``not required'' lie outside the main proof path. The core logical chain consists of the numbered Theorems, Propositions, and Lemmas; remarks never contain essential proof steps.}
\end{minipage}}
\end{center}


\section{K\"ahler Preliminaries}\label{sec:kahler-prelim}

In this section we fix the analytic and geometric setup used throughout \REVMZ{(see \cite{GH78,Wells,Voisin02} for standard references on K\"ahler geometry)}.
Unless stated otherwise, all norms, adjoints, differential operators, and identities are taken with respect to the K\"ahler metric $g(\cdot,\cdot)=\omega(\cdot,J\cdot)$ and the associated volume form $d\mathrm{vol}_\omega=\omega^{n}/n!$.
These conventions will be used systematically in the calibration and current formalisms \REVMZ{\cite{Fed69,HL82}} and in the quantitative estimates appearing later.




% ----------------------------------------------------------
\paragraph{Ambient setting.}
Let $X$ be a smooth projective complex manifold of complex dimension $n$, with
K\"ahler form $\omega$ and integrable complex structure $J$. Fix an ample line bundle $L\to X$ with a Hermitian metric whose curvature form equals $\omega$ (so $[\omega]=c_1(L)\in H^2(X,\Z)$).
We assume $\omega$ lies in the Chern class of a fixed ample line bundle $L\to X$ (so $[\omega]=c_1(L)$).
The associated Riemannian metric is
\[
g(\cdot,\cdot)=\omega(\cdot,J\cdot),
\qquad
d\mathrm{vol}_\omega=\frac{\omega^{n}}{n!}.
\]
Throughout the paper, all pointwise and $L^2$ norms are taken with respect to
$g$ (equivalently,~$\omega$).

% ----------------------------------------------------------
\paragraph{Currents, comass, and mass.}\label{par:currents-mass}
For a smooth real $k$--form $\eta$, its \emph{comass} is
\[
\|\eta\|_{\mathrm{comass}}
:=\sup_{x\in X}\ \sup\Bigl\{|\eta_x(v_1,\dots,v_k)|:\ v_1\wedge\cdots\wedge v_k\ \text{is a unit simple $k$--vector in }(T_xX,g)\Bigr\}.
\]
If $T$ is a $k$--current on $X$, its \emph{mass} is \REVMZ{(see Federer \cite[4.1.7]{Fed69})}
\[
\Mass(T):=\sup\Bigl\{\langle T,\eta\rangle:\ \eta\in C^\infty\Lambda^{k}T^*X,\ \|\eta\|_{\mathrm{comass}}\le 1\Bigr\}.
\]

% ----------------------------------------------------------
\paragraph{Forms, inner products, and energy.}
For $k\ge0$, let $\Lambda^{k}T^{*}X$ denote the bundle of real $k$--forms and
$\Lambda_{\C}^{k}T^{*}X=\Lambda^{k}T^{*}X\otimes\C$ its complexification.
The Hodge star \REVMZ{\cite{GH78,Wells}}
\[
*:\Lambda^{k}T^{*}X\longrightarrow\Lambda^{2n-k}T^{*}X
\]
satisfies
\[
\langle \alpha,\beta\rangle_{x}\,d\mathrm{vol}_\omega
=
\alpha\wedge *\beta,
\]
and the pointwise norm is $\|\alpha\|^{2}=\langle \alpha,\alpha\rangle$.
The $L^{2}$ inner product and norm are
\[
\langle \alpha,\beta\rangle_{L^{2}}
:=
\int_{X}\langle \alpha,\beta\rangle\,d\mathrm{vol}_\omega,
\qquad
\|\alpha\|^{2}_{L^{2}}
:=
\int_{X}\|\alpha\|^{2}\,d\mathrm{vol}_\omega.
\]
For any measurable $2p$--form $\alpha$, the Dirichlet energy agrees with its
$L^{2}$ norm:
\[
E(\alpha)
=
\|\alpha\|^{2}_{L^{2}}
=
\int_{X}\|\alpha\|^{2}\,d\mathrm{vol}_\omega.
\]

% ----------------------------------------------------------
\paragraph{Exterior calculus and Hodge theory.}
Let $d$ be the exterior derivative and $d^{*}$ its formal adjoint.
The Hodge Laplacian is
\[
\Delta = dd^{*}+d^{*}d.
\]
A smooth form $\eta$ is \emph{harmonic} if $\Delta\eta=0$.
Every de~Rham cohomology class on a compact Riemannian manifold has a unique
harmonic representative \REVMZ{(Hodge's theorem; see \cite[Ch.~6]{Wells} or \cite[Thm~5.23]{Voisin02})}.

If $\alpha$ is a smooth closed $k$--form representing a class $[\gamma]$, then
there exists a $(k-1)$--form $\xi$ with $d^{*}\xi=0$ (Coulomb gauge) such that
\[
\alpha=\gharm+d\xi,
\qquad
E(\alpha)-E(\gharm)=\|d\xi\|^{2}_{L^{2}}.
\tag{2}
\]

% ----------------------------------------------------------
\paragraph{Type decomposition.}
Complexifying the cotangent bundle gives
\[
T^{*}X\otimes\C
=
T^{1,0*}X\oplus T^{0,1*}X.
\]
Taking wedge powers yields the $(r,s)$--splitting
\[
\Lambda_{\C}^{k}T^{*}X
=
\bigoplus_{r+s=k}\Lambda^{r,s}T^{*}X.
\]
For a complex form $\alpha$, we write $\alpha^{(r,s)}$ for its $(r,s)$
component.  In particular, any complex $2p$--form decomposes as
\[
\alpha
=
\alpha^{(p+1,p-1)}
+
\alpha^{(p,p)}
+
\alpha^{(p-1,p+1)}.
\]
On a K\"ahler manifold \REVMZ{\cite[Ch.~0]{GH78}},
\[
d=\partial+\bar\partial,
\qquad
\partial:\Lambda^{r,s}\to\Lambda^{r+1,s},
\quad
\bar\partial:\Lambda^{r,s}\to\Lambda^{r,s+1}.
\]
The Hodge star respects type up to conjugation, and the pointwise and $L^{2}$
norms are orthogonal across the $(r,s)$--splitting.

% ----------------------------------------------------------
\paragraph{Lefschetz operators and primitive forms.}
The Lefschetz operator
\[
L:\Lambda_{\C}^{\bullet}T^{*}X\to\Lambda_{\C}^{\bullet+2}T^{*}X,
\qquad
L(\eta)=\omega\wedge\eta,
\]
has $L^{2}$--adjoint $\Lambda$ (contraction with $\omega$).
A form $\eta$ is \emph{primitive} if $\Lambda\eta=0$.

The Lefschetz decomposition \REVMZ{\cite[p.~122]{GH78}} expresses any $(p,p)$--form as an orthogonal sum
\[
\alpha^{(p,p)}=\sum_{r\ge0}L^{r}\eta_{r},
\qquad
\eta_{r}\ \text{primitive}.
\]
We write $(\cdot)_{\prim}$ for the orthogonal projection onto the primitive
subspace.

% ----------------------------------------------------------
\paragraph{K\"ahler identities (used implicitly).}
On a K\"ahler manifold one has the commutator identities \REVMZ{\cite[p.~111]{GH78}}
\[
[\Lambda,\partial]=i\,\bar\partial^{*},
\qquad
[\Lambda,\bar\partial]=-\,i\,\partial^{*},
\]
and their adjoints.
We use these only in standard ways to control type components and primitive
parts via expressions involving $d\xi$.

% ==========================================================
% SECTION 3 --- Calibrated Grassmannian and Pointwise Cone Geometry (Revised)
% ==========================================================

\section{Calibrated Grassmannian and Pointwise Cone Geometry}
\label{sec:calibrated-grassmannian}

We introduce the calibrated Grassmannian of complex $p$--planes and the pointwise cone geometry determined by the K\"ahler calibration \REVMZ{(see Harvey--Lawson \cite{HL82} for the foundational treatment of calibrated geometries)}.
We also record quantitative relations between distance to the calibrated cone, calibration defect, and angular misalignment; these are used later in the coercivity estimates.



\paragraph{Calibrated Grassmannian.}
Fix a point $x\in X$.  
Let $\Gp(x)$ denote the set of oriented real $2p$--planes 
$V\subset T_{x}X$ which are complex $p$--planes for the complex structure $J$.
Equivalently, $\Gp(x)$ is naturally identified with the complex
Grassmannian $G_{\C}(p,n)$ of $p$--dimensional complex subspaces of
$T^{1,0}_{x}X$.  

Given such a $V\in \Gp(x)$, let $\phi_{V}$ be the normalized
calibrated simple $(p,p)$--form associated to $V$, defined by
\[
\phi_{V}\bigl( v_{1},Jv_{1},\ldots,v_{p},Jv_{p} \bigr) = 1
\]
for any orthonormal basis $\{v_{1},\ldots,v_{p}\}$ of $V$.
Thus each $\phi_{V}$ has unit pointwise norm and determines the calibrated
direction corresponding to the holomorphic $p$--plane $V$.

\paragraph{Calibrated cone at a point.}
Let
\[
\varphi \;=\; \calibform.
\]
be the K\"ahler calibration.
Define the (closed, convex) calibrated cone in $\Lambda^{2p}T^{*}_{x}X$ by
\[
\mathcal{C}_{x}
:=
\Bigl\{
\sum_{j} a_{j} \phi_{V_{j}}
\;:\;
a_{j}\ge 0,\;
V_{j}\in \Gp(x)
\Bigr\}.
\]
Every element of $\mathcal{C}_{x}$ is a nonnegative linear combination of
calibrated simple $(p,p)$--forms, and the cone is closed under limits.


\begin{lemma}[Closure of the calibrated cone]\label{lem:calibrated-cone-closed}
For each $x\in X$, the cone $\mathcal{C}_{x}\subset \Lambda^{2p}T_x^*X$ is closed.
In particular, for every $\alpha_x$ the infimum in $\dist(\alpha_x,\mathcal{C}_x)$ is attained.
\end{lemma}


\begin{proof}
Let $\alpha_k\in\mathcal{C}_x$ be a convergent sequence with $\alpha_k\to \alpha$.
By Carath\'eodory's theorem \REVMZ{\cite{Schneider14}} for convex cones in finite-dimensional vector spaces, each $\alpha_k$ admits a representation
\[
\alpha_k=\sum_{j=1}^{M} a_{k,j}\,\phi_{V_{k,j}},
\qquad a_{k,j}\ge 0,\ \ V_{k,j}\in \Gp(x),
\]
where $M=\dim_{\R}\Lambda^{2p}T_x^*X$ (any fixed finite bound suffices).
Each generator has unit norm $\|\phi_{V_{k,j}}\|=1$ and, by the K\"ahler-angle formula,
$\langle \phi_{V},\phi_{W}\rangle\in[0,1]$ for all $V,W\in\Gp(x)$.
Therefore
\[
\|\alpha_k\|^2
=\sum_{i,j}a_{k,i}a_{k,j}\langle\phi_{V_{k,i}},\phi_{V_{k,j}}\rangle
\ \ge\ \sum_{j=1}^{M} a_{k,j}^2,
\]
so the coefficients $\{a_{k,j}\}$ are uniformly bounded (since $\{\alpha_k\}$ converges).
After passing to a subsequence we may assume $a_{k,j}\to a_j\ge 0$ for each $j$.
Since $\Gp(x)\cong G_{\C}(p,n)$ is compact, after further passing to a subsequence we may assume
$V_{k,j}\to V_j\in\Gp(x)$ for each $j$.
By continuity of $V\mapsto \phi_V$ we obtain
\[
\alpha=\lim_{k\to\infty}\alpha_k
=\sum_{j=1}^{M} a_j\,\phi_{V_j}\in\mathcal{C}_x,
\]
so $\mathcal{C}_x$ is closed.  Since $\mathcal{C}_x$ is a closed convex subset of a finite-dimensional inner-product space,
nearest-point projection exists and the distance infimum is attained.
\end{proof}


We write
\[
\distcone(\alpha_{x})
:=
\dist\!\bigl(\alpha_{x},\mathcal{C}_{x}\bigr)
\]
for the pointwise distance (with respect to the $g$--norm) from a real
$2p$--form $\alpha_{x}$ to the calibrated cone at $x$.

\paragraph{Finite calibrated frame (net viewpoint).}
Fix $\varepsilon = \tfrac{1}{10}$.
Choose a maximal $\varepsilon$--separated subset 
$\{V_{1},\ldots,V_{N}\}\subset \Gp(x)$, i.e.\ an $\varepsilon$--net
of the calibrated Grassmannian with respect to its standard homogeneous
Riemannian metric.  
Standard packing estimates on the complex Grassmannian yield the explicit
bound
\[
N \;\le\; 30^{\,2p(n-p)}.
\]





\noindent\textbf{Finite calibrated frame (net viewpoint, corrected).}
Fix $\varepsilon=\tfrac{1}{10}$ and let $\{V_{1},\ldots,V_{N}\}\subset \Gp(x)$ be an $\varepsilon$--net.
Set
\[
M(\alpha_{x}) := \max_{V\in \Gp(x)} \langle \alpha_{x},\phi_{V}\rangle_{+},
\qquad
M_{\varepsilon}(\alpha_{x}) := \max_{1\le j\le N}\langle \alpha_{x},\phi_{V_{j}}\rangle_{+}.
\]
Since $V\mapsto \phi_{V}$ is smooth on the compact manifold $\Gp(x)$, it is globally Lipschitz.
one has $\|\phi_{V}-\phi_{V'}\|\le C_{n,p}\,d_{\Gp}(V,V')$ for some $C_{n,p}$ depending only on $(n,p)$.
Therefore, for any $V$ and a net point $V_{j}$ with $d_{\Gp}(V,V_{j})\le \varepsilon$,
\[
\bigl|\langle \alpha_{x},\phi_{V}\rangle - \langle \alpha_{x},\phi_{V_{j}}\rangle\bigr|
\le \|\alpha_{x}\|\,\|\phi_{V}-\phi_{V_{j}}\|
\le C_{n,p}\,\varepsilon\,\|\alpha_{x}\|.
\]
In particular,
\[
0\le M(\alpha_{x})-M_{\varepsilon}(\alpha_{x})\le C_{n,p}\,\varepsilon\,\|\alpha_{x}\|,
\]
so the $\varepsilon$--net yields a quantitative \emph{discretization} of the support functional
$V\mapsto \langle \alpha_{x},\phi_{V}\rangle_{+}$ used in \eqref{eq:ray-defect-formula}.
No equivalence between $\distcone(\alpha_{x})=\dist(\alpha_{x},\mathcal{C}_{x})$ and the distance to a finite-dimensional
linear span is asserted or needed anywhere in the proof.


% ----------------------------------------------------------
% Ray distance vs. convex calibrated cone
% ----------------------------------------------------------

\subsection*{Ray distance vs.\ convex calibrated cone}

For a calibrated simple form $\phi_{V}$ and any real $2p$--form 
$\alpha_{x}\in \Lambda^{2p}T^{*}_{x}X$, consider the ray generated by $\phi_{V}$.
The pointwise distance from $\alpha_{x}$ to this ray is
\[
\dist\bigl(\alpha_{x}, \R_{\ge 0}\,\phi_{V}\bigr)
:=
\inf_{\lambda\ge 0} \|\alpha_{x}-\lambda\phi_{V}\|.
\]
Minimizing over all calibrated rays yields the \emph{ray defect}
\[
\Def_{\mathrm{ray}}(\alpha_{x})
:=
\inf_{V\in \Gp(x)}
\dist\!\left(
\alpha_{x},\,
\R_{\ge 0}\,\phi_{V}
\right).
\]

Since the convex calibrated cone
\[
\mathcal{C}_{x} = \cone\{\phi_{V} : V\in \Gp(x)\}
\]
contains every such ray, one always has
\[
\distcone(\alpha_{x})
\;=\;
\dist\bigl(\alpha_{x},\mathcal{C}_{x}\bigr)
\;\le\;
\Def_{\mathrm{ray}}(\alpha_{x}).
\]

\smallskip\noindent\textbf{Remark (no cone--to--span distortion needed).}
The only general relationship used later is the trivial inclusion
$\R_{\ge 0}\phi_{V}\subset \mathcal{C}_{x}$ for each $V\in\Gp(x)$, which gives
\[
\dist\bigl(\alpha_{x},\mathcal{C}_{x}\bigr)\;\le\;\Def_{\mathrm{ray}}(\alpha_{x}).
\]
We do \emph{not} use (and do not claim) any converse estimate comparing
$\dist(\alpha_{x},\mathcal{C}_{x})$ to the distance from $\alpha_{x}$ to the
linear span of a finite net.  When a finite net is needed, it is used only
to discretize the support functional in \eqref{eq:ray-defect-formula}, as explained above.


% ----------------------------------------------------------
% Radial minimization along a calibrated ray
% ----------------------------------------------------------





\begin{lemma}[Explicit minimization along a calibrated ray]
\label{lem:radial-min}
Fix $x\in X$ and write $\mathcal{C}_{x}:=\mathrm{cone}\{\phi_{V}:V\in\Gp(x)\}\subset \Lambda^{2p}T_x^*X$.
Define the \emph{ray defect} of a real $2p$--form $\alpha_{x}$ by
\[
\Def_{\mathrm{ray}}(\alpha_{x})
\;:=\;
\inf_{V\in\Gp(x)} \dist\bigl(\alpha_{x},\R_{\ge 0}\,\phi_{V}\bigr)
\;=\;
\inf_{V\in\Gp(x)}\ \inf_{\lambda\ge 0}\ \|\alpha_{x}-\lambda\,\phi_{V}\|.
\]
Then for each fixed $V\in\Gp(x)$, the inner minimization is attained at
$\lambda^{*}=\langle \alpha_{x},\phi_{V}\rangle_{+}:=\max\{0,\langle \alpha_{x},\phi_{V}\rangle\}$, and
\begin{equation}\label{eq:ray-defect-formula}
\Def_{\mathrm{ray}}(\alpha_{x})^{2}
\;=\;
\|\alpha_{x}\|^{2}
-
\Bigl(
\max_{V\in \Gp(x)} \langle \alpha_{x},\phi_{V}\rangle_{+}
\Bigr)^{2}.
\end{equation}
Moreover, since $\R_{\ge 0}\phi_{V}\subset \mathcal{C}_{x}$ for every $V$,
one always has the elementary comparison
\[
\distcone(\alpha_{x})=\dist(\alpha_{x},\mathcal{C}_{x})
\;\le\;
\Def_{\mathrm{ray}}(\alpha_{x}).
\]
\end{lemma}


\begin{proof}
Fix $V$ and consider $f(\lambda):=\|\alpha_{x}-\lambda\phi_{V}\|^{2}
=\|\alpha_{x}\|^{2}-2\lambda\langle \alpha_{x},\phi_{V}\rangle+\lambda^{2}$ for $\lambda\ge 0$.
This is a convex quadratic with unconstrained minimizer at $\lambda=\langle \alpha_{x},\phi_{V}\rangle$.
Imposing $\lambda\ge 0$ yields $\lambda^{*}=\langle \alpha_{x},\phi_{V}\rangle_{+}$ and
\[
\min_{\lambda\ge 0}\|\alpha_{x}-\lambda\phi_{V}\|^{2}
=\|\alpha_{x}\|^{2}-\langle \alpha_{x},\phi_{V}\rangle_{+}^{2}.
\]
Taking the infimum over $V\in\Gp(x)$ gives \eqref{eq:ray-defect-formula}.
The final inequality follows from $\bigcup_{V\in\Gp(x)}\R_{\ge 0}\phi_{V}\subset\mathcal{C}_{x}$.
\end{proof}
\iffalse
\begin{proof}
	Fix $\xi \in \Gp(x)$ with $\|\xi\| = 1$ and define
	\[
	f(\lambda)
	\;:=\;
	\|\alpha_{x} - \lambda \xi\|^{2},
	\qquad \lambda \in \R.
	\]
	Expanding using $\|\xi\|=1$ gives
	\[
	f(\lambda)
	\;=\;
	\|\alpha_{x}\|^{2}
	- 2\lambda\,\langle \alpha_{x}, \xi \rangle
	+ \lambda^{2},
	\]
	which is a strictly convex quadratic in $\lambda$.
	The unconstrained minimizer satisfies $f'(\lambda)=0$, namely
	\[
	\lambda_{\mathrm{unconstr}}
	\;=\;
	\langle \alpha_{x}, \xi \rangle.
	\]
	
	Imposing the constraint $\lambda \ge 0$ yields
	\[
	\lambda^{*}
	\;=\;
	\max\{0, \langle \alpha_{x}, \xi \rangle\}.
	\]
	If $\langle \alpha_{x}, \xi \rangle \ge 0$, then
	\[
	f(\lambda^{*})
	= \|\alpha_{x}\|^{2} - \langle \alpha_{x}, \xi \rangle^{2},
	\]
	while if $\langle \alpha_{x}, \xi \rangle < 0$, the minimum is attained
	at $\lambda^{*}=0$ with value $f(0) = \|\alpha_{x}\|^{2}$.
	Both cases are encoded by
	\[
	\min_{\lambda \ge 0} \|\alpha_{x} - \lambda \xi\|^{2}
	=
	\|\alpha_{x}\|^{2}
	-
	\bigl(\langle \alpha_{x}, \xi \rangle_{+}\bigr)^{2}.
	\]
	
	By definition of the pointwise calibration distance to the cone,
	\[
	\distcone(\alpha_{x})^{2}
	=
	\inf_{\lambda \ge 0,\;\xi \in \Gp(x)}
	\|\alpha_{x} - \lambda \xi\|^{2}.
	\]
	For each fixed $\xi$ we have already minimized over $\lambda \ge 0$, so
	\[
	\distcone(\alpha_{x})^{2}
	=
	\inf_{\xi \in \Gp(x)}
	\Bigl(
	\|\alpha_{x}\|^{2}
	-
	\bigl(\langle \alpha_{x}, \xi \rangle_{+}\bigr)^{2}
	\Bigr)
	=
	\|\alpha_{x}\|^{2}
	-
	\Bigl(
	\sup_{\xi \in \Gp(x)}
	\langle \alpha_{x}, \xi \rangle_{+}
	\Bigr)^{2},
	\]
	which is exactly \eqref{eq:ray-defect-formula}.
\end{proof}
\fi


% ----------------------------------------------------------
% Trace L^2 control (used later with Hermitian model)
% ----------------------------------------------------------

\begin{lemma}[Trace $L^{2}$ control]\label{lem:trace-L2}
	Let $\eta$ be the Coulomb potential with $d^{*}\eta = 0$ and
	\[
	\alpha = \gharm + d\eta.
	\]
	Define
	\[
	\beta := (d\eta)^{(p,p)},
	\]
	and let
	\[
	H_{\beta}(x) := \mathcal{I}(\beta_{x}) \in \Herm\bigl(\Lambda^{p,0}_{x}X\bigr),
	\]
	where $d := \dim_{\C}\Lambda^{p,0}_{x}X = \binom{n}{p}$ and
	$\mathcal{I}$ is the fixed identification from Lemma~\ref{lem:hermitian-model} between
	$\Lambda^{p,p}_{x}T^{*}X$ and $\Herm(\Lambda^{p,0}_{x}X)\,$.
	Set
	\[
	\mu(x) := \frac{1}{d}\,\tr H_{\beta}(x).
	\]
	Then
	\begin{equation}\label{eq:trace-L2-bound}
		\|\mu\|_{L^{2}}
		\;\le\;
		C_{\Lambda}(n,p)\,\|d\eta\|_{L^{2}},
		\qquad
		\text{where } C_{\Lambda}(n,p)>0 \text{ depends only on } (n,p).
	\end{equation}
\end{lemma}


\begin{proof}
	Pointwise at each $x\in X$, apply Cauchy--Schwarz for the Hilbert--Schmidt
	inner product on $\Herm(\Lambda^{p,0}_{x}X)$:
	\[
	\bigl|\tr H_{\beta}(x)\bigr|
	\;\le\;
	\sqrt{d}\,\|H_{\beta}(x)\|_{\HS}.
	\]
	Hence
	\[
	|\mu(x)|
	= \frac{1}{d}\,\bigl|\tr H_{\beta}(x)\bigr|
	\;\le\;
	d^{-1/2}\,\|H_{\beta}(x)\|_{\HS}.
	\]
	By Lemma~\ref{lem:hermitian-model}, the map $\mathcal I$ is a uniform linear isomorphism, hence there is a constant $C_I(n,p)$ (independent of $x$) such that
	Set $C_{\Lambda}(n,p):=d^{-1/2}C_I(n,p)$.
	\[
	\|H_{\beta}(x)\|_{\HS}\ \le\ C_I(n,p)\,\|\beta(x)\|.
	\]
	Moreover, since $\beta$ is the $(p,p)$--component of $d\eta$ and the
	$(r,s)$--components are orthogonal in the K\"ahler metric, we have the
	pointwise inequality
	\[
	\|\beta(x)\| \;\le\; \|d\eta(x)\|.
	\]
	Combining these estimates gives
	\[
	|\mu(x)|
	\;\le\;
	d^{-1/2}\,\|d\eta(x)\|
	\quad\text{for all } x\in X.
	\]
	Squaring and integrating over $X$ yields
	\[
	\|\mu\|_{L^{2}}
	\;\le\;
	C_{\Lambda}(n,p)\,\|d\eta\|_{L^{2}},
	\]
	which is exactly \eqref{eq:trace-L2-bound}.
\end{proof}

% ----------------------------------------------------------
% Basic properties of the calibration distance
% ----------------------------------------------------------





\begin{proposition}[Well-posedness and basic properties]
\label{prop:dist-cal-properties}
Fix $x\in X$ and consider the calibrated cone
$\mathcal{C}_{x}:=\mathrm{cone}\{\phi_{V}:V\in\Gp(x)\}\subset \Lambda^{2p}T_x^*X$,
which is a closed convex cone in the Euclidean space $(\Lambda^{2p}T_x^*X,\langle\cdot,\cdot\rangle)$.
Define $\distcone(\alpha_{x}):=\dist(\alpha_{x},\mathcal{C}_{x})$.
Then:
\begin{enumerate}
\item[\textnormal{(1)}] \textbf{Existence/uniqueness of projection.}
There exists a unique nearest point $\Pi_{\mathcal{C}_{x}}(\alpha_{x})\in\mathcal{C}_{x}$ such that
\[
\distcone(\alpha_{x})=\|\alpha_{x}-\Pi_{\mathcal{C}_{x}}(\alpha_{x})\|.
\]
\item[\textnormal{(2)}] \textbf{Positive homogeneity and $1$--Lipschitz continuity.}
For every $t\ge 0$,
$\distcone(t\alpha_{x})=t\,\distcone(\alpha_{x})$,
and for all $\alpha_{x},\beta_{x}$,
\[
\bigl|\distcone(\alpha_{x})-\distcone(\beta_{x})\bigr|\le \|\alpha_{x}-\beta_{x}\|.
\]
\item[\textnormal{(3)}] \textbf{Dependence on $x$.}
If $\alpha$ is measurable, then $x\mapsto \distcone(\alpha_{x})$ is measurable.
Moreover, in a local unitary trivialization of $TX$ (identifying $T_xX\simeq\C^{n}$),
the cone $\mathcal{C}_{x}$ identifies with a fixed model cone; hence if $\alpha$ is continuous
(respectively smooth) then $x\mapsto \distcone(\alpha_{x})$ is continuous (respectively locally Lipschitz).
\item[\textnormal{(4)}] \textbf{Zero characterization.}
One has $\distcone(\alpha_{x})=0$ if and only if $\alpha_{x}\in\mathcal{C}_{x}$.
In contrast, the \emph{ray defect} $\Def_{\mathrm{ray}}$ vanishes if and only if
$\alpha_{x}\in \R_{\ge 0}\phi_{V}$ for some $V\in\Gp(x)$ (equivalently, the maximum in
\eqref{eq:ray-defect-formula} equals $\|\alpha_{x}\|$).
\end{enumerate}
\end{proposition}



\noindent\textbf{Footprint-scaled variant (used later).} In the corner-exit construction, each face-slice current $\Sigma_y$ is supported in a region of diameter $\asymp s$ inside $F$ (with $s\ll h$).  In that regime the relevant per-face flat mismatch scaling uses $s^{k-1}$ (rather than $h^{k-1}$), provided one records the mass bound $\Mass(\Sigma_y)\lesssim s^{k-1}$ uniformly in $y$ and uses the translation estimate $|\Sigma_y(\eta)-\Sigma_{y'}(\eta)|\le \Lip(\eta)\,|y-y'|\,\Mass(\Sigma_y)$. The global parameter schedule therefore enforces $\varepsilon_j h_j\ll s_j$ so that geometric identification errors across faces remain negligible even when many sliver pieces are present.
\iffalse
\begin{proof}
Items (1)--(2) are standard facts for closed convex sets in finite-dimensional Hilbert spaces
(existence/uniqueness of the metric projection and $1$--Lipschitz property of the distance function).
Item (3) follows because in unitary coordinates the cone depends only on $(n,p)$, and $\dist(\cdot,\mathcal{C})$
is continuous (indeed, $1$--Lipschitz) in its argument.  Item (4) is immediate from the definition of distance.
\end{proof}
\fi



\begin{proof}
	(1) The calibrated Grassmannian $\Gp(x)$ is a compact homogeneous space
	(isomorphic to the complex Grassmannian $G_{\C}(p,n)$ \REVMZ{\cite{GH78}}), hence compact in the
	topology induced by the Riemannian metric.
	For fixed $\alpha_{x}$, the map
	\[
	\xi \longmapsto \langle \alpha_{x}, \xi \rangle
	\]
	is continuous on $\Gp(x)$, so the maximum in
	\eqref{eq:ray-defect-formula} is attained.  Therefore the infimum in the
	definition of $\distcone(\alpha_{x})$ (taken over rays
	$\R_{\ge 0}\xi$ with $\xi \in \Gp(x)$ and radial parameter
	$\lambda\ge 0$) is realized by some optimal pair
	$(\lambda^{*},\xi^{*})$.
	
	(2) The positive homogeneity follows directly from the definition:
	\[
	\distcone(t\alpha_{x})
	=
	\inf_{\lambda \ge 0,\;\xi \in \Gp(x)}
	\|t\alpha_{x} - \lambda \xi\|
	=
	t\inf_{\lambda' \ge 0,\;\xi \in \Gp(x)}
	\|\alpha_{x} - \lambda' \xi\|
	=
	t\,\distcone(\alpha_{x}).
	\]
	For the Lipschitz property, recall that the distance to any closed subset
	$C$ of a Hilbert space is $1$--Lipschitz:
	\[
	\bigl|\dist(u,C) - \dist(v,C)\bigr|
	\;\le\;
	\|u-v\|.
	\]
	Here $C = \mathcal{C}_{x}$, the calibrated cone at $x$, so
	\[
	\bigl|
	\distcone(\alpha_{x})
	-
	\distcone(\beta_{x})
	\bigr|
	=
	\bigl|
	\dist(\alpha_{x},\mathcal{C}_{x})
	-
	\dist(\beta_{x},\mathcal{C}_{x})
	\bigr|
	\;\le\;
	\|\alpha_{x} - \beta_{x}\|.
	\]
	
	(3) In a local trivialization of $\Lambda^{2p}T^{*}X$ and of the family of
	calibrated simple forms, the map
	\[
	(x,\xi) \longmapsto \langle \alpha_{x}, \xi \rangle
	\]
	is measurable in $x$ and continuous in $\xi$ whenever $\alpha$ is
	measurable.  Taking the supremum over the compact fiber
	$\Gp(x)$ produces a measurable function of $x$, and
	\eqref{eq:ray-defect-formula} then implies measurability of
	$x \mapsto \distcone(\alpha_{x})$.
	
	If $\alpha$ is continuous (resp.\ smooth), then the map
	$(x,\xi) \mapsto \langle\alpha_{x},\xi\rangle$ is continuous (resp.\ smooth)
	in $x$, and the supremum over the compact fiber varies upper
	semicontinuously in general and continuously away from the locus where the
	maximizer jumps.  Thus $x \mapsto \distcone(\alpha_{x})$ is
	continuous (resp.\ smooth off that ridge set).
	
	(4) If $\alpha_{x} = \lambda\xi$ with $\lambda \ge 0$ and
	$\xi \in \Gp(x)$, then by Lemma~\ref{lem:radial-min} the optimal
	radial parameter is $\lambda^{*}=\lambda$ and the minimum distance is zero,
	so $\distcone(\alpha_{x})=0$.
	
	Conversely, if $\distcone(\alpha_{x})=0$, then
	\eqref{eq:ray-defect-formula} gives
	\[
	\|\alpha_{x}\|^{2}
	=
	\Bigl(
	\max_{\xi \in \Gp(x)}
	\langle \alpha_{x}, \xi \rangle_{+}
	\Bigr)^{2}.
	\]
	For a maximizing direction $\xi^{*}$ with 
	$\langle\alpha_{x},\xi^{*}\rangle_{+} = \|\alpha_{x}\|$, equality holds in
	the Cauchy--Schwarz inequality, so $\alpha_{x}$ is a nonnegative multiple of
	$\xi^{*}$.  Hence $\alpha_{x} \in \R_{\ge 0}\cdot\Gp(x)$,
	as claimed.
\end{proof}

% ----------------------------------------------------------
% Optional: K\"ahler-angle parametrization (for intuition)
% ----------------------------------------------------------

\subsection*{Optional: K\"ahler-angle parametrization (for intuition)}

Let $x \in X$ and let $V,V' \in \Gp(x)$ be complex $p$--planes.
The relative position of $(V,V')$ is encoded by their $p$ K\"ahler angles
$\theta_{1},\ldots,\theta_{p} \in [0,\tfrac{\pi}{2})$ \REVMZ{\cite{GH78}}, the canonical angles
arising from the $U(n)$--invariant geometry of the Grassmannian.
In an adapted unitary frame one has the classical identity
\[
\langle \phi_{V},\phi_{V'} \rangle
= \prod_{j=1}^{p} \cos\theta_{j},
\]
where $\phi_{V}$ and $\phi_{V'}$ denote the associated unit calibrated
simple $(p,p)$--forms.

For small angles, the expansion
\[
\cos\theta
= 1 - \tfrac{1}{2}\theta^{2} + \tfrac{1}{24}\theta^{4}
+ O(\theta^{6})
\]
provides a second--order approximation of the inner product in terms of
$\sum_{j} \sin^{2}\theta_{j}$.  This relation between calibrated directions
and the K\"ahler angles yields the following quadratic control estimate.


\begin{lemma}[Quadratic control for small Jordan angles (principal angles)]
	\label{lem:kahler-angle}
	Let $V,V' \in \Gp(x)$ have K\"ahler angles
	$\theta_{1},\ldots,\theta_{p}$ satisfying
	\[
	\sum_{j=1}^{p} \theta_{j}^{2} \;\le\; \delta_{\mathrm{KA}}(p).
	\]
	Then the corresponding calibrated unit covectors $\phi_{V}$ and $\phi_{V'}$
	satisfy the estimate
	\begin{equation}\label{eq:kahler-angle-est}
		c_{1}(p)\sum_{j=1}^{p} \sin^{2}\theta_{j}
		\;\le\;
		1 - \langle \phi_{V}, \phi_{V'} \rangle
		\;\le\;
		c_{2}(p)\sum_{j=1}^{p} \sin^{2}\theta_{j}.
	\end{equation}
	(Only the existence of such constants depending on $p$ is used later; for concreteness one may take $\delta_{\mathrm{KA}}(p)=10^{-2}$, $c_{1}(p)=0.25$, and $c_{2}(p)=0.51$.)
\end{lemma}


\begin{proof}
	Using the standard principal-angle identity \REVMZ{\cite{GH78}}
	\(
	\langle \phi_{V},\phi_{V'}\rangle=\prod_{j=1}^{p}\cos\theta_{j},
	\)
	it suffices to control $1-\prod_j\cos\theta_j$.
	For $0\le\theta\le 0.1$ one has
	\[
	1-\cos\theta \;=\; 2\sin^2(\theta/2)
	\;\ge\; \tfrac12\,\sin^2\theta,
	\]
	and also, since $\cos(\theta/2)\ge \cos(0.05)$ on this range,
	\[
	1-\cos\theta \;=\; \frac{\sin^2\theta}{2\cos^2(\theta/2)}
	\;\le\; \frac{1}{2\cos^2(0.05)}\,\sin^2\theta
	\;\le\; 0.51\,\sin^2\theta.
	\]
	Let $a_j:=1-\cos\theta_j\ge 0$.  Since $\sum_j\theta_j^2\le 10^{-2}$, we have $0\le \theta_j\le 0.1$ and hence
	$\sum_j a_j \le 0.51\sum_j\sin^2\theta_j\le 0.51\cdot 10^{-2}<1$.
	Now
	\[
	1-\prod_{j=1}^p\cos\theta_j
	\;=\;1-\prod_{j=1}^p(1-a_j)
	\;\le\;\sum_{j=1}^p a_j
	\;\le\;0.51\sum_{j=1}^p\sin^2\theta_j.
	\]
	For the lower bound, use $\prod_j(1-a_j)\le e^{-\sum_j a_j}$ to get
	\[
	1-\prod_{j=1}^p\cos\theta_j
	\;=\;1-\prod_{j=1}^p(1-a_j)
	\;\ge\;1-e^{-\sum_j a_j}
	\;\ge\;\tfrac12\sum_{j=1}^p a_j
	\;\ge\;0.25\sum_{j=1}^p \sin^2\theta_j,
	\]
	using $1-e^{-t}\ge t/2$ for $t\in[0,1]$ and $a_j\ge \tfrac12\sin^2\theta_j$.
\end{proof}


\begin{remark}[Geometric meaning of Lemma~\ref{lem:kahler-angle}]
\REVMZ{\textbf{[Geometric intuition.]}}
	Lemma~\ref{lem:kahler-angle} shows that, when the K\"ahler angles between two
	complex $p$--planes are small, the deviation of their calibrated directions is
	quadratically controlled by the sum of the squared angles.  Since
	$\langle\phi_{V},\phi_{V'}\rangle = \prod_{j=1}^{p}\cos\theta_{j}$, the
	quantity
	\[
	1 - \langle \phi_{V},\phi_{V'}\rangle
	\]
	measures the pointwise misalignment between the two calibrated simple
	$(p,p)$--forms.  Lemma~\ref{lem:kahler-angle} asserts that this misalignment is
	comparable, up to uniform constants, to the elementary quadratic quantity
	$\sum_{j=1}^{p}\sin^{2}\theta_{j}$ whenever $\sum \theta_{j}^{2}$ is suitably
	small.  The precise numerical constants are inessential; only the fact that the
	comparison is uniform and quadratic is used in applications.
\end{remark}


	% ============================================================
%                    SECTION 4
% ============================================================

\section{Energy Gap and Primitive/Off--Type Controls}
\label{sec:energy-gap}

\REVMZ{\textbf{Optional background section.}} We record standard K\"ahler/Hodge estimates \REVMZ{\cite{Wells,Voisin02,GH78}} controlling off--type components and the primitive part of a closed form in terms of the energy of a Coulomb potential. \REVMZ{These estimates provide context but are not required for the main proof.}

Let $(X,\omega)$ be a compact K\"ahler manifold of complex dimension $n$.
Fix a real Hodge class
\[
[\alpha]\in H^{2p}(X,\RR)\cap H^{p,p}(X),
\]
and let $\alpha$ be a smooth real closed $2p$--form representing $[\alpha]$.



The purpose of this section is to record standard K\"ahler/Hodge estimates controlling
off--type components and the primitive part of a closed form in terms of the energy of its Coulomb potential.
\REVMZ{These estimates provide optional background for the coercivity discussion in Section~\ref{sec:cal-coercivity}.}




\subsection*{Coulomb potential}
Fix a representative $\alpha$ of $[\alpha]$.  Since $d\alpha = 0$, the elliptic
equation
\[
d^{*}d\eta = d^{*}\alpha
\]
admits a unique solution $\eta$ orthogonal to $\ker d$, giving the Hodge
decomposition
\[
\alpha
= \gamma_{\harm} + d\eta,
\]
where $\gamma_{\harm}$ is the unique harmonic representative of $[\alpha]$.
We define the energy of $\alpha$ by
\[
E(\alpha) := \|d\eta\|^{2}_{L^{2}}.
\]


\subsection*{Energy identity and type decomposition}
We recall a standard Hodge--theoretic fact: on a compact K\"ahler manifold, the space of harmonic forms decomposes into harmonic \((r,s)\) types, and the harmonic representative of a cohomology class in \(H^{r,s}(X)\) is of type \((r,s)\) (see, e.g., Wells \cite[Ch.~5]{Wells}).
In particular, since \( [\alpha]\in H^{p,p}(X)\), the harmonic representative \(\gamma_{\harm}\) of \([\alpha]\) has pure type \((p,p)\).

\smallskip
Fix a representative \(\alpha\) of \([\alpha]\).  Since \(d\alpha=0\), the elliptic equation
\[
d^{*}d\eta = d^{*}\alpha
\]
admits a unique solution \(\eta\) orthogonal to \(\ker d\) (equivalently, orthogonal to harmonic \((2p-1)\)-forms), giving the Hodge decomposition \REVMZ{\cite{Wells,Voisin02}}
\[
\alpha = \gamma_{\harm} + d\eta,
\qquad d^*\eta=0.
\]
We define the energy of \(\alpha\) by
\[
E(\alpha):=\|d\eta\|_{L^{2}}^{2}.
\]

\subsection*{Energy split}
Since \(\gamma_{\harm}\perp d\eta\) in \(L^2\), we have
\begin{equation}\label{eq:energy-split}
	E(\alpha)
	= \|d\eta\|_{L^{2}}^{2}
	= \|\alpha\|_{L^{2}}^{2} - \|\gamma_{\harm}\|_{L^{2}}^{2}.
	\tag{11}
\end{equation}

\subsection*{Type split}
Decompose \(\alpha\) into types \(\alpha=\sum_{r+s=2p}\alpha^{(r,s)}\).
Because \(\gamma_{\harm}\) has type \((p,p)\), all off--type components of \(\alpha\) are exact and belong to \(d\eta\):
\[
\alpha^{(r,s)}=(d\eta)^{(r,s)}
\qquad\text{for all }(r,s)\neq(p,p).
\]
Orthogonality of distinct types yields
\begin{equation}\label{eq:type-split}
	\|\alpha-\gamma_{\harm}\|_{L^{2}}^{2}
	=
	\sum_{\substack{r+s=2p\\(r,s)\neq(p,p)}}\|\alpha^{(r,s)}\|_{L^{2}}^{2}
	+\|(\alpha^{(p,p)}-\gamma_{\harm})\|_{L^{2}}^{2}.
	\tag{12}
\end{equation}

\subsection*{Primitive/off--type control}
Let \((\cdot)_{\prim}\) denote the \(L^2\)-orthogonal projection onto \(\omega\)-primitive \((p,p)\)-forms.
Elliptic control on the Coulomb slice gives a uniform bound (depending only on \(X,\omega,p\)):
\begin{equation}\label{eq:primitive-control}
	\sum_{\substack{r+s=2p\\(r,s)\neq(p,p)}}\|\alpha^{(r,s)}\|_{L^{2}}
	+\|(\alpha^{(p,p)}-\gamma_{\harm})_{\prim}\|_{L^{2}}
	\;\le\;
	C(X,\omega,p)\,\|d\eta\|_{L^{2}}.
	\tag{13}
\end{equation}

\smallskip
\noindent\emph{Remark.} If \([\alpha]\) has a nonzero harmonic off--type component, then no estimate of the form \eqref{eq:primitive-control} can hold with right--hand side \(\|d\eta\|_{L^2}\), since harmonic components are invisible to the Coulomb energy.



\begin{lemma}[Elliptic estimate on the Coulomb slice]\label{lem:elliptic-coulomb}

Let $\eta$ be a smooth $(2p-1)$--form on a compact K\"ahler manifold with $d^*\eta=0$ and $\eta\perp \ker d$.
Then there exists a constant $C=C(X,\omega,p)$ such that
\[
\|\eta\|_{H^1}\ \le\ C\,\|d\eta\|_{L^2}.
\]
In particular, the $L^2$ norms of all first-order type components $\partial\eta^{(r,s)}$ and $\bar\partial\eta^{(r,s)}$ are bounded by $C\,\|d\eta\|_{L^2}$.
\end{lemma}


\begin{proof}
\iffalse
We use standard Hodge-theoretic elliptic estimate
see \REV{\cite{Wells}}.
\fi

This is a standard elliptic estimate for the Hodge operator $d+d^*$ \REVMZ{\cite{Wells,Voisin02}} (equivalently for the Laplacian) on the Coulomb slice $d^*\eta=0$, restricted to the orthogonal complement of harmonic forms.
One convenient formulation is
\[
\|\eta\|_{H^1}\ \le\ C\bigl(\|d\eta\|_{L^2}+\|d^*\eta\|_{L^2}\bigr),
\]
valid on any compact Riemannian manifold; imposing $d^*\eta=0$ gives the stated bound.
See, for example, Wells, \emph{Differential Analysis on Complex Manifolds}, Chapter~5, or any standard Hodge theory reference.
\end{proof}



\begin{lemma}[Coulomb decomposition and energy identity]\label{lem:coulomb}
Let \([\alpha]\in H^{2p}(X,\RR)\cap H^{p,p}(X)\) and let \(\alpha\) be a smooth closed real \(2p\)--form representing \([\alpha]\).
Write \(\alpha=\gamma_{\harm}+d\eta\) for its Coulomb decomposition with \(d^*\eta=0\) and \(\eta\perp\ker d\).
Then:

\begin{enumerate}
\item
\(\displaystyle
E(\alpha)=\|d\eta\|_{L^{2}}^{2}
=\|\alpha\|_{L^{2}}^{2}-\|\gamma_{\harm}\|_{L^{2}}^{2},
\)
as in~\eqref{eq:energy-split}.

\item
The difference from the harmonic representative satisfies the orthogonal type split
\[
\|\alpha-\gamma_{\harm}\|_{L^{2}}^{2}
=
\sum_{\substack{r+s=2p\\(r,s)\neq(p,p)}}\|\alpha^{(r,s)}\|_{L^{2}}^{2}
+\|(\alpha^{(p,p)}-\gamma_{\harm})\|_{L^{2}}^{2},
\]
as in~\eqref{eq:type-split}.

\item
All off--type components and the primitive part of the \((p,p)\) component are controlled by the Coulomb energy:
\[
\sum_{\substack{r+s=2p\\(r,s)\neq(p,p)}}\|\alpha^{(r,s)}\|_{L^{2}}
+\|(\alpha^{(p,p)}-\gamma_{\harm})_{\prim}\|_{L^{2}}
\;\le\;
C(X,\omega,p)\,\sqrt{E(\alpha)},
\]
consistent with~\eqref{eq:primitive-control}.
\end{enumerate}
\end{lemma}


\begin{proof}
We use standard Hodge-theoretic elliptic estimate
see \REV{\cite{Wells}}.

Item (i) is the Pythagorean identity coming from the orthogonality \(\gamma_{\harm}\perp d\eta\).
Item (ii) is orthogonality of distinct \((r,s)\) types, using that \(\gamma_{\harm}\) has pure type \((p,p)\).

For (iii), since \(\gamma_{\harm}\) has type \((p,p)\), we have \(\alpha^{(r,s)}=(d\eta)^{(r,s)}\) for all \((r,s)\neq(p,p)\).
Each such component \((d\eta)^{(r,s)}\) is a linear combination of first--order operators \(\partial\) and \(\bar\partial\) applied to type components of \(\eta\),
so
\[
\sum_{\substack{r+s=2p\\(r,s)\neq(p,p)}}\|\alpha^{(r,s)}\|_{L^{2}}
\le C\,\|\eta\|_{H^{1}}.
\]
By Lemma~\ref{lem:elliptic-coulomb}, \(\|\eta\|_{H^{1}}\le C\,\|d\eta\|_{L^{2}}=C\sqrt{E(\alpha)}\), yielding the desired bound for off--type components.
Finally, \((\alpha^{(p,p)}-\gamma_{\harm})_{\prim}\) is the \(L^{2}\)-orthogonal projection of \((d\eta)^{(p,p)}\) to primitive \((p,p)\)-forms, hence
\(\|(\alpha^{(p,p)}-\gamma_{\harm})_{\prim}\|_{L^{2}}\le \|(d\eta)^{(p,p)}\|_{L^{2}}\le \|d\eta\|_{L^{2}}\),
and the same elliptic estimate completes the proof.
\end{proof}




% ------------------------------------------------------------
% SECTION 5 --- The Calibrated Grassmannian and an Explicit \varepsilon--Net
% ------------------------------------------------------------

\section{The Calibrated Grassmannian and an Explicit \texorpdfstring{$\varepsilon$}{epsilon}--Net}




We construct an explicit $\varepsilon$--net on the fiberwise calibrated Grassmannian \REVMZ{(see \cite{GH78} for the complex Grassmannian)} and record covering bounds used later to discretize calibrated directions and control angular errors in quantitative estimates.



\subsection*{Fiberwise geometry}

Fix $x\in X$ and set
\[
\varphi := \frac{\omega^{p}}{p!}.
\]
Define the calibrated Grassmannian at $x$ by
\[
G_{p}(x)
:=
\Big\{
\xi \in \Lambda^{2p}T^{*}_{x}X :
\|\xi\| = 1,\;
\xi\ \text{simple of type $(p,p)$},\;
\varphi_{x}(\xi)=1
\Big\}.
\]
This is the set of unit simple $(p,p)$ covectors saturated by the K\"ahler
calibration $\varphi_{x}$.  Equivalently, $G_{p}(x)$ is the image of the
complex Grassmannian $G_{\C}(p,n)$ under the map sending a $p$--plane
$V\subset T^{1,0}_{x}X$ to its associated calibrated covector $\phi_{V}$.
With the metric induced by $\omega$, this map is an isometric embedding
(up to normalization), and therefore
\[
G_{p}(x) \cong G_{\C}(p,n)
\]
with its standard Fubini--Study metric.


\noindent\textbf{Clarification (fiber model and metric).}
For the arguments below we only use that each fiber $G_{p}(x)$ is a \emph{fixed} compact homogeneous manifold
(isomorphic to $G_{\C}(p,n)$) and that $d_{\mathrm{FS}}$ denotes any fixed $U(n)$--invariant Riemannian distance on that model.
Different normalizations of the invariant metric are bi--Lipschitz equivalent with constants depending only on $(n,p)$,
so all covering/packing constants in Lemma~\ref{lem:covering-number} may be taken uniform in $x$.


In particular, $G_{p}(x)$ is
compact, smooth, homogeneous, and has real dimension
\[
d := \dim_{\R} G_{p}(x)
= 2p(n-p).
\]

\subsection{$\varepsilon$--nets and covering estimates}\label{sec:epsnets}

Fix $\varepsilon = \tfrac{1}{10}$.  


\smallskip\noindent\textbf{Remark (choice of the net).}
The maximal $\varepsilon$--separated set $\{\xi(x)_{\ell}\}_{\ell}$ is chosen \emph{independently for each} $x\in X$.
Only the uniform cardinality bound \eqref{eq:grass-cover} is used later; no global smooth/measurable dependence of the selection on $x$
is required in the main proof chain.



On each fiber $G_{p}(x)$ (with the Fubini--Study geodesic distance
$d_{\mathrm{FS}}$), choose a maximal $\varepsilon$--separated set
\[
\{\xi(x)_\ell\}_{\ell=1}^{N(x)}
\subset G_{p}(x),
\qquad
d_{\mathrm{FS}}(\xi(x)_\ell,\xi(x)_m) \ge \varepsilon
\ \text{for all }\ell\ne m,
\]
such that no additional point of $G_{p}(x)$ can be added while preserving
this separation property.

By compactness and the standard packing principle \REVMZ{\cite{GH78}} on compact homogeneous
spaces, such maximal $\varepsilon$--separated sets are automatically
$\varepsilon$--nets: for every $\xi \in G_{p}(x)$ there exists an index
$\ell$ with  
\[
d_{\mathrm{FS}}(\xi,\xi(x)_\ell) \le \varepsilon.
\]

\begin{lemma}[Covering number]\label{lem:covering-number}
	Let $d = 2p(n-p)$.  
	There exists a constant $C(n,p)$ depending only on $(n,p)$ such that every
	maximal $\varepsilon$--separated set in $G_{p}(x)$ satisfies
	\begin{equation}\label{eq:grass-cover}
		N(x) \;\le\; C(n,p)\,\varepsilon^{-d}.
		\tag{5.1}
	\end{equation}
\end{lemma}


\begin{proof}
	Cover $G_{p}(x)$ by the geodesic balls
	\[
	B\!\left(\xi(x)_\ell,\,\tfrac{\varepsilon}{2}\right),
	\qquad \ell=1,\dots,N(x),
	\]
	of radius $\varepsilon/2$ in the Fubini--Study metric.  
	Because the points are $\varepsilon$--separated, these balls are pairwise
	disjoint.  By maximality of the separated set, the $\varepsilon$--balls
	\[
	B\!\left(\xi(x)_\ell,\,\varepsilon\right)
	\]
	cover $G_{p}(x)$.
	
	Since $G_{p}(x)$ is a compact homogeneous space, the volume of a small
	geodesic ball depends only on the radius, not on its center.  
	Let $V(r)$ denote the volume of a geodesic ball of radius $r$.  
	Then disjointness gives
	\[
	N(x)\,V(\varepsilon/2)
	\;\le\; \Vol\bigl(G_{p}(x)\bigr),
	\]
	while the covering property yields
	\[
	\Vol\bigl(G_{p}(x)\bigr)
	\;\le\; N(x)\,V(\varepsilon).
	\]
	
	For small $r$ one has the uniform expansion
	\[
	V(r) = c_{d}\,r^{d} + O(r^{d+2}),
	\]
	with $c_{d}>0$ depending only on $d = \dim_{\R} G_{p}(x)$.  
	Since $G_{p}(x)$ is homogeneous, there exist constants $A(n,p)$ and $B(n,p)$
	such that
	\[
	A(n,p)\,r^{d} \le V(r) \le B(n,p)\,r^{d}
	\qquad\text{for } 0<r\le 1.
	\]
	
	Combining the two volume inequalities gives
	\[
	N(x)\,A(n,p)\,(\varepsilon/2)^{d}
	\;\le\; \Vol\bigl(G_{p}(x)\bigr)
	\;\le\; N(x)\,B(n,p)\,\varepsilon^{d},
	\]
	so cancelling $\Vol(G_{p}(x))$ yields
	\[
	N(x) \;\le\;
	\frac{B(n,p)}{A(n,p)}\,(2^{d})\,
	\varepsilon^{-d}.
	\]
	
	Absorbing the constants into
	\[
	C(n,p) := \frac{B(n,p)}{A(n,p)}\,2^{d},
	\]
	we obtain the desired estimate \eqref{eq:grass-cover}.
\end{proof}
% ============================================================
% SECTION 6 --- Pointwise Linear Algebra: Controlling the Net Distance
% ============================================================

\section{Pointwise Linear Algebra: Controlling the Net Distance}
\label{sec:linear-algebra}






We develop pointwise linear--algebra estimates converting wedge/Pl\"ucker closeness into distance to the calibrated cone, with explicit tracking of constants. These estimates interface with the $\varepsilon$--net from Section~\ref{sec:epsnets} and the calibration--coercivity inequalities \REVMZ{in Section~\ref{sec:cal-coercivity}}.



\REVMZ{\noindent\textbf{Optional background (Hermitian/PSD comparison).}}
\REVMZ{The material in this block provides context and comparison but is not required for the main proof.}


In this section we develop the pointwise linear--algebraic estimates
that control the distance of a real $2p$--form to the calibrated
span generated by the $\varepsilon$--net constructed in Section~\ref{sec:epsnets}.
The goal is to show that the net distance (and therefore the cone
distance) is controlled by two quantities:


\begin{itemize}
\item the full off--type component $\alpha_x^{\mathrm{off}}:=\sum_{\substack{r+s=2p\\(r,s)\neq(p,p)}}\alpha_x^{(r,s)}$;
\item the $(p,p)$--ray distance to the net, i.e.\ $\min_{1\le \ell\le N(x),\,\lambda\ge 0}\|\alpha_x^{(p,p)}-\lambda\,\xi_\ell(x)\|$.
\end{itemize}



\REVMZ{These pointwise inequalities are recorded as optional linear-algebra background (nets/Hermitian models) and are not required for the main proof.}


% ------------------------------------------------------------
\subsection*{Calibrated span}

Fix $x\in X$ and let 
\[
\{\xi_{\ell}(x)\}_{\ell=1}^{N(x)} \subset G_{p}(x)
\]
be the $\varepsilon$--net of Section~\ref{sec:epsnets}, with $\varepsilon=\tfrac{1}{10}$.
Define the calibrated span at $x$ by
\[
\Xi_{x}:=
\Span\{\xi_{\ell}(x):1\le \ell \le N(x)\}
\subset \Lambda^{p,p}T_{x}^{*}X.
\]

Each $\xi_{\ell}(x)$ is a unit simple $(p,p)$--covector, hence lies
entirely in the $(p,p)$--subspace of $\Lambda^{2p}T_{x}^{*}X$ and is
orthogonal to all off--type $(p+1,p-1)$ and $(p-1,p+1)$ components
with respect to the K\"ahler metric.

Thus every $\alpha_{x}\in\Lambda^{2p}T_{x}^{*}X$ admits an
orthogonal type decomposition

\begin{equation}\label{eq:typesplit-orth}
\alpha_{x}
=
\alpha_{x}^{(p,p)}
\;+\;
\alpha_{x}^{\mathrm{off}},
\qquad
\alpha_{x}^{\mathrm{off}}
:=
\sum_{\substack{r+s=2p\\(r,s)\neq(p,p)}}\alpha_{x}^{(r,s)}.
\tag{21}
\end{equation}


% ------------------------------------------------------------
\subsection*{Pointwise net distance}

Define the pointwise net distance
\[
D_{\mathrm{net}}(\alpha_{x})
:=
\min_{\ell,\;\lambda\ge 0}
\|\alpha_{x} - \lambda\xi_{\ell}(x)\|.
\]

\begin{lemma}[Off--type separation for $D_{\mathrm{net}}$]\label{lem:typesplit}

For every $x$ and every $\alpha_{x}\in\Lambda^{2p}T^{*}_{x}X$,
\begin{equation}\label{eq:Dnet-typesplit}
D_{\mathrm{net}}(\alpha_{x})^{2}
=
\|\alpha_{x}^{\mathrm{off}}\|^{2}
+
\min_{1\le \ell\le N(x),\,\lambda\ge 0}
\bigl\|\alpha_{x}^{(p,p)} - \lambda \,\xi_{\ell}(x)\bigr\|^{2}.
\tag{22}
\end{equation}

\end{lemma}


\begin{proof}

Since each $\xi_\ell(x)\in\Lambda^{p,p}T_x^*X$, the orthogonal splitting
\eqref{eq:typesplit-orth} implies that for every $\lambda\ge 0$,
\[
\|\alpha_x-\lambda\xi_\ell(x)\|^2
=
\|\alpha_x^{\mathrm{off}}\|^2
+
\|\alpha_x^{(p,p)}-\lambda\xi_\ell(x)\|^2.
\]
Taking $\inf_{\lambda\ge 0}$ and then $\min_{1\le \ell\le N(x)}$ gives \eqref{eq:Dnet-typesplit}.

\end{proof}

% ------------------------------------------------------------
\subsection*{Projection estimate}

We now show that the $(p,p)$--term in \eqref{eq:Dnet-typesplit}
is controlled by a purely $(p,p)$ quantity arising from the Hermitian
model for $(p,p)$--forms and a rank--one approximation inequality.

\begin{lemma}[Hermitian model for $(p,p)$]\label{lem:hermitian-model}
	Fix $x$ and identify $\Lambda^{p,0}T_x^{*}X$ with a Hermitian space 
	$\bigl(\mathcal{H},\langle\cdot,\cdot\rangle\bigr)$ of complex dimension 
	$d=\binom{n}{p}$.  
	There is an invertible linear map
	\[
	\mathcal{I} : \Lambda^{p,p}T_x^{*}X \;\longrightarrow\; \Herm(\mathcal{H})
	\]
	(with Hilbert--Schmidt norm on the right)
	and $\|\mathcal{I}\|,\|\mathcal{I}^{-1}\|\le C(n,p)$ for a constant depending only on $(n,p)$ (so the norms are uniformly equivalent). such that:
	\begin{enumerate}
		\item for $\alpha_x^{(p,p)}\in\Lambda^{p,p}$, the matrix 
		$H_\alpha := \mathcal{I}(\alpha_x^{(p,p)})$ is Hermitian;
		
		\item for any unit decomposable $p$--vector $v\in\Lambda^{p,0}$,  
		the calibrated covector $\xi_v$ satisfies
		\[
		\mathcal{I}(\xi_v) = P_v := v\otimes v^{*}
		\]
		(the rank--one projector);
		
		\item the contraction (trace) corresponds to the Lefschetz trace \REVMZ{\cite{GH78}}:  
		there exists $\mu(\alpha_x)\in\R$ such that
		\[
		\mathcal{I}\bigl( (\alpha_x^{(p,p)})_{\mathrm{prim}} \bigr)
		=
		H_\alpha - \mu(\alpha_x)\, I_{\mathcal{H}},
		\qquad
		\mu(\alpha_x) = \frac{1}{d}\operatorname{tr}(H_\alpha).
		\]
	\end{enumerate}

\begin{proof}
Fix unitary coordinates at $x$ and let $\mathcal H:=\Lambda^{p,0}T_x^*X$ with the induced Hermitian inner product.
Given a real $(p,p)$--form $\beta\in\Lambda^{p,p}T_x^*X$, define $H_\beta\in\Herm(\mathcal H)$ by
\[
\langle H_\beta u, v\rangle\ :=\ {(-i)^{-p}}\,\beta(u\wedge \overline{v}),
\qquad u,v\in\mathcal H.
\]
{\emph{Convention:} \(\beta(u\wedge\overline v)\) denotes the coefficient/tensor contraction in a unitary coframe (equivalently the pointwise inner product \(\langle \beta,\,u\wedge\overline v\rangle\)).}
Linearity is immediate.  The reality and $(p,p)$--type of $\beta$ imply $H_\beta$ is Hermitian.

Choose an orthonormal basis $\{e_I\}_{|I|=p}$ of $\mathcal H$ (wedges of an orthonormal basis of $(1,0)$--forms).  In this basis,
the matrix coefficients are $ (H_\beta)_{IJ}={(-i)^{-p}}\,\beta(e_I\wedge\overline{e_J}) $, so
\[
\|H_\beta\|_{\mathrm{HS}}^2=\sum_{I,J} |(H_\beta)_{IJ}|^2=\sum_{I,J}|\beta(e_I\wedge\overline{e_J})|^2=\|\beta\|^2,
\]
which shows that $\|H_\beta\|_{\mathrm{HS}}$ and $\|\beta\|$ are comparable by a constant depending only on $(n,p)$ (so $\mathcal I$ is a uniform isomorphism).
Surjectivity follows by reversing the construction: any Hermitian matrix $(h_{IJ})$ defines a unique real $(p,p)$--form by prescribing
its coefficients in the basis $\{e_I\wedge\overline{e_J}\}$ via $\beta(e_I\wedge\overline{e_J})={(-i)^p}\,h_{IJ}$.

For a unit decomposable $p$--vector $v\in\mathcal H$ define the associated simple $(p,p)$--form $\xi_v$ by
\[
\xi_v(u\wedge \overline{w})\ :=\ {(-i)^p}\,\langle u,v\rangle\,\langle v,w\rangle
\qquad (u,w\in\mathcal H),
\]
which is exactly the rank--one projector kernel.  By definition this gives $\mathcal I(\xi_v)=v\otimes v^*$.

Finally, under $\mathcal I$ the K\"ahler form $\omega^p/p!$ corresponds to the identity $I_{\mathcal H}$, so the Lefschetz trace component of $\beta$
corresponds to the scalar matrix component $(\operatorname{tr}H_\beta/d)\,I_{\mathcal H}$.
Thus subtracting $(\operatorname{tr}H_\beta/d)\,I_{\mathcal H}$ corresponds to the primitive (traceless) projection of $\beta$.
\end{proof}

\end{lemma}



\begin{remark}[Calibrated cone in the Hermitian model; not the full PSD cone for $1<p<n-1$]\label{rem:cone-not-full-psd}
\REVMZ{\textbf{[Technical clarification.]}}
Let $\mathcal H=\Lambda^{p,0}T_x^*X$ and let $\mathcal I:\Lambda^{p,p}T_x^*X\to\Herm(\mathcal H)$ be the isometry of Lemma~\ref{lem:hermitian-model}.
Let $\mathsf{Dec}\subset \mathcal H$ denote the set of \emph{decomposable} $p$--vectors.
Then the calibrated/strongly-positive cone $K_p(x)$ satisfies
\[
\mathcal I\bigl(K_p(x)\bigr)
\;=\;
\mathrm{cone}\{\, v\otimes v^*: v\in \mathsf{Dec}\,\}
\;\subset\;
\Herm(\mathcal H)_{\succeq 0}.
\]
For $p=1$ or $p=n-1$, every $v\in\mathcal H$ is decomposable, so the right-hand side is the full PSD cone.
For $1<p<n-1$, there exist non-decomposable $w\in\mathcal H$,
(Existence is standard: the decomposable locus is a proper algebraic subvariety of $\Lambda^{p,0}\C^{n}$ for $1<p<n-1$ (Pl\"ucker relations); for example, in a unitary frame with $n\ge 2p$ one may take $w:=e^{1}\wedge\cdots\wedge e^{p}+e^{p+1}\wedge\cdots\wedge e^{2p}$, which is not decomposable.) hence $w\otimes w^*$ is rank-one PSD but cannot lie in
$\mathrm{cone}\{v\otimes v^*: v\in\mathsf{Dec}\}$:
indeed, if $w\otimes w^*=\sum_j v_j\otimes v_j^*$ with $v_j\in\mathsf{Dec}$, then each summand has range contained in $\mathrm{span}\{w\}$ (because the left-hand side has rank one),
so every $v_j$ is collinear with $w$, forcing $w$ to be decomposable.  Thus the calibrated cone is a strict subcone of the PSD cone when $1<p<n-1$.
\end{remark}



\begin{lemma}[Rank--one approximation controls the traceless part]\label{lem:rankone}
	There exists a finite constant $C_{\mathrm{rank}}(d)>0$, depending only on
	$d=\dim_{\C}\mathcal{H}$, such that for every $H \in \Herm(\mathcal{H})$,
	\[
%
\min_{\substack{v\in\mathcal{H},\ \|v\|=1\\ \lambda\ge 0,\ \mu\in\mathbb{R}}}\,
\bigl\|\,H-\mu I_{\mathcal{H}}-\lambda\,(v\otimes v^*)\,\bigr\|_{\HS}^2
\;\le\; C_{\mathrm{rank}}(d)\,
\Bigl\|\,H-\frac{\tr(H)}{d}\,I_{\mathcal{H}}\,\Bigr\|_{\HS}^2.\]
\end{lemma}


\begin{proof}
Let $H_0:=H-\frac{\tr(H)}{d}I_{\mathcal{H}}$.  Choosing $\mu=\frac{\tr(H)}{d}$ and $\lambda=0$ gives
\[
\min_{\substack{v,\ \|v\|=1\\ \lambda\ge 0,\ \mu\in\mathbb{R}}}\,
\bigl\|\,H-\mu I_{\mathcal{H}}-\lambda\,(v\otimes v^*)\,\bigr\|_{\HS}^2
\;\le\;\|H_0\|_{\HS}^2,
\]
so the claim holds with $C_{\mathrm{rank}}(d)=1$.
\end{proof}

\begin{proposition}[PSD surrogate for the $(p,p)$ projection]\label{prop:pp-projection}

There exists a constant $C_{0}=C_{0}(n,p)>0$ such that for every $x\in X$ and every
$\alpha_{x}\in \Lambda^{2p}T_x^*X$, writing $\gamma_{\harm,x}\in \Lambda^{p,p}T_x^*X$
for the fixed reference $(p,p)$--form at $x$, one has
\begin{equation}\label{eq:pp-projection}
%
\min_{\substack{v\in \mathcal H_x,\ \|v\|=1\\ \lambda\ge 0,\ \mu\in\mathbb{R}}}
\bigl\|\alpha_{x}^{(p,p)}-\gamma_{\harm,x}-\mu\,\tfrac{\omega_x^p}{p!}-\lambda\,\xi_{v}(x)\bigr\|^{2}
\le
C_{0}\,\bigl\|(\alpha_{x}^{(p,p)}-\gamma_{\harm,x})_{\prim}\bigr\|^{2},
\tag{23}
\end{equation}
where $\xi_{v}(x):=\mathcal I_x^{-1}(v\otimes v^{*})\in \Lambda^{p,p}T_x^*X$ is the rank--one
\emph{positive} ray produced by the Hermitian model $\mathcal I_x$ from
Lemma~\ref{lem:hermitian-model}.

\end{proposition}


\begin{proof}
Let $\beta:=\alpha_{x}^{(p,p)}-\gamma_{\harm,x}$ and set $H:=\mathcal I_x(\beta)\in \Herm(\mathcal H_x)$.
Decompose $H=\frac{\tr H}{d}\,\Id + H_{0}$ with $\tr(H_{0})=0$.
By Lemma~\ref{lem:hermitian-model}, the map $\mathcal I_x$ is a uniform linear isomorphism, and the primitive/traceless decompositions correspond up to universal scalars. In particular, there is $C_{\mathcal I}(n,p)\ge 1$ such that $\|H_{0}\|_{\mathrm{HS}}\le C_{\mathcal I}(n,p)\,\|\beta_{\prim}\|$.
Lemma~\ref{lem:rankone} yields $\mu\in\mathbb R$, a unit vector $v\in\mathcal H_x$, and $\lambda\ge 0$ such that
\[
\bigl\|H-\mu\,\Id-\lambda\,v\otimes v^{*}\bigr\|_{\mathrm{HS}}^{2}
\le
C_{0}\,\|H_{0}\|_{\mathrm{HS}}^{2}.
\]
Applying $\mathcal I_x^{-1}$ and using $\mathcal I_x^{-1}(\Id)=\omega_x^p/p!$ gives \eqref{eq:pp-projection}.
\end{proof}

\begin{corollary}[Pointwise rank--one PSD surrogate]\label{cor:Dpsd-pointwise}

Define the rank--one PSD surrogate distance
\[
D_{\mathrm{PSD}}(\alpha_{x})
:=
\min_{\substack{v\in \mathcal H_x,\ \|v\|=1\\ \lambda\ge 0,\ \mu\in\mathbb{R}}}
\bigl\|\alpha_{x}-\mu\,\tfrac{\omega_x^p}{p!}-\lambda\,\xi_{v}(x)\bigr\|.
\]
Then
\begin{equation}\label{eq:Dpsd-pointwise}
D_{\mathrm{PSD}}(\alpha_{x})^{2}
\le
\|\alpha_{x}^{\mathrm{off}}\|^{2}
+
C_{0}\,\|(\alpha_{x}^{(p,p)}-\gamma_{\harm,x})_{\prim}\|^{2}.
\tag{24}
\end{equation}
\smallskip

\noindent\textbf{Warning.}
This controls distance to the \emph{full} set of rank--one PSD rays (all $v\in \mathcal H_x$),
and therefore does \emph{not} by itself upper--bound the calibrated net distance $D_{\mathrm{net}}$,
which only ranges over decomposable (calibrated) rays and then discretizes them.

\end{corollary}


\begin{proof}
By definition choose $\mu\in\mathbb R$, a unit vector $v\in\mathcal H_x$, and $\lambda\ge 0$ so that
$D_{\mathrm{PSD}}(\alpha_x)=\|\alpha_x-\mu\,\omega_x^p/p!-\lambda\,\xi_v(x)\|$.
Since $\omega_x^p/p!$ and $\xi_v(x)$ are of type $(p,p)$, Lemma~\ref{lem:typesplit} gives the orthogonal decomposition
\[
\|\alpha_{x}-\mu\,\tfrac{\omega_x^p}{p!}-\lambda\,\xi_{v}(x)\|^2
=
\|\alpha_{x}^{\mathrm{off}}\|^2
+
\bigl\|\alpha_{x}^{(p,p)}-\mu\,\tfrac{\omega_x^p}{p!}-\lambda\,\xi_{v}(x)\bigr\|^2.
\]
Taking the minimum over $(\mu,v,\lambda)$ and applying Proposition~\ref{prop:pp-projection} yields \eqref{eq:Dpsd-pointwise}.
\end{proof}

\paragraph{Fixing a constant.}

The arguments above produce some constant $C_{0}(n,p)>0$ depending only on $(n,p)$.
For the remainder of the paper, we fix \emph{one such} admissible choice of $C_{0}(n,p)$
and do not attempt to optimize it.


\section{Calibration--Coercivity (Explicit) and Its Proof}
\label{sec:cal-coercivity}

Let $(X,\omega)$ be a smooth complex projective manifold and let
$\gamma\in H^{2p}(X,\R)\cap H^{p,p}(X)$ be a de~Rham class.
Denote by $\gharm$ its unique $\omega$--harmonic representative \REVMZ{\cite{Wells,Voisin02}} and by
$E(\cdot)$ the Dirichlet energy.

For each $x\in X$, let $K_p(x)\subset \Lambda^{p,p}T_x^*X$ denote the \emph{closed convex cone} of \emph{strongly positive} $(p,p)$--forms
(equivalently, the closed convex cone generated by the extremal rays associated to complex $(n-p)$--planes; cf.\ Section~\ref{sec:calibrated-grassmannian} and \REVMZ{\cite{HL82}}).
The global cone defect of a form $\alpha$ is
\[
\Defcone(\alpha)
:= \int_X \distcone(\alpha_x)^2\,d\mathrm{vol}_\omega(x),
\qquad
\distcone(\alpha_x)
:= \inf_{\beta_x\in K_p(x)} \|\alpha_x - \beta_x\|.
\]

The main estimate of this section is the following explicit calibration--coercivity inequality.

\begin{theorem}[Calibration--coercivity (cone-valued harmonic classes, explicit)]
	\label{thm:cal-coercivity}
	Assume the $\omega$--harmonic representative satisfies $\gharm(x)\in K_p(x)$ for all $x\in X$.
	Then for every smooth closed representative $\alpha\in[\gamma]$ one has
	\begin{equation}\label{eq:global-coercivity}
		E(\alpha)-E(\gharm) \;\ge\; \Defcone(\alpha).
	\end{equation}
\end{theorem}

\begin{proof}
We use standard Hodge-theoretic elliptic estimate
see \REV{\cite{Wells}}.

Since $\alpha$ and $\gharm$ represent the same class and are closed, $\alpha-\gharm=d\eta$ is exact; by Hodge orthogonality
$\langle \gharm, d\eta\rangle_{L^2}=0$, hence
\[
E(\alpha)-E(\gharm)=\|\alpha-\gharm\|_{L^2}^2.
\]
Pointwise, because $\gharm(x)\in K_p(x)$ and $K_p(x)$ is a cone,
\[
\distcone(\alpha_x)\ =\ \inf_{\beta_x\in K_p(x)}\|\alpha_x-\beta_x\|
\ \le\ \|\alpha_x-\gharm(x)\|.
\]
Squaring and integrating yields $\Defcone(\alpha)\le \|\alpha-\gharm\|_{L^2}^2$, hence \eqref{eq:global-coercivity}.
\end{proof}


\begin{remark}[On the coercivity hypothesis]\label{rem:coercivity-hypothesis}
\REVMZ{\textbf{[Technical clarification.]}}
The inequality in Theorem~\ref{thm:cal-coercivity} is \emph{purely geometric}: once the energy minimizer $\gharm$ lies in the closed convex cone $K_p(x)$ pointwise,
the cone distance is trivially controlled by the $L^2$ distance to $\gharm$.
No Hermitian spectral/projection formula is needed.

\smallskip\noindent
Conversely, if $\gharm$ fails to be cone-valued, then any statement of the form
$E(\alpha)-E(\gharm)\ge c\,\Defcone(\alpha)$ with $c>0$ cannot hold in general (apply it to $\alpha=\gharm$).
\end{remark}


% ------------------------------------------------------------
\REVMZ{\subsection*{Optional: a penalized route (alternative approach)}}

Define the penalized functional on closed representatives of $[\gamma]$ by
\[
\mathcal{F}_\lambda(\alpha) := E(\alpha) + \lambda\,\Defcone(\alpha),
\qquad \lambda \ge 0.
\]
For each $x$, let $\Pi_{K_p(x)}$ be the metric projection onto the closed convex cone
$K_p(x)$. Since $K_p(x)$ is a closed convex cone containing $0$, the metric projection satisfies the Moreau decomposition
$\alpha_x=\Pi_{K_p(x)}(\alpha_x)+\Pi_{K_p(x)^\circ}(\alpha_x)$ with orthogonality (where $K_p(x)^\circ$ is the polar cone); in particular
\[
\|\alpha_x\|^2 = \|\Pi_{K_p(x)}(\alpha_x)\|^2 + \dist\!\bigl(\alpha_x,K_p(x)\bigr)^2.
\]
Integrating,
\begin{equation}\label{eq:projection-identity}
	E(\alpha) = E\!\bigl(\Pi_K(\alpha)\bigr) + \Defcone(\alpha),
\end{equation}
where $(\Pi_K\alpha)(x):=\Pi_{K_p(x)}(\alpha_x)$.

\begin{remark}[Limitation of pointwise projection]
\REVMZ{\textbf{[Technical clarification.]}}
While \eqref{eq:projection-identity} is a valid pointwise identity, the
fiberwise projection $\Pi_K(\alpha)$ does \emph{not} preserve closedness:
$d(\Pi_K(\alpha)) \neq 0$ in general, so $\Pi_K(\alpha)$ is not a closed
representative of $[\gamma]$.  Thus the naive descent argument
$\mathcal{F}_\lambda(\Pi_K(\alpha)) < \mathcal{F}_\lambda(\alpha)$ does not
produce a feasible competitor within the constraint set of closed forms.
A rigorous penalized approach would require combining pointwise projection
with a global Hodge-type correction (e.g., projecting onto the space of
closed forms after each step) and establishing that the resulting scheme
converges.  We do not pursue this route here; the main proof uses the
explicit SYR/microstructure construction in \REVMZ{Section~\ref{sec:realization}}.
\end{remark}

% ============================================================
\section{From Cone--Valued Minimizers to Calibrated Currents}\label{sec:realization}
% ============================================================

Let $\varphi=\omega^{p}/p!$ and let $\psi:=*\varphi=\omega^{n-p}/(n-p)!$ denote the
K\"ahler (Wirtinger) calibration \REVMZ{\cite{HL82}} of $\C$--dimension $(n-p)$ planes.  Set $k:=2n-2p$ and write $A=\mathrm{PD}(m[\gamma])\in H_{k}(X,\Z)$ for some $m\ge 1$.

% ------------------------------------------------------------

\subsection*{Spine theorem: a single checkable quantitative output}

\begin{theorem}[Quantitative almost--mass--minimizing cycles ]
\label{thm:spine-quantitative}
Let $(X,\omega)$ be a smooth projective K\"ahler manifold of complex dimension $n$, fix $1\le p\le n$, and set $\psi=\omega^{n-p}/(n-p)!$ and $k:=2n-2p$.
Let $[\gamma]\in H^{2p}(X,\Q)\cap H^{p,p}(X)$ admit a smooth closed cone--valued representative $\beta$.
Choose an integer $m\ge 1$ so that $m[\gamma]\in H^{2p}(X,\Z)$, and set
\[
A:=\mathrm{PD}(m[\gamma])\in H_k(X,\Z),
\qquad
c_0:=\langle A,[\psi]\rangle
=m\int_X \beta\wedge\psi.
\]
Fix any sequence of mesh scales $h_j\to 0$. For each $j$, apply the constructions assembled in the H1/H2 packages (Propositions~\ref{prop:h1-package} and \ref{prop:transport-flat-glue-weighted}) together with global coherence (Proposition~\ref{prop:global-coherence-all-labels}) to obtain, at scale $h_j$:
\begin{enumerate}
\item[\textnormal{(H1)}] a \emph{calibrated sheet--sum} integral current $S_j=\sum_Q S_{Q,j}$ built from holomorphic pieces in cells $Q$ (hence $\Mass(S_j)=\langle S_j,\psi\rangle$) and satisfying the single quantitative budget condition
\[
\Mass(S_j)=\langle S_j,\psi\rangle\ \le\ c_0+o(1);
\]
\item[\textnormal{(H2)}] a \emph{gluing current} $G_j$ and a fixed-class \emph{period/rounding choice} such that the corrected current
\[
T_j:=S_j-G_j
\]
satisfies $\partial T_j=0$ and $[T_j]=A$, and $G_j$ obeys the explicit mass bound
\[
\Mass(G_j)\ \le\ C_X\,h_j^2\sum_Q\ \sum_{a\in\mathcal S(Q,j)} m_{Q,a}^{\frac{k-1}{k}},
\qquad m_{Q,a}:=\Mass([Y^{Q,a}]\llcorner Q),
\]
where $\mathcal S(Q,j)$ indexes the holomorphic pieces in $Q$ at scale $h_j$ and $C_X$ depends only on $(X,\omega,n,p)$.
\end{enumerate}
Then the \emph{mass defect} satisfies the brutally simple bound
\[
0\ \le\ \Mass(T_j)-c_0\ \le\ 2\,\Mass(G_j).
\]
In particular, if the per-cell complexity satisfies $|\mathcal S(Q,j)|\le \Lambda_j$ for all $Q$, then
\[
\Mass(G_j)\ \le\ C_X\,c_0^{\frac{k-1}{k}}\,h_j^{\,2-\frac{2n}{k}}\,\Lambda_j^{1/k},
\]
so $\Mass(T_j)\to c_0$ whenever $h_j^{\,2-\frac{2n}{k}}\Lambda_j^{1/k}\to 0$.
\end{theorem}

\begin{proof}
Since $\psi$ is closed and $[T_j]=A$, the pairing is topological:
\[
\langle T_j,\psi\rangle=\langle [T_j],[\psi]\rangle=\langle A,[\psi]\rangle=c_0.
\]
The calibration inequality gives $\Mass(T_j)\ge \langle T_j,\psi\rangle=c_0$, hence $\Mass(T_j)-c_0\ge 0$.
Write $S_j=T_j+G_j$.  Since $S_j$ is $\psi$--calibrated, $\Mass(S_j)=\langle S_j,\psi\rangle$.
Thus, using the triangle inequality for mass and that $\psi$ has comass $\le 1$,
\[
\begin{aligned}
\Mass(T_j)
&\le \Mass(S_j)+\Mass(G_j) \\
&= \langle T_j+G_j,\psi\rangle+\Mass(G_j) \\
&\le c_0+\bigl|\langle G_j,\psi\rangle\bigr|+\Mass(G_j) \\
&\le c_0+2\,\Mass(G_j).
\end{aligned}
\]
which gives the stated defect estimate.

For the complexity bound, write $M_Q:=\sum_{a\in\mathcal S(Q,j)} m_{Q,a}=\Mass(S_{Q,j})$.
By H\"older/concavity,
\[
\sum_{a\in\mathcal S(Q,j)} m_{Q,a}^{\frac{k-1}{k}}
\ \le\ M_Q^{\frac{k-1}{k}}\,|\mathcal S(Q,j)|^{1/k}
\ \le\ M_Q^{\frac{k-1}{k}}\Lambda_j^{1/k}.
\]
Summing over $Q$, using $\sum_Q M_Q=\Mass(S_j)\le c_0+o(1)$, and that the number of $h_j$--cells is $\lesssim h_j^{-2n}$ gives
\[
\sum_Q M_Q^{\frac{k-1}{k}}
\ \le\ (\#\{Q\})^{1/k}\Bigl(\sum_Q M_Q\Bigr)^{\frac{k-1}{k}}
\ \lesssim\ h_j^{-\frac{2n}{k}}\,c_0^{\frac{k-1}{k}}.
\]
Substituting into (H2) yields $\Mass(G_j)\lesssim c_0^{\frac{k-1}{k}}h_j^{2-\frac{2n}{k}}\Lambda_j^{1/k}$.
\end{proof}


\begin{remark}[Where to look for (H1)--(H2) in this manuscript]
\REVMZ{\textbf{[Proof roadmap.]}}
In our implementation, (H1) is supplied by the projective tangential approximation / holomorphic patch manufacturing package
(Bergman/peak-section control and finite-template realization; see Lemma~\ref{lem:bergman-control} and the local sheet construction in Theorem~\ref{thm:local-sheets}).
The gluing estimate in (H2) is obtained by combining transport-to-filling on faces (Proposition~\ref{prop:transport-flat-glue-weighted}) with the slice boundary shrinkage
estimate on smooth uniformly convex cells (Lemma~\ref{lem:uniformly-convex-slice-boundary}), packaged globally as Corollary~\ref{cor:global-flat-weighted},
and then enforcing the required face-level matching and global period constraints via the corner-exit vertex-template coherence mechanism
(packaged in Proposition~\ref{prop:global-coherence-all-labels}, with the flat-norm filling estimate recorded in Proposition~\ref{prop:glue-gap}).
\end{remark}


\subsection*{Parameter conventions}
\label{sec:parameter-schedule}
We fix $[\gamma]\in H^{2p}(X,\Q)\cap H^{p,p}(X)$ and choose a smooth closed $(p,p)$-representative $\beta$.
All auxiliary parameters are chosen in the following order: first choose the integer $m$ (large enough to meet the
analytic and integrality requirements that arise later); then choose the mesh scale $h$ sufficiently small (depending on $m$
and the error tolerances); finally choose the local tolerances and holomorphic scales used in the constructions.
Whenever several ``choose small/large'' conditions occur, we enforce them by decreasing $h$ and increasing $m$ in this order.

\subsection*{H1/H2 packaged at the point of use (for Theorem~\ref{thm:spine-quantitative})}

\begin{proposition}[H1 package: local holomorphic multi-sheet manufacturing]\label{prop:h1-package}
In the parameter schedule of \S\ref{sec:parameter-schedule}, for each mesh cell $Q$ and each direction family prescribed by the local Carath\'eodory data of $\beta$ on $Q$,
Theorem~\ref{thm:local-sheets} and the projective holomorphic manufacturing machinery supply the required calibrated sheet--sum $S_Q$ satisfying $\Mass(S_Q)=\langle S_Q,\psi\rangle$
with quantitative within-direction disjointness, slope, and budget control.  Thus the hypothesis \textnormal{(H1)} in Theorem~\ref{thm:spine-quantitative} holds in this manuscript.


\noindent\textbf{tightening (make \textnormal{(H1)} verifiable from cited inputs).}
Fix an index $j$ and the mesh scale $h_j$.
For each cell $Q$, let $\{(P_{Q,a},m_{Q,a})\}_{a\in\mathcal S(Q,j)}$ denote the local Carath\'eodory data for the cone field $\beta$ on $Q$
(Lemma~\ref{lem:caratheodory-general}, as organized in \S\ref{sec:parameter-schedule}).
For each $a\in\mathcal S(Q,j)$, Proposition~\ref{prop:finite-template} realizes the corresponding finite translation template by holomorphic
complete intersections $Y^{Q,a}$, and Proposition~\ref{prop:cell-scale-linear-model-graph} (with Lemma~\ref{lem:sliver-stability})
upgrades the local model to a single $C^1$ graph over $P_{Q,a}$ on all of $Q$.
Define the per-cell sheet current and its global sum by
\[
S_{Q,j}:=\sum_{a\in\mathcal S(Q,j)} [Y^{Q,a}]\llcorner Q,
\qquad
S_j:=\sum_Q S_{Q,j}.
\]
Then each $S_{Q,j}$ is $\psi$--calibrated (hence $\Mass(S_{Q,j})=\langle S_{Q,j},\psi\rangle$), and the within-direction disjointness
from Theorem~\ref{thm:local-sheets} ensures additivity of the $\psi$--mass inside each cell.
Summing over $Q$ and using that the Carath\'eodory weights encode the $\psi$--pairing with $\beta$ at scale $h_j$
(as in the schedule), we obtain
\[
\Mass(S_j)=\langle S_j,\psi\rangle\ \le\ c_0+o(1),
\]
so $S_j$ satisfies the hypothesis \textnormal{(H1)} in Theorem~\ref{thm:spine-quantitative}.

\end{proposition}


\begin{proposition}[H2 package: global face coherence and gluing (corner-exit route)]\label{prop:h2-package}

In the parameter schedule of \S\ref{sec:parameter-schedule} (with fixed $m$ and $h_j\downarrow 0$), the corner-exit vertex-template coherence package yields
\begin{itemize}
\item per-face transverse matching (Proposition~\ref{prop:global-coherence-all-labels}, possibly with prefix-edits),
\item global flat-norm estimate $\mathcal F(\partial T^{\mathrm{raw}})\to 0$ for $p<n/2$ (Corollary~\ref{cor:global-flat-weighted}); in the borderline case $p=n/2$, the current bounds yield only $\mathcal F(\partial T^{\mathrm{raw}})\lesssim \varepsilon^{-1}$ and require an additional closure input (\REVMZ{Lemma~\ref{lem:borderline-p-half}}),
\item filling with vanishing mass (Proposition~\ref{prop:glue-gap}).
\end{itemize}
The exact-class conclusion is enforced by Proposition~\ref{prop:cohomology-match} together with the per-face matching/transport mechanism (Proposition~\ref{prop:integer-transport}). For $p=n/2$, this still requires an additional closure input to guarantee that the filling current $U_\epsilon$ in Proposition~\ref{prop:cohomology-match} can be taken with arbitrarily small mass (\REVMZ{Lemma~\ref{lem:borderline-p-half}}).
rather than relying on a decay exponent in $h$.
Thus the hypothesis \textnormal{(H2)} in Theorem~\ref{thm:spine-quantitative} holds for $p<n/2$ under the present estimates; the borderline case $p=n/2$ is reduced to supplying one of the additional closure inputs listed in \REVMZ{Lemma~\ref{lem:borderline-p-half}}.
\end{proposition}



% ------------------------------------------------------------
\subsection*{Closure from almost-calibrated sequences}

\begin{lemma}[Flat limits of cycles are cycles]\label{lem:flat_limit_of_cycles_is_cycle}
\label{lem:cone-closed-seq}
\label{lem:flat-compactness-setup}
\REVMZ{(See \cite{Fed69,Sim83} for general compactness results.)}
Let $T_k$ be integral currents of dimension $m$ on $X$ with $\partial T_k=0$ and
$\sup_k\Mass(T_k)<\infty$.  Assume $T_k\to T$ in the flat norm.  Then $T$ is an
integral $m$--cycle, i.e.\ $\partial T=0$.
\end{lemma}
\begin{proof}
The boundary operator is continuous with respect to flat convergence, hence
$\partial T_k\to \partial T$ in flat norm.  Since $\partial T_k=0$ for all $k$, we get
$\partial T=0$.
Therefore $T$ is integral (closure under flat convergence with uniform mass and boundary mass; see e.g.\ \cite[\S4.2]{Fed69}) and, by the first part, a cycle.
\end{proof}


\begin{theorem}[Realization from almost--calibrated sequences]\label{thm:realization-from-almost}
\label{thm:cone-closed-FF}
Let $(X^n,\omega)$ be a compact K\"ahler manifold, fix $1\le p\le n-1$, and set
\[
\psi := \frac{\omega^{\,n-p}}{(n-p)!}\in \Omega^{2n-2p}(X).
\]
Let $\gamma\in H^{p,p}(X)\cap H^{2p}(X;\Z)$ be an integral Hodge class and choose $m\in\N$
so that $A:=\mathrm{PD}(m[\gamma])\in H_{2n-2p}(X;\Z)$ is an integral homology class.
Define the cohomological lower bound
\[
c_0 := \int_X m\,\gamma\wedge \psi \;=\; \langle A,[\psi]\rangle .
\]
Assume there exists a sequence of integral $(2n-2p)$--cycles $\{T_k\}_{k\ge 1}$ with
$\partial T_k=0$ and $[T_k]=A$ in $H_{2n-2p}(X;\Z)/\mathrm{Tor}$ such that the calibration defect tends to zero:
\[
\Def_{\mathrm{cal}}(T_k)\ :=\ \Mass(T_k)-\langle T_k,\psi\rangle \ \longrightarrow\ 0.
\]
(Equivalently, since $\psi$ is closed and the homology class is fixed in $H_{2n-2p}(X;\Z)/\mathrm{Tor}$, one has $\langle T_k,\psi\rangle=c_0$ for all $k$ and thus
$\Def_{\mathrm{cal}}(T_k)\to 0$ iff $\Mass(T_k)\to c_0$.)
Then there exists an integral $(2n-2p)$--cycle $T$ with $\partial T=0$ and $[T]=A$ in $H_{2n-2p}(X;\Z)/\mathrm{Tor}$ such that
\[
\Mass(T)=\langle T,\psi\rangle=c_0.
\]
In particular, $T$ is $\psi$--calibrated.  Consequently $T$ is a $d$--closed positive locally integral current of bidimension $(p,p)$,
hence a holomorphic chain:
\[
T=\sum_{j=1}^N m_j [V_j],
\]
with $m_j\in\mathbb{N}$ and $V_j\subset X$ irreducible complex analytic subvarieties of codimension $p$.
If $X$ is projective, each $V_j$ is algebraic (Chow/GAGA (see \REV{\cite{Hartshorne77,GH78,Serre56}})), and therefore $[\gamma]\in H^{2p}(X;\Q)$ is an
algebraic cohomology class.
\end{theorem}


\begin{proof}
Since $[T_k]=A$ in $H_{2n-2p}(X;\Z)/\mathrm{Tor}$ is fixed and $\psi$ is closed, the pairing is constant:
\(
\langle T_k,\psi\rangle=\langle A,[\psi]\rangle=c_0
\)
for all $k$.  The hypothesis $\Def_{\mathrm{cal}}(T_k)\to 0$ therefore gives $\Mass(T_k)\to c_0$, hence $\sup_k\Mass(T_k)<\infty$.
\REVMZ{By the Federer--Fleming compactness theorem \cite[Thm~4.2.17]{Fed69}:}
\emph{Federer--Fleming compactness applies to \emph{integral} currents on a compact Riemannian manifold once $\sup_k\bigl(\Mass(T_k)+\Mass(\partial T_k)\bigr)<\infty$ (here $\partial T_k=0$ and $\Mass(T_k)\to c_0$).}
 compactness for integral currents on a compact manifold (e.g. \cite{Fed69}), after passing to a subsequence
we may assume $T_k\to T$ in the flat norm for some integral current $T$.
Because $\partial T_k=0$ for all $k$, the limit is also a cycle, $\partial T=0$ (cf.\ Lemma~\ref{lem:flat_limit_of_cycles_is_cycle}),
and the homology class agrees with $A$ in real homology (equivalently, in $H_{2n-2p}(X;\Z)$ modulo torsion). Indeed, for any smooth closed $(2n-2p)$--form $\eta$ on $X$ we have $\langle T_k,\eta\rangle=\langle A,[\eta]\rangle$ for all $k$; by flat convergence, $\langle T,\eta\rangle=\lim_k\langle T_k,\eta\rangle=\langle A,[\eta]\rangle$.

Flat convergence implies convergence against smooth forms, so
\(
\langle T,\psi\rangle=\lim_k \langle T_k,\psi\rangle=c_0.
\)
Lower semicontinuity of mass under flat convergence gives
\(
\Mass(T)\le \liminf_k \Mass(T_k)=c_0.
\)
On the other hand, since $\psi$ has comass $\le 1$ (Wirtinger inequality \REVMZ{\cite{Fed69,HL82}}),
\(
\langle T,\psi\rangle \le \Mass(T).
\)
Combining these inequalities yields
\(
\Mass(T)=\langle T,\psi\rangle=c_0,
\)
so $T$ is $\psi$--calibrated (equivalently, $\Def_{\mathrm{cal}}(T)=0$).
\REVMZ{By Harvey--Lawson \cite[Thm~4.2]{HL82}:} For the K\"ahler/Wirtinger calibration, $\psi$--calibration forces the approximate tangent planes of $T$ to be complex $(n-p)$--planes with the standard complex orientation (Harvey--Lawson
\emph{For the K\"ahler/Wirtinger calibration, $\psi$ is closed and comass-$1$; a $\psi$--calibrated \emph{integral} current has a.e.\ tangent plane complex and is strongly positive of bidimension $(p,p)$ (Harvey--Lawson).}
~\cite{HL82}). Hence $T$ is a \emph{positive} current of bidimension $(p,p)$.
Since $T$ is an integral \emph{cycle}, $\partial T=0$, it is $d$--closed as a current.
King's theorem
\emph{To invoke King/Harvey--Lawson, we use that $T$ is a positive $d$--closed locally integral current of bidimension $(p,p)$; then $T$ is a holomorphic chain with integer multiplicities.}
~\cite{King71} (in the form ``positive $d$--closed locally integral $(p,p)$--currents are holomorphic chains'') then yields
\(
T=\sum_j m_j[V_j],
\)
with $m_j\in\Z_{>0}$ and $V_j$ irreducible analytic subvarieties of codimension $p$.
If $X$ is projective, each $V_j$ is algebraic by Chow/GAGA
\emph{Chow/GAGA require $X$ projective (or more generally Moishezon); then complex analytic subvarieties of $X$ are algebraic.}
 (Chow/GAGA (see \REV{\cite{Hartshorne77,GH78,Serre56}})).
\end{proof}








\begin{lemma}[Almost--calibrated limits are calibrated]\label{lem:limit_is_calibrated}
Let $\psi$ be a smooth closed form with comass $\le 1$.
Let $T_k$ be integral currents with $\sup_k\Mass(T_k)<\infty$ and $T_k\rightharpoonup T$ weakly.
If the calibration deficits $\Def_{\mathrm{cal}}(T_k):=\Mass(T_k)-\langle T_k,\psi\rangle$ satisfy $\Def_{\mathrm{cal}}(T_k)\to 0$,
then $T$ is $\psi$--calibrated, i.e. $\Mass(T)=\langle T,\psi\rangle$.
\end{lemma}

\begin{proof}
Weak convergence gives $\langle T_k,\psi\rangle\to \langle T,\psi\rangle$.
Lower semicontinuity of mass under weak convergence (see e.g. \cite[Ch.~XIV]{LangGmT} or \cite{Sim83}) yields
$\Mass(T)\le \liminf_k \Mass(T_k)=\liminf_k\bigl(\langle T_k,\psi\rangle+\Def_{\mathrm{cal}}(T_k)\bigr)
=\langle T,\psi\rangle$.
On the other hand, $\mathrm{comass}(\psi)\le 1$ implies $|\langle T,\psi\rangle|\le \Mass(T)$ and hence $\langle T,\psi\rangle\le \Mass(T)$ (see e.g. \cite[Lemma~3.5]{HL82}).
Thus $\Mass(T)=\langle T,\psi\rangle$, so $T$ is $\psi$--calibrated.
\end{proof}





\begin{remark}[How to use Theorem~\ref{thm:realization-from-almost}]
\label{rem:cone-closed-what-used}
\REVMZ{\textbf{[Proof roadmap.]}}
	Theorem~\ref{thm:realization-from-almost} is an abstract closure principle: once one has a fixed-class sequence of integral cycles whose masses approach the cohomological lower bound $c_0$,
	the limit is automatically $\psi$--calibrated and hence analytic (Harvey--Lawson).
	The remainder of this section explains how to build such almost--calibrated integral cycles starting from a smooth closed cone--valued form $\beta$:
	first in classical situations (e.g.\ codimension one, complete intersections, and other LICD cases), and then (in general codimension) via the microstructure/gluing theorem proved below using the projective tangential approximation framework.
	
\end{remark}


% ------------------------------------------------------------
\subsection*{complete realizability in codimension one (Lefschetz (1,1))}


\begin{theorem}[Codimension one (Lefschetz $(1,1)$)]\label{thm:codim1}
\label{thm:lefschetz11}
	If $p=1$ and $[\gamma]\in H^{1,1}(X,\Q)$ on a smooth projective $X$, then
	$[\gamma]$ is algebraic.
\end{theorem}

\begin{proof}
Choose $m\ge 1$ so that $m[\gamma]\in H^{1,1}(X,\Z)$.
By the Lefschetz $(1,1)$ theorem \REV{(see, e.g., \cite{Voisin02,GH78,Wells})}
\emph{Lefschetz $(1,1)$ applies once $X$ is compact K\"ahler and $m[\gamma]\in H^{1,1}(X)\cap H^2(X;\Z)$; choose $m$ clearing denominators for a rational Hodge class.}
, there exists a holomorphic line bundle $L\to X$ with
\(
c_1(L)=m[\gamma].
\)
Equivalently, $m[\gamma]$ lies in the N\'eron--Severi group and is represented by an algebraic divisor class.
Thus the homology class $\mathrm{PD}(m[\gamma])\in H_{2n-2}(X,\Z)$ is represented by a codimension-one algebraic cycle
(\emph{a divisor with integer multiplicities}), and dividing by $m$ shows $[\gamma]$ is algebraic as a rational class.
\end{proof}


\begin{remark}[Mass equality in the effective codimension-one case]
\REVMZ{\textbf{[Technical clarification.]}}
If in addition $m[\gamma]$ is represented by an \emph{effective} divisor $D$ (so $D$ is a complex hypersurface with positive orientation),
then the current $[D]$ is $\psi$--calibrated by $\psi=\omega^{n-1}/(n-1)!$ and satisfies the exact mass identity
\(
\Mass([D])=\int_D\psi=\langle \mathrm{PD}(m[\gamma]),[\psi]\rangle.
\)
In particular, the constant sequence $T_k:=[D]$ is an almost-calibrated realizing sequence with $\Mass(T_k)$ equal to the cohomological pairing.
\end{remark}



% ------------------------------------------------------------
\subsection*{Complete--intersection realizability (very ample slicing)}

\begin{proposition}[Complete intersections]\label{prop:complete-intersection}
\label{thm:complete-intersection}
	Suppose $[\gamma]\in H^{p,p}(X,\Q)$ can be written as a rational linear
	combination of cohomology classes of complete intersections of $p$ very ample
	divisors\REV{ (see, e.g., \cite{Hartshorne77,Lazarsfeld04I,Lazarsfeld04II,Fulton98})}. Then there exists a sequence of integral cycles in the class
	$\mathrm{PD}(m[\gamma])$ with masses tending to $c_0$, and the limit is a calibrated
	sum of complex subvarieties realizing $[\gamma]$.
\end{proposition}


\begin{proof}[Idea]
	Very ample divisors are represented by smooth hypersurfaces \REVMZ{\cite{GH78,Hartshorne77}} calibrated by
	$\omega^{n-1}/(n-1)!$. Intersections of $p$ such hypersurfaces produce smooth
	complex submanifolds of codimension $p$ calibrated by $\psi=\omega^{n-p}/(n-p)!$ \REVMZ{(Wirtinger calibration \cite{HL82})}.
	Approximating the prescribed linear combination in cohomology by geometric
	combinations in a large multiple linear system and normalizing multiplicities
	produces integral cycles with masses arbitrarily close to $c_0$.
\end{proof}

% ------------------------------------------------------------
\subsection*{General realizability: a stationarity hypothesis}

\begin{definition}[Stationary Young--measure realizability (SYR)]\label{def:syr}
We say a cone--valued smooth closed $(p,p)$--form $\beta$ (representing the rational Hodge class $[\gamma]$)
is \emph{SYR--realizable} if there exists a sequence of integral $(2n-2p)$--cycles $T_k$ such that
\begin{enumerate}
\item $\partial T_k=0$ and $[T_k]=\mathrm{PD}(m[\gamma])$ in $H_{2n-2p}(X;\Z)/\mathrm{Tor}$ (equivalently in $H_{2n-2p}(X;\Q)$) for some fixed integer $m\ge 1$ (independent of $k$), and
\item the \emph{calibration defect} satisfies
\[
\Def_{\mathrm{cal}}(T_k)\ :=\ \Mass(T_k)-\langle T_k,\psi\rangle\ \longrightarrow\ 0.
\]
\end{enumerate}
Equivalently, since $\psi$ is closed and $[T_k]=\mathrm{PD}(m[\gamma])$ in $H_{2n-2p}(X;\Z)/\mathrm{Tor}$, one has the exact pairing identity
\[
\langle T_k,\psi\rangle=\bigl\langle [T_k],[\psi]\bigr\rangle
=\bigl\langle \mathrm{PD}(m[\gamma]),[\psi]\bigr\rangle
=m\int_X \beta\wedge\psi \;=:\; c_0
\qquad\text{for all }k,
\]
and therefore SYR is equivalent to $\Mass(T_k)\to c_0$.

\end{definition}


\begin{remark}\label{rem:syr-criterion}
Since $\|\psi\|_*\le 1$ one has $\langle T_k,\psi\rangle\le \Mass(T_k)$. Moreover $\psi$ is closed and $[T_k]=\operatorname{PD}(m[\gamma])$ in $H_{2n-2p}(X;\Z)\text{ modulo torsion}$, so the pairing $\langle T_k,\psi\rangle$ is the constant $c_0:=m\int_X\beta\wedge\psi$. Hence $\Def_{\mathrm{cal}}(T_k)\to0$ is equivalent to $\Mass(T_k)\to c_0$.
\end{remark}



{\noindent\textbf{Notation warning.}
The symbol $m$ is reserved for the \emph{fixed} cohomology multiplier in $\mathrm{PD}(m[\gamma])$ from
Definition~\ref{def:syr}.  All holomorphic/H\"ormander/Bergman constructions (including the $\mhol^{-1/2}$ scale)
use the independent tensor-power parameter $\mhol\in\mathbb{N}$ for $L^{\otimes \mhol}$.}




\begin{theorem}[Calibrated realization under SYR]\label{thm:syr}
\label{prop:realizability-stationary}
Assume $\beta$ is SYR--realizable in the sense of Definition~\ref{def:syr}, and let $\psi=\omega^{n-p}/(n-p)!$.
Then there exists an integral $(2n-2p)$--cycle $T$ with $\partial T=0$ and $[T]=\mathrm{PD}(m[\gamma])$ in $H_{2n-2p}(X;\Z)/\mathrm{Tor}$ such that
\[
\Mass(T)=\langle T,\psi\rangle=\bigl\langle \mathrm{PD}(m[\gamma]),[\psi]\bigr\rangle.
\]
In particular, $T$ is $\psi$--calibrated.  For the K\"ahler calibration $\psi=\omega^{n-p}/(n-p)!$, calibrated integral currents are $d$--closed and positive of bidimension $(p,p)$ (see \cite{HL82}).  By King's theorem on positive closed locally integral currents \cite{King71}, $T$ is a holomorphic chain
\[
T=\sum_j m_j[V_j],
\qquad m_j\in\mathbb{N},
\]
where $V_j\subset X$ are irreducible complex analytic subvarieties of codimension $p$.  If, moreover, $X$ is projective, then each $V_j$ is algebraic by Chow/GAGA (Chow/GAGA (see \REV{\cite{Hartshorne77,GH78,Serre56}})), so $[\gamma]\in H^{2p}(X;\Q)$ is an algebraic class.
\end{theorem}



\noindent\textbf{Clarification (holomorphic-chain conclusion).}
Once Theorem~\ref{thm:realization-from-almost} produces an \emph{integral} cycle $T$ with
$\Mass(T)=\langle T,\psi\rangle$ for $\psi=\omega^{n-p}/(n-p)!$, the current $T$ is $\psi$--calibrated.
By Wirtinger/Harvey--Lawson calibrated-geometry results \cite{HL82}, a $\psi$--calibrated integral current is strongly positive and
$d$--closed of the correct Hodge type. King's theorem then yields the holomorphic-chain representation
$T=\sum_j m_j[V_j]$ \cite{King71}. (Projective $\Rightarrow$ algebraic is handled separately via Chow/GAGA.)

\begin{proof}
Let $\{T_k\}$ be the SYR sequence.
By Definition~\ref{def:syr}, $\Def_{\mathrm{cal}}(T_k)\to 0$ and the homology classes $[T_k]=\mathrm{PD}(m[\gamma])$ are fixed.
Since $\psi$ is closed, the pairing $\langle T_k,\psi\rangle$ depends only on the homology class, so
\[
\langle T_k,\psi\rangle=\bigl\langle \mathrm{PD}(m[\gamma]),[\psi]\bigr\rangle=:c_0
\qquad\text{for all }k.
\]
Hence $\Mass(T_k)=\Def_{\mathrm{cal}}(T_k)+\langle T_k,\psi\rangle\to c_0$.
Applying Theorem~\ref{thm:realization-from-almost} to the fixed--class sequence $\{T_k\}$ yields an integral cycle $T$ with $[T]=\mathrm{PD}(m[\gamma])$ in $H_{2n-2p}(X;\Z)/\mathrm{Tor}$
in the same class with $\Mass(T)=\langle T,\psi\rangle=c_0$ and the holomorphic--chain representation
\(T=\sum_j m_j[V_j]\).
When $X$ is projective, algebraicity of analytic subvarieties follows by Chow/GAGA
\emph{Chow/GAGA require $X$ projective (or more generally Moishezon); then complex analytic subvarieties of $X$ are algebraic.}
 (see \REV{\cite{Hartshorne77,GH78,Serre56}}).
\end{proof}


\begin{remark}\label{rem:syr-remark}
	The SYR condition encodes the ``microstructure'' step in a purely geometric--measure
	framework (stationarity/compactness). The complete cases above (codimension
	one and complete intersections) provide two broad families where SYR holds
	constructively.
\end{remark}


% ------------------------------------------------------------
\subsection*{A classical sufficient criterion for SYR}

We now give a classical, fully geometric--measure--theoretic criterion under which
SYR holds, stated purely in standard language (coverings, Carath\'eodory
decompositions, isoperimetric fillings, and varifold compactness).

\begin{definition}[Locally integrable calibrated decomposition (LICD)]
	We say a smooth closed cone--valued $(p,p)$--form $\beta$ satisfies LICD if there
	exists a finite cover $\{U_\alpha\}$ of $X$ and for each $\alpha$:
	\begin{enumerate}
		\item smooth nonnegative coefficients $a_{\alpha,j}\in C^\infty(U_\alpha)$ and
		\item smooth fields of simple calibrated covectors $\xi_{\alpha,j}$ on $U_\alpha$,
	\end{enumerate}
	with $\beta=\sum_j a_{\alpha,j}\,\xi_{\alpha,j}$ on $U_\alpha$, where each
	$\xi_{\alpha,j}$ arises from a smooth integrable complex distribution of
	$(n-p)$--planes, i.e.\ through each $x\in U_\alpha$ there is a local
	$(n-p)$--dimensional complex submanifold whose oriented tangent plane is calibrated
	by $\psi$ and corresponds to $\xi_{\alpha,j}(x)$.
\end{definition}

\begin{theorem}[Classical SYR under LICD]\label{thm:classical-syr-licd}
	Let $(X,\omega)$ be smooth complex projective, $1\le p\le n$. If a smooth closed
	cone--valued $(p,p)$--form $\beta$ representing $[\gamma]$ satisfies LICD, then $\beta$
	is SYR--realizable. In particular, there exist integral cycles $T_k$ with $\partial T_k=0$,
	$[T_k]=\mathrm{PD}(m[\gamma])$ and $\Def_{\mathrm{cal}}(T_k)\to 0$ (equivalently, $\Mass(T_k)\to c_0$).
\end{theorem}

\begin{proof}[Proof (classical construction in charts)]
	Work in a single $U_\alpha$; a partition of unity reduces the global construction
	to a finite sum of local ones plus negligible overlaps.
	
	\emph{Step 1: Grid approximation and rationalization.} Fix a small mesh scale
	$\varepsilon>0$ and subordinate cubes $\{Q\}$ in a normal coordinate chart so that
	$\omega$ and $\psi$ vary by $O(\varepsilon)$ in each cell. By Carath\'eodory,
	$\beta=\sum_j a_j\,\xi_j$ with finitely many summands; approximate on each $Q$ by
	piecewise--constant smoothings
	\[
	\beta_Q \approx \sum_{j=1}^{N_Q} \theta_{Q,j}\,\xi_{Q,j},
	\qquad \theta_{Q,j}\in \Q_{\ge 0},\ \ \xi_{Q,j}\ \text{constant calibrated covectors},
	\]
	with $\sum_j \theta_{Q,j}$ bounded and the error $O(\varepsilon)$ in $C^0(Q)$.
	Write $\theta_{Q,j}=N_{Q,j}/M_Q$ with $N_{Q,j}\in\N$.
	
	\emph{Step 2: Local lamination by calibrated leaves.} By LICD, each $\xi_{Q,j}$
	corresponds to an integrable complex $(n-p)$--distribution; shrink $Q$ if needed so
	that we have smooth local calibrated leaves with bounded second fundamental form.
	Fix $j$. Work in a local foliation box for the integrable distribution underlying $\xi_{Q,j}$.
	Choose $N_{Q,j}$ local plaques (allowing repetition) contained in $Q$ after trimming a collar of width $O(\varepsilon)$.
	For fixed $j$ these plaques can be taken pairwise disjoint inside the foliation box; for different $j$ they may intersect transversely.
	This is harmless because we form an integral current by summing the plaques with multiplicity. Define $S_Q$ as this sum.
	resulting current $S_Q$ has tangent planes calibrated by $\psi$ almost everywhere
	in $Q$ and satisfies
	\[
	\Mass(S_Q) = \int S_Q\,\psi = \sum_j N_{Q,j}\int_{\mathrm{leaf}_{Q,j}}\psi
	= M_Q\int_Q \sum_j \theta_{Q,j}\,\langle \xi_{Q,j},\psi\rangle \,d\vol + O(\varepsilon\,|Q|),
	\]
	where the error arises from leaf boundaries near $\partial Q$ and the
	metric--calibration variation $O(\varepsilon)$. Since $\xi_{Q,j}$ are calibrated,
	$\langle\xi_{Q,j},\psi\rangle=1$ pointwise, hence $\Mass(S_Q)=M_Q\int_Q \sum_j
	\theta_{Q,j}\,d\vol + o_\varepsilon(1)$.
	
\emph{Step 3: Closure by isoperimetric filling.} The sum $\sum_Q S_Q$ has small
	boundary concentrated on cell interfaces with $\Mass(\partial \sum_Q S_Q)\lesssim
	C\,\varepsilon$ (uniform density and bounded geometry). By the isoperimetric
	inequality \REVMZ{\cite[Thm~6.1]{FF60}} on compact Riemannian manifolds and the Federer--Fleming Deformation
	Theorem \REVMZ{\cite[4.2.9]{Fed69}}, there exists a correction current $R_\varepsilon$ with
	$\partial R_\varepsilon = -\partial \sum_Q S_Q$ and $\Mass(R_\varepsilon)\to 0$ as
	$\varepsilon\to 0$. Then $T_\varepsilon:=\sum_Q S_Q+R_\varepsilon$ is closed,
	rectifiable, and calibrated almost everywhere.
	
	\emph{Step 4: Homology adjustment and mass control.} Pairing with $\psi$ shows
	\[
	\Mass(T_\varepsilon)=\int T_\varepsilon\,\psi
	= \sum_Q \int_Q \sum_j \theta_{Q,j}\,d\vol + o_\varepsilon(1)
	= \int_{U_\alpha}\beta\wedge\psi + o_\varepsilon(1).
	\]
	Using a finite cover $\{U_\alpha\}$ and partition of unity yields a global cycle
	with $\Mass(T_\varepsilon)=m\int_X\beta\wedge\psi + o_\varepsilon(1)$. Adjusting
	by a null--homologous small--mass cycle (via Deformation Theorem \cite{FF60,Fed69}) yields an integral
	cycle in class $\mathrm{PD}(m[\gamma])$ with the same mass asymptotics. Varifold
	compactness then provides a convergent subsequence.  The mass asymptotics imply
	$\Def_{\mathrm{cal}}(T_\varepsilon)\to 0$, hence $\beta$ is SYR--realizable in the
	sense of Definition~\ref{def:syr}.
\end{proof}

\begin{corollary}[Closure of the program under LICD]\label{cor:closure-licd}
	If a given cone--valued representative $\beta$ satisfies LICD, then the sequence produced by Theorem~\ref{thm:classical-syr-licd}
	and Theorem~\ref{thm:realization-from-almost} yields a calibrated integral current
	realizing $[\gamma]$ as a rational algebraic cycle. In particular, the paper's
	program closes without further assumptions in codimension $1$, for complete intersections,
	and for all classes whose cone--valued representatives admit LICD.
\end{corollary}
\begin{proof}
Assume the cone-valued representative $\beta$ satisfies LICD.
By the theorem ``Classical SYR under LICD'', there exists an integer $m\ge 1$ and a sequence of integral $(2n-2p)$-cycles $T_k$
with $\partial T_k=0$, $[T_k]=\mathrm{PD}(m[\gamma])$, and
\[
\Mass(T_k)\downarrow \bigl\langle \mathrm{PD}(m[\gamma]),[\psi]\bigr\rangle
= m\int_X \beta\wedge\psi.
\]
Applying the theorem ``Realization from almost--calibrated sequences'' yields, after passing to a subsequence, a weak limit $T$
with $[T]=\mathrm{PD}(m[\gamma])$, $\Mass(T)=m\int_X \beta\wedge\psi$, and $T$ $\psi$-calibrated.
By Harvey--Lawson
 structure theory \REVMZ{\cite{HL82}}, a $\psi$-calibrated integral cycle in a K\"ahler manifold is a positive sum of currents of integration
over irreducible complex analytic subvarieties of codimension $p$.
Since $X$ is projective, Chow's theorem \REVMZ{\cite{Chow49}}
 identifies these analytic cycles with algebraic cycles.
Dividing by $m$ expresses $[\gamma]$ as a rational algebraic cycle.
\end{proof}



% ============================================================
% RIGOROUS SYR CONSTRUCTION (GENERAL p)
% ============================================================

\subsection*{Step 1: Carath\'eodory decomposition in the Hermitian model}

At each $x\in X$, identify $\Lambda^{p,p}(T_x^*X)$ with a finite-dimensional
real vector space $\mathcal{V}_x$ equipped with the inner product induced by
the K\"ahler metric, and let $K_p(x)\subset \mathcal{V}_x$ be the closed convex
cone of strongly positive $(p,p)$-forms.
Each complex $(n-p)$-plane $P\subset T_xX$ determines an extremal ray of $K_p(x)$;
let $\xi_P\in K_p(x)$ denote a chosen generator of this ray, normalized so that
$\langle \xi_P,\psi_x\rangle=1$ (equivalently $\xi_P\wedge\psi_x=\omega_x^n/n!$).

Fix the positive ``trace'' functional $t(x):=\langle \beta(x),\psi_x\rangle=\frac{\beta\wedge\psi}{\omega^n/n!}(x)$.
Then $\widehat\beta(x):=\beta(x)/t(x)$ (on the set $\{t(x)>0\}$) lies in the convex
hull of the normalized generators $\{\xi_P:\ P\in \Gr_{n-p}(T_xX)\}$.
By Carath\'eodory's theorem \REVMZ{\cite{Schneider14}} in $\R^{D}$, $\widehat\beta(x)$ can be written as a convex
combination of at most $D+1$ such generators, where $D=\dim(\mathcal{V}_x)=\binom{n}{p}^2$
is independent of $x$.






\begin{definition}[Carath\'eodory decomposition at a point]\label{def:caratheodory-decomp}
Let $x\in X$ and let $v\in \mathcal V_x$ be a strongly positive $(p,p)$--covector of comass $\le 1$
(i.e.\ $v(\xi)\ge 0$ on every positive simple $(p,p)$--vector $\xi$ and $\sup_{\xi\in K_p(x)} v(\xi)\le 1$).
A \emph{Carath\'eodory decomposition} of $v$ is a representation
\[
v=\sum_{i=1}^N a_i\, v_{P_i},
\qquad
a_i\ge 0,\ \sum_{i=1}^N a_i=1,
\]
where each $P_i\subset T_xX$ is a complex $(n-p)$--plane and $v_{P_i}$ is the corresponding extremal ray element
(e.g.\ $v_{P_i}=\psi|_{P_i}$ in the normalization of Lemma~\ref{lem:caratheodory-general}).
\end{definition}

\begin{lemma}[Uniform Carath\'eodory decomposition]\label{lem:caratheodory-general}
Let $X$ be a compact K\"ahler $n$--fold and fix $p\in\{0,\dots,n\}$.
For each $x\in X$ let $\mathcal V_x:=\Lambda^{p,p}(T_x^*X)_{\R}$ (real $(p,p)$--forms) and let
$K_p(x)\subset \mathcal V_x$ be the closed convex cone of \emph{strongly positive} $(p,p)$--forms.
For each complex $(n-p)$--plane $P\subset T_xX$ let $\xi_P\in K_p(x)$ denote the (normalized) generator of the corresponding extremal ray,
chosen so that $\langle \xi_P,\psi_x\rangle=1$.

Then there exists a number $N=N(n,p)$ (one may take $N=\dim_{\R}(\mathcal V_x)+1=\binom{n}{p}^2+1$) such that for every $x\in X$ and every
$\beta(x)\in K_p(x)$ there exist complex $(n-p)$--planes $P_{x,1},\dots,P_{x,N}\subset T_xX$ and weights
$\theta_{x,j}\ge 0$ with $\sum_{j=1}^{N}\theta_{x,j}=1$ such that
\[
\beta(x)=t(x)\sum_{j=1}^{N}\theta_{x,j}\,\xi_{P_{x,j}},
\qquad t(x):=\langle \beta(x),\psi_x\rangle.
\]
\end{lemma}

\begin{proof}
\noindent\textbf{} $H_x$ is a real vector space of dimension $D:=\dim H_x$; identifying the affine hyperplane $\{h\in H_x:\ \psi_x(h)=1\}$ with $\R^{D-1}$, Carath\'eodory applies.\par
If $\beta(x)=0$ there is nothing to prove. Otherwise $t(x)=\langle \beta(x),\psi_x\rangle>0$ and the normalized form
$\widehat\beta(x):=\beta(x)/t(x)$ lies in the affine hyperplane
\[
H_x:=\{v\in \mathcal V_x:\ \langle v,\psi_x\rangle=1\}.
\]
By the standard description of the strongly positive cone (see e.g.\ Demailly
~\cite[\S~III.1]{Demailly12}),
$K_p(x)$ is the closed convex cone generated by the extremal rays $\R_{\ge 0}\,\xi_P$ as $P$ ranges over complex $(n-p)$--planes.
Intersecting with $H_x$ shows that $\widehat\beta(x)$ lies in the compact convex set
\[
\mathrm{conv}\{\xi_P:\ P\in \Gr_{n-p}(T_xX)\}\subset H_x.
\]
Set $D:=\dim_{\R}(\mathcal V_x)=\binom{n}{p}^2$, so $\dim(H_x)=D-1$.
Carath\'eodory's theorem \REVMZ{\cite{Schneider14}} in the affine space $H_x\cong \R^{D-1}$ yields a convex representation of $\widehat\beta(x)$
using at most $D$ points from $\{\xi_P\}$; padding with zero weights gives a representation with at most $N=D+1$ points.
Multiplying by $t(x)$ gives the claimed decomposition of $\beta(x)$.
\end{proof}

\begin{remark}[What is actually used later]\label{rem:caratheodory-quantitative}
\REVMZ{\textbf{[Technical clarification.]}}
In the global construction we only require such decompositions at the finitely many cube base points $x_Q$
(Substep~4.1), so no global measurable selection or continuity-in-$x$ statement is needed for the subsequent arguments.




\end{remark}

\begin{lemma}[Lipschitz weights from a strongly convex simplex fit]\label{lem:lipschitz-qp-weights}
Let $V$ be a finite-dimensional real inner-product space and let $\xi_1,\dots,\xi_M\in V$.
Let $\Delta_M:=\{w\in\R^M:\ w_i\ge 0,\ \sum_{i=1}^M w_i=1\}$ be the probability simplex.
Fix $\lambda>0$.
For each $b\in V$ define
\[
w(b)\ :=\ \arg\min_{w\in\Delta_M}\ \frac12\Bigl\|\sum_{i=1}^M w_i\xi_i-b\Bigr\|^2+\frac{\lambda}{2}\|w\|^2.
\]
Then:
\begin{enumerate}
\item[\textnormal{(i)}] The minimizer $w(b)$ exists and is unique.
\item[\textnormal{(ii)}] The map $b\mapsto w(b)$ is Lipschitz.  Writing $A:\R^M\to V$ for the linear map
$A e_i:=\xi_i$, one has
\[
\|w(b)-w(b')\|\ \le\ \frac{\|A\|_{\mathrm{op}}}{\lambda}\,\|b-b'\|\qquad\text{for all }b,b'\in V.
\]
\end{enumerate}
\end{lemma}

\begin{proof}
Existence follows from compactness of $\Delta_M$ and continuity of the objective.
Uniqueness follows because the objective is $\lambda$--strongly convex in $w$.

Let $w=w(b)$ and $w'=w(b')$.
The first-order optimality conditions for the constrained minimization read
\[
0\ \in\ A^\top(Aw-b)+\lambda w\ +\ N_{\Delta_M}(w),
\qquad
0\ \in\ A^\top(Aw'-b')+\lambda w'\ +\ N_{\Delta_M}(w'),
\]
where $N_{\Delta_M}$ is the normal cone mapping and $A^\top$ denotes the adjoint.
Choose $\nu\in N_{\Delta_M}(w)$ and $\nu'\in N_{\Delta_M}(w')$ realizing these inclusions.
Subtract the two relations and take the inner product with $(w-w')$ to obtain
\[
\langle A^\top A(w-w'),\,w-w'\rangle\ +\ \lambda\|w-w'\|^2\ +\ \langle \nu-\nu',\,w-w'\rangle
\ =\ \langle A^\top(b-b'),\,w-w'\rangle.
\]
Since $A^\top A$ is positive semidefinite and $N_{\Delta_M}$ is monotone, one has
$\langle A^\top A(w-w'),w-w'\rangle\ge 0$ and $\langle \nu-\nu',w-w'\rangle\ge 0$.
Hence
\[
\lambda\|w-w'\|^2\ \le\ \|A^\top(b-b')\|\,\|w-w'\|
\ \le\ \|A\|_{\mathrm{op}}\|b-b'\|\,\|w-w'\|.
\]
If $w\neq w'$, cancel $\|w-w'\|$; otherwise the desired bound is trivial.  This gives
$\|w-w'\|\le (\|A\|_{\mathrm{op}}/\lambda)\,\|b-b'\|$.
\end{proof}

\begin{remark}[Stable direction labeling via a growing net]\label{rem:direction-net-qp}
\REVMZ{\textbf{[Technical clarification.]}}
In a holomorphic chart $U\subset\C^n$, the calibrated directions are precisely the complex $(n-p)$--planes.
Fix a scale $h$ and choose an $\varepsilon_h$--net $\{P_1,\dots,P_M\}\subset G_{\C}(n-p,n)$ with $\varepsilon_h\ll h$.
For each $x\in U$, let $\xi_i(x)$ denote the corresponding normalized generator in $K_p(x)$ (so $\langle \xi_i(x),\psi_x\rangle=1$).

\smallskip\noindent
Given a smooth normalized target field $b(x)=\widehat\beta(x)$, one may choose \emph{globally labeled} coefficients by applying
Lemma~\ref{lem:lipschitz-qp-weights} (with $V=\Lambda^{p,p}(T_x^*X)$ in a fixed trivialization on $U$) to obtain
weights $w_i(x)$ depending \emph{Lipschitzly} on $b(x)$.  Since $b$ varies by $O(h)$ between adjacent mesh-$h$ cells,
the weights $w_i$ vary by $O(h)$ as well.  This gives a canonical pairing of directions across neighbors (index $i=i'$)
and reduces ``stable direction labeling'' to the quantitative choice of $\varepsilon_h$ and the regularization parameter $\lambda$.
\end{remark}


% ------------------------------------------------------------
\subsection*{Step 2: Projective tangential approximation with $C^1$ control}

Fix an ample line bundle $L\to X$ with a Hermitian metric whose curvature
form equals $\omega$.  For $\mhol\in\N$ large, consider the complete linear
system $|L^{\mhol}|$.
(Parameter convention: we use $\mhol\in\N$ for the tensor power of $L$ in the analytic/Bergman layer, so the natural local scale is $\mhol^{-1/2}$. Separately, we reserve $m\in\N$ for the fixed denominator-clearing integer in the cohomological quantization layer. When both appear, we write $\mhol$ for the tensor power and $m$ for the denominator.)


\begin{lemma}[Jet surjectivity for ample powers (pointwise and for finite sets)]\label{lem:jet-surjectivity}
Let $X$ be a smooth complex projective manifold and $L\to X$ an ample line bundle.
Fix an integer $k\ge 1$.

\smallskip
\noindent\textup{(i)} For each point $x\in X$ there exists an integer ${m_{\mathrm{hol},0}(k,x)}$ such that for all {$\mhol\ge m_{\mathrm{hol},0}(k,x)$}
the natural evaluation map on $k$-jets
\[
H^0(X,L^{\mhol})\longrightarrow J^k_x(L^{\mhol})
\]
Here $J^k_x(L^{\mhol}):=\mathcal O_X(L^{\mhol})_x/\mathfrak m_x^{k+1}\mathcal O_X(L^{\mhol})_x$ denotes the $k$-jet space at $x$.

is surjective.

\smallskip
\noindent\textup{(ii)} More generally, for any \emph{finite} set $S\subset X$ there exists ${m_{\mathrm{hol},0}(k,S)}$ such that for all {$\mhol\ge m_{\mathrm{hol},0}(k,S)$}
the joint evaluation map
\[
H^0(X,L^{\mhol})\longrightarrow \bigoplus_{x\in S} J^k_x(L^{\mhol})
\]
is surjective.  In particular (taking $k=1$), prescribed values and first derivatives can be realized simultaneously at finitely many points.
\end{lemma}

\begin{proof}
For \textup{(i)}, let $\mathfrak{m}_x\subset \mathcal{O}_X$ be the maximal ideal at $x$ and consider the exact sequence
\[
0\to L^{\mhol}\otimes \mathfrak{m}_x^{k+1}\to L^{\mhol} \to
L^{\mhol}\otimes \mathcal{O}_X/\mathfrak{m}_x^{k+1}\to 0.
\]
Since $\mathfrak{m}_x^{k+1}$ is coherent and $L$ is ample, Serre vanishing (see Hartshorne~\cite[Thm.~III.5.2]{Hartshorne77}) gives
$H^1(X,L^{\mhol}\otimes \mathfrak{m}_x^{k+1})=0$ for all {$\mhol\ge m_{\mathrm{hol},0}(k,x)$} (see \cite[III, Thm.~5.2]{Hartshorne77}).
Taking global sections yields the desired surjection
\[
H^0(X,L^{\mhol})\twoheadrightarrow H^0\!\left(X,L^{\mhol}\otimes \mathcal{O}_X/\mathfrak{m}_x^{k+1}\right)\cong J^k_x(L^{\mhol}).
\]

For \textup{(ii)}, apply the same argument to the finite ``fat point'' subscheme
$Z:=\sum_{x\in S}(k{+}1)\,x$ with ideal sheaf $\mathcal{I}_Z:=\bigcap_{x\in S}\mathfrak{m}_x^{k+1}$.
Serre vanishing (Hartshorne~\cite[Thm.~III.5.2]{Hartshorne77}) gives $H^1(X,L^{\mhol}\otimes \mathcal{I}_Z)=0$ for all {$\mhol\ge m_{\mathrm{hol},0}(k,S)$}, hence
\[
H^0(X,L^{\mhol})\twoheadrightarrow H^0\!\left(X,L^{\mhol}\otimes \mathcal{O}_X/\mathcal{I}_Z\right)\cong
\bigoplus_{x\in S} J^k_x(L^{\mhol}),
\]
as claimed.
\end{proof}



\begin{lemma}[Uniform $C^1$ control on $\mhol^{-1/2}$-balls via Bergman kernels]
\label{lem:bergman-control}

{
Fix $\varepsilon>0$. Since $X$ is compact, its injectivity radius $\operatorname{inj}(X)$ is positive.
There exist a constant $c=c(X,\omega)>0$ and an integer $m_{\mathrm{hol},1}=m_{\mathrm{hol},1}(\varepsilon)$ such that for every
tensor power $\mhol\ge m_{\mathrm{hol},1}$, every $x\in X$, and every collection of $p$ covectors
$\lambda_1,\ldots,\lambda_p\in T_x^{*(1,0)}X$, there exist sections $s_1,\ldots,s_p\in H^0(X,L^{\mhol})$
with the following properties.
Work in normal holomorphic coordinates $z$ centered at $x$ and a local unitary frame of $L$ over a ball $B_r(x)$ with
$r<\operatorname{inj}(X)$. For $y\in B_{c\,\mhol^{-1/2}}(x)$ we identify $T_y^{*(1,0)}X$ with $T_x^{*(1,0)}X$
via the coordinate coframe (equivalently, via Levi--Civita parallel transport along the minimizing geodesic).
Then:
\begin{enumerate}
\item[\textnormal{(i)}] $s_i(x)=0$ and $ds_i(x)=\lambda_i$ for each $i$;
\item[\textnormal{(ii)}] on the geodesic ball $B_{c\,\mhol^{-1/2}}(x)$,
\[
\|ds_i(y)-\lambda_i\|\le \varepsilon\,\max_{1\le j\le p}\|\lambda_j\|
\quad\text{for all } y\in B_{c\,\mhol^{-1/2}}(x).
\]
\end{enumerate}
}

\end{lemma}

\begin{proof}

{
Set $M:=\max_{1\le j\le p}\|\lambda_j\|$. If $M=0$, take $s_i\equiv 0$ for all $i$.
Otherwise, replace $\lambda_i$ by $\lambda_i/M$ and at the end multiply the resulting sections by $M$.

Choose normal holomorphic coordinates $z=(z^1,\dots,z^n)$ centered at $x$ on $B_r(x)$ with $r<\operatorname{inj}(X)$ and a local
unitary frame $e_L$ of $L$ over $B_r(x)$, so that $|e_L|_h^2=e^{-\phi}$ with $\phi(0)=0$ and $d\phi(0)=0$.
Let $P_{\mhol}(z,z')$ be the Bergman kernel \REVMZ{\cite{MaMarinescu07}}
 of $(L^{\mhol},h^{\mhol})$ written in this frame.
Set rescaled variables $Z=\sqrt{\mhol}\,z$ and $Z'=\sqrt{\mhol}\,z'$.

\noindent\textbf{Correct scaling input.}
We use the standard near off-diagonal Bergman kernel expansion \emph{with the local weight included}:
for each fixed $\sigma>0$ there exists $C_{\mathrm{BK}}=C_{\mathrm{BK}}(X,\omega,\sigma,2)>0$ such that for all large $\mhol$,
uniformly for $|Z|,|Z'|\le \sigma$,
\begin{equation}\label{eq:bergman-weighted-rescaled}
\max_{|\alpha|+|\beta|\le 2}\Bigl|
\partial_Z^\alpha\partial_{\overline{Z'} }^\beta\Bigl(
\mhol^{-n}e^{-\frac{\mhol}{2}(\phi(z)+\phi(z'))}\,P_{\mhol}(z,z')-P_{\mathrm{BF}}(Z,Z')
\Bigr)\Bigr|
\ \le\ C_{\mathrm{BK}}\,\mhol^{-1/2},
\end{equation}
where $P_{\mathrm{BF}}$ is the Bargmann--Fock kernel on $\C^n$ (normalized so that
$\partial_{\overline{Z'_a}}P_{\mathrm{BF}}(Z,0)=\pi Z_a e^{-\pi|Z|^2/2}$).
\REVMZ{See Ma--Marinescu \cite[Thm~4.2.1]{MaMarinescu07} and \cite[Thm~1.1]{MaMarinescu13OffDiag} for the near-diagonal expansion with $C^2$ control implying \eqref{eq:bergman-weighted-rescaled}.}

\noindent\textbf{Peak--derivative sections.}
For each $a=1,\dots,n$ define the holomorphic section (in the frame $e_L^{\otimes \mhol}$)
\[
u_{a,\mhol}(z)
:=\frac{1}{\pi}\,\mhol^{-(n+1)/2}\,
\partial_{\overline{z'_a}} P_{\mhol}(z,z')\Big|_{z'=0}.
\]
This is holomorphic in $z$ because $P_{\mhol}$ is holomorphic in the first variable and antiholomorphic in the second.
Differentiate \eqref{eq:bergman-weighted-rescaled} at $z'=0$.
Since $d\phi(0)=0$ in normal coordinates, the $z'$--derivative of the weight contributes no leading term at $z'=0$.
After undoing the constant factor $e^{-\frac{\mhol}{2}\phi(0)}=1$, one obtains the local Taylor control on the
$\mhol^{-1/2}$--ball:
\begin{align}
u_{a,\mhol}(z) &= z^a + O(|z|^2) + O(\mhol^{-1/2}|z|), \label{eq:u-linear}\\
du_{a,\mhol}(z) &= dz^a + O(|z|) + O(\mhol^{-1/2}), \label{eq:du-linear}
\end{align}
uniformly for $|z|\le \sigma\,\mhol^{-1/2}$, where the implicit constants depend only on $(X,\omega)$ and $\sigma$.

\noindent\textbf{Fix the radius $c$ once and for all.}
Set $c:=\sigma/2$ (so $B_{c\,\mhol^{-1/2}}(x)\subset B_r(x)$ for all large $\mhol$).
Then for all $y\in B_{c\,\mhol^{-1/2}}(x)$ we have $|z(y)|\le c\,\mhol^{-1/2}$, and \eqref{eq:du-linear} gives
\[
\|du_{a,\mhol}(y)-du_{a,\mhol}(x)\|
\le C\,|z(y)| + C\,\mhol^{-1/2}
\le C'(X,\omega)\,\mhol^{-1/2}.
\]
Also $du_{a,\mhol}(x)=dz^a+O(\mhol^{-1/2})$, hence for $\mhol$ large enough the matrix
$A_{\mhol}:=(du_{a,\mhol}(x))_{a=1}^n$ is invertible with $\|A_{\mhol}^{-1}\|$ uniformly bounded.

Write each normalized covector in coordinates: $\lambda_i=\sum_{a=1}^n \lambda_{i,a}\,dz^a$ with $|\lambda_{i,a}|\le 1$.
Define coefficients $c_i=(c_{i,1},\dots,c_{i,n})^\top:=A_{\mhol}^{-1}(\lambda_{i,1},\dots,\lambda_{i,n})^\top$ and set
\[
s_i:=\sum_{a=1}^n c_{i,a}\,u_{a,\mhol}\ \in H^0(X,L^{\mhol}).
\]
Then $s_i(x)=0$ and $ds_i(x)=\lambda_i$. Moreover, for $y\in B_{c\,\mhol^{-1/2}}(x)$,
\[
\|ds_i(y)-\lambda_i\|
\le \sum_{a=1}^n |c_{i,a}|\,\|du_{a,\mhol}(y)-du_{a,\mhol}(x)\|
\le C''\,\mhol^{-1/2}.
\]
Taking $\mhol\ge m_{\mathrm{hol},1}(\varepsilon)$ so that $C''\mhol^{-1/2}\le \varepsilon$ proves (ii) for the normalized data.
Rescaling back by $M$ completes the proof.
}

\end{proof}


\begin{remark}[Compatibility of the Bergman scale with footprints and cubes]
\REVMZ{\textbf{[Technical clarification.]}}
In the local-sheet constructions we do \emph{not} require an entire cubical cell $Q$
to lie inside a single Bergman-scale ball.
What we use is only: each local footprint $E\subset Q$ that we realize holomorphically
has diameter $\diam(E)\lesssim \mhol^{-1/2}$ (typically $\diam(E)\asymp \sigma\,\mhol^{-1/2}$ for fixed $\sigma>0$),
so for any anchor point $x\in E$ we have $E\subset B_{c\,\mhol^{-1/2}}(x)$ and Lemma~\ref{lem:bergman-control}
applies on a neighborhood containing $E$.
The cube side length $h=\diam(Q)$ may be much larger than $\mhol^{-1/2}$; the only requirement is that
$c\,\mhol^{-1/2}$ is below the injectivity radius scale of $(X,\omega)$ so normal holomorphic coordinates exist.

\end{remark}




\begin{lemma}[Graph control from uniform gradient control]\label{lem:graph-from-grad}\label{lem:linear-graph-geometry}
Let $U\subset\C^n$ be a ball and let $\lambda_1,\dots,\lambda_p\in(\C^n)^*$ be complex covectors with linearly independent real and imaginary parts,
so that $\Pi:=\bigcap_{i=1}^p \ker(\lambda_i)$ is a complex $(n-p)$-plane.
Let $s_1,\dots,s_p:U\to\C$ be holomorphic functions such that $s_i(0)=0$ and
\[
\sup_{y\in U}\|ds_i(y)-\lambda_i\|\le \varepsilon
\qquad\text{for all }i=1,\dots,p,
\]
with $\varepsilon$ small compared to $\min\{\|\lambda_i\|\}$.
Then the common zero set $Y:=\{s_1=\cdots=s_p=0\}\cap U$ is a smooth complex submanifold of $U$ and, after shrinking $U$ if needed,
$Y$ is a $C^1$ graph over $\Pi$ with slope $O(\varepsilon)$.
In particular,
\[
\sup_{y\in Y}\angle(T_yY,\Pi)\le C\,\varepsilon
\]
for a constant $C$ depending only on $(n,p)$ and the conditioning of $\{\lambda_i\}$.
\end{lemma}


\begin{proof}
Let $S=(s_1,\dots,s_p):U\to\C^p$.  The differential $dS(y)$ is uniformly close to the constant complex-linear map
$\Lambda=(\lambda_1,\dots,\lambda_p)$ in operator norm.  Since $\Lambda$ is surjective (its kernel is the complex $(n-p)$--plane $\Pi$),
for $\varepsilon$ sufficiently small the perturbation bound implies $dS(y)$ is surjective for all $y\in U$.
Hence $Y=S^{-1}(0)$ is a smooth complex submanifold of $U$ by the holomorphic implicit function theorem \REVMZ{\cite{GH78}}.

Write $\C^n=\Pi\oplus \Pi^\perp$ and let $(u,w)$ denote the corresponding coordinates.
Since $\partial_w S$ is uniformly close to $\partial_w\Lambda$ and $\partial_w\Lambda:\Pi^\perp\to\C^p$ is invertible,
the implicit function theorem yields (after shrinking $U$ if needed) a $C^1$ map $g$ with $Y=\{(u,g(u))\}$.
Differentiating $S(u,g(u))=0$ gives $Dg=-(\partial_w S)^{-1}\partial_u S$, so the same uniform closeness estimates imply
$\|Dg\|\le C\,\varepsilon$ for a constant $C$ depending only on $(n,p)$ and the conditioning of $\{\lambda_i\}$.
\end{proof}


\noindent\textbf{addendum (explicit slope constant and usable domain).}
In applications, we take $U=B_r(0)\subset\C^n$ and apply the above argument on a \emph{concentric subball}
$U_{1/2}:=B_{r/2}(0)$, so that the implicit-function graph is defined on the full base $\Pi\cap U_{1/2}$.
More precisely, write $\C^n=\Pi\oplus\Pi^\perp$ and let $\Lambda=(\lambda_1,\dots,\lambda_p)$.
Set $\kappa:=\|(\partial_w\Lambda)^{-1}\|$ (finite by transversality). For $\varepsilon\le (4\kappa)^{-1}$ one has
$\|(\partial_wS)^{-1}\|\le 2\kappa$ on $U$, hence the graph map satisfies the uniform estimate
\[
\|Dg\|_{C^0(\Pi\cap U_{1/2})}\ \le\ C_{\mathrm{graph}}\,\varepsilon,
\qquad C_{\mathrm{graph}}:=2\kappa,
\]
and $Y\cap U_{1/2}$ is a single $C^1$ graph over $\Pi\cap U_{1/2}$.




\begin{proposition}[Projective tangential approximation with $C^1$ control]

\label{prop:tangent-approx-full}\label{thm:tangent-approx}
Let $x\in X$ and let $\Pi\subset T_xX$ be a complex $(n-p)$-plane.
For every $\varepsilon>0$ there exist $\mhol=\mhol(\varepsilon)\gg 1$ and a (possibly singular) complete intersection complex analytic subset
\[
Y = \{s_1=0\}\cap \cdots \cap \{s_p=0\}\subset X,
\qquad s_i\in H^0(X,L^{\mhol}),
\]
such that $x\in Y$, $Y$ is smooth on $B_{c\,\mhol^{-1/2}}(x)$, and
\[
\angle\bigl(T_yY,\Pi\bigr)<\varepsilon
\quad\text{for all } y\in Y\cap B_{c\,\mhol^{-1/2}}(x).
\]
Here $c>0$ is the constant from Lemma~\ref{lem:bergman-control}, and angles are computed after identifying $T_yX$ with $T_xX$ by Levi--Civita parallel transport \REVMZ{\cite{GH78}} inside $B_{c\,\mhol^{-1/2}}(x)$.
Moreover, $Y$ is $\psi$-calibrated (as a complex analytic set, hence calibrated on its regular locus).
\end{proposition}




\noindent\textbf{Remark (removing an unnecessary Bertini/global-smoothness dependency).}
Downstream we only use that the defining sections are \emph{transverse on the Bergman ball}
$B_{c\,\mhol^{-1/2}}(x)$, which yields a \emph{single-sheet $C^1$ graph} there.
No later step requires $Y$ to be globally smooth on all of $X$.
Accordingly, the conclusion may be read as producing an \emph{algebraic complete-intersection cycle}
(possibly with singularities away from $B_{c\,\mhol^{-1/2}}(x)$), whose associated integration current is
$\psi$--calibrated in the sense of Wirtinger/Harvey--Lawson \REVMZ{\cite[Thm~4.2]{HL82}}.


\begin{proof}
Choose unitary holomorphic coordinates $(w,u)\in\C^p\times\C^{n-p}$ centered at $x$ such that
$\Pi=\{w=0\}\subset T_x^{1,0}X$.  Set $\lambda_i:=dw_i\in T_x^{*(1,0)}X$ for $i=1,\dots,p$, so that
$\max_i\|\lambda_i\|=1$ and the map $\Lambda:T_x^{1,0}X\to\C^p$, $v\mapsto(\lambda_1(v),\dots,\lambda_p(v))$ is
the standard projection.
Apply Lemma~\ref{lem:bergman-control} with tolerance $\varepsilon/p$ to obtain
$\mhol\ge m_{\mathrm{hol},1}(\varepsilon/p)$ and sections $s_i\in H^0(X,L^{\mhol})$ with $s_i(x)=0$ and
\[
\|ds_i(y)-\lambda_i\|\le \frac{\varepsilon}{p}
\quad\text{for all } y\in B_{c\,\mhol^{-1/2}}(x).
\]
Set $Y:=\bigcap_{i=1}^p\{s_i=0\}$.
By Lemma~\ref{lem:graph-from-grad} applied to $S=(s_1,\dots,s_p)$ in the $(w,u)$--coordinates, after replacing $\varepsilon$ by $\min\{\varepsilon,\tfrac14\}$ to meet the smallness threshold in Lemma~\ref{lem:graph-from-grad}, the Jacobian $\partial_w S$ is uniformly invertible on $B_{c\,\mhol^{-1/2}}(x)$, hence $Y$ is a smooth complex submanifold on that ball.
The complex normal space to $Y$ at $y$ is spanned by
$\{ds_1(y),\ldots,ds_p(y)\}$, which is $\varepsilon$-close to
$\{\lambda_1,\ldots,\lambda_p\}$ in the Grassmannian metric.
Hence $T_yY$ is $\varepsilon$-close to $\Pi$ for all $y$ in the ball.

Since $Y$ is a complex submanifold of a K\"ahler manifold, it is
automatically calibrated by $\psi=\omega^{n-p}/(n-p)!$.

More generally, even if $Y$ is only an analytic/algebraic complete intersection with singularities away from the ball,
the associated integration current $[Y]$ is $\psi$--calibrated (Wirtinger), hence strongly positive and closed; \REVMZ{see Harvey--Lawson \cite[Thm~4.2]{HL82} and King \cite[Thm~4.5]{King71}}.


\end{proof}
\begin{proposition}[Holomorphic finite-net of calibrated directions near $K$]
\label{prop:dense-holo}
{%
Let $K\subset X$ be compact and let $\varepsilon>0$.
There exist finitely many centers $x_\alpha\in K$ and finitely many $\psi$-calibrated
$(n-p)$-submanifolds
\[
Y_{\alpha,1},\ldots,Y_{\alpha,N_\alpha}\subset X
\]
{(each a smooth complete intersection in $|L^{m_{\mathrm{hol},\alpha,j}}|$ for some sufficiently large integer $m_{\mathrm{hol},\alpha,j}$),}
with $x_\alpha\in Y_{\alpha,j}$ for all $\alpha,j$, such that for every $x\in K$ and every calibrated plane
$\Pi\subset T_xX$ there exist indices $\alpha,j$ with
\[
d(x,x_\alpha)<\varepsilon
\quad\text{and}\quad
\dist\!\bigl(T_{x_\alpha}Y_{\alpha,j},\mathrm{PT}_{x\to x_\alpha}\Pi\bigr)<\varepsilon,
\]
where $\mathrm{PT}_{x\to x_\alpha}:T_xX\to T_{x_\alpha}X$ denotes parallel transport along the minimizing geodesic
inside the chosen normal ball.
}%
\end{proposition}

\begin{remark}[note]
\REVMZ{\textbf{[Technical clarification.]}}
{%
The originally stated ``global coverage'' form (requiring that for every $x\in K$ and every calibrated plane
$\Pi\subset T_xX$ there exists $j$ with \emph{$x\in Y_j$} and $\dist(T_xY_j,\Pi)<\varepsilon$)
cannot hold when $p\ge 1$ and $K$ has nonempty interior: each $(n-p)$-dimensional complex submanifold is a proper
real-analytic subset of real codimension $2p$, hence has empty interior, and a finite union still has empty interior.
The finite-net formulation above is the correct substitute and is exactly what the construction below provides.
}%
\end{remark}

\begin{proof}
{%
Fix $\varepsilon>0$.
Let $K_{n-p}(x)\subset \Gr_{n-p}(T_xX)$ denote the compact set of $\psi$-calibrated $(n-p)$-planes at $x$.
By smoothness of the calibration and compactness of $K$, the cone field $x\mapsto K_{n-p}(x)$ varies uniformly
continuously: hence there exists $r=r(\varepsilon)>0$ such that for any $x_0\in K$, any $x\in B_r(x_0)$, and any
$\Pi\in K_{n-p}(x)$, the transported plane $\mathrm{PT}_{x\to x_0}\Pi$ lies within distance $\varepsilon/3$ of
$K_{n-p}(x_0)$ in $\Gr_{n-p}(T_{x_0}X)$.

Choose finitely many centers $x_\alpha\in K$ with $K\subset\bigcup_\alpha B_r(x_\alpha)$ and $r<\varepsilon$.
For each $\alpha$, choose an $\varepsilon/3$-net of calibrated planes
$\{\Pi_{\alpha,1},\ldots,\Pi_{\alpha,N_\alpha}\}$ in the compact fiber $K_{n-p}(x_\alpha)\subset\Gr_{n-p}(T_{x_\alpha}X)$.
Apply Proposition~\ref{prop:tangent-approx-full}{ (Since the set of pairs $(x_\alpha,\Pi_{\alpha,j})$ used below is finite, fix one integer $\mhol(\varepsilon)\ge m_{\mathrm{hol},0}(\varepsilon,X)$ so the proposition applies simultaneously to each pair; we use this common $\mhol(\varepsilon)$ throughout.)} to realize each net direction by a smooth $\psi$-calibrated complete
intersection $Y_{\alpha,j}$ through $x_\alpha$ satisfying
\[
\dist\!\bigl(T_{x_\alpha}Y_{\alpha,j},\Pi_{\alpha,j}\bigr)<\varepsilon/3.
\]

Now given $x\in K$ and a calibrated plane $\Pi\in K_{n-p}(x)$, pick $\alpha$ with $x\in B_r(x_\alpha)$ and set
$\Pi':=\mathrm{PT}_{x\to x_\alpha}\Pi\in\Gr_{n-p}(T_{x_\alpha}X)$.
By the choice of $r$, there exists a calibrated plane $\widehat\Pi\in K_{n-p}(x_\alpha)$ with
$\dist(\Pi',\widehat\Pi)<\varepsilon/3$, and by the $\varepsilon/3$-net there exists $j$ with
$\dist(\widehat\Pi,\Pi_{\alpha,j})<\varepsilon/3$.
Therefore, by the triangle inequality,
\[
\dist\!\bigl(T_{x_\alpha}Y_{\alpha,j},\Pi'\bigr)
\le \dist\!\bigl(T_{x_\alpha}Y_{\alpha,j},\Pi_{\alpha,j}\bigr)
   +\dist(\Pi_{\alpha,j},\widehat\Pi)+\dist(\widehat\Pi,\Pi')
<\varepsilon.
\]
Finally $d(x,x_\alpha)<r<\varepsilon$ and $x_\alpha\in Y_{\alpha,j}$ by construction.
}%
\end{proof}
% ------------------------------------------------------------
\subsection*{Step 3: Local calibrated laminates on small cubes (Theorem B)}

This step constructs multiple calibrated sheet pieces on each cube $Q$ (multiplicity allowed; disjointness is optional) with prescribed
 tangent directions and mass fractions.




%=========================================================
% Block A (moved forward): Corner--exit templates and interface tools
% (Moved here to remove forward-dependencies in Thm.~\ref{thm:local-sheets} and Lem.~\ref{lem:local-bary}.)
%=========================================================

\begin{lemma}[A concrete \emph{complex} corner-exit translation template in a cube]\label{lem:complex-corner-exit-template}
Work in $\C^n=\C^{n-p}\times\C^p$ with coordinates $z=(u,w)$, where $u=(u_1,\dots,u_{n-p})$ and $w=(w_1,\dots,w_p)$.
Let $Q:=[0,h]^{2n}\subset\R^{2n}\cong\C^n$ be the coordinate cube with vertex $0$.
Fix a constant $0<c_0<1$ and choose a scale $s>0$ with $s\le c_0 h/100$.

\smallskip\noindent
Define a complex $(n-p)$--plane $P\subset\C^n$ as the graph of the linear map $A:\C^{n-p}\to\C^p$ given by
\[
w_1\ =\ -(1-i)\sum_{j=1}^{n-p}u_j,\qquad w_2=\cdots=w_p=0.
\]
For translation parameters $t=(t_1,\dots,t_p)\in\C^p$, write $P_t:=\{(u,Au+t):u\in\C^{n-p}\}$ (parallel translate of $P$).
Assume $t$ satisfies the \emph{interior-margin} bounds
\[
\Re t_1=s,\qquad 2s\le \Im t_1\le 3s,
\qquad
2s\le \Re t_j,\Im t_j\le 3s\ \ (2\le j\le p).
\]
Then:
\begin{enumerate}
\item[\textnormal{(i)}] (\textbf{Corner-exit simplex footprint}) The footprint $E(t):=P_t\cap Q$ is a $k$--simplex with $k=2n-2p$,
contained in $B(0,c_0h)$.
\item[\textnormal{(ii)}] (\textbf{Fixed designated exit faces}) The $k\!+\!1$ facets of $E(t)$ lie on the $k\!+\!1$ coordinate faces
\[
F_{\Re u_j=0},\ F_{\Im u_j=0}\ (1\le j\le n-p),\qquad\text{and}\qquad F_{\Re w_1=0},
\]
and $E(t)$ meets no other codimension-$1$ faces of $Q$.
\item[\textnormal{(iii)}] (\textbf{Uniform fatness and equal slice mass}) The family $E(t)$ is uniformly fat (with constants depending only on $(n,p)$),
and $\mathcal H^k(E(t))$ is independent of $t$ in the above parameter box (hence equal across indices).
\end{enumerate}
In particular, this admissible parameter box has real dimension $2p-1$, so for any separation scale $\delta>0$ one can choose an ordered $\delta$--separated
list $(t_a)_{a\ge 1}$ inside it with identical footprints $P_{t_a}\cap Q$.
\end{lemma}
\begin{proof}
Write $u_j=x_j+i y_j$ with $x_j=\Re u_j$ and $y_j=\Im u_j$.
On $P_t$ one computes
\[
\Re w_1\ =\ \Re t_1\ +\ \Re\!\Bigl(-(1-i)\sum_{j=1}^{n-p}u_j\Bigr)
\ =\ s\ -\ \sum_{j=1}^{n-p}(x_j+y_j),
\]
and
\[
\Im w_1\ =\ \Im t_1\ +\ \Im\!\Bigl(-(1-i)\sum_{j=1}^{n-p}u_j\Bigr)
\ =\ \Im t_1\ +\ \sum_{j=1}^{n-p}(x_j-y_j).
\]
The cube constraints on $w_2,\dots,w_p$ are automatic since $w_j\equiv t_j$ and $t_j\in(0,h)^2$ with margin $\gtrsim s$.
Moreover, on the region cut out by $x_j,y_j\ge 0$ and $\sum_j(x_j+y_j)\le s$, one has
$\bigl|\sum_j(x_j-y_j)\bigr|\le \sum_j(x_j+y_j)\le s$, hence
\[
\Im w_1\ \in\ [\Im t_1-s,\ \Im t_1+s]\ \subset\ [s,4s]\ \subset\ (0,h),
\]
so both faces $\{\Im w_1=0\}$ and $\{\Im w_1=h\}$ are avoided.
Likewise $\Re w_1\in[0,s]\subset(0,h)$ avoids $\{\Re w_1=h\}$, and $x_j,y_j\le s\ll h$ avoids the far faces
$\{\Re u_j=h\}$ and $\{\Im u_j=h\}$.

\smallskip\noindent
Consequently, $E(t)=P_t\cap Q$ is cut out on $P_t$ exactly by the inequalities
\[
x_j\ge 0,\quad y_j\ge 0\quad (1\le j\le n-p),\qquad\text{and}\qquad \Re w_1\ge 0,
\]
i.e.\ by $\sum_j (x_j+y_j)\le s$ together with nonnegativity of the
$k=2(n-p)$ coordinates $(x_1,y_1,\dots,\allowbreak x_{n-p},y_{n-p})$.
This is the standard $k$--simplex in $\R^{k}$ (embedded linearly as a graph in $\R^{2n}$), proving (i) and (ii).
Uniform fatness follows because this simplex is affine-equivalent to the standard simplex with distortion depending only on the fixed linear map $A$,
and the slice mass $\mathcal H^k(E(t))\asymp s^k$ is independent of $t$ since the defining inequalities do not depend on $t$ inside the admissible box.
Finally, packing a $\delta$--separated family inside a $(2p-1)$--dimensional box is elementary.
\end{proof}
\begin{remark}[Corner--exit intuition]\label{rem:corner-exit-intuition}
\REVMZ{\textbf{[Geometric intuition.]}}
The ``corner--exit'' template is a calibrated $(n\!-\!p)$--plane piece whose footprint in the base plane is a uniformly
fat simplex of diameter $\asymp s$ placed near a chosen cube vertex.  The defining inequalities force the footprint to \emph{touch only}
the designated ``near'' faces and to stay a definite distance from the ``far'' faces, so translating the sheet in transverse directions
does not change which faces are hit.  This is the mechanism that later allows consistent face--traces across adjacent cells while keeping
uniform mass scaling in $s$ and $h$.
\end{remark}


\begin{lemma}[Small-graph distortion]\label{lem:small-graph-distortion}

{
Let $k\ge 1$ and let $E\subset \R^k$ be Lebesgue measurable.  Let $G:E\to \R^q$ be $C^1$ with
\[
\|DG\|_{L^\infty(E)}\le \varepsilon,\qquad 0<\varepsilon\le 1.
\]
Then the graph map $\Gamma(x):=(x,G(x))$ satisfies the explicit Jacobian bounds
\[
1 \ \le\ J_\Gamma(x)\ :=\ \sqrt{\det\bigl(I_k + (DG(x))^{\!T}DG(x)\bigr)}\ \le\ (1+\varepsilon^2)^{k/2}\ \le\ 1+k\,\varepsilon^2,
\]
and consequently
\[
\mathcal{H}^k(E)\ \le\ \mathcal{H}^k(\Gamma(E))\ \le\ (1+k\,\varepsilon^2)\,\mathcal{H}^k(E).
\]
Moreover, for any $(k-1)$-rectifiable set $F\subset \R^k$ one has
\[
\mathcal{H}^{k-1}(F)\ \le\ \mathcal{H}^{k-1}(\Gamma(F))\ \le\ (1+(k-1)\varepsilon^2)\,\mathcal{H}^{k-1}(F).
\]
In particular, if $T$ is an integral (or normal) $(k-1)$-current carried by $F$, then
\[
\Mass(\Gamma_\# T)\ \le\ (1+(k-1)\varepsilon^2)\,\Mass(T).
\]
}

\end{lemma}

\begin{proof}

{
Since $A:=(DG)^T DG$ is positive semidefinite and $\|DG\|\le \varepsilon$, every eigenvalue of $A$ is $\le \varepsilon^2$.
Hence $\det(I_k+A)\le (1+\varepsilon^2)^k$, giving $J_\Gamma\le (1+\varepsilon^2)^{k/2}$.  The final bound
$(1+\varepsilon^2)^{k/2}\le 1+k\varepsilon^2$ holds for $0\le \varepsilon\le 1$.
The stated $\mathcal{H}^k$ and $\mathcal{H}^{k-1}$ estimates follow from the area formula \REVMZ{\cite[3.2.3]{Fed69}} applied to $\Gamma$ on $E$ and on rectifiable charts for $F$,
and the current mass bound is immediate from the definition of $\Mass(\Gamma_\#T)$ and the $(k-1)$-Jacobian estimate.}

\end{proof}

\begin{lemma}[Corner-exit simplex mass scale and no-heavy-tail uniformity]\label{lem:corner-exit-mass-scale}
In the setting of Lemma~\ref{lem:complex-corner-exit-template}, fix a scale $s>0$ and let $E(t)=P_t\cap Q$ be the resulting corner-exit simplex of
dimension $k=2n-2p$.
Then there exist constants $0<c\le C<\infty$ depending only on $(n,p)$ such that for every admissible $t$ (with the fixed scale $s$):
\[
c\,s^{k}\ \le\ \mathcal H^{k}(E(t))\ \le\ C\,s^{k},
\qquad
c\,s^{k-1}\ \le\ \mathcal H^{k-1}(E(t)\cap F_i)\ \le\ C\,s^{k-1}\ \ (i=0,\dots,k),
\]
where $F_0,\dots,F_k$ are the designated exit faces from Lemma~\ref{lem:complex-corner-exit-template}.
In particular, if one chooses $s=\theta\,h$ for a fixed $\theta\in(0,1)$ (so $s$ is a fixed fraction of the cell size), then each footprint has
$\mathcal H^k(E(t))\asymp h^k$ and each designated face slice has $\mathcal H^{k-1}(E(t)\cap F_i)\asymp h^{k-1}$.
Moreover, throughout the admissible parameter box in Lemma~\ref{lem:complex-corner-exit-template} (with fixed $\Re t_1=s$), the footprints are identical,
so $\mathcal H^{k}(E(t))$ and the facet measures $\mathcal H^{k-1}(E(t)\cap F_i)$ are in fact independent of $t$.

\smallskip\noindent
Consequently, an ordered $\delta$--separated list $(t_a)$ in that box yields a template whose pieces have \emph{exactly equal} footprint masses and
per-face slice masses (no heavy tails along the order).  If $Y^a\cap Q$ is an $\varepsilon$--slope graph over $E(t_a)$, then
Lemma~\ref{lem:small-graph-distortion} gives the corresponding holomorphic equal-mass/equal-slice-mass conclusions up to a common $(1+O(\varepsilon^2))$ factor.
\end{lemma}
\begin{proof}
In the proof of Lemma~\ref{lem:complex-corner-exit-template}, $E(t)$ is cut out on the $k$ real coordinates
$(x_1,y_1,\dots,x_{n-p},y_{n-p})\in\R^{k}$ by the inequalities
$x_j\ge 0$, $y_j\ge 0$, and $\sum_j(x_j+y_j)\le s$, which define a standard simplex of size $s$.
Thus $\mathcal H^{k}(E(t))\asymp s^{k}$ and each facet has $\mathcal H^{k-1}\asymp s^{k-1}$, with constants depending only on $k$ (hence only on $(n,p)$).
Independence of $t$ inside the parameter box is immediate because the defining inequalities on $P_t$ do not depend on $t$ once $\Re t_1=s$ is fixed.
Finally, Lemma~\ref{lem:small-graph-distortion} gives the $1+O(\varepsilon^2)$ distortion bounds for small-slope graphs, uniformly in $a$.
\end{proof}

\begin{lemma}[{Uniform per--face boundary mass for fat corner simplices}]\label{lem:corner-simplex-face-mass}

Fix $d\ge 2$, $1\le k<d$, and a fatness parameter $\Lambda\ge 1$.
Let $E\subset\R^d$ be a $k$--simplex and write $v_E:=\mathcal H^k(E)$.
Suppose that $E$ is \emph{$\Lambda$--fat} in the following quantitative sense: if $\Pi:=\mathrm{aff}(E)$ is the affine span of $E$, then there exists an affine isomorphism $A:\Pi\to\R^k$ such that
\[
\|DA\|\le \Lambda,\qquad \|(DA)^{-1}\|\le \Lambda,
\]
and $A(E)=\Delta_s$, the standard $k$--simplex of scale $s>0$.
Let $\sigma_0,\dots,\sigma_k$ denote the $(k-1)$--dimensional facets of $E$, and set $a_i:=\mathcal H^{k-1}(\sigma_i)$.
Then there exist constants $0<c_\star(k,\Lambda)\le C_\star(k,\Lambda)$ such that for every $i=0,\dots,k$,
\[
c_\star\, v_E^{(k-1)/k}\ \le\ a_i\ \le\ C_\star\, v_E^{(k-1)/k}.
\]

\end{lemma}

\begin{proof}

Because $A$ is affine on $\Pi$, both the $k$--Jacobian and the $(k-1)$--Jacobian of $A$ are constant on $\Pi$ and are controlled by the operator--norm bounds:
\[
\Lambda^{-k}\ \lesssim_k\ J_k(A)\ \lesssim_k\ \Lambda^{k},
\qquad
\Lambda^{-(k-1)}\ \lesssim_k\ J_{k-1}(A)\ \lesssim_k\ \Lambda^{k-1},
\]
and the same holds for $A^{-1}$ (here $\lesssim_k$ hides constants depending only on $k$).
Consequently,
\[
v_E=\mathcal H^k(E)\ \simeq_{k,\Lambda}\ \mathcal H^k(\Delta_s),
\qquad
a_i=\mathcal H^{k-1}(\sigma_i)\ \simeq_{k,\Lambda}\ \mathcal H^{k-1}(\partial\Delta_s),
\]
with comparability constants depending only on $(k,\Lambda)$.

In the standard simplex $\Delta_s$ the $k$--volume scales like $s^k$ and each facet area scales like $s^{k-1}$, i.e.
\[
\mathcal H^k(\Delta_s)=c_k\, s^k,
\qquad
\mathcal H^{k-1}(\text{any facet of }\Delta_s)=c_{k-1}\, s^{k-1},
\]
for explicit dimensional constants $c_k,c_{k-1}>0$.  Eliminating $s$ gives
$\mathcal H^{k-1}(\text{facet})\simeq_k \mathcal H^k(\Delta_s)^{(k-1)/k}$.
Combining with the previous comparability under $A$ yields
\[
a_i\ \simeq_{k,\Lambda}\ v_E^{(k-1)/k},
\]
uniformly in $i$, proving the claim.

\end{proof}



\begin{proposition}[{Corner--exit footprint geometry for small--slope graphs}]\label{prop:holomorphic-corner-exit-g1g2}

Fix $d\ge 2$ and $1\le k<d$.  Let $Q=[0,h]^d\subset\R^d$ and let $v$ be a vertex of $Q$.
Let $P\subset\R^d$ be an affine $k$--plane and set $E:=P\cap Q$.
Write $v_E:=\mathcal H^k(E)$.

Assume:
\begin{enumerate}
\item[\textup{(H1)}]\label{H1corner}
(\textbf{Corner--exit simplex footprint})
$E$ is a $k$--simplex with one vertex at $v$.  Moreover, there exist \emph{distinct} codimension--$1$ faces
$F_0,\dots,F_k$ of $Q$, each incident to $v$, such that the $k\!+\!1$ facets of $E$ are exactly the sets $E\cap F_i$
$(i=0,\dots,k)$; in particular, $E$ meets no other codimension--$1$ faces of $Q$.

\item[\textup{(H2)}]\label{H2cornerfat}
(\textbf{Uniform fatness})
$E$ is $\Lambda$--fat (in the quantitative sense of Lemma~\ref{lem:corner-simplex-face-mass}); hence
\[
\mathcal H^{k-1}(E\cap F_i)\ \simeq_{k,\Lambda}\ v_E^{(k-1)/k}\qquad (i=0,\dots,k).
\]
\end{enumerate}

Let $Y\subset\R^d$ be a smooth oriented $k$--dimensional submanifold such that $Y\cap Q$ is a single $C^1$ graph over $E$
with slope at most $\varepsilon$, i.e.\ there is a $C^1$ embedding $\Phi:E\to\R^d$ with $\Phi(E)=Y\cap Q$ and
$\|D\Phi-\mathrm{Id}\|_{C^0(E)}\le C\,\varepsilon$ in the coordinates of $P$.
Let
\[
\delta:=\min\{\dist(E,F):\ F\ \text{a codimension--$1$ face of $Q$ with }F\notin\{F_0,\dots,F_k\}\}\ >0.
\]
Assume in addition that
\[
\sup_{x\in E}|\Phi(x)-x|\ <\ \delta/2.
\]

Then:
\begin{enumerate}
\item[\textup{(G1)}]\label{G1iff}
(\textbf{Face incidence})
For any codimension--$1$ face $F$ of $Q$,
\[
Y\cap F\neq\emptyset
\quad\Longleftrightarrow\quad
F\in\{F_0,\dots,F_k\}.
\]

\item[\textup{(G2)}]\label{G2mass}
(\textbf{Per--face boundary mass comparability})
For each $i=0,\dots,k$, the intersection $Y\pitchfork F_i$ is a smooth oriented $(k-1)$--submanifold and
\[
\Mass\bigl(\partial([Y]\llcorner Q)\llcorner F_i\bigr)
=\mathcal H^{k-1}(Y\cap F_i)
=\bigl(1+O_k(\varepsilon^2)\bigr)\,\mathcal H^{k-1}(E\cap F_i)
\simeq_{k,\Lambda}\ v_E^{(k-1)/k}.
\]
\end{enumerate}

\end{proposition}

\begin{proof}

For \eqref{G1iff}, let $F$ be a codimension--$1$ face of $Q$ not in $\{F_0,\dots,F_k\}$.
By definition of $\delta$, we have $\dist(E,F)\ge \delta$.
If $y\in Y\cap F$, then $y=\Phi(x)$ for some $x\in E$, and hence
\[
\delta\ \le\ \dist(x,F)\ \le\ |x-\Phi(x)|\ <\ \delta/2,
\]
a contradiction.  Thus $Y\cap F=\emptyset$ for every non--designated face $F$.
Conversely, if $F=F_i$ is one of the designated faces, then $E\cap F_i$ is a facet of $E$ and is nonempty.
For $\varepsilon$ small the graph is transverse to $F_i$ along that facet, hence $Y\cap F_i\neq\emptyset$.

For \eqref{G2mass}, since $Y$ is smooth, $\partial[Y]=0$. Therefore
\[
\partial([Y]\llcorner Q)= [Y]\llcorner \partial Q,
\]
and restricting to a face $F_i$ gives
$\partial([Y]\llcorner Q)\llcorner F_i = [Y]\llcorner F_i$ with the induced orientation.
Thus $\Mass(\partial([Y]\llcorner Q)\llcorner F_i)=\mathcal H^{k-1}(Y\cap F_i)$.
Because $\Phi$ is a $C^1$ graph map with slope $\le \varepsilon$, the area formula gives
\[
\mathcal H^{k-1}(Y\cap F_i)=\bigl(1+O_k(\varepsilon^2)\bigr)\,\mathcal H^{k-1}(E\cap F_i),
\]
and the final comparison follows from the fatness estimate in \eqref{H2cornerfat}.

\end{proof}

\begin{lemma}[Sliver stability under $C^1$-graph perturbations]\label{lem:sliver-stability}

Let $Q\subset\R^{2n}$ be a cube of diameter $h$, and let $P$ be an affine calibrated $(2n-2p)$--plane.
Let $Y$ be a smooth $(2n-2p)$--submanifold such that $Y\cap Q$ is a $C^1$ graph over $P\cap Q$ with slope $\le \varepsilon$, i.e.\
in suitable coordinates
\[
Y\cap Q=\{x+u(x):x\in P\cap Q\},
\qquad
u:P\cap Q\to P^\perp,
\qquad
\|Du\|_{C^0}\le \varepsilon.
\]
Then:
\begin{enumerate}
\item[\textnormal{(i)}] (\textbf{Mass comparability})
\[
\Mass([Y]\llcorner Q) = \bigl(1+O(\varepsilon^2)\bigr)\,\Mass([P]\llcorner Q),
\]
where the implied constant depends only on $(n,p)$ (and in particular the ratio is $\ge 1$).
\item[\textnormal{(ii)}] (\textbf{Disjointness persistence, with an anchor}) 
Let $t_1,t_2\in P^\perp$ and suppose $Y_1,Y_2$ are $C^1$ graphs of slope $\le \varepsilon$ over the parallel planes
$P+t_1$ and $P+t_2$ on $(P+t_i)\cap Q$, realized as $Y_i\cap Q=\{x+u_i(x):x\in (P+t_i)\cap Q\}$ with $\|Du_i\|_{C^0}\le\varepsilon$.
Assume further that for each $i\in\{1,2\}$ there exists an \emph{anchor point} $x_i\in (P+t_i)\cap Q$ with $x_i\in Y_i$
(equivalently $u_i(x_i)=0$).
If $\|t_1-t_2\|\ge 10\,\varepsilon\,h$, then $Y_1\cap Q$ and $Y_2\cap Q$ are disjoint.
\end{enumerate}

\end{lemma}


\begin{proof}

\textnormal{(i)} Write $k:=2n-2p$ and parametrize $Y\cap Q$ as the graph of $u:P\cap Q\to P^\perp$ with $\|Du\|_{C^0}\le\varepsilon$.
By the area formula for graphs,
\[
\Mass([Y]\llcorner Q)
=\int_{P\cap Q}\sqrt{\det(I+Du^\top Du)}\,d\mathcal H^{k}.
\]
Since $Du^\top Du$ is positive semidefinite and $\|Du^\top Du\|\le \|Du\|^2\le \varepsilon^2$, one has
\[
1\ \le\ \sqrt{\det(I+Du^\top Du)}\ \le\ 1+C(n,p)\,\varepsilon^2,
\]
hence $\Mass([Y]\llcorner Q)=\bigl(1+O(\varepsilon^2)\bigr)\Mass([P]\llcorner Q)$.

\textnormal{(ii)} Fix $i\in\{1,2\}$ and let $x_i\in (P+t_i)\cap Q$ be an anchor with $u_i(x_i)=0$.
For any $x\in (P+t_i)\cap Q$,
\[
|u_i(x)|\ =\ |u_i(x)-u_i(x_i)|\ \le\ \|Du_i\|_{C^0}\,|x-x_i|\ \le\ \varepsilon\,h,
\]
since $\mathrm{diam}(Q)=h$.
Therefore every point $y=x+u_i(x)\in Y_i\cap Q$ satisfies
$\dist(y,P+t_i)\le \varepsilon h$, i.e.
\[
Y_i\cap Q\ \subset\ \mathcal N_{\varepsilon h}(P+t_i)\cap Q.
\]
If $\|t_1-t_2\|\ge 10\,\varepsilon\,h$, then the tubular neighborhoods
$\mathcal N_{\varepsilon h}(P+t_1)$ and $\mathcal N_{\varepsilon h}(P+t_2)$ are disjoint, hence so are $Y_1\cap Q$ and $Y_2\cap Q$.

\end{proof}


\begin{remark}[Sliver stability: what is used later]\label{rem:sliver-stability-discussion}
\REVMZ{\textbf{[Technical clarification.]}}
In later ``sliver'' bookkeeping we only use two consequences of Lemma~\ref{lem:sliver-stability}:
(i) the mass of a small $C^1$ graph over a calibrated plane is comparable to the plane mass in the same cell,
and (ii) disjointness of distinct translates persists provided one fixes an \emph{anchor} point on each graph.
No monotonicity or stationarity beyond the graph control is used at this stage.
\end{remark}



\begin{proposition}[Realizing a finite translation template locally]\label{prop:finite-template}

Fix a holomorphic chart $U\subset X$ with holomorphic coordinates $z=(u,w)\in\C^{n-p}\times\C^p$ and let
$P:=\{w=0\}\subset\C^n$.
Fix $\varepsilon\in(0,1)$ and choose an integer $\mhol\ge m_{\mathrm{hol},1}(\varepsilon)$ so that
Lemma~\ref{lem:bergman-control} applies at tolerance $\varepsilon$ on each Bergman ball $B_{c\,\mhol^{-1/2}}(\cdot)$.
Set
\[
r:=\frac{c}{2}\,\mhol^{-1/2}.
\]
Let $x_1,\dots,x_N\in U$ be points such that $B_{c\,\mhol^{-1/2}}(x_a)\subset U$ for all $a$, and write
$t_a:=w(x_a)\in\C^p\simeq P^\perp$.  Then for each $a$ there exists a holomorphic complete intersection
\(
Y^{(a)}\subset X
\)
cut out by $p$ holomorphic sections of $L^{\mhol}$ such that:
\begin{enumerate}
\item[(i)] On $B_r(x_a)$, the set $Y^{(a)}$ is a $C^1$ graph over the affine plane $\{w=t_a\}$ with slope bound
\[
\|D u^{(a)}\|\le C_{\mathrm{graph}}\,\varepsilon .
\]
Equivalently, for all $y\in Y^{(a)}\cap B_r(x_a)$ the angle between $T_yY^{(a)}$ and the translate of $P$
through $y$ is $\le C_{\mathrm{graph}}\varepsilon$ (with angles computed after Levi--Civita parallel transport).
\item[(ii)] (Mass control on the footprint scale.) Writing $k:=2n-2p$, we have
\[
\Mass\!\bigl([Y^{(a)}\cap B_r(x_a)]\bigr)
=\bigl(1+O(\varepsilon^2)\bigr)\,\mathcal{H}^{k}\!\bigl(P\cap B_r(0)\bigr),
\]
with the implied constant depending only on $(X,\omega)$ and the fixed rescaling radius.
\end{enumerate}
Moreover, if $a\neq b$ and $\|t_a-t_b\|\ge 10\,C_{\mathrm{graph}}\,\varepsilon\,r$, then the graph pieces
$Y^{(a)}\cap B_r(x_a)$ and $Y^{(b)}\cap B_r(x_b)$ are disjoint on any region where the two balls overlap.

\end{proposition}

\begin{proof}

Fix $a$.  Work in normal holomorphic coordinates centered at $x_a$ and identify $T_{x_a}X$ with $\C^n$
so that $P$ corresponds to the coordinate plane $\{w=0\}$.
Apply Lemma~\ref{lem:bergman-control} at $x_a$ with the target covectors $\lambda_i:=dw_i$.
This yields sections $s_{a,1},\dots,s_{a,p}\in H^0(X,L^{\mhol})$ whose differentials satisfy
\(
\sup_{B_{c\,\mhol^{-1/2}}(x_a)}\|ds_{a,i}-dw_i\|\le \varepsilon.
\)
By Lemma~\ref{lem:graph-from-grad} (and the implicit-function estimate it encodes), the common zero set
$Y^{(a)}:=\{s_{a,1}=\cdots=s_{a,p}=0\}$ is, on $B_r(x_a)$, the graph of a $C^1$ map over the plane through $x_a$
with direction $P$, with slope $\le C_{\mathrm{graph}}\varepsilon$.  Re-expressing in the fixed chart $(u,w)$ on $U$
yields (i) with the affine plane $\{w=t_a\}$.
The Jacobian bound for small-slope graphs gives (ii).  Finally, if $\|t_a-t_b\|\ge 10\,C_{\mathrm{graph}}\varepsilon r$,
then the base planes $\{w=t_a\}$ and $\{w=t_b\}$ are separated by that amount, and Lemma~\ref{lem:sliver-stability}
implies the corresponding graphs cannot intersect on any common region of definition, proving the disjointness claim.

\end{proof}

\begin{remark}[Geometric meaning for matching across adjacent cells]\label{rem:matching-cells-geometry}
\REVMZ{\textbf{[Geometric intuition.]}}
In later gluing steps, the translation parameters $t_a$ are encoded by the cell-scale coordinate $w$.
Because each $Y_a$ is a small-slope graph over the fixed affine plane $P_a$ on $B_{c\mhol^{-1/2}}(x)$,
the footprint $F_a$ (and its boundary faces) vary Lipschitzly with $t_a$; this is the geometric input needed
to match boundary masses face-by-face when adjoining neighboring cubes.
\end{remark}



\begin{proposition}[{Corner--exit: $L^1$ interface mass control on boundary faces}]\label{prop:holomorphic-corner-exit-L1}

Work in a holomorphic coordinate chart identifying a neighborhood of a cell $Q=[0,h]^d\subset\R^d$, with a chosen vertex $v$.
Fix $1\le k<d$ and let $\{P_a\}_{a=1}^N$ be a finite family of affine $k$--planes.
Set $E_a:=P_a\cap Q$ and $v_{E_a}:=\mathcal H^k(E_a)$.

Suppose that each $E_a$ is a $\Lambda$--fat corner--exit simplex footprint in the sense of
Proposition~\ref{prop:holomorphic-corner-exit-g1g2}, with designated exit faces
$F_0^{(a)},\dots,F_k^{(a)}$ incident to $v$.

Let $Y^{(a)}$ be the holomorphic complete intersections produced by Proposition~\ref{prop:finite-template} on $Q$ using anchor points
$x_a\in E_a$, so that $Y^{(a)}\cap Q$ is a single $C^1$ graph over $E_a$ with slope at most $C_{\mathrm{graph}}\varepsilon$,
realized by an embedding $\Phi_a:E_a\to\R^d$ with $\Phi_a(E_a)=Y^{(a)}\cap Q$ and $\Phi_a(x_a)=x_a$.
Assume moreover that $\varepsilon>0$ is chosen small enough (depending only on $(k,\Lambda)$ and on $C_{\mathrm{graph}}$) so that the
conclusions \eqref{G1iff}--\eqref{G2mass} of Proposition~\ref{prop:holomorphic-corner-exit-g1g2} apply to every pair $(E_a,Y^{(a)})$
whenever $\sup_{E_a}|\Phi_a-\mathrm{Id}|<\delta_\star/2$, where
\[
\delta_\star:=\min_{1\le a\le N}\ \min\{\dist(E_a,F):\ F\ \text{a codimension--$1$ face of $Q$ with }F\notin\{F_0^{(a)},\dots,F_k^{(a)}\}\}\ >0.
\]
Assume $\sup_{E_a}|\Phi_a-\mathrm{Id}|<\delta_\star/2$ for all $a$.

Then for each $a$,
\[
\operatorname{spt}\bigl(\partial([Y^{(a)}]\llcorner Q)\bigr)\ \subset\ \bigcup_{i=0}^k F_i^{(a)},
\]
and
\[
\Mass\bigl(\partial([Y^{(a)}]\llcorner Q)\bigr)
\le \sum_{i=0}^k \Mass\bigl(\partial([Y^{(a)}]\llcorner Q)\llcorner F_i^{(a)}\bigr)
\ \lesssim_{k,\Lambda,\varepsilon}\ v_{E_a}^{(k-1)/k}.
\]
In particular,
\[
\sum_{a=1}^N \Mass\bigl(\partial([Y^{(a)}]\llcorner Q)\bigr)
\ \lesssim_{k,\Lambda,\varepsilon}\ \sum_{a=1}^N v_{E_a}^{(k-1)/k}.
\]

\end{proposition}

\begin{proof}

The geometric graph and disjointness conclusions (existence of $\Phi_a$ and the bound $\sup_{E_a}|\Phi_a-\mathrm{Id}|\lesssim \varepsilon h$)
are the output of Proposition~\ref{prop:finite-template}.
Since the family $\{E_a\}_{a=1}^N$ is finite, the uniform gap $\delta_\star$ is positive, and for $\varepsilon$ small the displacement bound
ensures $\sup_{E_a}|\Phi_a-\mathrm{Id}|<\delta_\star/2$ for all $a$.

Apply the corresponding inheritance corollary proved immediately after this proposition (equivalently, Proposition~\ref{prop:holomorphic-corner-exit-g1g2})
to each pair $(E_a,Y^{(a)})$ (with slope parameter $\varepsilon':=C_{\mathrm{graph}}\varepsilon$).
By \eqref{G1iff}, $Y^{(a)}$ meets only the designated faces $F_i^{(a)}$ of $Q$.

Finally, because each $Y^{(a)}$ is a holomorphic complete intersection, it is a closed oriented $k$--cycle in $Q$, i.e.\ $\partial[Y^{(a)}]=0$.
Thus
\[
\partial([Y^{(a)}]\llcorner Q)= [Y^{(a)}]\llcorner \partial Q,
\]
and $\partial Q$ is the disjoint union of its codimension--$1$ faces.  Using \eqref{G2mass} on each designated face gives
\[
\Mass\bigl(\partial([Y^{(a)}]\llcorner Q)\llcorner F_i^{(a)}\bigr)
=\mathcal H^{k-1}(Y^{(a)}\cap F_i^{(a)})
\ \lesssim_{k,\Lambda,\varepsilon}\ v_{E_a}^{(k-1)/k},
\]
hence the stated bound for $\Mass(\partial([Y^{(a)}]\llcorner Q))$, and summing over $a$ yields the final estimate.

\end{proof}

\begingroup

\begin{definition}[Holomorphic scale]\label{def:holomorphic-scale}
Fix a holomorphic tensor power $\mhol\ge 1$ and a parameter $\sigma\in\bigl(0,\tfrac{c(X,\omega)}{100}\bigr)$.
The associated \emph{holomorphic scale} is
\[
s:=\sigma\,\mhol^{-1/2}.
\]

\end{definition}
\begin{theorem}[Local multi-sheet construction]\label{thm:local-sheets}
Fix a cube $Q\subset X$ of side length $h$ contained in a normal holomorphic coordinate chart.
Fix a tolerance $\varepsilon_{\mathrm{hol}}\in(0,1)$ and a holomorphic tensor power $\mhol\ge m_{\mathrm{hol},1}(\varepsilon/100)$
(from Lemma~\ref{lem:bergman-control}).  Let $s$ be the holomorphic scale from Definition~\ref{def:holomorphic-scale} (so $s=\sigma\,\mhol^{-1/2}$ for a fixed $\sigma\in(0,\tfrac{c(X,\omega)}{100})$) and assume $s\le h/100$ (equivalently, $\mhol$ is large compared to $h^{-2}$).
Let $k:=2n-2p$.

For each direction label $j$ in a fixed finite set, let $\widetilde\Pi_j\subset T^{1,0}X$ be the associated complex
$(n-p)$--plane (in the chosen chart), and choose a finite family of translation parameters
$t_{j,a}\in \widetilde\Pi_j^\perp\cap B^{2p}(0,Ch)$ for $a=1,\dots,N_j$.
{Let $\mathcal E_j\subset \widetilde\Pi_j$ be a fixed \emph{corner--exit footprint shape} of diameter $\asymp s$.
In the $j$--adapted splitting $\C^n=\widetilde\Pi_j\oplus \widetilde\Pi_j^\perp$ write points as $(u,w)$ with
$u\in \widetilde\Pi_j$, $w\in \widetilde\Pi_j^\perp$.
For each parameter $t_{j,a}$ define the \emph{physical footprint} in $Q$ by
\[
E_{j,a}:=\{(u,w)\in Q:\ u\in \mathcal E_j,\ w=t_{j,a}\},
\]
and let $x_{j,a}$ denote its Euclidean barycenter (so $B_s(x_{j,a})$ is the natural holomorphic-scale neighborhood of the footprint).
No separation/disjointness is required among the $E_{j,a}$; repeated parameters $t_{j,a}$ are allowed and correspond to multiplicity.}

Then for each $(j,a)$ there exists a (possibly singular) global complete intersection
$Y_{j}^{(a)}\subset X$ of complex codimension $p$ such that $Y_{j}^{(a)}\cap Q$ contains a connected component
$S_{j}^{(a)}$ with the following properties:
\begin{enumerate}
\item \textbf{Graph/tangent control on the holomorphic scale.}
Inside $B_{c\,\mhol^{-1/2}}(x_{j,a})$ (where $x_{j,a}$ is the footprint center), the set $S_{j}^{(a)}$
is a $C^1$ graph over the affine plane $\{w=t_{j,a}\}$ (in the $j$--adapted splitting) and satisfies
\[
\angle\bigl(T_y S_{j}^{(a)},\,\widetilde\Pi_j\bigr)\le \varepsilon_{\mathrm{hol}}
\qquad\text{for all }y\in S_{j}^{(a)}\cap B_{s}(x_{j,a}).
\]

\item \textbf{Mass scale.}
There is a constant $C_X$ depending only on the ambient geometry and the fixed template dictionary such that
\[
\Mass\bigl(S_{j}^{(a)}\bigr)=A_j(s)\ +\ O(\varepsilon_{\mathrm{hol}})\,s^{k},
\qquad\text{with }A_j(s)\asymp s^{k}.
\]

\item \textbf{Face--trace template decomposition.}
For each interior face $F$ of $Q$ there is a decomposition of the slice
$\langle S_{j}^{(a)},\partial Q\rangle\llcorner F$ into a translated reference template on $F$ plus a graph error
current supported where the $C^1$--graph parametrization fails; cf.\ Lemma~\ref{lem:face-trace-decomposition}.
\end{enumerate}

Remark (optional geometric disjointness).
If one additionally chooses the parameters $t_{j,a}$ so that the footprints $E_{j,a}$ are
pairwise separated by $\gtrsim s$, then the corresponding components $S_{j}^{(a)}$ are pairwise disjoint inside $Q$,
and necessarily $N_j\lesssim (h/s)^{2p}$ by a packing bound.  None of the subsequent arguments requires such
disjointness; multiplicities are allowed.
\end{theorem}

\begin{proof}
Fix $(j,a)$.
Work in the chosen normal holomorphic chart on $Q$ and in $j$--adapted coordinates.
Apply Proposition~\ref{prop:tangent-approx-full} with tolerance $\varepsilon_{\mathrm{hol}}$ and tensor power $\mhol$
to obtain global sections $s_1,\dots,s_p\in H^0(X,L^{\mhol})$ whose common zero set $Y_{j}^{(a)}$ is, inside
$B_{c\,\mhol^{-1/2}}(x_{j,a})$, a $C^1$ graph over the affine plane $\{w=t_{j,a}\}$ with
$\angle(T_yY_{j}^{(a)},\widetilde\Pi_j)\le\varepsilon_{\mathrm{hol}}$.
Using the corner--exit normal forms encoded in Propositions~\ref{prop:holomorphic-corner-exit-L1}
and~\ref{prop:holomorphic-corner-exit-g1g2} (with Lemma~\ref{lem:bergman-control} as the analytic input),
we arrange that on the holomorphic scale $s=\sigma\,\mhol^{-1/2}$ the component $S_{j}^{(a)}\subset Y_{j}^{(a)}\cap Q$
has the asserted face--trace template decomposition.
The mass estimate follows from the scaling computation for the corner--exit template
(Lemma~\ref{lem:corner-exit-mass-scale}) and stability of mass under $C^1$ perturbations.
No packing argument is needed because we allow multiplicity and do not require geometric disjointness.
\end{proof}



\begin{lemma}[Face--trace decomposition as translated template slices]\label{lem:face-trace-decomposition}
Assume the hypotheses and conclusions of Theorem~\ref{thm:local-sheets} for a fixed $(j,a)$ and write
$S_{j}^{(a)}=[Y_{j}^{(a)}]\llcorner Q$.
Let $F\subset \partial Q$ be a codimension-$1$ face.  Work in the holomorphic normal chart for $Q$ and
use the orthogonal splitting $\C^n=\widetilde\Pi_j\oplus \widetilde\Pi_j^\perp$.
Denote by $\tau_t$ translation in the transverse factor $\widetilde\Pi_j^\perp$ (so $\tau_t(u,w)=(u,w+t)$).

\smallskip\noindent
Then there exists an integral $(k\!-\!1)$--current $R_{j,F}$ supported in $F\cap(\widetilde\Pi_j\times\{0\})$
(depending only on the \emph{flat} corner--exit template and on which face $F$ is involved) such that, if the
flat footprint $E_{j,a}$ exits $Q$ through $F$, one has the decomposition
\[
(\partial S_{j}^{(a)})\llcorner F \;=\; (\tau_{t_{j,a}})_\# R_{j,F}\;+\;E_{j,a,F},
\]
where $t_{j,a}\in \widetilde\Pi_j^\perp$ is the translation parameter of the footprint $E_{j,a}=E_j+t_{j,a}$ and
the error current $E_{j,a,F}$ is a normal $(k\!-\!1)$--current supported in $F$ satisfying the quantitative bound
\[
\Mass(E_{j,a,F})\ \le\ C_{\mathrm{tr}}\,\varepsilon_{\mathrm{hol}}\,s^{k-1}.
\]
If $E_{j,a}$ does \emph{not} meet $F$ (so that the flat template has no boundary contribution on $F$), then
$(\partial S_{j}^{(a)})\llcorner F=E_{j,a,F}$ with the same mass bound.

\smallskip\noindent
Moreover, if one takes integer multiplicity $N\in\N$ (i.e.\ the current $N\,S_{j}^{(a)}$), then the same
decomposition holds with $(\tau_{t_{j,a}})_\# R_{j,F}$ replaced by $N(\tau_{t_{j,a}})_\# R_{j,F}$ and
$\Mass(E_{j,a,F})\le C_{\mathrm{tr}}\,\varepsilon_{\mathrm{hol}}\,N\,s^{k-1}$.
\end{lemma}

\begin{proof}
By Theorem~\ref{thm:local-sheets}(i), inside the chart the set $Y_{j}^{(a)}\cap Q$ is a $C^1$--small graph
over the flat corner--exit footprint $E_{j,a}\subset \widetilde\Pi_j$ with slope bounded by $O(\varepsilon_{\mathrm{hol}})$.
Hence the boundary trace on a face $F$ is a $C^1$--small graph over the corresponding flat boundary face piece
$\partial(E_{j,a}\cap Q)\cap F$.  Define $R_{j,F}$ to be the integral current carried by this flat boundary piece
in $F\cap(\widetilde\Pi_j\times\{0\})$, with the orientation induced by the calibrated template.
Translation by $t_{j,a}$ in the transverse factor carries $R_{j,F}$ to the flat reference slice for $E_{j,a}$.

The error current $E_{j,a,F}$ is defined as the difference between the graph trace and the translated flat slice.
Its mass is controlled by the standard area estimate for $C^1$ graphs (see Lemma~\ref{lem:small-graph-distortion}):
the Jacobian differs from $1$ by $O(\varepsilon_{\mathrm{hol}})$ and the base $(k\!-\!1)$--area of the flat boundary
piece is $\asymp s^{k-1}$ (because the template diameter is $\asymp s$), giving
$\Mass(E_{j,a,F})\le C_{\mathrm{tr}}\varepsilon_{\mathrm{hol}}s^{k-1}$.
Linearity gives the multiplicity statement.
\end{proof}




\begin{proof}
For each fixed direction label $j$, apply Proposition~\ref{prop:holomorphic-corner-exit-L1}
(with the chosen footprint family $\{E_{j,a}\}_{a=1}^{N_j}$) to obtain the corresponding complete intersections
$\{Y_{j}^{(a)}\}_{a=1}^{N_j}$ and the stated properties on $Q$.
Items (i)--(iv) are exactly the conclusions of that proposition together with the quantitative graph control in
Lemma~\ref{lem:bergman-control}.
\end{proof}
\endgroup


Fix a finite normal coordinate atlas by geodesic balls of radii $\ll 1$
and subordinate cubes $\{Q\}$ small enough so that the Carath\'eodory
data from Lemma~\ref{lem:caratheodory-general} are $\varepsilon$-stable
on each cube.  For each cube $Q$ and each index $j\in\{1,\ldots,N\}$,
let $\Pi_{Q,j}$ denote a constant complex $(n-p)$-plane approximating
$P_{x,j}$ on $Q$.  {Apply Theorem~\ref{thm:local-sheets} to each cube to obtain, for each direction label $(Q,j)$, a $\psi$--calibrated sheet piece $Y_{Q,j}\subset Q$.
We allow integer multiplicity and do not require any disjointness: replace $\sum_{a=1}^{N_{Q,j}}[Y_{Q,j}^a]$ by $N_{Q,j}[Y_{Q,j}]$.}

Define the local current
\[
S_Q := \sum_{j=1}^{N} N_{Q,j}\,[Y_{Q,j}]\llcorner Q.
\]
By construction, each $Y_{Q,j}$ is $\psi$-calibrated; hence $S_Q$ is a
positive $\psi$-calibrated integral current on $Q$.  Its tangent-plane
distribution on $Q$ is a convex combination of directions within
$\varepsilon$ of $\{\Pi_{Q,j}\}$ with weights proportional to the $\psi$--masses
for each direction (equivalently proportional to $N_{Q,j}A_{Q,j}$, where $A_{Q,j}$ is
the $\psi$--mass of a single $(Q,j)$-sheet in $Q$).

\begin{lemma}[Local barycenter and mass matching on a cube]\label{lem:local-bary}
Fix a cube $Q$ of side length $h:=\mathrm{side}(Q)$ and a basepoint $x_Q\in Q$.
Let
\[
\beta(x_Q)=t_Q\sum_{j=1}^{J(Q)}\theta_{Q,j}\,\xi_{Q,j},
\qquad
\theta_{Q,j}\ge 0,\ \sum_{j=1}^{J(Q)}\theta_{Q,j}=1,
\qquad
\langle \xi_{Q,j},\psi_{x_Q}\rangle=1,
\]
be a Carath\'eodory decomposition as in Lemma~\ref{lem:caratheodory-general}, where $J(Q)\le N(n,p)$.
Let $P_{Q,j}$ be the complex $(n-p)$--plane corresponding to $\xi_{Q,j}$ and let $Y_{Q,j}\subset Q$
be the $\psi$--calibrated sheet piece from Theorem~\ref{thm:local-sheets} associated to $(Q,j)$.
Set the cube budget
\[
M_Q:=m\int_Q \beta\wedge\psi.
\]

Then for every $\delta\in(0,1)$ one can choose auxiliary parameters $(s,\mhol)$ with $0<s\ll h$ and
$s=\sigma_{\mathrm{tpl}}\mhol^{-1/2}$, and integers $N_{Q,1},\ldots,N_{Q,J(Q)}\ge 0$ such that the local current
\[
S_Q := \sum_{j=1}^{J(Q)} N_{Q,j}\,[Y_{Q,j}]\llcorner Q
\]
satisfies the quantitative mass and barycenter bounds
\[
\bigl|\Mass(S_Q)-M_Q\bigr|\le \delta\,\max\{M_Q,h^{2n}\},
\]
and, writing $\mathsf{Bar}(S_Q)$ for the (mass--normalized) barycenter of the tangent--plane Young measure of $S_Q$, \REV{(Young measures: see \cite{KinderlehrerPedregal91,Pedregal97}).}
\[
\|\mathsf{Bar}(S_Q)-\widehat\beta(x_Q)\|_{\mathrm{HS}}
\le C\,\delta,
\]
where $C$ depends only on $(X,\omega)$ and $(n,p)$.
In particular, if $h$ is chosen so that $\|\widehat\beta(x)-\widehat\beta(x_Q)\|_{\mathrm{HS}}\le \delta$ for all $x\in Q$,
then $\|\mathsf{Bar}(S_Q)-\widehat\beta(x)\|_{\mathrm{HS}}\le (C+1)\delta$ on $Q$.
\end{lemma}

\begin{proof}
{\noindent\textbf{Step 1: choose the footprint scale so integer rounding is possible.}}
Fix $\delta\in(0,1)$.
Choose $s\ll h$ so that for each direction label $(Q,j)$ the corner--exit template mass scale satisfies
\[
A_{Q,j}(s)\ \le\ \frac{\delta}{2J(Q)}\,\max\{M_Q,h^{2n}\},
\]
which is possible since $A_{Q,j}(s)\asymp s^{2n-2p}\to 0$ as $s\downarrow 0$.
Then choose $\mhol$ so that $s=\sigma_{\mathrm{tpl}}\mhol^{-1/2}$ and Lemma~\ref{lem:bergman-control}
applies on the required $B_{c\mhol^{-1/2}}$ neighborhoods, with graph error $\ll s$ on each footprint.

{\noindent\textbf{Step 2: realize one sheet per direction and use multiplicity (no packing).}}
For each $j\le J(Q)$, pick one translation parameter $t=t(Q,j)$ in the template box from
Lemma~\ref{lem:complex-corner-exit-template}, so that the flat footprint $E_{Q,j}(t)=(P_{Q,j})_t\cap Q$
is the reference $k$--simplex (dimension $k=2n-2p$) of $\psi$--mass $A_{Q,j}(s)$.
Realize this footprint holomorphically by Theorem~\ref{thm:local-sheets} to obtain a $\psi$--calibrated
sheet piece $Y_{Q,j}\subset Q$ with tangent planes within $\varepsilon_{\mathrm{ang}}$ of $P_{Q,j}$ and
\[
\Mass([Y_{Q,j}]\llcorner Q) = A_{Q,j}(s)\,(1+\eta_{Q,j}),\qquad |\eta_{Q,j}|\le c_0\,\delta,
\]
after choosing $\varepsilon_{\mathrm{ang}}\le c_1\delta$ and $\mhol$ large (constants $c_0,c_1$ depend only on $(X,\omega)$).

{\noindent\textbf{Step 3: integer rounding.}}
If $M_Q=0$, set $N_{Q,j}=0$ for all $j$ and the conclusion is immediate.
Assume $M_Q>0$ and set
\[
N_{Q,j}:=\Bigl\lfloor \frac{\theta_{Q,j}M_Q}{A_{Q,j}(s)}\Bigr\rfloor.
\]
Then $\bigl|N_{Q,j}A_{Q,j}(s)-\theta_{Q,j}M_Q\bigr|\le A_{Q,j}(s)$, hence
\[
\Bigl|\sum_{j=1}^{J(Q)}N_{Q,j}A_{Q,j}(s)-M_Q\Bigr|
\le \sum_{j=1}^{J(Q)}A_{Q,j}(s)\le \delta\,\max\{M_Q,h^{2n}\}.
\]
Using $\Mass([Y_{Q,j}]\llcorner Q)=A_{Q,j}(s)(1+\eta_{Q,j})$ and $|\eta_{Q,j}|\le c_0\delta$,
we obtain
\[
\Bigl|\Mass(S_Q)-\sum_{j}N_{Q,j}A_{Q,j}(s)\Bigr|
\le c_0\delta\sum_j N_{Q,j}A_{Q,j}(s)
\le c_0\delta\,\Mass(S_Q),
\]
and combining with the previous display yields (after absorbing constants into $\delta$) the stated quantitative mass bound.

{\noindent\textbf{Step 4: barycenter control.}}
{We identify an oriented complex $(n-p)$--plane $\Pi$ with its associated Hilbert--Schmidt unit tensor $\xi_{\Pi}$
(e.g.\ via the orthogonal projection onto $\Pi$, normalized in $\mathrm{HS}$ norm).
For a rectifiable current $T$ with approximate tangent plane $T_yT$ for $\|T\|$--a.e.\ $y$, define the normalized
tangent-plane barycenter by
\[
\mathsf{Bar}(T):=\frac{1}{\Mass(T)}\int \xi_{T_yT}\,d\|T\|(y).
\]
A standard Lipschitz estimate on the Grassmannian \REVMZ{\cite{GH78}} gives $\|\xi_{\Pi}-\xi_{\Pi'}\|_{\mathrm{HS}}\le C\,\angle(\Pi,\Pi')$,
hence if $\angle(T_yT,\Pi)\le \varepsilon$ $\|T\|$--a.e.\ then $\|\mathsf{Bar}(T)-\xi_\Pi\|_{\mathrm{HS}}\le C\varepsilon$.}
Let $w_j:=\Mass(N_{Q,j}[Y_{Q,j}]\llcorner Q)/\Mass(S_Q)$ so that $\sum_j w_j=1$.
Because each $Y_{Q,j}$ has tangent planes within $\varepsilon_{\mathrm{ang}}$ of $P_{Q,j}$,
the barycenter of the tangent-plane Young measure satisfies
\[
\bigl\|\mathsf{Bar}(S_Q)-\sum_{j=1}^{J(Q)} w_j\,\xi_{Q,j}\bigr\|_{\mathrm{HS}}
\le C_2\,\varepsilon_{\mathrm{ang}}.
\]
Moreover, the rounding and mass bound above imply $|w_j-\theta_{Q,j}|\le C_3\,\delta$ uniformly in $j$,
hence
\[
\bigl\|\sum_j w_j\,\xi_{Q,j}-\sum_j\theta_{Q,j}\,\xi_{Q,j}\bigr\|_{\mathrm{HS}}
\le C_4\,\delta.
\]
Since $\widehat\beta(x_Q)=\sum_j\theta_{Q,j}\,\xi_{Q,j}$ by normalization, and $\varepsilon_{\mathrm{ang}}\le c_1\delta$,
we obtain $\|\mathsf{Bar}(S_Q)-\widehat\beta(x_Q)\|_{\mathrm{HS}}\le C\,\delta$, as claimed.
\end{proof}

% ------------------------------------------------------------
%=========================================================
% Corner--exit template direction variation tools (moved out of Theorem~\ref{thm:global-cohom} proof)
%=========================================================

\begin{remark}[Parameter schedule for $\mhol$, $h$, $s$, and $m$]\label{rem:weighted-scaling}
\REVMZ{\textbf{[Technical clarification.]}}
{
We use two unrelated integers throughout:
\begin{itemize}
\item the \emph{holomorphic tensor power} $\mhol$ (Bergman/H\"ormander scale), and
\item the \emph{cohomology quantization integer} $m$ (integrality/rounding scale).
\end{itemize}
The \emph{corner--exit footprint} scale is tied to $\mhol$ by
\[
s:=\sigma_{\mathrm{tpl}}\,\mhol^{-1/2},\qquad 0<\sigma_{\mathrm{tpl}}\le \frac{c(X,\omega)}{20},
\]
where $c(X,\omega)$ is the radius constant from Lemma~\ref{lem:bergman-control}.  All SYR/gluing arguments are invoked with the following
\emph{non-circular} order:
\begin{enumerate}
\item Choose the mesh size $h>0$ so small that every cube of side $h$ lies in a normal holomorphic coordinate chart (e.g.\ $h<\operatorname{inj}(X)/100$).
\item Choose $\mhol\ge m_{\mathrm{hol},1}$ so large that $s=\sigma_{\mathrm{tpl}}\mhol^{-1/2}\le h/100$ and Lemma~\ref{lem:bergman-control}
applies on balls of radius $c(X,\omega)\,\mhol^{-1/2}$.
\item Fix a graph-slope tolerance $\varepsilon_{\mathrm{hol}}>0$ used in the local realization layer; then (if needed) increase $\mhol$ further so that
all holomorphic sheets produced there satisfy $\|DG\|\le \varepsilon_{\mathrm{hol}}$ on $s$-balls.
\item Finally, require that the (already fixed) $m$ is large enough for the rounding/coherence step; this choice does \emph{not} affect the already-fixed $s$.
\end{enumerate}
In particular, we do \emph{not} assume any relation $h\asymp \mhol^{-1/2}$, and we do not rely on packing/disjointness bounds:
we allow multiplicity (cf.\ the direction-net proposition below).}

\end{remark}

\begin{lemma}[Corner-exit translation templates for a quantitative family of complex planes]\label{lem:corner-exit-template-open}
Work in $\C^n=\C^{n-p}\times\C^p$ with coordinates $z=(u,w)$ and identify $\C^n\cong\R^{2n}$.
Let $h>0$ denote the mesh (cube side-length) parameter, and let $Q:=[0,h]^{2n}$ be the coordinate cube.
Fix $0<c_0<1$ and parameters $\alpha_*,\alpha^*,A_*>0$.

\smallskip\noindent
Let $P\subset\C^n$ be a complex $(n-p)$--plane written as a graph
\[
P\ =\ \{(u,Au):u\in\C^{n-p}\},
\]
for some complex linear map $A:\C^{n-p}\to\C^p$ with operator norm $\|A\|\le A_*$.
Suppose that for some choice of a \emph{slanted} coordinate $w_r$ (one of the $p$ components of $w$), the corresponding row of $A$ has coefficients
$c_j=a_j+i b_j$ ($1\le j\le n-p$) satisfying the quantitative nondegeneracy bounds
\[
\alpha_*\ \le\ |a_j|\ \le\ \alpha^*,\qquad \alpha_*\ \le\ |b_j|\ \le\ \alpha^*\qquad (1\le j\le n-p).
\]
Define the conditioning ratio $\Lambda:=\alpha^*/\alpha_*$.


\smallskip\noindent\textbf{Remark (tracking the conditioning constants).}
The parameters $\alpha_*,\alpha^*,A_*$ (hence $\Lambda=\alpha^*/\alpha_*$) are \emph{inputs} describing a quantitative transversality of the chosen
``slanted'' coordinate row of $A$ for the specific plane $P$.
When this lemma is applied to a \emph{finite direction net} (the direction-net proposition below), one typically takes
\[
\begin{aligned}
	\alpha_*(h) &:= \min_{P_i\in\mathcal N_h}\alpha_*(P_i), &
	\alpha^*(h) &:= \max_{P_i\in\mathcal N_h}\alpha^*(P_i), \\
	A_*(h) &:= \max_{P_i\in\mathcal N_h}A_*(P_i), &
	\Lambda(h) &:= \frac{\alpha^*(h)}{\alpha_*(h)} .
\end{aligned}
\]

so these constants may depend on $(h,\varepsilon_h)$ as the net is refined.
Unless a uniform-in-$h$ lower bound on $\alpha_*(h)$ is proved, the later scaling schedule must keep the dependence on $(1+A_*(h))\Lambda(h)$ explicit.


\smallskip\noindent
Then there exists a choice of a \emph{vertex} $v$ of $Q$ (equivalently, a choice of which incident coordinate faces of $Q$ provide the ``orthant'' constraints)
and a choice of a translation parameter $t\in\C^p$ with a scale $s:=|\,\Re t_r\,|$ satisfying
\[
s\ \le\ \frac{c_0}{C(n,p)}\cdot \frac{h}{(1+A_*)\,\Lambda},
\]
such that, writing $P_t:=P+t$ and $E:=P_t\cap Q$, the footprint $E$ is a $k$--simplex ($k=2n-2p$) contained in $B(v,c_0h)$ whose $k\!+\!1$ facets lie on
exactly $k\!+\!1$ coordinate faces of $Q$ incident to $v$ (a designated exit-face set), and the simplex is uniformly fat with constant depending only on
$(n,p,\Lambda)$.

\smallskip\noindent
Moreover, one may choose $t$ from a $(2p\!-\!1)$--dimensional parameter box (fixing $\Re t_r=\pm s$ and varying the remaining real components with margin $\asymp s$),
so that the resulting footprints are \emph{identical} (hence have equal slice mass) throughout that box.  In particular, for any separation scale $\delta>0$ one can
extract an ordered $\delta$--separated list of translations producing identical corner-exit simplex footprints.
\end{lemma}

\begin{proof}
Write $u_j=x_j+i y_j$.
By reflecting real coordinates $x_j\mapsto h-x_j$ and/or $y_j\mapsto h-y_j$ (which corresponds to choosing a vertex $v$ of $Q$),
we may replace $(x_j,y_j)$ by nonnegative coordinates $(x'_j,y'_j)\in[0,h]$ so that the affine inequality
$\Re w_r\ge 0$ restricted to $P_t$ becomes
\[
\sum_{j=1}^{n-p} (|a_j|\,x'_j+|b_j|\,y'_j)\ \le\ s,
\]
after absorbing the resulting additive constants into the choice of $\Re t_r$.
Together with the orthant constraints $x'_j\ge 0$, $y'_j\ge 0$, this cuts out a $k$--simplex in the $k=2(n-p)$ real variables.
The bound $s\ll h$ prevents meeting the far faces in the $u$-coordinates.

\smallskip\noindent
By $\|A\|\le A_*$ and the simplex bound $|u|\lesssim s/\alpha_*$, all other cube coordinates (the remaining $w$ components and the $\Im w_r$ coordinate)
vary by at most $O(A_* s/\alpha_*)$ on $E$.  Choosing the remaining components of $t$ with margin $\asymp s$ and taking
$s\le c_0\,h/(C(1+A_*)\Lambda)$ forces these coordinates to stay in $(0,h)$, so no additional faces are met.
Uniform fatness and volume scaling follow by an affine change of variables on $\R^k$ controlled by $\Lambda$.
\smallskip

\noindent
Finally, to obtain a template family with identical footprints, fix $\Re t_r=\pm s$ and vary the remaining real components of $t$
in a box of sidelength $\asymp s$ chosen so that all the non-$u$ cube coordinates remain strictly inside $(0,h)$ as above.
On this parameter box, the defining inequalities in the $(x'_j,y'_j)$ variables are unchanged, so the footprint in $Q$ is identical for all such $t$.
Extracting a $\delta$--separated ordered list from the box is a standard packing argument in dimension $2p-1$.
\end{proof}

\begin{proposition}[Robust corner-exit templates for a finite direction net]\label{prop:corner-exit-template-net}
Fix $h>0$ and a tolerance $\varepsilon_h>0$.
In any fixed holomorphic coordinate chart, there exists a finite set of calibrated directions
\[
\mathcal N_h=\{P_1,\dots,P_M\}\subset G_\C(n-p,n)
\]
which is an $\varepsilon_h$--net in $G_\C(n-p,n)$ and has the following property:
for each $P_i\in\mathcal N_h$ there is a corner-exit translation template family in the cube $Q=[0,h]^{2n}$ (allowing choice of vertex and exit-face set)
whose footprints are uniformly fat corner-exit simplices, and which supplies an arbitrarily long $\delta$--separated ordered list of translations (for any $\delta>0$)
with identical footprint geometry (hence uniform per-piece slice mass within each label).
Moreover, because $\mathcal N_h$ is finite, the fatness/locality constants may be chosen \emph{uniformly} over all directions in $\mathcal N_h$.
\end{proposition}

\begin{proof}
Let $\mathcal U\subset G_\C(n-p,n)$ be the set of planes for which there exists some coordinate splitting and some choice of slanted coordinate $w_r$
so that the corresponding row coefficients satisfy $a_j\neq 0$ and $b_j\neq 0$ for all $j$; this is a finite union of complements of algebraic
degeneracy loci (vanishing of Pl\"ucker minors \REVMZ{\cite{GH78}} and coordinate coefficients), hence dense.
Start with any $\varepsilon_h/2$--net and perturb each point by $<\varepsilon_h/2$ into $\mathcal U$; compactness gives a finite net $\mathcal N_h\subset\mathcal U$.

\smallskip\noindent
For each $P_i\in\mathcal N_h$, choose a witnessing splitting and slanted coordinate, and let $\alpha_*(i),\alpha^*(i),A_*(i)$ be the resulting quantitative constants.
Since $\mathcal N_h$ is finite and all required coefficients are nonzero, one has $\alpha_*:=\min_i\alpha_*(i)>0$ and $A_*:=\max_iA_*(i)<\infty$.
Apply Lemma~\ref{lem:corner-exit-template-open} with these uniform constants to obtain uniform corner-exit templates for every $P_i$.

\noindent\textbf{Net constants $\alpha_*(h),A_*(h),\Lambda(h)$.}
For each fixed $h$ we choose the net $\mathcal N_h\subset\mathcal U$ inside the open ``nondegenerate'' set $\mathcal U$ (defined below) so that the
row-coefficient nonvanishing required by Lemma~\ref{lem:corner-exit-template-open} holds for every $P_i\in\mathcal N_h$.
Because $\mathcal N_h$ is finite, we may define the \emph{net constants}
\[
\alpha_*(h):=\min_i\alpha_*(i),\qquad 
\alpha^*(h):=\max_i\alpha^*(i),\qquad 
A_*(h):=\max_iA_*(i),\qquad 
\Lambda(h):=\alpha^*(h)/\alpha_*(h),
\]
which are uniform \emph{across labels} for this fixed $h$.

\smallskip\noindent
A uniform-in-$h$ positive lower bound for $\alpha_*(h)$ is \emph{not} established here (as $\varepsilon_h\downarrow0$ the net may approach
degeneracy loci where some row coefficients become small).
Therefore we adopt option \textbf{(2)} in the closure chain:
the later parameter schedule (Remark~\ref{rem:weighted-scaling}) keeps the dependence on $(1+A_*(h))\Lambda(h)$ explicit and enforces the
corner-exit scale condition from Lemma~\ref{lem:corner-exit-template-open},
\[
s\ \le\ \frac{c_0}{C(n,p)}\cdot \frac{h}{(1+A_*(h))\,\Lambda(h)},
\]
whenever Proposition~\ref{prop:holomorphic-corner-exit-L1} is applied uniformly over all labels.


\end{proof}


\subsection*{Step 4: Global cohomology quantization (Theorem C)}

This step forces the global integral current to represent exactly the
correct homology class $\mathrm{PD}(m[\gamma])$ by using lattice
discreteness.

\begin{theorem}[Global cohomology quantization]\label{thm:global-cohom}
Let $X$ be a compact K\"ahler $n$-fold with rational Hodge class
$[\gamma]\in H^{2p}(X,\Q)$ represented by a smooth closed $(p,p)$-form
$\beta$ with $\beta(x)\in K_p(x)$ pointwise.  {There exists an integer $m\ge 1$ (clearing denominators of $[\gamma]$) such that for every $\varepsilon>0$
there exist a mesh size $h=h(\varepsilon)>0$ and an $h$--cubulation $\{Q\}$ of $X$ subordinate to holomorphic normal coordinate charts, and there exist:}
\begin{itemize}
\item A closed integral $(2n-2p)$-current $T_\varepsilon$ with
$[T_\varepsilon]=\mathrm{PD}(m[\gamma])$;
\item A correction current $R_\varepsilon$ with $\Mass(R_\varepsilon)<\varepsilon$;
\end{itemize}
{\noindent\textbf{Parameter schedule.} Fix the denominator--clearing integer $m$ once and for all. Given a target error $\varepsilon>0$, choose a small tolerance $\delta=\delta(\varepsilon)\in(0,1)$, then choose the cubulation side length $h=h(\delta)\in(0,r)$ so that (i) $\beta$ and the metric vary by at most $\delta$ on each cube, and (ii) the face identifications of Section~\ref{sec:atlas-face-identifications} are well-defined. Next choose a footprint scale $s=s(h,\delta)\ll h$ for the corner--exit templates so that each per--piece mass satisfies $A_{Q,j}(s)\le \frac{\delta}{2J(Q)}\max\{M_Q,h^{2n}\}$ on every cube and direction. Finally choose a holomorphic tensor power $\mhol=\mhol(h,\varepsilon)$ large enough to apply the Bergman/H\"ormander local realization on each cube of side $h$ (more precisely, choose $\mhol$ so that the footprint scale $s:=\sigma_{\mathrm{tpl}}\,\mhol^{-1/2}$ satisfies $s\ll h$ and is small enough that $A(s)\asymp s^{2n-2p}\ll h^{2n}/m$, and so that the $C^1$ graph error from Lemma~\ref{lem:bergman-control} is $\ll s$ on each footprint neighborhood.). In particular $m$ is fixed and does not depend on $\varepsilon$; only $(\delta,h,s,\mhol)$ vary with the approximation.}

such that the local tangent-plane mass proportions on each $Q$ match
those of $\beta$ up to error $o_{\varepsilon\to 0}(1)$.
{\noindent\textbf{Clarification of the meaning of ``mass proportions''.\label{rem:mass-proportions-meaning}}
For each cube $Q$ of side length $h:=\mathrm{side}(Q)$ fix a basepoint $x_Q\in Q$ and choose a Carath\'eodory decomposition
$\beta(x_Q)=t_Q\sum_{j=1}^{J(Q)}\theta_{Q,j}\,\xi_{Q,j}$ with $\langle \xi_{Q,j},\psi_{x_Q}\rangle=1$ as in Substep~4.1 below.
Set the \emph{cube budget}
$M_Q:=m\int_Q \beta\wedge\psi$.
Then the local piece $T_\varepsilon\llcorner Q$ can be chosen of the form
$S_Q=\sum_{j=1}^{J(Q)} S_{Q,j}$ where each $S_{Q,j}$ is carried by $\psi$--calibrated sheets
with tangent planes within $\varepsilon$ of the direction $P_{Q,j}$ corresponding to $\xi_{Q,j}$, and
\[
\bigl|\Mass(S_{Q,j})-\theta_{Q,j}M_Q\bigr|\le \varepsilon\,\max\{M_Q,h^{2n}\},
\qquad
\bigl|\Mass(S_Q)-M_Q\bigr|\le \varepsilon\,\max\{M_Q,h^{2n}\}.
\]
This is the precise sense in which the ``local tangent-plane mass proportions'' match those of $\beta$.}

\end{theorem}

\begin{remark}[Roadmap and scope of Theorem~\ref{thm:global-cohom}]\label{rem:global-cohom-commentary}
\REVMZ{\textbf{[Proof roadmap.]}}
Theorem~\ref{thm:global-cohom} is proved by (i) per-cube integer \emph{mass budgeting} (Substep~4.1), 
(ii) a \emph{coherent template} and transport--gluing scheme that controls $\mathcal F(\partial T^{\mathrm{raw}})$ (Substep~4.2), and 
(iii) a final \emph{filling} of the small boundary (Proposition~\ref{prop:glue-gap}) together with cohomology matching (Proposition~\ref{prop:cohomology-match}) and almost-calibration (Proposition~\ref{prop:almost-calibration}) (Substep~4.3).
\end{remark}



\begin{proof}
The proof proceeds in three substeps.

\medskip\noindent
\textbf{Substep 4.1: Local quantization.}
Choose the partition $\{Q\}$ fine enough that on each $Q$, $\beta(x)$
is within $\delta$ (in operator norm) of $\beta(x_Q)$ for a base point
$x_Q\in Q$, and the K\"ahler metric is nearly constant (Jacobian and
volume distortion $\le 1+\delta$).

By Lemma~\ref{lem:caratheodory-general}, write
\[
\beta(x_Q)=t_Q\sum_{j=1}^{J(Q)}\theta_{Q,j}\,\xi_{Q,j},
\qquad
t_Q:=\langle \beta(x_Q),\psi_{x_Q}\rangle,
\]
where $\xi_{Q,j}\in K_p(x_Q)$ are normalized extremal generators (coming from
complex $(n-p)$-planes) satisfying $\langle \xi_{Q,j},\psi_{x_Q}\rangle=1$,
the weights satisfy $\theta_{Q,j}\ge 0$, $\sum_j\theta_{Q,j}=1$, and
$J(Q)\le N=N(n,p)$ uniformly bounded.

Since $[\gamma]$ is rational, all its periods lie in $(1/M)\Z$ for some
fixed $M$.  Choose $m\gg 1$ divisible by $M$.

Let $P_{Q,j}\subset T_{x_Q}X$ be the complex $(n-p)$-plane corresponding to $\xi_{Q,j}$.

\smallskip\noindent\textbf{Remark (per--cube integer matching without a forbidden quantifier swap).}
It is tempting to assert that \emph{any} affine translate of a fixed calibrated plane has constant $\psi$--mass in a cube $Q$.
This is false in general (even for a line intersecting a square in $\R^2$), and it leads to the scaling obstruction when the mesh is refined.

Instead, we tie the footprint size to the holomorphic scaling used to realize sheets.
Fix a small constant $\sigma\in\bigl(0,\tfrac{c(X,\omega)}{100}\bigr)$ (with $c(X,\omega)$ as in Lemma~\ref{lem:bergman-control})
and choose an auxiliary holomorphic tensor power $\mhol\ge m_{\mathrm{hol},1}(\varepsilon_{\mathrm{hol}})$ so that the
\emph{holomorphic scale} $s:=\sigma\,\mhol^{-1/2}$ satisfies $s\le h/100$ and
$A(s)\ll h^{2n}/m$ (equivalently, $s^{2n-2p}\ll h^{2n}/m$ since $A(s)\asymp s^{2n-2p}$).
This removes the fixed--$m$ scaling obstruction when the mesh is refined.
For each direction label $(Q,j)$, apply Proposition~\ref{prop:corner-exit-template-net} (in the flattened cube $Q$) to select a direction label $j$ with an associated corner--exit plane $P_{Q,j}$
to obtain a parameter box of translations $t$ for which the footprints
\[
E_{Q,j}(t):=(P_{Q,j})_t\cap Q
\]
are \emph{identical} $k$--simplices of dimension $k:=2n-2p$; in particular their $\psi$--mass in $Q$ is independent of $t$.
Denote this common per--piece $\psi$--mass by
\[
A_{Q,j}=A_{Q,j}(s):=\Mass\bigl(E_{Q,j}(t)\bigr)\asymp s^{k}
\quad\text{(independent of the chosen }t\text{ in the template box).}
\]
Define the \emph{cube budget}
\[
M_Q := m\int_Q \beta\wedge\psi ,
\]
which satisfies $M_Q=O(h^{2n})$ uniformly (since $\beta$ is smooth and $m$ is fixed).

Choose $s$ small enough that
\[
A_{Q,j}\ \le\ {\frac{\delta}{2J(Q)}}\,\max\{M_Q,h^{2n}\}
\qquad\text{for all }j,
\]
which is possible because $A_{Q,j}(s)\to0$ as $s\downarrow0$.
Then set
\[
N_{Q,j}:=\Bigl\lfloor \frac{\theta_{Q,j}M_Q}{A_{Q,j}}\Bigr\rfloor
\quad(\text{and }N_{Q,j}=0\text{ if }M_Q=0).
\]
This yields the quantitative per--cube matching
\[
\bigl|N_{Q,j}A_{Q,j}-\theta_{Q,j}M_Q\bigr|\le A_{Q,j},
\qquad
\Bigl|\sum_j N_{Q,j}A_{Q,j}-M_Q\Bigr|\le \sum_j A_{Q,j}\le \delta\,\max\{M_Q,h^{2n}\}.
\]
Thus no step uses ``$m\to\infty$'': $m$ is fixed once (clearing denominators), and the integer rounding is enabled by the tunable footprint scale $s\ll h$.
{Apply Theorem~\ref{thm:local-sheets} to realize each direction $(Q,j)$ by a single $\psi$--calibrated sheet piece $Y_{Q,j}\subset Q$
with angle control and boundary supported on $\partial Q$.  We do \emph{not} require packing or disjointness: for each integer $N_{Q,j}\ge 0$ we use the integral current with
multiplicity $N_{Q,j}[Y_{Q,j}]$ (equivalently, one may take $Y_{Q,j}^a=Y_{Q,j}$ for all $a$).}

Define the raw local current
\[
S_Q:=\sum_{j=1}^{J(Q)} N_{Q,j}\,[Y_{Q,j}]\llcorner Q.
\]



\begin{remark}[Main quantifier schedule]\label{rem:main-quantifiers}
\REVMZ{\textbf{[Technical clarification.]}}
The order of choices in the construction is: fix the integer $m$ (the desired cohomology multiple) first; then choose the mesh size $h$ and holomorphic scale parameters; then choose the tolerance parameters $\delta,\varepsilon_{\mathrm{temp}},\varepsilon_{\mathrm{hol}},\varepsilon_{\mathrm{edit}}$ as explicit functions of $h$ and the fixed geometric constants. In particular, no step requires re-choosing $m$ after passing to a finer mesh.
\end{remark}

\medskip\noindent
\textbf{Substep 4.2: Gluing across cubes.}
Consider the global raw current
\[
T^{\mathrm{raw}}:=\sum_Q S_Q.
\]
This is integral but not closed: $\partial T^{\mathrm{raw}}$ lives on
the union of cube faces.  View the cube adjacency as a finite graph:
vertices $=$ cubes $Q$, edges $=$ codimension-1 faces $F=Q\cap Q'$.
On each oriented face $F$, the restriction of $\partial S_Q$ induces
a $(2n-2p-1)$-current $B_{Q\to F}$ living on $F$.  Summed over all cubes:
\[
\partial T^{\mathrm{raw}}=\sum_F B_F,
\]

\begin{definition}[Coherent templates]\label{def:coherent-templates}
We say that the cubical family $\{S_Q\}_{Q\in\mathcal Q}$ of local template currents is \emph{coherent} if, for every interior face $F=Q\cap Q'$, the induced face traces agree as integral currents after the canonical identification of $F$ from the two sides (via the face maps): 
\[
S_Q\llcorner F = S_{Q'}\llcorner F.
\]
Equivalently, all face-mismatch currents $B_F$ vanish.
\end{definition}

\begin{lemma}[Coherence implies closed]\label{lem:coherence-implies-closed}
If $\{S_Q\}$ is coherent in the sense of Definition~\ref{def:coherent-templates}, then the raw sum $T^{\mathrm{raw}}:=\sum_Q S_Q$ is a cycle: $\partial T^{\mathrm{raw}}=0$. More generally (without coherence), $\partial T^{\mathrm{raw}}=\sum_{F\in\mathcal F_{\mathrm{int}}} B_F$ where each $B_F$ is the corresponding face-mismatch current.
\end{lemma}
\begin{proof}
Apply Stokes to each $S_Q$ and sum over $Q$. Each interior face $F=Q\cap Q'$ occurs twice with opposite orientations. After identifying the two copies of $F$, the contributions cancel in the coherent case, and in general the leftover is exactly the mismatch current $B_F$ by definition.
\end{proof}


where $B_F$ is the mismatch between the two neighboring cubes.

\textbf{Key point (flat norm, not mass):} In general the individual face currents $B_F$
need not have small mass (cancellation-heavy boundaries can have large mass), so the robust
quantity to control is the \emph{flat norm} of the total mismatch $\partial T^{\mathrm{raw}}$.
Recall the flat norm on $(2n-2p-1)$-currents:
\[
\mathcal F(S):=\inf\{\Mass(R)+\Mass(Q):\ S=R+\partial Q\},
\]
where $R$ is an integral $(2n-2p-1)$-current and $Q$ is an integral $(2n-2p)$-current.
On a compact manifold one has the dual characterization (Federer--Fleming
):
\[
\mathcal F(S)=\sup\{S(\eta):\ \eta\in C^\infty\Lambda^{2n-2p-1},\ \|\eta\|_{\mathrm{comass}}\le 1,\
\|d\eta\|_{\mathrm{comass}}\le 1\}.
\]
For $S=\partial T^{\mathrm{raw}}$ and such $\eta$, Stokes
 gives
$S(\eta)=\partial T^{\mathrm{raw}}(\eta)=T^{\mathrm{raw}}(d\eta)$.



%=========================================================
\subsubsection*{Atlas and face identifications for the cubulation}
\label{sec:atlas-face-identifications}
%=========================================================

{The transport/gluing estimates across faces require a precise identification of the transverse
parameter spaces on adjacent cubes. We record here an explicit atlas choice and the induced linear
\emph{face maps} that were used implicitly in earlier drafts.}

\begin{definition}[Face basepoints and linear face maps]
\label{def:face-maps}
Fix a cubulation $\mathcal{Q}_h$ of $X$ by coordinate cubes $Q$ of side $h$ (with $h>0$ sufficiently small).
For each cube $Q$ and each direction label $j$ used in the local holomorphic models,
choose a holomorphic chart
\[
\Psi_{Q,j}\colon U_{Q,j}\longrightarrow V_{Q,j}\subset \C^n=\C^{n-p}\times \C^p,
\qquad z=(u,w),
\]
such that (i) $Q\subset U_{Q,j}$ and $\Psi_{Q,j}$ is a biholomorphism onto its image, and (ii) in these
coordinates the local calibrated sheet families are parametrized by \emph{transverse translations} in the
$w$--variables (so $w\in\C^p\cong \R^{2p}$ is the template/displacement parameter).

Let $F$ be an interior face shared by adjacent cubes $Q,Q'$ and fix a \emph{face basepoint} $x_F\in F$.
Choose a reference holomorphic chart $\Psi_{F,j}$ centered at $x_F$ (for instance, any chart of the above
type defined on a neighborhood of $x_F$).
For each adjacent cube $Q$ define the \emph{shifted} chart
\[
\widetilde\Psi_{Q,F,j}:=\tau_{-\Psi_{Q,j}(x_F)}\circ \Psi_{Q,j},
\qquad\text{so that }\ \widetilde\Psi_{Q,F,j}(x_F)=0,
\]
and define the holomorphic transition map
\[
\Theta_{Q,F,j}:=\Psi_{F,j}\circ \widetilde\Psi_{Q,F,j}^{-1}
\quad\text{(defined near $0\in\C^n$).}
\]
Writing $\Theta_{Q,F,j}(u,w)=(u'(u,w),\,w'(u,w))$ with $w'\in \C^p$, we define the
\emph{linear face map}
\[
\Phi_{Q,F}:=\mathrm{d}\,w'(0,0)\colon \C^p\to \C^p
\qquad(\text{identified with } \R^{2p}\to \R^{2p}).
\]
\end{definition}

\begin{lemma}[Uniform stability of face maps]
\label{lem:face-map-stability}
There exist constants $C_\Phi,C_\Phi'>0$ depending only on $(X,\omega)$ and the fixed choice of local
holomorphic charts in Definition~\ref{def:face-maps} such that, for all sufficiently small $h$ and every
interior face $F=Q\cap Q'$ (with $Q,Q'\in\mathcal{Q}_h$), the associated face maps satisfy
\[
\|\Phi_{Q,F}\|_{\mathrm{op}}+\|\Phi_{Q,F}^{-1}\|_{\mathrm{op}}\le C_\Phi,
\qquad
\|\Phi_{Q,F}-\Phi_{Q',F}\|_{\mathrm{op}}\le C_\Phi'\,h.
\]
\end{lemma}

\begin{proof}
Because $X$ is compact and all charts involved are smooth with uniformly bounded derivatives on their
(common) domains of definition, the differentials of the transition maps
$\Theta_{Q,F,j}$ and $\Theta_{Q',F,j}$ (and their inverses) are uniformly bounded; this gives the
first estimate.

For the second estimate, note that $Q$ and $Q'$ share a face of diameter $\asymp h$, so the relevant
centering points and chart shifts differ by $O(h)$ in the underlying coordinate domains.
Since the transition maps are $C^2$ with uniformly bounded second derivatives, the mean value theorem
implies that their differentials vary at most linearly with the basepoint; hence the operator norm
difference of the $w$--block differentials at $(0,0)$ is $O(h)$, yielding the stated bound.
\end{proof}


\begin{proposition}[Transport control $\Rightarrow$ flat-norm gluing]\label{prop:transport-flat-glue}\label{prop:transport-cell}
Let $k=2n-2p$.  Fix a cubulation of $X$ by coordinate cubes of side length $h=\mathrm{mesh}$, fix a footprint scale
$0<s\le h/100$, and write $T^{\mathrm{raw}}=\sum_Q S_Q$, where each $S_Q$ is a sum of calibrated sheet pieces
restricted to $Q$ (with integer multiplicity allowed).

For each interior face $F=Q\cap Q'$ and each direction label $j$, assume that in the chart for $Q$ the face trace
of each sheet family admits the translated-template decomposition of Lemma~\ref{lem:face-trace-decomposition},
so that one may write
\[
(\partial S_Q)\llcorner F
=\sum_{j}\int (\tau_t)_\# R_{j,F}\, d\mu_{Q\to F,j}(t)\;+\;E_{Q,F},
\]
where $R_{j,F}$ are fixed integral $(k\!-\!1)$--currents (depending only on the flat templates and on the face type),
$\mu_{Q\to F,j}$ are discrete integer-valued measures on the transverse factor $\widetilde\Pi_j^\perp$, and the error
current satisfies
\[
\Mass(E_{Q,F})\ \le\ C_{\mathrm{tr}}\,\varepsilon_{\mathrm{hol}}\,s^{k-1}\sum_j \Mass(\mu_{Q\to F,j}).
\]
Assume the analogous decomposition holds for $Q'$ in its chart, and compare the two transverse coordinate systems
along $F$ by the linear face maps of Definition~\ref{def:face-maps} and Lemma~\ref{lem:face-map-stability}

For each label $j$ we set the comparison map
\[
\Phi_{F,j}:=\Phi_{Q,F,j}\circ(\Phi_{Q',F,j})^{-1}\colon \C^p\to \C^p
\qquad(\text{identified with }\R^{2p}\to\R^{2p}),
\]
which is $(1+O(h))$--biLipschitz by Lemma~\ref{lem:face-map-stability}.


(which is $(1+O(h))$--biLipschitz on the relevant transverse ball).  Define, for each $j$,
\[
\tau_{F,j}:=W_1\!\left(\mu_{Q\to F,j},\,(\Phi_{F,j})_\#\mu_{Q'\to F,j}\right),
\qquad
\Delta_{F,j}:=\Bigl|\Mass(\mu_{Q\to F,j})-\Mass(\mu_{Q'\to F,j})\Bigr|.
\]

%
\noindent\textbf{Discrete integrality convention.}
Since the face measures $\mu_{Q\to F,j}$ and $\mu_{Q'\to F,j}$ are discrete and integer-valued,
we may (and do) fix for each $(F,j)$ an \emph{optimal} coupling realizing $\tau_{F,j}$ whose coefficients are integers,
by Lemma~\ref{lem:w1-integral-optimal}.


Then there exists a normal $k$--current $B_F$ supported in a fixed tubular neighborhood of $F$ such that
\[
\partial B_F
=\bigl((\partial S_Q)\llcorner F\bigr)\;-\;\bigl((\partial S_{Q'})\llcorner F\bigr)\;+\;\widetilde E_{Q,F}+\widetilde E_{Q',F},
\]
where $\widetilde E_{Q,F},\widetilde E_{Q',F}$ are normal error currents with
$\Mass(\widetilde E_{Q,F})\le \Mass(E_{Q,F})$ and $\Mass(\widetilde E_{Q',F})\le \Mass(E_{Q',F})$, and
\[
\Mass(B_F)\ \le\ C_X\,s^{k-1}\sum_j \tau_{F,j}\;+\;C_X\,h\,s^{k-1}\sum_j \Delta_{F,j}.
\]
Consequently, the boundary mismatch satisfies the flat-norm bound
\[
\mathcal F\!\left(\bigl((\partial S_Q)-(\partial S_{Q'})\bigr)\llcorner F\right)
\ \le\ C_X\,s^{k-1}\sum_j \tau_{F,j}\;+\;C_X\,h\,s^{k-1}\sum_j \Delta_{F,j}
\;+\; \Mass(E_{Q,F})+\Mass(E_{Q',F}).
\]
Summing over all interior faces $F$ gives a bound on $\mathcal F(\partial T^{\mathrm{raw}})$ of the same form.
\end{proposition}


\begin{proof}
Fix an interior face $F=Q\cap Q'$.
Work in the product chart for $Q$ identifying a neighborhood of $F$ with
$F\times B^{2p}(0,C h)$ and use the splitting adapted to each label $j$ so that translations $\tau_t$ act in the
transverse factor.  (All constants below may depend on the fixed ambient geometry of $X$ and on the chosen finite
template dictionary, but not on $h,s$.)

\smallskip\noindent
\textbf{Step 1: reduce to transport between discrete measures.}
By the decomposition hypothesis, the only non-error part of the face trace is a finite sum of translated copies
of the fixed reference currents $R_{j,F}$, parameterized by the discrete measures $\mu_{Q\to F,j}$ and
$(\Phi_{F,j})_\#\mu_{Q'\to F,j}$.  The chart compatibility lemma ensures that comparing the two cube charts changes
transport costs by at most a $(1+O(h))$ factor, which is absorbed into $C_X$.

\smallskip\noindent
\textbf{Step 2: build the filling current from a transport plan (normality and mass).}

On a compact metric space, the Kantorovich problem defining $W_1$ admits an optimal plan; moreover
$W_1$ is characterized by the Kantorovich--Rubinstein duality. \cite{Villani03}

Fix $j$.  Let $\pi_{F,j}$ be an optimal transport plan between $\mu_{Q\to F,j}$ and $(\Phi_{F,j})_\#\mu_{Q'\to F,j}$
for the $W_1$ cost.  Since these measures are finite sums of Dirac masses with \emph{integer} weights,
we may expand each atom into unit masses (repeating points by multiplicity), thereby reducing to a finite assignment problem.
Choosing an optimizer at an extreme point, we may take $\pi_{F,j}$ supported on finitely many pairs and with integer
transported masses; in particular
\[
\pi_{F,j}=\sum_{\ell=1}^{N_{F,j}} m_\ell\,\delta_{(t_\ell,t'_\ell)},\qquad m_\ell\in\N,
\qquad \sum_\ell m_\ell |t_\ell-t'_\ell|=\tau_{F,j}
 =W_1\bigl(\mu_{Q\to F,j},(\Phi_{F,j})_\#\mu_{Q'\to F,j}\bigr).
\]

For each atom $(t_\ell,t'_\ell)$ define the \emph{prism current}
\[
P_{j,\ell}:=\bigl([0,1]\times R_{j,F}\bigr)\ \text{pushed forward by}\ (s,x)\mapsto (x,(1-s)t_\ell+s t'_\ell),
\]
and then set $B_{F,j}:=\sum_\ell m_\ell\,P_{j,\ell}$.
A direct boundary computation gives
\[
\partial B_{F,j}=\int (\tau_t)_\# R_{j,F}\,d\mu_{Q\to F,j}(t)\;-\;\int (\tau_t)_\# R_{j,F}\,d(\Phi_{F,j})_\#\mu_{Q'\to F,j}(t).
\]
Each $P_{j,\ell}$ is a pushforward of a product of integral currents, hence is an integral (in particular normal)
$k$--current; therefore $B_{F,j}$ is normal.  Its mass satisfies
\[
\Mass(B_{F,j})\ \le\ C_X\,\Mass(R_{j,F})\sum_\ell m_\ell |t_\ell-t'_\ell|
\ \le\ C_X\,s^{k-1}\,\tau_{F,j},
\]
since $\Mass(R_{j,F})\asymp s^{k-1}$ by construction of the flat template on scale $s$.

\smallskip\noindent
\textbf{Step 3: handle unequal total masses (the $\Delta_{F,j}$ term).}
If $\Mass(\mu_{Q\to F,j})\neq \Mass(\mu_{Q'\to F,j})$, the above coupling matches only the common part.
The remaining unmatched mass $\Delta_{F,j}$ is filled by inserting $\Delta_{F,j}$ many vertical slabs of thickness
$\asymp h$ over $R_{j,F}$ inside the tubular neighborhood of $F$, producing an additional normal current
$\widetilde B_{F,j}$ with $\partial \widetilde B_{F,j}$ equal to the leftover translated slices and
\[
\Mass(\widetilde B_{F,j})\ \le\ C_X\,h\,\Mass(R_{j,F})\,\Delta_{F,j}\ \le\ C_X\,h\,s^{k-1}\Delta_{F,j}.
\]

\smallskip\noindent
\textbf{Step 4: add the analytic graph errors.}
Finally incorporate the error currents $E_{Q,F}$ and $E_{Q',F}$ from the face-trace decomposition.
Set $B_F:=\sum_j(B_{F,j}+\widetilde B_{F,j})$ and absorb the graph errors into $\widetilde E_{Q,F},\widetilde E_{Q',F}$.
The stated mass and flat-norm bounds follow.
\end{proof}





\begin{remark}[Why hypotheses (a)--(b) hold for the local sheet model]\label{rem:transport-hypotheses}
In the flat model of Substep~3.4, each sheet in family $(Q,j)$ is literally an affine calibrated plane
$(\widetilde\Pi_{Q,j}+t_{j,a})\cap Q$, with translation parameter $t_{j,a}\in N_{Q,j}^\perp\cong\R^{2p}$.
For a fixed face $F\subset\partial Q$, the boundary slice current
\[
\Sigma_{F,j}(t):=\partial\big([\widetilde\Pi_{Q,j}+t]\llcorner Q\big)\llcorner F
\]
depends only on $t$ through its component normal to the $(2n-2p-1)$-plane $\widetilde\Pi_{Q,j}\cap TF$.
Thus, in the flat model, $\partial S_Q\llcorner F$ can be written as a finite sum
$\sum_a \Sigma_{F,j}(t_{j,a})$, i.e.\ it is parameterized by the discrete transverse measure
$\mu_{Q\to F}:=\sum_a \delta_{t_{j,a}}$ (with integer weights).

After upgrading to algebraic complete intersections in Substep~3.5, the sheets remain $C^1$-graphs over the flat model on $Q$
(for $k$ large), so the same parameterization persists in a tubular neighborhood of $F$ up to an $O(\varepsilon)$ error
controlled by the graph distortion.  This justifies the use of transverse measures on faces and the small-angle graph model
in Proposition~\ref{prop:transport-flat-glue}.

What is \emph{not} automatic is hypothesis (c): arranging $W_1$ matching across faces simultaneously for all cubes, subject to
the constraint that each sheet's translation parameter determines its intersection with \emph{all} faces of $Q$ at once.
\smallskip
Equivalently, for a fixed cube $Q$ and family $(Q,j)$, the face measures $\mu_{Q\to F}$ for different faces $F\subset\partial Q$
are not independent choices: they arise as pushforwards of the \emph{same} discrete translation multiset $\{t_{j,a}\}$ under
the corresponding face-slice maps.  Thus the remaining task is a \emph{simultaneous} matching problem.


\end{remark}

\begin{lemma}[Automatic $W_1$-matching from smooth dependence of face maps]\label{lem:w1-auto}
\REVMZ{(Wasserstein distance; see \cite{Villani03}.)}
Let $\mu$ be a finite Borel measure on $\R^{2p}$ supported in a ball of radius $O(\varrho h)$ and with total mass $\mu(\R^{2p})=N$.
Let $\Phi,\Phi':\R^{2p}\to\R^{2p}$ be linear maps with $\|\Phi-\Phi'\|_{\mathrm{op}}\le C\,h$.
Then
\[
W_1(\Phi_\#\mu,\Phi'_\#\mu)\ \le\ C\,h\int_{\R^{2p}}\|y\|\,d\mu(y)\ \le\ C'\,\varrho\,h^2\,N.
\]
\end{lemma}


\begin{proof}
Define a coupling $\pi$ of $\Phi_\#\mu$ and $\Phi'_\#\mu$ by pushing $\mu$ forward under the map
$y\mapsto (\Phi y,\Phi' y)$.
Then $\pi$ has first marginal $\Phi_\#\mu$ and second marginal $\Phi'_\#\mu$, and therefore
\[
W_1(\Phi_\#\mu,\Phi'_\#\mu)
\le
\int_{\R^{2p}\times\R^{2p}} \|u-u'\|\,d\pi(u,u')
\;=\;
\int_{\R^{2p}} \|\Phi y-\Phi' y\|\,d\mu(y).
\]
Estimating $\|\Phi y-\Phi' y\|\le \|\Phi-\Phi'\|_{\mathrm{op}}\|y\|$ gives
\[
W_1(\Phi_\#\mu,\Phi'_\#\mu)
\le \|\Phi-\Phi'\|_{\mathrm{op}}\int_{\R^{2p}}\|y\|\,d\mu(y).
\]
If $\operatorname{supp}\mu\subset B(0,C_0 \varrho h)$, then $\int\|y\|\,d\mu\le C_0 \varrho h\,\mu(\R^{2p})=C_0 \varrho h\,N$.
Absorbing constants yields the stated bound.
\end{proof}



\begin{definition}[Cell transport plan]\label{def:cell-transport-plan}
Fix an interior face $F=Q\cap Q'$ and a direction label $j$. Suppose the $j$--sheets induce two finite atomic measures on $F$,
\[
\mu_{F,j}=\sum_{\alpha} a_\alpha\,\delta_{x_\alpha},
\qquad
\nu_{F,j}=\sum_{\beta} b_\beta\,\delta_{y_\beta},
\]
with integer weights $a_\alpha,b_\beta\in\mathbb N$. A \emph{cell transport plan} is a coupling matrix $\pi=(\pi_{\alpha\beta})_{\alpha,\beta}$ with $\pi_{\alpha\beta}\in\mathbb N$ and prescribed marginals $\sum_\beta \pi_{\alpha\beta}=a_\alpha$, $\sum_\alpha \pi_{\alpha\beta}=b_\beta$. Its (Euclidean) cost is 
\[
\mathrm{Cost}(\pi):=\sum_{\alpha,\beta}\pi_{\alpha\beta}\,|x_\alpha-y_\beta|.
\]

\end{definition}

\begin{lemma}[Integral optimal couplings for discrete integer measures]\label{lem:w1-integral-optimal}\label{lem:transport-exists}
Let $\mu=\sum_{a=1}^A m_a\,\delta_{x_a}$ and $\nu=\sum_{b=1}^B n_b\,\delta_{y_b}$ be finite Borel measures on $\R^d$
with $m_a,n_b\in\mathbb{N}$ and $\mu(\R^d)=\nu(\R^d)=:N$.
Then
\[
W_1(\mu,\nu)
=\min\Bigl\{\;\sum_{a=1}^A\sum_{b=1}^B k_{ab}\,\|x_a-y_b\|\;:\;
k_{ab}\in\mathbb{N},\ \sum_{b=1}^B k_{ab}=m_a,\ \sum_{a=1}^A k_{ab}=n_b\;\Bigr\}.
\]
In particular, the infimum in the Kantorovich formulation of $W_1$ is attained by a coupling with \emph{integer} coefficients
$k_{ab}$ (equivalently, an integral min-cost flow).
\end{lemma}


\begin{proof}
\noindent\textbf{}
By the Kantorovich formulation of $W_1$ for atomic measures (see, e.g., \cite{Villani03}),
any coupling $\pi\in\Pi(\mu,\nu)$ corresponds to a nonnegative matrix $(k_{ab})$ with the stated row/column sums, and the
transport cost is exactly $\sum_{a,b} k_{ab}\|x_a-y_b\|$.
This is the classical transportation linear program.  Since the constraint matrix is totally unimodular, whenever the supplies
and demands are integers there exists an \emph{optimal} solution with integer entries $k_{ab}$; see, e.g., \cite{Schrijver86}.
\end{proof}





\begin{lemma}[Pointwise displacement bound under nearby face maps]\label{lem:face-displacement}
Let $y_1,\dots,y_N\in\R^{2p}$ satisfy $\|y_a\|\le C_0\,\varrho\,h$ and let $\Phi,\Phi':\R^{2p}\to\R^{2p}$ be linear maps with
$\|\Phi-\Phi'\|_{\mathrm{op}}\le C_1\,h$.
Define two multisets $u_a:=\Phi y_a$ and $u'_a:=\Phi' y_a$.
Then the index-wise matching satisfies
\[
\|u_a-u'_a\|\ \le\ C_0C_1\,\varrho\,h^2\qquad\text{for all }a.
\]
In particular, when adjacent cells use the \emph{same} translation template $\{y_a\}$ and their face parameterizations differ by $O(h)$ in operator norm,
the hypothesis of Corollary holds with $\Delta_F=O(\varrho\,h^2)$.
\end{lemma}

\begin{proof}
\[
\|u_a-u'_a\|
=\|(\Phi-\Phi')y_a\|
\le \|\Phi-\Phi'\|_{\mathrm{op}}\|y_a\|
\le (C_1h)(C_0\varrho h)=C_0C_1\varrho h^2.
\qedhere
\]
\end{proof}


\begin{lemma}[Template stability under small multiset edits]\label{lem:w1-template-edit}
Let $\Omega\subset\R^{2p}$ be a bounded domain of diameter $\mathrm{diam}(\Omega)\le C\,\varrho h$.
Let $\mu=\sum_{a=1}^{N}\delta_{y_a}$ and $\mu'=\sum_{b=1}^{N}\delta_{y'_b}$ be two integer-weighted discrete measures on $\Omega$
with the \emph{same total mass} $N$.
Assume there is a matching of atoms such that $\|y_a-y'_a\|\le \Delta$ for all $a$ (after relabeling).
Then
\[
W_1(\mu,\mu')\ \le\ \Delta\,N.
\]
More generally, if $\mu'$ is obtained from $\mu$ by deleting $r$ atoms and inserting $r$ atoms (so total mass stays $N$), then
\[
W_1(\mu,\mu')\ \le\ r\cdot \mathrm{diam}(\Omega)\ \le\ C\,r\,\varrho\,h.
\]
\end{lemma}

\begin{proof}
For the first claim, couple $\mu$ and $\mu'$ by pairing each $y_a$ to $y'_a$; the transport cost is $\sum_a\|y_a-y'_a\|\le \Delta N$.
For the second claim, transport each deleted atom to an inserted atom at cost at most $\mathrm{diam}(\Omega)$ and keep the unchanged atoms fixed.
\end{proof}

\begin{remark}[How Lemma~\ref{lem:w1-auto} reduces the remaining matching task]\label{rem:w1-auto}
\REVMZ{\textbf{[Technical clarification.]}}
If, for each cube $Q$ and sheet family $(Q,j)$, we choose the translation multiset $\{t_{j,a}\}$ by a \emph{fixed} template in
$N_{Q,j}^\perp$ (e.g.\ a scaled lattice/low-discrepancy set of diameter $O(h)$), then across a shared face $F=Q\cap Q'$ the two
induced transverse measures are related by applying two nearby face-slice maps (coming from nearby plane directions and nearby normal-coordinate identifications).
Since $\beta$ is smooth, these maps differ by $O(h)$ in operator norm, so Lemma~\ref{lem:w1-auto} yields
\[
W_1(\mu_{Q\to F},\mu_{Q'\to F})\ \lesssim\ \varrho\,h^2\,N_F,
\]
where $N_F$ is the number of sheets contributing to that face.
Inserting this into Proposition~\ref{prop:transport-flat-glue} yields a global bound of the form
\[
\mathcal F(\partial T^{\mathrm{raw}})\ \lesssim\ m\,\varrho\,h \;+\; O(\varepsilon\,m),
\]

so choosing a refinement schedule $h=h_j\downarrow 0$ and $\varepsilon=\varepsilon_j\downarrow 0$ forces the right-hand side to $0$ for fixed $m$, hence the gluing correction $U_{h_j}$ becomes negligible in the mass equality.

The remaining task is then to implement this ``fixed template'' choice while still meeting the cohomological constraints (Substep 4.3).
\smallskip
In the \emph{sliver} regime, the count $N_F$ is not controlled by total mass; see Remark and
Corollary for the weighted replacement.
\end{remark}


\begin{remark}[Sliver regime: what changes in the global counting estimate]\label{rem:sliver-vs-template}
\REVMZ{\textbf{[Technical clarification.]}}
The global $m\,h$ bound in Remark~\ref{rem:w1-auto} uses an implicit \emph{counting step}: it treats the total face mismatch as scaling like
``(per-sheet mismatch) $\times$ (number of sheet pieces meeting faces)''.  In the constant-mass-per-sheet model this count is controlled by total mass,
because each sheet piece carries $\psi$--mass $\asymp h^{2(n-p)}$ in a cube.

\smallskip\noindent
In the \emph{sliver} regime (Remark), one deliberately allows many pieces of very small mass per cube.
Then the raw counts $N_F$ (or the total number of sheet pieces meeting faces) can grow without bound at fixed total mass, so the crude reduction
to $\Mass(T^{\mathrm{raw}})$ is no longer available.
To make the sliver escape compatible with flat-norm gluing, we therefore use a \emph{weighted} replacement that tracks the actual size of each face slice,
for example a bound in terms of the boundary-size functional
\[
\sum_{F}\sum_{a\in\mathcal S(F)} \Mass\!\big(\partial([Y^a]\llcorner Q)\llcorner F\big),
\]
or an equivalent transverse-parameter integral.  Concretely, Proposition bounds each face flat mismatch by
displacement $\times$ (slice boundary mass), and Lemma converts slice boundary mass into a power of the
interior piece mass on smooth curvature-pinched cells.  This is packaged globally as Corollary.
\end{remark}


\begin{definition}[Sliver cell]\label{def:sliver-cell}
Fix a cell decomposition at mesh $h$ (by cubes or smooth uniformly convex cells).
A cell $Q$ is said to be in the \emph{sliver regime} for a given direction family $j$ if it contains a collection of
$\psi$--calibrated pieces $(Y^{Q,a})_{a=1}^{N_Q}$ with the following properties:
\begin{enumerate}
\item[\textnormal{(a)}] (\textbf{Many pieces}) $N_Q\ge c\,h^{-1}$ whenever the target budget $M_Q$ is not negligible;
\item[\textnormal{(b)}] (\textbf{Graph control}) each $Y^{Q,a}\cap Q$ is a $C^1$ graph over the same affine calibrated plane
in $Q$ with slope $\le \varepsilon_{\mathrm{hol}}$ on the footprint scale (as provided by Theorem~\ref{thm:local-sheets});
\item[\textnormal{(c)}] (\textbf{Template parameters}) the transverse parameters of the pieces are taken from a fixed ordered master
template $(y_a)_{a\ge 1}$ (prefixes in each cell).
\end{enumerate}
\end{definition}



\begin{proposition}[Quantizing a cell mass budget by many equal-scale pieces]\label{prop:sliver-cell-mass-proportions}
Assume the setting of Theorem~\ref{thm:local-sheets} in a cell $Q$ for a fixed direction family $j$ and footprint scale $s$,
so that each local piece $S^{(a)}_j$ has mass
\[
\Mass(S^{(a)}_j)=A_j(s)+O(\varepsilon_{\mathrm{hol}})\,s^{k},\qquad k:=2n-2p,
\]
with $A_j(s)\asymp s^{k}$ depending only on $(X,\omega,n,p)$ and the fixed footprint template.
Given any target budget $M_Q\ge 0$, choose the integer
\[
N_Q:=\left\lfloor \frac{M_Q}{A_j(s)}\right\rceil \in \Z_{\ge 0},
\]
and select $N_Q$ pieces from the master template (disjointness not required).
Then the total mass in $Q$ satisfies
\[
\Bigl|\sum_{a=1}^{N_Q}\Mass(S^{(a)}_j)-M_Q\Bigr|\ \le\ A_j(s)\ +\ O(\varepsilon_{\mathrm{hol}})\,N_Q\,s^{k}.
\]
In particular, if $N_Q\to\infty$ as $h\to 0$ and $\varepsilon_{\mathrm{hol}}\to 0$, then
$\sum_{a\le N_Q}\Mass(S^{(a)}_j)=M_Q+o(M_Q)$ uniformly over $Q$ with $M_Q\gg s^{k}$.
\end{proposition}

\begin{proof}
Write $\Mass(S^{(a)}_j)=A_j(s)+e_a$ with $|e_a|\le C\,\varepsilon_{\mathrm{hol}}\,s^{k}$.
Then $\sum_{a\le N_Q}\Mass(S^{(a)}_j)=N_QA_j(s)+\sum_{a\le N_Q}e_a$ and
$|\sum_{a\le N_Q}e_a|\le C\,\varepsilon_{\mathrm{hol}}\,N_Q\,s^{k}$.
By the definition of rounding, $|N_QA_j(s)-M_Q|\le A_j(s)$.
Combining yields the displayed bound.
\end{proof}





%=========================================================
% Block B tools (moved here to avoid forward references)
%=========================================================

\begin{lemma}[Flat-norm stability under translation]\label{lem:flat-translate}
Let $S$ be an integral $\ell$--current in $\R^d$ with finite mass and finite boundary mass.
For any translation vector $v\in\R^d$, write $\tau_v(x):=x+v$ and $(\tau_v)_\#S$ for the pushforward.
Then
\[
\mathcal F\!\bigl((\tau_v)_\#S-S\bigr)\ \le\ \|v\|\Bigl(\Mass(S)+\Mass(\partial S)\Bigr).
\]
In particular, if $S$ is a cycle ($\partial S=0$) this reduces to
$\mathcal F((\tau_v)_\#S-S)\le \|v\|\,\Mass(S)$.
\end{lemma}

\begin{proof}
We use the standard cone (straight-line homotopy) construction; see \cite{Fed69}.

Let $H:[0,1]\times\R^d\to\R^d$ be the straight-line homotopy $H(t,x)=x+t v$.
Consider the product current $[0,1]\times S$ in $[0,1]\times\R^d$ and set
\(
Q:=H_\#([0,1]\times S).
\)
Set also
\(
R:=H_\#([0,1]\times \partial S).
\)
Since $\partial([0,1]\times S)=\{1\}\times S-\{0\}\times S-[0,1]\times \partial S$, we have
\[
\partial Q
=H_\#(\{1\}\times S)-H_\#(\{0\}\times S)-H_\#([0,1]\times \partial S)
=(\tau_v)_\#S-S-R.
\]
Thus $(\tau_v)_\#S-S=R+\partial Q$.
Moreover, $H$ has Jacobian bounded by $\|v\|$ in the $t$-direction, so the mass estimate for pushforwards gives
\(
\Mass(Q)\le \|v\|\,\Mass(S).
\)
Likewise $\Mass(R)\le \|v\|\,\Mass(\partial S)$.
Taking these $R,Q$ in the definition of $\mathcal F$ yields
\[
\mathcal F((\tau_v)_\#S-S)\le \Mass(R)+\Mass(Q)\le \|v\|\Bigl(\Mass(S)+\Mass(\partial S)\Bigr),
\]
as claimed.
\end{proof}

\begin{lemma}[Flat-norm stability under small $C^0$ deformations]\label{lem:flat-C0-deform}
Let $S$ be an integral $\ell$--current in $\R^d$ with finite mass and finite boundary mass.
Let $\phi_0,\phi_1:\R^d\to\R^d$ be Lipschitz maps with
\[
\sup_{x\in\spt S}\|\phi_1(x)-\phi_0(x)\|\ \le\ \delta,
\qquad
\Lip(\phi_0)+\Lip(\phi_1)\ \le\ L.
\]
Then there exists a constant $C_\ell$ depending only on $\ell$ such that
\[
\mathcal F\!\bigl(\phi_{1\#}S-\phi_{0\#}S\bigr)\ \le\ C_\ell\,\delta\,L^{\ell}\Bigl(\Mass(S)+\Mass(\partial S)\Bigr).
\]
\end{lemma}

\begin{proof}
We use the standard cone (straight-line homotopy) construction; see \cite{Fed69}.

Consider the straight-line homotopy $H:[0,1]\times\R^d\to\R^d$ given by
$H(t,x):=(1-t)\phi_0(x)+t\,\phi_1(x)$.
Set $Q:=H_\#([0,1]\times S)$ and $R:=H_\#([0,1]\times \partial S)$.
Since $\partial([0,1]\times S)=\{1\}\times S-\{0\}\times S-[0,1]\times\partial S$, the homotopy formula gives
\[
\phi_{1\#}S-\phi_{0\#}S\ =\ R+\partial Q.
\]
On $\spt S$, the differential of $H$ has one ``$t$--direction'' column $\,\partial_t H=\phi_1-\phi_0\,$ whose norm is $\le\delta$,
and $\ell$ ``spatial'' columns bounded by $L$.
Therefore the $(\ell+1)$--Jacobian of $H$ is bounded by $C_\ell\,\delta\,L^\ell$ on $\spt([0,1]\times S)$, and the $\ell$--Jacobian
of $H$ restricted to $\spt([0,1]\times\partial S)$ is bounded by $C_\ell\,\delta\,L^{\ell-1}$.
The standard mass estimate for pushforwards yields
\[
\Mass(Q)\ \le\ C_\ell\,\delta\,L^\ell\,\Mass(S),
\qquad
\Mass(R)\ \le\ C_\ell\,\delta\,L^\ell\,\Mass(\partial S),
\]
(after enlarging $C_\ell$ to absorb the $L^{\ell-1}$ factor).
Taking these $R,Q$ in the definition of $\mathcal F$ gives the claim.
\end{proof}

\begin{lemma}[Flat norm of a cycle supported in diameter $\lesssim h$]\label{lem:flat-diameter}
Let $S$ be an integral $\ell$-cycle in $\R^d$ with finite mass.
Assume $\mathrm{diam}(\mathrm{spt}\,S)\le D$.
Then
\[
\mathcal F(S)\ \le\ C(\ell)\,D\,\Mass(S).
\]
In particular, if $\mathrm{diam}(\mathrm{spt}\,S)\lesssim h$ then $\mathcal F(S)\lesssim h\,\Mass(S)$.
\end{lemma}

\begin{proof}
We use the standard cone (straight-line homotopy) construction; see \cite{Fed69}.

Fix $x_0$ in the convex hull of $\mathrm{spt}\,S$, so that $\|x-x_0\|\le D$ for all $x\in \mathrm{spt}\,S$.
Consider the straight-line homotopy $H:[0,1]\times\R^d\to\R^d$ given by
\(
H(t,x)=(1-t)x+t x_0.
\)
Let $Q:=H_\#([0,1]\times S)$.
Since $S$ is a cycle, $\partial([0,1]\times S)=\{1\}\times S-\{0\}\times S$, and therefore
\[
\partial Q
=H_\#(\{1\}\times S)-H_\#(\{0\}\times S)
=0-S
=-S,
\]
because $H(1,\cdot)\equiv x_0$ is constant and pushes any positive-dimensional current to $0$.
Thus $\partial(-Q)=S$, so taking $R=0$ in the definition of $\mathcal F$ gives $\mathcal F(S)\le \Mass(Q)$.

Finally, the cone/Jacobian estimate for $H$ yields $\Mass(Q)\le C(\ell)\,D\,\Mass(S)$ for a constant $C(\ell)$ depending only on $\ell$.
Combining gives the claim.
\end{proof}

\begin{lemma}[Template displacement $\Rightarrow$ per-face flat-norm mismatch]\label{lem:template-displacement}
Work in the setting of Proposition~\ref{prop:transport-flat-glue}\textnormal{(a)}--\textnormal{(b)} on an interior interface $F=Q\cap Q'$ at mesh $h$.
In the global-coherence regime (\REVMZ{Proposition~\ref{prop:global-coherence-all-labels}}), the boundary slices on $F$ are parameterized by the \emph{same} integer-weighted discrete measure
$\nu=\sum_{a=1}^{N_F} w_a\,\delta_{y_a}$ supported in a ball of radius $C_0\,\varrho h\subset\R^{2p}$ via linear face maps
$\mu_{Q\to F}=(\Phi_{Q,F})_\#\nu$ and $\mu_{Q'\to F}=(\Phi_{Q',F})_\#\nu$.
Assume $\|\Phi_{Q,F}\|_{\mathrm{op}}+\|\Phi_{Q',F}\|_{\mathrm{op}}\le C_{\Phi,0}$ and $\|\Phi_{Q,F}-\Phi_{Q',F}\|_{\mathrm{op}}\le C_\Phi h$.
Then, after pairing atoms by the identity pairing $y_a\leftrightarrow y_a$, the mismatch current $B_F$ satisfies
\begin{align*}
\mathcal F(B_F)
&\le C\,\varrho\,h^2\Biggl[
\sum_{a=1}^{N_F} w_a\Bigl(\Mass(\Sigma_{\Phi_{Q,F}y_a})+\Mass(\partial\Sigma_{\Phi_{Q,F}y_a})\Bigr) \\
&\qquad\qquad\quad
+\sum_{a=1}^{N_F} w_a\Bigl(\Mass(\Sigma_{\Phi_{Q',F}y_a})+\Mass(\partial\Sigma_{\Phi_{Q',F}y_a})\Bigr)
\Biggr]
+ C_{\angle}\,\varepsilon\,M_F.
\end{align*}
where $M_F$ denotes the total $(2n-2p)$-mass of pieces meeting the interface (so $M_F\lesssim M_Q+M_{Q'}$) and
$\varepsilon$ is the small-angle/graph parameter from Proposition~\ref{prop:transport-flat-glue}\textnormal{(a)}.
\end{lemma}

\begin{proof}
Write $\nu=\sum_{a=1}^{N_F} w_a\,\delta_{y_a}$.
In the flat/parallel model ($\varepsilon=0$), the slice current on $F$ associated to a parameter $z\in\R^{2p}$ is a translate of a fixed model slice:
$\Sigma_z=(\tau_z)_\#\Sigma_0$ in the face chart.
Thus
\[
(\partial S_Q)\llcorner F=\sum_{a=1}^{N_F} w_a\,\Sigma_{\Phi_{Q,F}y_a},
\qquad
(\partial S_{Q'})\llcorner F=\sum_{a=1}^{N_F} w_a\,\Sigma_{\Phi_{Q',F}y_a},
\]
and hence
\[
B_F=\sum_{a=1}^{N_F} w_a\bigl(\Sigma_{\Phi_{Q,F}y_a}-\Sigma_{\Phi_{Q',F}y_a}\bigr).
\]
For each atom $y_a$ define the translation vector $v_a:=(\Phi_{Q,F}-\Phi_{Q',F})y_a$.
Since $\|y_a\|\le C_0\varrho h$ and $\|\Phi_{Q,F}-\Phi_{Q',F}\|_{\mathrm{op}}\le C_\Phi h$, we have $\|v_a\|\le C\,\varrho\,h^2$.
Lemma~\ref{lem:flat-translate} then gives

\[
\begin{aligned}
	\mathcal F\!\bigl(\Sigma_{\Phi_{Q,F}y_a}-\Sigma_{\Phi_{Q',F}y_a}\bigr)
	&\le \|v_a\|\Bigl(\Mass(\Sigma_{\Phi_{Q,F}y_a})+\Mass(\partial\Sigma_{\Phi_{Q,F}y_a})\Bigr) \\
	&\le C\,\varrho\,h^2\Bigl(\Mass(\Sigma_{\Phi_{Q,F}y_a})+\Mass(\partial\Sigma_{\Phi_{Q,F}y_a})\Bigr).
\end{aligned}
\]

By subadditivity of $\mathcal F$ and summing over $a$ (with weights $w_a$),
\[
\mathcal F(B_F)\le C\,\varrho\,h^2\sum_{a=1}^{N_F} w_a\,\Bigl(\Mass(\Sigma_{\Phi_{Q,F}y_a})+\Mass(\partial\Sigma_{\Phi_{Q,F}y_a})\Bigr).
\]
The same bound holds with $Q$ and $Q'$ swapped; combining yields the symmetric form stated.

For $\varepsilon>0$, write each actual boundary slice on $F$ as a pushforward of its flat/parallel model slice
by a Lipschitz graph map in the tubular chart.
Hypothesis \textnormal{(a)} gives a uniform displacement bound $\delta\lesssim\varepsilon h$ and a uniform Lipschitz bound.
Applying Lemma~\ref{lem:flat-C0-deform} to each slice yields a flat-norm error of size
\[
\mathcal F\!\bigl(\Sigma^{\mathrm{act}}_y-\Sigma^{\mathrm{flat}}_y\bigr)\ \le\ C\,\varepsilon\,h\Bigl(\Mass(\Sigma^{\mathrm{flat}}_y)+\Mass(\partial\Sigma^{\mathrm{flat}}_y)\Bigr).
\]
Summing over the integer-weighted family of slices meeting $F$ and using that the total $(2n-2p)$--mass of pieces meeting $F$
controls the sum of slice masses at scale $h$ gives the additional term
$C_{\angle}\,\varepsilon\,M_F$ in the statement.
\end{proof}

\begin{lemma}[Template displacement with insertions/deletions]\label{lem:template-displacement-edits}
Work in the setting of Lemma~\ref{lem:template-displacement} on an interior interface $F=Q\cap Q'$ at mesh $h$.
Assume the two sides admit template representations
\[
\mu_{Q\to F}=(\Phi_{Q,F})_\#\nu,
\qquad
\mu_{Q'\to F}=(\Phi_{Q',F})_\#\nu',
\]
where $\nu$ and $\nu'$ are integer-weighted discrete measures supported in $B_{C_0\varrho h}(0)\subset\R^{2p}$ {and the face maps satisfy (by Lemma~\ref{lem:face-map-stability})}
$\|\Phi_{Q,F}\|_{\mathrm{op}}+\|\Phi_{Q',F}\|_{\mathrm{op}}\le C_{\Phi,0}$ and $\|\Phi_{Q,F}-\Phi_{Q',F}\|_{\mathrm{op}}\le C_\Phi h$.
Write $\nu=\nu^{\wedge}+\nu^{+}$ and $\nu'=\nu^{\wedge}+\nu^{-}$, where $\nu^{\wedge}$ is any common submeasure (matched part) and
$\nu^{\pm}$ are the unmatched remainders (insertions/deletions).
Let $B_F^{\wedge}$ be the mismatch current coming from the matched part $\nu^{\wedge}$ and let $B_F^{\mathrm{un}}$ be the mismatch current
coming from the unmatched part (so $B_F=B_F^{\wedge}+B_F^{\mathrm{un}}$).
Then
\[
\mathcal F(B_F^{\wedge})\ \le\ C\,\varrho\,h^2\Bigl(\Mass(\partial S_Q\llcorner F)+\Mass(\partial S_{Q'}\llcorner F)\Bigr)\ +\ C_{\angle}\,\varepsilon\,M_F,
\]
and, moreover,
\[
\mathcal F(B_F^{\mathrm{un}})\ \le\ C\,h\,\Mass(B_F^{\mathrm{un}})\ \le\ C\,h\Bigl(\Mass(\partial S_Q\llcorner F)+\Mass(\partial S_{Q'}\llcorner F)\Bigr),
\]
where $C$ depends only on $(n,p,X)$ and the uniform tubular-face charts.
\end{lemma}

\begin{proof}
The matched part $B_F^{\wedge}$ is obtained by applying the two face maps to the \emph{same} common submeasure $\nu^{\wedge}$.
Therefore Lemma~\ref{lem:template-displacement} applies directly and yields the stated bound for $B_F^{\wedge}$.

For the unmatched part, $B_F^{\mathrm{un}}$ is an integral $(k-1)$--cycle supported on the face patch $F$.
Since $\mathrm{diam}(F)\lesssim h$, Lemma~\ref{lem:flat-diameter} gives
\[
\mathcal F(B_F^{\mathrm{un}})\ \le\ C\,h\,\Mass(B_F^{\mathrm{un}}).
\]
Finally, $\Mass(B_F^{\mathrm{un}})$ is bounded by the total face boundary mass coming from the unpaired sheets, hence by
\(
\Mass(\partial S_Q\llcorner F)+\Mass(\partial S_{Q'}\llcorner F).
\)
Combining these yields the claimed inequalities.
\end{proof}

\begin{lemma}[If edits are an $O(h)$ fraction, they are $h^2$ in flat norm]\label{lem:template-edits-oh}
In the setting of Lemma~\ref{lem:template-displacement-edits}, assume moreover that the unmatched part satisfies
\[
\Mass(B_F^{\mathrm{un}})\ \le\ \theta_F\Bigl(\Mass(\partial S_Q\llcorner F)+\Mass(\partial S_{Q'}\llcorner F)\Bigr)
\]
for some $\theta_F\in[0,1]$.
Then
\[
\begin{aligned}
	\mathcal F(B_F)
	&\le C\,h^2\Bigl(\Mass(\partial S_Q\llcorner F)+\Mass(\partial S_{Q'}\llcorner F)\Bigr) \\
	&\quad + C\,h\,\theta_F\Bigl(\Mass(\partial S_Q\llcorner F)+\Mass(\partial S_{Q'}\llcorner F)\Bigr) \\
	&\quad + C_{\angle}\,\varepsilon\,M_F.
\end{aligned}
\]

In particular, if $\theta_F\lesssim h$ then the unmatched contribution is of the same $h^2\times(\text{boundary mass})$ order as the matched displacement term.
\end{lemma}


\begin{proof}
Decompose $B_F=B_F^{\wedge}+B_F^{\mathrm{un}}$ as in Lemma~\ref{lem:template-displacement-edits}.
Lemma~\ref{lem:template-displacement-edits} gives the $h^2$--scale bound for $\mathcal F(B_F^{\wedge})$ (plus the $C_{\angle}\,\varepsilon\,M_F$ term), and also gives
\(
\mathcal F(B_F^{\mathrm{un}})\le C h\,\Mass(B_F^{\mathrm{un}}).
\)
Using the hypothesis $\Mass(B_F^{\mathrm{un}})\le \theta_F(\Mass(\partial S_Q\llcorner F)+\Mass(\partial S_{Q'}\llcorner F))$ and subadditivity of $\mathcal F$
yields the stated inequality for $\mathcal F(B_F)$.
\end{proof}

\begin{proposition}[Prefix templates $\Rightarrow$ interface coherence up to $O(h)$ edits]\label{prop:prefix-template-coherence}
Work in the setting of Lemma~\ref{lem:template-displacement-edits} on an interior interface $F=Q\cap Q'$ at mesh $h$.
Fix an ordered template of transverse atoms $(y_a)_{a\ge 1}\subset B_{C_0\varrho h}(0)\subset\R^{2p}$ and define prefixes
\[
\nu^{(N)}\ :=\ \sum_{a=1}^{N}\delta_{y_a}.
\]
Assume the two sides arise from prefixes:
\[
\mu_{Q\to F}=(\Phi_{Q,F})_\#\nu^{(N_Q)},\qquad
\mu_{Q'\to F}=(\Phi_{Q',F})_\#\nu^{(N_{Q'})},
\]
and write $B_F$ for the resulting mismatch current on $F$.
If the unmatched part satisfies the $O(h)$-fraction hypothesis
\[
\Mass(B_F^{\mathrm{un}})\ \le\ \theta_F\Bigl(\Mass(\partial S_Q\llcorner F)+\Mass(\partial S_{Q'}\llcorner F)\Bigr)
\qquad\text{with}\qquad \theta_F\lesssim h,
\]
then
\[
\mathcal F(B_F)\ \le\ C\,h^2\Bigl(\Mass(\partial S_Q\llcorner F)+\Mass(\partial S_{Q'}\llcorner F)\Bigr)\ +\ C_{\angle}\,\varepsilon\,M_F,
\]
with $C$ depending only on $(n,p,X)$ and the uniform tubular-face charts.
\end{proposition}


\begin{proof}
Let $N_{\min}:=\min\{N_Q,N_{Q'}\}$ and decompose the two prefixes into a common matched prefix plus tails:
\[
\nu^{(N_Q)}=\nu^{(N_{\min})}+\nu^{+},
\qquad
\nu^{(N_{Q'})}=\nu^{(N_{\min})}+\nu^{-}.
\]
This is exactly the decomposition in Lemma~\ref{lem:template-displacement-edits} with $\nu^{\wedge}=\nu^{(N_{\min})}$.
Applying Lemma~\ref{lem:template-displacement-edits} controls the matched displacement contribution and bounds the unmatched part by the diameter estimate.
Then Lemma~\ref{lem:template-edits-oh} (using $\theta_F\lesssim h$) upgrades the unmatched contribution to the same $h^2$ scale.
\end{proof}


\begin{remark}[Why ordered prefixes?]\label{rem:why-prefix}
Across an interface $F=Q\cap Q'$, using \emph{prefixes} of a single ordered master template $(y_a)_{a\ge 1}$
forces the two induced transverse counting measures to have the form
$\nu^{(N_Q)}=\sum_{a\le N_Q}\delta_{y_a}$ and $\nu^{(N_{Q'})}=\sum_{a\le N_{Q'}}\delta_{y_a}$.
Consequently their symmetric difference is \emph{combinatorially explicit}: it is supported only on indices
$N_{\min}<a\le N_{\max}$, so any mismatch current decomposes into a matched part (paired indices $a\le N_{\min}$)
plus an \emph{unmatched tail} (indices $N_{\min}<a\le N_{\max}$).
This is the mechanism that reduces interface bookkeeping to (i) a transport term on matched indices and
(ii) an $O(h)$--fraction hypothesis on the unmatched tail in Proposition~\ref{prop:prefix-template-coherence}.
\end{remark}



\begin{proposition}[Prefix mismatch decomposition]\label{prop:sliver-template-extension}
Work in the setting of Proposition~\ref{prop:prefix-template-coherence} on a face $F=Q\cap Q'$.
Assume the two sides arise from prefixes of a common ordered template $(y_a)_{a\ge 1}$, i.e.
\[
\mu_{Q\to F}=(\Phi_{Q,F})_\#\nu^{(N_Q)},\qquad
\mu_{Q'\to F}=(\Phi_{Q',F})_\#\nu^{(N_{Q'})},
\qquad \nu^{(N)}:=\sum_{a\le N}\delta_{y_a}.
\]
Let $N_{\min}:=\min\{N_Q,N_{Q'}\}$ and $N_{\max}:=\max\{N_Q,N_{Q'}\}$.
Then the face currents admit a decomposition
\[
S_{Q\to F}=S_{Q\to F}^{\mathrm{ma}}+S_{Q\to F}^{\mathrm{un}},\qquad
S_{Q'\to F}=S_{Q'\to F}^{\mathrm{ma}}+S_{Q'\to F}^{\mathrm{un}},
\]
where the matched parts are indexed by $a\le N_{\min}$ and the unmatched parts are indexed by
$N_{\min}<a\le N_{\max}$ (pulled back by the same face chart).
In particular, the mismatch current $B_F:=S_{Q\to F}-S_{Q'\to F}$ splits as
$B_F=B_F^{\mathrm{ma}}+B_F^{\mathrm{un}}$ with $B_F^{\mathrm{un}}$ supported on the unmatched tail indices
$N_{\min}<a\le N_{\max}$.
\end{proposition}

\begin{proof}
This is immediate from the prefix representations: write each side as a sum over $a\le N_Q$ (resp.\ $a\le N_{Q'}$)
and split the index sets into $\{a\le N_{\min}\}$ and $\{N_{\min}<a\le N_{\max}\}$.
\end{proof}


\begin{remark}[On the ``extension'' step]\label{rem:sliver-template-extension}
Proposition~\ref{prop:sliver-template-extension} is purely combinatorial: it identifies the unmatched tail as
the sole source of \emph{unpaired} indices when adjacent cells have different prefix lengths.
To apply Proposition~\ref{prop:prefix-template-coherence} one still needs a quantitative bound of the form
$\Mass(B_F^{\mathrm{un}})\le \theta_F(\Mass(\partial S_Q\llcorner F)+\Mass(\partial S_{Q'}\llcorner F))$ with $\theta_F\lesssim h$.
In this manuscript we treat that $O(h)$--fraction bound as an explicit hypothesis (see the ``$O(h)$ edit regime'' assumption
in Theorem~\ref{thm:sliver-mass-matching-on-template}), because it can fail for sharp cubical cells or poorly shaped slices
(cf.\ Remark~\ref{rem:sliver-cell-shape}).
\end{remark}



%=========================================================


\begin{proposition}[Weighted transport $\Rightarrow$ flat-norm face control (sliver-compatible)]\label{prop:transport-flat-glue-weighted}
Work in the tubular/flat model on an interior face $F=Q\cap Q'$.
Assume each sheet piece meeting $F$ contributes an integral slice current $\Sigma(u)$ on $F$ depending on a transverse parameter
$u\in\Omega_F\subset\R^{2p}$, and that $\Sigma(u)$ is obtained from $\Sigma(0)$ by translation in the face chart.
Let the two adjacent cubes induce two multisets of parameters $\{u_a\}_{a=1}^N$ and $\{u'_a\}_{a=1}^N$ (same cardinality), hence two face currents
\[
S_{Q\to F}:=\sum_{a=1}^N \Sigma(u_a),\qquad
S_{Q'\to F}:=\sum_{a=1}^N \Sigma(u'_a),
\qquad
B_F:=S_{Q\to F}-S_{Q'\to F}.
\]
Then
\[
\mathcal F(B_F)\ \le\ \inf_{\sigma\in S_N}\ \sum_{a=1}^N \|u_a-u'_{\sigma(a)}\|\Bigl(\Mass(\Sigma(u_a))+\Mass(\partial\Sigma(u_a))\Bigr).
\]
In particular, if $\Mass(\Sigma(u_a))+\Mass(\partial\Sigma(u_a))\le b_F$ for all $a$ and if
\[
\tau_F:=\inf_{\sigma\in S_N}\ \sum_{a=1}^N \|u_a-u'_{\sigma(a)}\|
\]
(the equal-weight matching cost, i.e.\ $W_1$ of the counting measures), then
\[
\mathcal F(B_F)\ \le\ b_F\,\tau_F.
\]
\end{proposition}


\begin{proof}
Fix a permutation $\sigma\in S_N$.
For each index $a$, apply Lemma~\ref{lem:flat-translate} in the face chart to the translated pair
$\Sigma(u_a)$ and $\Sigma(u'_{\sigma(a)})$.
This yields integral currents $R_a$ and $Q_a$ such that
\[
\begin{aligned}
	\Sigma(u_a)-\Sigma\bigl(u'_{\sigma(a)}\bigr) &= R_a+\partial Q_a, \\
	\Mass(R_a)+\Mass(Q_a)
	&\le \|u_a-u'_{\sigma(a)}\|
	\Bigl(\Mass(\Sigma(u_a))+\Mass(\partial\Sigma(u_a))\Bigr).
\end{aligned}
\]

Summing $R:=\sum_{a=1}^N R_a$ and $Q:=\sum_{a=1}^N Q_a$ gives $B_F=R+\partial Q$ and
\[
\Mass(R)+\Mass(Q)\ \le\ \sum_{a=1}^N \|u_a-u'_{\sigma(a)}\|\Bigl(\Mass(\Sigma(u_a))+\Mass(\partial\Sigma(u_a))\Bigr).
\]
Taking the infimum over $\sigma$ in the definition of $\mathcal F$ proves the claim.
\end{proof}






\begin{proposition}[Integer transverse matching from the master prefix template (constructed here)]
\label{prop:integer-transport}
Let $F=Q\cap Q'$ be an interior $(2n-1)$-face of the cubulation at mesh $h$.
Fix a transverse grid scale $\delta_{\perp}\in(0,h)$ and an ordered master template
$\mathbf y=(y_a)_{a\ge1}\subset B_{C_0\varrho h}(0)\cap \delta_{\perp}\mathbb Z^{2p}$.
For $N\in\mathbb N$ write $\nu^{(N)}:=\sum_{a=1}^N\delta_{y_a}$ (cf.\ Proposition~\ref{prop:prefix-template-coherence}).

Let $\Phi_{Q,F},\Phi_{Q',F}:B_{C_0\varrho h}(0)\to \Omega_F$ be the face maps from
Lemma~\ref{lem:template-displacement-edits} (in particular
$\|\Phi_{Q,F}-\Phi_{Q',F}\|_{\mathrm{op}}\le C_\Phi h$ and
$\|\Phi_{Q,F}\|_{\mathrm{op}}+\|\Phi_{Q',F}\|_{\mathrm{op}}\le C_{\Phi,0}$).

Define the (balanced) transverse measures on $F$ by choosing an integer $N_F\ge0$
(the common prefix length after the face-balancing/prefix-edit step of
\REVMZ{Proposition~\ref{prop:prefix-template-coherence}}) and setting
\[
\mu_{Q\to F}\ :=\ (\Phi_{Q,F})_\#\nu^{(N_F)},
\qquad
\mu_{Q'\to F}\ :=\ (\Phi_{Q',F})_\#\nu^{(N_F)}.
\]
Then $\mu_{Q\to F}$ and $\mu_{Q'\to F}$ are integer-weighted, supported on the
$\delta_{\perp}$-grid images in $\Omega_F$, and satisfy
\[
\int_{\Omega_F}\mu_{Q\to F}\ =\ \int_{\Omega_F}\mu_{Q'\to F}\ =\ N_F.
\]
Moreover their $W_1$-distance is controlled by the template displacement:
\[
W_1(\mu_{Q\to F},\mu_{Q'\to F})
\ \le\ C_\Phi\,C_0\,\varrho\,h^2\,N_F.
\]
In particular, hypothesis \textnormal{(c)} in Proposition~\ref{prop:transport-flat-glue-weighted}
holds with
\[
\tau_F\ :=\ C_\Phi\,C_0\,\varrho\,h^2\,N_F.
\]
\end{proposition}

\begin{proof}
Let $\pi$ be the coupling obtained by matching the same template atom on both sides:
\[
\pi\ :=\ \sum_{a=1}^{N_F}\delta_{(\Phi_{Q,F}(y_a),\,\Phi_{Q',F}(y_a))}.
\]
Then $\pi$ has marginals $\mu_{Q\to F}$ and $\mu_{Q'\to F}$ by definition, hence
\[
W_1(\mu_{Q\to F},\mu_{Q'\to F})
\ \le\ \int_{\Omega_F\times\Omega_F}|u-v|\,d\pi(u,v)
\ =\ \sum_{a=1}^{N_F}\big|\Phi_{Q,F}(y_a)-\Phi_{Q',F}(y_a)\big|.
\]
Using Lemma~\ref{lem:template-displacement-edits} and $|y_a|\le C_0\varrho h$ we get
\[
\big|\Phi_{Q,F}(y_a)-\Phi_{Q',F}(y_a)\big|
\ \le\ \|\Phi_{Q,F}-\Phi_{Q',F}\|_{\mathrm{op}}\,|y_a|
\ \le\ C_\Phi h\cdot C_0\varrho h\ =\ C_\Phi C_0 \varrho h^2.
\]
Summing over $a=1,\dots,N_F$ yields the claimed bound.
\end{proof}





\begin{remark}[Exact geometric inequality needed for slivers]
Proposition~\ref{prop:transport-flat-glue-weighted} shows that, in the sliver regime, the face mismatch is controlled by a \emph{weighted} matching cost:
displacement $\times$ (slice boundary mass), rather than displacement $\times$ (number of sheets).
Thus the missing geometric input is precisely an estimate of the form
\[
\Mass(\Sigma(u))\ \lesssim\ \Mass([Y]\llcorner Q)^{\frac{k-1}{k}}
\qquad (k:=2n-2p),
\]
uniformly for the relevant family of slices in the chosen cell geometry (balls / rounded cubes).  In the ball model this holds with an explicit sharp constant;
for general smooth uniformly convex cells it is the content of the ``boundary shrinkage for plane slices'' estimate.
\end{remark}



\begin{lemma}[Boundary shrinkage for plane slices in smooth uniformly convex cells]\label{lem:uniformly-convex-slice-boundary}
Let $Q\subset\R^d$ be a bounded $C^2$ \emph{uniformly convex} domain of diameter $\asymp h$.
Assume the principal curvatures of $\partial Q$ satisfy
\[
\frac{c}{h}\ \le\ \kappa_i\ \le\ \frac{C}{h}
\qquad\text{everywhere on }\partial Q,
\]
for fixed constants $0<c\le C$.
Fix $1\le k<d$ and a $k$-plane $P$.
For each translate $P+t$ with nonempty intersection, set
\[
v(t):=\mathcal H^{k}\bigl((P+t)\cap Q\bigr),
\qquad
a(t):=\mathcal H^{k-1}\bigl((P+t)\cap \partial Q\bigr).
\]
Then there exists $C_*=C_*(d,k,c,C)$ such that
\[
a(t)\ \le\ C_*\,\bigl(v(t)\bigr)^{\frac{k-1}{k}}
\qquad\text{for all such }t.
\]
\end{lemma}

\begin{proof}
The estimate is scale-invariant, so rescale so that $h\asymp 1$.
Write $K_t:=(P+t)\cap Q\subset P+t\cong\R^k$, so $v(t)=\mathcal H^k(K_t)$ and $a(t)=\mathcal H^{k-1}(\partial K_t)$.

If $v(t)\ge v_0>0$, then $K_t$ is a convex body contained in a fixed $k$--ball of radius $O(1)$, hence $a(t)\le A_0(d,k)$, and the desired bound follows
after increasing $C_*$.

Assume $v(t)\le v_0$ with $v_0$ small.  The curvature pinching implies an interior/exterior rolling-ball condition with radii
$r_{\mathrm{in}},r_{\mathrm{out}}\asymp 1$ (depending only on $c,C$) at every boundary point of $Q$.
Let $\pi:\R^d\to P^\perp$ be orthogonal projection and set $D:=\pi(Q)\subset P^\perp$.
Choose a nearest point $t_0\in\partial D$ and an outward normal $u\in P^\perp$ to a supporting hyperplane of $D$ at $t_0$, and write $t=t_0-s u$.
Let $x_0\in\partial Q$ be the unique supporting point with outward normal $u$ (uniqueness by uniform convexity), so $\pi(x_0)=t_0$.

Intersect the tangent balls at $x_0$ with the affine plane $P+t$.  Since $u\perp P$, these intersections are $k$-balls of radii
$\rho_{\mathrm{in}}(s)=\sqrt{2r_{\mathrm{in}}s-s^2}$ and $\rho_{\mathrm{out}}(s)=\sqrt{2r_{\mathrm{out}}s-s^2}$, hence
\[
\omega_k\,\rho_{\mathrm{in}}(s)^k\ \le\ v(t)\ \le\ \omega_k\,\rho_{\mathrm{out}}(s)^k,
\qquad
a(t)\ \le\ \omega_{k-1}\,\rho_{\mathrm{out}}(s)^{k-1}.
\]
For $s$ small one has $\rho_{\mathrm{in}}(s)\gtrsim \sqrt{s}$ and $\rho_{\mathrm{out}}(s)\lesssim \sqrt{s}$, so $v(t)\gtrsim s^{k/2}$ and
$a(t)\lesssim s^{(k-1)/2}$, hence $s\lesssim v(t)^{2/k}$ and $a(t)\lesssim v(t)^{(k-1)/k}$.
\end{proof}


\begin{remark}[References for the geometric inputs]
The implication ``principal curvatures pinched at scale $h$ $\Rightarrow$ interior/exterior tangent balls of radius $\asymp h$'' is the classical
\emph{rolling ball} principle in convex geometry \REVMZ{\cite{Schneider14}} (often attributed to Blaschke).
The supporting-hyperplane/unique-support-point facts used above are standard consequences of strict convexity and $C^2$ regularity of $\partial Q$
(see any standard text on convex bodies, e.g.\ Schneider's \emph{Convex Bodies: The Brunn--Minkowski Theory}\REV{ \cite{Schneider14}}).
\end{remark}












\begin{lemma}[{Interface face-slices are cycles with controlled mass}]\label{lem:face-slice-cycle-mass}

Work on an interior interface face $F=Q\cap Q'$ in the flat/tubular chart, and assume the holomorphic sliver pieces
$Y^{Q,a}\cap Q$ satisfy the single-sheet small-slope graph control of Proposition
(and hence Lemma~\ref{lem:sliver-stability}) on a neighborhood of $Q$.
Let $\Sigma_F(u_a)$ denote the (integral) $(k-1)$--current on $F$ contributed by the boundary trace of the $a$--th sheet on $F$
(as in Proposition~\ref{prop:transport-flat-glue-weighted}), where $k:=2n-2p$.

\smallskip\noindent
Then, after the standard edge-trimming/prefix-edit localization of Proposition
(which ensures the face trace is supported away from the $(2n-2)$--skeleton of the mesh):
\begin{enumerate}
\item[\textnormal{(i)}] $\partial\Sigma_F(u_a)=0$ as a current on $F$ (i.e.\ the face slice is a cycle on the interior face).
\item[\textnormal{(ii)}] There exists a constant $C=C(X,n,p)>0$ such that
\[
\Mass(\Sigma_F(u_a))\ \le\ C\,m_{Q,a}^{\frac{k-1}{k}},
\qquad
m_{Q,a}:=\Mass([Y^{Q,a}]\llcorner Q).
\]
\end{enumerate}

\smallskip\noindent
\begin{proof}
For \textnormal{(i)}, Proposition performs the edge-trimming/prefix edits so that
the face trace associated to the $a$--th sheet is supported in the \emph{open} face
\[
F^\circ:=F\setminus \mathcal N_{\rho h}(\mathrm{skel}),
\]
where $\mathrm{skel}$ is the $(2n-2)$--skeleton of the mesh and $\rho>0$ is fixed.
On $F^\circ$ the trace is an integral $(k-1)$--current with no contribution from the boundary of $F^\circ$,
hence its boundary vanishes as a current on $F^\circ$. Since $\Sigma_F(u_a)$ is supported in $F^\circ$,
this is equivalent to $\partial\Sigma_F(u_a)=0$ as a current on the interior interface face.

For \textnormal{(ii)}, by \REVMZ{Proposition~\ref{prop:holomorphic-corner-exit-g1g2}} the sheet piece $Y^{Q,a}\cap Q$ is a single
$C^1$ graph of slope $\le \varepsilon$ over its calibrated template $k$--simplex $E\subset \Pi$ in the flat chart,
and the face slice $\Sigma_F(u_a)$ is (up to translation by $u_a$ in the face chart) the graph image of the corresponding
template facet $\sigma\subset E$ lying in $F$.
Translation does not change mass, so it suffices to estimate the slice at $u=0$.
By the area formula and Lemma~\ref{lem:small-graph-distortion},
\[
\Mass(\Sigma_F(u_a))\ \le\ (1+C\varepsilon^2)\,\mathcal H^{k-1}(\sigma).
\]
Since $E$ is uniformly fat, Lemma~\ref{lem:corner-simplex-face-mass} gives
$\mathcal H^{k-1}(\sigma)\le C_\star\,v_E^{(k-1)/k}$ where $v_E:=\mathcal H^{k}(E)$.
Finally, Lemma~\ref{lem:sliver-stability} compares the graph mass in $Q$ with the template volume:
$v_E\asymp m_{Q,a}=\Mass([Y^{Q,a}]\llcorner Q)$ (with constants depending only on $X,n,p$ and the fixed fatness bounds).
Combining these estimates yields
\[
\Mass(\Sigma_F(u_a))\ \le\ C\,m_{Q,a}^{\frac{k-1}{k}},
\]
as claimed.
\end{proof}


\end{lemma}


\begin{corollary}[Global flat-norm bound from weighted face control (sliver-compatible)]\label{cor:global-flat-weighted}

Assume the hypotheses of Proposition~\ref{prop:transport-flat-glue-weighted} on each interior interface face $F$ between adjacent mesh cells,
and let $T^{\mathrm{raw}}$ be the global raw current obtained by summing the cell-wise template currents.
For each such face $F$, denote by $B_F$ the boundary mismatch current supported near $F$.
If the parameter multisets on the two sides admit a matching $\sigma$ with
$\|u_a-u'_{\sigma(a)}\|_\infty\le \Delta_F$, then
\[
\mathcal F(B_F)\ \le\ \Delta_F \sum_{a\in \mathcal S(F)}\Bigl(\Mass(\Sigma_F(u_a))+\Mass(\partial\Sigma_F(u_a))\Bigr),
\]
where $\mathcal S(F)$ indexes the pieces meeting $F$ and $\Sigma_F(u_a)$ is the associated sliver slice along $F$.
Consequently,
\[
\mathcal F(\partial T^{\mathrm{raw}})\ \le\ \sum_F \mathcal F(B_F)
\ \le\ \sum_F \Delta_F \sum_{a\in \mathcal S(F)}\Bigl(\Mass(\Sigma_F(u_a))+\Mass(\partial\Sigma_F(u_a))\Bigr).
\]

In the vertex-template holomorphic-sliver regime, Lemma~\ref{lem:face-slice-cycle-mass} supplies
$\partial\Sigma_F(u_a)=0$ and $\Mass(\Sigma_F(u_a))\le C\, m_{Q,a}^{\frac{k-1}{k}}$ (with $k:=2n-2p$)
for every slice meeting a cell $Q$.
Assuming in addition the schedule/face parameterization control of Lemma~\ref{lem:face-displacement} (so that $\Delta_F=O(\varrho\,h^2)$ on all interior faces),
we obtain
\[
\mathcal F(\partial T^{\mathrm{raw}})\ \le\ C\,\varrho\, h^2 \sum_Q \sum_{a\in \mathcal S(Q)} m_{Q,a}^{\frac{k-1}{k}},
\]
for a constant $C$ depending only on $X$ (and the fixed geometric data), not on $h$ or the multiplicities.

\end{corollary}

\begin{proof}
Apply Proposition~\ref{prop:transport-flat-glue-weighted} facewise, then sum over interfaces and use the triangle inequality for $\mathcal F$.

Since $T^{\mathrm{raw}}=\sum_Q S_Q$, we have
\[
\partial T^{\mathrm{raw}}=\sum_Q \partial S_Q.
\]
On each interface $F=Q\cap Q'$, the restriction of $\partial T^{\mathrm{raw}}$ to $F$ is exactly the mismatch current
\(
B_F=(\partial S_Q)\llcorner F-(\partial S_{Q'})\llcorner F
\)
(with the induced orientations), and hence
\(
\partial T^{\mathrm{raw}}=\sum_F B_F
\)
as a sum over all interfaces $F$.
By the triangle inequality for the flat norm,
\[
\mathcal F(\partial T^{\mathrm{raw}})
\ \le\
\sum_F \mathcal F(B_F).
\]

For a fixed interface $F$, the translation model hypothesis and a matching $\sigma$ with
$\|u_a-u'_{\sigma(a)}\|\le \Delta_F$ give the per-face estimate
\[
\mathcal F(B_F)\ \le\ \Delta_F\sum_{a=1}^N \Bigl(\Mass(\Sigma_F(u_a))+\Mass(\partial\Sigma_F(u_a))\Bigr),
\]
so summing over $F$ yields the first bound.

Under the additional assumptions $\Delta_F\le C\,\varrho\,h^2$ and
$\Mass(\Sigma_F(u_a))+\Mass(\partial\Sigma_F(u_a))\lesssim m_a^{\frac{k-1}{k}}$ (with $k=2n-2p$),
we obtain
\[
\mathcal F(B_F)\ \lesssim\ \varrho\,h^2\sum_{a\in\mathcal S(F)} m_{F,a}^{\frac{k-1}{k}}.
\]
Finally, each piece $Y^{Q,a}\llcorner Q$ meets only $O(1)$ interfaces of its cell, so reorganizing the sum over faces into a sum over
cells and their pieces gives
\[
\mathcal F\!\left(\partial T^{\mathrm{raw}}\right)
\ \lesssim\ \varrho\,h^2\sum_Q\ \sum_{a\in\mathcal S(Q)} m_{Q,a}^{\frac{k-1}{k}},
\]
as claimed.
\end{proof}


\begin{lemma}[Borderline ($p=n/2$): closure via a refined displacement schedule]\label{lem:borderline-p-half}
Assume $p=n/2$ (equivalently $k=2n-2p=n$).
Under the hypotheses of Corollary~\ref{cor:global-flat-weighted} with the \emph{refined} displacement control
\[
\Delta_F\ \lesssim\ \varrho\,h^{2}
\qquad\text{(as in Lemma~\ref{lem:w1-auto}, Lemma~\ref{lem:template-displacement}, and Lemma~\ref{lem:w1-template-edit}),}
\]
and the packing bound $|\mathcal S(Q)|\lesssim \varepsilon^{-2p}=\varepsilon^{-n}$ (\REVMZ{Lemma~\ref{lem:sliver-packing}}),
one has the quantitative estimate
\[
\mathcal F(\partial T^{\mathrm{raw}})\ \le\ C\,\frac{\varrho}{\varepsilon}\,\Mass(T^{\mathrm{raw}})^{\frac{n-1}{n}},
\]
where $C$ depends only on $X,n$ (and the fixed geometric data), not on $h,\varepsilon,\varrho$.
In particular, if $\varrho=o(\varepsilon)$ as $h\downarrow 0$ and $\sup_h \Mass(T^{\mathrm{raw}})<\infty$ (as ensured by the construction),
then
\[
\mathcal F(\partial T^{\mathrm{raw}})\ \longrightarrow\ 0
\qquad\text{as }h\downarrow 0
\]
also in the middle-dimensional regime $p=n/2$.
\end{lemma}

\begin{proof}
Start from Corollary~\ref{cor:global-flat-weighted}.  In the refined schedule one has
\[
\mathcal F(\partial T^{\mathrm{raw}})
\ \le\ C\,\varrho\,h^2 \sum_Q \sum_{a\in \mathcal S(Q)} m_{Q,a}^{\frac{n-1}{n}},
\]
with $m_{Q,a}:=\Mass([Y^{Q,a}]\llcorner Q)$ and $M_Q:=\sum_{a\in\mathcal S(Q)} m_{Q,a}$.
Since the map $t\mapsto t^{(n-1)/n}$ is concave on $\R_+$, Jensen (or the standard $\ell^1$--$\ell^{(n-1)/n}$ inequality) yields
\[
\sum_{a\in \mathcal S(Q)} m_{Q,a}^{\frac{n-1}{n}}
\ \le\ |\mathcal S(Q)|^{\frac1n}\,M_Q^{\frac{n-1}{n}}.
\]
Using the packing bound $|\mathcal S(Q)|\lesssim \varepsilon^{-n}$ gives $|\mathcal S(Q)|^{1/n}\lesssim \varepsilon^{-1}$, hence
\[
\mathcal F(\partial T^{\mathrm{raw}})
\ \le\ C\,\frac{\varrho}{\varepsilon}\,h^2 \sum_Q M_Q^{\frac{n-1}{n}}.
\]
Finally, H\"older with exponents $\frac{n}{n-1}$ and $n$ implies
\[
\sum_Q M_Q^{\frac{n-1}{n}}
\ \le\ \bigl(\#\{Q\}\bigr)^{\frac1n}\Bigl(\sum_Q M_Q\Bigr)^{\frac{n-1}{n}}
\ =\ \bigl(\#\{Q\}\bigr)^{\frac1n}\,\Mass(T^{\mathrm{raw}})^{\frac{n-1}{n}}.
\]
Since $\#\{Q\}\asymp h^{-2n}$ for a cubulation at mesh $h$, we have $(\#\{Q\})^{1/n}\lesssim h^{-2}$.
Substituting yields
\[
\mathcal F(\partial T^{\mathrm{raw}})
\ \le\ C\,\frac{\varrho}{\varepsilon}\,\Mass(T^{\mathrm{raw}})^{\frac{n-1}{n}},
\]
as claimed.
\end{proof}





\begin{remark}[Consistency with the constant-mass-per-sheet template regime]
\REVMZ{\textbf{[Technical clarification.]}}
If every piece in a cell has comparable mass $m_{Q,a}\asymp h^{k}$ (the naive ``one sheet type'' model), then
$m_{Q,a}^{(k-1)/k}\asymp h^{k-1}$ and $\sum_a m_{Q,a}^{(k-1)/k}\asymp N_Q h^{k-1}\asymp M_Q/h$, where $M_Q=\sum_a m_{Q,a}$ is the total mass in $Q$.
The corollary then yields $\mathcal F(\partial T^{\mathrm{raw}})\lesssim \varrho\,h^2\sum_Q(M_Q/h)=\varrho\,h\,\sum_Q M_Q\asymp m\,\varrho\,h$,
recovering the unweighted ``template'' scaling from Remark~\ref{rem:w1-auto}.
\end{remark}



\begin{remark}[On vanishing per-piece masses (no hidden lower bound)]\label{rem:no-vanishing-piece-mass}
The weighted flat-norm estimate of Corollary~\ref{cor:global-flat-weighted}
\[
\mathcal F(\partial T^{\mathrm{raw}})\ \lesssim\ \varrho\,h^2\sum_Q\sum_{a\in\mathcal S(Q)} m_{Q,a}^{\frac{k-1}{k}}
\]
holds \emph{without} any hypothesis that the individual piece masses $m_{Q,a}$ are bounded below by a fixed multiple of $h^{k}$.
This is crucial in the sliver regime, where one may intentionally split a cell budget $M_Q$ into many tiny pieces in order to obtain large
template degrees of freedom and good interface matching.

\smallskip\noindent
What the gluing bookkeeping needs is instead a \emph{no-heavy-tail} condition: along each face, tail pieces created by a prefix edit must not carry
disproportionately large face-slice boundary mass compared to the matched prefix.  In the corner-exit route this is enforced by deterministic
face incidence (G1-iff) and uniform per-face comparability (G2) for holomorphic corner-exit slivers
(Proposition~\ref{prop:holomorphic-corner-exit-g1g2} and \REVMZ{Corollary~\ref{cor:holomorphic-corner-exit-inherits}}), together with the prefix-tail reduction
in \REVMZ{Lemma~\ref{lem:template-edits-oh}}.
\end{remark}




\begin{remark}[Model scaling at the Bergman cell size]\label{rem:sliver-bergman-scaling}
This remark records a simplified scaling calculation explaining why a ``sliver'' mechanism could, in principle, coexist with the intrinsic
holomorphic control scale $h\sim \mhol^{-1/2}$.

\smallskip\noindent
Assume cells have diameter $h\asymp \mhol^{-1/2}$ (as suggested by Lemma~\ref{lem:bergman-control}) so that uniform $C^1$ graph control holds on each cell.
Then the number of cells is $\asymp h^{-2n}\asymp m^{n}$, and the target mass per cell is
\[
M_Q\ \sim\ m\int_Q \beta\wedge\psi\ \asymp\ m\,h^{2n}\ \asymp\ m^{1-n}.
\]
In a smooth convex flat model (e.g.\ a ball cell), if $M_Q$ is split into $N_Q$ \emph{equal} sliver pieces of mass $M_Q/N_Q$, then the
$(2n-2p-1)$--dimensional boundary size of a single piece scales like $(M_Q/N_Q)^{\frac{k-1}{k}}$ (with $k:=2n-2p$), hence the total boundary size
on the cell boundary scales like
\[
\mathrm{Bdry}(Q)\ \asymp\ N_Q\Bigl(\frac{M_Q}{N_Q}\Bigr)^{\frac{k-1}{k}}
\ =\ M_Q^{\frac{k-1}{k}}\,N_Q^{\frac1k}.
\]
If, across a shared interface, the corresponding face slices are displaced by $\|v\|=O(h^2)$ (as in the template/face-map variation estimate (cf.\ Lemma~\ref{lem:template-displacement})),
then Lemma~\ref{lem:flat-translate} gives a per-piece flat mismatch $\lesssim \|v\|\times$(boundary mass).  A crude summation therefore yields a
per-face mismatch of order $h^n\,\varepsilon_{\mathcal T}$, and summing over the $\sim h^{-n}$ faces yields a
\[
\mathcal F(B_F)\ \lesssim\ \varrho\,h^2\,\mathrm{Bdry}(Q)\ \asymp\ \varrho\,h^2\,M_Q^{\frac{k-1}{k}}\,N_Q^{\frac1k}.
\]
Summing over $\asymp h^{-2n}$ faces gives the global scaling bound
\[
\mathcal F(\partial T^{\mathrm{raw}})\ \lesssim\ h^{-2n}\cdot h^2\cdot M_Q^{\frac{k-1}{k}}\,N_Q^{\frac1k}
\ \asymp\ m^{\frac{n-1}{k}}\,N_Q^{\frac1k}.
\]
Since $(n-1)/k<1$ for $k=2n-2p\ge 2$, this is automatically sublinear in $m$ provided $N_Q$ grows at most polynomially in $m$ with exponent $<k-(n-1)$.
Making any version of this calculation rigorous inside the cubical/face framework requires precisely the weighted bookkeeping estimate flagged in
Remark~\ref{rem:sliver-vs-template}.
\end{remark}


\begin{remark}[Handling slowly varying multiplicities]\label{rem:w1-multiplicity}
In practice the number of sheets in a given family $(Q,j)$ will vary with $Q$ because the target weights depend on $\beta(x_Q)$.
If adjacent cubes $Q,Q'$ have sheet counts differing by $r=|N_{Q,j}-N_{Q',j}|$, one can view their face measures as arising from the
same template after $r$ insertions/deletions.  Lemma~\ref{lem:w1-template-edit} then gives an additional contribution
$W_1\lesssim r\,h$ (since the transverse domain has diameter $O(h)$).
Thus, once one has a quantitative bound $r\le C\,h\,N_{Q,j}$ (slow variation), this term is of order
$W_1\lesssim h^2 N_{Q,j}$ and is absorbed into the $h^2 N$ scaling of Lemma~\ref{lem:w1-auto}.
Making this ``slow variation of integer counts'' rigorous is a rounding/Diophantine bookkeeping problem, separate from the geometric transport estimates.
\end{remark}














\begin{remark}[Bounded global corrections do not spoil the $O(h)$ edit regime]\label{rem:bounded-corrections}
In applications, one often needs to adjust rounded counts by a bounded amount (e.g.\ to enforce finitely many global period constraints).
If $N_Q\gtrsim h^{-1}$ uniformly and $\widetilde N_Q:=N_Q+\Delta_Q$ with $|\Delta_Q|\le C_0$, then
\[
\frac{|\widetilde N_Q-N_Q|}{\widetilde N_Q}\ \le\ \frac{C_0}{\widetilde N_Q}\ \lesssim\ C_0\,h.
\]
Thus such bounded corrections create only an $O(h)$ \emph{fraction} of insertions/deletions in a nested prefix-template scheme
(Remark) and are absorbed by Lemma~\ref{lem:template-edits-oh} for $h\ll 1$.
\end{remark}

\begin{remark}[Nested prefix-template scheme]\label{rem:nested-template-scheme}
Fix, for each direction label, an \emph{ordered} master template of transverse atoms $(y_a)_{a\ge 1}\subset B_{C_0\varrho h}(0)\subset\R^{2p}$.
For example, \REVMZ{Lemma~\ref{lem:sphere-quantize-nested}} produces a nested ordered sequence on a sphere (uniform density), and scaling embeds it into $B_{C_0\varrho h}(0)$.
For each cell $Q$ choose an integer count $N_Q$ and take the cell template to be the prefix
\(
\nu^{(N_Q)}:=\sum_{a=1}^{N_Q}\delta_{y_a}.
\)
Then across an interface $F=Q\cap Q'$ the two sides differ by a \emph{prefix edit} of size $|N_Q-N_{Q'}|$.
If the target counts come from rounding a smooth density, \REVMZ{Lemma~\ref{lem:slow-variation-rounding}} implies $|N_Q-N_{Q'}|/N_Q=O(h)$ in the ``many pieces'' regime.
Thus it suffices to ensure the \emph{unpaired boundary slice mass} on $F$ is an $O(h)$ fraction of the total face boundary mass; Lemma~\ref{lem:template-edits-oh}
then upgrades this to an $O(h^2)$ flat-norm contribution, matching the displacement bookkeeping.
\end{remark}







\begin{lemma}[Flat gluing bound from prefix coherence]\label{lem:sliver-template-glue-flat}
On an interior face $F=Q\cap Q'$, assume the hypotheses of Proposition~\ref{prop:prefix-template-coherence}.
Then the mismatch current $B_F$ satisfies
\[
\mathcal F(B_F)\ \le\ C\,h^2\Bigl(\Mass(\partial S_Q\llcorner F)+\Mass(\partial S_{Q'}\llcorner F)\Bigr)\ +\ C_{\angle}\,\varepsilon\,M_F,
\]
with constants as in Proposition~\ref{prop:prefix-template-coherence}.
\end{lemma}

\begin{proof}
This is exactly the estimate proved in Proposition~\ref{prop:prefix-template-coherence}.
\end{proof}


\begin{proposition}[Summing face mismatches]\label{prop:sliver-template-glue}
Let $T^{\mathrm{raw}}$ be the raw current obtained by assembling the cell currents $\{S_Q\}$.
Assume that on each interior face $F=Q\cap Q'$ the hypotheses of Lemma~\ref{lem:sliver-template-glue-flat} hold.
Then
\[
\mathcal F(\partial T^{\mathrm{raw}})\ \le\ \sum_{F}\mathcal F(B_F),
\]
where the sum runs over interior faces and $B_F$ is the face mismatch current.
\end{proposition}

\begin{proof}
Write $\partial T^{\mathrm{raw}}=\sum_F B_F$ as a sum over oriented interior faces.
By subadditivity of $\mathcal F$ we have $\mathcal F(\partial T^{\mathrm{raw}})\le \sum_F \mathcal F(B_F)$.
\end{proof}


\begin{proposition}[Global flat bound in the sliver regime]\label{prop:sliver-template-global}
Assume the hypotheses of Proposition~\ref{prop:sliver-template-glue}, and in addition that each face slice mass satisfies
the uniformly convex slice bound of Lemma~\ref{lem:uniformly-convex-slice-boundary}.
Then
\[
\mathcal F(\partial T^{\mathrm{raw}})\ \lesssim\ h^2\sum_Q\sum_{a\in\mathcal S(Q)} m_{Q,a}^{\frac{k-1}{k}}\ +\ O(\varepsilon\,m),
\qquad k:=2n-2p,
\]
where $m_{Q,a}:=\Mass([Y^{Q,a}]\llcorner Q)$ and $\mathcal S(Q)$ indexes the pieces in cell $Q$.
\end{proposition}

\begin{proof}
Combine Lemma~\ref{lem:sliver-template-glue-flat} with Proposition~\ref{prop:sliver-template-glue} to obtain
\[
\mathcal F(\partial T^{\mathrm{raw}})\ \le\ C\,h^2\sum_F\Bigl(\Mass(\partial S_Q\llcorner F)+\Mass(\partial S_{Q'}\llcorner F)\Bigr)\ +\ O(\varepsilon\,m).
\]
Each $\Mass(\partial S_Q\llcorner F)$ is a sum of slice masses contributed by pieces meeting $F$.
Lemma~\ref{lem:uniformly-convex-slice-boundary} bounds each such slice by $\lesssim m_{Q,a}^{(k-1)/k}$.
Summing over faces, each piece contributes to only $O(1)$ faces (cell-complex bounded overlap), yielding the claimed estimate.
\end{proof}


\begin{remark}[Quantifiers in the sliver estimate]\label{rem:sliver-quantifier-remark}
The estimate in Proposition~\ref{prop:sliver-template-global} is purely geometric.
To conclude $\mathcal F(\partial T^{\mathrm{raw}})=o(m)$ one needs a parameter schedule ensuring both
(i) the small-angle term $O(\varepsilon\,m)$ is $o(m)$ and (ii) the $h^2$--weighted boundary sum is $o(m)$.
This scaling estimate is recorded in Remark~\ref{rem:weighted-scaling}.
\end{remark}



\begin{proposition}[Local mass matching on a fixed template]\label{prop:sliver-mass-matching}
In the setting of Theorem~\ref{thm:local-sheets}, fix a direction family $j$ and footprint scale $s$ and a global ordered list of
transverse parameters $(t_{j,a})_{a\ge 1}$ used to generate the local sheets $S^{(a)}_j$.
For any cell $Q$ with target budget $M_Q\ge 0$, choose $N_Q$ as in Proposition~\ref{prop:sliver-cell-mass-proportions}
and take the $N_Q$ corresponding sheets inside $Q$.
Then the resulting local current $S_Q:=\sum_{a\le N_Q}S^{(a)}_j$ satisfies the mass matching property
\[
\Mass(S_Q)=M_Q+o(M_Q)
\]
whenever $N_Q\to\infty$ and $\varepsilon_{\mathrm{hol}}\to 0$ (uniformly over cells with $M_Q\gg s^{k}$).
\end{proposition}

\begin{proof}
This is an immediate restatement of Proposition~\ref{prop:sliver-cell-mass-proportions}.
\end{proof}


\begin{remark}[What ``mass matching'' accomplishes]\label{rem:sliver-matching-remark}
The local mass matching property in Proposition~\ref{prop:sliver-mass-matching} supplies hypothesis
\textnormal{(iii)} in Theorem~\ref{thm:sliver-mass-matching-on-template} for each direction family.
The remaining nontrivial input for global boundary control is the interface coherence provided by prefixes together with
the $O(h)$--fraction edit hypothesis on the unmatched tail (see Proposition~\ref{prop:prefix-template-coherence}
and Remark~\ref{rem:sliver-template-extension}).
\end{remark}


\begin{theorem}[Global prefix-template activation / mass matching (template bookkeeping)]\label{thm:sliver-mass-matching-on-template}
Fix a mesh-$h$ decomposition by smooth uniformly convex cells (rounded cubes) and fix a direction label $j$ with paired calibrated reference planes across neighbors.
Fix an \emph{ordered} master template of transverse atoms $(y_a)_{a\ge 1}\subset B_{C_0\varrho h}(0)\subset\R^{2p}$.
For each cell $Q$, let $N_Q\in\Z_{\ge 0}$ be the desired integer count for family $j$ (derived from the Lipschitz target weights) and let
$M_Q\ge 0$ be the corresponding target mass budget for that family (obtained from the smooth form $m\beta$).
Assume:
\begin{enumerate}
\item[\textnormal{(i)}] (\textbf{Many pieces}) $N_Q\gtrsim h^{-1}$ on the region where $M_Q$ is not negligible;
\item[\textnormal{(ii)}] (\textbf{Slow variation}) $|N_Q-N_{Q'}|\le C\,h\,\min\{N_Q,N_{Q'}\}$ for adjacent cells $Q\sim Q'$;
\item[\textnormal{(iii)}] (\textbf{Local realizability on a fixed template}) for each $Q$ {there exist $\psi$--calibrated holomorphic pieces (no disjointness is assumed; repetitions/multiplicity are allowed)}
$Y^1,\dots,Y^{N_Q}$ in $Q$ whose transverse parameters are the prefix $\{y_a\}_{a\le N_Q}$, and whose total mass satisfies
\[
\sum_{a=1}^{N_Q}\Mass([Y^a]\llcorner Q)\ =\ M_Q\ +\ o(M_Q)
\]
as $h\to 0$ (uniformly over $Q$).
\item[\textnormal{(iv)}] (\textbf{$O(h)$ edit regime on faces}) For every interior interface $F=Q\cap Q'$, the unmatched part satisfies the
$O(h)$--fraction hypothesis of Proposition~\ref{prop:prefix-template-coherence}.
\end{enumerate}
Then the resulting raw current built from these pieces satisfies the per-face flat-norm mismatch bound of Proposition~\ref{prop:prefix-template-coherence}.
Consequently one obtains the global estimate
\[
\mathcal F(\partial T^{\mathrm{raw}})\ \lesssim\ \varrho\,h^2\sum_Q\sum_{a\in\mathcal S(Q)} m_{Q,a}^{\frac{k-1}{k}}\ +\ O(\varepsilon\,m),
\qquad k:=2n-2p,
\]
where $m_{Q,a}:=\Mass([Y^{Q,a}]\llcorner Q)$ and $\varepsilon$ is the small-angle parameter.
In particular, under the parameter regime of Remark~\ref{rem:weighted-scaling} (e.g.\ Bergman scale $h\sim \mhol^{-1/2}$, polynomial piece count per cell, and $p\le n/2$),
one has $\mathcal F(\partial T^{\mathrm{raw}})=o(m)$.
\end{theorem}



\begin{proof}
By Proposition~\ref{prop:sliver-template-global} we have the global flat-norm estimate.
The $o(m)$ conclusion follows from the scaling computation in Remark~\ref{rem:weighted-scaling}.

For completeness we record the short argument below.

For each interior interface $F=Q\cap Q'$, Proposition~\ref{prop:prefix-template-coherence} provides a bound of the form
\[
\mathcal F(B_F)
\ \le\ C\,h^2\Bigl(\Mass(\partial S_Q\llcorner F)+\Mass(\partial S_{Q'}\llcorner F)\Bigr)\ +\ C_{\angle}\,\varepsilon\,M_F,
\]
where $M_F$ is the total interior mass of pieces meeting $F$.
Summing over all interior faces and using subadditivity of $\mathcal F$ gives
\[
\mathcal F(\partial T^{\mathrm{raw}})
\le \sum_F \mathcal F(B_F)
\le C\,h^2\sum_F\Bigl(\Mass(\partial S_Q\llcorner F)+\Mass(\partial S_{Q'}\llcorner F)\Bigr)\ +\ O(\varepsilon\,m),
\]
since $\sum_F M_F\lesssim m$ (each piece meets only $O(1)$ faces).

\smallskip\noindent
Each face boundary mass is a sum of slice masses $\Mass(\Sigma_F(u_a))$ coming from pieces $Y^{Q,a}\cap Q$ meeting $F$.
By Lemma~\ref{lem:uniformly-convex-slice-boundary},
\[
\Mass(\Sigma_F(u_a))\ \lesssim\ m_{Q,a}^{\frac{k-1}{k}},
\qquad m_{Q,a}:=\Mass([Y^{Q,a}]\llcorner Q),\qquad k:=2n-2p.
\]
Therefore,
\[
\sum_F\Bigl(\Mass(\partial S_Q\llcorner F)+\Mass(\partial S_{Q'}\llcorner F)\Bigr)
\ \lesssim\ \sum_Q\sum_{a\in\mathcal S(Q)} m_{Q,a}^{\frac{k-1}{k}},
\]
because each piece contributes to only finitely many faces. (Finite-overlap depends only on the fixed cell-complex combinatorics.)
Substituting yields the stated global estimate for $\mathcal F(\partial T^{\mathrm{raw}})$.
Finally, the $o(m)$ conclusion follows from the scaling/packing computation in Remark~\ref{rem:weighted-scaling}.
\end{proof}



\begin{remark}[Status of the activation hypotheses in the corner-exit route]\label{rem:activation-hypotheses-status}
Theorem~\ref{thm:sliver-mass-matching-on-template} is stated as a bookkeeping reduction: it converts per-cell realization and an $O(h)$ face-edit regime
into the global flat-norm bound needed for gluing.
In the corner-exit vertex-template construction, the hypotheses are verified as follows.
\begin{itemize}
\item \textbf{(i)--(ii)} Many pieces and slow variation follow from rounding Lipschitz targets: see Lemma and the
$0$--$1$ stability Lemma (the lower bound $N_Q\gtrsim h^{-1}$ holds on regions where the target density is bounded below).
\item \textbf{(iii)--(iv)} Local realizability on a fixed ordered template and the $O(h)$ face-edit regime are certified for corner-exit vertex templates by
Corollary (using Propositions~\ref{prop:holomorphic-corner-exit-L1}, ,
and  / ).
\item \textbf{All labels simultaneously (B1)} The all-direction packaged execution is recorded in \REVMZ{Proposition~\ref{prop:global-coherence-all-labels}}.
\end{itemize}
Thus the ``global activation gate'' is complete in the corner-exit route; the remaining work is purely expository (keeping these references prominent at the point of use).
\end{remark}



\begin{remark}[Sliver cube templates]\label{rem:sliver-cube-templates}
\REVMZ{\textbf{[Technical clarification.]}}
In the corner-exit regime, each cell $Q$ carries an ordered list of candidate translation sites (a \emph{cube template})
near its vertex stars.  The template is chosen from a finite net so adjacent cells agree on the shared-face prefix,
up to an $O(h)$ unmatched tail.
\end{remark}



\begin{remark}[Template candidates and ordering]\label{rem:sliver-template-candidates}
The ordered master template $(y_a)_{a\ge1}$ is fixed globally; each local realization selects a finite prefix
(and possibly discards an $O(h)$-fraction tail) to match face populations.  The ordering is what makes
prefix coherence a meaningful, checkable condition.
\end{remark}




\begin{proposition}[Flat-ball model: prefix activation is feasible]\label{prop:prefix-activation-flat-ball}
In the Euclidean ball-cell model of \REVMZ{Proposition~\ref{prop:flat-sliver-local}}, fix a radius $r\in(0,h)$ so that each affine piece
$[P+t]\llcorner B_h(0)$ with $t\in S^{2p-1}(r)$ has the same mass $\mu(r)$.
Fix an ordered $\delta$--separated template $(t_a)_{a\ge 1}\subset S^{2p-1}(r)$ and define prefixes
\(
\nu^{(N)}:=\sum_{a=1}^N \delta_{t_a}.
\)
Then for any target mass $M\ge 0$, choosing $N=\lfloor M/\mu(r)\rceil$ gives
\[
\Bigl|\sum_{a=1}^N \Mass([P+t_a]\llcorner B_h(0))\ -\ M\Bigr|\ \le\ \mu(r),
\qquad
\frac{\mu(r)}{M}\ =\ O\!\left(\frac1N\right)\ \text{ when }M\gg \mu(r).
\]
Moreover, if two neighboring cells choose counts $N$ and $N'$ with $|N-N'|\le \theta\,\min\{N,N'\}$, then the induced prefix edit is a $\theta$--fraction
of the pieces (hence of the face-boundary mass, since all pieces have comparable slice boundary by the ball scaling law).
\end{proposition}

\begin{proof}
Since $Q=B_h(0)$ is rotationally symmetric, the cross-sectional volume
\(
\Mass([P+t]\llcorner B_h(0))=\mathcal H^{2(n-p)}\bigl((P+t)\cap B_h(0)\bigr)
\)
depends only on $\|t\|$ (equivalently, only on the distance from the center to the affine plane $P+t$).
Hence it is constant on the sphere $S^{2p-1}(r)$; denote this constant by $\mu(r)$.

For the mass-budget estimate, take $N=\lfloor M/\mu(r)\rceil$.  Then by nearest-integer rounding,
\(
|N\mu(r)-M|\le \mu(r),
\)
which is exactly the displayed inequality.

For the edit claim, suppose two cells choose counts $N$ and $N'$, and assume (as in the ball model) that the relevant face-slice boundary masses are equal
or uniformly comparable across indices.
Then the unmatched tail has size $|N-N'|$, so the unmatched face boundary mass is a fraction $\asymp |N-N'|/\min\{N,N'\}\le \theta$ of the total.
\end{proof}


\begin{corollary}[Holomorphic prefix activation on a Bergman-scale ball cell]\label{cor:prefix-activation-holo}
In the setting of Corollary, take $\rho\equiv 1$ on the sphere $S^{2p-1}(r)$ and choose a separated ordered template
$(t_a)_{a=1}^{N}$ as in Proposition~\ref{prop:prefix-activation-flat-ball}.
Then the resulting holomorphic pieces $Y^1,\dots,Y^N$ on the cell $Q$ satisfy
\[
\Mass([Y^a]\llcorner Q)=(1+O(\varepsilon^2))\,\mu(r)
\qquad\text{for all }a,
\]
so selecting a prefix of length $N_Q$ matches a target mass budget $M_Q$ up to a relative error $O(1/N_Q)+O(\varepsilon^2)$, and prefix edits of size
$|N_Q-N_{Q'}|$ contribute only an $O(|N_Q-N_{Q'}|/\min\{N_Q,N_{Q'}\})$ fraction of face-boundary mass.
\end{corollary}
\begin{proof}
When $\rho$ is constant on the sphere $S^{2p-1}(r)$, the flat slices in the template have equal mass:
each affine piece over $P+t_a$ contributes the same interior mass $\mu(r)$ on the cell $Q$.
The ordered template from the flat prefix-activation construction therefore has the property that taking a prefix of length $N_Q$
produces total interior mass $N_Q\,\mu(r)$, so choosing $N_Q$ by rounding a target mass budget produces a relative error $O(1/N_Q)$.
Moreover, on a fixed face $F$ the per-piece face-slice boundary masses are equal (or uniformly comparable) across the template,
so changing from $N_Q$ to $N_{Q'}$ across a neighbor interface affects only the unmatched tail of size $|N_Q-N_{Q'}|$ and hence changes
the face boundary mass by an $O(|N_Q-N_{Q'}|/\min\{N_Q,N_{Q'}\})$ fraction.

The holomorphic upgrade replaces each affine slice by a holomorphic complete intersection piece $Y^a$ that is a $C^1$ graph of slope $O(\varepsilon)$,
hence its Jacobian differs from the affine Jacobian by $1+O(\varepsilon^2)$.
In particular,
\(
\Mass([Y^a]\llcorner Q)=(1+O(\varepsilon^2))\,\mu(r)
\)
for every $a$, and the same $1+O(\varepsilon^2)$ comparability holds for the face-slice boundary masses on any interface.
Therefore the flat prefix activation conclusions transfer verbatim, with the additional $O(\varepsilon^2)$ relative error claimed in the statement.
\end{proof}



\begin{lemma}[A sufficient condition for the $O(h)$ face-edit regime]\label{lem:oh-face-edit-regime}

Fix an interior interface $F=Q\cap Q'$ and a paired direction label $j$, and assume $N_Q\ge N_{Q'}$.
Write $N_{\min}:=N_{Q'}$ and $r:=N_Q-N_{Q'}$.
Let the face-slice boundary masses on $F$ of the pieces indexed by the master template be
\[
b_a(F)\ :=\ \Mass\!\big(\partial([Y^a]\llcorner Q)\llcorner F\big)\ \ge\ 0,
\qquad a=1,\dots,N_Q,
\]
so that $\Mass(\partial S_Q\llcorner F)=\sum_{a=1}^{N_Q} b_a(F)$.
Assume:
\begin{enumerate}
\item[\textnormal{(a)}] (\textbf{Prefix activation on the face}) after aligning the order of indices across $Q$ and $Q'$,
the paired part on $F$ is exactly the common prefix $\{1,\dots,N_{\min}\}$, and the unpaired part on $F$ is the tail
$\{N_{\min}+1,\dots,N_{\min}+r\}$ coming from the larger side;
\item[\textnormal{(b)}] (\textbf{No heavy tail}) there exists $\kappa\ge 1$ such that every tail term is bounded by the prefix average:
\[
b_{a}(F)\ \le\ \kappa\cdot \frac{1}{N_{\min}}\sum_{i=1}^{N_{\min}} b_i(F)
\qquad\text{for all }a>N_{\min};
\]
\item[\textnormal{(c)}] (\textbf{Slow count variation}) $r\le C\,h\,N_{\min}$.
\end{enumerate}
Then the unpaired face boundary mass satisfies the $O(h)$-fraction hypothesis
\[
\sum_{a>N_{\min}} b_a(F)\ \le\ \theta_F\sum_{a\le N_Q} b_a(F)
\qquad\text{with}\qquad \theta_F\ \le\ (\kappa C)\,h.
\]
In particular, hypothesis (iv) in Theorem~\ref{thm:sliver-mass-matching-on-template} holds (after absorbing constants).

\end{lemma}

\begin{proof}
By (b),
\[
\sum_{a>N_{\min}} b_a(F)\ \le\ r\cdot \kappa\,\frac{1}{N_{\min}}\sum_{i=1}^{N_{\min}} b_i(F).
\]
By (c), $r\le C h N_{\min}$, hence the right-hand side is $\le (\kappa C)h \sum_{i=1}^{N_{\min}} b_i(F)\le (\kappa C)h \sum_{a\le N_Q} b_a(F)$.
\end{proof}

\begin{remark}[Item \textnormal{(iv)}: tail-heaviness and how it is enforced]\label{rem:iv-what-remains}

Lemma~\ref{lem:oh-face-edit-regime} isolates the only nontrivial ingredient needed for the $O(h)$ face-edit estimate in item~\textnormal{(iv)} of Theorem~\ref{thm:sliver-mass-matching-on-template}: when passing from a prefix template to a longer template, the added ``tail'' pieces must not contribute a disproportionate amount of face-slice boundary mass on any interior face $F$.

In the present paper this requirement is built into the \emph{finite corner-exit direction net} and the associated activation rule.
By Proposition~\ref{prop:corner-exit-template-net}, all slivers in a fixed template label have identical footprint geometry; in particular, for each fixed interior face $F$ the per-piece face-slice boundary masses are \emph{uniform} within the label.
Consequently the tail-heaviness hypothesis \textnormal{(b)} in Lemma~\ref{lem:oh-face-edit-regime} holds with $\kappa=1$ (uniformly in the refinement parameter), and item~\textnormal{(iv)} follows by combining Lemma~\ref{lem:oh-face-edit-regime} with the checkerboard face-edit estimate of \REVMZ{Proposition~\ref{prop:checkerboard-face-oh-edit}}.


\smallskip
For intuition, in the dense-sheet translation-invariant model the same uniformity is automatic because each sheet is a translate of a fixed slice current; the net construction above is the robust replacement used in the sliver regime.
\end{remark}


\begin{remark}[Parameter tension and the chosen regime]\label{rem:param-tension}

There is a genuine tension between (a) taking many pieces per cube (to average fluctuations in counting) and (b) keeping the face mismatch small enough that the flat gluing error tends to zero as the mesh is refined.
The present paper resolves this tension by working in the \emph{sliver / corner-exit regime} and by using \emph{weighted} matching rather than raw counting:

\begin{itemize}
\item The finite direction net and the corner-exit realization produce, within each activated template label, pieces with identical footprint geometry (Proposition~\ref{prop:corner-exit-template-net}). In particular, per-piece slice and boundary masses are uniform at the label level.
\item The global matching is then performed at the level of integer-weighted face measures and transported with flat control (Propositions and~\ref{prop:transport-flat-glue-weighted}), yielding the required $W_1$-type matching while keeping the induced boundary mismatch $o(m)$ in the weighted scaling (Remark~\ref{rem:weighted-scaling}).
\end{itemize}

In this way, the proof does not rely on a dense-sheet cancellation principle; the smallness of the gluing error is obtained directly from the weighted transport bounds together with the finite-net uniformity.

\end{remark}
\begin{remark}[Hard Lefschetz reduction to $p\le n/2$]\label{rem:lefschetz-reduction} \cite[Ch.~6]{Voisin02}
\REVMZ{\textbf{[Key conceptual point.]}}
Because $X$ is projective, the K\"ahler class $[\omega]=c_1(L)$ is algebraic (hyperplane class).
By hard Lefschetz, for $p>\frac{n}{2}$ the map
\[
L^{2p-n}:\ H^{2(n-p)}(X,\Q)\longrightarrow H^{2p}(X,\Q),\qquad \eta\mapsto [\omega]^{2p-n}\wedge \eta,
\]
is an isomorphism.  Hence any rational Hodge class $\gamma\in H^{2p}(X,\Q)\cap H^{p,p}(X)$ can be written uniquely as
$\gamma=[\omega]^{2p-n}\wedge\eta$ with $\eta\in H^{2(n-p)}(X,\Q)\cap H^{n-p,n-p}(X)$.
If $\eta$ is represented by an algebraic cycle $Z$ of codimension $(n-p)$, then intersecting $Z$ with $(2p-n)$ generic hyperplanes produces
an algebraic cycle representing $\gamma$.
Therefore, for the complete closure of the Hodge conjecture, it is enough to prove the realization step for $p\le \frac{n}{2}$.
\end{remark}



\begin{lemma}[Mass tunability of plane slices in the flat model]\label{lem:mass-tunable}
In the flat chart model, fix a calibrated affine $(2n-2p)$-plane $P\subset\R^{2n}$ and a \emph{smooth convex} cell $Q$ of diameter $h$
(e.g.\ a Euclidean ball, or a cube with rounded corners).
The function
\[
t\ \longmapsto\ \Mass\big([P+t]\llcorner Q\big)
\]
is continuous in the translation parameter $t\in P^\perp\cong\R^{2p}$ and takes values in an interval $[0,A_{\max}]$ with $A_{\max}\asymp h^{2(n-p)}$.
In particular, for any $a\in(0,A_{\max})$ there exist translations $t$ such that $\Mass([P+t]\llcorner Q)=a$.
\end{lemma}


\begin{proof}
Write $k:=2(n-p)$.  In the flat model one has
\[
\Mass([P+t]\llcorner Q)=\mathcal H^{k}\bigl((P+t)\cap Q\bigr).
\]
Continuity in $t$ follows because this is the integral of the indicator function $\mathbf 1_Q$ over the translated plane:
for any sequence $t_\nu\to t$, the sets $(P+t_\nu)\cap Q$ converge to $(P+t)\cap Q$ in the sense of characteristic functions on $P$
after identifying $P+t_\nu$ with $P$ by translation, and dominated convergence applies since $\mathbf 1_Q$ is bounded.

The maximum $A_{\max}$ is achieved by some translate intersecting the bulk of $Q$ and satisfies $A_{\max}\asymp h^{k}$
because $Q$ contains and is contained in Euclidean balls of radii comparable to $h$ (uniform convexity/diameter control).
The value $0$ occurs for translates $P+t$ far enough that $(P+t)\cap Q=\emptyset$.
Therefore the image contains an interval $[0,A_{\max}]$, and the intermediate value theorem yields translations realizing any $a\in(0,A_{\max})$.
\end{proof}


\begin{remark}[Sliver pieces and fixed-$m$ microstructure]\label{rem:sliver}
\REVMZ{\textbf{[Geometric intuition.]}}
Lemma~\ref{lem:mass-tunable} indicates a potential escape from the dense-vs-gluing tension at fixed $m$:
one may take \emph{many} parallel calibrated sheets in a cube but choose their translations so that each sheet contributes only a tiny mass
(``sliver pieces''), with the total mass still matching $m\int_Q\beta\wedge\psi$.
If such tunability persists under the holomorphic complete-intersection upgrade (Substep~3.5) with uniform control, then one can have
large sheet counts per face (good for $W_1$ matching) while keeping the total mass $O(m)$.
Making this quantitative in the projective setting is part of the remaining realization problem.
\end{remark}


\begin{lemma}[Quantizing a Lipschitz density on a sphere]\label{lem:sphere-quantize}
Let $d\ge 2$ and let $S^{d-1}(r)\subset\R^d$ be the Euclidean sphere of radius $r>0$.
Let $\rho$ be a nonnegative Lipschitz function on $S^{d-1}(r)$ with total mass
\[
M:=\int_{S^{d-1}(r)} \rho\,d\sigma.
\]
Then for every $N\in\N$ there exist points $t_1,\dots,t_N\in S^{d-1}(r)$ such that the equal-weight atomic measure
\[
\mu_N:=\sum_{a=1}^N \frac{M}{N}\,\delta_{t_a}
\]
satisfies the transport bound
\[
W_1(\mu_N,\rho\,d\sigma)\ \le\ C(d)\,r\,\Bigl(M+\mathrm{Lip}(\rho)\,r^{d-1}\Bigr)\,N^{-\frac{1}{d-1}}.
\]
Moreover, the points may be chosen $\delta$--separated with
\[
\|t_a-t_b\|\ \ge\ c(d)\,r\,N^{-\frac{1}{d-1}}
\qquad (a\neq b).
\]
\end{lemma}


\begin{proof}
This is a standard $W_1$ quantization bound on the $(d\!-\!1)$--sphere.
One concrete route is to start from a maximal $\delta$--separated set $\{t_a\}\subset S^{d-1}(r)$ with
\(
\delta\asymp r\,N^{-1/(d-1)},
\)
which has cardinality $\asymp N$ by packing, and then trim/duplicate finitely many points to obtain exactly $N$ points while preserving separation at the stated scale.
Let $\{C_a\}$ be the associated Voronoi cells; then $\mathrm{diam}(C_a)\lesssim \delta$.

Define the cell-averaged atomic measure $\widetilde\mu:=\sum_a \bigl(\int_{C_a}\rho\,d\sigma\bigr)\delta_{t_a}$.
Transporting the mass of each cell $C_a$ to its representative $t_a$ gives
\[
W_1(\widetilde\mu,\rho\,d\sigma)\ \le\ \sum_a \mathrm{diam}(C_a)\int_{C_a}\rho\,d\sigma\ \lesssim\ \delta\,M.
\]
To convert $\widetilde\mu$ to the equal-weight measure $\mu_N=\sum_{a=1}^N \frac{M}{N}\delta_{t_a}$, rebalance the atomic weights.
Since $\rho$ is Lipschitz and each cell has diameter $\lesssim\delta$, the discrepancy between the cell masses and the equal weight $M/N$
is controlled at scale $\lesssim \mathrm{Lip}(\rho)\,\delta\,r^{d-1}$.
Rebalancing these weights can be done by transporting mass between nearby cells at cost $\lesssim \delta$ per unit mass, yielding the stated bound
\(
W_1(\mu_N,\rho\,d\sigma)\lesssim \delta\,(M+\mathrm{Lip}(\rho)\,r^{d-1}).
\)
We record the rate and dependencies here; a detailed implementation of this standard quantization argument can be found, for example, in texts on optimal quantization
or empirical $W_1$ convergence on compact manifolds \REVMZ{\cite{Villani03}}.
\end{proof}


\begin{lemma}[Nested equal-weight quantization of the uniform sphere]\label{lem:sphere-quantize-nested}
Let $d\ge 2$ and let $S^{d-1}(r)\subset\R^d$ be the Euclidean sphere of radius $r>0$, with normalized surface measure $\sigma_r$.
There exists an (infinite) sequence of points $(t_a)_{a\ge 1}\subset S^{d-1}(r)$ such that for every $N\ge 1$ the equal-weight empirical measure
\[
\mu_N\ :=\ \frac{1}{N}\sum_{a=1}^N \delta_{t_a}
\]
satisfies
\[
W_1(\mu_N,\sigma_r)\ \le\ C(d)\,r\,N^{-\frac{1}{d-1}}.
\]
\end{lemma}

\begin{proof}
Build a nested sequence of partitions of $S^{d-1}(r)$ into $\asymp 2^{(d-1)k}$ measurable cells at level $k$, each of diameter $\lesssim r\,2^{-k}$
and with $\sigma_r$-mass exactly $2^{-(d-1)k}$ (for example, by inductively bisecting cells by smooth hypersurfaces; existence of equal-area partitions with
controlled diameter is standard on the sphere).
Choose one representative point in each cell and enumerate these points in increasing level order to obtain a single infinite sequence $(t_a)_{a\ge 1}$.

For $N\asymp 2^{(d-1)k}$, the first $N$ points consist of one representative from each cell at level $k$.
Transporting the mass of each cell to its representative costs at most $\mathrm{diam}(\text{cell})\cdot\sigma_r(\text{cell})\lesssim r\,2^{-k}\cdot 2^{-(d-1)k}$,
and summing over the $2^{(d-1)k}$ cells yields $W_1(\mu_N,\sigma_r)\lesssim r\,2^{-k}\asymp r\,N^{-1/(d-1)}$.
For intermediate $N$, compare to the nearest dyadic level and absorb constants.
\end{proof}




\begin{proposition}[Flat ball model slivers achieve $W_1$ transverse approximation]\label{prop:flat-sliver-local}
Work in the flat decomposition $\R^{2n}=\R^{2(n-p)}\oplus\R^{2p}$ and let $P:=\R^{2(n-p)}\times\{0\}$.
Let $Q:=B_h(0)\subset\R^{2n}$ be the Euclidean ball of radius $h$.
Fix a radius $r\in(0,h)$ and let $\sigma_r$ denote surface measure on $S^{2p-1}(r)\subset P^\perp\cong\R^{2p}$.
Let $\rho$ be a nonnegative Lipschitz density on $S^{2p-1}(r)$ with total mass
$M=\int_{S^{2p-1}(r)}\rho\,d\sigma_r$.
Then for every $N\in\N$ there exist translations $t_1,\dots,t_N\in S^{2p-1}(r)$ such that the affine calibrated pieces
\[
T_N\ :=\ \sum_{a=1}^N \bigl([P+t_a]\llcorner Q\bigr)
\]
are pairwise disjoint and:
\begin{enumerate}
\item[\textnormal{(i)}] (\textbf{Equal sliver masses}) $\Mass([P+t_a]\llcorner Q)=\Mass([P+t_1]\llcorner Q)$ for all $a$ (depends only on $r$);
\item[\textnormal{(ii)}] (\textbf{Transverse $W_1$ approximation}) with $\mu_N:=\sum_{a=1}^N \frac{M}{N}\delta_{t_a}$ one has
\[
W_1(\mu_N,\rho\,d\sigma_r)\ \le\ C(p)\,r\,\Bigl(M+\mathrm{Lip}(\rho)\,r^{2p-1}\Bigr)\,N^{-\frac{1}{2p-1}}.
\]
\end{enumerate}
\end{proposition}


\begin{proof}
For \textnormal{(i)}, note that $\Mass([P+t]\llcorner Q)=\mathcal H^{2(n-p)}((P+t)\cap B_h(0))$ depends only on the distance from the center to the affine plane
$P+t$, i.e.\ only on $\|t\|$, by rotational symmetry of the Euclidean ball.  Hence it is constant on $S^{2p-1}(r)$.

For \textnormal{(ii)}, apply Lemma~\ref{lem:sphere-quantize} with $d=2p$ to the Lipschitz density $\rho$ on $S^{2p-1}(r)$ to obtain points $t_a\in S^{2p-1}(r)$
such that the equal-weight atomic measure $\mu_N=\sum_{a=1}^N \frac{M}{N}\delta_{t_a}$ satisfies the stated $W_1$ bound.

Disjointness of the pieces $[P+t_a]\llcorner Q$ is immediate because the affine planes $P+t_a$ are parallel and distinct whenever $t_a\neq t_b$.
\end{proof}




\begin{corollary}[Holomorphic upgrade on a ball cell]\label{cor:holomorphic-flat-sliver-local}
In the setting of Proposition~\ref{prop:flat-sliver-local}, assume $Q$ lies in a holomorphic chart and that $P$ is a calibrated complex
$(n-p)$-plane in those coordinates with normal covectors $\lambda_1,\dots,\lambda_p$.
Fix $\varepsilon>0$ and choose {$\mhol\ge m_{\mathrm{hol},1}(\varepsilon)$} (Lemma~\ref{lem:bergman-control}) with $\mathrm{diam}(Q)\le c\,\mhol^{-1/2}$.
Then, after possibly reducing $N$ by a dimensional constant (absorbed into $C(p)$), the translations $t_a$ may be chosen so that
\[
\|t_a-t_b\|\ \ge\ 10\,\varepsilon\,\mathrm{diam}(Q)\qquad (a\neq b),
\]
and Proposition~\ref{prop:finite-template} produces $\psi$-calibrated holomorphic complete intersections $Y^1,\dots,Y^N$ whose restricted
pieces on $Q$ are disjoint $C^1$ graphs over $P+t_a$ with
\[
\Mass([Y^a]\llcorner Q)=(1+O(\varepsilon^2))\,\Mass([P+t_a]\llcorner Q).
\]
Consequently, the induced transverse measure $\sum_a \Mass([Y^a]\llcorner Q)\,\delta_{t_a}$ approximates $\rho\,d\sigma_r$ in $W_1$ with error
bounded by the right-hand side of Proposition~\ref{prop:flat-sliver-local} plus an additional $O(\varepsilon^2)\,M$ term.
\end{corollary}
\begin{proof}
Apply the flat model construction to obtain translations $t_1,\dots,t_N$ and the corresponding affine calibrated pieces over $P+t_a$
with the stated $W_1$ approximation to $\rho\,d\sigma_r$.
By a standard packing/subselection argument on the sphere (discarding at most a dimensional constant fraction of the points),
we may replace the family by a subfamily (renaming and keeping the same notation) so that
\(
\|t_a-t_b\|\ge 10\,\varepsilon\,\mathrm{diam}(Q)
\)
for all $a\neq b$.

With {$\mhol\ge m_{\mathrm{hol},1}(\varepsilon)$} and $\mathrm{diam}(Q)\le c\,\mhol^{-1/2}$, the Bergman-scale
 $C^1$ control and the holomorphic finite-template
construction apply at each translation parameter $t_a$, producing $\psi$-calibrated holomorphic complete intersections
$Y^1,\dots,Y^N$ whose restrictions to $Q$ are disjoint $C^1$ graphs over $P+t_a$ with slope $O(\varepsilon)$.
In particular their masses satisfy
\[
\Mass([Y^a]\llcorner Q)=(1+O(\varepsilon^2))\,\Mass([P+t_a]\llcorner Q)
\qquad\text{for each }a.
\]

Let
\(
\mu_{\mathrm{flat}}:=\sum_a \Mass([P+t_a]\llcorner Q)\,\delta_{t_a}
\)
and
\(
\mu_{\mathrm{holo}}:=\sum_a \Mass([Y^a]\llcorner Q)\,\delta_{t_a}.
\)
The mass comparison gives $\mu_{\mathrm{holo}}=(1+O(\varepsilon^2))\,\mu_{\mathrm{flat}}$, hence
\(
W_1(\mu_{\mathrm{holo}},\mu_{\mathrm{flat}})\lesssim \varepsilon^2\,M
\)
(with the domain diameter absorbed into the implicit constant), where $M=\int_\Omega\rho$ is the total target mass.
Combining this with the $W_1(\mu_{\mathrm{flat}},\rho\,d\sigma_r)$ estimate from the flat model yields the stated conclusion.
\end{proof}





\begin{remark}[Interpretation]
\REVMZ{\textbf{[Optional context.]}}
Proposition~\ref{prop:flat-sliver-local} shows that the \emph{transverse-measure approximation} requirement in the sliver program is achievable
in a clean flat ball model using exact affine calibrated pieces.
The remaining nontrivial step in this \emph{sliver program} is the \emph{holomorphic complete-intersection upgrade with uniform $C^1$ control}
(captured by Lemma~\ref{lem:bergman-control} and Proposition~\ref{prop:finite-template}) together with cube/face compatibility for gluing.
This conjectural sliver route is included only for context; the complete proof in this manuscript proceeds instead via the corner-exit vertex-template mechanism
(Propositions~\ref{prop:holomorphic-corner-exit-L1}, , , and the all-label package )
and does \emph{not} rely on Conjecture.
\end{remark}



\begin{conjecture}[Local sliver-sheet realizability (quantitative target)]\label{conj:sliver-local}
\REVMZ{\textbf{Note.} This conjecture provides a quantitative target for an alternative ``sliver'' route and is not required for the main proof.}
\smallskip
Fix a sufficiently small \emph{smooth convex} coordinate cell $Q$ of diameter $h$ inside a holomorphic chart
(e.g.\ a geodesic ball, or a cubical cell with rounded corners), and fix a calibrated direction
$P\in K_{n-p}(x_Q)$ with normal space $P^\perp\cong\R^{2p}$.
Let $\rho$ be a nonnegative Lipschitz density on a bounded transverse domain $\Omega\subset P^\perp$ with total mass
$\int_\Omega \rho = M$.
Then for every $N\in\N$ there exist \emph{calibrated} holomorphic complete intersections
$Y^1,\dots,Y^N\subset X$ such that:
\begin{enumerate}
\item[\textnormal{(i)}] (\textbf{Small-angle / graph control}) each $Y^a$ is $C^1$-close to an affine translate $P+t_a$ on $Q$
with $\sup_{y\in Q}\angle(T_yY^a,P)\le \varepsilon(h)$ and $\varepsilon(h)\to 0$ as $h\to 0$;
\item[\textnormal{(ii)}] (\textbf{Sliver masses}) the restricted pieces satisfy
\[
\Mass([Y^a]\llcorner Q)\ \le\ C\,\frac{M}{N}
\qquad\text{for all }a,
\]
and $\sum_a \Mass([Y^a]\llcorner Q)=M+o(1)$;
\item[\textnormal{(iii)}] \textbf{(Transverse measure approximation).}
The induced transverse measure
\[
\mu_N:=\sum_a \Mass([Y^a]\llcorner Q)\,\delta_{t_a}
\]
satisfies
\[
W_1(\mu_N,\rho\,dt)\le \tau(N,h),
\qquad
\tau(N,h)\xrightarrow[N\to\infty,\ h\to 0]{}0 .
\]

\end{enumerate}
\end{conjecture}

\begin{remark}[Why we ask for a smooth convex cell]\label{rem:sliver-cell-shape}
The ``sliver'' mechanism relies on being able to make \emph{both} the interior mass and the induced boundary slices small when a sheet translate
approaches the edge of the cell.  This behavior is clean in smooth convex models (e.g.\ balls), where plane sections shrink in a controlled way.
For sharp cubical cells, a plane section can have arbitrarily small $k$-volume while still having $O(h^{k-1})$ boundary on a face (thin long slices),
so additional geometry would be needed to keep boundary slices small.  Thus smooth convexity is a natural technical condition for any rigorous
sliver bookkeeping estimate.
One explicit alternative is a \emph{corner-exit / simplex} mechanism, combined with \emph{global vertex templates}: force each sliver footprint inside a cube
to meet only a fixed set of $k\!+\!1$ faces adjacent to a vertex and to have uniformly nondegenerate simplex shape, and choose the slivers from a fixed ordered template
anchored at each grid vertex.  This yields $a\lesssim v^{(k-1)/k}$ even in sharp cubes and also resolves the face-population/prefix obstruction for gluing;
see \REVMZ{Proposition~\ref{prop:vertex-template-face-edits}}.
\end{remark}


\subsection*{Sharp-cube variant: corner-exit slivers and global vertex templates (model)}

\begin{remark}[Status of the sharp-cube variant]\label{rem:sharp-cube-variant-status}
\REVMZ{\textbf{[Optional geometric intuition.]}}
This subsection is an optional geometric model meant to explain why vertex-anchored templates and corner-exit slivers
avoid one-sided face population and prefix mismatch.  None of the main proofs should \emph{depend} on this model
unless a later statement explicitly invokes it.
\end{remark}



\begin{remark}[Why corners (vertex stars) are the natural anchors]\label{rem:why-corners}
\REVMZ{\textbf{[Geometric intuition.]}}
Placing microstructure pieces near vertex stars ensures that every shared face sees comparable populations from both sides,
so mismatches become an $O(h)$ tail effect rather than an $O(1)$ obstruction.  This is the geometric reason the
corner-exit route is compatible with global gluing.
\REVMZ{See Proposition~\ref{prop:vertex-template-face-edits} for the detailed estimate.}
\end{remark}

\begin{remark}[Why templates should live at vertices (pan-vertex distribution)]
\label{rem:sharp-cube-why}
\REVMZ{\textbf{[Geometric intuition.]}}
If one concentrates all slivers in a cube $Q$ near a single vertex, then an interior face $F=Q\cap Q'$ can be populated on one side and essentially empty on the other,
creating a one-sided mismatch that is not a tail effect.
Moreover, even if both sides use the same \emph{cellwise} master template, it is not automatic that the pieces that actually meet a given face $F$ are the \emph{early}
pieces in the chosen prefix.

\smallskip\noindent
A clean way to remove both issues is to define templates at the \emph{grid vertices} and to distribute each cube's mass among its vertices.
Then any two cubes sharing a vertex $v$ use the same ordered geometric sequence of slivers anchored at $v$, so across every shared face the mismatch reduces to a
pure prefix-count difference at the shared vertices.
\end{remark}

\begin{definition}[Global vertex template (flat cubical model)]\label{def:vertex-template}

Fix a cubical grid in $\R^{2n}$ with mesh $h$ and vertex set $\Lambda:=(h\Z)^{2n}$, and fix a calibrated $(2n-2p)$--plane $P$.
For each vertex $v\in\Lambda$, fix an infinite ordered family of affine planes
\[
P_{v,a}\ :=\ P+v+t_{v,a},\qquad a\ge 1,
\]
with translation vectors $t_{v,a}\in P^\perp$ satisfying the following \emph{cellwise corner-exit} properties.
Let $Q$ be any cube of the grid containing $v$, set $k:=2n-2p$, and write
\[
E_{v,a}(Q)\ :=\ P_{v,a}\cap Q\ \subset\ Q.
\]
\begin{enumerate}
\item[\textnormal{(i)}] (\textbf{Corner localization}) there exists $c_0\in(0,1)$, independent of $h,v,Q,a$, such that
\[
E_{v,a}(Q)\ \subset\ B(v,c_0h)
\qquad\text{for every }a\ge 1.
\]
\item[\textnormal{(ii)}] (\textbf{Uniform corner-exit simplex type}) for each $Q\ni v$, every footprint $E_{v,a}(Q)$ is a $k$--simplex
which meets \emph{exactly} the same set of $k\!+\!1$ coordinate $(2n\!-\!1)$--faces of $Q$ through $v$ (the ``designated exit faces''),
and meets no other codimension--$1$ faces of $Q$.
Equivalently, for any face $F\subset\partial Q$ incident to $v$, either $E_{v,a}(Q)\cap F\neq\emptyset$ for all $a$,
or $E_{v,a}(Q)\cap F=\emptyset$ for all $a$.
\item[\textnormal{(iii)}] (\textbf{Equal / uniformly comparable face-slice masses}) for each $Q\ni v$ and each designated exit face $F\subset\partial Q$
met by $E_{v,a}(Q)$, the slice masses
\[
b_{v,a}(F;Q)\ :=\ \mathcal H^{k-1}\bigl(E_{v,a}(Q)\cap F\bigr)
\]
are independent of $a$ (or, more generally, satisfy
$c\,b_{v,1}(F;Q)\le b_{v,a}(F;Q)\le C\,b_{v,1}(F;Q)$ for all $a$ with constants $c,C$ independent of $h,v,Q$).
\end{enumerate}
We refer to $(P_{v,a})_{a\ge 1}$ as a \emph{global vertex template} for direction $P$.

\end{definition}

\begin{lemma}[Existence of vertex-template families in the flat cubical model]\label{lem:vertex-template-exists}
Fix a calibrated $k$--plane $P\subset\R^{2n}$ and a cube $Q$ of side length $h$.
Assume there exists \emph{one} translation $t_\star\in P^\perp$ for which the footprint $(P+v+t_\star)\cap Q$
is a $k$--simplex meeting exactly $k\!+\!1$ of the $(2n\!-\!1)$--faces adjacent to the vertex $v$
(and lying in $B(v,c_0h)$). Then, for all translations $t$ in a sufficiently small neighborhood of $t_\star$ in $P^\perp$,
the footprint $(P+v+t)\cap Q$ is a $k$--simplex with the same set of exit faces.
In particular, one can choose a countably infinite ordered family $(t_{v,a})_{a\ge 1}$ converging to $t_\star$
that satisfies items (i)--(ii) in Definition~\ref{def:vertex-template}.
\end{lemma}

\begin{proof}
The intersection of an affine $k$--plane with the fixed polytope $Q$ is a convex polytope in that plane,
whose combinatorial type changes only when the plane passes through a lower-dimensional face of $Q$.
Thus the set of translations $t$ for which the active set of face constraints is constant is open in $P^\perp$.
Under the hypothesis that $t_\star$ yields a simplex with the desired $k\!+\!1$ active faces, the same active-face pattern
persists for all $t$ in a small neighborhood of $t_\star$, giving the same simplex type and the same exit faces.
\end{proof}






















\begin{remark}[Supplying corner-exit template families for the direction net]\label{rem:corner-exit-direction-net}\label{rem:vertex-template-remarks}
The global activation/gluing bookkeeping (Theorem~\ref{thm:sliver-mass-matching-on-template} and \REVMZ{Proposition~\ref{prop:global-coherence-all-labels}})
is \emph{direction-by-direction}: one fixes a calibrated direction label $j$ and activates an ordered template by choosing only prefix lengths.
Thus, to run the corner-exit route in the holomorphic setting, it suffices to ensure the following for each direction label $j$
in the finite direction net used to approximate $m\beta$ on the mesh:
\begin{itemize}
\item \textbf{(Template existence)} in the local holomorphic chart for a cell $Q$, there is a complex reference plane $P_j$ and a supply of translation
parameters $t_{v,a}^{(j)}$ near each vertex $v$ so that the footprints $(P_j+v+t_{v,a}^{(j)})\cap Q$ are uniformly fat corner-exit simplices with a fixed
designated exit-face set (hence satisfy the geometric hypotheses of Proposition~\ref{prop:holomorphic-corner-exit-g1g2}),
and
\item \textbf{(Holomorphic realization)} these translated templates can be realized by disjoint holomorphic complete intersections on $Q$ with cell-scale single-sheet
graph control.
\end{itemize}

\smallskip\noindent
Lemma~\ref{lem:complex-corner-exit-template} provides a completely explicit complex corner-exit translation template in a coordinate cube, and
Lemma~\ref{lem:corner-exit-template-open} + Proposition~\ref{prop:corner-exit-template-net} provide the robust finite-net supply needed for the global scheme:
one can choose the direction dictionary/net used to approximate $\widehat\beta$ so that \emph{every} direction label admits a corner-exit translation template, with
constants uniform over the finite net.
In practice, one chooses the direction net \emph{inside} the open set of calibrated planes for which an analogous ``one-coordinate slanted inequality''
produces a corner simplex in $Q$ (after choosing the appropriate anchored vertex $v$ and designated faces among those incident to $v$).
Since the net is finite at each mesh scale, all geometric constants (fatness, locality radius $c_0$, and per-face comparability constants) may be taken uniform by
min/max over the finitely many labels.
\end{remark}

\begin{lemma}[Corner-exit simplex slices have optimal boundary scaling]\label{lem:cube-vertex-slice-boundary}

Let $Q=[0,h]^{2n}\subset\R^{2n}$ and fix $1\le k<2n$.
Let $E\subset Q$ be a $k$--dimensional simplex such that:
(i) $E\subset B(0,c_0h)$ for some fixed $c_0\in(0,1)$, and
(ii) exactly $k\!+\!1$ of the $(k\!-\!1)$--faces of $E$ lie in the $k\!+\!1$ coordinate $(2n\!-\!1)$--faces of $Q$ through $0$,
with all dihedral angles of $E$ bounded below by a fixed constant (uniform nondegeneracy).
Then there exists $C=C(k,c_0,\textnormal{nondeg})$ such that
\[
\mathcal H^{k-1}(\partial E)\ \le\ C\,\bigl(\mathcal H^{k}(E)\bigr)^{\frac{k-1}{k}}.
\]

\end{lemma}


\begin{proof}
Let $\Pi$ be the affine $k$--plane containing $E$.  By the uniform nondegeneracy assumption (dihedral angles bounded below),
there exists an affine isomorphism $A:\Pi\to\R^k$ whose distortion (operator norm and inverse norm) is bounded in terms of $k$ alone, such that
$A(E)=\Delta_s$ is a standard $k$--simplex of scale $s$ (i.e.\ affine-equivalent to $\{x\in\R^k: x_i\ge 0,\ \sum_{i=1}^k x_i\le s\}$).

For the standard simplex one computes explicitly
\(
\mathcal H^{k}(\Delta_s)=c_k\,s^k
\)
and
\(
\mathcal H^{k-1}(\partial\Delta_s)=c'_k\,s^{k-1}
\)
for dimensional constants $c_k,c'_k>0$.
Eliminating $s$ yields
\(
\mathcal H^{k-1}(\partial\Delta_s)\le C(k)\,\bigl(\mathcal H^k(\Delta_s)\bigr)^{(k-1)/k}.
\)
Applying the change-of-variables bounds under $A$ (which distort $k$-- and $(k-1)$--dimensional Hausdorff measures by at most a multiplicative factor depending only on $k$)
gives the stated inequality for $E$.
\end{proof}


\begin{proposition}[Vertex-template prefix lengths match local mass budgets (L2, cube model)]\label{prop:vertex-template-mass-matching}

Work in the setting of Definition~\ref{def:vertex-template} for a fixed cube $Q$, and suppose that the vertex templates have equal (or uniformly comparable) slice masses as in
Definition~\ref{def:vertex-template}\textnormal{(iii)}.  Assume further that the geometric templates are realized in $Q$ by holomorphic pieces with small-slope graph control,
so that Lemma~\ref{lem:sliver-stability}\textnormal{(i)} applies uniformly on $Q$.

\smallskip\noindent
Let $M_Q\ge 0$ be the target mass budget for this direction family in $Q$.  Choose any vertex splitting $M_{Q,v}\ge 0$ with
\(
\sum_{v\in\mathrm{Vert}(Q)} M_{Q,v}=M_Q
\)
(for instance the equal split $M_{Q,v}=2^{-d}M_Q$).
For each vertex $v\in\mathrm{Vert}(Q)$, let $\mu_{Q,v}>0$ denote the common mass scale in $Q$ for the $v$-anchored template pieces, i.e.
\[
\Mass([Y_{Q,v}^a]\llcorner Q)\ =\ \bigl(1+O(\varepsilon^2)\bigr)\,\mu_{Q,v}\qquad\text{uniformly in }a,
\]
with the implied constant independent of $Q,v,a$.
Define the prefix length by nearest-integer rounding
\[
N_{Q,v}\ :=\ \Bigl\lfloor \frac{M_{Q,v}}{\mu_{Q,v}}\Bigr\rceil .
\]
Then the realized total mass satisfies
\[
\sum_{v\in\mathrm{Vert}(Q)}\sum_{a=1}^{N_{Q,v}} \Mass([Y_{Q,v}^a]\llcorner Q)
\ =\ M_Q\ +\ O\!\left(\sum_{v}\mu_{Q,v}\right)\ +\ O(\varepsilon^2)\,M_Q.
\]
In particular, whenever $M_{Q,v}\gg \mu_{Q,v}$ (equivalently $N_{Q,v}\to\infty$), the relative error per vertex is
\(O(1/N_{Q,v})+O(\varepsilon^2)\).

\end{proposition}


\begin{proof}

Fix a vertex $v$.  By nearest-integer rounding,
\[
\Bigl|N_{Q,v}\,\mu_{Q,v}-M_{Q,v}\Bigr|\ \le\ \frac12\,\mu_{Q,v}.
\]
By the holomorphic small-slope graph control and Lemma~\ref{lem:sliver-stability}\textnormal{(i)}, each realized piece satisfies
\[
\Mass([Y_{Q,v}^a]\llcorner Q)\ =\ \bigl(1+\theta_{Q,v,a}\bigr)\,\mu_{Q,v},
\qquad |\theta_{Q,v,a}|\ \le\ C\,\varepsilon^2,
\]
with $C$ uniform in $Q,v,a$.
Therefore
\[
\sum_{a=1}^{N_{Q,v}}\Mass([Y_{Q,v}^a]\llcorner Q)
\ =\ N_{Q,v}\,\mu_{Q,v}\ +\ O(\varepsilon^2)\,N_{Q,v}\,\mu_{Q,v}.
\]
Using $N_{Q,v}\,\mu_{Q,v}=M_{Q,v}+O(\mu_{Q,v})$ from rounding, we obtain
\[
\Bigl|\sum_{a=1}^{N_{Q,v}}\Mass([Y_{Q,v}^a]\llcorner Q)-M_{Q,v}\Bigr|
\ \le\ \frac12\,\mu_{Q,v}\ +\ O(\varepsilon^2)\,M_{Q,v}\ +\ O(\varepsilon^2)\,\mu_{Q,v}.
\]
Absorbing the last term into the $O(\mu_{Q,v})$ contribution (since $\varepsilon\le 1$ in the regime of interest) yields
\[
\Bigl|\sum_{a=1}^{N_{Q,v}}\Mass([Y_{Q,v}^a]\llcorner Q)-M_{Q,v}\Bigr|
\ \le\ O(\mu_{Q,v})\ +\ O(\varepsilon^2)\,M_{Q,v}.
\]
Summing over the finitely many vertices $v\in\mathrm{Vert}(Q)$ and using $\sum_v M_{Q,v}=M_Q$ gives
\[
\sum_{v}\sum_{a=1}^{N_{Q,v}} \Mass([Y_{Q,v}^a]\llcorner Q)
\ =\ M_Q\ +\ O\!\left(\sum_{v}\mu_{Q,v}\right)\ +\ O(\varepsilon^2)\,M_Q.
\]
Finally, if $M_{Q,v}\gg \mu_{Q,v}$ then $M_{Q,v}\simeq N_{Q,v}\mu_{Q,v}$ and dividing the per-vertex estimate by $M_{Q,v}$ gives the relative error
\(O(1/N_{Q,v})+O(\varepsilon^2)\).

\end{proof}



\begin{proposition}[Vertex templates $\Rightarrow$ face-level $O(h)$ edit regime (hypothesis \textnormal{(iv)})]\label{prop:vertex-template-face-edits}

Work in the setting of Definition~\ref{def:vertex-template}, and fix one paired direction family.
For each cube $Q$ and each vertex $v\in\mathrm{Vert}(Q)$, let $N_{Q,v}\in\Z_{\ge 0}$ and suppose that inside $Q$ we realize the
vertex-prefix $\{P_{v,a}\}_{1\le a\le N_{Q,v}}$ by corresponding (disjoint) pieces, so that the face-slice boundary masses along a face
are indexed by the same order $a=1,2,\dots$.

Assume the \emph{slow-variation} bound holds at every shared vertex: for any two adjacent cubes $Q\sim Q'$ and any shared vertex
$v\in Q\cap Q'$,
\[
|N_{Q,v}-N_{Q',v}|\ \le\ C\,h\,\min\{N_{Q,v},N_{Q',v}\}.
\]
Assume moreover that the face-slice boundary masses of $v$-anchored pieces meeting a fixed interior face $F$ are uniformly comparable
in the index $a$ (with constant $\kappa$ independent of $h,Q,Q',v$), i.e.\ for each such $F$ and $v\in\mathrm{Vert}(F)$,
\[
b_{v,a}(F)\ \le\ \kappa\cdot \frac{1}{N_{\min}}\sum_{i=1}^{N_{\min}} b_{v,i}(F)
\qquad\text{for all }a>N_{\min},
\]
where $N_{\min}:=\min\{N_{Q,v},N_{Q',v}\}$ and $b_{v,a}(F)$ denotes the face-slice boundary mass on $F$ of the $a$-th $v$-anchored piece
(from the side where it is present).

Then for every interior interface face $F=Q\cap Q'$, the unmatched part of the boundary on $F$ satisfies the $O(h)$--fraction hypothesis
of Proposition~\ref{prop:prefix-template-coherence}:
there exists $\theta_F\lesssim h$ (depending only on $C,\kappa$ and dimension) such that
\[
\Mass(B_F^{\mathrm{un}})\ \le\ \theta_F\Bigl(\Mass(\partial S_Q\llcorner F)+\Mass(\partial S_{Q'}\llcorner F)\Bigr),
\]
where $B_F^{\mathrm{un}}$ is the unpaired (tail) part of the mismatch on $F$.

\end{proposition}


\begin{proof}

Fix an interior interface face $F=Q\cap Q'$.
By the corner localization property in Definition~\ref{def:vertex-template}\textnormal{(i)}, any piece anchored at a vertex
$v\notin\mathrm{Vert}(F)$ is supported in $B(v,c_0h)$, which does not intersect $F$.
Hence only pieces anchored at vertices $v\in\mathrm{Vert}(F)$ can contribute to the face-restricted boundaries
$\partial S_Q\llcorner F$ and $\partial S_{Q'}\llcorner F$, and therefore
\[
\Mass(B_F^{\mathrm{un}})\ \le\ \sum_{v\in\mathrm{Vert}(F)} \Mass\bigl(B_{F,v}^{\mathrm{un}}\bigr),
\]
where $B_{F,v}^{\mathrm{un}}$ denotes the unpaired (tail) mismatch on $F$ coming from the $v$-anchored prefixes.

Fix such a vertex $v\in\mathrm{Vert}(F)$ and, without loss of generality, assume $N_{Q,v}\ge N_{Q',v}$.
Set $N_{\min}:=N_{Q',v}$ and $r:=N_{Q,v}-N_{Q',v}$, and write $b_a(F):=b_{v,a}(F)$ for the corresponding ordered face-slice masses
from the $Q$-side (so that the $Q'$-side contributes only the prefix $a\le N_{\min}$).
Because both cubes activate prefixes of the \emph{same} vertex order at $v$, the paired part is $\{1,\dots,N_{\min}\}$ and the unpaired
part is the tail $\{N_{\min}+1,\dots,N_{\min}+r\}$, i.e.\ hypothesis \textnormal{(a)} of Lemma~\ref{lem:oh-face-edit-regime} holds.
The assumed uniform comparability of the face-slice masses gives hypothesis \textnormal{(b)} with constant $\kappa$, and the vertex-wise
slow-variation bound gives hypothesis \textnormal{(c)} with the same constant $C$.
Therefore Lemma~\ref{lem:oh-face-edit-regime} yields
\[
\sum_{a>N_{\min}} b_{v,a}(F)\ \le\ (\kappa C)\,h\sum_{a\le N_{Q,v}} b_{v,a}(F).
\]
Summing this bound over the finitely many vertices $v\in\mathrm{Vert}(F)$ and absorbing the fixed vertex-count into the constant
gives $\Mass(B_F^{\mathrm{un}})\le \theta_F\bigl(\Mass(\partial S_Q\llcorner F)+\Mass(\partial S_{Q'}\llcorner F)\bigr)$ with
$\theta_F\lesssim h$, as claimed.

\end{proof}



\begin{corollary}[Corner-exit vertex templates satisfy the activation hypotheses (iii)--(iv)]\label{cor:corner-exit-iii-iv}

Fix one direction label $j$ and work on a mesh-$h$ cubulation in the cube model of Definition~\ref{def:vertex-template}.
Suppose that for this label the following are implemented:
\begin{enumerate}
\item[\textnormal{(1)}] (\textbf{Holomorphic corner-exit manufacturing (L1)}) for each cube $Q$ and each vertex $v\in\mathrm{Vert}(Q)$,
the affine planes $\{P_{v,a}\}_{a\ge 1}$ {are realized in $Q$ by $\psi$--calibrated holomorphic pieces (no disjointness is assumed; one may repeat pieces and interpret repetition as multiplicity)}
$\{Y_{Q,v}^a\}_{a\ge 1}$ with uniform small-slope graph control, and each pair $(E_{Q,v}^a,Y_{Q,v}^a)$
satisfies the corner-exit face-control conclusions of Proposition~\ref{prop:holomorphic-corner-exit-L1}
(with vertex-star coherence as in Remark);
\item[\textnormal{(2)}] (\textbf{Local mass-budget matching (L2)}) for each cube $Q$, the prefix lengths $N_{Q,v}$ are chosen to match the local
vertex budgets as in Proposition~\ref{prop:vertex-template-mass-matching}, so that
\[
\sum_{v\in\mathrm{Vert}(Q)}\sum_{a=1}^{N_{Q,v}} \Mass([Y_{Q,v}^a]\llcorner Q)
\ =\ M_Q\ +\ O\!\left(\sum_{v}\mu_{Q,v}\right)\ +\ O(\varepsilon^2)\,M_Q;
\]
\item[\textnormal{(3)}] (\textbf{Slow variation of counts}) at shared vertices $v\in Q\cap Q'$ one has
$|N_{Q,v}-N_{Q',v}|\lesssim h\,\min\{N_{Q,v},N_{Q',v}\}$
(e.g.\ by \REVMZ{Lemma~\ref{lem:slow-variation-rounding}} applied to Lipschitz target budgets).
\end{enumerate}
Then, for this direction label $j$, the two nontrivial activation hypotheses \textnormal{(iii)}--\textnormal{(iv)} in
Theorem~\ref{thm:sliver-mass-matching-on-template} hold (after absorbing constants).
Moreover, if one prefers to express the activation via a single ordered per-cube prefix (rather than per-vertex prefixes),
one may implement the block-uniform coded interleaving of \REVMZ{Definitions~\ref{def:checkerboard-anchoring}--\ref{def:block-uniform-codes}}
and then use \REVMZ{Proposition~\ref{prop:checkerboard-face-oh-edit}} for the face-edit estimate.

\end{corollary}
\begin{proof}

\emph{Hypothesis \textnormal{(iii)}.}
For each cube $Q$, consider the disjoint family of holomorphic pieces
\[
\mathcal Y_Q\ :=\ \{\,Y_{Q,v}^a:\ v\in\mathrm{Vert}(Q),\ 1\le a\le N_{Q,v}\,\},
\qquad N_Q:=\#\mathcal Y_Q=\sum_{v\in\mathrm{Vert}(Q)}N_{Q,v}.
\]
Assumption \textnormal{(1)} provides the local realizability (existence, calibration, and disjointness) on $Q$,
with transverse parameters inherited from the fixed vertex-template ordering at each $v$.
Assumption \textnormal{(2)} gives the mass match on $Q$ in the quantitative form stated in
Proposition~\ref{prop:vertex-template-mass-matching}.
In particular, on the region where the \emph{many pieces} hypothesis \textnormal{(i)} of
Theorem~\ref{thm:sliver-mass-matching-on-template} holds (so $N_{Q,v}\to\infty$ and $\sum_v\mu_{Q,v}=o(M_Q)$),
and under the parameter regime $\varepsilon\to 0$, the right-hand side equals $M_Q+o(M_Q)$ as required.

\emph{Hypothesis \textnormal{(iv)}.}
Let $F=Q\cap Q'$ be an interior interface face.
By assumption \textnormal{(3)}, the slow-variation bound holds at each shared vertex $v\in\mathrm{Vert}(F)$.
Therefore Proposition~\ref{prop:vertex-template-face-edits} applies and yields that the unmatched boundary mass on $F$
is an $O(h)$ fraction of the total face boundary mass; this is exactly the face-edit regime \textnormal{(iv)} of
Theorem~\ref{thm:sliver-mass-matching-on-template}.
(Alternatively, under the block-uniform coded interleaving, one may invoke \REVMZ{Proposition~\ref{prop:checkerboard-face-oh-edit}}.)

With \textnormal{(iii)}--\textnormal{(iv)} verified for label $j$, Theorem~\ref{thm:sliver-mass-matching-on-template}
applies to this direction family.

\end{proof}


\begin{remark}\label{rem:L1-downstream-map}

For later proof-spine checks, Proposition~\ref{prop:holomorphic-corner-exit-L1} is used downstream only through the following interfaces:
\begin{enumerate}
\item Corollary~\ref{cor:corner-exit-iii-iv}: it supplies the holomorphic corner-exit slivers (with (G1-iff)/(G2) and the $L^1$ boundary-face mass control) needed to certify activation hypothesis \textnormal{(iii)} for the vertex-template program.
\item Proposition: it invokes Corollary~\ref{cor:corner-exit-iii-iv} label-by-label and then applies only the rounding/slow-variation and face-edit machinery; no additional geometric input beyond Proposition~\ref{prop:holomorphic-corner-exit-L1} enters there.
\end{enumerate}
Earlier forward references to Proposition~\ref{prop:holomorphic-corner-exit-L1} (e.g.\ Remark~\ref{rem:weighted-scaling} and Proposition~\ref{prop:corner-exit-template-net})
are parameter-synchronization notes rather than additional proof dependencies.
Constants in the $L^1$ bounds may depend on $(k,\Lambda,\varepsilon)$ as recorded in Proposition~\ref{prop:holomorphic-corner-exit-L1};
uniformity across labels is enforced by the finite direction net and the schedule in Remark~\ref{rem:weighted-scaling}.

\end{remark}





\begin{proposition}[Global coherence across all direction labels (B1, packaged)]\label{prop:global-coherence-all-labels}
Fix a mesh-$h$ cubulation by coordinate cubes $Q$ (subordinate to a holomorphic atlas) and let $\beta$ be a smooth closed strongly positive $(p,p)$-form.
Fix a small scale $\varepsilon_h\ll h$ and choose, in each chart, an $\varepsilon_h$--net of calibrated directions
$\{P_1,\dots,P_M\}\subset G_\C(n-p,n)$ together with uniform corner-exit translation templates as in Proposition~\ref{prop:corner-exit-template-net}.

\smallskip\noindent
Choose \emph{globally labeled} Lipschitz weights $w_i(x)$ against this dictionary (e.g.\ by the strongly convex simplex fit of
Lemma~\ref{lem:lipschitz-qp-weights} applied to $\widehat\beta(x)$ in local trivializations), and define per-cell target mass budgets
$M_{Q,i}\ge 0$ accordingly, with $\sum_i M_{Q,i}=M_Q$ and Lipschitz variation across neighbors.
For each label $i$, realize the corresponding corner-exit template holomorphically on each vertex star by applying
Proposition~\ref{prop:holomorphic-corner-exit-L1} (with vertex-star coherence as in Remark) to the template planes provided by
Proposition~\ref{prop:corner-exit-template-net}; this yields corner-exit holomorphic slivers with (G1-iff)/(G2) and equal/comparable per-piece masses
(hence Proposition~\ref{prop:vertex-template-mass-matching} applies).

\smallskip\noindent
Then one can choose integer counts $N_{Q,v,i}$ simultaneously for all $(Q,v,i)$ so that:
\begin{enumerate}
\item[\textnormal{(a)}] (\textbf{Local mass/barycenter accuracy}) for each cube $Q$ and label $i$ the realized mass in direction $i$ matches $M_{Q,i}$
up to the rounding error $O(1/N)+O(\varepsilon^2)$ from Proposition~\ref{prop:vertex-template-mass-matching};
\item[\textnormal{(b)}] (\textbf{Slow variation}) for each interior adjacency $Q\sim Q'$ and each shared vertex $v\in Q\cap Q'$, one has
$|N_{Q,v,i}-N_{Q',v,i}|\lesssim h\min\{N_{Q,v,i},N_{Q',v,i}\}$ on the region where $M_{Q,i}$ is not negligible (e.g.\ via \REVMZ{Lemma~\ref{lem:slow-variation-rounding}}
and the $0$--$1$ stability \REVMZ{Lemma~\ref{lem:slow-variation-discrepancy}});
\item[\textnormal{(c)}] (\textbf{Cohomology periods}) after clearing denominators by choosing $m$ and applying fixed-dimension discrepancy rounding
(\REVMZ{Lemma~\ref{lem:barany-grinberg}}), one can choose the integer activations so that the \emph{raw} current has the desired periods up to an error $<\tfrac14$
on a fixed integral cohomology basis; after applying the gluing correction with sufficiently small mass, the resulting \emph{closed} glued cycle has the
exact integral periods and hence the exact class $\mathrm{PD}(m[\gamma])$ in rational homology (Proposition~
%=========================================================
% Auxiliary propositions used in the proof (placed here to avoid forward references)
%=========================================================

\begin{proposition}[Integral cohomology constraints]\label{prop:cohomology-match}
Given $\epsilon>0$, by refining the cube decomposition and choosing the
integers $N_{Q,j}$ appropriately, one can achieve simultaneously for all
$\ell=1,\ldots,b$ that
\[
\biggl|\sum_Q S_Q(\Theta_\ell) - m\,I_\ell\biggr| < \tfrac14.
\]
Let $S:=\sum_Q S_Q$ and let $U_\epsilon$ be any integral $(2n-2p)$--current with $\partial U_\epsilon=\partial S$ and
\[
\Mass(U_\epsilon)\ <\ \min\Bigl\{\epsilon,\ \frac{1}{4\,\max_\ell\|\Theta_\ell\|_{C^0}}\Bigr\}.
\]
Then $T_\epsilon:=S-U_\epsilon$ is a closed integral cycle and
\[
\int_{T_\epsilon}\Theta_\ell\ =\ m\,I_\ell\qquad\text{for all }\ell=1,\dots,b.
\]
(Here $S_Q:=\sum_{j=1}^N\sum_{a=1}^{N_{Q,j}}[Y_{Q,j}^a]\llcorner Q$ is the local integral current built from the sheet pieces, and $S_Q(\Theta_\ell):=\int_{S_Q}\Theta_\ell=\sum_{j,a}\int_{Y_{Q,j}^a\cap Q}\Theta_\ell$.)

In particular, $[T_\epsilon]=\mathrm{PD}(m[\gamma])$ in $H_{2(n-p)}(X,\Z)/\mathrm{tors}$ (equivalently in $H_{2(n-p)}(X,\Q)$).
\end{proposition}

\begin{proof}
We make the fixed-dimension rounding in Substep~4.3 explicit.

\smallskip\noindent
\textbf{Step 1: Real targets and base--marginal decomposition.}

Fix a fine cube decomposition $\{Q\}$ (mesh $h$) and the associated families of sheet pieces
$\{Y_{Q,j}^a\}_{a\ge 1}$ produced in the preceding prefix--template construction.
For each pair $(Q,j)$ let $n_{Q,j}\in\R_{\ge 0}$ denote the \emph{real} target sheet count coming from the local bookkeeping.
Write
\[
n_{Q,j}=B_{Q,j}+a_{Q,j},\qquad  B_{Q,j}:=\lfloor n_{Q,j}\rfloor\in\Z_{\ge 0},\quad a_{Q,j}\in[0,1).
\]
Define the \emph{base} (integral) current and the \emph{marginal} sheet-current by
\[
S^{0}:=\sum_{Q,j}\sum_{a=1}^{B_{Q,j}}[Y_{Q,j}^a]\llcorner Q,
\qquad
Z_{Q,j}:=[Y_{Q,j}^{B_{Q,j}+1}]\llcorner Q.
\]
(If $a_{Q,j}=0$ we may set $Z_{Q,j}:=0$; then it plays no role in the fractional combination below.)
For any choice $\varepsilon_{Q,j}\in\{0,1\}$ set
\[
S(\varepsilon):=S^{0}+\sum_{Q,j}\varepsilon_{Q,j}\,Z_{Q,j},
\qquad
N_{Q,j}:=B_{Q,j}+\varepsilon_{Q,j}.
\]
Then $S(\varepsilon)$ is exactly the current obtained by taking the prefix of length $N_{Q,j}$ in each family.
The corresponding \emph{fractional} (real) combination is
\[
S^{\mathrm{frac}}:=S^{0}+\sum_{Q,j}a_{Q,j}\,Z_{Q,j}.
\]
Thus the rounding problem is to choose $\varepsilon_{Q,j}\in\{0,1\}$ so that the period error
$\int_{S(\varepsilon)}\Theta_\ell-\int_{S^{\mathrm{frac}}}\Theta_\ell$ is uniformly small for all $\ell$.


\smallskip\noindent
\textbf{Step 2: Set up the rounding vectors.}

For each $(Q,j)$ define the \emph{marginal contribution vector}
\[
v_{Q,j}:=\Bigl(\int_{Z_{Q,j}}\Theta_1,\ \dots,\ \int_{Z_{Q,j}}\Theta_b\Bigr)\in\R^b.
\]
By Theorem~\ref{thm:local-sheets}, each marginal sheet $Y_{Q,j}^{B_{Q,j}+1}\cap Q$ is a $C^1$ graph over its template plane on a region containing $Q$,
with slope $\lesssim \varepsilon$ and uniform $C^1$ control. In particular, there is a constant $C_0$ such that
\[
\Mass(Z_{Q,j})\le C_0\,h^{2(n-p)}
\qquad\text{and hence}\qquad
\|v_{Q,j}\|_{\ell^\infty}\le C_0\,h^{2(n-p)}\cdot \max_{\ell}\|\Theta_\ell\|_{C^0}.
\]
Choosing the mesh $h$ small (depending on $\max_\ell\|\Theta_\ell\|_{C^0}$ and $b$) we may assume
\[
\|v_{Q,j}\|_{\ell^\infty}\le \frac{1}{8b}\qquad\text{for all }(Q,j).
\]


\smallskip\noindent
\textbf{Step 3: Apply B\'ar\'any--Grinberg.}
Apply Lemma in dimension $d=b$ to the normalized vectors
$\widetilde v_{Q,j}:=(8b)\,v_{Q,j}$ (so $\|\widetilde v_{Q,j}\|_{\ell^\infty}\le 1$) with coefficients $a_{Q,j}$.
This yields choices $\varepsilon_{Q,j}\in\{0,1\}$ such that
\[
\Bigl\|\sum_{Q,j}(\varepsilon_{Q,j}-a_{Q,j})\,\widetilde v_{Q,j}\Bigr\|_{\ell^\infty}\le b.
\]
Undoing the normalization gives
\[
\Bigl\|\sum_{Q,j}(\varepsilon_{Q,j}-a_{Q,j})\,v_{Q,j}\Bigr\|_{\ell^\infty}\le \frac18.
\]

Equivalently, for each $\ell=1,\dots,b$,
\[
\Bigl|\int_{S(\varepsilon)}\Theta_\ell-\int_{S^{\mathrm{frac}}}\Theta_\ell\Bigr|
=\Bigl|\sum_{Q,j}(\varepsilon_{Q,j}-a_{Q,j})\int_{Z_{Q,j}}\Theta_\ell\Bigr|
\le \frac18.
\]
It therefore suffices to choose the continuous targets $\{n_{Q,j}\}$ (equivalently the fractional current $S^{\mathrm{frac}}$) so that
\[
\Bigl|\int_{S^{\mathrm{frac}}}\Theta_\ell - m I_\ell\Bigr|<\frac18\qquad\text{for all }\ell,
\]
which is the quantitative period-matching output of the local Carath{\'e}odory decomposition of $m\beta$
(Lemma~\ref{lem:caratheodory-general} together with the error bounds in the preceding construction, obtained by taking $\delta$ and $h$ sufficiently small).
Combining the two inequalities yields
\[
\Bigl|\sum_Q S_Q(\Theta_\ell)-mI_\ell\Bigr|
=\Bigl|\int_{S(\varepsilon)}\Theta_\ell- m I_\ell\Bigr|
<\frac14
\qquad (\ell=1,\dots,b).
\]
Set $S:=S(\varepsilon)=\sum_Q S_Q$.


\smallskip\noindent

\textbf{Step 4: Lock the periods via a small boundary correction.}
Choose an integral $(2n-2p)$--current $U_\epsilon$ with $\partial U_\epsilon=\partial S$ and
\[
\Mass(U_\epsilon)<\min\Bigl\{\epsilon,\ \frac{1}{4\max_\ell\|\Theta_\ell\|_{C^0}}\Bigr\},
\]
so that $|\int_{U_\epsilon}\Theta_\ell|<\frac14$ for all $\ell$.
(Existence of such $U_\epsilon$ is established in \emph{Step~5} below, using Proposition
and Corollary~\ref{cor:global-flat-weighted}.)

\smallskip\noindent
\textbf{Step 5: Construct the tiny-mass boundary correction $U_\epsilon$.}
Let $S$ be the raw microstructure current produced in Steps~1--3 and set $R:=\partial S$.
Define $\delta:=\mathcal F(R)$ using Definition~\ref{def:flat-norm}.
By Corollary~\ref{cor:global-flat-weighted} (together with the parameter schedule in \S\ref{sec:parameter-schedule}),
we have $\delta=\delta(h)\to 0$ as the mesh size $h\downarrow 0$.
Fix
\[
M_\Theta:=\max_\ell \|\Theta_\ell\|_{C^0}.
\]
Choose the mesh sufficiently fine so that
\[
\delta + C_X\,\delta^{\frac{k}{k-1}}
< \min\Bigl\{\epsilon,\ \frac{1}{4\,M_\Theta}\Bigr\}.
\]
Apply Proposition (with $T^{\mathrm{raw}}=S$) to obtain an integral $k$--current $R_{\mathrm{glue}}$ with
\[
\partial R_{\mathrm{glue}}=-R,\qquad \Mass(R_{\mathrm{glue}})\le \delta + C_X\,\delta^{\frac{k}{k-1}}.
\]
Set $U_\epsilon:=-R_{\mathrm{glue}}$. Then $\partial U_\epsilon=R=\partial S$, and $\Mass(U_\epsilon)=\Mass(R_{\mathrm{glue}})$ satisfies the bound required for Step~4.

Then $T_\epsilon:=S-U_\epsilon$ is a closed integral cycle. Since each $\Theta_\ell$ represents an integral cohomology class, the pairing $\int_{T_\epsilon}\Theta_\ell=\langle[\Theta_\ell],[T_\epsilon]\rangle$ lies in $\Z$.
For each $\ell$ we have
\[
\int_{T_\epsilon}\Theta_\ell=\int_{S}\Theta_\ell-\int_{U_\epsilon}\Theta_\ell,
\]
so the previous estimate implies $\bigl|\int_{T_\epsilon}\Theta_\ell-mI_\ell\bigr|<\frac12$.
Because $\int_{T_\epsilon}\Theta_\ell\in\Z$ and we just proved $|\int_{T_\epsilon}\Theta_\ell-mI_\ell|<\tfrac12$, we conclude $\int_{T_\epsilon}\Theta_\ell=mI_\ell$ for all $\ell$.
This identifies the Poincar\'e dual class of $T_\epsilon$ with $m\gamma$ (modulo torsion), as claimed.

\end{proof}


\begin{proposition}[Almost--calibration and global mass convergence for the glued cycles]\label{prop:almost-calibration}
Let $\psi$ be a smooth closed $(2n-2p)$--form with comass $\le 1$.
Let $S$ be an integral $(2n-2p)$--current built as a finite sum of local $\psi$--calibrated sheet pieces (so $\Mass(S)=\int_S\psi$).
Let $U_\epsilon$ be integral currents such that
\[
\partial U_\epsilon=\partial S,
\qquad
\Mass(U_\epsilon)\xrightarrow[\epsilon\to 0]{}0,
\]
for instance the gluing corrections $U_h$ constructed in Proposition (with $\epsilon\sim h$).
Define the closed integral cycles
\[
T_\epsilon := S-U_\epsilon,
\qquad
\partial T_\epsilon=0.
\]
Assume moreover that
\[
[T_\epsilon]=\mathrm{PD}(m[\gamma])\quad\text{in }H_{2(n-p)}(X,\Z)/\mathrm{tors}\ \text{for all }\epsilon
\]
(for instance by Proposition~\ref{prop:cohomology-match}).

Then:
\begin{enumerate}
\item[\textnormal{(i)}] \textbf{Exact cohomological pairing.}
Since $d\psi=0$ and $[T_\epsilon]$ is fixed, the number
\[
\int_{T_\epsilon}\psi
=\bigl\langle [T_\epsilon],[\psi]\bigr\rangle
=\bigl\langle \mathrm{PD}(m[\gamma]),[\psi]\bigr\rangle
=:c_0
\]
is independent of $\epsilon$.
\item[\textnormal{(ii)}] \textbf{Almost--calibration.}
Writing the calibration defect
\[
\Def_{\mathrm{cal}}(T_\epsilon)\ :=\ \Mass(T_\epsilon)-\int_{T_\epsilon}\psi\ \ge\ 0,
\]
one has the explicit estimate
\[
0\ \le\ \Def_{\mathrm{cal}}(T_\epsilon)\ \le\ 2\,\Mass(U_\epsilon)\ \xrightarrow[\epsilon\to 0]{}\ 0.
\]
\item[\textnormal{(iii)}] \textbf{Mass convergence.}
In particular,
\[
c_0\ \le\ \Mass(T_\epsilon)\ \le\ c_0+2\,\Mass(U_\epsilon),
\qquad\text{so}\qquad
\Mass(T_\epsilon)\to c_0.
\]
\end{enumerate}
\end{proposition}

\begin{proof}
By the comass bound, $|\int_{U_\epsilon}\psi|\le \Mass(U_\epsilon)$ and $\int_{T_\epsilon}\psi\le \Mass(T_\epsilon)$.
Since each sheet piece of $S$ is $\psi$--calibrated, $\Mass(S)=\int_S\psi$.

For \textnormal{(i)}, $d\psi=0$ and the homology hypothesis give
$\int_{T_\epsilon}\psi=\langle[T_\epsilon],[\psi]\rangle=\langle\mathrm{PD}(m[\gamma]),[\psi]\rangle=:c_0$.

For \textnormal{(ii)}, write
\[
\Def_{\mathrm{cal}}(T_\epsilon)
=\Mass(T_\epsilon)-\int_{S}\psi+\int_{U_\epsilon}\psi
\le \bigl(\Mass(S)+\Mass(U_\epsilon)\bigr)-\Mass(S)+\bigl|\int_{U_\epsilon}\psi\bigr|
\le 2\,\Mass(U_\epsilon),
\]
using $\Mass(T_\epsilon)\le \Mass(S)+\Mass(U_\epsilon)$ and $|\int_{U_\epsilon}\psi|\le \Mass(U_\epsilon)$.
The lower bound $\Def_{\mathrm{cal}}(T_\epsilon)\ge 0$ is $\int_{T_\epsilon}\psi\le \Mass(T_\epsilon)$.

Finally \textnormal{(iii)} follows from $\Mass(T_\epsilon)=\int_{T_\epsilon}\psi+\Def_{\mathrm{cal}}(T_\epsilon)=c_0+\Def_{\mathrm{cal}}(T_\epsilon)$.
\end{proof}

\ref{prop:cohomology-match}).
\end{enumerate}
Consequently, for each label $i$ the activation hypotheses (iii)--(iv) in Theorem~\ref{thm:sliver-mass-matching-on-template} hold (by Corollary~\ref{cor:corner-exit-iii-iv}),
and summing the resulting per-label flat-norm mismatch bounds yields $\mathcal F(\partial T^{\mathrm{raw}})=o(m)$ under the parameter regime of
Remark~\ref{rem:weighted-scaling}.
\end{proposition}

\begin{proof}

{
This proposition is a \emph{combinatorial/rounding and bookkeeping} step: all analytic holomorphic-realization
constraints are delegated to the cited local template realization inputs (notably
Proposition~\ref{prop:holomorphic-corner-exit-L1} and Proposition~\ref{prop:finite-template}).

\smallskip\noindent
\textbf{Step 1: Fix a finite direction/template dictionary.}
By Proposition~\ref{prop:corner-exit-template-net} we may choose, in each chart, a finite
$\varepsilon_h$--net $\{P_1,\dots,P_M\}$ of calibrated directions together with a finite list of translation templates per direction,
with uniform separation and local face-compatibility data (the ``dictionary'').

\smallskip\noindent
\textbf{Step 2: Choose globally labeled real weights and per-cell target budgets.}
Apply Lemma~\ref{lem:lipschitz-qp-weights} in local trivializations to the cone field $\beta$ to obtain
Lipschitz weights $w_i(\cdot)\ge 0$ with $\sum_i w_i(\cdot)=1$ such that
$\beta$ is approximated at scale $h$ by the convex combination of the dictionary rays.
Define the per-cell target budgets by
\[
M_{Q,i}:=\int_Q w_i(x)\,\langle\beta(x),\psi\rangle\,dV,
\qquad \sum_i M_{Q,i}=M_Q,
\]
and note that the Lipschitz control on $w_i$ implies the neighbor variation asserted in the statement.

\smallskip\noindent
\textbf{Step 3: Integer rounding with slow variation across neighbors.}
Fix the large integer parameter $N$ from the template net.
Using \REVMZ{Lemma~\ref{lem:slow-variation-rounding}} and \REVMZ{Lemma~\ref{lem:slow-variation-discrepancy}}, round the real budgets
$\{M_{Q,i}\}$ to integer multiplicities $\{N_{Q,v,i}\}$ at each vertex $v$ (for each label $i$) so that:
(i) the rounded masses approximate $M_{Q,i}$ up to $O(1/N)+O(\varepsilon_h^2)$,
and (ii) adjacent cubes differ by a controlled amount whenever the relevant multiplicities are non-negligible.
This yields items \textnormal{(a)} and \textnormal{(b)}.

\smallskip\noindent
\textbf{Step 4: Holomorphic realization of each label on vertex stars and coherence on overlaps.}
For each label $i$ and each vertex star, apply Proposition~\ref{prop:holomorphic-corner-exit-L1}
(using the coherence guarantee in Remark) to realize the corresponding finite
translation template holomorphically with the integer multiplicities $N_{Q,v,i}$.
By Proposition~\ref{prop:vertex-template-mass-matching}, the realized pieces have equal/comparable per-piece masses
and satisfy the local sliver conditions needed for later face-gluing.

\smallskip\noindent
\textbf{Step 5: Period correction and exact homology class.}
Choose the cohomology clearing integer $m$ and perform the fixed-dimension discrepancy rounding
(\REVMZ{Lemma~\ref{lem:barany-grinberg}}) to achieve item \textnormal{(c)}.
Then Proposition~\ref{prop:cohomology-match} upgrades the approximate periods to exact integral periods after adding a
small-mass correction, yielding the claimed rational homology class.

\smallskip\noindent
\textbf{Step 6: Face mismatch control (flat-norm gluing).}
Fix a label $i$ and fix once and for all a finite, ordered list of translation vectors
$\{t_{i,\ell}\}_{\ell=1}^{L_i}\subset \widetilde\Pi_i^\perp$ (a separated net in the transverse ball of radius $\asymp h$,
with mutual spacing $\gtrsim s$).
For each cube $Q$ and each vertex $v\in \mathrm{Vert}(Q)$, realize the local translated templates using the \emph{first}
$N_{Q,v,i}$ translation vectors $t_{i,1},\dots,t_{i,N_{Q,v,i}}$.
This canonical choice makes the induced transverse measures on faces explicit and comparable:

\smallskip\noindent
for an interior face $F=Q\cap Q'$ and label $i$, define the discrete measures
\[
\mu_{Q\to F,i}:=\sum_{v\in \mathrm{Vert}(F)}\ \sum_{\ell=1}^{N_{Q,v,i}}\delta_{t_{i,\ell}},
\qquad
\mu_{Q'\to F,i}:=\sum_{v\in \mathrm{Vert}(F)}\ \sum_{\ell=1}^{N_{Q',v,i}}\delta_{t_{i,\ell}}.
\]
By construction, $\mu_{Q\to F,i}$ and $\mu_{Q'\to F,i}$ share the same atoms $t_{i,\ell}$ for the common part
$\ell\le \min\{N_{Q,v,i},N_{Q',v,i}\}$ at each shared vertex $v$, hence the \emph{equal--mass part} matches at zero
transport cost.  Consequently one may take, in Proposition~\ref{prop:transport-flat-glue},
\[
\tau_{F,i}=0,
\qquad
\Delta_{F,i}\le \sum_{v\in \mathrm{Vert}(F)}\bigl|N_{Q,v,i}-N_{Q',v,i}\bigr|.
\]
Item \textnormal{(b)} (slow variation of the integers $N_{Q,v,i}$) gives a quantitative bound on $\Delta_{F,i}$, so the
slab term $h\,s^{k-1}\Delta_{F,i}$ is $o(m)$ after summing over faces under the stated $(h,s,m)$ scaling regime.

\smallskip\noindent
Finally, Lemma~\ref{lem:face-trace-decomposition} upgrades these discrete measures to the actual face traces of the
holomorphic sheet currents, with only the analytic error $E_{Q,F}$ of size
$O(\varepsilon_{\mathrm{hol}} s^{k-1}\Mass(\mu_{Q\to F,i}))$ on each face.
Applying Proposition~\ref{prop:transport-flat-glue} label-by-label and summing over all faces yields the global flat-norm
gluing estimate claimed in item \textnormal{(d)}.
}

\end{proof}



\noindent\textbf{closure of the ``simultaneous matching'' hinge.}
For each interior face $F=Q\cap Q'$, Proposition~\ref{prop:global-coherence-all-labels} furnishes a \emph{common prefix length} $N_F$
(and hence equal integer masses on both sides after the prefix-edit/balancing step).
With this $N_F$, Proposition~\ref{prop:integer-transport} \emph{constructs} the integer-weighted measures
$\mu_{Q\to F}$ and $\mu_{Q'\to F}$ and gives the quantitative $W_1$ bound needed in the transport/glue estimate.
Because the construction is facewise and uses the globally coherent master template, the hypothesis (c) in
Proposition~\ref{prop:transport-flat-glue-weighted} is discharged \emph{simultaneously for all interior faces}.



\begin{proof}

\noindent\textbf{Dependency packaging (no new axioms).}
This statement is a \emph{packaging} of previously proved components; the role of this proof is to
make the dependency chain explicit and eliminate any hidden ``assume-as-needed'' steps.

\smallskip\noindent
\textbf{(1) Existence of globally labeled weights.}
The Lipschitz weights $w_i(x)$ are produced by Lemma~\ref{lem:lipschitz-qp-weights} applied to the local coefficient vector
$\widehat\beta(x)$ in the chosen trivializations; no additional hypothesis is introduced here.

\smallskip\noindent
\textbf{(2) Local holomorphic realizability for each label.}
For each calibrated direction label $i$, the template planes and their coherence data come from
Proposition~\ref{prop:corner-exit-template-net} and Remark.
Applying Proposition~\ref{prop:holomorphic-corner-exit-L1} in each vertex star yields the corresponding holomorphic corner-exit slivers
with the uniform geometry properties (G1-iff)/(G2) and the per-piece comparability needed to invoke
Proposition~\ref{prop:vertex-template-mass-matching}.
No extra ``activation'' assumption is made: the hypotheses required by Proposition~\ref{prop:holomorphic-corner-exit-L1}
are exactly those ensured by the template-net construction together with the parameter regime fixed in Remark~\ref{rem:weighted-scaling}.

\smallskip\noindent
\textbf{(3) Integer rounding with slow variation.}
Given the per-cell mass budgets $M_{Q,i}$, the simultaneous choice of integer counts $N_{Q,v,i}$
satisfying \textnormal{(a)} and \textnormal{(b)} is precisely the content of \REVMZ{Lemma~\ref{lem:slow-variation-rounding}}
together with the $0$--$1$ stability \REVMZ{Lemma~\ref{lem:slow-variation-discrepancy}}, applied label-by-label.

\smallskip\noindent
\textbf{(4) Period control and gluing.}
The cohomology/period statement \textnormal{(c)} is obtained by the fixed-dimension discrepancy rounding
\REVMZ{Lemma~\ref{lem:barany-grinberg}} and the class-identification Proposition~\ref{prop:cohomology-match}.
The passage from the raw current to a closed glued cycle with the exact integral periods uses the gluing correction mechanism
(\REVMZ{Proposition~\ref{prop:glue-gap}}) with the small-boundary input supplied by the global flat-weighted estimates
(Corollary~\ref{cor:global-flat-weighted} under Remark~\ref{rem:weighted-scaling}).

\smallskip\noindent
\textbf{(5) Activation hypotheses (iii)--(iv) and boundary mismatch.}
Finally, we check that the sliver/template activation hypotheses \textnormal{(iii)}--\textnormal{(iv)} hold for each label
is exactly Corollary~\ref{cor:corner-exit-iii-iv}, once the local corner-exit realizability from (2) is in place.
Summing the per-label flat-norm mismatch bounds is then a direct application of Corollary~\ref{cor:global-flat-weighted},
yielding $\mathcal F(\partial T^{\mathrm{raw}})=o(m)$ in the parameter regime of Remark~\ref{rem:weighted-scaling}.

\end{proof}



\begin{corollary}[Flat boundary of the raw current in the weighted scaling regime]\label{cor:raw-boundary-flat-small}\label{prop:global-flat-control}

Assume the hypotheses of Proposition~\ref{prop:global-coherence-all-labels} and work in the parameter regime of
Remark~\ref{rem:weighted-scaling} (in particular, $h\sim c\,\mhol^{-1/2}$ and the corner-exit/graph parameters are synchronized as there).
Then the associated raw current $T^{\mathrm{raw}}$ satisfies
\[
\mathcal F(\partial T^{\mathrm{raw}})=o(m)\qquad\text{as }h\downarrow 0\ \ (\text{equivalently }\mhol\to\infty).
\]
Equivalently, for every $\eta>0$ there exists $h_0>0$ such that for all sufficiently small mesh sizes $h<h_0$,
\[
\mathcal F(\partial T^{\mathrm{raw}})\le \eta\,m.
\]


In particular, for fixed cohomology multiplier $m$, any bound of the form $\mathcal{F}(\partial T^{\mathrm{raw}})=o(m)$ yields $\mathcal{F}(\partial T^{\mathrm{raw}})\to 0$ as $h\downarrow 0$.


\end{corollary}


\begin{proof}

By Proposition~\ref{prop:global-coherence-all-labels}, each interior interface mismatch $B_F$ fits the weighted translation model
of Corollary~\ref{cor:global-flat-weighted} with a uniform displacement control $\Delta_F\lesssim h^2$.
Moreover, the face-level $O(h)$ edit regime required there is ensured by the local face-edit estimates
(e.g.\ Proposition~\ref{prop:vertex-template-face-edits} for vertex-prefix activation, or its checkerboard analogue),
so Corollary~\ref{cor:global-flat-weighted} gives
\[
\mathcal F\!\left(\partial T^{\mathrm{raw}}\right)
\ \lesssim\ \varrho\,h^2\sum_Q\ \sum_{a\in\mathcal S(Q)} m_{Q,a}^{\frac{k-1}{k}},
\qquad k:=2n-2p.
\]
Remark~\ref{rem:weighted-scaling} identifies the local packing bound in each cell (coming from the holomorphic corner-exit realization)
and converts the right-hand side into the global scaling estimate
\[
\mathcal F(\partial T^{\mathrm{raw}})
\ \lesssim\ m^{\frac{k-1}{k}}\,h^{\,2-\frac{2n}{k}}\;\varepsilon^{-\frac{2p}{k}}.
\]
At the intrinsic Bergman cell size $h\sim \mhol^{-1/2}$, one can record the indicative scaling {
\[
\frac{\mathcal F(\partial T^{\mathrm{raw}})}{m}\ \lesssim\ m^{-1/k}\,\mhol^{-\left(1-\frac{n}{k}\right)}\;\varepsilon^{-\frac{2p}{k}}.
\]
}
{We emphasize again: $m$ is fixed in the SYR definition; no $m\to\infty$ limit is taken. Smallness is obtained by taking $\delta,\varepsilon\to0$ and refining the mesh (equivalently choosing $\mhol$ large enough for the local realization scale), as quantified in \REVMZ{Lemma~\ref{lem:bergman-control}}.}

\end{proof}

\begin{remark}[Making the ``prefix-balanced face population'' explicit]
The previous proposition treats each vertex template separately.
If one prefers a \emph{single} global ordered template whose prefixes automatically populate every interior face in a balanced way, one can interleave the vertex templates
by a deterministic block scheme (a ``vertex-code'' ordering) and align the vertex anchoring across the grid by a checkerboard parity rule.
This removes the possibility that the $F$-hitting pieces concentrate in a tail of the master order.
See \REVMZ{Proposition~\ref{prop:checkerboard-face-oh-edit}} below.
\end{remark}

\begin{definition}[Cubical grid parity and checkerboard vertex anchoring]\label{def:checkerboard-anchoring}
Fix $d\ge 2$ and mesh $h>0$ and index cubes by $g\in\Z^d$ via
\(
Q_g:=\prod_{\ell=1}^d[g_\ell h,(g_\ell+1)h].
\)
Define the parity vector $\pi(g)\in\{0,1\}^d$ by $\pi(g)_\ell:=g_\ell\bmod 2$, and let $\oplus$ denote bitwise XOR.
For a vertex-code $u\in\{0,1\}^d$, define the anchored vertex of $Q_g$ by
\[
v_g(u)\ :=\ \bigl(g+(u\oplus\pi(g))\bigr)h\ \in\ \R^d,
\]
so $u$ selects a cube-vertex in a checkerboard-consistent way across neighbors.
\end{definition}

\begin{definition}[Block-uniform vertex-code sequence]\label{def:block-uniform-codes}
Let $\mathcal V:=\{0,1\}^d$ and fix any bijection $\sigma:\{1,\dots,2^d\}\to\mathcal V$.
Define an infinite sequence $(u_a)_{a\ge 1}\subset\mathcal V$ by repeating $\sigma$ in blocks:
\[
u_{b\cdot 2^d+r}\ :=\ \sigma(r)\qquad (b\ge 0,\ 1\le r\le 2^d).
\]
\end{definition}

\begin{lemma}[Prefix discrepancy for block-uniform codes]\label{lem:prefix-discrepancy}\label{lem:template-vertex-control}
Let $S\subset\mathcal V$ and define
\(
A_S(N):=\#\{1\le a\le N:\ u_a\in S\}.
\)
Then for all $N\ge 1$,
\[
\Bigl|A_S(N) - \frac{|S|}{2^d}\,N\Bigr|\ \le\ 2^d,
\]
and for all $N,N'\ge 1$,
\[
|A_S(N)-A_S(N')|\ \le\ \frac{|S|}{2^d}\,|N-N'| + 2^{d+1}.
\]
\end{lemma}


\begin{proof}
Write $N=q\,2^d+r$ with $q\in\Z_{\ge 0}$ and $0\le r<2^d$.  By Definition~\ref{def:block-uniform-codes}, each complete block of
length $2^d$ contains each code in $\mathcal V$ exactly once, so each complete block contributes exactly $|S|$ hits.  Hence
\[
A_S(N)\ =\ q\,|S| + A_S(r),
\]
where $A_S(r):=\#\{1\le a\le r:\ u_a\in S\}$ counts hits in the initial segment of one block.  In particular
$0\le A_S(r)\le \min\{r,|S|\}\le 2^d$.
Since $\frac{|S|}{2^d}N=q|S|+\frac{|S|}{2^d}r$, we obtain
\[
\Bigl|A_S(N)-\frac{|S|}{2^d}N\Bigr|
\ =\ \Bigl|A_S(r)-\frac{|S|}{2^d}r\Bigr|
\ \le\ A_S(r)+\frac{|S|}{2^d}r
\ \le\ 2^d+|S|
\ \le\ 2^{d+1}.
\]
For the Lipschitz bound, write
\[
A_S(N)-A_S(N')\ =\ \Bigl(A_S(N)-\frac{|S|}{2^d}N\Bigr) - \Bigl(A_S(N')-\frac{|S|}{2^d}N'\Bigr)
\ +\ \frac{|S|}{2^d}(N-N'),
\]
and use the previous estimate for each bracketed term to get
\[
|A_S(N)-A_S(N')|
\ \le\ \frac{|S|}{2^d}|N-N'| + 2^{d+1}.
\]
\end{proof}


\begin{lemma}[Two-sided face population is automatic under checkerboarding]\label{lem:two-sided-face-pop}
Fix a coordinate direction $\ell\in\{1,\dots,d\}$ and an interior interface face $F:=Q_g\cap Q_{g+e_\ell}$.
Let $S_{g,\ell}^+\subset\mathcal V$ be the set of codes whose anchored vertex in $Q_g$ lies on the positive $\ell$-face of $Q_g$, and let
$S_{g+e_\ell,\ell}^-\subset\mathcal V$ be the set of codes whose anchored vertex in $Q_{g+e_\ell}$ lies on the negative $\ell$-face of $Q_{g+e_\ell}$ (the same hyperplane).
Then $S_{g,\ell}^+=S_{g+e_\ell,\ell}^-$ and hence, for every $N$,
\[
\{a\le N:\ v_g(u_a)\in F\}\ =\ \{a\le N:\ v_{g+e_\ell}(u_a)\in F\}.
\]
\end{lemma}

\begin{proof}
By Definition~\ref{def:checkerboard-anchoring}, the anchored vertex $v_g(u)$ lies on the \emph{positive} $\ell$-face of $Q_g$
if and only if the $\ell$-th coordinate of $g+(u\oplus\pi(g))$ equals $g_\ell+1$, i.e.
\[
(u\oplus\pi(g))_\ell=1.
\]
Since $\pi(g+e_\ell)=\pi(g)\oplus e_\ell$, we have
\[
(u\oplus\pi(g+e_\ell))_\ell \ =\ (u\oplus\pi(g)\oplus e_\ell)_\ell \ =\ (u\oplus\pi(g))_\ell\oplus 1,
\]
so $(u\oplus\pi(g))_\ell=1$ if and only if $(u\oplus\pi(g+e_\ell))_\ell=0$.
But $(u\oplus\pi(g+e_\ell))_\ell=0$ is exactly the condition that $v_{g+e_\ell}(u)$ lies on the \emph{negative} $\ell$-face
of $Q_{g+e_\ell}$, which is the same hyperplane $F=Q_g\cap Q_{g+e_\ell}$.  Therefore $S_{g,\ell}^+=S_{g+e_\ell,\ell}^-$.
The equality of the index-sets for every $N$ follows immediately from the definition of $A_S(N)$.
\end{proof}



\begin{proposition}[Checkerboard corner assignment implies a face-level $O(h)$ edit regime]\label{prop:checkerboard-face-oh-edit}\label{prop:template-edits}
Fix $d\ge 2$ and a cubical grid $(Q_g)$ of mesh $h>0$.
Assume the ordered sliver activation in each cube $Q_g$ uses the block-uniform code sequence $(u_a)_{a\ge 1}$
(Definition~\ref{def:block-uniform-codes}), anchored by the checkerboard vertex rule $a\mapsto v_g(u_a)$
(Definition~\ref{def:checkerboard-anchoring}).
Assume the following geometric features hold uniformly for the activated slivers in each cube:
\begin{enumerate}
\item[\textnormal{(G1)}] (\textbf{Locality}) For an interior face $F=Q_g\cap Q_{g+e_\ell}$ and an index $a$,
the boundary slice $\partial([Y_g^a]\llcorner Q_g)\llcorner F$ is nonzero if and only if $v_g(u_a)\in F$,
and in that case it is supported in a patch of diameter $\lesssim h$ near $v_g(u_a)$.
\item[\textnormal{(G2)}] (\textbf{Comparable face mass}) There exist constants $0<c_0\le C_0$ and face-scale parameters $b_g\ge 0$
such that for every interior face $F$ and every $a$ with $v_g(u_a)\in F$,
\[
c_0\,b_g\ \le\ \Mass\!\bigl(\partial([Y_g^a]\llcorner Q_g)\llcorner F\bigr)\ \le\ C_0\,b_g.
\]
\end{enumerate}
Let $F=Q_g\cap Q_{g+e_\ell}$ be an interior face and let $N:=N_g$, $N':=N_{g+e_\ell}$ be the chosen prefix lengths on the two sides,
with $N_{\min}:=\min\{N,N'\}$.
Assume $N_{\min}\ge 2^{d+3}$ (in particular this holds in the regime $N_{\min}\gtrsim h^{-1}$ for $h\ll 1$).
Then the unmatched boundary mass on $F$ coming from tail indices $\{N_{\min}+1,\dots,\max\{N,N'\}\}$ satisfies
\[
\Mass(B_F^{\mathrm{un}})\ \le\ C\left(\frac{|N-N'|}{N_{\min}}+\frac{2^d}{N_{\min}}\right)
\Bigl(\Mass(\partial S_{Q_g}\llcorner F)+\Mass(\partial S_{Q_{g+e_\ell}}\llcorner F)\Bigr),
\]
with $C$ depending only on $(d,c_0,C_0)$.
In particular, if $|N-N'|\le \theta\,N_{\min}$ with $\theta\lesssim h$ and $N_{\min}\gtrsim h^{-1}$, then
\[
\Mass(B_F^{\mathrm{un}})\ \le\ C'\,h\,
\Bigl(\Mass(\partial S_{Q_g}\llcorner F)+\Mass(\partial S_{Q_{g+e_\ell}}\llcorner F)\Bigr),
\]
so the $O(h)$ face-edit regime (item \textnormal{(iv)} in Theorem~\ref{thm:sliver-mass-matching-on-template}) holds.
\end{proposition}


\begin{proof}
Let $S\subset\mathcal V=\{0,1\}^d$ be the set of codes whose anchored vertex in $Q_g$ lies on the interface face
$F=Q_g\cap Q_{g+e_\ell}$.  Then $|S|=2^{d-1}$.
By Lemma~\ref{lem:two-sided-face-pop}, the set of indices $\{a\le N:\ v_g(u_a)\in F\}$ agrees with
$\{a\le N:\ v_{g+e_\ell}(u_a)\in F\}$ for every $N$, so the only unmatched boundary contributions on $F$ come from those
tail indices $a\in(N_{\min},N_{\max}]$ with $u_a\in S$, where $N_{\max}:=\max\{N,N'\}$.

\smallskip\noindent\emph{Counting unmatched tail indices.}
By Lemma~\ref{lem:prefix-discrepancy} applied to the set $S$,
\[
\#\{N_{\min}<a\le N_{\max}:\ u_a\in S\}
\ =\ |A_S(N_{\max})-A_S(N_{\min})|
\ \le\ \frac{|S|}{2^d}|N-N'|+2^{d+1}
\ =\ \frac12|N-N'|+2^{d+1}.
\]
Each such unmatched index contributes at most $C_0\,b_g$ (or $C_0\,b_{g+e_\ell}$) to the boundary mass on the side where it appears,
by (G2).  Hence
\[
\Mass(B_F^{\mathrm{un}})
\ \le\ C_0\Bigl(\tfrac12|N-N'|+2^{d+1}\Bigr)\,(b_g+b_{g+e_\ell}).
\]

\smallskip\noindent\emph{Lower bound for the total activated boundary mass on $F$.}
Again by Lemma~\ref{lem:prefix-discrepancy},
\[
A_S(N_{\min})\ \ge\ \frac{|S|}{2^d}N_{\min}-2^{d+1}\ =\ \frac12 N_{\min}-2^{d+1}.
\]
Since $N_{\min}\ge 2^{d+3}$, the right-hand side is $\ge \frac14 N_{\min}$.
Each of these $A_S(N_{\min})$ indices appears on \emph{both} sides of $F$, and by (G2) contributes at least $c_0 b_g$ and
at least $c_0 b_{g+e_\ell}$ to $\Mass(\partial S_{Q_g}\llcorner F)$ and $\Mass(\partial S_{Q_{g+e_\ell}}\llcorner F)$ respectively.
Therefore,
\[
\Mass(\partial S_{Q_g}\llcorner F)+\Mass(\partial S_{Q_{g+e_\ell}}\llcorner F)
\ \ge\ c_0\,A_S(N_{\min})\,(b_g+b_{g+e_\ell})
\ \ge\ \frac{c_0}{4}\,N_{\min}\,(b_g+b_{g+e_\ell}).
\]

\smallskip\noindent\emph{Conclusion.}
Combining the previous two displays yields
\[
\Mass(B_F^{\mathrm{un}})
\ \le\ C\left(\frac{|N-N'|}{N_{\min}}+\frac{2^d}{N_{\min}}\right)
\Bigl(\Mass(\partial S_{Q_g}\llcorner F)+\Mass(\partial S_{Q_{g+e_\ell}}\llcorner F)\Bigr),
\]
with $C$ depending only on $(d,c_0,C_0)$ (absorbing fixed powers of $2$ into the constant), as claimed.
The final $O(h)$ specialization follows immediately under $|N-N'|\le \theta N_{\min}$ with $\theta\lesssim h$ and $N_{\min}\gtrsim h^{-1}$.
\end{proof}

\begin{remark}[Interpretation of the $O(h)$ template edits]\label{rem:template-edits-remark}
Proposition~\ref{prop:template-edits} and Lemma~\ref{lem:template-edits-oh} are used only to ensure that face-level discrepancies can be absorbed into the transport/gluing budget. No additional geometric input is hidden here: the estimates are purely combinatorial (prefix counts and rounding) plus the fixed-scale template smoothness.
\end{remark}





\begin{remark}[Rounded cubes]\label{rem:smooth-cells}
For the combinatorics of Substep~4.2 (adjacency graph, faces, cochain constraints), it is convenient to work with a cubulation.
For the sliver bookkeeping, it is convenient to replace each sharp cube by a \emph{rounded cube} of comparable diameter $h$ whose boundary is $C^2$
and uniformly convex with principal curvatures pinched at scale $h$ (so Lemma~\ref{lem:uniformly-convex-slice-boundary} applies).
This rounding changes only constants and does not change the adjacency graph.
\end{remark}

\begin{remark}[Where the remaining analytic difficulty really lives]\label{rem:bergman-not-enough}
\REVMZ{\textbf{[Key technical point.]}}
It is tempting to argue that Bergman kernel localization or Tian--Yau--Zelditch universality \REVMZ{\cite{Tian90,Zelditch98,Catlin99}} alone forces the desired
face-incidence and per-face boundary-mass properties of slivers.  However, \emph{pointwise decay of a holomorphic section does not
localize its exact zero set} in the strong sense needed for gluing.

\smallskip\noindent
The correct ``critical checkpoint'' is instead the following: on a \emph{whole cell} $Q$ (not just infinitesimally near one point),
the defining holomorphic map must be \emph{uniformly $C^1$-close} to a fixed linear model so that the zero set in $Q$ is a \emph{single sheet}
graph over the intended template plane.  Once this global-graph property holds, the corner-exit geometry immediately forces
(G1-iff) and (G2) (exit-face stability and per-face mass comparability), and the remaining face bookkeeping is purely combinatorial.
\end{remark}

\begin{lemma}[Global quantitative graph lemma (contraction criterion)]\label{lem:global-graph-contraction}
Let $U=U_u\times U_w\subset \R^{k}\times \R^{d-k}$ be a product of convex sets and fix $r>0$ with $B_w(0,r)\subset U_w$.
Let $F:U\to \R^{d-k}$ be $C^{1}$ and fix an invertible matrix $A\in GL(d-k,\R)$.
Assume:
\begin{enumerate}
\item[\textnormal{(i)}] (\textbf{Uniform linearization in the $w$-directions})
\[
\sup_{(u,w)\in U}\,\|\partial_w F(u,w)-A\|\ \le\ \eta,
\qquad
\|A^{-1}\|\,\eta\ \le\ \frac12;
\]
\item[\textnormal{(ii)}] (\textbf{Small offset on the $w=0$ slice})
\[
\sup_{u\in U_u}\,\|A^{-1}F(u,0)\|\ \le\ \frac{r}{2}.
\]
\end{enumerate}
Then for every $u\in U_u$ there exists a \emph{unique} $w=g(u)\in B_w(0,r)$ such that $F(u,g(u))=0$.
Hence $\{F=0\}\cap (U_u\times B_w(0,r))$ is the graph of $g$.

\smallskip\noindent
If in addition $\sup_{(u,w)\in U}\|\partial_u F(u,w)\|\le \eta$, then $g$ is Lipschitz and, wherever differentiable,
\[
\|Dg\|\ \le\ \frac{\|A^{-1}\|\,\eta}{1-\|A^{-1}\|\,\eta}\ \le\ 2\,\|A^{-1}\|\,\eta.
\]

In particular, since $F$ is $C^1$ and $\partial_wF(u,g(u))$ is invertible for all $u\in U_u$, the implicit function theorem \REVMZ{\cite{LangGmT}} implies $g\in C^1(U_u)$; hence the displayed bound holds for every $u\in U_u$.


\end{lemma}
\begin{proof}
Fix $u\in U_u$ and define $T_u:B_w(0,r)\to \R^{d-k}$ by
\[
T_u(w)\ :=\ w - A^{-1}F(u,w).
\]
Write
\[
T_u(w)= -A^{-1}F(u,0)\ +\ \Bigl[w - A^{-1}(F(u,w)-F(u,0))\Bigr].
\]
By the mean value theorem in the $w$-variable,
\[
F(u,w)-F(u,0)=\Bigl(\int_0^1 \partial_w F(u,tw)\,dt\Bigr)\,w,
\]
hence
\[
w - A^{-1}(F(u,w)-F(u,0))
=\Bigl(I - A^{-1}\int_0^1 \partial_wF(u,tw)\,dt\Bigr)\,w.
\]
Using $\|\partial_wF-A\|\le \eta$ and $\|A^{-1}\|\eta\le \tfrac12$ gives
\[
\Bigl\|I - A^{-1}\int_0^1 \partial_wF(u,tw)\,dt\Bigr\|\ \le\ \|A^{-1}\|\,\eta\ \le\ \frac12,
\]
so for $w\in B_w(0,r)$,
\[
\|T_u(w)\|\ \le\ \|A^{-1}F(u,0)\| + \frac12\|w\|\ \le\ \frac r2 + \frac12 r = r.
\]
Thus $T_u$ maps $B_w(0,r)$ into itself.

Similarly, for $w,w'\in B_w(0,r)$, the mean value theorem yields
\[
T_u(w)-T_u(w')
=\Bigl(I - A^{-1}\int_0^1 \partial_wF(u,w'+t(w-w'))\,dt\Bigr)\,(w-w'),
\]
so $\|T_u(w)-T_u(w')\|\le \tfrac12\|w-w'\|$.  Hence $T_u$ is a contraction, and Banach's fixed point theorem
gives a unique fixed point $g(u)\in B_w(0,r)$ with $T_u(g(u))=g(u)$, i.e.\ $F(u,g(u))=0$.

For the slope bound, differentiate $F(u,g(u))=0$ where $g$ is differentiable:
\[
\partial_uF(u,g(u)) + (\partial_wF(u,g(u)))\,Dg(u)=0,
\qquad\text{so}\qquad
Dg(u)=-(\partial_wF)^{-1}\partial_uF.
\]
Since $\|\partial_wF-A\|\le \eta$ and $\|A^{-1}\|\eta\le \tfrac12$, Neumann series gives
$\|(\partial_wF)^{-1}\|\le \|A^{-1}\|/(1-\|A^{-1}\|\eta)$, yielding the stated estimate.


Finally, since $F$ is $C^1$ and $\partial_wF(u,g(u))$ is invertible, the implicit function theorem upgrades $g$ to a $C^1$ map on $U_u$, so the derivative identity and bound hold for all $u\in U_u$.


\end{proof}


\begin{remark}[Global single-sheet graph control on a cell]\label{rem:graph-whole-cell}
With the corner-exit Euclidean templates and the small-slope stability package in hand, the remaining microstructure/gluing
difficulty becomes sharply focused.

\smallskip\noindent

\textbf{Cell-scale single-sheet control.}
This step is now achieved by \REVMZ{Proposition~\ref{prop:cell-scale-linear-model-graph}}, which builds holomorphic complete intersections
whose local defining map $F(u,w)$ is a small perturbation of the invertible linear model in the $w$-variables and hence yields a \emph{unique}
$C^1$ graph $w=g(u)$ on all of $Q$ by Lemma~\ref{lem:global-graph-contraction}.
In particular, each holomorphic sliver in a cell is a \emph{single sheet} over its template plane on a region containing $Q$ (with slope as small as desired).

\smallskip\noindent
\textbf{Per-sliver mass control.}
Once the single-sheet small-slope graph property holds on $Q$, mass and face-slice masses are quantitatively controlled by area distortion:
Lemma~\ref{lem:sliver-stability} gives $\Mass([Y]\llcorner Q)=(1+O(\varepsilon^2))\Mass([P]\llcorner Q)$ for the underlying template plane $P$,
and Proposition~\ref{prop:holomorphic-corner-exit-g1g2} (hence Proposition~\ref{prop:holomorphic-corner-exit-L1}) controls the boundary-face contributions.
Therefore there are no ``heavy tails'' at cell scale, and the remaining mass-budget matching (L2) reduces to the discrete prefix-length bookkeeping
(with $O(1/N)+O(\varepsilon^2)$ rounding error) in the template-matching stage.

\smallskip\noindent
\textbf{How to apply Lemma~\ref{lem:global-graph-contraction} to holomorphic complete intersections.}
In a holomorphic chart, write the local coefficients of the defining sections as a map
$F=(f_1,\dots,f_p):U\to \C^p\cong\R^{2p}$.  Choose real coordinates $(u,w)\in\R^{k}\times\R^{2p}$ so that the template
plane is $\{w=0\}$ and the linear model is $w\mapsto Aw$ with $A$ invertible.
If one can construct the sections so that, on a ball containing $Q$,
\[
\|\partial_wF-A\|_{L^\infty}\le \eta,\qquad \|\partial_uF\|_{L^\infty}\le \eta,\qquad
\|F(\cdot,0)\|_{L^\infty(U_u)}\le \eta\,h,
\]
with $\|A^{-1}\|\eta\ll 1$, then Lemma~\ref{lem:global-graph-contraction} gives a global graph $w=g(u)$ on all of $Q$.
This is exactly the ``graph on the whole cell'' step highlighted in the microstructure roadmap.

\smallskip\noindent
\textbf{Two standard routes to produce the needed uniform $C^1$ control} are:
\begin{itemize}
\item peak sections plus $\bar\partial$-solving (H\"ormander $L^2$ estimates) to approximate prescribed affine-linear holomorphic models on
Bergman-scale balls, and
\item Bergman kernel asymptotics / jet right-inverses \REVMZ{\cite{Tian90,Catlin99,Zelditch98,Donaldson01}} to achieve the same $C^1$ control directly.
\end{itemize}
\end{remark}


\begin{lemma}[Bergman-scale affine model approximation via $\bar\partial$-solving]\label{lem:bergman-affine-approx-hormander}
Fix a point $x\in X$ and choose holomorphic normal coordinates $z=(z^1,\dots,z^n)$ on a ball
$U\cong B_\rho(0)\subset\C^n$ centered at $x$, together with a local holomorphic frame $e$ for $L$
such that $|e|_h^2=e^{-\phi}$ and $\phi(z)=|z|^2+O(|z|^3)$ on $B_\rho(0)$.
Let $\ell(z)=\sum_{i=1}^n a_i z^i$ be a complex affine-linear function with $|a|\le 1$.

{\noindent\textbf{notation fix.}
In this lemma the tensor-power parameter is denoted by $\mhol$ (so sections live in $H^0(X,L^{\mhol})$);
this is distinct from the fixed cohomology-clearing integer $m$ used in Theorem~\ref{thm:global-cohom}.}

Then for all sufficiently large {$\mhol$} there exists a global section
{$s_{\ell,\mhol}\in H^0(X,L^{\mhol})$} such that, writing
{$s_{\ell,\mhol}=f_{\ell,\mhol}\,e^{\otimes \mhol}$} on $B_{\rho/8}(0)$, one has on the Bergman-scale ball
{$B_{R/\sqrt{\mhol}}(0)\subset B_{\rho/8}(0)$}:
\[
\sup_{|z|\le R/\sqrt{\mhol}}
\Bigl(|f_{\ell,\mhol}(z)-\ell(z)|+\sqrt{\mhol}\,|\nabla(f_{\ell,\mhol}-\ell)(z)|\Bigr)
\ \le\ {\varepsilon_{\mathrm{hol}}(\mhol)},
\qquad
{\varepsilon_{\mathrm{hol}}(\mhol)\xrightarrow[\mhol\to\infty]{}0},
\]
with constants uniform in $x$ (over a fixed finite atlas) and in $\ell$ with $|a|\le 1$.
{Moreover one may take $\varepsilon_{\mathrm{hol}}(\mhol)\le C_R e^{-c\mhol}$ for some $c>0$ depending only on the
positivity of $(L,h)$ and the cutoff scale.}
\end{lemma}

\begin{proof}
Choose a cutoff $\chi$ supported in $B_{\rho/2}(0)$ with $\chi\equiv 1$ on $B_{\rho/4}(0)$ and set
{$\tilde s:=\chi\,\ell\,e^{\otimes \mhol}$} (extended by $0$ outside $U$).
Then $\bar\partial\tilde s=(\bar\partial\chi)\,\ell\,{e^{\otimes \mhol}}$ is supported in the annulus
$\{\rho/4\le |z|\le \rho/2\}$.
\REVMZ{Apply the H\"ormander $L^2$ $\bar\partial$ estimate \cite[Thm~4.2.1]{Hormander65} (see also \cite[Ch.~VIII]{Demailly12}):}
Solve $\bar\partial u=\bar\partial\tilde s$ using H\"ormander $L^2$ estimates for the positive bundle
{$(L^{\mhol},h^{\mhol})$}; the weight $e^{-{\mhol}\phi}$ forces
$\|u\|_{L^2(h^{\mhol})}\le C\,e^{-c\,{\mhol}}$.
On the inner ball $B_{\rho/4}(0)$ one has $\bar\partial u=0$, so $u$ is holomorphic there.
Standard local $L^2\to C^1$ estimates for holomorphic sections
(e.g.\ mean-value inequality plus Cauchy estimates
 at scale ${\mhol^{-1/2}}$)
give $\|u\|_{C^1({B_{R/\sqrt{\mhol}}})}\le C_R\,e^{-c\,{\mhol}}$.
Setting {$s_{\ell,\mhol}:=\tilde s-u$} yields $s_{\ell,\mhol}$ holomorphic and
$f_{\ell,\mhol}=\ell-\text{(holomorphic error)}$ on ${B_{R/\sqrt{\mhol}}}$ with the stated bound.
\end{proof}



\begin{proposition}[Cell-scale linear-model complete intersections are single-sheet graphs]\label{prop:cell-scale-linear-model-graph}

Fix a holomorphic chart identifying a neighborhood of a cell $Q$ with a domain in $\C^{n}=\C^{n-p}\times\C^{p}$ with coordinates $z=(u,w)$, and assume $Q\subset B_{R/\sqrt{\mhol}}(0)$ for some fixed $R$. Assume moreover that the cell diameter satisfies $h\asymp \mhol^{-1/2}$.
Let $t\in\C^p$ satisfy $|t|\le c\,h$ (with $h\asymp \mhol^{-1/2}$).
Then {for all sufficiently large $\mhol$ there exist sections} $\sigma_1,\dots,\sigma_p\in H^0(X,L^{\mhol})$ such that, writing
$\sigma_j=F_j\,e^{\otimes \mhol}$ in a local frame on $B_{R/\sqrt{\mhol}}(0)$ and setting $F=(F_1,\dots,F_p)$, one has
\[
\|\partial_w F-I\|_{L^\infty(B_{R/\sqrt{\mhol}})}\ +\ \|\partial_u F\|_{L^\infty(B_{R/\sqrt{\mhol}})}\ \le\ \eta_{\mhol},
\qquad
\sup_{u:\ (u,t)\in B_{R/\sqrt{\mhol}}} |F(u,t)|\ \le\ \eta_{\mhol}\,h,
\]
{with $\eta_{\mhol}\to 0$ as $\mhol\to\infty$.}
Consequently, for $\mhol$ large enough, the common zero set
$Y_t:=\{\sigma_1=\cdots=\sigma_p=0\}$ satisfies that $Y_t\cap Q$ is a \emph{single} $C^1$ graph over the affine complex plane
$\{w=t\}$ on all of $Q$, with slope $O(\eta_{\mhol})$ (hence as small as desired).
\end{proposition}

\begin{proof}

Apply Lemma~\ref{lem:bergman-affine-approx-hormander} to the affine-linear holomorphic functions
$\ell_0\equiv 1$ and $\ell_j(z)=w_j$ ($1\le j\le p$).  Thus for each $j=0,1,\dots,p$ we obtain a holomorphic section
$s_j\in H^0(X,L^{\mhol})$ with local coefficient $f_j$ on $B_{R/\sqrt{\mhol}}$ such that
\[
\begin{aligned}
	\sup_{B_{R/\sqrt{\mhol}}}\Bigl(|f_0-1|+\sqrt{\mhol}\,|\nabla(f_0-1)|\Bigr)
	&\le \varepsilon_{\mhol},\\
	\sup_{B_{R/\sqrt{\mhol}}}\Bigl(|f_j-w_j|+\sqrt{\mhol}\,|\nabla(f_j-w_j)|\Bigr)
	&\le \varepsilon_{\mhol}\qquad (1\le j\le p).
\end{aligned}
\]
for some $\varepsilon_{\mhol}\to 0$ (as in Lemma~\ref{lem:bergman-affine-approx-hormander}).
\noindent\textbf{} on $B_{R/\sqrt{\mhol}}$ the $(u,w)$ chart is Euclidean to the required order, so $\partial_w(f_j-w_j)$ and $\partial_u(f_j-w_j)$ are $O(\varepsilon_{\mhol})$. Thus the map $F=(F_1,\dots,F_p)$ satisfies the small-slope conditions of Lemma~\ref{lem:graph-from-grad}.\par


For $t=(t_1,\dots,t_p)$ define $\sigma_j:=s_j-t_j s_0$ and write
$\sigma_j=F_j\cdot e^{\otimes \mhol}$ in the chosen local frame, so
\[
F_j(u,w)=f_j(u,w)-t_j f_0(u,w).
\]
Since $\nabla(w_j)=e_j$ and $\nabla(1)=0$ in the Euclidean chart, the above estimates imply
\[
\|\partial_w F-I\|_{L^\infty(B_{R/\sqrt{\mhol}})}+\|\partial_u F\|_{L^\infty(B_{R/\sqrt{\mhol}})}
\ \le\ C\,\frac{\varepsilon_{\mhol}}{\sqrt{\mhol}},
\]
and at $w=t$ we have
\[
F_j(u,t)=(f_j-w_j)(u,t)-t_j(f_0-1)(u,t),
\qquad\text{hence}\qquad
\sup_{u:(u,t)\in B_{R/\sqrt{\mhol}}}|F(u,t)|\le C\,\varepsilon_{\mhol} .
\]
{Set $\eta_{\mhol}:=C\,\varepsilon_{\mhol}$. Since the ball has radius $R/\sqrt{\mhol}$, the above $C^1$ bounds imply}
\[
\sup_{B_{R/\sqrt{\mhol}}}|f_0-1|+\max_{1\le j\le p}\sup_{B_{R/\sqrt{\mhol}}}|f_j-w_j|
\le C\,\varepsilon_{\mhol}/\sqrt{\mhol}.
\]
Hence the affine error satisfies
\[
\begin{aligned}
	\|\partial_w F-I\|_{L^\infty(B_{R/\sqrt{\mhol}})}
	+\|\partial_u F\|_{L^\infty(B_{R/\sqrt{\mhol}})}
	&\le \eta_{\mhol},\\
	\sup_{\substack{u\\(u,t)\in B_{R/\sqrt{\mhol}}}} |F(u,t)|
	&\le \frac{\eta_{\mhol}}{\sqrt{\mhol}}
	\le \eta_{\mhol}\,h.
\end{aligned}
\]

since $h=\mathrm{diam}(Q)\asymp \mhol^{-1/2}$.

Introduce the translated variable $\widetilde w:=w-t$ and the translated map
\[
\widetilde F(u,\widetilde w):=F(u,\widetilde w+t).
\]
Then $\partial_{\widetilde w}\widetilde F=\partial_w F$ and $\partial_u\widetilde F=\partial_u F$, and
$\sup_{u}|\widetilde F(u,0)|=\sup_{u}|F(u,t)|\le \eta_{\mhol} h$.

Choose $r\simeq h$ and a product set $U_u\times U_{\widetilde w}\subset B_{R/\sqrt{\mhol}}$
with $Q\subset U_u\times (t+U_{\widetilde w})$ and $U_{\widetilde w}\subset B_{\widetilde w}(0,r)$.
For $\mhol$ large we have $\eta_{\mhol}\ll 1$ and $\eta_{\mhol} h\le r/2$, so Lemma~\ref{lem:global-graph-contraction}
applies to $\widetilde F$ on $U_u\times U_{\widetilde w}$ with $A=I$.
It produces a unique $C^1$ graph $\widetilde w=g(u)$ solving $\widetilde F(u,\widetilde w)=0$ on $U_u$,
hence $w=t+g(u)$ solves $F(u,w)=0$.
Therefore,
\[
Y_t\cap Q=\{(u,w)\in Q:\sigma_1=\cdots=\sigma_p=0\}
\]
is a single $C^1$ graph over the affine plane $\{w=t\}$ on all of $Q$,
and the slope estimate follows from Lemma~\ref{lem:global-graph-contraction}:
$\|Dg\|_{L^\infty}\le 2\eta_{\mhol}$.

\end{proof}





% ================================
\subsubsection*{Graph-template package (whole-cell graphs)}
%
\begin{definition}[Graph template on a cube]\label{def:graph-template}
Fix a cube $Q\subset X$ of side length $h$ contained in a holomorphic coordinate chart, and fix a splitting
$\R^{2n}\cong \R^{2n-2p}\times \R^{2p}$ adapted to an affine complex plane $\{w=t\}$ (so $u\in\R^{2n-2p}$ are tangential coordinates and
$w\in\R^{2p}$ are transverse coordinates).
A \emph{graph template} in $Q$ (over $\{w=t\}$) is a pair $(t,\Gamma)$ where $t\in\R^{2p}$ and $\Gamma\subset Q$ is a $C^1$ submanifold
of real dimension $2n-2p$ such that
\[
\Gamma\;=\;\{(u,\,t+g(u)):\ u\in U_Q\}
\]
for some domain $U_Q\subset \R^{2n-2p}$ projecting $Q$ (e.g. $U_Q=\pi_u(Q)$) and a $C^1$ map $g:U_Q\to \R^{2p}$ with
$\|Dg\|_{L^\infty(U_Q)}\le \eta_{\mathrm{graph}}$ for a prescribed small slope parameter $\eta_{\mathrm{graph}}\in(0,1)$.
\end{definition}


%
\begin{remark}[What this package is for]\label{rem:graph-template-remark}
This is a bookkeeping device for the sharp-cube variant: when a holomorphic complete intersection is known to be a \emph{single} small-slope
graph on \emph{all} of a cube $Q$, we treat that sheet as an atom in a discrete transverse measure (as in the sliver regime), but now each atom
carries a globally defined graph on $Q$.
Nothing in the definition assumes holomorphicity; holomorphicity enters only through the existence lemma below.
\end{remark}


%
\begin{lemma}[Linear-model sheets yield whole-cell graph templates]\label{lem:graph-template-exists}
In the setting of Proposition~\ref{prop:cell-scale-linear-model-graph}, fix a translation parameter $t$ and let
$Y_t:=\{\sigma_1=\cdots=\sigma_p=0\}$ be the associated complete intersection in the chart. Then
$\Gamma_{t,Q}:=Y_t\cap Q$ is a graph template on $Q$ in the sense of Definition~\ref{def:graph-template}, with slope
$\eta_{\mathrm{graph}}=O(\eta_{\mhol})$.
\end{lemma}

\begin{proof}
Proposition~\ref{prop:cell-scale-linear-model-graph} provides $C^1$ bounds
$\|\partial_w F-I\|_{L^\infty}\!+\!\|\partial_u F\|_{L^\infty}\le \eta_{\mhol}$ on $B_{R/\sqrt{\mhol}}$
and an offset bound $\sup_{u:\,(u,t)\in B_{R/\sqrt{\mhol}}}|F(u,t)|\le \eta_{\mhol}h$.
Translate $w'=w-t$ and set $\widetilde F(u,w'):=F(u,w'+t)$ on the cube $Q$ (shrunk by a harmless constant so that a transverse ball
$B_{w'}(0,r)$ with $r\asymp h$ lies in the $w'$-projection of $Q$).
Then $\partial_{w'}\widetilde F=\partial_wF$ and $\widetilde F(u,0)=F(u,t)$, so Lemma~\ref{lem:global-graph-contraction}
(with $A=I$) applies and yields a unique $C^1$ solution $w'=g(u)$ of $\widetilde F(u,w')=0$ for each $u$.
Equivalently, $Y_t\cap Q=\{(u,t+g(u))\}$ is a single graph on $Q$, and the slope bound is controlled by $\eta_{\mhol}$ as stated.
\end{proof}

%
\begin{remark}[Global plan in the graph-template variant]\label{rem:global-graph-plan}
Once one has whole-cell graphs (Definition~\ref{def:graph-template}) for the required translation net, the transport-and-glue steps can be
run exactly as in the sliver regime: one couples the induced transverse measures across faces using an integral transport plan, then uses
the transport to build a filling current with controlled flat norm, and finally corrects the residual boundary by the boundary-correction
machinery.
The substantive checkpoint is therefore the existence of whole-cell graphs (cf. Remark~\ref{rem:graph-whole-cell}).
\end{remark}


%
\begin{definition}[Face/vertex restriction maps for graph templates]\label{def:graph-restriction-maps}
Fix an ordered translation net $\mathcal N=\{t_\ell\}_{\ell=1}^L\subset \R^{2p}$.
A \emph{graph-template measure} on a cube $Q$ (for a fixed direction label $i$) is an atomic measure
\[
\mu_{Q,i}=\sum_{\ell=1}^L a_{Q,i,\ell}\,\delta_{t_\ell},
\qquad a_{Q,i,\ell}\in\Z_{\ge 0}.
\]
For any face $F\subset\partial Q$ and any vertex $v\in F$, we define the induced face and vertex measures by
\[
\mu_{Q\to F,i}:=\mu_{Q,i},
\qquad
\mu_{Q,v,i}:=\mu_{Q,i}.
\]
Equivalently, the restriction maps $\operatorname{Res}_{Q\to F}$ and $\operatorname{Res}_{F\to v}$ are the identity on atoms; only the bookkeeping
of which cubes share the same net enters.
When the measures arise from whole-cell graph sheets $(t,\Gamma)\subset Q$ (Definition~\ref{def:graph-template}),
the translation parameter $t$ determines the unique atom, and the above restriction agrees with geometric restriction of the
sheet to $F$ and to the vertex star, provided the sheet is realized by a single holomorphic object on the overlap
(as in Remark~\ref{rem:vertex-star-coherence}).
\end{definition}



%
\begin{proposition}[Integral transport for graph-template measures]\label{prop:graph-template-transport}
Let $\mu=\sum_{\ell} a_\ell\,\delta_{x_\ell}$ and $\nu=\sum_{\ell} b_\ell\,\delta_{x_\ell}$ be atomic measures on a common finite set of atoms
$\{x_\ell\}$ (e.g. the translation net in the transverse space), with integer weights $a_\ell,b_\ell\ge 0$.
Then there exists an \emph{integral} transport plan $\pi$ from $\mu$ to $\nu$ with optimal $W_1$ cost \REVMZ{\cite{Villani03}}, and in particular
\[
W_1(\mu,\nu)\;=\;\inf_{\pi\in\Pi(\mu,\nu)}\int |x-y|\,d\pi(x,y)
\]
is attained by a plan $\pi$ that is a sum of Dirac masses with integer weights.
\end{proposition}

\begin{proof}
This is exactly Proposition~\ref{prop:integer-transport} (applied to the discrete measures obtained from the graph-template multiplicities),
and we only restate it here as the dedicated graph-template transport input.
\end{proof}

%
\begin{lemma}[Flat-norm control from face transport in the graph-template variant]\label{lem:graph-template-flat-control}
Assume the hypotheses of Proposition~\ref{prop:transport-flat-glue} for the facewise assemblies determined by a graph-template configuration.
Then the corresponding filling current $B_F$ satisfies
\[
\mathcal F(B_F)\ \le\ C\,h\,W_1(\mu_{Q\to F},\mu_{Q'\to F}),
\]
with the same constant $C$ as in Proposition~\ref{prop:transport-flat-glue}.
\end{lemma}

\begin{proof}
This is the conclusion of Proposition~\ref{prop:transport-flat-glue}; no change is needed beyond the observation that the graph-template
slices are still assembled by transporting atomic masses across the face.
\end{proof}


%
\begin{proposition}[Global gluing for graph templates]\label{prop:graph-template-global}
Fix a mesh $\mathcal Q_h$ of side length $h$ and a direction label $i$.
Assume a graph-template configuration on each cube $Q\in\mathcal Q_h$ in the sense of Definition~\ref{def:graph-template},
together with atomic measures $\mu_{Q,i}$ on a common translation net and induced face measures $\mu_{Q\to F,i}$
as in Definition~\ref{def:graph-restriction-maps}.
Assume further that for every interior face $F=Q\cap Q'$ we have chosen an \emph{integral} transport plan between
$\mu_{Q\to F,i}$ and $\mu_{Q'\to F,i}$ as in Proposition~\ref{prop:graph-template-transport}, and let
$T^{\mathrm{raw}}$ denote the resulting facewise assembly.

Finally, assume that the global boundary-flat estimate of Corollary~\ref{cor:raw-boundary-flat-small}
holds for this raw assembly (equivalently, that the hypotheses of that corollary are satisfied by the graph-template discrete data).
Then $T^{\mathrm{raw}}$ is an integral current and there exists an integral filling $U_{\mathrm{glue}}$ with
\[
\partial U_{\mathrm{glue}}=\partial T^{\mathrm{raw}},
\qquad
\Mass(U_{\mathrm{glue}})\ \le\ C\,\mathcal F(\partial T^{\mathrm{raw}}),
\]
so that $T:=T^{\mathrm{raw}}-U_{\mathrm{glue}}$ is a closed integral cycle.
\end{proposition}

\begin{proof}
Integrality of $T^{\mathrm{raw}}$ follows from integrality of the chosen transport plans (Proposition~\ref{prop:graph-template-transport})
and the fact that the assembly is performed by integer-multiplicity sheet pieces.
Under the stated hypothesis, Corollary~\ref{cor:raw-boundary-flat-small} supplies the quantitative bound
$\mathcal F(\partial T^{\mathrm{raw}})\to 0$ in the regime under consideration.
Applying Proposition~\ref{prop:glue-cell} to $T^{\mathrm{raw}}$ yields an integral filling $U_{\mathrm{glue}}$ with
$\partial U_{\mathrm{glue}}=\partial T^{\mathrm{raw}}$ and the displayed mass bound.
The cycle $T:=T^{\mathrm{raw}}-U_{\mathrm{glue}}$ is then closed and integral.
\end{proof}


%
\begin{remark}[What is actually used from ``graph'' geometry]\label{rem:graph-flat-remark}
\REVMZ{\textbf{[Technical clarification.]}}
All subsequent uses of this package only require: (i) whole-cell graphs exist with a uniform small-slope bound, and
(ii) the induced transverse measures are atomic on a common net so that integral optimal transport applies.
No additional holomorphic structure is needed at the transport stage.
\end{remark}


%
\begin{lemma}[Boundary correction for graph-template assemblies]\label{lem:graph-boundary-correction}
Assume $T^{\mathrm{raw}}$ is a graph-template raw assembly as in Proposition~\ref{prop:graph-template-global} and that the hypotheses of
Theorem~\ref{thm:boundary-correction} apply to $T^{\mathrm{raw}}$. Then there exists an integral correction current $S$ with
$\partial (T^{\mathrm{raw}}+S)=0$ and with the same quantitative bounds as in Theorem~\ref{thm:boundary-correction}.
\end{lemma}

\begin{proof}
Apply Theorem~\ref{thm:boundary-correction} to $T^{\mathrm{raw}}$.
\end{proof}

%
\begin{remark}[Boundary-correction remark for the graph package]\label{rem:graph-boundary-remark}
Any remaining issue at this stage is therefore \emph{not} geometric but logical: one must ensure that the hypotheses of
Theorem~\ref{thm:boundary-correction} are explicitly checked for the particular raw assembly produced by the graph-template transport-and-glue steps.
\end{remark}

% ================================
\begin{lemma}[Vertex-ball locality excludes nonincident faces]\label{lem:ball-excludes-faces}
Let $Q=[0,h]^d\subset\R^d$ and let $v$ be a vertex of $Q$.
Let $F\subset\partial Q$ be any codimension-$1$ face. If $v\notin F$, then $\dist(v,F)=h$.
Consequently, if $E\subset Q$ satisfies
\[
E\subset B(v,c_0h)\qquad\text{for some }0<c_0<1,
\]
then $E\cap F=\emptyset$ for every face $F$ not containing $v$.
\end{lemma}
\begin{proof}
After translation we may assume $v=0$.
Every codimension-$1$ face of $Q$ is of the form $\{x_j=0\}$ or $\{x_j=h\}$.
If $0\notin F$, then $F=\{x_j=h\}$ for some $j$, hence $\dist(0,F)=h$.
If $E\subset B(0,c_0h)$ with $c_0<1$, then $E$ cannot intersect any set at distance $h$ from $0$.
\end{proof}

\begin{lemma}[{Fat corner simplices force ``if'' on the designated exit faces}]\label{lem:corner-simplex-hits-designated-faces}

Fix $d\ge 2$ and $1\le k<d$.  Let $Q=[0,h]^d$ and let $v$ be a vertex.
Assume $E\subset Q$ is a $k$--simplex satisfying:
\begin{enumerate}
\item[\textup{(C1)}]\label{C1cornerfaces}
There exist \emph{distinct} codimension--$1$ faces $F_0,\dots,F_k$ of $Q$ incident to $v$ such that, for each $i$, the intersection $E\cap F_i$ is a $(k-1)$--dimensional \emph{facet} of $E$ (equivalently, $E\cap F_i$ has nonempty relative interior inside the affine hyperplane $F_i$).
\item[\textup{(C2)}]\label{C2cornerboundary}
The boundary footprint meets no other codimension--$1$ faces:
\[
E\cap \partial Q \subset \bigcup_{i=0}^k F_i.
\]
\item[\textup{(C3)}]\label{C3cornerlocal}
$E$ is localized near $v$, i.e.\ $E\subset B(v,c_0h)$ for some $0<c_0<1$.
\end{enumerate}
Then for any codimension--$1$ face $F$ of $Q$,
\[
\mathcal H^{k-1}(E\cap F)>0
\quad\Longleftrightarrow\quad
F\in\{F_0,\dots,F_k\}.
\]
Moreover, if $F$ is not incident to $v$, then $E\cap F=\emptyset$.

\end{lemma}

\begin{proof}

For each $i$, since $E\cap F_i$ is a facet of the $k$--simplex $E$, it contains a relatively open subset of the $(k-1)$--dimensional affine hyperplane $F_i$. Hence $\mathcal H^{k-1}(E\cap F_i)>0$.

Let $F$ be a codimension--$1$ face of $Q$ \emph{not} incident to $v$.  Using \eqref{C3cornerlocal} with $c_0<1$, Lemma~\ref{lem:ball-excludes-faces} implies $E\cap F=\emptyset$.

Now let $F$ be incident to $v$ but $F\notin\{F_0,\dots,F_k\}$.  By \eqref{C2cornerboundary},
\[
E\cap F \subset E\cap\partial Q \subset \bigcup_{i=0}^k F_i,
\]
so $E\cap F\subset\bigcup_{i=0}^k (E\cap F\cap F_i)$.  For each $i$, since $F\neq F_i$ the intersection $F\cap F_i$ is contained in a codimension--$2$ face of $Q$, and $E\cap F_i$ has nonempty relative interior in $F_i$; therefore $E\cap F\cap F_i$ is contained in a proper boundary piece of the facet $E\cap F_i$ and has Hausdorff dimension at most $k-2$.  In particular $\mathcal H^{k-1}(E\cap F\cap F_i)=0$ for all $i$, hence $\mathcal H^{k-1}(E\cap F)=0$.

This proves the claimed ``if and only if'' statement.

\end{proof}

























\begin{proof}

\textbf{Step 1: no new faces.}
If $F$ is a codimension--$1$ face of $Q$ with $E\cap F=\emptyset$, then $\dist(E,F)\ge\delta$ by definition of $\delta$.
Since $\sup_{x\in E}|\Phi(x)-x|<\delta/2$, we have $\dist(\Phi(E),F)\ge\delta/2$, hence $Y\cap F=(\Phi(E))\cap F=\emptyset$.


This proves the ``$\Rightarrow$'' direction in \textup{(G1)}. The ``$\Leftarrow$'' direction follows because each facet

$E\cap F_i$ is nonempty, hence $\Phi(E)$ intersects $F_i$ for $\varepsilon$ small (by continuity) and $Y=\Phi(E)$ on $Q$.

\textbf{Step 2: transverse intersections with $F_i$.}
Fix $i\in\{0,\dots,k\}$.  Since $E\cap F_i$ is a facet of the simplex $E$, it contains a relatively open subset of the
$(k-1)$--dimensional affine plane $P\cap F_i$.  Because $\Phi$ is $C^1$--close to the identity on $E$,
the tangent planes $T_yY$ are $\varepsilon$--close to $P$ uniformly, so $Y$ meets $F_i$ transversely for $\varepsilon$ sufficiently small.
In particular, $Y\cap F_i$ is a smooth oriented $(k-1)$--submanifold and
\[
\Mass\bigl(\partial([Y]\llcorner Q)\llcorner F_i\bigr)=\mathcal H^{k-1}(Y\cap F_i).
\]

\textbf{Step 3: area distortion on the facets.}
On $E\cap F_i$, the restriction $\Phi|_{E\cap F_i}$ is a $C^1$ embedding whose differential is $O(\varepsilon)$--close to the identity
on the $(k-1)$--plane $P\cap F_i$.  By the area formula for graphs (as in Lemma~\ref{lem:sliver-stability}),
\[
\mathcal H^{k-1}(Y\cap F_i)=\bigl(1+O_k(\varepsilon^2)\bigr)\,\mathcal H^{k-1}(E\cap F_i).
\]
Combining with \eqref{H2cornerfat} gives \eqref{G2mass}.

\end{proof}


\begin{proof}

\emph{Step 1: localization away from non--incident faces.}
By \eqref{H1corner} we have $E\subset B(v,c_0h)$ with $c_0<1$.  Choosing $\varepsilon_0(c_0)>0$ small enough, the $C^1$--graph assumption implies
$Y\cap Q=\Phi(E)\subset B(v,(c_0+\tfrac12)h)$.  Hence Lemma~\ref{lem:ball-excludes-faces} shows that $Y$ cannot meet any codimension--$1$ face of $Q$ not incident to $v$.

\emph{Step 2: the designated faces are hit with positive $(k-1)$--measure.}
Fix $i\in\{0,\dots,k\}$.  Since $E\cap F_i$ is a facet of the simplex $E$, it has nonempty relative interior in $F_i$ and therefore $\mathcal H^{k-1}(E\cap F_i)>0$.
Because $\Phi$ is a homeomorphism $E\to Y\cap Q$ and $\partial E=\bigcup_{j=0}^k (E\cap F_j)$ (all facets lie on $\partial Q$ by \eqref{H1corner}),
we have $\partial(Y\cap Q)=\Phi(\partial E)\subset\partial Q$.
On the relative interior of the facet $E\cap F_i$ there is a positive distance to every other codimension--$1$ face of $Q$; for $\varepsilon$ sufficiently small,
continuity forces $\Phi(\mathrm{relint}(E\cap F_i))\subset F_i$.  In particular $\mathcal H^{k-1}(Y\cap F_i)>0$.

\emph{Step 3: no other incident faces carry positive $(k-1)$--measure.}
Let $F$ be a codimension--$1$ face of $Q$ incident to $v$ but $F\notin\{F_0,\dots,F_k\}$.  By \eqref{H1corner}, $E\cap F=\emptyset$.
Since $E$ is compact and disjoint from $F$, there is a positive gap $\dist(E,F)\ge \delta>0$.  Under the standing assumption $\sup_{x\in E}|\Phi(x)-x|<\delta/2$,
we get $\dist(\Phi(E),F)\ge \delta/2$, hence $Y\cap F=\emptyset$ and $\mathcal H^{k-1}(Y\cap F)=0$.
Together with Steps 1--2 this proves \eqref{G1iff}.

\emph{Step 4: boundary current and mass on each designated face.}
Fix $i$.  For $\varepsilon$ small, the $C^1$--graph condition implies that $Y$ is $C^1$--close to $P$ on $Q$, and since $P$ meets $F_i$ transversely along the facet $E\cap F_i$,
the perturbed submanifold $Y$ also meets $F_i$ transversely; hence $Y\cap F_i$ is a smooth oriented $(k-1)$--submanifold.
In this situation, the restriction of the boundary current to the face is the integration current over the slice:
\[
\partial([Y]\llcorner Q)\llcorner F_i = [Y\cap F_i],
\]
so $\Mass(\partial([Y]\llcorner Q)\llcorner F_i)=\mathcal H^{k-1}(Y\cap F_i)$.

\emph{Step 5: compare $\mathcal H^{k-1}(Y\cap F_i)$ to $\mathcal H^{k-1}(E\cap F_i)$.}
On each facet $E\cap F_i$ the map $\Phi$ restricts to a $C^1$ map into $F_i$ whose differential differs from the identity by $O(\varepsilon)$; in particular
$Y\cap F_i=\Phi(E\cap F_i)$ is a small--slope $(k-1)$--graph over $E\cap F_i$ inside $F_i$.
Applying Lemma~\ref{lem:small-graph-distortion} with $m=k-1$ gives
\[
\mathcal H^{k-1}(Y\cap F_i)=\mathcal H^{k-1}(\Phi(E\cap F_i))
=(1+O_k(\varepsilon^2))\,\mathcal H^{k-1}(E\cap F_i).
\]
Finally Lemma~\ref{lem:corner-simplex-face-mass} yields $\mathcal H^{k-1}(E\cap F_i)\simeq_{k,\Lambda} v_E^{(k-1)/k}$, completing \eqref{G2mass}.

\end{proof}


\begin{corollary}[{Corner--exit faces persist uniformly across a finite template family}]\label{cor:holomorphic-corner-exit-inherits}

Fix $d\ge 2$ and $1\le k<d$.  Let $Q=[0,h]^d$ and let $v$ be a vertex.
Let $\{P_a\}_{a=1}^N$ be a finite family of affine $k$--planes and set $E_a:=P_a\cap Q$.
Suppose that for each $a$:
\begin{enumerate}
\item[\textup{(T1)}]\label{T1}
$E_a$ is a $k$--simplex satisfying the footprint hypotheses \eqref{H1corner} of Proposition~\ref{prop:holomorphic-corner-exit-g1g2}
(with designated exit faces $F_0^{(a)},\dots,F_k^{(a)}$ incident to $v$).
\item[\textup{(T2)}]\label{T2}
$E_a$ is $\Lambda$--fat with the same fatness parameter $\Lambda$.
\end{enumerate}
Assume moreover that $\varepsilon>0$ is chosen small enough (depending only on $k$ and $\Lambda$) so that Proposition~\ref{prop:holomorphic-corner-exit-g1g2} applies to every pair $(E_a,Y^{(a)})$.

Define the uniform gap
\[
\delta_\star:=\min_{1\le a\le N}\ \min\{\dist(E_a,F): F\ \text{a codimension--$1$ face of $Q$ with }F\notin\{F_0^{(a)},\dots,F_k^{(a)}\}\}\ >0.
\]
Let $Y^{(a)}$ be smooth oriented $k$--submanifolds such that each $Y^{(a)}\cap Q$ is a single $C^1$ graph over $E_a$
with slope at most $\varepsilon$, realized by an embedding $\Phi_a:E_a\to\R^d$ with $\Phi_a(E_a)=Y^{(a)}\cap Q$ and
$\sup_{x\in E_a}|\Phi_a(x)-x|<\delta_\star/2$.

Then for every $a$ the conclusions \eqref{G1iff}--\eqref{G2mass} of Proposition~\ref{prop:holomorphic-corner-exit-g1g2} hold for $(E_a,Y^{(a)})$,
with constants depending only on $(k,\Lambda)$ and on the graph--slope bound (equivalently on $\varepsilon$).

\end{corollary}

\begin{proof}

Apply Proposition~\ref{prop:holomorphic-corner-exit-g1g2} to each pair $(E_a,Y^{(a)})$.
The only additional point is that the smallness requirement on the graph (encoded there by the gap parameter $\delta$) can be chosen uniformly in $a$,
because the family is finite and $\delta_\star>0$ by definition.
This is legitimate because the smallness threshold in Proposition~\ref{prop:holomorphic-corner-exit-g1g2} depends only on $(k,\Lambda)$ (and the slope bound), not on $a$.

\end{proof}












\begin{lemma}[Packing bound for disjoint sliver graphs]\label{lem:sliver-packing}
Let $Q\subset\R^{2n}$ be a bounded domain of diameter $h$ and fix an affine $(2n-2p)$-plane $P$ with transverse space $P^\perp\cong\R^{2p}$.
Assume we have affine translates $P+t_1,\dots,P+t_N$ such that each $(P+t_a)\cap Q\neq\emptyset$ and
\[
\|t_a-t_b\|\ \ge\ 10\,\varepsilon\,h
\qquad (a\neq b).
\]
Then $N\le C(n,p)\,\varepsilon^{-2p}$.
\end{lemma}

\begin{proof}
Since $(P+t_a)\cap Q\neq\emptyset$ and $\mathrm{diam}(Q)=h$, the translation parameters $t_a$ all lie in a transverse ball $B_{Ch}(0)\subset P^\perp$
for a dimensional constant $C$ (depending only on the choice of identification of $P^\perp$ with $\R^{2p}$).
The balls $B(t_a,5\varepsilon h)\subset P^\perp$ are pairwise disjoint and contained in $B_{(C+5\varepsilon)h}(0)$.
Comparing Euclidean volumes in $\R^{2p}$ gives
\[
N\,(5\varepsilon h)^{2p}\ \lesssim\ (Ch)^{2p},
\]
hence $N\lesssim \varepsilon^{-2p}$ as claimed.
\end{proof}



















\begin{remark}[Vertex-star coherence (how to make the same template live across adjacent cubes)]\label{rem:vertex-star-coherence}
For the global gluing/plumbing, one wants the \emph{same index-$a$ sliver} anchored at a vertex $v$ to be used by every cube incident to $v$,
so that across any shared face the mismatch reduces to a pure prefix-count difference (rather than a geometric displacement mismatch).

\smallskip\noindent
This is achieved by choosing the anchor points $x_a$ in Proposition~\ref{prop:finite-template} (hence the Bergman balls on which the $C^1$ control holds)
to be \emph{vertex-centered}: take $x_a\in (P+t_a)\cap B(v,c_0h)$ (for instance $x_a=v+t_a$ in a coordinate model).
If the mesh satisfies $h\lesssim \mhol^{-1/2}$ with a small enough constant, then the Bergman ball $B_{c\,\mhol^{-1/2}}(x_a)$ contains the entire
vertex star $\mathrm{Star}(v)$ (the union of the finitely many cubes meeting at $v$), so the resulting holomorphic complete intersection $Y^a$
is a single-sheet graph over the same affine translate $P+t_a$ \emph{on every cube in $\mathrm{Star}(v)$ simultaneously}.
Thus the vertex template is realized by a single global holomorphic object $Y^a$, and restricting to each cube produces coherent
face slices at that vertex.
\end{remark}


\begin{lemma}[Slow variation under rounding of Lipschitz targets]\label{lem:slow-variation-rounding}
Let $\{Q\}$ be a cubulation of mesh $h$, and let $f: X\to\R_{\ge 0}$ be a Lipschitz function with constant
$\mathrm{Lip}(f)\le L$ on each chart used for the cubulation.
Fix $m\ge 1$ and set the target real counts
\[
n_Q := m\,h^{2p}\, f(x_Q),
\]
for chosen basepoints $x_Q\in Q$.
Define integer counts by nearest-integer rounding $N_Q:=\lfloor n_Q\rceil$.
Then for adjacent cubes $Q\sim Q'$ one has
\[
|N_Q-N_{Q'}|\ \le\ L\,m\,h^{2p+1}\ +\ 1.
\]
If moreover $f\ge f_0>0$ and $m\,h^{2p+1}\ge 2/f_0$, then there is a constant $C=C(L,f_0)$ such that
\[
|N_Q-N_{Q'}|\ \le\ C\,h\,N_Q.
\]
\end{lemma}

\begin{proof}
Nearest-integer rounding satisfies $|N_Q-N_{Q'}|\le |n_Q-n_{Q'}|+1$.
By the Lipschitz bound, $|f(x_Q)-f(x_{Q'})|\le L\,\mathrm{dist}(x_Q,x_{Q'})\le Lh$, hence
$|n_Q-n_{Q'}|\le m\,h^{2p}\cdot Lh = L\,m\,h^{2p+1}$, proving the first inequality.

If $f\ge f_0$, then $n_Q\ge m\,h^{2p} f_0$, so $N_Q\ge n_Q-1 \ge m\,h^{2p}f_0-1$.
Under $m\,h^{2p+1}\ge 2/f_0$ one has $m\,h^{2p}f_0\ge 2/h$, hence $N_Q\ge (1/h)$.
Therefore $1\le hN_Q$ and
\[
|N_Q-N_{Q'}|\le L\,m\,h^{2p+1}+1 \le \Bigl(\frac{L}{f_0}+1\Bigr)\,hN_Q,
\]
which yields the stated form.
\end{proof}


% ================================
% Vertex/discrete bookkeeping for graph templates (Batch 8)
%
\begin{lemma}[Discrete vertex-template selection]\label{lem:vertex-template-discrete}
In the setting of Remark~\ref{rem:vertex-star-coherence}, suppose the per-cube vertex counts $N_{Q,v,i}\in\Z_{\ge 0}$ are assigned so that
whenever cubes $Q,Q'$ meet along an edge through $v$, one has $|N_{Q,v,i}-N_{Q',v,i}|\le 1$.
Then there exists a canonical choice of subsets $\mathcal T_{Q,v,i}\subset \{t_{i,\ell}\}_\ell$ with $|\mathcal T_{Q,v,i}|=N_{Q,v,i}$ such that
for any two incident cubes $Q,Q'$ at $v$, the symmetric difference $\mathcal T_{Q,v,i}\triangle \mathcal T_{Q',v,i}$ has cardinality at most $1$.
\end{lemma}

\begin{proof}
Order the translation net $\{t_{i,\ell}\}_\ell$ once and for all and define $\mathcal T_{Q,v,i}$ to be the first $N_{Q,v,i}$ atoms in that order.
If $|N_{Q,v,i}-N_{Q',v,i}|\le 1$ then the two initial segments differ by at most one atom, proving the claim.
\end{proof}

%
\begin{remark}[Why Lemma~\ref{lem:vertex-template-discrete} is enough]\label{rem:vertex-template-discrete}
The lemma reduces all vertex-to-vertex variation to adding/removing a single terminal atom, which is precisely the regime in which the
slow-variation estimates (Lemmas~\ref{lem:slow-variation-rounding} and \ref{lem:slow-variation-discrepancy}) can be applied with no further combinatorics.
\end{remark}


%
\begin{proposition}[Integral transport at vertices]\label{prop:vertex-template-transport}
Let $\mu_{Q,v,i}:=\sum_{t\in\mathcal T_{Q,v,i}} \delta_t$ and $\mu_{Q',v,i}:=\sum_{t\in\mathcal T_{Q',v,i}} \delta_t$ be the vertex measures
associated with Lemma~\ref{lem:vertex-template-discrete}. Then there exists an integral optimal transport plan from $\mu_{Q,v,i}$ to $\mu_{Q',v,i}$,
and its cost is supported entirely on the at most one differing atom.
\end{proposition}

\begin{proof}
Apply Proposition~\ref{prop:integer-transport} to the two atomic measures on the common net. Since the supports differ by at most one atom,
an optimal plan moves only the unmatched unit mass and keeps all common atoms fixed.
\end{proof}


%
\begin{proposition}[Compatibility of graph and vertex bookkeeping]\label{prop:graph-vertex-compatibility}
Fix a vertex $v$ and an ordered translation net $\mathcal N=\{t_\ell\}_{\ell=1}^L$.
Assume that for each cube $Q$ incident to $v$ (and each direction label $i$) we have a graph-template measure
$\mu_{Q,i}$ supported on $\mathcal N$ and define the induced face and vertex measures by
Definition~\ref{def:graph-restriction-maps}.
Then for every face $F\subset\partial Q$ with $v\in F$,
\[
\operatorname{Res}_{F\to v}\bigl(\mu_{Q\to F,i}\bigr)=\mu_{Q,v,i},
\]
so the face and vertex bookkeeping are compatible.
In particular, if the vertex measures across the star of $v$ satisfy the one-atom discrepancy of
Lemma~\ref{lem:vertex-template-discrete}, then the same discrepancy persists after restriction from faces.
\end{proposition}

\begin{proof}
By Definition~\ref{def:graph-restriction-maps}, the restriction maps are the identity on atoms and
$\mu_{Q\to F,i}=\mu_{Q,v,i}=\mu_{Q,i}$ as atomic measures on the common ordered net.
The final sentence is immediate from Lemma~\ref{lem:vertex-template-discrete}.
\end{proof}


%
\begin{remark}[Graph/vertex compatibility remark]\label{rem:graph-vertex-compatibility}
In the discrete graph-template formalism adopted here, face and vertex ``restrictions'' of the transverse atomic measures
are defined by the identity-on-atoms maps of Definition~\ref{def:graph-restriction-maps}.  The geometric meaning is that a whole-cell
graph sheet is indexed by its translation parameter $t\in\mathcal N$, and the same parameter indexes its traces on faces and
on vertex stars whenever the overlap charts are chosen consistently (cf.\ Remark~\ref{rem:vertex-star-coherence}).
Thus no additional ``exit through $v$'' rule is required at the bookkeeping level.
\end{remark}



%
\begin{proposition}[Global coherence in the graph-template regime]\label{prop:graph-global-coherence}
Assume that the discrete graph-template data $\{\mu_{Q,i}\}$ on a common translation net satisfy the coherence conditions of
Definition~\ref{def:coherent-templates} when equipped with the restriction maps of Definition~\ref{def:graph-restriction-maps}.
Then every subsequent transport/gluing argument in the manuscript that depends only on this discrete template bookkeeping
applies in the graph-template regime \emph{verbatim} once the corresponding facewise transport plans are chosen integrally
(as in Proposition~\ref{prop:graph-template-transport}).
\end{proposition}

\begin{proof}
With Definition~\ref{def:graph-restriction-maps} in place, the graph-template bookkeeping is a special case of the abstract
template framework used throughout the transport/gluing steps: the proofs only use the atomic measures, their restriction maps,
and integrality of the transport plans.
\end{proof}


%
\begin{remark}[Global-coherence remark for graphs]\label{rem:graph-global-coherence}
Definition~\ref{def:graph-restriction-maps} supplies the missing face/vertex restriction maps for whole-cell graphs at the discrete level.
With these maps in place, the hypotheses of Proposition~\ref{prop:global-coherence-all-labels} are formulated on the same data
$\{\mu_{Q,i}\}$, so Proposition~\ref{prop:graph-global-coherence} is a genuine specialization of the general coherence lemma.
\end{remark}


%
\begin{remark}[Activation status for the graph-template branch]\label{rem:graph-activation}
\REVMZ{\textbf{[Optional proof branch.]}}
This branch is activated only if the main construction truly achieves the ``whole-cell graph'' checkpoint (Remark~\ref{rem:graph-whole-cell})
for the required translation net.  Once the restriction maps are fixed as in Definition~\ref{def:graph-restriction-maps}, the remaining steps
depend only on the discrete measures and the integrality of the chosen transport plans.
\end{remark}

%
\begin{remark}[Global summary of the graph-template branch]\label{rem:global-graph-summary}
\REVMZ{\textbf{[Optional proof branch summary.]}}
The graph-template branch isolates a particularly clean regime: on each activated cube $Q$ the relevant complete intersection is
a \emph{single} small-slope sheet on \emph{all} of $Q$ (Remark~\ref{rem:graph-whole-cell}).  In that regime, transport and gluing
reduce to discrete bookkeeping on atomic face-masses, exactly as in the sliver-template branch.

\end{remark}


% ================================
\begin{lemma}[Slow variation persists under $0$--$1$ discrepancy rounding]\label{lem:slow-variation-discrepancy}
In the setting of Lemma~\ref{lem:slow-variation-rounding}, suppose instead of nearest-integer rounding we choose integers of the form
\[
N_Q\ :=\ \lfloor n_Q\rfloor\ +\ \varepsilon_Q,
\qquad \varepsilon_Q\in\{0,1\}.
\]
Then for adjacent cubes $Q\sim Q'$ one has
\[
|N_Q-N_{Q'}|\ \le\ L\,m\,h^{2p+1}\ +\ 2.
\]
If moreover $f\ge f_0>0$ and $m\,h^{2p+1}\ge 4/f_0$, then there is a constant $C=C(L,f_0)$ such that
\[
|N_Q-N_{Q'}|\ \le\ C\,h\,N_Q.
\]
\end{lemma}
\begin{proof}
For adjacent $Q\sim Q'$, one has
\[
|N_Q-N_{Q'}|
\ \le\ |\lfloor n_Q\rfloor-\lfloor n_{Q'}\rfloor|\ +\ |\varepsilon_Q-\varepsilon_{Q'}|
\ \le\ |n_Q-n_{Q'}|\ +\ 1\ +\ 1.
\]
The Lipschitz estimate from Lemma~\ref{lem:slow-variation-rounding} gives $|n_Q-n_{Q'}|\le L\,m\,h^{2p+1}$, proving the first claim.

For the relative bound, if $f\ge f_0$ then $n_Q\ge m\,h^{2p}f_0$ and hence
$N_Q\ge \lfloor n_Q\rfloor \ge n_Q-1 \ge m\,h^{2p}f_0-1$.
Under $m\,h^{2p+1}\ge 4/f_0$ we have $m\,h^{2p}f_0\ge 4/h$, so $N_Q\ge 3/h$ and thus $2\le hN_Q$.
Therefore
\[
|N_Q-N_{Q'}|
\ \le\ L\,m\,h^{2p+1}+2
\ \le\ \Bigl(\frac{L}{f_0}+1\Bigr)\,h\,(m\,h^{2p}f_0)\ +\ hN_Q
\ \le\ \Bigl(\frac{L}{f_0}+2\Bigr)\,h\,N_Q,
\]
after absorbing $m\,h^{2p}f_0\le n_Q\le N_Q+1$ into the constant and using $1\le hN_Q$.
\end{proof}

The local sheet construction is designed so that, uniformly for these test forms $d\eta$,
\[
T^{\mathrm{raw}}(d\eta)\approx \int_X (m\beta)\wedge d\eta,
\]
with an error controlled by {$(\delta+\varepsilon+\mathrm{mesh})\cdot m$} (with $m$ fixed).
Since $\beta$ is closed and $X$ has no boundary, $\int_X (m\beta)\wedge d\eta=\pm\int_X d(m\beta\wedge \eta)=0$.



\begin{lemma}[Per-face flat control from transport]\label{lem:per-cell-flat-control}
For each interior face $F=Q\cap Q'$ and each direction label $j$, let $B_{F,j}$ be the face-mismatch current produced by the transport construction in Proposition~\ref{prop:transport-cell} (equivalently, Proposition~\ref{prop:transport-flat-glue}). Then the flat norm of $B_{F,j}$ is controlled by the corresponding discrete transport cost and template-edit errors; in particular one has an estimate of the form
\[
\mathcal F(B_{F,j})\;\le\;\mathrm{Cost}(\pi_{F,j})\;+\;C(n,p)\,\varepsilon_{\mathrm{temp}}\,h^{2n-1},
\]
for some admissible cell transport plan $\pi_{F,j}$ as in Definition~\ref{def:cell-transport-plan}.
\end{lemma}
\begin{proof}
This is exactly the flat-norm estimate established in Proposition~\ref{prop:transport-flat-glue} (and its weighted variant when invoked), once $B_{F,j}$ is written as the boundary of the corresponding prism current plus the residual template-edit term.
\end{proof}

\begin{lemma}[Flat-norm control of the gluing mismatch]\label{lem:flatnorm-gluing-mismatch}
With notation as above and with the integer $m\ge1$ \emph{fixed once and for all} (clearing denominators of $[\gamma]$),
there exists a function $\varepsilon_{\mathrm{glue}}=\varepsilon_{\mathrm{glue}}(\delta,\varepsilon,\mathrm{mesh})$ such that
\[
\mathcal F\!\left(\partial T^{\mathrm{raw}}\right)\ \le\ {\varepsilon_{\mathrm{glue}}(\delta,\varepsilon,\mathrm{mesh})}\cdot m,
\qquad
{\varepsilon_{\mathrm{glue}}\xrightarrow[\delta,\varepsilon\to0,\ \mathrm{mesh}\to0]{}0}.
\]
\end{lemma}

\begin{proof}
Let $k:=2n-2p$ so that $\partial T^{\mathrm{raw}}$ is a $(k-1)$--current.
By the Federer--Fleming
 dual characterization of the flat norm \REVMZ{\cite{FF60}}, it suffices to test $\partial T^{\mathrm{raw}}$
against smooth compactly-supported $(k-1)$--forms $\eta$ with $\|\eta\|_{\infty}\le 1$ and $\|d\eta\|_{\infty}\le 1$.
Decompose $\partial T^{\mathrm{raw}}$ as the alternating sum of face-mismatch currents across adjacent cubes in the partition.

For each codimension-one face $F$, Proposition~\ref{prop:transport-flat-glue} bounds the contribution of the face mismatch
to $\langle \partial T^{\mathrm{raw}},\eta\rangle$ by the Wasserstein transport cost $\tau_F$ plus the explicit cubewise
template/rounding error terms. Summing over all interior faces and invoking the global bookkeeping estimates from
Theorem~\ref{thm:sliver-mass-matching-on-template}, Corollary~\ref{cor:global-flat-weighted}, and
Proposition~\ref{prop:global-coherence-all-labels} yields the stated bound with
{$\varepsilon_{\mathrm{glue}}(\delta,\varepsilon,\mathrm{mesh})\to 0$ as $\delta,\varepsilon\to 0$ and $\mathrm{mesh}\to 0$} (with $m$ fixed).
\end{proof}

\begin{remark}[note: this is the quantitative bottleneck]\label{rem:lean-bottleneck-flatnorm}\label{rem:global-flat-note}
\REVMZ{\textbf{[Key technical point.]}}
For the Lean formalization, the nontrivial input encapsulated here is precisely the quantitative estimate delivered by
Proposition~\ref{prop:transport-flat-glue} and the cited bookkeeping results.
All subsequent uses of this estimate (Proposition and Proposition~\ref{prop:almost-calibration})
require only flat-norm calculus and standard filling/isoperimetric inequalities.
\end{remark}





\begin{lemma}[Federer--Fleming filling on $X$ for bounding cycles]\label{lem:FF-filling-X}
Let $X$ be a fixed compact smooth Riemannian manifold and fix an integer $k\ge 2$.
There exists a constant $C_X>0$ (depending only on $(X,k)$) with the following property:

\smallskip\noindent
If $R$ is an integral $(k-1)$--current in $X$ with $\partial R=0$ which bounds in $X$
(i.e.\ $R=\partial W$ for some integral $k$--current $W$ in $X$), then there exists an integral $k$--current $Q_R$ in $X$
such that $\partial Q_R=R$ and
\[
\Mass(Q_R)\ \le\ C_X\,\Mass(R)^{\frac{k}{k-1}}.
\]
\end{lemma}

\begin{proof}
Choose an (isometric) Nash embedding $\iota:X\hookrightarrow \R^N$ for some $N$.
Let $U$ be a tubular neighborhood of $\iota(X)$ and let $\pi:U\to X$ be the nearest--point projection.
Then $\pi$ is Lipschitz with $\Lip(\pi)$ depending only on $X$.

Since $R$ bounds in $X$, the pushforward $\iota_\# R$ bounds in $\R^N$.
\REVMZ{By the Federer--Fleming isoperimetric inequality \cite[Thm~6.1]{FF60} (see also \cite[Thm~4.2.10]{Fed69}):}
there exists an integral $k$--current $Q$ in $\R^N$ with $\partial Q=\iota_\# R$ and
\[
\Mass(Q)\ \le\ C_{N,k}\,\Mass(\iota_\# R)^{\frac{k}{k-1}}.
\]
Define $Q_R:=\pi_\# Q$, which is an integral $k$--current in $X$ \REVMZ{(pushforward under a Lipschitz map preserves integrality by \cite[Thm~4.1.14]{Fed69})}.
Then
\[
\partial Q_R\ =\ \pi_\#(\partial Q)\ =\ \pi_\#(\iota_\# R)\ =\ (\pi\circ \iota)_\# R\ =\ R.
\]
Moreover,
\[
\Mass(Q_R)\ \le\ \Lip(\pi)^k\,\Mass(Q)
\ \le\ \bigl(\Lip(\pi)^k C_{N,k}\bigr)\,\Mass(R)^{\frac{k}{k-1}},
\]
since $\iota$ is an isometric embedding and hence $\Mass(\iota_\# R)=\Mass(R)$.
Absorb the constants into $C_X$.
\end{proof}






\begin{proposition}[Microstructure/gluing estimate]\label{prop:glue-gap}\label{prop:glue-cell}
Let $T^{\mathrm{raw}}\in \mathcal I_k(X)$ be the (generally non-closed) integral $k$--current built from the microstructure pieces on a mesh of size $h$.
Set $R:=\partial T^{\mathrm{raw}}\in \mathcal I_{k-1}(X)$ and let $\delta:=\mathcal F(R)$ be the \emph{integral flat norm} from Definition~\ref{def:flat-norm}.
Then there exists an integral $k$--current $R_{\mathrm{glue}}\in\mathcal I_k(X)$ with
\[
\partial R_{\mathrm{glue}}=-R,\qquad
\Mass(R_{\mathrm{glue}})\le \delta + C_X\,\delta^{\frac{k}{k-1}},
\]
where $C_X>0$ depends only on $X$ (and $k$).
Equivalently, $U_{\mathrm{glue}}:=-R_{\mathrm{glue}}$ satisfies $\partial U_{\mathrm{glue}}=R$ and the same mass bound.
\end{proposition}


\begin{proof}
Fix $\eta>0$. By Definition~\ref{def:flat-norm} choose integral currents $R_0\in\mathcal I_{k-1}(X)$ and $Q\in\mathcal I_k(X)$ such that
\[
R=R_0+\partial Q,\qquad \Mass(R_0)+\Mass(Q)\le \delta+\eta .
\]
Then $\partial R_0=\partial R-\partial^2Q=0$, so $R_0$ is an integral $(k-1)$--cycle. Moreover $R_0$ bounds in $X$ since
\[
R_0=\partial(T^{\mathrm{raw}}+Q).
\]
Apply Lemma~\ref{lem:FF-filling-X} to $R_0$ to obtain an integral $k$--current $Q_0\in\mathcal I_k(X)$ with
\[
\partial Q_0 = R_0,\qquad \Mass(Q_0)\le C_X\,\Mass(R_0)^{\frac{k}{k-1}}.
\]
Define
\[
R_{\mathrm{glue}}:=-(Q+Q_0).
\]
Then $\partial R_{\mathrm{glue}}=-(\partial Q+\partial Q_0)=-(R-R_0+R_0)=-R$, and
\[
\Mass(R_{\mathrm{glue}})\le \Mass(Q)+\Mass(Q_0)
\le \Mass(Q)+C_X\,\Mass(R_0)^{\frac{k}{k-1}}
\le (\delta+\eta)+C_X\,(\delta+\eta)^{\frac{k}{k-1}}.
\]
Letting $\eta\downarrow 0$ yields the claimed bound.
\end{proof}




\begin{remark}[Choosing the glue scale to make the correction negligible]\label{rem:glue-scaling}\label{rem:glue-gap-remark}
Let $k=2n-2p$ and set $\delta:=\mathcal F(\partial T^{\mathrm{raw}})$.
Proposition~\ref{prop:glue-gap} yields
\[
\Mass(R_{\mathrm{glue}})\ \le\ \delta\ +\ C_X\,\delta^{\frac{k}{k-1}}.
\]
Hence $\Mass(R_{\mathrm{glue}})=o(m)$ whenever $\delta=o\!\left(m^{\frac{k-1}{k}}\right)$.
In the regime where Lemma~\ref{lem:flatnorm-gluing-mismatch} gives $\delta\le \varepsilon_{\mathrm{glue}}(m,\delta,\varepsilon,\mathrm{mesh})\cdot m$,
it is enough to choose parameters so that $\varepsilon_{\mathrm{glue}}(m,\delta,\varepsilon,\mathrm{mesh})=o\!\left(m^{-1/k}\right)$.
\end{remark}









We now return to the global construction.
Fix $\varepsilon>0$, and choose the mesh/activation parameters so that the gluing correction $R_{\mathrm{glue}}$ from
Proposition~\ref{prop:glue-gap} satisfies $\Mass(R_{\mathrm{glue}})\le\varepsilon/2$.
Define the closed glued cycle
\[
T^{(1)}:=T^{\mathrm{raw}}+R_{\mathrm{glue}}.
\]
Then $T^{(1)}$ is closed and integral.

\medskip\noindent
\textbf{Substep 4.3: Forcing the cohomology class via lattice discreteness.}
Fix harmonic $(2n-2p)$-forms $\{\eta_\ell\}_{\ell=1}^b$ whose cohomology classes form an integral basis of the free part
$H^{2n-2p}(X,\Z)/\mathrm{tors}$ \REVMZ{(see \cite{GH78,Voisin02} for Hodge theory on compact K\"ahler manifolds)}.
These harmonic representatives detect only the free part of integral cohomology, hence the period computation determines the class in
$H_{2n-2p}(X,\Z)/\mathrm{tors}$.
If one wants an equality in full integral homology, let $m_{\mathrm{tors}}$ be the exponent of the torsion subgroup of $H_{2n-2p}(X,\Z)$ and replace
$(m,T^{(1)})$ by $(m_{\mathrm{tors}}m,\ m_{\mathrm{tors}}T^{(1)})$ (and correspondingly shrink the target $\varepsilon$), which kills any possible torsion discrepancy.
The homology class of any closed integral current $T$ is determined (up to torsion) by the pairings
\[
\langle[T],[\eta_\ell]\rangle=\int_T\eta_\ell.
\]
Since $[\gamma]$ is rational, for each integral cohomology generator $\eta_\ell$
the period
\[
I_\ell:=\int_X \beta\wedge \eta_\ell\in\Q
\]
has bounded denominator.  Choose $m\ge 1$ so that $m\,I_\ell\in\Z$ for all $\ell$.

\begin{lemma}[Fixed-dimension discrepancy rounding (B\'ar\'any--Grinberg)]\label{lem:barany-grinberg}
\REVMZ{(See \cite{BaranyGrinberg81,Schrijver86}.)}
Let $d\ge 1$ and let $v_1,\dots,v_M\in\R^d$ satisfy $\|v_i\|_{\ell^\infty}\le 1$.
For any coefficients $a_1,\dots,a_M\in[0,1]$, there exist $\varepsilon_1,\dots,\varepsilon_M\in\{0,1\}$ such that
\[
\Bigl\|\sum_{i=1}^M (\varepsilon_i-a_i)\,v_i\Bigr\|_{\ell^\infty}\ \le\ d.
\]
\end{lemma}


\begin{proof}
Set $x:=\sum_{i=1}^M a_i v_i\in\R^d$ and let $V$ be the $d\times M$ matrix whose $i$th column is $v_i$.
Consider the (nonempty) polytope
\[
P\ :=\ \bigl\{\,t\in[0,1]^M:\ Vt=x\,\bigr\},
\]
which contains $a:=(a_1,\dots,a_M)$.  Choose an extreme point $t^*\in P$.
Let $F:=\{\,i:\ 0<t_i^*<1\,\}$ be the set of fractional coordinates.

Write $r:=\mathrm{rank}(V)\le d$.  The affine constraints $Vt=x$ impose $r$ independent linear equalities.
At an extreme point of $P$, at least $M$ linearly independent constraints are active; at most $r$ of them come from $Vt=x$,
so at least $M-r$ of the box constraints $t_i=0$ or $t_i=1$ must be active.
Hence $|F|\le r\le d$.

Now define $\varepsilon_i:=t_i^*$ for $i\notin F$ (so $\varepsilon_i\in\{0,1\}$) and choose any $\varepsilon_i\in\{0,1\}$ for $i\in F$.
Since $Vt^*=Va$, we have
\[
\sum_{i=1}^M (\varepsilon_i-a_i)v_i
\;=\;
\sum_{i=1}^M (\varepsilon_i-t_i^*)v_i,
\]
and only indices in $F$ contribute on the right-hand side.
For each coordinate $1\le j\le d$,
\[
\Bigl|\sum_{i=1}^M (\varepsilon_i-t_i^*)\,v_{i,j}\Bigr|
\;\le\;
\sum_{i\in F} |\varepsilon_i-t_i^*|\,|v_{i,j}|
\;\le\;
\sum_{i\in F} 1
\;=\;
|F|
\;\le\;
d,
\]
because $|\varepsilon_i-t_i^*|\le 1$ and $\|v_i\|_{\ell^\infty}\le 1$.
Taking the maximum over $j$ gives the claimed $\ell^\infty$ bound.
\end{proof}

\begin{remark}
Lemma~\ref{lem:barany-grinberg} is a standard ``rounding in fixed dimension'' discrepancy estimate
\REVMZ{\cite{BaranyGrinberg81}}.
The key feature is that the bound depends only on the dimension $d$, not on $M$.
\end{remark}

By refining the cube decomposition (so each individual sheet piece has very small contribution
to each pairing) and choosing the integers $N_{Q,j}$ using Lemma~\ref{lem:barany-grinberg}
(applied to the fractional parts of the target real counts), one can ensure that for all $\ell$,
\[
\Bigl|\int_{T^{\mathrm{raw}}}\eta_\ell - m\,I_\ell\Bigr|<\tfrac12.
\]
Moreover, the gluing correction $R_{\mathrm{glue}}$ has arbitrarily small mass (Proposition~\ref{prop:glue-gap}), hence
its pairing with each fixed smooth $\eta_\ell$ is arbitrarily small:
$\bigl|\int_{R_{\mathrm{glue}}}\eta_\ell\bigr|\le \|\eta_\ell\|_{C^0}\Mass(R_{\mathrm{glue}})$.
Choosing parameters so that this error is $<\tfrac12$ as well yields
\[
\Bigl|\int_{T^{(1)}}\eta_\ell - m\,I_\ell\Bigr|<1,
\qquad T^{(1)}=T^{\mathrm{raw}}+R_{\mathrm{glue}}.
\]
Since $\int_{T^{(1)}}\eta_\ell\in\Z$ (integral current against an integral class),
we conclude $\int_{T^{(1)}}\eta_\ell = m\,I_\ell$ for all $\ell$.
Hence
\[
[T^{(1)}]=\mathrm{PD}(m[\gamma]).
\]

Set $R_\varepsilon:=R_{\mathrm{glue}}$ and $T_\varepsilon:=T^{(1)}$.  This satisfies all requirements.
\end{proof}

Let $\{\Theta_\ell\}_{\ell=1}^{b}$ be a fixed integral basis of
$H^{2(n-p)}(X,\Z)$ represented by smooth closed forms.  Since $\beta$
represents $[\gamma]$, we have for every $\ell$,
\[
I_\ell := \int_X \beta\wedge \Theta_\ell
= \langle [\gamma], [\Theta_\ell]\rangle \in \Q.
\]
Choose a common positive integer multiplier $m=m(\gamma)$ so that
$m\,I_\ell\in\Z$ for all $\ell$.

On each cube $Q$, the current $S_Q$ constructed above satisfies, for
each $\ell$,
\[
S_Q(\Theta_\ell)
= \sum_{j,a} \int_{Y_{Q,j}^a\cap Q} \Theta_\ell
= \int_Q \Bigl(\sum_{j}\tfrac{N_{Q,j}}{m_Q}\,\xi_{\Pi_{Q,j}}\Bigr)
  \wedge \Theta_\ell + O(\eta_Q),
\]
with $\eta_Q\to 0$ as $\varepsilon,\delta\to 0$.  Summing over all cubes yields
\[
\sum_Q S_Q(\Theta_\ell)
= \int_X \beta\wedge \Theta_\ell + O\Bigl(\sum_Q \eta_Q\Bigr).
\]



\begin{lemma}[Integral periods of integral cycles]\label{lem:integral-periods}
Let $X$ be a compact manifold and let $T$ be a closed integral $k$--cycle (equivalently, an integral $k$--current with $\partial T=0$).
Let $\eta$ be a smooth closed $k$--form whose de~Rham cohomology class lies in the image of
$H^{k}(X,\Z)\to H^{k}(X,\R)$ (i.e.\ $[\eta]\in H^{k}(X,\Z)$ is an integral class).
Then
\[
\int_{T}\eta \;=\; T(\eta)\ \in\ \Z.
\]
\end{lemma}

\begin{proof}
A closed integral current $T$ determines an integral homology class $[T]\in H_{k}(X,\Z)$ (Federer--Fleming
).
An integral cohomology class $[\eta]\in H^{k}(X,\Z)$ defines an integer-valued homomorphism on $H_{k}(X,\Z)$ via the Kronecker pairing,
so $\langle [\eta],[T]\rangle\in\Z$.
Under the de~Rham isomorphism, this pairing is represented by integration of a smooth closed form representative, hence
$\langle [\eta],[T]\rangle=\int_{T}\eta$.
\end{proof}





% (Lemma~\ref{lem:lattice-discreteness} relocated to the lattice-template subsection.)

% ------------------------------------------------------------
\subsection*{Step 5: Boundary correction with vanishing mass}


\begin{lemma}[Boundary template bookkeeping]\label{lem:boundary-template}
Let $S=\sum_Q S_Q$ be the raw microstructure current built on a mesh of size $h$, and set $R:=\partial S$.
Then $R$ is supported on the union of inter-cell faces, and can be written as a finite sum of face-restrictions
of the per-cell pieces with uniformly bounded overlap (depending only on the fixed cell-complex).
\end{lemma}

\begin{proof}
Each $S_Q$ is supported in $Q$ and has boundary supported in $\partial Q$. Summing over $Q$ cancels interior contributions
except on shared faces; bounded overlap follows from bounded face-incidence in the mesh.
\end{proof}


\begin{remark}[Comment on the boundary-template lemma]\label{rem:boundary-template-comment}
Lemma~\ref{lem:boundary-template} is purely combinatorial: it records where the boundary lives and how many times each face
can be counted.  All quantitative smallness enters through the flat-norm bounds (e.g.\ Proposition~\ref{prop:glue-gap}).
\end{remark}



\begin{theorem}[Boundary correction with vanishing mass]\label{thm:boundary-correction}
Let $S\in\mathcal I_k(X)$ be an integral current and set $\delta:=\mathcal F(\partial S)$ using Definition~\ref{def:flat-norm},
with $k=2n-2p$.  Then there exists an integral $k$--current $U$ with
\[
\partial U=\partial S,\qquad \Mass(U)\le \delta + C_X\,\delta^{\frac{k}{k-1}},
\]
where $C_X$ depends only on $X$ and $k$.  In particular, if $\delta=o\!\left(m^{\frac{k-1}{k}}\right)$ then $\Mass(U)=o(m)$.
\end{theorem}

\begin{proof}
Apply Proposition~\ref{prop:glue-gap} to $T^{\mathrm{raw}}:=S$ to obtain $R_{\mathrm{glue}}$ with $\partial R_{\mathrm{glue}}=-\partial S$
and $\Mass(R_{\mathrm{glue}})\le \delta + C_X\delta^{k/(k-1)}$.  Set $U:=-R_{\mathrm{glue}}$.
\end{proof}
%
\subsection*{Optional branch: Lattice templates (integer-valued invariants)}

\begin{definition}[Lattice template]\label{def:lattice-template}
Fix an integer $b\ge 1$ (typically $b=b_k(X)=\dim H^k(X,\R)$) and write $\Z^b\subset \R^b$ for the standard lattice.
A \emph{lattice template} on a mesh $\mathcal Q_h$ is an assignment
\[
  Q\ \longmapsto\ \tau_Q\in \Z^b
\]
together with a face-restriction rule $r_F:\Z^b\to \Z^b$ for each oriented face $F$,
such that for every adjacent pair $Q,Q'$ sharing $F=Q\cap Q'$ one has \emph{compatibility}
\[
  r_F(\tau_Q)\;=\;r_F(\tau_{Q'}).
\]
The tuple $(\tau_Q)_Q$ is called \emph{coherent} if all face compatibilities hold.
\end{definition}


%
\begin{remark}[How this differs from graph/sliver templates]\label{rem:lattice-template-remark}
\REVMZ{\textbf{[Technical clarification.]}}
This package abstracts the only extra feature needed for ``integral quantization'' variants: the template data lives in an \emph{integer lattice}.
It is intentionally agnostic about the geometric origin of $\tau_Q$ (graph sheet, sliver sheet, etc.).
In applications, $\tau_Q$ is typically built from a finite set of period integrals or other integer-valued invariants.
\end{remark}


%
\begin{lemma}[Lattice template exists from integer data]\label{lem:lattice-template-exists}
Suppose one has a mesh $\mathcal Q_h$ and integer vectors $\tau_Q\in\Z^b$ with a specified family of face restriction maps $r_F$
such that $r_F(\tau_Q)=r_F(\tau_{Q'})$ for every shared face $F=Q\cap Q'$.
Then $(\tau_Q)_Q$ is a coherent lattice template in the sense of Definition~\ref{def:lattice-template}.
\end{lemma}

\begin{proof}
This is immediate from the definition: the stated face compatibilities are exactly the coherence conditions.
\end{proof}

%
\begin{remark}[Qualitative use]\label{rem:lattice-template-qualitative}
In the main proof, \emph{existence} of coherent integer data is the substantive point (usually provided by a rounding/discrepancy lemma).
Once such integer data is available, all subsequent transport/gluing arguments are formal consequences of the already-established
integral transport and flat-norm control propositions.
\end{remark}


%
\begin{proposition}[Integral transport for lattice templates]\label{prop:lattice-template-transport}
Assume a coherent lattice template $(\tau_Q)_Q$ in the sense of Definition~\ref{def:lattice-template}, and assume that for each shared face
$F=Q\cap Q'$ the induced atomic measures determined by $r_F(\tau_Q)=r_F(\tau_{Q'})$ fall under the hypotheses of
Proposition~\ref{prop:integer-transport}.  Then one can choose the per-face transport/filling currents integrally and the resulting face
correction obeys the same flat-norm bounds as in Proposition~\ref{prop:transport-flat-glue-weighted}.
\end{proposition}

\begin{proof}
This is a direct wrapper: apply Proposition~\ref{prop:integer-transport} on each face to obtain an integral optimal transport plan, and then
apply Proposition~\ref{prop:transport-flat-glue-weighted} to realize that plan by an integral filling current with the stated flat control.
Summing over faces preserves integrality and adds the bounds linearly.
\end{proof}

%
\begin{proposition}[Global assembly for lattice templates]\label{prop:lattice-template-global}
Assume a coherent lattice template $(\tau_Q)_Q$ and that a raw assembly current $T^{\mathrm{raw}}=\sum_Q T_Q$ has been built so that:
(i) each $T_Q$ is integral, (ii) across each face $F=Q\cap Q'$ the mismatch is corrected using the integral face fillings from
Proposition~\ref{prop:lattice-template-transport}, and (iii) the global flat-norm estimate of Proposition~\ref{prop:global-flat-control} applies
to this assembly.  Then the glued current is integral and its boundary flat norm satisfies the same global bound as in
Proposition~\ref{prop:global-flat-control}.
\end{proposition}

\begin{proof}
Under the stated hypotheses, this is exactly the conclusion of Proposition~\ref{prop:global-flat-control}, and integrality is preserved because
each correction piece is integral and $\mathcal I_k(X)$ is closed under finite sums.
\end{proof}

%
\begin{remark}[What is used from the lattice structure]\label{rem:lattice-flat-remark}
\REVMZ{\textbf{[Technical clarification.]}}
Beyond coherence, the only use of ``lattice'' is to enable invoking Proposition~\ref{prop:integer-transport}.  No further arithmetic input is used
in the flat-norm or boundary-correction steps.
\end{remark}


%
\begin{lemma}[Vertex control for lattice templates]\label{lem:lattice-vertex-control}
Assume a coherent lattice template $(\tau_Q)_Q$ and that the same vertex-prefix control hypotheses as in Lemma~\ref{lem:template-vertex-control}
are satisfied by the restriction maps $r_F$.  Then the vertex discrepancies are controlled by the same constants as in
Lemma~\ref{lem:template-vertex-control}.
\end{lemma}

\begin{proof}
This is the same combinatorial argument as Lemma~\ref{lem:template-vertex-control}, applied to the integer-valued restrictions $r_F(\tau_Q)$.
\end{proof}

%
\begin{remark}[Vertex control remark]\label{rem:lattice-vertex-control}
This lemma is only needed if the global coherence proof is routed through a vertex-prefix bookkeeping step.
\end{remark}


%
\begin{proposition}[Vertex compatibility for lattice templates]\label{prop:lattice-vertex-compatibility}
Assume the hypotheses of Proposition~\ref{prop:prefix-template-coherence} and Lemma~\ref{lem:lattice-vertex-control} hold for the lattice
restriction data. Then the induced vertex restrictions are compatible across the mesh in the same sense as for sliver templates.
\end{proposition}

\begin{proof}
Apply Proposition~\ref{prop:prefix-template-coherence} with the integer-valued restriction maps; the proof is unchanged.
\end{proof}

%
\begin{remark}[Vertex compatibility remark]\label{rem:lattice-vertex-compatibility}
Once vertex compatibility is established, the global coherence criterion of Proposition~\ref{prop:global-coherence-all-labels} applies.
\end{remark}


%
\begin{proposition}[Global coherence for lattice templates]\label{prop:lattice-global-coherence}
Assume the face/vertex restriction maps for the lattice template are defined and satisfy the hypotheses of
Proposition~\ref{prop:global-coherence-all-labels}. Then the lattice template is globally coherent (no forward dependencies), and hence the
global gluing/quantization pipeline may be applied exactly as in the sliver-template branch.
\end{proposition}

\begin{proof}
This is a direct invocation of Proposition~\ref{prop:global-coherence-all-labels}.
\end{proof}

%
\begin{remark}[Global coherence remark]\label{rem:lattice-global-coherence}
The only non-formal work is checking the restriction maps are actually well defined for the chosen geometric representatives.
\end{remark}


%
\begin{remark}[Lattice summary]\label{rem:lattice-summary}
\REVMZ{\textbf{[Optional proof branch summary.]}}
The lattice-template branch is a purely arithmetic overlay: once integer-valued coherent template data is produced, the previously established
transport/gluing/boundary-correction machinery applies unchanged.
\end{remark}


%
\begin{remark}[Activation status for the lattice-template branch]\label{rem:lattice-activation}
\REVMZ{\textbf{[Optional proof branch.]}}
This branch is activated only when the manuscript provides an explicit construction of integer template data $(\tau_Q)_Q$ (typically by a
discrepancy/rounding argument) \emph{and} verifies the geometric restriction maps needed for
Proposition~\ref{prop:global-coherence-all-labels}.  Otherwise, it remains a conditional wrapper.
\end{remark}


%
\begin{remark}[Activation remark]\label{rem:lattice-activation-remark}
In particular, ``integrality'' is not automatic from small calibration defect: one must specify where the integer invariants come from
(period integrals, algebraic cycle counts, etc.) and prove they are stable under the chosen approximation scheme.
\end{remark}


\begin{lemma}[Lattice discreteness]\label{lem:lattice-discreteness}
Let $z\in\Z$ and $r\in\R$ satisfy $|z-r|<\tfrac12$.  Then $z$ is the unique integer within distance $<\tfrac12$ of $r$.
\end{lemma}
\begin{proof}
If $z\neq c$ are integers then $|z-c|\ge 1$.  Thus no other integer can lie within $<\tfrac12$ of $r$.
\end{proof}



The sum $S:=\sum_Q S_Q$ is supported in the union of cubes and typically
has a boundary supported on the inter-cube faces.
By the microstructure/gluing estimate established in Proposition~\ref{prop:glue-gap}
(i.e.\ a quantitative bound forcing $\mathcal F(\partial S)\to 0$ as the local errors $\delta,\varepsilon\to 0$ and the mesh size $\to 0$).


Write $k:=2n-2p$ and $\delta:=\mathcal F(\partial S)$.
By Definition~\ref{def:flat-norm}, for any $\eta>0$ there exist \emph{integral} currents
$R$ (a $(k-1)$--current) and $Q$ (a $k$--current) in $X$ such that
\[
\partial S\ =\ R+\partial Q,
\qquad
\Mass(R)+\Mass(Q)\ \le\ \delta+\eta.
\]
Since $\partial S$ is a boundary, $R$ is null-homologous and hence bounds in $X$.
Applying Lemma~\ref{lem:FF-filling-X} yields an integral $k$--current $Q_R$ with $\partial Q_R=R$ and
\[
\Mass(Q_R)\ \le\ C_X\,\Mass(R)^{\frac{k}{k-1}}.
\]
Define
\[
U_\epsilon:=-(Q+Q_R).
\]
Then $\partial U_\epsilon=\partial S$ and
\[
\Mass(U_\epsilon)\ \le\ \Mass(Q)+\Mass(Q_R)
\ \le\ (\delta+\eta)\ +\ C_X\,(\delta+\eta)^{\frac{k}{k-1}}.
\]
Letting $\eta\downarrow 0$ gives $\Mass(U_\epsilon)\le \delta + C_X\,\delta^{\frac{k}{k-1}}$.
Therefore, once $\delta$ is arranged small enough (using Corollary~\ref{cor:global-flat-weighted} and, if $p=n/2$,
Lemma~\ref{lem:borderline-p-half}), the bound
\[
\Mass(U_\epsilon)\ <\ \min\Bigl\{\epsilon,\ \frac{1}{4\,\max_\ell\|\Theta_\ell\|_{C^0}}\Bigr\}
\]
required in Proposition~\ref{prop:cohomology-match} holds.



\[
\Mass(T_\epsilon)
\le \Mass(S) + \Mass(U_\epsilon)
\to m\int_X \beta\wedge \psi,
\]

since $\Mass(U_\epsilon)\to 0$.



\noindent\textbf{bridge (Step 21 $\rightarrow$ Step 22 $\rightarrow$ Proposition~\ref{prop:almost-calibration}).}
Let $T^{\mathrm{raw}}=T^{\mathrm{raw}}_h$ be the raw mesh current.
By Proposition~\ref{prop:glue-gap} there is a correction $R_{\mathrm{glue}}=R_{\mathrm{glue}}(h)$ with
$\partial R_{\mathrm{glue}}=-\partial T^{\mathrm{raw}}$ and $\Mass(R_{\mathrm{glue}})\to 0$.
Set $U_h:=-R_{\mathrm{glue}}$ and $T_h:=T^{\mathrm{raw}}-U_h=T^{\mathrm{raw}}+R_{\mathrm{glue}}$; then $\partial T_h=0$ and $\Mass(U_h)\to 0$.
Moreover Proposition~\ref{prop:cohomology-match} gives the required period equalities, hence $[T_h]=\mathrm{PD}(m[\gamma])$
(mod torsion), so the hypotheses of Proposition~\ref{prop:almost-calibration} are met with $\epsilon=h$.







\begin{proof}
By construction, each local sheet current $S_Q$ is holomorphic and hence $\psi$--calibrated, so their sum $S$ is $\psi$--calibrated.
In particular,
\[
\Mass(S)=\int_S\psi.
\]
Using the triangle inequality for the mass norm and $T_\epsilon=S-U_\epsilon$,
\[
\Mass(T_\epsilon)\le \Mass(S)+\Mass(U_\epsilon)=\int_S\psi+\Mass(U_\epsilon).
\]
Also,
\[
\int_{T_\epsilon}\psi=\int_S\psi-\int_{U_\epsilon}\psi.
\]
Since $\psi$ has comass $\le 1$, one has $\bigl|\int_{U_\epsilon}\psi\bigr|\le \Mass(U_\epsilon)$.
Therefore
\[
\begin{aligned}
\Def_{\mathrm{cal}}(T_\epsilon)
&= \Mass(T_\epsilon)-\int_{T_\epsilon}\psi \\
&\le \Bigl(\int_S\psi+\Mass(U_\epsilon)\Bigr)
     -\Bigl(\int_S\psi-\int_{U_\epsilon}\psi\Bigr) \\
&\le \Mass(U_\epsilon)+\bigl|\int_{U_\epsilon}\psi\bigr| \\
&\le 2\,\Mass(U_\epsilon).
\end{aligned}
\]
which proves (ii).
Item (i) is the de~Rham pairing for the fixed homology class $[T_\epsilon]$ against the closed form $\psi$,
and (iii) follows by combining (i)--(ii).
\end{proof}


\begin{remark}[The correction current need not be positive]\label{rem:correction-not-positive}
\REVMZ{\textbf{[Technical clarification.]}}
The filling currents $U_\epsilon$ (or $R_{\mathrm{glue}}$ in Substep~4.2) are produced by the flat--norm decomposition and the Federer--Fleming isoperimetric inequality. They are \emph{not required} to be $\psi$--calibrated, nor to have any positivity/type property. This causes no difficulty: the only later input is the vanishing--mass estimate $\Mass(U_\epsilon)\to 0$. In particular, if $S$ is the glued current built from coherent templates, then $T_\epsilon:=S-U_\epsilon$ represents the same homology class as $S$ and satisfies $\Mass(U_\epsilon)\to 0$; together with Proposition~\ref{prop:almost-calibration}\textnormal{(ii)} this implies the calibration defect of $T_\epsilon$ tends to $0$. Any weak limit of $T_\epsilon$ is therefore $\psi$--calibrated, and hence (by King/Harvey--Lawson) a holomorphic chain.
\end{remark}



% ------------------------------------------------------------
\subsection*{Step 6: SYR realization via varifold compactness (Theorem D)}

This step establishes that the limit of the approximating cycles is
$\psi$-calibrated and realizes the SYR property \REVMZ{(see Allard \cite{Allard72} and Simon \cite{Sim83} for varifold compactness)}.

\begin{theorem}[SYR Realization]\label{thm:syr-realization}
Assume the local-sheet construction of Theorem~\ref{thm:local-sheets} producing a $\psi$--calibrated sheet current $S$, and let $U_\varepsilon$ be the gluing correction current constructed in Proposition~\ref{prop:glue-gap} so that $T_\varepsilon:=S-U_\varepsilon$ is a closed integral current.  Assume Proposition~\ref{prop:cohomology-match} so that $[T_\varepsilon]=\mathrm{PD}(m[\gamma])$ in $H_{2n-2p}(X;\R)$ (hence in $H_{2n-2p}(X;\Q)$ modulo torsion), and Proposition~\ref{prop:almost-calibration} so that $0\le \Mass(T_\varepsilon)-\int_{T_\varepsilon}\psi\le 2\,\Mass(U_\varepsilon)\to 0$.
the sequence $T_\varepsilon$ has:
\begin{enumerate}
\item[\textnormal{(i)}] $\Mass(T_\varepsilon)\to m\int_X\beta\wedge\psi$;
\item[\textnormal{(ii)}] A subsequential limit $T$ that is $\psi$-calibrated
and represents $\mathrm{PD}(m[\gamma])$ in $H_{2n-2p}(X;\R)$ (equivalently in $H_{2n-2p}(X;\Z)/\mathrm{Tor}$).
\end{enumerate}
In particular, $\beta$ is SYR-realizable in the sense of Definition~\ref{def:syr}.
\end{theorem}

\begin{proof}
The proof proceeds in four substeps.

\medskip\noindent
\textbf{Substep 6.1: Uniform mass bound and homology class.}
Set $T_k:=T_{1/k}$ and write $U_{1/k}$ for the gluing correction from Proposition~\ref{prop:glue-gap}.  By Proposition~\ref{prop:almost-calibration} we have

\[
\Mass(T_k)\le \int_{T_k}\psi + 2\,\Mass(U_{1/k})
= \langle[T_k],[\psi]\rangle + 2\,\Mass(U_{1/k})
= m\int_X\beta\wedge\psi + 2\,\Mass(U_{1/k}),
\]

and $\Mass(U_{1/k})\to 0$ as $k\to\infty$.
By construction, $T_k$ is obtained from the raw current $T^{\mathrm{raw}}_{1/k}$ by the gluing correction from Proposition~\ref{prop:glue-gap}:
write $R_{\mathrm{glue}}=R_{\mathrm{glue}}(1/k)$ and set $U_{1/k}:=-R_{\mathrm{glue}}$, so $T_k=T^{\mathrm{raw}}_{1/k}-U_{1/k}=T^{\mathrm{raw}}_{1/k}+R_{\mathrm{glue}}$ and $\partial T_k=0$.
Proposition~\ref{prop:cohomology-match} gives the period equalities against the chosen integral basis $\{\Theta_\ell\}$, hence $[T_k]=\mathrm{PD}(m[\gamma])$ (mod torsion).
(Equivalently, Proposition~\ref{prop:almost-calibration} isolates this global mass control in the sharper ``almost--calibration'' form
$0\le \Mass(T_k)-\int_{T_k}\psi\le 2\,\Mass(U_{1/k})=o(1)$, together with the exact pairing
$\int_{T_k}\psi=m\int_X\beta\wedge\psi$.)
By the calibration inequality applied to any
cycle $S$ in class $\mathrm{PD}(m[\gamma])$:
\[
\Mass(S)\ge\langle[S],[\psi]\rangle=\langle\mathrm{PD}(m[\gamma]),[\psi]\rangle
=m\int_X\gamma\wedge\psi=m\int_X\beta\wedge\psi.
\]
Thus $\Mass(T_k)\ge m\int_X\beta\wedge\psi-o(1)$ as well.  We conclude:
\begin{itemize}
\item $\sup_k\Mass(T_k)<\infty$;
\item All $T_k$ are cycles: $\partial T_k=0$;
\item Their homology class is constant: $[T_k]=\mathrm{PD}(m[\gamma])$.
\end{itemize}
This is the compactness/\allowbreak normalization needed for Federer--Fleming


\medskip\noindent
\textbf{Substep 6.2: Compactness (currents) \cite[\S\S 4.1--4.2]{Fed69}.}

By Federer--Fleming compactness \REVMZ{\cite{FF60,Fed69}} for integral currents on the compact manifold $X$, using $\sup_k\Mass(T_k)<\infty$ and $\partial T_k=0$, after passing to a subsequence we obtain $T_k\to T$ in the flat norm. \begin{itemize}
\item $T_k\to T$ as integral currents in the flat norm;
\item By Lemma~\ref{lem:flat_limit_of_cycles_is_cycle}, since $T_k\to T$ in the flat norm (hence weakly) and $\partial T_k=0$, we have $\partial T=0$; thus $T$ is an integral $(2n-2p)$-cycle;
\item (Homology identification.) For every smooth closed $(2n-2p)$--form $\eta$ on $X$,
\[
\langle T,\eta\rangle=\lim_{k\to\infty}\langle T_k,\eta\rangle
=\bigl\langle \mathrm{PD}(m[\gamma]),[\eta]\bigr\rangle.
\]
Hence $[T]=\mathrm{PD}(m[\gamma])$ in $H_{2n-2p}(X;\R)$ (and therefore in $H_{2n-2p}(X;\Q)$ after quotienting by torsion).
\end{itemize}
Lower semicontinuity gives
\begin{equation}\label{eq:mass-lsc}
\Mass(T)\le\liminf_{k\to\infty}\Mass(T_k)\le m\int_X\beta\wedge\psi.
\end{equation}

\medskip\noindent
\textbf{Substep 6.4: Calibration of the limit.}
Since $\psi$ is closed and $[T_k]=\mathrm{PD}(m[\gamma])$, the pairing $\langle T_k,\psi\rangle$ is constant in $k$.
\[
\langle T_k,\psi\rangle=\langle[T_k],[\psi]\rangle=m\int_X\beta\wedge\psi\qquad\text{for all }k.
\]
By weak convergence $T_k\rightharpoonup T$ and closedness of $\psi$, we have
\[
\langle T,\psi\rangle=\lim_{k\to\infty}\langle T_k,\psi\rangle=m\int_X\beta\wedge\psi.
\]
By Proposition~\ref{prop:almost-calibration}, the calibration defect satisfies $\Def_{\mathrm{cal}}(T_k)=\Mass(T_k)-\int_{T_k}\psi\le 2\,\Mass(U_{1/k})\to 0$, hence $\Mass(T_k)\to m\int_X\beta\wedge\psi$.
Combining with \eqref{eq:mass-lsc} and the calibration inequality $\langle T,\psi\rangle\le \Mass(T)$ yields $\Mass(T)=\langle T,\psi\rangle$, so $T$ is $\psi$-calibrated. (This is exactly Lemma~\ref{lem:limit_is_calibrated} in this setting.)
In particular, $\Mass(T)=m\int_X\beta\wedge\psi$ and $[T]=\mathrm{PD}(m[\gamma])$.

\textbf{Conclusion:} We have established:
\begin{enumerate}
\item Mass convergence / vanishing calibration defect:
$\Mass(T_k)\to m\int_X\beta\wedge\psi$ and $\Def_{\mathrm{cal}}(T_k)\to 0$;
\item Limit cycle: $T$ is an integral $\psi$-calibrated $(2n-2p)$-cycle
with $[T]=\mathrm{PD}(m[\gamma])$.
\end{enumerate}
Thus $\beta$ is SYR-realizable in the sense of Definition~\ref{def:syr}.
\end{proof}



\begin{remark}[Comment on Theorem~\ref{thm:syr-realization}]\label{rem:syr-realization-comment}
\REVMZ{\textbf{[Key technical point.]}}
Theorem~\ref{thm:syr-realization} is the point where the analytic construction is converted into a calibrated limit cycle.
The main inputs are: uniform mass/boundary control, compactness (as currents/varifolds), and the calibration inequality
$\Mass(T)\ge \int_T\psi$ with equality characterizing $\psi$--calibrated currents.
\end{remark}



\begin{remark}[What compactness is used]\label{rem:varifold-compactness-used}
\REVMZ{\textbf{[Technical clarification.]}}
In this paper we only use Federer--Fleming compactness for integral currents: uniform mass bounds and $\partial T_k=0$ on the compact manifold $X$
yield a subsequence converging in the flat norm.
No stationarity/first-variation hypotheses are required for this step.  \REVMZ{(One may also consider the associated integral varifolds, but this is optional.)}
\end{remark}




\begin{corollary}[SYR limit is a holomorphic (hence algebraic) cycle]\label{cor:syr-limit-holomorphic-chain}
Under the hypotheses of Theorem~\ref{thm:syr-realization}, any subsequential flat limit $T$ is a $\psi$--calibrated integral $(2n-2p)$--cycle representing $\mathrm{PD}(m[\gamma])$ in real homology. In particular, $T$ is a positive holomorphic $(n-p)$--cycle (a holomorphic chain).
If $X$ is projective, then by Chow/GAGA this holomorphic cycle is algebraic, so $\gamma$ is represented by an algebraic cycle with rational coefficients.
\end{corollary}

\begin{proof}
By Theorem~\ref{thm:syr-realization}, any subsequential limit $T$ of the cycles $T_\varepsilon$ is an integral $(2n-2p)$--cycle, is $\psi$--calibrated, and satisfies $[T]=\mathrm{PD}(m[\gamma])$.

\REVMZ{By the Harvey--Lawson structure theorem \cite[Thm~4.2]{HL82} (see also King \cite[Thm~4.5]{King71}):} Since $T$ is an integral $\psi$--calibrated cycle where $\psi$ is the Wirtinger calibration, $T$ is a positive holomorphic $(n-p)$--cycle.

If $X$ is projective, then each analytic component is algebraic by Chow/GAGA
; see \REV{\cite{Serre56,GH78,Hartshorne77}}. Hence $[\gamma]$ is represented by an algebraic cycle with rational coefficients.
\end{proof}


\begin{remark}[Why the mass defect vanishes]\label{rem:mass-vanishes}
If $T_j\to T$ in flat norm and $\int_{T_j}\psi\to \int_T\psi$ while $\Mass(T_j)-\int_{T_j}\psi\to 0$,
then lower semicontinuity of mass gives $\Mass(T)\le \liminf\Mass(T_j)=\lim\int_{T_j}\psi=\int_T\psi\le \Mass(T)$,
hence $\Mass(T)=\int_T\psi$ and $T$ is $\psi$--calibrated.
\end{remark}



\begin{remark}[What about integrality of the limit?]\label{rem:what-about-integrality}
\REVMZ{\textbf{[Technical clarification.]}}
Under uniform bounds on $\Mass(T_j)$ and $\Mass(\partial T_j)$, Federer--Fleming compactness \REVMZ{\cite[Thm~4.2.17]{Fed69}} implies that any flat limit of integral
currents is again an integral current.  Thus integrality is preserved along the approximating sequence in Theorem~\ref{thm:syr-realization}.
\end{remark}




% ------------------------------------------------------------

% ------------------------------------------------------------
\subsection*{Clarifications on the SYR construction}
% ------------------------------------------------------------

The SYR scheme produces a sequence of integral $(2n-2p)$--cycles $(T_k)$ representing a fixed
homology class and whose \emph{calibration defect} tends to zero with respect to the fixed
K\"ahler calibration $\psi$ used throughout.

\begin{remark}[A calibrated limit need not approximate $\beta$ pointwise]\label{rem:density-mass}
\REVMZ{\textbf{[Technical clarification.]}}
The construction is not a pointwise approximation of a smooth form by a measure on all of $X$.
Rather, each $T_k$ is an \emph{integral cycle} (a rectifiable current with integer multiplicity and
$\partial T_k=0$) in the fixed class $\mathrm{PD}(m[\gamma])$, and the quantitative control is through
the calibration inequality
\[
\langle T_k,\psi\rangle \ \le\ \Mass(T_k),
\qquad
\Def_{\mathrm{cal}}(T_k)\ :=\ \Mass(T_k)-\langle T_k,\psi\rangle\ \ge\ 0 .
\]
Since $\psi$ is closed, the pairing $\langle T_k,\psi\rangle$ depends only on the homology class
of $T_k$; hence it is constant in $k$.  Therefore
\[
\Mass(T_k)\ =\ \langle T_k,\psi\rangle + \Def_{\mathrm{cal}}(T_k)
\]
is uniformly bounded once $\Def_{\mathrm{cal}}(T_k)\to 0$.  No ``everywhere density'' of $\operatorname{supp}(T_k)$
is asserted or required; the limiting object is a calibrated integral cycle supported on a
$(2n-2p)$--rectifiable set, which is exactly the geometric output used in the final algebraicity step.
\end{remark}

\begin{remark}[From K\"ahler calibration to analytic cycles]\label{rem:hl-applicable}
\REVMZ{\textbf{[Key conceptual point.]}}
Let $T$ be an integral $(2n-2p)$--cycle on a complex manifold $(X,\omega)$ which is calibrated by the
K\"ahler $(2n-2p)$--form $\psi=\omega^{n-p}/(n-p)!$ (equivalently, $\Def_{\mathrm{cal}}(T)=0$ with respect
to $\psi$).  Then $T$ is a closed, strongly positive current of bidimension $(p,p)$.
By the calibrated--geometry regularity and the complex--analytic characterization of such currents,
a calibrated integral cycle is a \emph{positive holomorphic chain}, i.e. a finite sum of irreducible
complex analytic $(n-p)$--subvarieties with positive integer multiplicities; see
Harvey--Lawson~\cite{HL82} and the analytic--current structure theorem of King~\cite{King71}.
This is the point at which the argument passes from geometric measure theory to complex geometry.

\REV{For classical discussion of analytic cycles and related topological obstructions (including torsion phenomena), see \cite{AtiyahHirzebruch62}.}
\end{remark}

\begin{remark}[Non-integrability does not obstruct the construction]\label{rem:gluing}
\REVMZ{\textbf{[Technical clarification.]}}
The local plane decomposition $x\mapsto\beta(x)$ is used only to select, on each sufficiently small cube
$Q$, finitely many \emph{separate} calibrated model planes (via Carath\'eodory) and then to realize each
such plane by an algebraic complete intersection inside $Q$ (Bertini-type input).
No step requires integrating the varying plane field $\beta(x)$ into a single foliation, so the Frobenius
integrability condition is irrelevant here.

The global cycle is obtained by summing the cube-wise pieces and then correcting the boundary mismatch by a
flat-norm filling.  The correction uses the Federer--Fleming deformation/isoperimetric mechanism (classical
input~\cite{FF60}) in the quantitative form already recorded in the gluing theorem of the paper.
The resulting corrected current is still an \emph{integral} cycle in the same homology class, and its
calibration defect remains controlled by the established flat-norm and mass estimates.
\end{remark}


\subsection*{Automatic SYR: summary theorem}

\begin{theorem}[Automatic SYR for cone-valued forms]\label{thm:automatic-syr}
Let $(X,\omega)$ be a smooth complex projective manifold of complex
dimension $n$, and let $1\le p\le \frac{n}{2}$.
(For $p>\frac{n}{2}$ one reduces to the complementary degree $n-p$ by Hard Lefschetz; see Remark~\ref{rem:lefschetz-reduction}.)
Let $\beta$ be a smooth closed cone--valued $(p,p)$--form representing a rational Hodge class $[\gamma]\in H^{p,p}(X;\Q)$.
Then $\beta$ is SYR--realizable in the sense of Definition~\ref{def:syr}; equivalently,
there exist integral $(2n-2p)$--cycles $T_k$ with $\partial T_k=0$ and
$[T_k]=\mathrm{PD}(m[\gamma])$ in $H_{2n-2p}(X;\Z)/\mathrm{Tor}$ (equivalently in $H_{2n-2p}(X;\Q)$) for some fixed $m\in\N$ independent of $k$, such that
{\noindent\textbf{Parameter schedule.} The integer $m$ is fixed first (clearing denominators / co{\,(invoked with fixed clearing integer $m$, then choose $(h,s)$, then choose $\mhol$ so that Lemma~\ref{lem:bergman-control} applies on the needed $B_{c\mhol^{-1/2}}$ and the graph displacement is $\ll s$).}homology quantization).  Given a tolerance for the calibration defect, one then chooses auxiliary parameters (cubulation scale $h$, footprint scale $s$, and holomorphic power $\mhol$) as in Theorem~\ref{thm:global-cohom} to construct the approximating cycles $T_k$; only these auxiliary parameters depend on the tolerance.}
\[
\Def_{\mathrm{cal}}(T_k)=\Mass(T_k)-\langle T_k,\psi\rangle\ \longrightarrow\ 0.
\]
Consequently, $[\gamma]$ is algebraic.
\end{theorem}
\begin{remark}\label{rem:automatic-syr-comment}
this packaged theorem invokes (i) Hard Lefschetz to reduce $p>\frac n2$ to $n-p$ (Remark~\ref{rem:lefschetz-reduction}); (ii) Bergman/H\"ormander input to build local sheets at the holomorphic scale (Lemma~\ref{lem:bergman-control} and Theorem~\ref{thm:local-sheets}); (iii) Federer--Fleming flat--norm gluing and boundary correction (e.g., Proposition~\ref{prop:glue-gap}); and (iv) the calibrated-limit identification (King/Harvey--Lawson) to conclude the limiting integral current is a holomorphic chain.
\end{remark}


\begin{proof}
Fix a mesh parameter $\epsilon>0$.
The construction in Proposition~\ref{prop:global-coherence-all-labels} (built from Theorem~\ref{thm:local-sheets} and Proposition~\ref{prop:finite-template}) produces a ``raw'' integral current $S_\epsilon$
built as a finite sum of local $\psi$--cal{\,(invoked with fixed clearing integer $m$, then choose $(h,s)$, then choose $\mhol$ so that Lemma~\ref{lem:bergman-control} applies on the needed $B_{c\mhol^{-1/2}}$ and the graph displacement is $\ll s$).}ibrated sheets (so $\Mass(S_\epsilon)=\langle S_\epsilon,\psi\rangle$ on each cell),
whose boundary is supported in the gluing region.
Proposition~\ref{prop:glue-gap} provides an integral current $U_\epsilon$ with
\(
\partial U_\epsilon=\partial S_\epsilon
\)
and
\(
\Mass(U_\epsilon)\to 0
\)
as $\epsilon\to 0$.
Define the closed cycle
\(
T_\epsilon:=S_\epsilon-U_\epsilon.
\)

The cohomological bookkeeping (Theorem~\ref{thm:global-cohom}, together with Proposition~\ref{prop:cohomology-match})
(Here $m$ is fixed first; the auxiliary parameters are chosen after a target error $\varepsilon>0$ and may depend on $\varepsilon$.)
implies that for some fixed $m\in\N$ independent of $\epsilon$,
\(
[T_\epsilon]=\mathrm{PD}(m[\gamma])\in H_{2n-2p}(X;\Z)/\mathrm{Tor}\quad\text{(equivalently in }H_{2n-2p}(X;\Q)\text{)}.
\)
Finally, Proposition~\ref{prop:almost-calibration} yields the quantitative almost--calibration estimate
\[
\Def_{\mathrm{cal}}(T_\epsilon)\le 2\,\Mass(U_\epsilon)\ \longrightarrow\ 0.
\]
Choosing any sequence $\epsilon_k\downarrow 0$ and setting $T_k:=T_{\epsilon_k}$ gives the SYR sequence required by
Definition~\ref{def:syr}.
Applying Theorem~\ref{thm:syr} concludes that $[\gamma]$ is represented by a holomorphic chain and, since $X$ is projective (Chow/GAGA
 (see \REV{\cite{Hartshorne77,GH78,Serre56}})),
is algebraic.
\end{proof}






% ============================================================
\subsection*{Signed decomposition: the complete step}
% ============================================================

The preceding machinery applies to \emph{cone--positive} classes---those admitting
smooth closed cone-valued representatives.  The following lemma shows that \emph{every} rational
Hodge class reduces to this case.

\begin{definition}[Cone--positive class (smooth $K_p$--positive)]
A cohomology class $\gamma \in H^{2p}(X,\R) \cap H^{p,p}(X)$ is called
\emph{cone--positive} if there exists a smooth closed $(p,p)$--form $\beta$
representing $\gamma$ such that $\beta(x) \in K_p(x)$ for all $x \in X$.
(We avoid the word ``effective'' here, which in algebraic geometry refers to \emph{algebraic} cycles with nonnegative coefficients.)
\end{definition}

\begin{lemma}[Strict interior positivity of the K\"ahler power]\label{lem:kahler-positive}
The $(p,p)$--form $\omega^p$ is strictly positive in the calibrated cone: for all $x\in X$,
\[
\omega^p(x)\in \mathrm{int}\,K_p(x).
\]
Moreover, there exists a uniform radius $r_0=r_0(X,\omega,p)>0$ such that for every $x\in X$,
\[
B\bigl(\omega^p(x),\,r_0\bigr)\ \subset\ K_p(x),
\]
where $B(\cdot,r_0)$ denotes the ball in $\Lambda^{p,p}T_x^*X$ for the pointwise metric induced by $\omega$.
\end{lemma}


\begin{proof}
Fix $x\in X$ and choose a unitary frame for $(T^{1,0}_xX,\omega_x)$.
In these coordinates, $\omega_x=\frac{i}{2}\sum_{j=1}^n dz^j\wedge d\bar z^j$, hence
$\omega_x^p$ is a strictly (strongly) positive $(p,p)$-form: it is a positive linear
combination of decomposable forms $i^p\,\eta\wedge\bar\eta$ with $\eta\in\Lambda^{p,0}$.
Equivalently, $\omega_x^p$ lies in the interior of the cone of strongly positive $(p,p)$-forms.
Since $X$ is compact and the cone varies continuously with $x$, there exists a uniform
$\delta>0$ such that the Euclidean ball $B_\delta(\omega_x^p)$ is contained in the
cone at every $x$.  For background on positivity cones see Harvey--Lawson
~\cite{HL82}
or Demailly
~\cite{Demailly12}.
\end{proof}


\begin{lemma}[Signed Decomposition]\label{lem:signed-decomp}\label{thm:signed-decomp}
Let $\gamma \in H^{2p}(X,\Q) \cap H^{p,p}(X)$ be any rational Hodge class. Assume moreover that the fixed K\"ahler class $[\omega]\in H^2(X,\Q)$ (so in particular $[\omega^p]\in H^{2p}(X,\Q)$).
Then there exist cone--positive classes $\gamma^+$ and $\gamma^-$ such that
\[
\gamma \;=\; \gamma^+ - \gamma^-.
\]
Moreover, both $\gamma^+$ and $\gamma^-$ are rational Hodge classes,
and $\gamma^-$ can be taken to be a positive rational multiple of $[\omega^p]$.
\end{lemma}

\begin{proof}
Let $\alpha$ be any smooth closed $(p,p)$--form representing $\gamma$.

Let $r_0>0$ be the uniform interior radius from Lemma~\ref{lem:kahler-positive}.
Set
\[
M\ :=\ \sup_{x\in X}\|\alpha(x)\|\ <\ \infty,
\]
finite by compactness of $X$ and smoothness of $\alpha$.
Choose $N\in\Q_{>0}$ with $N> M/r_0$ (possible since $\Q$ is dense in $\R$).
Then for every $x\in X$ we have $\|\alpha(x)/N\|<r_0$, hence
\[
\omega^p(x) + \frac{1}{N}\alpha(x)\ \in\ B\bigl(\omega^p(x),r_0\bigr)\ \subset\ K_p(x).
\]
Since $K_p(x)$ is a cone, multiplying by $N$ yields $\alpha(x)+N\,\omega^p(x)\in K_p(x)$ for all $x$.


Define $\gamma^+ := \gamma + N \cdot [\omega^p]$ and
$\gamma^- := N \cdot [\omega^p]$.
Then $\gamma = \gamma^+ - \gamma^-$ by construction,
$\gamma^+$ is cone--positive (represented by the cone-valued form
$\alpha + N \cdot \omega^p$),
$\gamma^-$ is cone--positive (represented by $N \cdot \omega^p$),
and both are rational Hodge classes since $[\omega^p]\in H^{2p}(X,\Q)$ by the hypothesis $[\omega]\in H^2(X,\Q)$.
\end{proof}
\begin{corollary}[Reduction to cone--positive classes]\label{cor:signed-decomp-usage}
To prove algebraicity of a rational Hodge class $\gamma\in H^{2p}(X,\Q)\cap H^{p,p}(X)$, it suffices to prove algebraicity for the cone--positive summands appearing in Lemma~\ref{lem:signed-decomp}, because $\gamma$ is a $\Q$--linear combination of those summands and the space of rational $(p,p)$ classes is a $\Q$--vector space.
\end{corollary}

\begin{remark}\label{rem:signed-decomp-comment}
the signed decomposition is finite-dimensional linear algebra inside $H^{2p}(X,\Q)\cap H^{p,p}(X)$ \REVMZ{\cite{Voisin02}}; it uses only rationality and the existence of a K\"ahler class (after clearing denominators), with no geometric measure input.
\end{remark}



\begin{lemma}[$\omega^p$ is algebraic]\label{lem:gamma-minus-alg}
Let $X$ be a smooth complex projective manifold and fix an ample line bundle $L\to X$
as in the global assumptions, so that $[\omega]=c_1(L)$.
Then the class $[\omega^p]=c_1(L)^p\in H^{2p}(X,\Q)\cap H^{p,p}(X)$ lies in the $\Q$--span
of algebraic cycle classes.  More concretely, for $\mhol\gg 0$ there exists a smooth complete
intersection $Z\subset X$ of codimension $p$, cut out by $p$ generic divisors in the
linear system $|L^{\otimes \mhol}|$, such that
\[
\mathrm{PD}([Z]) \;=\; c_1(L^{\otimes \mhol})^p \;=\; \mhol^p\,[\omega^p].
\]
In particular, any rational multiple of $[\omega^p]$ is an algebraic class.
\end{lemma}

\begin{proof}
Choose $\mhol\gg 0$ so that $L^{\otimes \mhol}$ is very ample and basepoint free.
Let $D_1,\dots,D_p\in |L^{\otimes \mhol}|$ be generic divisors and set
\[
Z \;:=\; D_1\cap\cdots\cap D_p.
\]


By Bertini
\emph{Bertini/complete-intersection smoothness requires $L^{\otimes \mhol}$ very ample (or basepoint free) and generic sections; check very ampleness/basepoint-freeness and generic choice.}
's theorem, $Z$ is smooth of codimension $p$ for generic choices
(see e.g.\ Hartshorne~\cite{Hartshorne77}).
In cohomology,
\[
\mathrm{PD}([Z]) \;=\; c_1(\mathcal O(D_1))\smile\cdots\smile c_1(\mathcal O(D_p))
\;=\; c_1(L^{\otimes \mhol})^p \;=\; \mhol^p\,c_1(L)^p \;=\; \mhol^p\,[\omega^p],
\]
as claimed.
\end{proof}


\begin{theorem}[Cone--positive classes are algebraic]\label{thm:effective-algebraic}
Let $\gamma^+ \in H^{2p}(X,\Q) \cap H^{p,p}(X)$ be a cone--positive rational
Hodge class on a smooth complex projective manifold, and assume $p\le n/2$.
Then $\gamma^+$
is algebraic.
\end{theorem}

\begin{proof}

Let $\beta$ be a smooth closed cone--valued $(p,p)$--form representing the class $\gamma^+$ (this is the meaning of cone--positivity in the manuscript).
By Theorem~\ref{thm:automatic-syr}, $\beta$ is SYR--realizable in the sense of Definition~\ref{def:syr}; in particular there exists $m\in\N$ and integral cycles $T_k$ with
\noindent\emph{Parameter discipline.} Fix this integer $m$ once; all subsequent choices of mesh size $h$, holomorphic power $\mhol$, and tolerances are made with $m$ held fixed.
$[T_k]=\mathrm{PD}(m[\gamma^+])$ in $H_{2n-2p}(X;\Z)/\mathrm{Tor}$ (equivalently in $H_{2n-2p}(X;\Q)$) and $\Def_{\mathrm{cal}}(T_k)\to 0$.
Applying Theorem~\ref{thm:syr} to this SYR data produces a $\psi$--calibrated integral cycle
\(T=\sum_j m_j[V_j]\)
with $[T]=\mathrm{PD}(m[\gamma^+])$ in $H_{2n-2p}(X;\Z)/\mathrm{Tor}$ (equivalently in $H_{2n-2p}(X;\Q)$), hence an analytic cycle representative of $m[\gamma^+]$.
Since $X$ is projective, each $V_j$ is algebraic by Chow/GAGA
 (see \REV{\cite{Hartshorne77,GH78,Serre56}}), so $m[\gamma^+]$ is algebraic and therefore $\gamma^+$ is algebraic.


\end{proof}



\begin{remark}[Chow/GAGA for analytic subvarieties]\label{rem:chow-gaga}
\REVMZ{\textbf{[Standard background.]}}
If $X$ is projective, any complex analytic subvariety of $X$ is algebraic.
\REVMZ{This follows from Chow's theorem \cite{Chow49} for projective space together with Serre's GAGA \cite[Thm~1]{Serre56}; see also \cite[App.~B]{GH78} and \cite[Ch.~I, \S1]{Hartshorne77}.}
\end{remark}


% ============================================================

\begin{conjecture}[Hodge Conjecture]\label{conj:hodge-conjecture}
\REV{(See \cite{Voisin02,Lewis99} for standard references and surveys.)}
Let $X$ be a smooth complex projective manifold. For each $p$, every rational cohomology class $\gamma\in H^{2p}(X,\Q)\cap H^{p,p}(X)$ is represented by an algebraic cycle with rational coefficients.
\end{conjecture}

\begin{remark}[Context for the main theorem]\label{rem:hodge-context}
\REVMZ{\textbf{[Proof roadmap.]}}
The main theorem below proves Conjecture~\ref{conj:hodge-conjecture} by reducing to cone--positive classes via signed decomposition (Lemma~\ref{lem:signed-decomp}) and then applying the SYR/calibrated realization pipeline (Theorems~\ref{thm:automatic-syr} and~\ref{thm:syr}).
\end{remark}
\subsection*{Main theorem: Hodge conjecture for rational $(p,p)$ classes}
% ============================================================

\begin{theorem}[Hodge Conjecture for rational $(p,p)$ classes]
\label{thm:main-hodge}
Let $X$ be a smooth complex projective manifold.  Then every rational Hodge
class $\gamma \in H^{2p}(X,\Q) \cap H^{p,p}(X)$ is algebraic.
\end{theorem}

\begin{proof}

By Remark~\ref{rem:lefschetz-reduction} it suffices to treat the range $p\le n/2$.
Let $\gamma\in H^{2p}(X,\Q)\cap H^{p,p}(X)$.
Apply Lemma~\ref{lem:signed-decomp} to write
$\gamma=\gamma^{+}-\gamma^{-}$ where $\gamma^{+}$ and $\gamma^{-}$ are cone--positive
rational Hodge classes and $\gamma^{-}=N\,[\omega^{p}]$ for some $N\in\Q_{>0}$.
By Lemma~\ref{lem:gamma-minus-alg}, the class $[\omega^{p}]$ (hence $\gamma^{-}$) is algebraic.
By Theorem~\ref{thm:effective-algebraic}, the cone--positive class $\gamma^{+}$ is algebraic.
Since algebraic classes form a $\Q$--vector subspace of $H^{2p}(X,\Q)$, the difference
$\gamma=\gamma^{+}-\gamma^{-}$ is algebraic.

\end{proof}
\begin{remark}[Discussion of Theorem~\ref{thm:main-hodge}]\label{rem:main-hodge-discussion}
\REVMZ{\textbf{[Key conceptual point.]}}
The proof of Theorem~\ref{thm:main-hodge} isolates the analytic realization step: it suffices to construct,
for a fixed integer $m$, an effective holomorphic chain representing $m[\gamma^{+}]$ with the required calibration and mass
control.  Once such a holomorphic cycle is obtained, the passage to algebraic cycles on a projective manifold is standard:
Chow's theorem \REVMZ{\cite{Chow49}} identifies analytic cycles with algebraic cycles, and GAGA \REVMZ{\cite{Serre56}} transfers the resulting cycle class
to the algebraic category \REVMZ{(see also \cite{GH78,Hartshorne77})}.
In this sense, the substantial work lies in the existence and quantitative control of the holomorphic realization
within the pipeline developed in Sections~8--9.
\end{remark}






\begin{corollary}[Full Hodge conjecture]\label{cor:full-hodge}
	Every rational $(p,p)$ class on a smooth complex projective manifold is represented
	by an algebraic cycle.
\end{corollary}

\begin{proof}
Let $\alpha\in H^{2p}(X;\Q)\cap H^{p,p}(X)$.  This is exactly the hypothesis of Theorem~\ref{thm:main-hodge}, which shows that
$\alpha$ is an \emph{algebraic class}.  Equivalently, there exists an algebraic cycle (with rational coefficients) whose cohomology
class equals $\alpha$.  Hence $\alpha$ is represented by an algebraic cycle.

\end{proof}

\begin{remark}[Why signed decomposition is the key]
\REVMZ{\textbf{[Key conceptual point.]}}
(Invoked after fixing $m$ and choosing $\mhol$ so that the SYR construction is closed and the integral filling is at scale $\ll 1$; no new ``choose small'' parameters are introduced here.)
The signed decomposition sidesteps the fundamental obstruction that the
harmonic representative $\gamma_{\mathrm{harm}}$ of a general Hodge class
need not be cone-valued.  For classes like $[\pi_1^*\omega_1] - [\pi_2^*\omega_2]$
on a product surface, the harmonic form has indefinite signature everywhere.
We do \emph{not} claim that every Hodge class has a cone-valued representative;
we only use that every Hodge class is a \emph{difference} of two that do.
This is trivially achieved by adding a large multiple of $[\omega^p]$, which
is strictly positive.
\end{remark}






%=========================================================


\begin{remark}[Conclusion of the proof]\label{rem:hodge-final-remark}
\REVMZ{\textbf{[Proof summary.]}}
Fix the auxiliary parameters (in particular $m$ and the holomorphic tensor power $\mhol$) as chosen in Sections~8--9.
Theorem~\ref{thm:main-hodge} is obtained by combining:
(i) the signed decomposition $\gamma=\gamma^{+}-\gamma^{-}$ with $\gamma^{-}$ algebraic (a complete intersection);
(ii) the analytic realization of $m[\gamma^{+}]$ by an effective holomorphic cycle with the required calibration and mass control; and
(iii) the algebraization of holomorphic cycles on a projective manifold via Chow's theorem and GAGA.
No further analytic input is needed beyond these steps.
\end{remark}
\begin{thebibliography}{99}

\bibitem{Allard72}
W.~K. Allard.
\newblock On the first variation of a varifold.
\newblock {\em Annals of Mathematics}, 95(3):417--491, 1972.

\bibitem{Catlin99}
D.~Catlin.
\newblock The Bergman kernel and a theorem of Tian.
\newblock In {\em Analysis and Geometry in Several Complex Variables}, Trends in Mathematics,
pages 1--23. Birkh\"auser, 1999.

\bibitem{Demailly12}
J.-P. Demailly.
\newblock {\em Complex Analytic and Differential Geometry}.
\newblock Open book/lecture notes, version 2012. Available at \url{https://www-fourier.ujf-grenoble.fr/~demailly/manuscripts/agbook.pdf}.

\bibitem{Donaldson01}
S.~K. Donaldson.
\newblock Scalar curvature and projective embeddings. I.
\newblock {\em Journal of Differential Geometry}, 59(3):479--522, 2001.

\bibitem{FF60}
H.~Federer and W.~H. Fleming.
\newblock Normal and integral currents.
\newblock {\em Annals of Mathematics}, 72(3):458--520, 1960.

\bibitem{Fed69}
H.~Federer.
\newblock {\em Geometric Measure Theory}.
\newblock Springer, 1969.

\bibitem{GH78}
\REV{P.~Griffiths and J.~Harris.
\newblock {\em Principles of Algebraic Geometry}.
\newblock Wiley-Interscience, 1978.}
\bibitem{Hartshorne77}
R.~Hartshorne.
\newblock {\em Algebraic Geometry}.
\newblock Graduate Texts in Mathematics 52. Springer, 1977.

\bibitem{Hormander65}
L.~H\"ormander.
\newblock $L^2$ estimates and existence theorems for the $\bar\partial$ operator.
\newblock {\em Acta Mathematica}, 113:89--152, 1965.

\bibitem{HL82}
R.~Harvey and H.~B. Lawson, Jr.
\newblock Calibrated geometries.
\newblock {\em Acta Mathematica}, 148:47--157, 1982.

\bibitem{King71}
J.~R. King.
\newblock The currents defined by analytic varieties.
\newblock {\em Acta Mathematica}, 127:185--220, 1971.

\bibitem{LangGmT}
S.~Lang.
\newblock {\em Fundamentals of Differential Geometry}.
\newblock Graduate Texts in Mathematics 191. Springer, 1999.
\newblock (See Ch.~XIV for currents and the compactness theorem.)

\bibitem{MaMarinescu07}
X.~Ma and G.~Marinescu.
\newblock {\em Holomorphic Morse Inequalities and Bergman Kernels}.
\newblock Progress in Mathematics 254. Birkh\"auser, 2007.

\bibitem{Serre56}
J.-P. Serre.
\newblock G\'eom\'etrie alg\'ebrique et g\'eom\'etrie analytique ({GAGA}).
\newblock {\em Annales de l'Institut Fourier}, 6:1--42, 1956.

\bibitem{Sim83}
L.~Simon.
\newblock {\em Lectures on Geometric Measure Theory}.
\newblock Proceedings of the Centre for Mathematical Analysis, Australian National University,
Vol.~3, 1983.

\bibitem{MaMarinescu13OffDiag}
X.~Ma and G.~Marinescu.
\newblock Remark on the off-diagonal expansion of the Bergman kernel on compact K\"ahler manifolds.
\newblock {\em Communications in Mathematics and Statistics}, 1:37--41, 2013.
\newblock arXiv:1302.2346.

\bibitem{Tian90}
G.~Tian.
\newblock On a set of polarized {K}\"ahler metrics on algebraic manifolds.
\newblock {\em Journal of Differential Geometry}, 32(1):99--130, 1990.

\bibitem{Voisin02}
\REV{C.~Voisin.
\newblock {\em Hodge Theory and Complex Algebraic Geometry I}.
\newblock Cambridge Studies in Advanced Mathematics 76. Cambridge University Press, 2002.}
\bibitem{Zelditch98}
S.~Zelditch.
\newblock Szeg\H{o} kernels and a theorem of Tian.
\newblock {\em International Mathematics Research Notices}, 1998(6):317--331, 1998.



\bibitem{Villani03}
C.~Villani.
\newblock \emph{Topics in Optimal Transportation}.
\newblock Graduate Studies in Mathematics, Vol.~58, American Mathematical Society, 2003.

\bibitem{Wells}
\REV{R.~O.~Wells, Jr.\ \emph{Differential Analysis on Complex Manifolds}. Graduate Texts in Mathematics, vol.~65. Springer, New York, 3rd ed., 2008.}
\bibitem{BBS08}
R.~Berman, B.~Berndtsson, and J.~Sj\"ostrand,
\emph{A direct approach to Bergman kernel asymptotics for positive line bundles},
Arkiv f\"or Matematik \textbf{46} (2008), 197--217.

\bibitem{BaranyGrinberg81}
I.~B\'ar\'any and V.~S.~Grinberg,
\emph{On some combinatorial questions in finite dimensional spaces},
Linear Algebra and its Applications \textbf{41} (1981), 1--9.

\bibitem{Schrijver86}
A.~Schrijver.
\newblock \emph{Theory of Linear and Integer Programming}.
\newblock John Wiley \& Sons, Chichester, 1986.


% ------------------------------------------------------------
% Added references (from hodge-references.tex)
% ------------------------------------------------------------

\bibitem{AtiyahHirzebruch62}
\REV{M.~F. Atiyah and F.~Hirzebruch,
\emph{Analytic cycles on complex manifolds},
Topology \textbf{1} (1962), 25--45.}
\bibitem{Chow49}
\REV{W.-L. Chow,
\emph{On compact complex analytic varieties},
Amer. J. Math. \textbf{71} (1949), 893--914.}
\bibitem{Fulton98}
\REV{W.~Fulton,
\emph{Intersection Theory},
2nd ed., Ergebnisse der Mathematik und ihrer Grenzgebiete (3), Vol.~2, Springer-Verlag, Berlin, 1998.}
\bibitem{Grothendieck69}
\REV{A.~Grothendieck,
\emph{Standard conjectures on algebraic cycles},
in Algebraic Geometry (Bombay Colloquium, 1968), Oxford University Press, 1969.}
\bibitem{KinderlehrerPedregal91}
\REV{D.~Kinderlehrer and P.~Pedregal,
\emph{Characterizations of Young measures generated by gradients},
Arch. Ration. Mech. Anal. \textbf{115} (1991), 329--365.}
\bibitem{Kleiman68}
\REV{S.~L. Kleiman,
\emph{Algebraic cycles and the Weil conjectures},
in Dix Expos\'es sur la Cohomologie des Sch\'emas, North-Holland, 1968.}
\bibitem{Lazarsfeld04I}
\REV{R.~Lazarsfeld,
\emph{Positivity in Algebraic Geometry~I},
Ergebnisse der Mathematik und ihrer Grenzgebiete (3), Vol.~48, Springer-Verlag, Berlin, 2004.}
\bibitem{Lazarsfeld04II}
\REV{R.~Lazarsfeld,
\emph{Positivity in Algebraic Geometry~II},
Ergebnisse der Mathematik und ihrer Grenzgebiete (3), Vol.~49, Springer-Verlag, Berlin, 2004.}
\bibitem{Lewis99}
\REV{J.~D. Lewis,
\emph{A Survey of the Hodge Conjecture},
2nd ed., CRM Monograph Series, Vol.~10, American Mathematical Society, Providence, RI, 1999.}
\bibitem{Pedregal97}
\REV{P.~Pedregal,
\emph{Parametrized Measures and Variational Principles},
Birkh\"auser, 1997.}
\bibitem{Schneider14}
\REV{R. Schneider,
\emph{Convex Bodies: The Brunn--Minkowski Theory},
2nd ed., Encyclopedia of Mathematics and its Applications, Vol. 151, Cambridge University Press, Cambridge, 2014.}
\end{thebibliography}


\end{document}
