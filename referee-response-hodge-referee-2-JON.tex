\documentclass[11pt]{article}
\usepackage[margin=1in]{geometry}
\usepackage{amsmath,amssymb,amsthm}
\usepackage{hyperref}

\newcommand{\Q}{\mathbb{Q}}
\newcommand{\Mass}{\mathbf{M}}

\newcommand{\ms}{\texttt{hodge-SAVE-dec-12-handoff.tex}}
\newcommand{\lbl}[1]{\texttt{#1}}

\title{Response Note to \texttt{hodge-referee-2.tex} (Dec 2025)}
\author{Jonathan Washburn}
\date{\today}

\begin{document}
\maketitle

\section*{Executive summary}
This note answers each referee-critical item in \texttt{hodge-referee-2.tex} by pointing to the exact, numbered statements in \ms\ that close the issue.
The two ``bridges'' singled out by the referee are now explicit:
\begin{itemize}
\item \textbf{Template $\rightarrow$ holomorphic:} uniform $C^1$ single-sheet control on an entire cell, plus stability of face incidence and face-slice masses.
\item \textbf{Local $\rightarrow$ global gluing:} a per-face flat-norm bound that sums to a global estimate, followed by a standard flat-norm filling with vanishing mass.
\end{itemize}

\section*{Key bridge A: template $\rightarrow$ holomorphic (the core stability verification)}
The referee asks for a proved statement that holomorphic realizations preserve the geometric hypotheses used by the mismatch bookkeeping.
This is provided by:
\begin{itemize}
\item \textbf{Uniform holomorphic $C^1$ control at Bergman scale:} \lbl{lem:bergman-control}.
\item \textbf{Global ``graph on the whole cell'' criterion:} \lbl{lem:global-graph-contraction} and its holomorphic implementation \lbl{prop:cell-scale-linear-model-graph}.
\item \textbf{Realization of a finite translation template with uniform graph control and disjointness:} \lbl{prop:finite-template}.
\item \textbf{No accidental face hits + localized trace + per-face mass comparability:} \lbl{prop:holomorphic-corner-exit-g1g2} and \lbl{cor:holomorphic-corner-exit-inherits}.
\item \textbf{Corner-exit holomorphic slivers (L1):} \lbl{prop:holomorphic-corner-exit-L1} (plus vertex-star coherence \lbl{rem:vertex-star-coherence}).
\end{itemize}

\section*{Key bridge B: local $\rightarrow$ global gluing (flat norm + filling)}
The referee asks for an explicit derivation of face-by-face mismatch bounds and a global small-boundary conclusion in a clearly specified norm.
This is provided by:
\begin{itemize}
\item \textbf{Per-face mismatch from prefix templates (with $O(h)$ edit regime):} \lbl{prop:prefix-template-coherence}.
\item \textbf{Vertex-template route that proves the face-edit regime from boundary-mass control (no-heavy-tail):}
\lbl{prop:vertex-template-face-edits} (and the tail-vs-prefix reduction \lbl{lem:oh-face-edit-regime}).
\item \textbf{Global summation in the sliver-compatible weighted form:} \lbl{cor:global-flat-weighted} and scaling bookkeeping \lbl{rem:weighted-scaling}.
\item \textbf{Microstructure/gluing estimate (now a numbered proposition):} \lbl{prop:glue-gap}.
\item \textbf{Boundary correction with vanishing mass (flat-norm decomposition + Federer--Fleming):} Step 5 in \ms.
\end{itemize}

\section*{Referee short list: ``Missing steps (what a referee would still require)''}
The referee isolates four bridge-closure items (Section ``Missing steps (what a referee would still require)'' in \texttt{hodge-referee-2.tex}).
They are now closed in \ms\ as follows:
\begin{enumerate}
\item \textbf{(G1)--(G2) verification for the \emph{holomorphic} corner-exit slivers.}
This is proved in \lbl{prop:holomorphic-corner-exit-g1g2} and packaged in \lbl{cor:holomorphic-corner-exit-inherits} (using the corner-exit existence \lbl{prop:holomorphic-corner-exit-L1}).

\item \textbf{Jet-to-global / ``graph on all of $Q$'' lemma (closing the Prop.~8.95-type gap).}
The deterministic global-sheet criterion is \lbl{lem:global-graph-contraction}; its holomorphic implementation is \lbl{prop:cell-scale-linear-model-graph}, and the resulting finite-template realization (disjointness + mass comparison) is \lbl{prop:finite-template}.

\item \textbf{Global face-cancellation / Thm.~8.46(iv) from mismatch estimates.}
Per-face control is \lbl{prop:prefix-template-coherence} (with the vertex-template edit regime \lbl{prop:vertex-template-face-edits}), global summation is \lbl{cor:global-flat-weighted}, and the combined global conclusion is the numbered estimate \lbl{prop:glue-gap}.

\item \textbf{Positivity/type hypothesis for the holomorphic-chain theorem.}
The promotion is stated and invoked in \lbl{thm:realization-from-almost}, with applicability discussed in \lbl{rem:hl-applicable}; algebraicity in the projective case is recorded in \lbl{rem:chow-gaga}.
\end{enumerate}

\section*{Responses to the referee's ``Ten key items''}
\begin{enumerate}
\item \textbf{Core bottleneck trio / uniformity.}
The three named bottlenecks correspond (in \ms) to:
\lbl{prop:glue-gap} (global gluing), \lbl{prop:holomorphic-corner-exit-L1} (corner-exit holomorphic slivers), and
\lbl{lem:bergman-control} + \lbl{lem:global-graph-contraction} (uniform holomorphic control on an entire cell).
Uniformity in cell index and label is enforced by the finite direction net package \lbl{prop:corner-exit-template-net} and the all-label execution \lbl{prop:global-coherence-all-labels}.

\item \textbf{Global gluing theorem.}
The global small-boundary statement is recorded as \lbl{prop:glue-gap} and is proved by combining:
\lbl{prop:global-coherence-all-labels} (all labels), \lbl{thm:sliver-mass-matching-on-template} (template bookkeeping),
\lbl{cor:global-flat-weighted} (global weighted sum), and the standard flat-norm filling argument in Substep~4.2 and Step 5.

\item \textbf{Combinatorial bookkeeping conditional on (G1)--(G2).}
The needed holomorphic verification of (G1)--(G2) is explicit in \lbl{prop:holomorphic-corner-exit-g1g2} and \lbl{cor:holomorphic-corner-exit-inherits}.

\item \textbf{Decisive gap: flat templates vs holomorphic realizations.}
Closed by the explicit stability statement \lbl{prop:holomorphic-corner-exit-g1g2} (no accidental face hits; localized trace; per-face mass comparability),
whose hypotheses are produced by \lbl{prop:finite-template} and \lbl{prop:holomorphic-corner-exit-L1}.

\item \textbf{Holomorphic realization of separated planes / global-control gap.}
The ``single sheet on all of $Q$'' mechanism is formalized by \lbl{lem:global-graph-contraction} and implemented for holomorphic sections in
\lbl{prop:cell-scale-linear-model-graph}.  The packaged finite-template realization is \lbl{prop:finite-template}.

\item \textbf{Corner-exit holomorphic slivers inherit weaknesses of the previous step.}
Closed by \lbl{prop:holomorphic-corner-exit-L1} together with the inheritance corollary \lbl{cor:holomorphic-corner-exit-inherits}.

\item \textbf{``Microstructure/gluing estimate established'' must be numbered.}
This has been promoted to the numbered proposition \lbl{prop:glue-gap} (and is the cited input in the mass-correction Step~5 and in \lbl{thm:automatic-syr}).

\item \textbf{SYR / promotion step needs exact hypotheses.}
The calibrated-limit bridge is stated as \lbl{thm:realization-from-almost}.
The paper’s ``automatic SYR'' summary is \lbl{thm:automatic-syr}, and Harvey--Lawson applicability is discussed in \lbl{rem:hl-applicable}.

\item \textbf{Flat-norm filling must be parameter-free.}
The filling argument used is explicit (flat-norm decomposition + Federer--Fleming isoperimetric inequality) in Step~5 of \ms,
and does not impose extra dimension restrictions beyond standard GMT hypotheses for integral currents.

\item \textbf{Structure / dependency chain.}
The manuscript already contains an explicit proof-structure list in the Introduction (``Proof structure'') and local-to-global steps (Steps~1--6 in \ms).
For the specific activation gate, \lbl{rem:activation-hypotheses-status} gives a compact pointer list showing where (i)--(iv) are proved in the corner-exit route.
\end{enumerate}

\section*{Responses to the referee's ``Missing steps'' list}
\begin{enumerate}
\item \textbf{Global microstructure matching (MM/edit-regime gate).}
Closed by \lbl{thm:sliver-mass-matching-on-template} together with the verification remark \lbl{rem:activation-hypotheses-status} and the all-label package
\lbl{prop:global-coherence-all-labels}.

\item \textbf{``No vanishing sliver mass''.}
The construction does not assume a uniform lower bound $m_{Q,a}\gtrsim h^{2n-2p}$ as a \emph{formal hypothesis} of the weighted bookkeeping.
Instead, the bookkeeping is formulated in the sliver-compatible weighted form \lbl{cor:global-flat-weighted}, and the absence of hidden lower bounds is stated in
\lbl{rem:no-vanishing-piece-mass}.
For corner-exit templates specifically, the footprint mass scale is explicit in terms of the template parameter $s$ (and is uniform along the template order):
see \lbl{lem:corner-exit-mass-scale}.  In particular, choosing $s=\theta h$ yields the absolute-scale bound $\Mass(\text{one piece})\asymp h^{2n-2p}$ if desired.

\item \textbf{Exponent/parameter regime barrier.}
The scaling condition is computed in \lbl{rem:weighted-scaling} and combined with the standard reduction \lbl{rem:lefschetz-reduction}
to cover the full Hodge statement.

\item \textbf{Slow variation from rounding.}
Derived quantitatively in \lbl{lem:slow-variation-rounding} and preserved under $0$--$1$ discrepancy rounding in \lbl{lem:slow-variation-discrepancy}.

\item \textbf{Cohomology quantization / fixed $m$.}
The integrality/period constraints are enforced in \lbl{prop:cohomology-match}, and fixed-$m$ (no drift) is built into
\lbl{thm:global-cohom} and summarized in \lbl{thm:automatic-syr}\,(iii).

\item \textbf{Flat-norm filling vs positivity.}
The filling current is not required to be positive; only its mass must vanish.  This is isolated in \lbl{prop:almost-calibration} and explained in
\lbl{rem:correction-not-positive}.

\item \textbf{Harvey--Lawson/Siu promotion.}
The calibrated-limit theorem is \lbl{thm:realization-from-almost}, which cites Harvey--Lawson and uses Federer--Fleming compactness.

\item \textbf{Signed decomposition rationality.}
Handled in \lbl{lem:signed-decomp}, with $N\in\Q_{>0}$ chosen explicitly (see the proof).

\item \textbf{Local holomorphic sheets / disjointness / uniform control.}
The uniform $C^1$ control input is \lbl{lem:bergman-control}, and the global-sheet criterion is \lbl{lem:global-graph-contraction}.
Disjointness and mass comparability under separation are handled by \lbl{lem:sliver-stability}, \lbl{lem:sliver-packing}, and \lbl{prop:finite-template}.

\item \textbf{Calibration--coercivity constants.}
The quantitative coercivity inequality is stated explicitly as \lbl{thm:cal-coercivity}, with dependence recorded in the surrounding discussion.
\end{enumerate}

\section*{Counterexample / failure-mode stress tests (and where they are excluded)}
The referee lists several plausible failure modes.  In the present draft they are ruled out as follows:
\begin{itemize}
\item \textbf{Accidental face hits (a sliver hits a non-designated face).}
Excluded by the deterministic face-incidence statement (G1-iff) in \lbl{prop:holomorphic-corner-exit-g1g2} and its packaged form \lbl{cor:holomorphic-corner-exit-inherits}.

\item \textbf{Multiplicity/orientation mismatch across an interface.}
The activation scheme uses a common labeling/template across interfaces (vertex-star coherence \lbl{rem:vertex-star-coherence} and the all-label packaging \lbl{prop:global-coherence-all-labels}),
so mismatches reduce to controlled prefix edits and are accounted for in the interface bookkeeping (\lbl{prop:prefix-template-coherence} and \lbl{prop:vertex-template-face-edits}).

\item \textbf{Heavy-tail mismatch (few unmatched indices but disproportionately large boundary mass).}
Excluded by the ``no heavy tail'' reduction \lbl{lem:oh-face-edit-regime} together with uniform per-face slice comparability (G2) in \lbl{prop:holomorphic-corner-exit-g1g2};
for corner-exit templates, uniformity along the order is made explicit in \lbl{lem:corner-exit-mass-scale}.

\item \textbf{Sheet splitting (local control but multi-sheet behavior elsewhere in the cell).}
The ``graph on the whole cell'' criterion is formalized in \lbl{lem:global-graph-contraction} and implemented for holomorphic complete intersections in \lbl{prop:cell-scale-linear-model-graph};
the finite-template realization is \lbl{prop:finite-template}.

\item \textbf{Positivity gap (cone alignment weaker than the positivity notion needed for the promotion theorem).}
The promotion step is invoked only for the \emph{limit} current after almost-calibration has been established (\lbl{prop:almost-calibration} and \lbl{thm:realization-from-almost}).
The limit is a $\psi$-calibrated \emph{integral} current, so Harvey--Lawson applies (see also \lbl{rem:hl-applicable}).
\end{itemize}

\section*{Required references (where they are used)}
\begin{itemize}
\item \textbf{Federer--Fleming / flat norm / isoperimetric filling:} used in Step~5 (boundary correction) and in \lbl{prop:glue-gap}.
\item \textbf{Harvey--Lawson calibrated structure theorem:} used in \lbl{thm:realization-from-almost} and discussed in \lbl{rem:hl-applicable}.
\item \textbf{Analytic $\Rightarrow$ algebraic in the projective case:} Chow/GAGA, recorded in \lbl{rem:chow-gaga}.
\item \textbf{Bergman/peak sections and jet control:} used in \lbl{lem:bergman-control} (with standard references listed there).
\end{itemize}

\section*{Notes on references}
The manuscript cites:
Federer--Fleming compactness/isoperimetric filling (Ann.\ of Math.\ 72 (1960)),
Harvey--Lawson calibrated-geometry structure (Acta Math.\ 148 (1982)),
and standard Bergman kernel/peak-section sources (Tian; Zelditch; Donaldson).
The algebraicity of analytic cycles in the projective setting is concluded by Chow/GAGA (standard references include Hartshorne or Griffiths--Harris).

\end{document}


