% ==========================================================
% TECHNICAL MEMO: POSITIVE HODGE CONJECTURE RISK ASSESSMENT
% ==========================================================
\documentclass[11pt]{article}

\usepackage[utf8]{inputenc}
\usepackage[T1]{fontenc}
\usepackage{amsmath, amssymb, amsfonts, amsthm}
\usepackage{geometry}
\geometry{margin=1.25in}

\usepackage[colorlinks=true,linkcolor=blue,citecolor=blue]{hyperref}

\theoremstyle{plain}
\newtheorem{claim}{Claim}
\newtheorem{observation}{Observation}

\theoremstyle{remark}
\newtheorem*{concern}{Concern}
\newtheorem*{response}{Response}
\newtheorem*{risk}{Risk Assessment}

\newcommand{\Mass}{\mathrm{Mass}}
\newcommand{\Def}{\mathrm{Def}}
\newcommand{\calF}{\mathcal{F}}
\newcommand{\R}{\mathbb{R}}
\newcommand{\Q}{\mathbb{Q}}
\newcommand{\Z}{\mathbb{Z}}

\title{\bfseries Technical Memo:\\
Does the Proof Imply the (False) Positive Hodge Conjecture?}
\author{Internal Review}
\date{\today}

\begin{document}
\maketitle

\section*{Executive Summary}

\textbf{Short answer:} The proof does \emph{not} imply the positive Hodge conjecture in the sense of Demailly--Voisin. The construction is sound for its stated hypotheses, but the manuscript would benefit from an explicit remark clarifying this distinction.

\textbf{Risk level:} \colorbox{yellow!30}{LOW--MEDIUM} (requires clarifying exposition, not a gap in logic).

\hrulefill

\section{The Concern}

A referee raised the following objection regarding Proposition~8.108 (the Microstructure/Gluing Estimate, \texttt{prop:glue-gap}) and its application to Harvey--Lawson:

\begin{concern}[Paraphrased]
``Approximating a non-integrable plane field with integrable holomorphic sheets incurs a systematic `filling mass' penalty proportional to the field's curvature.  If this mass does not vanish in the limit, the resulting current is not calibrated, which invalidates the Harvey--Lawson theorem.  Check whether this implies the (known-false) positive Hodge conjecture.''
\end{concern}

The ``positive Hodge conjecture'' referred to is the statement (known to be \textbf{false} in intermediate codimension by work of Voisin and others) that every \emph{limit of positive classes} is algebraic.  The gap between differential positivity and algebraic positivity is well-documented.

\section{Key Propositions Under Scrutiny}

The chain of logic is:
\begin{enumerate}
    \item \textbf{Proposition 8.108} (\texttt{prop:glue-gap}): Given the raw current $T^{\mathrm{raw}} = \sum_Q S_Q$, if face mismatches admit a translation model with displacement $\Delta_F \lesssim h^2$, then
    \[
    \calF(\partial T^{\mathrm{raw}}) \le \Mass(U_h) = o(m).
    \]
    
    \item \textbf{Proposition 8.114} (\texttt{prop:almost-calibration}): If $\Mass(U_\epsilon) \to 0$, then the corrected cycles $T_\epsilon = S - U_\epsilon$ satisfy
    \[
    \Def_{\mathrm{cal}}(T_\epsilon) := \Mass(T_\epsilon) - \int_{T_\epsilon}\psi \longrightarrow 0.
    \]
    
    \item \textbf{Theorem 8.116} (\texttt{thm:syr-realization} / \texttt{thm:realization-from-almost}): A sequence with vanishing calibration defect converges to a $\psi$-calibrated integral current, which Harvey--Lawson identifies as a positive sum of complex analytic subvarieties.
\end{enumerate}

The concern is: \emph{Can the filling mass $\Mass(U_h)$ actually be forced to zero, or is there a geometric obstruction from non-integrability?}

\section{Analysis of the Argument}

\subsection{What the manuscript claims}

The manuscript's parameter schedule (Section 8, lines 2144--2162) specifies:
\begin{enumerate}
    \item Fix $m$ first (to clear denominators in the integral lattice).
    \item Send mesh size $h_j \downarrow 0$.
    \item The local parameters $\varepsilon_j, \delta_j \to 0$ as functions of $h_j$.
\end{enumerate}

The key scaling estimate (Remark~8.110, \texttt{rem:weighted-scaling}) gives, at mesh scale $h$:
\[
\calF(\partial T^{\mathrm{raw}}) \lesssim m^{\frac{k-1}{k}} h^{2 - \frac{2n}{k}} \varepsilon^{-\frac{2p}{k}},
\qquad k := 2n - 2p.
\]

For \textbf{fixed} $m$ and $\varepsilon$, as $h \to 0$:
\begin{itemize}
    \item When $p < n/2$: exponent $2 - 2n/k > 0$, so estimate $\to 0$. \checkmark
    \item When $p = n/2$: exponent $= 0$ (borderline).  Handled by Lemma~8.164 via slow-variation.  \checkmark
    \item When $p > n/2$: reduced to $p \le n/2$ by Hard Lefschetz (Remark~8.170). \checkmark
\end{itemize}

\subsection{Why non-integrability does not obstruct}

Remark~8.116 (\texttt{rem:gluing}) directly addresses this:

\begin{quote}
\textbf{Objection:} ``The plane field $x \mapsto \beta(x)$ is generically non-integrable.  Local sheets cannot be glued without accumulating mass.''

\textbf{Response:} This conflates (a) integrating a plane field into a foliation, with (b) building many separate calibrated sheets.  The construction does (b), not (a).
\end{quote}

The crucial points are:
\begin{enumerate}
    \item We \textbf{never integrate} the cone-valued form $\beta$.  It is only a ``design target'' for local Carath\'eodory decompositions.
    
    \item Each cell $Q$ uses a \textbf{finite dictionary} of calibrated directions (from a $\varepsilon$-net on the calibrated Grassmannian).  The holomorphic sheets are algebraic complete intersections---their existence is guaranteed by Bertini, independent of $\beta$'s integrability.
    
    \item The \textbf{gluing mismatch} on faces $F = Q \cap Q'$ is bounded by displacement $\times$ slice mass.  The displacement is $O(h^2)$ when adjacent cells use \textbf{the same translation template} (corner-exit coherence).
    
    \item The non-integrability of $\beta$ affects \emph{which directions} appear in the Carath\'eodory decomposition, but \textbf{not the approximation error}, which depends only on mesh size.
\end{enumerate}

\subsection{Why this is NOT the positive Hodge conjecture}

The positive Hodge conjecture (false) would assert:
\begin{quote}
Every class that is a \emph{limit} of positive (1,1)-classes (or more generally, is represented by a positive current) is algebraic.
\end{quote}

The manuscript's claim is different:
\begin{quote}
Every \textbf{rational Hodge class} $\gamma \in H^{2p}(X,\Q) \cap H^{p,p}(X)$ is algebraic.
\end{quote}

The distinction:
\begin{itemize}
    \item The manuscript starts with a class that \emph{becomes cone-positive} after adding $N[\omega^p]$ for $N \gg 1$.  This is guaranteed for any Hodge class by the signed decomposition lemma.
    
    \item A cone-positive class \textbf{by definition} admits a smooth closed cone-valued representative $\beta$.  This is \emph{stronger} than being a limit of positive classes.
    
    \item The Demailly--Voisin counterexamples concern classes that are limits of positive classes but do \textbf{not} themselves admit smooth positive representatives.  Such classes are not cone-positive in the manuscript's sense.
\end{itemize}

\section{Potential Vulnerabilities}

\begin{observation}[Where the argument could fail---but doesn't]
\leavevmode
\begin{enumerate}
    \item \textbf{Hidden $m$-dependence:}  If the estimate required $m \to \infty$ along with $h \to 0$, the fixed-class hypothesis would fail.  
    
    \textit{Status:} The parameter schedule explicitly fixes $m$ first.  The scaling exponents are negative in $h$ for $p \le n/2$, independent of $m$.
    
    \item \textbf{Curvature-dependent lower bound:}  If there were a geometric lower bound $\Mass(U_h) \ge C(\kappa_\beta) > 0$ depending on the Frobenius curvature $\kappa_\beta$ of the plane field, the argument would fail.
    
    \textit{Status:}  No such bound exists because the construction does not integrate the plane field.  The filling is produced by Federer--Fleming isoperimetry, which depends only on the flat norm of $\partial T^{\mathrm{raw}}$, not on $\beta$'s integrability.
    
    \item \textbf{Cohomology mismatch:}  The discrepancy rounding (Proposition 8.113) requires approximation errors $< 1/2$.
    
    \textit{Status:}  The filling mass $\Mass(U_h) \to 0$ makes the pairing error with integral classes arbitrarily small.
\end{enumerate}
\end{observation}

\section{Recommendations}

\begin{enumerate}
    \item \textbf{Add an explicit remark} distinguishing the manuscript's claim from the positive Hodge conjecture.  Suggested location: after Theorem~\ref{thm:automatic-syr} or in the introduction.
    
    \item \textbf{Clarify in Remark~8.116} (\texttt{rem:gluing}) that non-integrability affects direction \emph{selection} but not the \emph{approximation error bounds}, which are purely metric/mesh-dependent.
    
    \item \textbf{Consider adding a subsection} titled ``Relation to the Positive Hodge Conjecture'' that:
    \begin{itemize}
        \item States the (false) positive Hodge conjecture precisely.
        \item Explains why cone-positivity (smooth closed cone-valued representative) is strictly stronger than differential positivity (positive current representative).
        \item Notes that the signed decomposition guarantees cone-positivity for $\gamma + N[\omega^p]$.
    \end{itemize}
\end{enumerate}

\section{Conclusion}

\begin{risk}
The reviewer's concern identifies a place where the exposition could be clearer, not a logical gap.  The construction does not imply the positive Hodge conjecture because:
\begin{enumerate}
    \item It applies to \textbf{cone-positive} classes (smooth closed cone-valued representatives), not arbitrary limits of positive classes.
    \item The filling mass estimate depends on \textbf{mesh geometry}, not on the curvature/integrability of the target form.
    \item The parameter schedule fixes $m$ first and sends $h \to 0$, so no secret blow-up occurs.
\end{enumerate}

\textbf{Recommended action:}  Add 1--2 clarifying remarks; no structural changes needed.
\end{risk}

\end{document}

