% ==========================================================
% PROOF COMPARISON AND VALIDITY ANALYSIS
% Calibration--Coercivity and the Hodge Conjecture
% ==========================================================

\documentclass[11pt]{article}

\usepackage[utf8]{inputenc}
\usepackage[T1]{fontenc}
\usepackage{amsmath, amssymb, amsfonts, amsthm}
\usepackage{geometry}
\usepackage{xcolor}
\usepackage[colorlinks=true,linkcolor=blue,citecolor=blue,urlcolor=blue]{hyperref}

\geometry{margin=1in}

% Custom colors
\definecolor{oldcolor}{RGB}{180,80,80}
\definecolor{newcolor}{RGB}{60,120,60}
\definecolor{warncolor}{RGB}{200,120,0}

% Commands for highlighting
\newcommand{\oldtext}[1]{\textcolor{oldcolor}{\textbf{[Original:]} #1}}
\newcommand{\newtext}[1]{\textcolor{newcolor}{\textbf{[New:]} #1}}
\newcommand{\warn}[1]{\textcolor{warncolor}{\textbf{[Warning:]} #1}}

% Theorem environments
\theoremstyle{plain}
\newtheorem{theorem}{Theorem}[section]
\newtheorem{lemma}[theorem]{Lemma}
\newtheorem{proposition}[theorem]{Proposition}

\theoremstyle{definition}
\newtheorem{definition}[theorem]{Definition}

\theoremstyle{remark}
\newtheorem{remark}[theorem]{Remark}
\newtheorem*{remark*}{Remark}

% Notation
\newcommand{\R}{\mathbb{R}}
\newcommand{\C}{\mathbb{C}}
\newcommand{\Q}{\mathbb{Q}}
\newcommand{\Z}{\mathbb{Z}}

\title{\bfseries Proof Comparison and Validity Analysis:\\
Calibration--Coercivity Approaches to the Hodge Conjecture}

\author{Analysis of \texttt{hodg-dec-6d.tex} vs.\ Earlier Approaches}
\date{\today}

\begin{document}
\maketitle

\begin{abstract}
This document compares two approaches to proving the Hodge Conjecture via 
calibration--coercivity methods: the \emph{original conditional approach} and 
the \emph{new signed decomposition approach}. We identify the critical flaw 
in the original approach, explain how the new approach remedies it, and 
assess the validity of the resulting proof. The two files \texttt{hodg-dec-6d.tex} 
and \texttt{hodge-dec-6c.tex} are \emph{textually identical}---both contain the 
new approach. The ``original approach'' is documented in auxiliary files 
(\texttt{hodge-blocker.tex}, \texttt{change-report.tex}).
\end{abstract}

\tableofcontents
\newpage

% ==========================================================
\section{Executive Summary}
% ==========================================================

\subsection{File Comparison Result}

A direct comparison of \texttt{hodg-dec-6d.tex} and \texttt{hodge-dec-6c.tex} 
reveals that \textbf{these files are identical}. Both contain the same proof 
using the signed decomposition approach. The version numbering (6c vs.\ 6d) 
likely reflects incremental saves of the same document.

The meaningful comparison is between:
\begin{itemize}
    \item \textbf{Original approach:} Documented in \texttt{hodge-blocker.tex} 
    and the ``before'' sections of \texttt{change-report.tex}
    \item \textbf{New approach:} Contained in both \texttt{hodg-dec-6d.tex} 
    and \texttt{hodge-dec-6c.tex}
\end{itemize}

\subsection{Key Finding}

The original approach was \textbf{conditional} on an unproven (and often false) 
assumption. The new approach is designed to be \textbf{unconditional} by 
employing signed decomposition. However, critical analysis reveals that the 
proof still has unresolved dependencies that affect its validity.

% ==========================================================
\section{The Original Approach (Conditional)}
% ==========================================================

\subsection{Core Strategy}

The original proof attempted the following:

\begin{enumerate}
    \item Let $\gamma \in H^{2p}(X,\Q) \cap H^{p,p}(X)$ be a rational Hodge class.
    \item Let $\gamma_{\mathrm{harm}}$ be its unique $\omega$-harmonic representative.
    \item \textbf{Assume:} $\gamma_{\mathrm{harm}}(x) \in K_p(x)$ for all $x \in X$ 
    (the harmonic representative lies in the calibrated cone pointwise).
    \item Under this assumption, prove calibration--coercivity:
    \[
        E(\alpha) - E(\gamma_{\mathrm{harm}}) \;\ge\; c\,\mathrm{Def}_{\mathrm{cone}}(\alpha).
    \]
    \item Use energy minimization to force convergence to calibrated currents.
    \item Conclude via Harvey--Lawson that the class is algebraic.
\end{enumerate}

\subsection{The Critical Assumption}

The key step was:
\begin{quote}
\textit{``Since $\gamma_{\mathrm{harm}}(x) \in K_p(x)$ for all $x$, the cone 
distance satisfies $\mathrm{dist}_{\mathrm{cone}}(\alpha_x) \le 
\|\alpha_x - \gamma_{\mathrm{harm},x}\|$.''}
\end{quote}

This inequality requires the harmonic representative to lie \emph{inside} 
the calibrated cone at every point, serving as a reference point for 
distance estimates.

\subsection{Why the Assumption Fails}

\begin{proposition}[Failure of the cone-valued harmonic assumption]
There exist rational Hodge classes $\gamma$ on smooth projective varieties 
whose harmonic representatives $\gamma_{\mathrm{harm}}$ do \textbf{not} lie 
in the calibrated cone $K_p(x)$ at any point $x \in X$.
\end{proposition}

\begin{proof}[Counterexample]
Consider $X = S_1 \times S_2$, a product of two smooth projective surfaces, 
with K\"ahler form $\omega = \pi_1^*\omega_1 + \pi_2^*\omega_2$. Let 
$p = 2$ and consider the $(2,2)$-class:
\[
    \gamma \;:=\; [\pi_1^*\omega_1^2] - [\pi_2^*\omega_2^2].
\]
This is a rational Hodge class. Its harmonic representative is:
\[
    \gamma_{\mathrm{harm}} \;=\; \pi_1^*\omega_1^2 - \pi_2^*\omega_2^2.
\]
In the Hermitian model at any point $x = (x_1, x_2) \in X$, this corresponds 
to a matrix with both positive and negative eigenvalues---it has 
\emph{indefinite signature} everywhere. Thus:
\[
    \gamma_{\mathrm{harm}}(x) \;\notin\; K_2(x)
    \quad\text{for all } x \in X.
\]
The calibrated cone $K_2(x)$ consists of positive semi-definite matrices 
(weakly positive forms), so a form with indefinite signature cannot 
belong to it.
\end{proof}

\subsection{Consequence}

The original proof was \textbf{conditional} on the assumption 
$\gamma_{\mathrm{harm}}(x) \in K_p(x)$, which:
\begin{itemize}
    \item Is true for some special classes (e.g., effective classes).
    \item Is \textbf{false} for general Hodge classes, including difference classes.
    \item Was never proven in the original manuscript.
\end{itemize}

Therefore, the original approach \textbf{did not constitute a proof} of the 
Hodge Conjecture.

% ==========================================================
\section{The New Approach (Signed Decomposition)}
% ==========================================================

\subsection{Core Innovation}

The new approach in \texttt{hodg-dec-6d.tex} bypasses the problematic 
assumption entirely through the following insight:

\begin{quote}
\textit{We do not need the harmonic representative to be cone-valued. We only 
need that every Hodge class is a \textbf{difference} of two classes that 
admit cone-valued representatives.}
\end{quote}

\subsection{The Signed Decomposition Lemma}

\begin{lemma}[Signed Decomposition---Lemma 8.7 in the manuscript]
Let $\gamma \in H^{2p}(X,\Q) \cap H^{p,p}(X)$ be any rational Hodge class. 
Then there exist \emph{effective} classes $\gamma^+$ and $\gamma^-$ such that:
\[
    \gamma \;=\; \gamma^+ - \gamma^-.
\]
Moreover:
\begin{enumerate}
    \item $\gamma^- := N[\omega^p]$ for some rational $N > 0$.
    \item $\gamma^+ := \gamma + N[\omega^p]$ admits a cone-valued representative.
    \item Both $\gamma^+$ and $\gamma^-$ are rational Hodge classes.
\end{enumerate}
\end{lemma}

\begin{proof}[Proof sketch]
Let $\alpha$ be any smooth closed $(p,p)$-form representing $\gamma$. In the 
Hermitian model, $\alpha(x)$ corresponds to a Hermitian matrix $A(x)$ with 
(possibly negative) minimum eigenvalue $\lambda_{\min}(A(x))$.

The K\"ahler power $\omega^p(x)$ corresponds to a strictly positive definite 
matrix $W(x)$ with $\lambda_{\min}(W(x)) \ge c_0 > 0$ uniformly (by 
compactness of $X$).

Choose $N > \sup_x |\lambda_{\min}(A(x))| / c_0$. Then:
\[
    \lambda_{\min}(A(x) + N \cdot W(x)) 
    \;\ge\; 
    \lambda_{\min}(A(x)) + N c_0 
    \;>\; 0.
\]
Thus $\alpha + N \omega^p$ is strictly positive (cone-valued), and:
\[
    \gamma^+ := \gamma + N[\omega^p], 
    \qquad 
    \gamma^- := N[\omega^p].
\]
\end{proof}

\subsection{Why $\gamma^-$ is Automatically Algebraic}

\begin{lemma}[Lemma 8.8 in the manuscript]
The class $[\omega^p] = H^p$ (where $H$ is the hyperplane class) is algebraic, 
represented by a complete intersection of $p$ generic hyperplane sections.
\end{lemma}

\begin{proof}
By Bertini's theorem, generic hyperplanes $H_1, \ldots, H_p$ intersect $X$ 
in a smooth subvariety $Z = X \cap H_1 \cap \cdots \cap H_p$ of codimension $p$. 
Its Poincar\'e dual is $H^p = [\omega^p]$.
\end{proof}

\subsection{New Proof Structure}

The new proof proceeds as:

\begin{enumerate}
    \item \textbf{Signed decomposition:} Write $\gamma = \gamma^+ - \gamma^-$.
    
    \item \textbf{$\gamma^-$ is algebraic:} It equals $N[\omega^p]$, represented 
    by complete intersections.
    
    \item \textbf{$\gamma^+$ is effective:} It admits a cone-valued representative 
    $\beta = \alpha + N\omega^p$ with $\beta(x) \in K_p(x)$ for all $x$.
    
    \item \textbf{Effective $\Rightarrow$ Algebraic:} For effective classes, the 
    calibration--coercivity machinery produces calibrated currents via:
    \begin{itemize}
        \item Projective tangential approximation (Lemma 8.4)
        \item Automatic SYR (Theorem 8.6)
        \item Harvey--Lawson structure theory
        \item Chow's theorem
    \end{itemize}
    
    \item \textbf{Conclusion:} $\gamma = [Z^+] - [Z^-]$ is algebraic.
\end{enumerate}

% ==========================================================
\section{How the New Approach Remedies the Original Flaw}
% ==========================================================

\subsection{The Key Insight}

\begin{center}
\renewcommand{\arraystretch}{1.4}
\begin{tabular}{|p{6cm}|p{6cm}|}
\hline
\textbf{Original Approach} & \textbf{New Approach} \\
\hline
Required: $\gamma_{\mathrm{harm}}(x) \in K_p(x)$ for all $x$ 
    & No assumption on $\gamma_{\mathrm{harm}}$ \\
\hline
Applied to: All Hodge classes directly 
    & Applied to: Only \emph{effective} classes \\
\hline
Failed for: Difference classes with indefinite signature 
    & Handles differences by decomposition \\
\hline
Cone-valued form: The harmonic representative 
    & Cone-valued form: Constructed $\alpha + N\omega^p$ \\
\hline
\end{tabular}
\end{center}

\subsection{Detailed Comparison}

\subsubsection{What Changed in the Coercivity Proof}

\textbf{Original Step 3:}
\begin{quote}
``Since $\gamma_{\mathrm{harm}}(x) \in K_p(x)$ for all $x$, the cone distance 
satisfies $\mathrm{dist}_{\mathrm{cone}}(\alpha_x) \le 
\|\alpha_x - \gamma_{\mathrm{harm},x}\|$.''
\end{quote}

\textbf{Revised Step 3:}
\begin{quote}
``By Proposition 6.6 (pointwise cone projection bound),
\[
    \mathrm{dist}_{\mathrm{cone}}(\alpha_x)^2
    \;\le\;
    |\alpha^{(p+1,p-1)}_x|^2
    + |\alpha^{(p-1,p+1)}_x|^2
    + \|(\alpha^{(p,p)}_x - \gamma_{\mathrm{harm},x})_{\mathrm{prim}}\|^2
    + d\,\mu(x)^2.
\]
Integrating and using established estimates yields the bound unconditionally.''
\end{quote}

The revised version uses the Hermitian/PSD-cone structure directly, without 
requiring $\gamma_{\mathrm{harm}}$ to be in the cone.

\subsubsection{Why the Decomposition Works}

The signed decomposition sidesteps the fundamental obstruction because:

\begin{itemize}
    \item We do \textbf{not} claim that $\gamma$ has a cone-valued representative.
    
    \item We do \textbf{not} claim that $\gamma_{\mathrm{harm}}$ is cone-valued.
    
    \item We \textbf{only} claim that $\gamma^+ = \gamma + N[\omega^p]$ has a 
    cone-valued representative (namely $\alpha + N\omega^p$).
    
    \item This is trivially true: adding a sufficiently large positive form 
    makes any form positive.
\end{itemize}

% ==========================================================
\section{Validity Assessment of the New Proof}
% ==========================================================

\subsection{What is Established}

The following components of the proof are mathematically rigorous:

\begin{enumerate}
    \item \textbf{Signed decomposition (Lemma 8.7):} 
    \textcolor{newcolor}{\textbf{Valid.}} The construction is elementary 
    linear algebra in the Hermitian model.
    
    \item \textbf{$\gamma^-$ is algebraic (Lemma 8.8):} 
    \textcolor{newcolor}{\textbf{Valid.}} This follows from Bertini's theorem 
    and is classical.
    
    \item \textbf{Calibration--coercivity inequality (Theorem 7.1):} 
    \textcolor{newcolor}{\textbf{Valid for effective classes.}} The pointwise 
    linear algebra and integration arguments are sound.
    
    \item \textbf{Projective tangential approximation (Lemma 8.4):} 
    \textcolor{newcolor}{\textbf{Valid.}} Uses Bertini's theorem and projective 
    automorphisms---standard tools.
    
    \item \textbf{Realization from almost-calibrated sequences (Theorem 8.1):} 
    \textcolor{newcolor}{\textbf{Valid.}} This is a direct application of 
    Federer--Fleming compactness and Harvey--Lawson structure theory.
\end{enumerate}

\subsection{Critical Dependencies}

The proof's validity hinges on two key claims:

\subsubsection{Automatic SYR (Theorem 8.6)}

\begin{quote}
\textit{``Every smooth cone-valued $(p,p)$ form $\beta$ representing a rational 
Hodge class satisfies the Stationary Young-measure Realizability property.''}
\end{quote}

\textbf{Assessment:} The proof sketch appeals to:
\begin{itemize}
    \item Carath\'eodory decomposition on each cube
    \item Dense family of calibrated submanifolds from Proposition 8.5
    \item ``Boundary correction and varifold compactness arguments identical to 
    Theorem 8.3''
\end{itemize}

\warn{This is the most delicate step.} The argument requires:
\begin{enumerate}
    \item Constructing rectifiable currents $S_Q$ with prescribed barycentric weights
    \item Controlling boundary masses across cube interfaces
    \item Ensuring the homology class is exactly correct (not just approximately)
    \item Varifold compactness producing the right Young-measure convergence
\end{enumerate}

The manuscript provides a proof sketch for the LICD case (Theorem 8.3) but 
claims Theorem 8.6 follows ``identically.'' This equivalence needs more 
detailed justification, as the general cone-valued case lacks the integrability 
structure of LICD.

\subsubsection{Effective Classes are Algebraic (Theorem 8.7)}

This theorem relies on Automatic SYR (Theorem 8.6), so its validity is 
contingent on that result.

\subsection{Potential Gaps}

\begin{enumerate}
    \item \textbf{SYR construction details:} The passage from ``dense family of 
    calibrated directions'' to ``rectifiable currents with exact homology class'' 
    involves non-trivial geometric measure theory. The Federer--Fleming deformation 
    theorem handles boundary filling, but ensuring exact (not approximate) 
    cohomology requires careful attention to lattice quantization.
    
    \item \textbf{Young-measure convergence:} Varifold compactness gives 
    subsequential convergence, but the claim that tangent-plane Young measures 
    converge to a field with barycenter $\beta(x)$ ``almost everywhere'' needs 
    the weak-$*$ limit to be supported on calibrated planes. This follows if 
    the approximating currents are exactly calibrated, which they are by 
    construction.
    
    \item \textbf{Stationarity:} The SYR definition requires the sequence 
    $\{T_k\}$ to be ``stationary,'' but the construction produces $\psi$-calibrated 
    cycles, which are automatically stationary (zero first variation). This is fine.
\end{enumerate}

\subsection{Overall Verdict}

\begin{center}
\fbox{\parbox{0.9\textwidth}{
\textbf{Verdict:} The proof is \textbf{substantially more complete} than the 
original conditional version. The signed decomposition strategy is sound and 
eliminates the false assumption about harmonic representatives.

The remaining concern is the \textbf{Automatic SYR theorem} (Theorem 8.6), 
which is stated with a brief proof appealing to techniques from the LICD case. 
A fully rigorous treatment would require more detailed verification that the 
lamination/filling construction works for arbitrary cone-valued forms, not 
just those satisfying LICD.

\textbf{Conditional status:} The proof is unconditional \textbf{if} Theorem 8.6 
(Automatic SYR) holds. The manuscript's argument for Theorem 8.6 is plausible 
but abbreviated.
}}
\end{center}

% ==========================================================
\section{Summary: What Changed and Why}
% ==========================================================

\subsection{The Fundamental Shift}

\begin{center}
\renewcommand{\arraystretch}{1.5}
\begin{tabular}{|p{5.5cm}|p{5.5cm}|}
\hline
\textcolor{oldcolor}{\textbf{Original Approach}} & 
\textcolor{newcolor}{\textbf{New Approach}} \\
\hline
Prove: Harmonic representative is cone-valued 
    & Accept: Harmonic representative may not be cone-valued \\
\hline
Strategy: Direct attack on general $\gamma$ 
    & Strategy: Reduce to effective classes via decomposition \\
\hline
Assumption: $\gamma_{\mathrm{harm}} \in K_p$ (unproven, often false) 
    & Assumption: None (construction yields cone-valued form) \\
\hline
Status: Conditional 
    & Status: Unconditional (modulo SYR details) \\
\hline
\end{tabular}
\end{center}

\subsection{Why the Original Approach was Insufficient}

\begin{enumerate}
    \item \textbf{False assumption:} The claim $\gamma_{\mathrm{harm}}(x) \in K_p(x)$ 
    is false for difference classes.
    
    \item \textbf{No bypass:} The original coercivity argument \emph{required} 
    this assumption to bound cone distance by $L^2$ distance.
    
    \item \textbf{Gap not identified:} The original manuscript did not flag 
    this as an assumption; it was stated as if proven.
\end{enumerate}

\subsection{How the New Approach Remedies It}

\begin{enumerate}
    \item \textbf{Signed decomposition:} Writes any $\gamma$ as $\gamma^+ - \gamma^-$ 
    with both pieces effective.
    
    \item \textbf{$\gamma^-$ handled classically:} Complete intersections are 
    algebraic by Bertini/Lefschetz.
    
    \item \textbf{$\gamma^+$ handled by construction:} The form 
    $\alpha + N\omega^p$ is cone-valued by choice of $N$, not by any property 
    of harmonic representatives.
    
    \item \textbf{Coercivity reformulated:} The revised Theorem 7.1 bounds cone 
    distance using pointwise Hermitian/PSD structure, not by assuming the 
    harmonic form is in the cone.
\end{enumerate}

% ==========================================================
\section{Conclusion}
% ==========================================================

The documents \texttt{hodg-dec-6d.tex} and \texttt{hodge-dec-6c.tex} are 
\textbf{identical} and both contain the \textbf{new signed decomposition 
approach}. The comparison requested is therefore between:

\begin{itemize}
    \item The \textbf{original conditional approach} (documented in earlier 
    files and the change report), which assumed the harmonic representative 
    is cone-valued---an assumption that is \textbf{false in general}.
    
    \item The \textbf{new unconditional approach}, which uses signed 
    decomposition to reduce the problem to effective classes, for which 
    cone-valued representatives can be \textbf{constructed explicitly}.
\end{itemize}

\subsection*{Validity Summary}

\begin{itemize}
    \item[$\checkmark$] Signed decomposition is \textbf{valid and elementary}.
    \item[$\checkmark$] $\gamma^-$ algebraic is \textbf{classical and valid}.
    \item[$\checkmark$] Calibration--coercivity for effective classes is \textbf{valid}.
    \item[$\checkmark$] Projective tangential approximation is \textbf{valid}.
    \item[$\sim$] Automatic SYR (Theorem 8.6) is \textbf{plausible but abbreviated}.
    \item[$?$] Full rigor requires detailed verification of the lamination/filling 
    construction for general cone-valued forms.
\end{itemize}

The proof represents a \textbf{major conceptual advance} over the original 
conditional version. Whether it constitutes a complete proof of the Hodge 
Conjecture depends on the detailed validity of the SYR construction, which 
merits careful scrutiny by experts in geometric measure theory.

\end{document}

