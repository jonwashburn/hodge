% Standalone proof for p=2 (strong G1) — rigorous detailed version
\documentclass[11pt]{article}
\usepackage{amsmath,amsthm,amssymb,amsfonts,mathtools}
\usepackage[margin=1in]{geometry}
\usepackage{hyperref}
\hypersetup{colorlinks=true,linkcolor=blue,citecolor=blue,urlcolor=blue}

\newtheorem{theorem}{Theorem}[section]
\newtheorem{lemma}[theorem]{Lemma}
\newtheorem{proposition}[theorem]{Proposition}
\newtheorem{corollary}[theorem]{Corollary}
\theoremstyle{definition}
\newtheorem{definition}[theorem]{Definition}
\theoremstyle{remark}
\newtheorem{remark}[theorem]{Remark}

\newcommand{\C}{\mathbb{C}}
\newcommand{\R}{\mathbb{R}}
\newcommand{\Z}{\mathbb{Z}}
\newcommand{\N}{\mathbb{N}}
\newcommand{\Gr}{\mathrm{Gr}}
\newcommand{\Mass}{\mathbf{M}}
\newcommand{\HS}{\mathrm{HS}}
\newcommand{\harm}{\mathrm{harm}}
\newcommand{\cone}{\mathrm{cone}}
\newcommand{\Def}{\mathrm{Def}}
\newcommand{\dist}{\mathrm{dist}}
\newcommand{\Vol}{\mathrm{Vol}}

\title{Case p=2: Strong G1 Realizability via Projective Tangential Approximation and Calibrated Laminates}
\author{}
\date{\today}

\begin{document}
\maketitle

\begin{abstract}
We give a rigorous, fully detailed proof of the strong G1 statement in the case $p=2$ on a smooth projective K\"ahler manifold $(X^n,\omega)$. Concretely: let $\gamma\in H^{4}(X,\mathbb{Q})\cap H^{2,2}(X)$ be rational, and let $\beta$ be the smooth, closed, cone--valued $(2,2)$--form produced by the unconditional calibration--coercivity minimization (or penalized) procedure, representing $[\gamma]$. We prove that $\beta$ is Stationary Young--measure Realizable (SYR): there exists a sequence of $\psi$--calibrated integral $(n-2)$--cycles whose tangent--plane Young measures converge almost everywhere to a field supported on complex $(n-2)$--planes with barycenter $\beta(x)$. Consequently, by varifold compactness and Harvey--Lawson, one obtains a calibrated integral current representing $\mathrm{PD}(m[\gamma])$, hence an algebraic cycle representing $\gamma$. This establishes strong G1 for $p=2$.
\end{abstract}

\section{Statement (strong G1 for $p=2$)}

Let $(X^n,\omega)$ be smooth projective K\"ahler and
\[
\psi := \frac{\omega^{n-2}}{(n-2)!}
\]
be the K\"ahler calibration for $(n-2)$--planes. For a rational Hodge class $\gamma\in H^{2,2}(X)\cap H^{4}(X,\Q)$, let $\beta$ denote the smooth, closed, cone--valued $(2,2)$--form produced by the unconditional calibration--coercivity minimization in $[\gamma]$; in particular $\beta(x)$ lies in the closed convex $\psi$--calibrated cone $K_2(x)\subset \Lambda^{2,2}T_x^*X$ for all $x$ and $\beta$ represents $[\gamma]$.

\begin{theorem}[Strong G1 for $p=2$]\label{thm:G1p2}
The cone--valued minimizer $\beta$ is SYR--realizable: there exist $\psi$--calibrated integral $(n-2)$--cycles $T_k$ with $\partial T_k=0$ and
\begin{itemize}
	\item $\Mass(T_k)\to \displaystyle\int_X \beta\wedge\psi$,
	\item the tangent--plane Young measures of $T_k$ converge a.e.\ to a measurable field $\nu_x$ supported on complex $(n-2)$--planes with barycenter $\int \xi_P\,d\nu_x(P)=\beta(x)$,
	\item $[T_k]=\mathrm{PD}(m[\gamma])$ for some fixed $m\in\N$ (independent of $k$).
\end{itemize}
Consequently there is a $\psi$--calibrated integral current $T$ representing $\mathrm{PD}(m[\gamma])$. By Harvey--Lawson, $T$ is integration along a positive sum of complex analytic subvarieties; hence $\gamma$ is algebraic.
\end{theorem}

The remainder of the paper proves Theorem~\ref{thm:G1p2} in six steps.

\section{Step 1: Carath\'eodory decomposition in the Hermitian model}

At each $x\in X$, identify $\Lambda^{2,2}(T_x^*X)$ with a finite--dimensional real vector space $\mathcal{V}_x$ equipped with the inner product induced by the metric, and let $K_2(x)\subset \mathcal{V}_x$ be the closed convex cone generated by simple $\psi$--saturating \((2,2)\)--forms $\xi_P$ associated to complex $(n-2)$--planes $P\subset T_xX$. By Carath\'eodory's theorem in $\R^{D}$, every $\beta(x)\in K_2(x)$ can be written as a convex combination of at most $D+1$ extremal generators, where $D=\dim(\mathcal{V}_x)$ is independent of $x$.

\begin{lemma}[Uniform Carath\'eodory]\label{lem:caratheodory}
There exists $N=N(n)$ such that for all $x\in X$ there exist $(n-2)$--planes $P_{x,1},\dots,P_{x,N}$ and weights $\theta_{x,j}\ge 0$, $\sum_{j=1}^{N}\theta_{x,j}=1$, with
\[
\beta(x)=\sum_{j=1}^{N}\theta_{x,j}\,\xi_{P_{x,j}}.
\]
Moreover, for every $\varepsilon>0$ there exist measurable choices with $\theta_{x,j}$ piecewise continuous in $x$ and the fields $x\mapsto P_{x,j}$ measurable, such that on sufficiently small coordinate cubes $Q$ the data vary by at most $\varepsilon$.
\end{lemma}

\begin{proof}
Carath\'eodory is classical. The measurability and local stabilization follow from standard measurable selection on Grassmannians and the compactness of $G:=\Gr_{n-2}(T X)$ together with a partition of unity subordinate to normal coordinate charts. Details are routine and omitted.
\end{proof}

\section{Step 2: Projective tangential approximation with 1--jet control (p=2)}

Fix an ample line bundle $L\to X$ with a Hermitian metric of positive curvature proportional to $\omega$. For $m\in\N$ large, consider the complete linear system $|L^m|$.

\begin{lemma}[1--jet surjectivity for high powers]\label{lem:1jet}
For each integer $k\ge 1$ there exists $m_0(k)$ such that for all $m\ge m_0(1)$ and all $x\in X$, the evaluation map on first jets
\[
H^0(X,L^m)\longrightarrow (L^m)_x\oplus (L^m)_x\otimes T_x^*X
\]
is surjective. Equivalently, any prescribed value and first derivative at $x$ is realized by a global section of $L^m$.
\end{lemma}

\begin{proof}
Consider the exact sequence \(0\to L^m\otimes \mathfrak{m}_x^2\to L^m \to L^m\otimes \mathcal{O}_X/\mathfrak{m}_x^2\to 0\). For $m\gg 0$, $H^1(X,L^m\otimes \mathfrak{m}_x^2)=0$ by Serre vanishing (ampleness of $L$). Hence $H^0(X,L^m)\twoheadrightarrow H^0(X,L^m\otimes \mathcal{O}_X/\mathfrak{m}_x^2)$ which identifies with first jets at $x$. See e.g.\ Lazarsfeld, Positivity in Algebraic Geometry I, Theorem 1.8.5 and Remark 1.8.6.
\end{proof}

We will also need control of the first derivatives on a small ball. This is provided by Bergman kernel asymptotics for high powers.

\begin{lemma}[Uniform $C^1$ control on $m^{-1/2}$--balls]\label{lem:bergman}
Fix $\varepsilon>0$. There exists $m_1(\varepsilon)$ such that for all $m\ge m_1(\varepsilon)$, each $x\in X$, and each pair of complex covectors $\lambda_1,\lambda_2\in T_x^*X$, there exist sections $s_1,s_2\in H^0(X,L^m)$ with the following properties in normal holomorphic coordinates centered at $x$:
\begin{itemize}
	\item $s_i(x)=0$ and $ds_i(x)=\lambda_i$,
	\item on the geodesic ball $B_{c\,m^{-1/2}}(x)$ (for a universal $c>0$), the gradients satisfy $\|ds_i(y)-\lambda_i\|\le \varepsilon$ for all $y\in B_{c\,m^{-1/2}}(x)$.
\end{itemize}
\end{lemma}

\begin{proof}
This is standard from peak section/Bergman kernel asymptotics (Tian, Catlin, Zelditch, Lu). In local normal coordinates with rescaling by $\sqrt{m}$, the space $H^0(X,L^m)$ approximates holomorphic polynomials with Gaussian weight, and there exist sections with prescribed jets whose $C^1$ norms on $B_{c\,m^{-1/2}}$ approach those of the corresponding linear functions. See e.g.\ Zelditch, ``Szeg\H{o} kernels and a theorem of Tian,'' IMRN 1998; and Donaldson, ``Scalar curvature and projective embeddings,'' J.\ Diff.\ Geom.\ 2001, Section 2, for uniform $C^k$ control in the rescaled charts.
\end{proof}

\begin{proposition}[Projective tangential approximation for $p=2$]\label{prop:tangent-approx}
Let $x\in X$ and let $\Pi\subset T_xX$ be a complex $(n-2)$--plane. For every $\varepsilon>0$ there exist $m\gg 0$ and a smooth complete intersection
\[
Y = \{s_1=0\}\cap \{s_2=0\}\subset X,\qquad s_i\in H^0(X,L^m),
\]
such that $x\in Y$, $Y$ is smooth near $x$, and
\[
\angle\bigl(T_yY,\Pi\bigr)<\varepsilon \quad\text{for all } y\in B_{c\,m^{-1/2}}(x).
\]
\end{proposition}

\begin{proof}
Choose covectors $\lambda_1,\lambda_2\in T_x^*X$ whose common kernel equals $\Pi$. By Lemma~\ref{lem:bergman}, pick $s_1,s_2$ with $s_i(x)=0$, $ds_i(x)=\lambda_i$ and $\|ds_i(y)-\lambda_i\|<\varepsilon$ on $B_{c\,m^{-1/2}}(x)$. For $m\gg 0$ and after a small generic perturbation inside the finite--dimensional linear system (which does not change jets at $x$ nor the $C^1$ estimates on the small ball), Bertini's theorem gives that $Y$ is smooth and $\{ds_1(y),ds_2(y)\}$ are linearly independent on the ball. The complex normal space to $Y$ at $y$ is spanned by $\{ds_1(y),ds_2(y)\}$, which is $\varepsilon$--close to $\{\lambda_1,\lambda_2\}$, hence $T_yY$ is $\varepsilon$--close to $\Pi$ for all $y$ in the ball.
\end{proof}

\section{Step 3: Local calibrated laminates on small cubes}

Fix a finite normal coordinate atlas by geodesic balls of radii $\ll 1$ and subordinate cubes $\{Q\}$ small enough so that the Carath\'eodory data from Lemma~\ref{lem:caratheodory} are $\varepsilon$--stable on each cube. For each cube $Q$ and each index $j\in\{1,\dots,N\}$, let $\Pi_{Q,j}$ denote a constant complex $(n-2)$--plane approximating $P_{x,j}$ on $Q$.

By Proposition~\ref{prop:tangent-approx}, for each $(Q,j)$ and for $m\gg 0$ there exists a smooth complete intersection $Y_{Q,j}\subset X$ given by $|L^m|$ such that on $Q$,
\[
\angle\bigl(T_yY_{Q,j},\Pi_{Q,j}\bigr)<\varepsilon.
\]
By shrinking $Q$ slightly and choosing generic sections in pencils, we obtain $N_{Q,j}$ disjoint parallel copies of $Y_{Q,j}$ crossing $Q$ with the same uniform angle control, for any prescribed integers $N_{Q,j}\ge 0$ (this uses again Bertini and the dimension of $H^0(X,L^m)$ for $m\gg 0$). Define the local current
\[
S_Q := \sum_{j=1}^{N} N_{Q,j}\,[Y_{Q,j}]\llcorner Q.
\]
By construction each $Y_{Q,j}$ is $\psi$--calibrated; hence each $S_Q$ is a positive $\psi$--calibrated integral current on $Q$. Its tangent--plane distribution on $Q$ is a convex combination of directions within $\varepsilon$ of $\{\Pi_{Q,j}\}$ with weights proportional to $N_{Q,j}$.

\begin{lemma}[Local barycenter matching]\label{lem:local-bary}
For any $\delta>0$ there exist integers $N_{Q,1},\dots,N_{Q,N}$ and a normalization factor $m_Q\in\N$ such that the tangent--plane Young measure of $S_Q$ has barycenter within $\delta$ (in $\HS$--norm) of $\beta$ on $Q$, and
\[
\Mass(S_Q) \to \int_Q \beta\wedge \psi \quad \text{as }\delta\to 0.
\]
\end{lemma}

\begin{proof}
This is a finite--dimensional rational approximation: approximate the weights $\theta_{x,j}$ (nearly constant on $Q$) by rationals $N_{Q,j}/m_Q$ so that $\sum_j (N_{Q,j}/m_Q)\,\xi_{\Pi_{Q,j}}$ is within $\delta$ of $\beta$ on $Q$. Because the tangent angles are $<\varepsilon$ and $\varepsilon\ll \delta$, the $\HS$--distance of barycenters is $\le C(\varepsilon+\delta)$. Calibratedness gives $\Mass([Y_{Q,j}]\llcorner Q) = \int_Q \psi\llcorner [Y_{Q,j}]$, yielding the displayed convergence.
\end{proof}

\section{Step 4: Exact cohomology matching by integer approximation}

Let $\{\Theta_\ell\}_{\ell=1}^{b}$ be a fixed integral basis of $H^{2(n-2)}(X,\Z)$ represented by smooth closed forms. Since $\beta$ represents $[\gamma]$, we have for every $\ell$,
\[
I_\ell := \int_X \beta\wedge \Theta_\ell = \langle [\gamma], [\Theta_\ell]\rangle \in \Q.
\]
Choose a common positive integer multiplier $m=m(\gamma)$ so that $m\,I_\ell\in\Z$ for all $\ell$.

On each cube $Q$, the current $S_Q$ constructed above satisfies, for each $\ell$,
\[
S_Q(\Theta_\ell) = \sum_{j} N_{Q,j}\int_{Y_{Q,j}\cap Q} \Theta_\ell = \int_Q \Bigl(\sum_{j}\tfrac{N_{Q,j}}{m_Q}\,\xi_{\Pi_{Q,j}}\Bigr)\wedge \Theta_\ell + O(\eta_Q),
\]
with $\eta_Q\to 0$ as $\varepsilon,\delta\to 0$. Summing over all cubes yields
\[
\sum_Q S_Q(\Theta_\ell) = \int_X \beta\wedge \Theta_\ell + O\Bigl(\sum_Q \eta_Q\Bigr).
\]

\begin{proposition}[Integral cohomology constraints]\label{prop:cohomology-match}
Given $\epsilon>0$, by refining the cube decomposition and choosing the integers $N_{Q,j}$ appropriately, one can achieve simultaneously for all $\ell=1,\dots,b$ that
\[
\biggl|\sum_Q S_Q(\Theta_\ell) - m\,I_\ell\biggr| < \tfrac12.
\]
Consequently, by integrality, $\sum_Q S_Q(\Theta_\ell) = m\,I_\ell$ for all $\ell$, i.e.\ the class of $\sum_Q S_Q$ in $H_{2(n-2)}(X,\Z)$ equals $\mathrm{PD}(m[\gamma])$.
\end{proposition}

\begin{proof}
This is a finite system of linear constraints with integer unknowns $\{N_{Q,j}\}$. By making the partition fine, the contribution of each $(Q,j)$ to the vector $(\Theta_\ell)$ can be made arbitrarily small and the feasible set becomes $\epsilon$--dense. Standard Diophantine approximation yields simultaneous approximation within $<\tfrac12$ for all $\ell$. Since the left side is an integer, equality follows.
\end{proof}

\section{Step 5: Boundary correction with vanishing mass}

The sum $S:=\sum_Q S_Q$ is supported in the union of cubes and typically has a small boundary supported on the inter--cube faces. By the Federer--Fleming Deformation Theorem (see Federer, GMT, 4.2.9) and the isoperimetric inequality on compact manifolds, there exist integral $(n-1)$--currents $U_\epsilon$ with
\[
\partial U_\epsilon = \partial S,\qquad \Mass(U_\epsilon)\xrightarrow[\epsilon\to 0]{}0.
\]
Define the closed integral current
\[
T_\epsilon := S - \partial U_\epsilon,\qquad \partial T_\epsilon=0.
\]
By construction, the cohomology class $[T_\epsilon]=[S]=\mathrm{PD}(m[\gamma])$ (Proposition~\ref{prop:cohomology-match}). Moreover, calibratedness of the $S_Q$ pieces gives
\[
\Mass(T_\epsilon) \le \Mass(S) + \Mass(\partial U_\epsilon) \to \int_X \beta\wedge \psi,
\]
since $\Mass(U_\epsilon)\to 0$ and $\Mass(\partial U_\epsilon)$ is controlled by $\Mass(U_\epsilon)$.

\section{Step 6: Young--measure convergence and calibrated limit}

Each $S_Q$ is a finite sum of $\psi$--calibrated sheets with uniformly controlled angles to fixed planes on $Q$; hence the tangent--plane distributions of $T_\epsilon$ generate Young measures $\nu_x^\epsilon$ supported in the calibrated Grassmann bundle. By Lemma~\ref{lem:local-bary}, the barycenters converge in $L^1$ to $\beta(x)$. Compactness of varifolds (Allard) gives a subsequence $T_{\epsilon_k}\to T$ as integral varifolds and currents. Lower semicontinuity and calibration yield
\[
\Mass(T)\le \liminf_k \Mass(T_{\epsilon_k}) = \int_X \beta\wedge \psi,
\]
while the calibration inequality gives the reverse bound; hence equality holds and $T$ is $\psi$--calibrated. The Young measures $\nu_x^{\epsilon_k}$ converge a.e.\ to a limiting field $\nu_x$ with barycenter $\beta(x)$. Thus $\beta$ is SYR--realizable, and $T$ is a $\psi$--calibrated integral cycle in class $\mathrm{PD}(m[\gamma])$. Harvey--Lawson identify $T$ with a positive combination of complex subvarieties, completing the proof of Theorem~\ref{thm:G1p2}.

\section*{References (standard sources)}

\begin{itemize}
	\item H.\ Federer, \emph{Geometric Measure Theory}, Springer, 1969.
	\item R.\ Harvey and H.\ B.\ Lawson, ``Calibrated geometries,'' Acta Math.\ 148 (1982), 47--157.
	\item R.\ Lazarsfeld, \emph{Positivity in Algebraic Geometry I}, Springer, 2004.
	\item G.\ Tian, ``On a set of polarized K\"ahler metrics on algebraic manifolds,'' J.\ Diff.\ Geom.\ 32 (1990), 99--130.
	\item D.\ Catlin, ``The Bergman kernel and a theorem of Tian,'' in \emph{Analysis and geometry in several complex variables}, 1999.
	\item S.\ Zelditch, ``Szeg\H{o} kernels and a theorem of Tian,'' IMRN 1998.
	\item S.\ K.\ Donaldson, ``Scalar curvature and projective embeddings, I,'' J.\ Diff.\ Geom.\ 59 (2001), 479--522.
	\item W.\ K.\ Allard, ``On the first variation of a varifold,'' Ann.\ of Math.\ 95 (1972), 417--491.
\end{itemize}

\end{document}

