\documentclass[11pt]{article}

\usepackage{amsmath,amssymb,amsthm}
\usepackage[margin=1in]{geometry}
\usepackage{hyperref}
\usepackage{xcolor}

\newcommand{\C}{\mathbb{C}}
\newcommand{\R}{\mathbb{R}}
\newcommand{\Z}{\mathbb{Z}}
\newcommand{\Q}{\mathbb{Q}}
\newcommand{\F}{\mathcal{F}}

\title{\bfseries Reviewer Memo: December 15, 2025 Update\\[0.3em]
\large From Conditional to Unconditional Hodge Closure}

\author{Jonathan Washburn\\
Recognition Science, Recognition Physics Institute\\
\texttt{jon@recognitionphysics.org}}

\date{December 15, 2025}

\begin{document}

\maketitle

\section*{Executive Summary}

The manuscript \texttt{hodge-SAVE-dec-12-handoff.tex} has been updated from a \textbf{conditional} proof (pending the ``microstructure/gluing'' step) to an \textbf{unconditional} proof of the Hodge Conjecture for rational $(p,p)$ classes on smooth projective K\"ahler manifolds.

The previous version you reviewed (\texttt{hodge-fix-dec-8-old.tex}) established the calibration--coercivity framework and the quantitative approximation to the calibrated cone.
That machinery remains unchanged.
What has been added is the \emph{realization/microstructure} step that was previously flagged as open.

\medskip
\noindent\textbf{Key change:} The proof now includes a complete ``corner-exit vertex-template'' construction that manufactures $\psi$-calibrated holomorphic complete intersections with controlled geometry, enabling the flat-norm gluing estimate $\F(\partial T^{\mathrm{raw}}) = o(m)$.

\section{What Was Missing Before (and Why)}

The previous manuscript reduced the Hodge Conjecture to the following:

\begin{quote}
\textbf{Microstructure/Gluing Theorem (informal).}
For every smooth closed cone-valued $(p,p)$-form $\beta$ representing an effective rational Hodge class, and for $m$ large and mesh $h$ small, there exists a construction of $\psi$-calibrated holomorphic complete-intersection pieces whose sum $T^{\mathrm{raw}}$ satisfies:
\begin{enumerate}
\item $T^{\mathrm{raw}}$ approximates $m\beta$ cellwise in the tangential/Young-measure sense, and
\item $\F(\partial T^{\mathrm{raw}}) = o(m)$.
\end{enumerate}
\end{quote}

The obstruction was that while we could build \emph{local} holomorphic pieces inside each cell, we could not control their \emph{boundary mismatch} across cell interfaces well enough to guarantee the flat-norm bound (ii).

The difficulty is combinatorial/geometric: on each interior face $F = Q \cap Q'$, the pieces from $Q$ and $Q'$ contribute boundary slices that must nearly cancel.
Without careful coordination, these mismatches can accumulate to $O(m)$ rather than $o(m)$.

\section{The Solution: Corner-Exit Vertex Templates}

The key insight is to use a \textbf{vertex-anchored, corner-exit} construction:

\begin{enumerate}
\item \textbf{Corner localization.} Each holomorphic sliver is anchored at a vertex $v$ of a cube $Q$ and has footprint contained in $B(v, c_0 h)$ with $c_0 < 1$.
This automatically prevents the sliver from reaching any face \emph{not} incident to $v$.

\item \textbf{Designated exit faces.} The footprint is a uniformly fat $k$-simplex ($k = 2n - 2p$) whose $k+1$ facets lie on exactly $k+1$ coordinate faces incident to $v$ (the ``exit faces'').
This gives a clean ``if and only if'' characterization of which faces a sliver meets.

\item \textbf{Prefix-based activation.} At each vertex, slivers are organized into an ordered template $(P_{v,1}, P_{v,2}, \ldots)$.
To realize $N$ slivers at vertex $v$, we activate the prefix $\{P_{v,a}\}_{a \leq N}$.
Because the template is the same on both sides of a shared vertex, slow variation of counts ($|N_{Q,v} - N_{Q',v}| \lesssim h \cdot N$) means mismatches are confined to the ``tail'' of the template.

\item \textbf{No heavy tails.} Because all slivers in a template have comparable mass (by equal footprint geometry), the tail mismatch is an $O(h)$ fraction of the total---not a rare heavy piece that dominates.
\end{enumerate}

This reduces the ``global face consistency'' problem to a \emph{local geometric} problem: can we actually manufacture holomorphic slivers with the required corner-exit footprint?

\section{New Lemmas and Propositions (Manuscript References)}

The following are the main new results added to achieve unconditional closure:

\subsection*{A. Corner-Exit Geometry (Euclidean Model)}

\begin{itemize}
\item \textbf{\texttt{lem:ball-excludes-faces}} (ball locality excludes nonincident faces): If $E \subset B(v, c_0 h)$ with $c_0 < 1$, then $E$ cannot meet any face not incident to $v$.

\item \textbf{\texttt{lem:corner-simplex-hits-designated-faces}} (fat corner simplex hits designated faces): A uniformly fat $k$-simplex with the corner-exit structure meets each designated exit face in a genuine $(k-1)$-patch.

\item \textbf{\texttt{lem:corner-simplex-face-mass}} (uniform per-face boundary mass): Each facet has $(k-1)$-mass comparable to $(\text{sliver mass})^{(k-1)/k}$.

\item \textbf{\texttt{lem:small-graph-distortion}} (graph area distortion): A $C^1$ graph with slope $\leq \varepsilon$ distorts $k$- and $(k-1)$-areas by $1 + O(\varepsilon^2)$.
\end{itemize}

\subsection*{B. Complex Corner-Exit Templates}

\begin{itemize}
\item \textbf{\texttt{lem:complex-corner-exit-template}} (explicit complex example): Constructs a concrete complex $(n-p)$-plane family with the corner-exit property, showing existence.

\item \textbf{\texttt{lem:corner-exit-template-open}} (quantitative template family): For a complex plane satisfying quantitative nondegeneracy bounds on its coefficients, there exists a corner-exit translation template with uniform fatness.
The key point: one can choose vertex and exit-face set, and there is a $(2p-1)$-parameter box of translations giving \emph{identical} footprints.

\item \textbf{\texttt{prop:corner-exit-template-net}} (robust templates for a finite net): For any $\varepsilon_h$-net of calibrated directions, one can perturb to ensure \emph{every} direction admits corner-exit templates, with uniform constants over the finite net.
\end{itemize}

\subsection*{C. Holomorphic Realization}

\begin{itemize}
\item \textbf{\texttt{lem:global-graph-contraction}} (contraction criterion for global graphs): If a holomorphic map $F(u,w)$ satisfies $|\partial_w F - I| \leq \eta$ uniformly on a product domain, then $\{F = 0\}$ is a single-sheet $C^1$ graph $w = g(u)$ on the entire domain.

\item \textbf{\texttt{lem:bergman-affine-approx-hormander}} (Bergman-scale affine approximation): Using cutoff + H\"ormander $\bar\partial$-solving, one can construct global holomorphic sections whose coefficient functions are uniformly $C^1$-close to any prescribed affine-linear model on a ball of radius $R/\sqrt{m}$.
The error is $O(e^{-cm})$---exponentially small.

\item \textbf{\texttt{prop:cell-scale-linear-model-graph}} (cell-scale single-sheet graphs): Combining the above, holomorphic complete intersections at Bergman scale ($h \lesssim m^{-1/2}$) are single-sheet graphs on entire cells.

\item \textbf{\texttt{prop:holomorphic-corner-exit-g1g2}} (holomorphic slivers inherit corner-exit geometry): Holomorphic small-slope graphs over fat corner-exit footprints satisfy (G1-iff) and (G2).
\end{itemize}

\subsection*{D. Global Assembly}

\begin{itemize}
\item \textbf{\texttt{prop:vertex-template-mass-matching}} (L2: mass budget matching): Nearest-integer rounding of prefix lengths matches cell mass budgets with relative error $O(1/N) + O(\varepsilon^2)$.

\item \textbf{\texttt{prop:vertex-template-face-edits}} (face-level $O(h)$ edits): Slow variation of counts implies unmatched boundary mass on each face is an $O(h)$ fraction.

\item \textbf{\texttt{prop:global-coherence-all-labels}} (B1: global coherence across all direction labels): Packages the full execution: choose corner-exit-admissible direction nets, Lipschitz weights, slow-variation integer counts, and cohomology discrepancy rounding; sum labelwise flat-norm bounds to get $\F(\partial T^{\mathrm{raw}}) = o(m)$.
\end{itemize}

\section{Proof Strategy Overview}

The complete proof now has the following structure:

\begin{enumerate}
\item \textbf{Signed decomposition (unchanged).}
Any rational Hodge class $\gamma$ decomposes as $\gamma = \gamma^+ - \gamma^-$ with both effective.
Here $\gamma^- = N[\omega^p]$ is already algebraic (complete intersections of hyperplanes).

\item \textbf{Calibration--coercivity (unchanged).}
For effective $\gamma^+$, the coercivity inequality forces energy-minimizing sequences toward the calibrated cone.
This is the content of the earlier version you reviewed.

\item \textbf{SYR construction (now complete).}
Build a sequence of integral $\psi$-calibrated cycles $T_k$ with:
\begin{itemize}
\item $\mathrm{Mass}(T_k) \to m \int_X \beta \wedge \psi$,
\item Tangent-plane Young measures converging to the prescribed distribution,
\item $[T_k] = \mathrm{PD}(m[\gamma^+])$.
\end{itemize}

The new corner-exit vertex-template machinery provides the ``$\F(\partial T^{\mathrm{raw}}) = o(m)$'' estimate that was previously missing.

\item \textbf{Federer--Fleming compactness + Harvey--Lawson (standard).}
The limit $T$ is a $\psi$-calibrated integral current, hence a positive sum of complex analytic subvarieties.
By Chow's theorem, these are algebraic.

\item \textbf{Conclusion.}
$\gamma = [Z^+] - [Z^-]$ is algebraic.
\end{enumerate}

\section{What to Check During Review}

For a careful review, I suggest focusing on:

\begin{enumerate}
\item \textbf{\texttt{prop:corner-exit-template-net}}: Does the perturbation argument correctly produce a finite net where every direction has corner-exit templates?
The key is that the ``bad'' set (where some coefficient vanishes) is a finite union of proper algebraic subvarieties.

\item \textbf{\texttt{lem:global-graph-contraction}}: Is the contraction-mapping argument correctly set up?
The domain must contain the cell $Q$, not just an infinitesimal neighborhood.

\item \textbf{\texttt{lem:bergman-affine-approx-hormander}}: The cutoff + H\"ormander construction is standard, but verify that the exponential decay $O(e^{-cm})$ propagates correctly to $C^1$ estimates on the inner ball.

\item \textbf{Parameter regime consistency}: Multiple constraints must coexist:
\begin{itemize}
\item $h \lesssim m^{-1/2}$ (Bergman control on cells),
\item $\delta \gtrsim \varepsilon h$ (disjointness of slivers),
\item $N_Q \gtrsim h^{-1}$ (many pieces for rounding),
\item The weighted flat-glue bound is $o(m)$.
\end{itemize}
These are recorded in \texttt{rem:weighted-scaling}; verify they are mutually satisfiable.

\item \textbf{Slow variation + discrepancy rounding compatibility}: \texttt{lem:slow-variation-discrepancy} shows that $0$--$1$ discrepancy rounding (B\'ar\'any--Grinberg style) preserves slow variation.
Check that the ``$+2$'' error term doesn't break the $O(h)$-fraction bound.
\end{enumerate}

\section{Files in the Repository}

\begin{itemize}
\item \texttt{hodge-SAVE-dec-12-handoff.tex} --- the master manuscript (now unconditional)
\item \texttt{mg-microstructure-gluing-target.txt} --- dependency DAG for the microstructure step
\item \texttt{strategy-and-progress.md} --- lab notebook of the proof development
\item \texttt{proof-completion-plan.md} --- standalone completion plan
\item \texttt{hodge-fix-dec-8-old.tex} --- the earlier version you reviewed (calibration--coercivity)
\end{itemize}

\section*{Summary of Changes from the Previous Version}

\begin{center}
\renewcommand{\arraystretch}{1.3}
\begin{tabular}{|p{5cm}|p{9cm}|}
\hline
\textbf{Previous Version} & \textbf{Current Version} \\
\hline
Calibration--coercivity established & Unchanged \\
\hline
Quantitative approximation to calibrated cone & Unchanged \\
\hline
SYR realization flagged as ``conditional on \texttt{rem:glue-gap}'' & SYR realization is now unconditional \\
\hline
No explicit corner-exit geometry & Full corner-exit vertex-template construction \\
\hline
No Bergman-scale graph control & Global graph lemma + H\"ormander construction \\
\hline
Headline theorems marked ``Conditional'' & Headline theorems are unconditional \\
\hline
\end{tabular}
\end{center}

\bigskip
\noindent
Thank you for your continued review.
Please contact me with any questions or concerns.

\end{document}

