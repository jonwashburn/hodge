\documentclass[11pt]{article}

\usepackage{amsmath,amssymb}
\usepackage[margin=1in]{geometry}
\usepackage{hyperref}

\newcommand{\C}{\mathbb{C}}
\newcommand{\R}{\mathbb{R}}
\newcommand{\Z}{\mathbb{Z}}
\newcommand{\Q}{\mathbb{Q}}
\newcommand{\F}{\mathcal{F}}
\newcommand{\Mass}{\mathrm{Mass}}

\title{\bfseries Note for Review: sanity check of the new microstructure step in codimension $p=2$}
\author{Jonathan Washburn}
\date{\today}

\begin{document}
\maketitle

\section*{Question being answered}

You asked: \emph{``Can you test the new proof in the first genuinely nontrivial case $p=2$?  If the proof is solid, can you claim it for $p=2$ before I invest in a full review?''}

\medskip
\noindent
This note explains how the manuscript specializes to $p=2$, and performs a concrete parameter/scaling check showing the core new estimate
\[
\F(\partial T^{\mathrm{raw}})=o(m)
\]
still closes in the $p=2$ regime.

\section{Scope of the claim for $p=2$}

The master manuscript \texttt{hodge-SAVE-dec-12-handoff.tex} states the main result for all $1\le p\le n$ and is intended to be uniform in $p$.
Therefore, \emph{as stated}, it includes the codimension--$2$ case $p=2$.

\medskip
\noindent
Two easy sanity checks about the surrounding classical landscape:
\begin{itemize}
\item If $n=3$ and $p=2$, then codimension $2$ cycles are Poincar\'e dual to $(1,1)$ classes; by Lefschetz $(1,1)$, the Hodge conjecture is already known in this case.  The manuscript also reduces large $p$ to small $p$ via Hard Lefschetz.
\item The first genuinely ``middle codimension'' test is $n\ge 4$, $p=2$ (e.g.\ $n=4$).  This is exactly where the naive PSD-identification for the calibrated cone fails, and where the quantitative replacement (the Dec 8 note \texttt{hodge-fix-dec-8-old.tex}) is needed.
\end{itemize}

\section{Where the value $p=2$ enters the new microstructure construction}

In the microstructure/gluing step, the dependence on $p$ is explicit and benign:
\begin{itemize}
\item The K\"ahler calibration is $\psi=\omega^{n-p}/(n-p)!$.  For $p=2$, $\psi=\omega^{n-2}/(n-2)!$ calibrates complex $(n-2)$--planes.
\item The \emph{real} dimension of each holomorphic sliver is
\[
k:=2n-2p=2n-4.
\]
\item The normal translation parameter lives in $\C^p=\C^2$ (real dimension $4$).  In the corner-exit template lemma, one typically fixes one real component (to force a chosen ``slanted'' inequality), leaving a $(2p-1)=3$ real dimensional box from which to pack many separated translations.
\item The key exponent in the weighted flat-norm bound is $(k-1)/k=(2n-5)/(2n-4)$.
\end{itemize}

\section{Local corner-exit template supply for $p=2$}

In the manuscript this is handled by:
\begin{itemize}
\item \texttt{lem:complex-corner-exit-template} (an explicit complex model), and
\item \texttt{lem:corner-exit-template-open} + \texttt{prop:corner-exit-template-net} (robust supply for a finite direction net, with uniform constants over the net).
\end{itemize}

\medskip
\noindent
Specializing those statements to $p=2$:
\begin{itemize}
\item we work in $\C^n=\C^{n-2}\times\C^2$ with $w=(w_1,w_2)$;
\item for each direction label, after choosing a vertex and a ``slanted'' coordinate (typically $w_1$), the footprint
\[
E(t) := (P+t)\cap Q
\]
is a $k$--simplex ($k=2n-4$) in the cube $Q=[0,h]^{2n}$ with $k+1=2n-3$ facets on a fixed designated set of cube faces incident to the chosen vertex;
\item the translation parameter box has real dimension $3$, so for any separation scale $\delta>0$ one can extract a long $\delta$--separated ordered list of translations $(t_a)$ producing identical footprints (hence identical per-piece slice masses within a label).
\end{itemize}

\section{Holomorphic realization for $p=2$ (two equations)}

For codimension $p=2$, each sliver is produced as a local piece of a holomorphic complete intersection
\[
Y=\{s_1=s_2=0\}
\]
with uniform single-sheet $C^1$ graph control on an entire cell.
In the manuscript this is routed through:
\begin{itemize}
\item \texttt{lem:bergman-affine-approx-hormander} (cutoff + H\"ormander $L^2$ solution gives $C^1$ approximation of affine-linear holomorphic models on a ball $B_{R/\sqrt m}$), and
\item \texttt{lem:global-graph-contraction} (a contraction criterion on a product domain giving a unique global graph $w=g(u)$), packaged into
\item \texttt{prop:cell-scale-linear-model-graph}.
\end{itemize}
None of these steps changes in substance for $p=2$; one simply has a two-component map $F=(F_1,F_2)$ instead of a $p$-component map.

\section{A concrete scaling/consistency check for $p=2$}

This is the main ``test'' one can do without re-proving every analytic lemma: verify that the parameter regime needed by the weighted flat-norm bound
is internally consistent for $p=2$ (and yields $o(m)$).

\subsection*{Set-up and notation}

Let $n\ge 4$ and $p=2$, hence $k=2n-4$.
Let $h$ be the cubical mesh and assume we are in the Bergman scale regime
\[
h \asymp m^{-1/2}.
\]
Let the corner-exit simplex scale inside each cube be
\[
s := \varepsilon\,h,
\]
so each sliver has mass on the order of
\[
\mu \asymp s^k = (\varepsilon h)^k.
\]
The total target mass per cube is on the order of
\[
M_Q \asymp m\,h^{2n}.
\]
Therefore the number of slivers needed per cube (per direction budget) is
\[
N_Q \asymp \frac{M_Q}{\mu} \asymp \frac{m\,h^{2n}}{(\varepsilon h)^k}
= m\,h^{2n-k}\,\varepsilon^{-k}
= m\,h^{4}\,\varepsilon^{-k}.
\]

\subsection*{Choose explicit parameters}

Choose
\[
h := m^{-1/2},\qquad \varepsilon := m^{-1/3},\qquad s:=\varepsilon h = m^{-5/6},\qquad \delta := 10\,\varepsilon h.
\]
Then:
\begin{itemize}
\item $N_Q \asymp m\,h^{4}\,\varepsilon^{-k} = m\cdot m^{-2}\cdot m^{k/3} = m^{k/3-1}$, which tends to $+\infty$ for all $n\ge 4$ (since then $k=2n-4\ge 4$ and $k/3-1>0$).
\item The local rounding error in mass-budget matching scales like $O(1/N_Q)+O(\varepsilon^2)$ and therefore tends to $0$.
\end{itemize}

\subsection*{Check the weighted flat-norm bound}

The weighted bound used in the manuscript has the schematic form
\[
\F(\partial T^{\mathrm{raw}})\ \lesssim\ h^2 \sum_{Q}\sum_{a\in\mathcal S(Q)} m_{Q,a}^{\frac{k-1}{k}},
\qquad k=2n-4,
\]
where $m_{Q,a}$ are the individual piece masses.
In the uniform model above, $m_{Q,a}\asymp \mu$, and the total number of pieces is $N_Q$ per cube times $h^{-2n}$ cubes.
Therefore
\[
\sum_{Q,a} m_{Q,a}^{\frac{k-1}{k}}
\ \asymp\ (N_Q\,h^{-2n})\cdot \mu^{\frac{k-1}{k}}
\ =\ N_Q\,h^{-2n}\cdot (\varepsilon h)^{k-1}.
\]
Using $N_Q \asymp m\,h^{4}\,\varepsilon^{-k}$ and $k=2n-4$, we get
\[
\sum_{Q,a} m_{Q,a}^{\frac{k-1}{k}}
\ \asymp\ m\,h^{4}\,\varepsilon^{-k}\,h^{-2n}\,\varepsilon^{k-1}\,h^{k-1}
\ =\ m\,\varepsilon^{-1}\,h^{-1}.
\]
Hence
\[
\F(\partial T^{\mathrm{raw}})\ \lesssim\ h^2\cdot \bigl(m\,\varepsilon^{-1}\,h^{-1}\bigr)
= m\,\frac{h}{\varepsilon}.
\]
With $h=m^{-1/2}$ and $\varepsilon=m^{-1/3}$,
\[
\F(\partial T^{\mathrm{raw}})\ \lesssim\ m\cdot \frac{m^{-1/2}}{m^{-1/3}}
= m^{5/6}
= o(m).
\]
This is the desired scaling in the microstructure/gluing step for $p=2$.

\section*{Conclusion}

In summary: the new microstructure/gluing mechanism in the manuscript is uniform in $p$ and specializes cleanly to $p=2$.
The first nontrivial test case $n\ge 4$, $p=2$ admits a consistent parameter regime (explicitly exhibited above) in which:
\begin{itemize}
\item one has many disjoint holomorphic corner-exit slivers per cell (so rounding is effective), and
\item the weighted flat-norm estimate yields $\F(\partial T^{\mathrm{raw}})=o(m)$.
\end{itemize}
Accordingly, the manuscript \emph{as written} claims the codimension--$2$ ($p=2$) case as a special case of the stated general theorem.

\end{document}


