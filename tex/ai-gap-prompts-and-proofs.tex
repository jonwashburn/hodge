% AI gap prompts + proofs (working document)
\documentclass[11pt]{article}

\usepackage{amsmath,amssymb,amsthm,mathtools}
\usepackage[margin=1in]{geometry}
\usepackage{hyperref}

\newtheorem{theorem}{Theorem}
\newtheorem{lemma}{Lemma}
\newtheorem{proposition}{Proposition}
\newtheorem{corollary}{Corollary}
\theoremstyle{definition}
\newtheorem{definition}{Definition}
\newtheorem{remark}{Remark}

\title{Gap Prompts and Proof Notes (Calibration / K\"ahler / Hodge Program)}
\author{}
\date{\today}

\begin{document}
\maketitle

\begin{remark}
This document is a working collection of (i) standalone prompts suitable for external solvers and (ii) proof notes / status checks.
Some items below are classical linear algebra or calibrated-geometry facts; others are essentially the missing heart of the Hodge conjecture.
\end{remark}

\tableofcontents

\section{Prompt 1 (Pointwise cone identification --- referee concern)}

\subsection*{Standalone prompt (copy/paste)}
\begin{quote}\small
Let $(X,\omega)$ be a K\"ahler manifold, fix $p$. At a point $x$, define $C_x\subset \Lambda^{p,p}T_x^*X$ to be the closed convex cone generated by the ``simple calibrated rays'' $\phi_V$ associated to complex $p$-planes $V\subset T_xX$ (equivalently $i^{p^2}v\wedge \bar v$ for decomposable $v\in \Lambda^{p,0}T_x^*X$).

Prove: $C_x$ equals the classical cone of strongly positive $(p,p)$-forms at $x$. Give an explicit linear-algebra identification and describe the extreme rays. Also clarify whether $C_x$ is (or is not) the full PSD cone in $\mathrm{Herm}(\Lambda^{p,0}T_x^*X)$ when $p>1$.
\end{quote}

\subsection*{Status}
\textbf{Provable (standard).} The identification $C_x=$ strong positivity is essentially definitional once one fixes conventions. The important clarification: \textbf{for $p>1$, $C_x$ is strictly smaller than the full PSD cone in $\mathrm{Herm}(\Lambda^{p,0})$} because ``rank-one PSD'' in $\mathrm{Herm}(\Lambda^{p,0})$ need not correspond to \emph{decomposable} $p$-vectors.

\subsection*{Proof (full linear algebra statement)}
Fix $x$ and write $E:=T_x^{1,0}X\cong \mathbb C^n$ with the Hermitian metric induced by $\omega$.
All statements below are pointwise in $(E,\omega_x)$.

\begin{definition}[Decomposables and rays]
Let $W:=\Lambda^{p,0}E^*$.
A vector $v\in W$ is \emph{decomposable} if $v=\alpha_1\wedge\cdots\wedge \alpha_p$ for some $\alpha_i\in E^*$.
For $v\in W$, define the real $(p,p)$-form
\[
\phi_v := i^{p^2}\, v\wedge \bar v\in \Lambda^{p,p}E^*_\mathbb R.
\]
If $v$ is decomposable and nonzero, then $\mathbb R_{\ge 0}\phi_v$ depends only on the complex $p$-plane $V:=\ker(\alpha_1)\cap\cdots\cap \ker(\alpha_p)$ (equivalently on the oriented complex $p$-plane in $E$).
\end{definition}

\begin{definition}[Strongly positive cone]
The \emph{strongly positive cone} $\mathrm{SP}^p(E)\subset \Lambda^{p,p}E^*_\mathbb R$ is the set of all finite sums
\[
\beta=\sum_{j=1}^N \lambda_j\,\phi_{v_j},\qquad \lambda_j\ge 0,\ \ v_j\in \Lambda^{p,0}E^*\ \text{decomposable}.
\]
\end{definition}

\begin{proposition}[Cone identification]\label{prop:prompt1-cone-id}
Let $C_x$ be the closed convex cone generated by the rays $\mathbb R_{\ge0}\phi_V$ as $V$ ranges over complex $p$-planes in $E$.
Then
\[
C_x = \mathrm{SP}^p(E).
\]
\end{proposition}
\begin{proof}
By definition, $\mathrm{SP}^p(E)$ is exactly the (finite) conic hull of the generating rays $\mathbb R_{\ge0}\phi_v$ with $v$ decomposable.
Taking closures, one obtains the closed convex cone generated by those rays, which is $C_x$ by construction.
\end{proof}

\begin{proposition}[Extreme rays]\label{prop:prompt1-extreme}
The extreme rays of $\mathrm{SP}^p(E)$ are precisely the rays $\mathbb R_{\ge0}\phi_v$ with $v\in W$ decomposable and nonzero.
\end{proposition}
\begin{proof}
Let $r:=\mathbb R_{\ge0}\phi_v$ with $v$ decomposable, $v\neq 0$.
Suppose $\phi_v=\beta_1+\beta_2$ with $\beta_i\in \mathrm{SP}^p(E)$.
Under the Hermitian-model map from Proposition~\ref{prop:prompt1-psd-compare} below, $\phi_v$ corresponds to a rank-one PSD Hermitian form.
A sum of PSD Hermitian forms has rank one only if each summand has range contained in the same one-dimensional subspace; hence each summand is itself a nonnegative multiple of $\phi_v$.
Therefore $r$ is extreme.

Conversely, if $\rho$ is an extreme ray of $\mathrm{SP}^p(E)$, then by Minkowski--Carath\'eodory in finite dimensions, any nonzero element of $\rho$ admits a representation as a finite sum of generators $\phi_{v_j}$.
If at least two non-proportional generators occur with positive coefficient, this yields a nontrivial decomposition of the ray, contradicting extremality.
Hence $\rho$ must be generated by a single $\phi_v$ with $v$ decomposable.
\end{proof}

\begin{proposition}[Weak positivity and the PSD cone; strictness for $p>1$]\label{prop:prompt1-psd-compare}
Let $W=\Lambda^{p,0}E^*$. Define a real-linear map
\[
\mathcal I:\Lambda^{p,p}E^*_\mathbb R \longrightarrow \mathrm{Herm}(W)
\]
by polarization from
\[
\big(\mathcal I(\beta)\big)(u,u) := \beta\big(i^{p^2}u\wedge \bar u\big),\qquad u\in W.
\]
Then:
\begin{enumerate}
\item $\beta$ is \emph{weakly positive} iff $\mathcal I(\beta)$ is PSD on $W$ (nonnegative on all $u\in W$).
\item $\beta$ is \emph{strongly positive} iff $\mathcal I(\beta)$ lies in the closed convex cone generated by rank-one rays $u\mapsto | \langle u, v\rangle|^2$ with $v$ decomposable.
\item If $p=1$, strong positivity = weak positivity = PSD.
\item If $p>1$, then $\mathrm{SP}^p(E)$ is a \emph{proper} subset of the PSD cone: there exist PSD Hermitian forms on $W$ that do not lie in $\mathcal I(\mathrm{SP}^p(E))$.
\end{enumerate}
\end{proposition}
\begin{proof}
(1)--(2) are standard and follow immediately from the definitions and the fact that $\mathcal I(\phi_v)$ is a rank-one PSD Hermitian form.

(3) If $p=1$, then $W=E^*$ and every $u\in W$ is decomposable, so the cones coincide.

(4) If $p>1$, there exist \emph{nondecomposable} $u\in W$.
Consider the rank-one PSD Hermitian form $H_u:=u\otimes \bar u$ (i.e.\ $w\mapsto |\langle w,u\rangle|^2$).
If $H_u$ were in the strong cone, then $H_u$ would be a finite sum of rank-one PSD forms $H_{v_j}$ with $v_j$ decomposable and nonnegative coefficients:
\[
H_u = \sum_j c_j\, H_{v_j},\qquad c_j\ge 0.
\]
But the range of $H_u$ is the one-dimensional subspace $\mathbb C u\subset W$, and the range of each $H_{v_j}$ is $\mathbb C v_j$.
If the sum has rank one, each $\mathbb C v_j$ must be contained in $\mathbb C u$, hence each $v_j$ is proportional to $u$.
Therefore $u$ would be decomposable, contradiction.
So $H_u$ is PSD but not in the strong cone, proving strictness.
\end{proof}

\subsection*{Citations}
Standard references for positivity cones: Demailly, \emph{Complex Analytic and Differential Geometry} (notes), section on positive $(p,p)$-forms/currents; also Harvey--Lawson \emph{Calibrated geometries} for calibration viewpoint.

\section{Prompt 2 (Carath\'eodory with uniform finite support + measurability)}

\subsection*{Standalone prompt (copy/paste)}
\begin{quote}\small
Fix $(n,p)$. For each $x$, let $C_x$ be the strongly positive cone (or the cone generated by $\phi_V$). Let $\beta$ be a smooth $(p,p)$-form on $X$ with $\beta(x)\in C_x$ for all $x$.

Prove (or disprove): there exists $N=N(n,p)$ and measurable (ideally piecewise-smooth) choices of complex $p$-planes $V_1(x),\dots,V_N(x)$ and weights $\lambda_i(x)\ge 0$ such that
\[
\beta(x)=\sum_{i=1}^N \lambda_i(x)\, \phi_{V_i(x)}\quad\text{for all }x,
\]
with $\sum_i\lambda_i(x)$ uniformly bounded and with quantitative control of $x\mapsto (\lambda_i,V_i)$ in terms of $\|\beta\|_{C^1}$ (or Lipschitz on cubes).
\end{quote}

\subsection*{Status}
\textbf{Partially provable; the ``quantitative $C^1$ control'' is the hard part.}
\begin{itemize}
\item \textbf{Pointwise finite support with uniform $N$:} yes, by Carath\'eodory in a fixed finite-dimensional vector space (works pointwise).
\item \textbf{Measurable selection:} plausible (Borel measurable selections exist under mild hypotheses), but one must formulate it carefully because decompositions are highly non-unique.
\item \textbf{Piecewise-smooth / Lipschitz control:} not automatic; generally false without extra structure because the decomposition map need not admit continuous selections globally.
\end{itemize}

\subsection*{Proof (finite $N$ and measurable selection)}
\paragraph{Setup.}
Work pointwise on $(E,\omega_x)\cong(\mathbb C^n,\omega_0)$ as in Prompt~1.
Let
\[
V:=\Lambda^{p,p}E^*_\mathbb R,\qquad d:=\dim_\mathbb R V,
\]
so $d$ depends only on $(n,p)$.
Let $K\subset V$ be the compact set of \emph{normalized generators}
\[
K:=\Big\{\frac{\phi_v}{\langle \phi_v,\omega^p\rangle}:\ v\in \Lambda^{p,0}E^*\ \text{decomposable},\ v\neq 0\Big\}.
\]
Then $K$ is compact, and its convex hull $\mathrm{conv}(K)\subset V$ is compact and convex.

\paragraph{Step 1: pointwise Carath\'eodory with uniform support size.}
Let $\beta$ be a smooth $(p,p)$-form with $\beta(x)$ strongly positive for all $x$.
Define the (positive) trace density
\[
t(x):=\langle \beta(x),\omega_x^p\rangle\ \ge\ 0.
\]
If $t(x)=0$, then $\beta(x)=0$ (since $\omega^p$ lies in the interior of the strong cone and the pairing with $\omega^p$ is strictly positive on nonzero strong forms), so take the trivial representation $\beta(x)=0$.
Assume $t(x)>0$ and define the normalized form
\[
\sigma(x):=\frac{1}{t(x)}\,\beta(x)\in V.
\]
By strong positivity, $\sigma(x)\in \mathrm{conv}(K)$ for every such $x$.
By Carath\'eodory's theorem in $\mathbb R^d$, each $\sigma(x)\in \mathrm{conv}(K)$ admits a representation as a convex combination of at most $d+1$ points of $K$:
\[
\sigma(x)=\sum_{i=1}^{N} a_i(x)\,k_i(x),\qquad a_i(x)\ge 0,\ \sum_{i=1}^N a_i(x)=1,\ \ k_i(x)\in K,
\]
with $N:=d+1$.
Multiplying by $t(x)$ yields
\[
\beta(x)=\sum_{i=1}^{N}\lambda_i(x)\,k_i(x),
\qquad \lambda_i(x):=t(x)a_i(x)\ge 0,\quad \sum_i \lambda_i(x)=t(x).
\]
Finally, by definition of $K$, each $k_i(x)$ is a normalized $\phi_{V_i(x)}$, so this is exactly
\[
\beta(x)=\sum_{i=1}^N \lambda_i(x)\,\phi_{V_i(x)}
\]
after reabsorbing the normalization into $\lambda_i(x)$.
Thus we may take $N=d+1=N(n,p)$ uniformly.

\paragraph{Step 2: uniform bound on $\sum_i\lambda_i(x)$.}
With the above normalization, $\sum_i\lambda_i(x)=t(x)=\langle \beta(x),\omega_x^p\rangle$.
Since $\beta$ is smooth and $X$ is compact, $t$ is bounded, hence $\sum_i\lambda_i(x)$ is uniformly bounded.

\paragraph{Step 3: measurable selection (existence of measurable $(\lambda_i,V_i)$).}
The above representation is nonunique; to obtain measurability, use a measurable selection theorem.

Let $\Delta_N:=\{a\in\mathbb R_{\ge0}^N:\sum_i a_i=1\}$ (compact simplex).
Consider the compact metric space $\mathcal Y:=\Delta_N\times K^N$.
Define the continuous map $G:\mathcal Y\to V$ by
\[
G(a,k):=\sum_{i=1}^N a_i k_i.
\]
For each $x$ with $t(x)>0$, define the (nonempty, compact) fiber
\[
F(x):=G^{-1}(\{\sigma(x)\})\subset \mathcal Y.
\]
The set $\mathrm{Graph}(F):=\{(x,y): y\in F(x)\}\subset X\times \mathcal Y$ is closed because $G$ is continuous and $\sigma$ is continuous.
Hence $F$ is a measurable set-valued map with nonempty compact values.
By the Kuratowski--Ryll-Nardzewski measurable selection theorem, there exists a Borel measurable selection $s(x)\in F(x)$.
Writing $s(x)=(a(x),k(x))$ and setting $\lambda_i(x):=t(x)a_i(x)$ yields Borel measurable $\lambda_i$ and $k_i$ (hence $V_i$) giving the desired decomposition for all $x$.

\paragraph{Step 4: why Lipschitz/piecewise-smooth control is not automatic.}
Even in finite-dimensional convex geometry, choosing a Carath\'eodory decomposition continuously can fail because extreme-point representations are highly nonunique and can ``jump'' when the point crosses loci where the set of supporting faces changes.
Thus \emph{measurable} selections are standard; \emph{continuous/Lipschitz} selections generally require additional structure (e.g.\ strict convexity/unique decomposition, or restricting to a region where a canonical algorithmic decomposition is stable).

\subsection*{Citations}
Carath\'eodory's theorem: any convex-geometry text.
Measurable selection: Kuratowski--Ryll-Nardzewski theorem (e.g.\ Castaing--Valadier, \emph{Convex Analysis and Measurable Multifunctions}).

\section{Prompt 3 (Quantitative relation: calibration deficit vs K\"ahler-angle defect)}

\subsection*{Standalone prompt (copy/paste)}
\begin{quote}\small
Let $\psi = \omega^{n-p}/(n-p)!$ on a K\"ahler manifold $(X,\omega)$. For an integral $(2n-2p)$-cycle $T$ with approximate unit tangent $\xi_T(x)$ defined $|T|$-a.e., define the calibration deficit
\[
\mathrm{Def}_{\mathrm{cal}}(T):= \mathrm{Mass}(T) - \langle [T],[\psi]\rangle = \int (1-\psi(\xi_T(x)))\,d|T|(x).
\]
Define a geometric ``K\"ahler-angle'' defect, e.g.
\[
D_{\mathrm{angle}}(T):=\int \mathrm{dist}(\xi_T(x),\mathrm{Gr}^{\mathrm{cal}})^2\,d|T|(x),
\]
where $\mathrm{Gr}^{\mathrm{cal}}$ is the calibrated Grassmannian of complex $(n-p)$-planes.
Prove explicit two-sided inequalities $\mathrm{Def}_{\mathrm{cal}}(T) \asymp D_{\mathrm{angle}}(T)$ with constants depending only on $(n,p)$.
\end{quote}

\subsection*{Status}
\textbf{Provable (compactness + Taylor expansion).} Pointwise, $1-\psi(\xi)$ is a smooth nonnegative function on the Grassmannian vanishing exactly on $\mathrm{Gr}^{\mathrm{cal}}$, with nondegenerate quadratic behavior transverse to $\mathrm{Gr}^{\mathrm{cal}}$. This yields global constants by compactness.

\subsection*{Proof (pointwise inequality + integration)}
Write $m:=n-p$ so $\psi$ is a $2m$-calibration.
Fix $x$ and identify $(T_xX,g_x,J_x,\omega_x)$ with $(\mathbb C^n,\langle\cdot,\cdot\rangle,J_0,\omega_0)$.
Let $\mathrm{Gr}_{2m}$ be the oriented Grassmannian of real $2m$-planes in $T_xX$ with its standard compact Riemannian metric.
Let $\mathrm{Gr}^{\mathrm{cal}}\subset \mathrm{Gr}_{2m}$ be the subset of complex $m$-planes (the calibrated Grassmannian).

\paragraph{Step 1: K\"ahler angles and Wirtinger formula.}
For each $\xi\in \mathrm{Gr}_{2m}$, there exist \emph{K\"ahler angles} $\theta_1(\xi),\dots,\theta_m(\xi)\in[0,\pi/2]$ such that, for the unit simple $2m$-vector $\vec\xi$ associated to $\xi$,
\begin{equation}\label{eq:wirt-kahler-angles}
\psi(\vec\xi)=\frac{\omega^{m}}{m!}(\vec\xi)=\prod_{j=1}^{m}\cos\theta_j(\xi).
\end{equation}
Moreover, $\xi\in \mathrm{Gr}^{\mathrm{cal}}$ iff $\theta_j(\xi)=0$ for all $j$.
(This is the classical Wirtinger inequality and its equality characterization.)

\paragraph{Step 2: $1-\psi$ is comparable to the squared K\"ahler-angle defect.}
Define the pointwise angle defect
\[
\delta(\xi):=\sum_{j=1}^{m}\theta_j(\xi)^2.
\]
Using \eqref{eq:wirt-kahler-angles} and elementary inequalities for $\cos$ on $[0,\pi/2]$, one gets \emph{explicit} constants:
\begin{lemma}\label{lem:cos-product-angle}
For all $\theta_1,\dots,\theta_m\in[0,\pi/2]$,
\[
\frac{2}{m\pi^2}\sum_{j=1}^{m}\theta_j^2
\;\le\;
1-\prod_{j=1}^{m}\cos\theta_j
\;\le\;
\frac{1}{2}\sum_{j=1}^{m}\theta_j^2.
\]
\end{lemma}
\begin{proof}
For the upper bound, use $1-\prod a_j \le \sum (1-a_j)$ for $a_j\in[0,1]$, and $1-\cos t\le t^2/2$ for $t\in[0,\pi/2]$:
\[
1-\prod_{j=1}^m\cos\theta_j \le \sum_{j=1}^m (1-\cos\theta_j)\le \frac12\sum_{j=1}^m \theta_j^2.
\]
For the lower bound, note $\prod_{j=1}^m \cos\theta_j \le \cos\theta_k$ for each $k$, hence
\[
1-\prod_{j=1}^m \cos\theta_j \ge 1-\cos\theta_{\max},
\quad \theta_{\max}:=\max_j\theta_j.
\]
Also $1-\cos t = 2\sin^2(t/2)\ge 2(t/\pi)^2$ for $t\in[0,\pi/2]$ (since $\sin s\ge 2s/\pi$ on $[0,\pi/2]$).
Thus $1-\cos\theta_{\max}\ge \tfrac{2}{\pi^2}\theta_{\max}^2$.
Finally $\theta_{\max}^2 \ge \tfrac{1}{m}\sum_j\theta_j^2$, giving the stated bound.
\end{proof}

Combining \eqref{eq:wirt-kahler-angles} with Lemma~\ref{lem:cos-product-angle} yields the pointwise estimate
\begin{equation}\label{eq:pointwise-defect-angle}
\frac{2}{m\pi^2}\,\delta(\xi)\ \le\ 1-\psi(\vec\xi)\ \le\ \frac12\,\delta(\xi),
\qquad \forall \xi\in \mathrm{Gr}_{2m}.
\end{equation}

\paragraph{Step 3: Relate $\delta(\xi)$ to distance from $\mathrm{Gr}^{\mathrm{cal}}$.}
In the standard Grassmannian metric, the squared distance from $\xi$ to the complex Grassmannian is comparable to $\delta(\xi)$:
there exist constants $a=a(m),b=b(m)>0$ such that
\begin{equation}\label{eq:dist-vs-angles}
a\,\delta(\xi)\ \le\ \mathrm{dist}(\xi,\mathrm{Gr}^{\mathrm{cal}})^2\ \le\ b\,\delta(\xi),
\qquad \forall \xi\in \mathrm{Gr}_{2m}.
\end{equation}
(This is a standard principal-angle fact: the geodesic distance is controlled by the Euclidean norm of the relevant angle coordinates; in particular, near $\mathrm{Gr}^{\mathrm{cal}}$ it is equivalent to $\sqrt{\delta(\xi)}$, and global constants follow by compactness.)

\paragraph{Step 4: Integrate against $|T|$.}
For an integral cycle $T$, the calibration identity gives
\[
\mathrm{Def}_{\mathrm{cal}}(T)=\int \bigl(1-\psi(\xi_T(x))\bigr)\,d|T|(x).
\]
Apply the pointwise bounds \eqref{eq:pointwise-defect-angle}--\eqref{eq:dist-vs-angles} to $\xi=\xi_T(x)$ and integrate to obtain constants
$c=c(n,p),C=C(n,p)>0$ such that
\[
c\int \mathrm{dist}(\xi_T(x),\mathrm{Gr}^{\mathrm{cal}})^2\,d|T|(x)
\;\le\;
\mathrm{Def}_{\mathrm{cal}}(T)
\;\le\;
C\int \mathrm{dist}(\xi_T(x),\mathrm{Gr}^{\mathrm{cal}})^2\,d|T|(x).
\]
This is exactly $\mathrm{Def}_{\mathrm{cal}}(T)\asymp D_{\mathrm{angle}}(T)$.
\qedhere
\end{proof}

\section{Prompt 4 (Calibrated integral current $\Rightarrow$ analytic cycle)}

\subsection*{Standalone prompt (copy/paste)}
\begin{quote}\small
Let $(X,\omega,J)$ be a complex manifold with K\"ahler form $\omega$ and $\psi = \omega^{n-p}/(n-p)!$. Let $T$ be an integral $(2n-2p)$-cycle calibrated by $\psi$ (i.e. $\psi(\xi_T)=1$ $|T|$-a.e.).
Prove: $T$ is a positive closed $(p,p)$-current and in fact equals $\sum m_i [V_i]$ where $V_i$ are irreducible complex analytic subvarieties of codimension $p$ and $m_i\in\mathbb N$.
\end{quote}

\subsection*{Status}
\textbf{Provable with standard theorems.} The key chain is:
calibrated $\Rightarrow$ complex tangent a.e. (Wirtinger equality) $\Rightarrow$ positive rectifiable current of bidimension $(p,p)$; together with $\partial T=0$ and integer multiplicities $\Rightarrow$ holomorphic chain $\Rightarrow$ analytic cycle.

\subsection*{Proof (with precise theorem inputs)}
Let $X$ be a complex $n$-fold with K\"ahler form $\omega$, and let $\psi=\omega^{n-p}/(n-p)!$.
Let $T$ be an integral $(2n-2p)$-cycle calibrated by $\psi$.

\paragraph{Step 1: calibration equality forces complex tangency a.e.}
By definition of calibration, $\psi(\xi)\le 1$ for every unit simple $(2n-2p)$-vector $\xi$.
Since $T$ is calibrated, $\psi(\xi_T(x))=1$ for $|T|$-a.e.\ $x$.
By the equality case in Wirtinger's inequality, this implies that the approximate tangent plane $\mathrm{Tan}(T,x)$ is a complex $(n-p)$-plane for $|T|$-a.e.\ $x$.

\paragraph{Step 2: type $(p,p)$ and positivity.}
The current $T$ is represented by integration over an $(n-p)$-rectifiable set with integer multiplicity and orientation.
Since its tangent planes are complex almost everywhere and its multiplicities are nonnegative (integral current orientation agrees with the complex orientation on calibrated planes), the associated current defines a \emph{strongly positive} current of bidimension $(p,p)$ in the sense of complex geometry.
Moreover, $T$ is closed as a current because $\partial T=0$.

\paragraph{Step 3: invoke the holomorphic-chain characterization.}
There is a classical theorem (Harvey--Shiffman) which characterizes holomorphic chains among rectifiable currents:
\begin{quote}\small
If $S$ is a locally rectifiable current of bidimension $(p,p)$ on a complex manifold which is \emph{positive} and \emph{$d$-closed} and has \emph{integer multiplicities}, then $S$ is a holomorphic $p$-chain, i.e.\ $S=\sum_i m_i [V_i]$ where $V_i$ are irreducible complex analytic subvarieties of codimension $p$ and $m_i\in\mathbb N$.
\end{quote}
Applying this theorem to $T$ yields the desired representation $T=\sum_i m_i [V_i]$.

\subsection*{Citations (canonical)}
\begin{itemize}
\item R.\ Harvey and B.\ Shiffman, \emph{A characterization of holomorphic chains}, Ann.\ of Math.\ (2) \textbf{99} (1974), 553--587.
\item R.\ Harvey and H.\ B.\ Lawson, Jr., \emph{Calibrated geometries}, Acta Math.\ \textbf{148} (1982), 47--157.
\end{itemize}

\section{Prompt 5 (Projective: analytic $\Rightarrow$ algebraic)}

\subsection*{Standalone prompt (copy/paste)}
\begin{quote}\small
Let $X$ be smooth complex projective. Assume $T$ is a $\psi$-calibrated integral cycle and hence an analytic cycle.
Prove that each analytic component is algebraic (Chow) and that $\mathrm{PD}([T])$ is a rational Hodge class of type $(p,p)$.
\end{quote}

\subsection*{Status}
\textbf{Provable (standard).} Chow's theorem + the fact that algebraic cycles define integral $(p,p)$-classes.

\subsection*{Proof (make ``therefore $\gamma$ is algebraic'' explicit)}
Assume $X$ is smooth complex projective, and $T$ is a $\psi$-calibrated integral cycle of real dimension $2(n-p)$.
By Prompt~4, $T=\sum_i m_i [V_i]$ where $V_i\subset X$ are irreducible complex analytic subvarieties of codimension $p$ and $m_i\in\mathbb N$.

\paragraph{Step 1: analytic $\Rightarrow$ algebraic on a projective variety.}
By Chow's theorem, every closed complex analytic subvariety of a complex projective variety is algebraic.
Hence each $V_i$ is an algebraic subvariety of codimension $p$ and $T$ is an algebraic cycle (with integer multiplicities).

\paragraph{Step 2: cohomology class is integral and of type $(p,p)$.}
For each irreducible codimension-$p$ algebraic subvariety $V\subset X$, the fundamental class $[V]\in H_{2n-2p}(X,\mathbb Z)$ has Poincar\'e dual
\[
\mathrm{PD}([V])\in H^{2p}(X,\mathbb Z)\cap H^{p,p}(X),
\]
because integration over $V$ defines a closed current of type $(p,p)$.
Therefore
\[
\mathrm{PD}([T])=\sum_i m_i\,\mathrm{PD}([V_i])\in H^{2p}(X,\mathbb Z)\cap H^{p,p}(X).
\]

\paragraph{Step 3: ``therefore $\gamma$ is algebraic''.}
If a rational Hodge class $\gamma\in H^{2p}(X,\mathbb Q)\cap H^{p,p}(X)$ satisfies $\mathrm{PD}(m\gamma)=[T]$ for some $m\in\mathbb N$ and an algebraic cycle $T$, then
\[
\gamma=\frac{1}{m}\sum_i m_i\, [V_i]^\vee
\]
in cohomology, i.e.\ $\gamma$ is a rational linear combination of algebraic cycle classes.
This is exactly the Hodge-conjecture conclusion for $\gamma$.

\subsection*{Citations}
Chow's theorem is standard (e.g.\ Griffiths--Harris).

\section{Prompt 6 (Core missing theorem)}

\subsection*{Standalone prompt (copy/paste)}
\begin{quote}\small
Let $X$ be a smooth complex projective manifold of dimension $n$ with K\"ahler form $\omega$ and $\psi=\omega^{n-p}/(n-p)!$. Let $\gamma^+ \in H^{2p}(X,\mathbb Q)\cap H^{p,p}(X)$ be such that it has a smooth closed representative $\beta$ with $\beta(x)$ in the strongly positive cone for all $x$.
Prove (or give a counterexample): there exists $m\ge1$ and an integral $(2n-2p)$-cycle $T$ with $[T]=\mathrm{PD}(m[\gamma^+])$ and $\mathrm{Mass}(T)=\langle [T],[\psi]\rangle$ (equivalently $T$ is $\psi$-calibrated).
\end{quote}

\subsection*{Status}
\textbf{Open in general; known in special cases.}
For $p=1$ (divisors), it reduces to Lefschetz $(1,1)$ plus effectivity.
For $p=n-1$ (curve classes), it follows from hard Lefschetz + Lefschetz $(1,1)$ (intersection with hyperplane powers).
For general $1<p<n-1$, this is essentially a positivity-strengthened Hodge-type statement.

\subsection*{What would count as a proof here (and why it is ``the missing heart'')}
By Proposition~\ref{prop:prompt7-sandwich}, the desired conclusion for $\gamma^+$ is equivalent to the existence of at least one $\psi$-calibrated integral representative of $h=\mathrm{PD}(m[\gamma^+])$.
By Prompt~4, any $\psi$-calibrated integral cycle is an analytic cycle, and by Prompt~5 (projective case) it is algebraic.
Therefore, in the projective K\"ahler setting, Prompt~6 is essentially equivalent to:
\begin{quote}\small
If a rational Hodge class admits a \emph{smooth closed strongly positive} representative, then (after clearing denominators) it lies in the cone generated by algebraic codimension-$p$ cycles.
\end{quote}
This is a strong ``positivity $\Rightarrow$ algebraicity'' principle; beyond low codimension/degree it is not currently known in general.

\subsection*{Special cases (what is actually provable)}
\paragraph{Case $p=1$ (divisors).}
If $\gamma^+\in H^{1,1}(X)\cap H^2(X,\mathbb Q)$ admits a smooth semipositive representative, then it is nef; on a projective manifold, nef integral $(1,1)$ classes are limits of effective divisors, and when the class is also integral and big, one gets effective representatives via Kodaira-type results.
At the level of \emph{Hodge conjecture}, Lefschetz $(1,1)$ already gives algebraicity of integral $(1,1)$ classes.

\paragraph{Case $p=n-1$ (curves).}
Curve classes are dual to divisor classes by hard Lefschetz. Combined with Lefschetz $(1,1)$, one can often reduce to algebraic curve classes via intersection theory.

\section{Prompt 8 (Local-to-global realization / microstructure)}

\section{Prompt 7 (Stable norm linearity)}

\subsection*{Standalone prompt (copy/paste)}
\begin{quote}\small
Let $\psi$ be a calibration on a compact Riemannian manifold $X$. Define the stable mass norm
\[
M(h):=\inf\{\mathrm{Mass}(T): T \text{ integral cycle},\ [T]=h\}
\]
on $H_k(X,\mathbb Z)$. Fix $h$ and define $L(h):=\langle h,[\psi]\rangle$.
Prove: $M(h)=L(h)$ iff there exists a $\psi$-calibrated integral representative of $h$.
Then, for $X$ projective K\"ahler and $\psi=\omega^{n-p}/(n-p)!$, characterize the cone of $h$ with $M(h)=L(h)$.
\end{quote}

\subsection*{Status}
\textbf{First equivalence is provable (one-line sandwich).} The characterization in the K\"ahler projective case reduces to ``which classes admit calibrated representatives'', i.e. to Prompts 4--6.

\subsection*{Proof (equivalence $M(h)=L(h)$)}
\begin{proposition}\label{prop:prompt7-sandwich}
Let $X$ be compact, and let $\psi$ be a closed calibration of degree $k$ (comass $\le 1$).
Fix $h\in H_k(X,\mathbb Z)$ and define
\[
L(h):=\langle h,[\psi]\rangle,\qquad
M(h):=\inf\{\mathrm{Mass}(T): T \text{ integral cycle},\ [T]=h\}.
\]
Then $M(h)=L(h)$ if and only if there exists a $\psi$-calibrated integral cycle $T$ with $[T]=h$.
\end{proposition}
\begin{proof}
For any integral cycle $T$ with $[T]=h$, calibration gives $\langle T,\psi\rangle\le \mathrm{Mass}(T)$, and closedness gives $\langle T,\psi\rangle=\langle h,[\psi]\rangle=L(h)$.
Thus $\mathrm{Mass}(T)\ge L(h)$ for all such $T$, hence $M(h)\ge L(h)$.

If there exists a $\psi$-calibrated $T_0$ with $[T_0]=h$, then
$\mathrm{Mass}(T_0)=\langle T_0,\psi\rangle=L(h)$, so $M(h)\le \mathrm{Mass}(T_0)=L(h)$.
Therefore $M(h)=L(h)$.

Conversely, if $M(h)=L(h)$, let $T_{\min}$ be a mass minimizer in class $h$ (existence is standard by GMT compactness).
Then $\mathrm{Mass}(T_{\min})=M(h)=L(h)=\langle T_{\min},\psi\rangle$, so $T_{\min}$ saturates the calibration inequality and is $\psi$-calibrated.
\end{proof}

\section{Prompt 8 (Local-to-global realization / microstructure)}

\subsection*{Standalone prompt (copy/paste)}
\begin{quote}\small
In a coordinate chart $U\subset X$ with approximately constant $\omega$, suppose you are given a smooth strongly positive closed $(p,p)$-form $\beta$ on $U$ representing an integral cohomology class (after scaling).
Design (or prove impossible): a construction producing an integral $(2n-2p)$-cycle $T_U$ in $U$ whose mass is controlled by $\int_U \beta\wedge\psi$ (no blow-up), whose boundary errors are controllable on $\partial U$, and whose homology contribution matches $\mathrm{PD}([\beta|_U])$ in relative homology. Give explicit quantitative bounds and explain how to glue across a cube decomposition to get a global closed integral cycle.
\end{quote}

\subsection*{Status}
\textbf{Essentially the hard realization step; unknown at this strength.}
This is a precise formulation of the missing geometric measure theory ``realization'' mechanism that would turn diffuse cone-valued forms into calibrated integral cycles without mass blow-up.

\subsection*{Why this is hard (one-paragraph obstruction summary)}
The difficulty is that a smooth form $\beta$ is diffuse (nonzero on a set of full measure), whereas an integral cycle is supported on a rectifiable set of codimension $p$.
Producing a sequence of cycles whose tangent measures ``sample'' $\beta$ without increasing total mass is a microstructure/lamination problem.
In many contexts, sufficiently fine ``wrinkling'' forces mass growth; avoiding that requires a new structure theorem that connects closedness of $\beta$ and strong positivity to integrability into genuine complex-analytic strata.
At present, such a theorem in the generality needed to imply Hodge is not available.

\subsection*{Concrete quantitative target (flat-norm gluing lemma)}
The global SYR gluing step can be reduced to the following \emph{quantitative flat-norm estimate}, which is robust under cancellation.

\begin{lemma}[Flat-norm gluing criterion]\label{lem:prompt8-flat-glue}
Let $X$ be a compact oriented Riemannian manifold and let $k\in\{1,\dots,\dim X-1\}$.
Let $T^{\mathrm{raw}}$ be an integral $k$-current on $X$ (not necessarily closed) and set $S:=\partial T^{\mathrm{raw}}$.
Assume there is a number $\varepsilon>0$ such that for every smooth $(k-1)$-form $\eta$ with
$\|\eta\|_{\mathrm{comass}}\le 1$ and $\|d\eta\|_{\mathrm{comass}}\le 1$ one has
\[
|S(\eta)|=|\partial T^{\mathrm{raw}}(\eta)|=|T^{\mathrm{raw}}(d\eta)|\le \varepsilon.
\]
Then $\mathcal F(S)\le \varepsilon$, where $\mathcal F$ is the flat norm.
Moreover, there exists an integral $k$-current $R_{\mathrm{glue}}$ with
\[
\partial R_{\mathrm{glue}}=-S
\]
and a mass bound
\[
\Mass(R_{\mathrm{glue}})\le C\bigl(\varepsilon+\varepsilon^{\frac{k}{k-1}}\bigr),
\]
where $C=C(X,k)$ depends only on the geometry of $X$ and the Federer--Fleming isoperimetric constant.
\end{lemma}

\begin{proof}[Sketch]
The first claim is the dual characterization of the flat norm (Federer--Fleming), which identifies
$\mathcal F(S)$ with the supremum of $S(\eta)$ over test forms with $\|\eta\|_{\mathrm{comass}},\|d\eta\|_{\mathrm{comass}}\le 1$.

For the second claim, use the definition of $\mathcal F(S)$ to write $S=R+\partial Q$ with
$\Mass(R)+\Mass(Q)\le 2\mathcal F(S)\le 2\varepsilon$.  Since $S$ is a boundary, $R=S-\partial Q$ is also a boundary,
hence null-homologous.  Apply the Federer--Fleming isoperimetric inequality to fill $R$ by an integral current $Q_R$ with
$\partial Q_R=R$ and $\Mass(Q_R)\le C\,\Mass(R)^{k/(k-1)}$.  Then $R_{\mathrm{glue}}:=-(Q+Q_R)$ satisfies
$\partial R_{\mathrm{glue}}=-S$ and the stated mass bound.
\end{proof}

\paragraph{How this connects to closedness of $\beta$.}
In the SYR program one has $k=2n-2p$ and $T^{\mathrm{raw}}=\sum_Q S_Q$ built from local calibrated sheet-stacks on cubes.
The key missing estimate is to prove (uniformly in test forms $\eta$ with $\|\eta\|_{\mathrm{comass}},\|d\eta\|_{\mathrm{comass}}\le 1$) that
\[
T^{\mathrm{raw}}(d\eta)\approx m\int_X \beta\wedge d\eta.
\]
If $\beta$ is closed, then $\int_X \beta\wedge d\eta=\pm\int_X d(\beta\wedge\eta)=0$, so such an approximation implies the hypothesis of
Lemma~\ref{lem:prompt8-flat-glue} and hence produces a small-mass gluing correction $R_{\mathrm{glue}}$.
Making this approximation quantitative in terms of $(\delta,\varepsilon,\mathrm{mesh},m)$ is exactly the ``microstructure/gluing'' heart of Prompt~8.

\subsection*{A potentially viable route: transport on transverse parameters (flat model)}
One way to make the needed flat-norm estimate plausible is to exploit the fact that the dual constraint
$\|d\eta\|_{\mathrm{comass}}\le 1$ forces \emph{Lipschitz control} of boundary integrals under small transverse shifts.
In a flat chart, families of (almost) parallel sheets are naturally parameterized by their transverse translation.
This suggests a quantitative gluing strategy based on \emph{optimal transport} in the transverse parameter space.

\begin{lemma}[Parallel-sheet mismatch controlled by Wasserstein transport (flat chart)]\label{lem:prompt8-w1}
Let $k\ge 1$ and $q\ge 1$, and consider $\mathbb R^{k+q}=\mathbb R^k_x\times\mathbb R^q_y$ with the standard metric.
Fix the oriented $k$-plane $P:=\mathbb R^k\times\{0\}$.
Let $Q^-:=[-1,0]\times[0,1]^{k-1}\times[0,1]^q$ and $Q^+:=[0,1]\times[0,1]^{k-1}\times[0,1]^q$ be two adjacent unit cubes
sharing the interface face $F:=\{0\}\times[0,1]^{k-1}\times[0,1]^q$.

For $y\in[0,1]^q$, let $S_y^\pm$ denote the oriented $k$-current given by integration over
the $k$-dimensional slab $(\mathbb R^k\times\{y\})\cap Q^\pm$.
Given two finite nonnegative integer-valued measures on $[0,1]^q$,
\[
\mu:=\sum_{a=1}^N m_a\,\delta_{y_a},\qquad
\mu':=\sum_{b=1}^{N'} m'_b\,\delta_{y'_b},
\]
define integral currents $T^-:=\sum_a m_a S_{y_a}^-$ and $T^+:=\sum_b m'_b S_{y'_b}^+$ and let
$B_F$ be the interface mismatch current on $F$:
\[
B_F:=\bigl(\partial T^-\bigr)\llcorner F\;+\;\bigl(\partial T^+\bigr)\llcorner F.
\]
Then there is a constant $C=C(k,q)$ such that for every smooth $(k-1)$-form $\eta$ on $\mathbb R^{k+q}$ with
$\|\eta\|_{\mathrm{comass}}\le 1$ and $\|d\eta\|_{\mathrm{comass}}\le 1$,
\[
|B_F(\eta)|\ \le\ C\,W_1(\mu,\mu'),
\]
where $W_1$ is the $1$-Wasserstein distance on $[0,1]^q$ (with Euclidean cost).
Consequently, the flat norm of $B_F$ (as a $(k-1)$-current on $F$) obeys
\[
\mathcal F(B_F)\ \le\ C\,W_1(\mu,\mu').
\]
\end{lemma}

\begin{proof}[Idea]
For fixed $y$, the slice current $(\partial S_y^-)\llcorner F$ is just the oriented $(k-1)$-slab
in $F$ at transverse parameter $y$; denote it by $\Sigma_y$.
Define $f_\eta(y):=\Sigma_y(\eta)$.
If $y,y'$ are close, the difference $\Sigma_y-\Sigma_{y'}$ is the boundary of the obvious $k$-dimensional cylinder
between the two slabs, whose mass is $O(|y-y'|)$; Stokes gives
$
|f_\eta(y)-f_\eta(y')|
=
|(\Sigma_y-\Sigma_{y'})(\eta)|
\le \|d\eta\|_{\mathrm{comass}}\cdot \Mass(\text{cylinder})
\le C|y-y'|.
$
Thus $f_\eta$ is $C$-Lipschitz on $[0,1]^q$.  Since
$
B_F(\eta)=\int f_\eta\,d\mu-\int f_\eta\,d\mu',
$
Kantorovich--Rubinstein duality implies the stated bound by $C\,W_1(\mu,\mu')$.
\end{proof}

\begin{remark}[Robustness under small angle errors]
In the actual K\"ahler sheet construction, sheets are only \emph{approximately} parallel on a cube $Q$:
$\sup_{y\in Q}\angle(T_yY,\Pi)\le \varepsilon$.
In a flat chart, one can compare each such sheet to an exactly-parallel translate by projecting along normal coordinates.
Because $\|\eta\|_{\mathrm{comass}}\le 1$ and $\|d\eta\|_{\mathrm{comass}}\le 1$, the induced slice function $f_\eta$ remains Lipschitz
up to an additional error $O(\varepsilon)$ (coming from the distortion of the slice geometry and the comparison cylinder).
Thus one expects an inequality of the schematic form
\[
|B_F(\eta)|\ \le\ C\bigl(W_1(\mu,\mu')+\varepsilon\cdot \Mass(\text{reference slice stack})\bigr),
\]
and hence $\mathcal F(B_F)\lesssim W_1(\mu,\mu')+O(\varepsilon)$.
Making this precise is technical but conceptually straightforward once a uniform tubular-coordinate model for the sheets is fixed.
\end{remark}

\paragraph{Why this matters for Prompt 8.}
In Substep 4.2 of the SYR construction, each cube $Q$ builds many nearly-parallel calibrated sheets;
their restrictions to a face $F$ can be viewed as a discrete measure on a $(2p)$-dimensional transverse parameter space.
Lemma~\ref{lem:prompt8-w1} shows that, at least in the flat/parallel model, controlling the face mismatch in flat norm reduces to
controlling a \emph{transport distance} between these discrete transverse measures.
This suggests a concrete program:
\begin{itemize}
\item Prove that closedness of $\beta$ implies a discrete ``conservation law'' across faces for the target transverse measures.
\item Choose the integer sheet counts/placements so that adjacent-face transverse measures match up to $W_1$-error $o(1)$.
\item Deduce $\mathcal F(\partial T^{\mathrm{raw}})=o(1)$ by summing face bounds and invoking Lemma~\ref{lem:prompt8-flat-glue}.
\end{itemize}
This does not complete Prompt~8, but it upgrades the gap from an amorphous ``microstructure'' issue to a concrete
transport-and-rounding problem in the transverse parameter space, which is closer in spirit to CPM/finite-resolution ideas.

\begin{remark}[Closedness as a discrete cancellation law]\label{rem:prompt8-closedness-cancel}
The identity $d\beta=0$ gives an exact cancellation mechanism that is well matched to the flat-norm dual formulation.
Indeed, for any smooth $(k-1)$-form $\eta$ on a cubulated manifold $X$ (here $k=2n-2p$),
\[
\int_X (m\beta)\wedge d\eta = \pm \int_X d(m\beta\wedge\eta)=0.
\]
If $X=\bigcup_Q Q$ is a cubulation, then
\[
0=\sum_Q \int_Q (m\beta)\wedge d\eta
= \pm \sum_Q \int_{\partial Q} (m\beta)\wedge \eta,
\]
and the interior-face contributions cancel in pairs (opposite orientations).
Thus, to show $\partial T^{\mathrm{raw}}(\eta)=T^{\mathrm{raw}}(d\eta)$ is small for all test $\eta$ with
$\|\eta\|_{\mathrm{comass}},\|d\eta\|_{\mathrm{comass}}\le 1$, it suffices to show a \emph{local cubewise approximation}
of the form
\[
S_Q(d\eta)\approx \int_Q (m\beta)\wedge d\eta
\quad\text{(equivalently }\partial S_Q(\eta)\approx \int_{\partial Q}(m\beta)\wedge\eta\text{)}.
\]
In the sheet-stack setting, $\partial S_Q(\eta)$ is a sum of face-slice integrals of $\eta$, and the transport viewpoint
explains why bounding $\|d\eta\|_{\mathrm{comass}}$ is the right hypothesis: it makes those slice integrals Lipschitz under small
transverse shifts.
\end{remark}

\subsection*{Quantization estimates in $W_1$ (rounding a Lipschitz density)}
To make the transport route usable, one needs quantitative control of how well a smooth (or Lipschitz) transverse density
can be approximated by an \emph{integer-weighted} discrete measure (sheet counts), and how the $W_1$ error scales.
The following standard estimates capture the relevant scaling.

\begin{lemma}[Grid quantization gives $W_1$ error $O(\delta)$]\label{lem:w1-quantize}
Let $\Omega:=[0,1]^q$ with Euclidean metric and let $\rho\in L^1(\Omega)$ be nonnegative with total mass
$M:=\int_\Omega \rho\,dy<\infty$.
Let $\delta\in(0,1)$ and partition $\Omega$ into a grid of axis-aligned cubes $\{C_i\}$ of side length $\delta$.
Pick a point $y_i\in C_i$ (e.g.\ the center) and define the discrete measure
\[
\mu_\delta:=\sum_i \Bigl(\int_{C_i}\rho\,dy\Bigr)\,\delta_{y_i}.
\]
Then
\[
W_1(\rho\,dy,\mu_\delta)\ \le\ \sqrt{q}\,\delta\,M.
\]
\end{lemma}

\begin{proof}
Define a coupling $\pi$ by transporting each point $y\in C_i$ to $y_i$:
$\pi:=(\mathrm{id},T)_\#(\rho\,dy)$ where $T(y):=y_i$ for $y\in C_i$.
Then the $\pi$-transport cost is bounded by $\sup_{y\in C_i}\|y-y_i\|\le \frac{\sqrt q}{2}\delta$ on each cell, hence
\[
W_1(\rho\,dy,\mu_\delta)\le \int_{\Omega}\|y-T(y)\|\,\rho(y)\,dy \le \sqrt q\,\delta \int_\Omega \rho\,dy=\sqrt q\,\delta\,M.
\]
\end{proof}

\begin{remark}[Integer rounding]
Lemma~\ref{lem:w1-quantize} produces real weights.  In the sheet setting, one further rounds weights to integers.
Rounding changes total mass by at most one unit per cell; for large total mass (large $m$) this can be arranged so that the
relative $W_1$ error is arbitrarily small by choosing a sufficiently fine grid and then scaling by a common denominator.
\end{remark}

\subsection*{Cube-consistency: simultaneous face constraints from one translation set}
Even if one can enforce $W_1$-matching across each \emph{individual} face, there is an additional “cube-consistency’’ constraint:
within a given cube $Q$, each sheet is determined by a single translation parameter $t\in N^\perp\cong\R^{2p}$, and that single $t$
determines the sheet’s intersection with \emph{all} faces of $Q$ simultaneously.  Thus one cannot choose independent transverse measures
for different faces; they must arise as pushforwards of a \emph{single} discrete measure on translation space under the relevant face-slice maps.

\begin{conjecture}[Cube-consistent transverse matching]\label{conj:cube-consistent}
Fix a cube $Q$ and a calibrated direction (or family of directions) with normal translation space $N^\perp\cong\R^{2p}$.
For each face $F\subset\partial Q$, let $\Phi_F:N^\perp\to \Omega_F$ be the (approximately linear) map sending a translation parameter
to the induced transverse parameter on that face.  Given target face measures $\nu_F$ (coming from the smooth form $\beta$),
prove that there exists a discrete integer-weighted measure $\mu$ on $N^\perp$ such that simultaneously
\[
W_1\bigl((\Phi_F)_\#\mu,\ \nu_F\bigr)\le \tau_F
\qquad\text{for all faces }F\subset\partial Q,
\]
with quantitative bounds on $\tau_F$ that are compatible with the global gluing estimate (and with $\mu(\,N^\perp\,)\asymp m$ at the appropriate scaling).
\end{conjecture}

\paragraph{Partial progress (pure combinatorics).}
If the face maps $\Phi_F$ are coordinate projections (or coordinate permutations) on a finite grid model, then the problem reduces to realizing
compatible 1D marginals by a multiset of grid points, which is always possible by standard flow/matching arguments (see `hodge-blocker.tex`, Lemma~\ref{lem:discrete-marginals}).
For genuinely nontrivial linear face maps, this becomes a higher-order multi-marginal realization problem and is a plausible source of real obstruction.

\subsection*{A sharpened fixed-$m$ obstruction and a “sliver microstructure” escape hatch}
In the constant-mass-per-sheet toy model on a cubulation of mesh $h$, the expected sheet count per cube scales like
$N_Q\sim m h^{2p}$ while the template-based gluing bound scales like $\mathcal F(\partial T^{\mathrm{raw}})\lesssim m h$.
For $p>1$ and fixed $m$, taking $h\to 0$ drives the gluing error down but forces $N_Q\to 0$, leaving too few degrees of freedom per face for transport averaging.
This suggests that any successful fixed-$m$ realization must use \emph{many} sheet pieces per cube whose individual masses are correspondingly smaller
(``sliver pieces'').

\begin{conjecture}[Sliver microstructure local model]\label{conj:sliver-micro}
Fix a small coordinate cell $Q$ of size $h$ and a calibrated direction (complex $(n-p)$-plane) with normal translation space $N^\perp\cong\R^{2p}$.
Given a nonnegative density $\rho$ on $N^\perp$ with total mass $\int\rho = O(mh^{2p})$ and any $N\in\N$, there exist
$N$ calibrated sheet pieces in $Q$ (restrictions of calibrated sheets) whose induced transverse measure $\mu_N$ on $N^\perp$ satisfies:
\begin{enumerate}
\item[\textnormal{(i)}] (\textbf{Fixed total mass}) $\mu_N(N^\perp)=\int \rho$;
\item[\textnormal{(ii)}] (\textbf{Arbitrary fineness}) $W_1(\mu_N,\rho\,dy)\to 0$ as $N\to\infty$ (quantitative rate);
\item[\textnormal{(iii)}] (\textbf{Tiny pieces}) each sheet piece has mass $\lesssim (\int\rho)/N$ in $Q$ (so the number of pieces can grow without mass blow-up).
\end{enumerate}
Prove (or refute) that such a construction is possible in the \emph{K\"ahler/projective} setting with calibrated holomorphic sheets.
\end{conjecture}

\paragraph{Comment.}
In the flat affine model, the intersection mass of a translated plane with a cell varies continuously down to $0$, so (iii) is plausible.
The hard part is combining (ii)+(iii) with holomorphic/algebraic realizability and with compatibility across faces in a global gluing scheme.

\section{Prompt 9 (Signed decomposition positivity lemma)}

\subsection*{Standalone prompt (copy/paste)}
\begin{quote}\small
Let $\gamma_{\mathrm{harm}}$ be the harmonic representative of a rational $(p,p)$ class $\gamma$ on a compact K\"ahler manifold $(X,\omega)$.
Prove: there exists an explicit $N$ (depending on $\|\gamma_{\mathrm{harm}}\|_{C^0}$ and $\omega$) such that the smooth closed form
\[
\beta:=\gamma_{\mathrm{harm}} + N\, \omega^p
\]
is strongly positive pointwise. State the sharp cone notion (weak/strong positivity) and prove the uniform-in-$x$ choice of $N$.
\end{quote}

\subsection*{Status}
\textbf{Provable (pointwise domination on decomposables).}

\subsection*{Proof (uniform $N$ via interior of the strong cone)}
Fix $x$ and write $V_x:=\Lambda^{p,p}T_x^*X$ (real vector space) with any norm $\|\cdot\|_x$ induced by the K\"ahler metric.
Let $K_x\subset V_x$ be the strongly positive cone (Prompt~1), and let $e_x:=\omega_x^p\in K_x$.

\begin{lemma}\label{lem:prompt9-interior}
For each $x$, $K_x$ is full-dimensional in $V_x$ (hence has nonempty interior), and $e_x=\omega_x^p$ lies in $\mathrm{int}(K_x)$.
\end{lemma}
\begin{proof}
Full-dimensionality follows because the generators $\phi_v=i^{p^2}v\wedge\bar v$ with $v$ decomposable span $V_x$ linearly.
Hence $K_x=\mathrm{cone}(\{\phi_v\})$ has nonempty interior.

The cone $K_x$ and the ray $\mathbb R_{>0}\omega_x^p$ are invariant under the unitary group $U(T_x^{1,0}X)$.
Choose any interior point $\eta\in \mathrm{int}(K_x)$ and average its unitary orbit:
\[
\bar\eta:=\int_{U(T_x^{1,0}X)} U\cdot \eta\ dU.
\]
Each $U\cdot\eta$ lies in $\mathrm{int}(K_x)$, and $\mathrm{int}(K_x)$ is convex, so $\bar\eta\in \mathrm{int}(K_x)$.
By unitary invariance, $\bar\eta$ is $U(n)$-invariant, hence spans the unique invariant ray in $\Lambda^{p,p}$, i.e.\ $\bar\eta=c\,\omega_x^p$ for some $c>0$.
Thus $\omega_x^p\in \mathrm{int}(K_x)$.
\end{proof}

By Lemma~\ref{lem:prompt9-interior}, the distance
\[
r_x:=\mathrm{dist}\bigl(e_x,\ V_x\setminus K_x\bigr)>0.
\]
Because all $(V_x,K_x,e_x)$ are canonically isometric (they are the same $U(n)$-model at each point), $r_x$ is a \emph{constant} $r=r(n,p,\omega)>0$ independent of $x$.
Hence the ball $B_x(e_x,r)\subset K_x$ for all $x$.

\begin{proposition}\label{prop:prompt9-N}
Let $(X,\omega)$ be compact K\"ahler and let $\gamma_{\mathrm{harm}}$ be a smooth real $(p,p)$-form.
Let $\|\gamma_{\mathrm{harm}}\|_{C^0}:=\sup_x \|\gamma_{\mathrm{harm},x}\|_x$.
Choose
\[
N\ \ge\ \frac{\|\gamma_{\mathrm{harm}}\|_{C^0}}{r}.
\]
Then $\beta:=\gamma_{\mathrm{harm}}+N\,\omega^p$ is strongly positive at every point.
\end{proposition}
\begin{proof}
Fix $x$. Write
\[
\beta_x = N\Bigl(\omega_x^p + \frac{1}{N}\gamma_{\mathrm{harm},x}\Bigr).
\]
By the choice of $N$, one has $\big\|\tfrac{1}{N}\gamma_{\mathrm{harm},x}\big\|_x\le r$, hence
$\omega_x^p+\tfrac{1}{N}\gamma_{\mathrm{harm},x}\in B_x(e_x,r)\subset K_x$.
Since $K_x$ is a cone, multiplying by $N$ gives $\beta_x\in K_x$.
\end{proof}

\section{Prompt 10 (Reality check / falsification)}

\subsection*{Standalone prompt (copy/paste)}
\begin{quote}\small
Find (or prove none exists): a smooth complex projective $X$ and a rational Hodge class $\alpha\in H^{2p}(X,\mathbb Q)\cap H^{p,p}(X)$ that admits a smooth closed strongly positive representative, but such that no positive rational combination of codimension-$p$ algebraic cycles represents $\alpha$ (equivalently $\alpha$ is not in the algebraic effective cone). If you believe it’s impossible, explain which known conjecture/theorem it would imply.
\end{quote}

\subsection*{Status}
\textbf{Open / would have major consequences.} Existence of such an $\alpha$ would refute the Hodge conjecture (in the relevant degree) under standard identifications; nonexistence amounts to a strong positivity-implies-algebraicity principle.

\subsection*{Logical implication (what it would prove/contradict)}
\begin{itemize}
\item If one could prove ``every rational Hodge class with a smooth strongly positive representative is algebraic'', then in particular one would prove Prompt~6 for all $p$.
\item Conversely, producing a counterexample $\alpha$ as stated in Prompt~10 (strongly positive smooth representative but not in the algebraic effective cone) would show that ``strong positivity in cohomology'' is strictly weaker than ``algebraic effectivity'', and it would obstruct any realization strategy based only on cone-valued smooth representatives.
\end{itemize}

\section{Recognition/CPM bridge: what would actually close Prompts 6/8/10}

\begin{remark}
The CPM/recognition documents emphasize \emph{finite resolution} and \emph{finite structured modes}.
In the Hodge setting, the gap is exactly: ``a diffuse strongly positive class should be realized by a \emph{discrete} calibrated cycle.''
Below are two crisp intermediate statements which, if proved (by RS/Recognition input or otherwise), would close the remaining prompts.
\end{remark}

\subsection*{RS-PROMPT A (finite-mode / polyhedrality hypothesis)}
\begin{quote}\small
Let $X$ be smooth complex projective and fix $p$. Let
$\mathcal C_{\mathrm{strong}}^p(X)\subset H^{2p}(X,\mathbb R)\cap H^{p,p}(X)$
be the cone of classes admitting a smooth \emph{strongly positive} representative.
Assume a ``finite resolution / finite mode'' principle: $\mathcal C_{\mathrm{strong}}^p(X)$ is a \emph{rational polyhedral cone}, generated by finitely many extreme rays
$\mathbb R_{\ge 0}\,[V_i]^\vee$ where each $V_i$ is an algebraic codimension-$p$ cycle.
Prove that every rational class $\gamma^+\in \mathcal C_{\mathrm{strong}}^p(X)\cap H^{2p}(X,\mathbb Q)$ is algebraic (i.e.\ a rational combination of the $[V_i]^\vee$).
\end{quote}

\paragraph{Consequence.}
Under this hypothesis, Prompt~6 holds immediately by clearing denominators.

\paragraph{Proof (cone rationality).}
If $\mathcal C_{\mathrm{strong}}^p(X)=\mathrm{cone}\{[V_1]^\vee,\dots,[V_N]^\vee\}$ is rational polyhedral, then any $\gamma^+\in \mathcal C_{\mathrm{strong}}^p(X)\cap H^{2p}(X,\mathbb Q)$ can be written
$\gamma^+=\sum_i q_i [V_i]^\vee$ with $q_i\in\mathbb Q_{\ge0}$ (standard linear programming / rational polyhedral cones).
Choose $m$ clearing denominators so $mq_i\in\mathbb Z_{\ge0}$, and set
$T:=\sum_i (mq_i)[V_i]$.
Then $[T]=\mathrm{PD}(m[\gamma^+])$ and $T$ is $\psi$-calibrated, hence satisfies
$\mathrm{Mass}(T)=\langle [T],[\psi]\rangle$ (Wirtinger equality).
This is exactly the conclusion of Prompt~6.

\subsection*{RS-PROMPT B (finite-resolution realization of barycentric plane measures)}
\begin{quote}\small
Let $(X,\omega)$ be K\"ahler projective and fix $p$. Let $\beta$ be a smooth closed strongly positive $(p,p)$-form representing a rational Hodge class $\gamma^+$.
Assume a \emph{finite-resolution} hypothesis: there exists a triangulation (or cubulation) of $X$ and, on each top-dimensional cell $Q$, a representation
\[
\beta|_Q = \sum_{i=1}^{N} \lambda_i\,\phi_{V_i}
\]
with \emph{constant} coefficients $\lambda_i\in\mathbb Q_{\ge0}$ and \emph{constant} complex $p$-planes $V_i$ in local holomorphic coordinates on $Q$ (i.e.\ a piecewise-constant calibrated mode decomposition).
Prove that, after multiplying by a common denominator $m$, one can construct an integral $\psi$-calibrated cycle $T$ with $[T]=\mathrm{PD}(m[\gamma^+])$ and $\mathrm{Mass}(T)=\langle[T],[\psi]\rangle$.
\end{quote}

\paragraph{Comment.}
This isolates the true hard part of Prompt~8: the \emph{gluing/patching} of local calibrated sheets across cell boundaries while keeping the global cycle closed and calibrated.
If finite resolution forces a piecewise-constant mode decomposition (as RG4/RG5 style principles suggest), then the remaining question is a calibrated-gluing theorem.

\end{document}


