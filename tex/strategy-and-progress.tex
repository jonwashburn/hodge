\documentclass[11pt]{article}

\usepackage[margin=1in]{geometry}

\usepackage{iftex}
\ifPDFTeX
  \usepackage[utf8]{inputenc}
  \usepackage[T1]{fontenc}
  \usepackage{lmodern}
\else
  \usepackage{fontspec}
  \defaultfontfeatures{Ligatures=TeX}
\fi

\usepackage{amsmath,amssymb,mathtools}
\usepackage{microtype}
\usepackage{hyperref}

\setlength{\emergencystretch}{2em}

\title{Strategy and Progress (Unconditional Hodge Closure)}
\date{Dec 2025}
\author{}

\begin{document}
\maketitle

This file is a living lab notebook for pushing the manuscript toward a \textbf{valid, unconditional} proof.

\section{Current status (Dec 2025)}

\subsection{The core dependency}
\begin{itemize}
\item The manuscript's claimed ``unconditional Hodge'' conclusion flows through
  \textbf{\texttt{thm:automatic-syr}} (``Automatic SYR for cone-valued forms'').
\item \textbf{Unconditional closure requires a fully rigorous proof of the realization/microstructure step}
  that produces a \textbf{\(\psi\)-calibrated integral cycle} in the target class (after clearing denominators).
\end{itemize}

\subsection{What is standard (once the key existence is granted)}
\begin{itemize}
\item \textbf{Federer--Fleming compactness + calibration inequality} (already packaged as
  \texttt{thm:realization-from-almost}).
\item \textbf{Harvey--Lawson} (calibrated integral current \(\Rightarrow\) analytic cycle) and \textbf{Chow}
  (projective analytic \(\Rightarrow\) algebraic).
\item Signed decomposition reduction is algebraic bookkeeping (not the core obstacle).
\end{itemize}

\subsection{What is not yet proven (the ``life-or-death'' blockers)}
\begin{enumerate}
\item \textbf{Prompt 6 / Prompt 8 (core missing theorem):}
  \begin{itemize}
  \item From a smooth closed strongly positive \((p,p)\) representative \(\beta\) of a rational class,
    prove existence of a multiple \(m[\gamma^+]\) represented by a \textbf{\(\psi\)-calibrated integral cycle}
    (equivalently stable mass equals the calibration pairing).
  \end{itemize}
\item \textbf{Microstructure / gluing control in Step 4.2:}
  \begin{itemize}
  \item The manuscript explicitly flags the missing piece: a quantitative estimate making
    \(\mathcal F(\partial T^{\mathrm{raw}})\) small, robustly under cancellation.
  \end{itemize}
\item \textbf{A scheme that avoids the ``diffuse density vs codimension-\(p\) mass'' trap,} or proves a new
  rigidity principle that collapses smooth strong positivity into algebraicity.
\end{enumerate}

\section{Work completed so far (high-signal)}

\subsection{1) Flat-norm gluing route made explicit in the manuscript}
\begin{itemize}
\item In \texttt{hodge-dec-6-handoff.tex} Substep 4.2:
  \begin{itemize}
  \item Added \texttt{prop:transport-flat-glue}: \textbf{small-angle sheets + transverse \(W_1\) matching across
    faces \(\Rightarrow\) \(\mathcal F(\partial T^{\mathrm{raw}})\) small} (quantitative).
  \item Added \texttt{prop:integer-transport} + \texttt{rem:integer-transport-algo}: reduced ``produce \(W_1\) matching''
    to \textbf{grid quantization + integer rounding + integer network flow} (discrete combinatorics), isolating
    the remaining \emph{geometric} difficulty.
  \end{itemize}
\end{itemize}

\subsection{2) Prompt docs upgraded}
\begin{itemize}
\item In \texttt{ai-gap-prompts-and-proofs.tex}:
  \begin{itemize}
  \item Added a flat-norm criterion (\texttt{lem:prompt8-flat-glue}) and transport control (\texttt{lem:prompt8-w1})
    + \(W_1\) quantization (\texttt{lem:w1-quantize}).
  \item Added a closedness-cancellation remark explaining why the flat-norm dual is the right target.
  \end{itemize}
\end{itemize}

\subsection{3) Obstruction knowledge improved}
\begin{itemize}
\item We explicitly identified the need to \textbf{separate}:
  \begin{itemize}
  \item what is purely discrete/rounding/transport (solvable), vs
  \item what is geometric: realizing the discrete transverse measures by actual calibrated holomorphic sheets
    with controlled angles and controlled face-slice geometry.
  \end{itemize}
\end{itemize}

\section{The plan (what we try next, inch by inch)}

\subsection{A) Make the transport hypotheses provable (not assumed)}

Target: prove a lemma of the form:

\begin{quote}
For the specific holomorphic sheet families produced by \texttt{thm:local-sheets}, the restriction of \(\partial S_Q\)
 to a face admits a transverse-parameter measure model (with uniform tubular charts), and the induced measures
 on adjacent cubes can be chosen/adjusted to satisfy the required \(W_1\) face matching.
\end{quote}

Concrete subtasks:
\begin{itemize}
\item Prove uniform tubular-coordinate control for each sheet family on a face (geometry estimate).
\item Define explicitly the transverse measures \(\mu_{Q\to F}\) in the K\"ahler chart.
\item Prove the Lipschitz slice-functional estimate needed for Kantorovich--Rubinstein.
\end{itemize}

\subsection{B) Resolve the scaling tension in the current cube-local paradigm}

We must reconcile three competing needs in any cube-by-cube sampling approach:
\begin{itemize}
\item \textbf{integer availability} (need many sheets per cube to match weights),
\item \textbf{small variation} (need mesh small so the target data changes slowly),
\item \textbf{small boundary mismatch} (need face mismatches small in flat norm).
\end{itemize}

If this cannot be reconciled, we must switch paradigms (see C).

\subsubsection{Scaling sanity check (important)}

Let \(h\) be the cube side length and \(m\) the denominator-clearing multiplier.
The ``target cube mass'' is \(M_Q \sim m\,h^{2n}\), while a single calibrated sheet crossing \(Q\) has mass
\(A_Q\sim h^{2(n-p)}\).
Thus the expected number of sheets per cube is
\[
N_Q \sim \frac{M_Q}{A_Q} \sim m\,h^{2p}.
\]
\begin{itemize}
\item To have \textbf{enough integer degrees of freedom} locally (and make rounding errors small), we want
  \(N_Q\gg 1\), i.e. \(m\,h^{2p}\gg 1\).
\item To make \(\beta\) nearly constant on cubes, we want \(h\to 0\).
\item In naive estimates, \textbf{total face mismatch} scales like ``(variation across a face) \(\times\) (total boundary mass)'',
  which heuristically leads to a global mismatch of order \(\sim m\,h\) (variation \(\sim h\), total flux \(\sim m\)),
  so driving mismatch to \(0\) at fixed \(m\) suggests \(h\ll 1/m\).
\end{itemize}

For \(p>1\), the requirements \(m\,h^{2p}\gg 1\) and \(h\ll 1/m\) are incompatible at large \(m\):
\[
h\ll m^{-1} \quad\Rightarrow\quad m\,h^{2p}\ll m\cdot m^{-2p}=m^{1-2p}\to 0.
\]
This indicates that any successful construction must use \textbf{cancellation/closedness in an essential way}
(flat-norm route), not naive mass-of-boundary estimates.

\subsection{C) Search for a replacement mechanism (avoid cube-local diffuse matching)}

Possible directions to explore (each must end in a classical proof, not just RS motivation):
\begin{itemize}
\item \textbf{Rigidity/Bochner-type}: show a rational class admitting a smooth strongly positive representative must be algebraic
  (or severely restricted).
\item \textbf{Global probabilistic/algebraic averaging}: represent smooth positive forms as averages of algebraic cycles and attempt
  a rational/integral extraction argument.
\item \textbf{Stationarity constraints on Young measures}: find a PDE-type necessary condition that forces realizable barycenters
  to come from genuine analytic strata.
\end{itemize}

\subsection{D) RS/Recognition-guided hypotheses (kept separate from classical proof)}

We will record RS-inspired ``finite resolution'' hypotheses (polyhedrality / finite-mode decompositions) as optional bridges,
but every step needed for unconditional closure must be proven classically or reduced to known theorems.

\section{Next concrete actions}
\begin{enumerate}
\item Strengthen \texttt{prop:transport-flat-glue} into a fully cited lemma package: tubular neighborhood existence + slice Lipschitz estimate
  + KR duality.
\item Audit \texttt{thm:local-sheets} to see whether it can actually realize prescribed transverse placements across faces (not just inside one cube).
\item Decide whether Step 4's cube-local scheme is fundamentally incompatible with fixed-class realization for \(1<p<n-1\), and if so,
  pivot to a global replacement mechanism.
\end{enumerate}

\subsection{Latest incremental progress (today)}
\begin{itemize}
\item Added \texttt{rem:transport-hypotheses} in \texttt{hodge-dec-6-handoff.tex} clarifying that hypotheses (a)--(b) of
  \texttt{prop:transport-flat-glue} really do hold for the flat-sheet model and persist (up to \(O(\varepsilon)\)) after the holomorphic upgrade.
\item This isolates the remaining unknown in the transport route to a single hard requirement:
  \textbf{construct translation parameters so that adjacent cubes satisfy face-by-face \(W_1\) matching (and do so consistently across all faces of each cube).}

\item Added a new subsection in \texttt{hodge-blocker.tex} formalizing the ``cube-consistency'' issue as a finite-dimensional
  \textbf{marginal realization / coupling problem} for discrete translation parameters, and proved a basic existence lemma
  (\texttt{lem:discrete-marginals}) for the coordinate-projection case (via flow/matching + induction). This is not the full geometric problem,
  but it cleanly separates ``combinatorics is not the blocker'' from ``geometry/plane-slice maps are.''

\item Added \texttt{lem:w1-linear-stability} in \texttt{hodge-blocker.tex}: a clean estimate
  \(W_1(L_\#\mu,L'_\#\mu)\le \|L-L'\|\,\int \|y\|\,d\mu\), useful for bounding face-measure mismatches when adjacent cubes use the
  \emph{same} underlying transverse translation measure but have slightly different face-slice maps (due to small angle changes).

\item Added \texttt{cor:angle-to-face} in \texttt{hodge-blocker.tex}: combines transport control + linear stability to give an explicit
  ``small angle change \(\Rightarrow\) small flat-norm face mismatch'' scaling bound in the flat model.

\item Added \texttt{lem:slice-angle-lip} in \texttt{hodge-blocker.tex}: a clean (sketched) estimate that the face-slice functional
  \(\Sigma_F^{P}(t)(\eta)\) varies Lipschitzly with the plane angle, at scale \(\alpha\,h^{k-1}\).

\item Added a ``cube-consistency'' formulation to Prompt 8 in \texttt{ai-gap-prompts-and-proofs.tex} (\texttt{conj:cube-consistent}):
  the remaining transport route can be seen as a \textbf{simultaneous pushforward approximation problem} (one discrete measure on translation space
  must approximate many face measures at once).

\item Updated \texttt{hodge-dec-6-handoff.tex} \texttt{rem:transport-hypotheses} to explicitly state this ``same translation multiset pushes
  forward to all faces'' viewpoint, so the reader sees that the remaining obstruction is truly a \emph{simultaneous} matching constraint,
  not independent per-face tuning.

\item Added \texttt{lem:w1-auto} + \texttt{rem:w1-auto} in \texttt{hodge-dec-6-handoff.tex}: if adjacent cubes use the \textbf{same translation template}
  and their face-slice maps differ by \(O(h)\) (smooth geometry), then the induced transverse measures are automatically \(W_1\)-close at scale
  \(O(h^2\cdot N)\). This is a key simplification: it suggests we may not need to \emph{solve} the full cube-consistent matching problem,
  only keep the translation template coherent and handle cohomology constraints globally.

\item \textbf{Correction:} the resulting global flat-norm bound from this mechanism scales like
  \(\mathcal F(\partial T^{\mathrm{raw}})\lesssim m\,h+O(\varepsilon m)\), not \(m h^2\). (Counting faces in a \(2n\)-dimensional cubulation gives
  \(O(h^{-2n})\) faces.)

\item Added \texttt{lem:w1-template-edit} + \texttt{rem:w1-multiplicity} in \texttt{hodge-dec-6-handoff.tex}: if adjacent cubes use the same template but
  integer sheet counts vary slowly, then the extra \(W_1\) error from insertions/deletions is \(O(rh)\) and is absorbed into the same \(h^2N\) scaling
  provided \(r\lesssim hN\). This reduces the remaining matching analysis to a \textbf{rounding/Diophantine ``slow variation of counts''} lemma.

\item Added \texttt{lem:slow-variation-rounding} in \texttt{hodge-dec-6-handoff.tex}: a concrete bound showing that rounding a Lipschitz target
  \(n_Q=m h^{2p} f(x_Q)\) automatically yields neighbor differences
  \(|N_Q-N_{Q'}|\le L m h^{2p+1}+1\), and hence \(|N_Q-N_{Q'}|\lesssim hN_Q\) once \(f\) is bounded below and \(m h^{2p+1}\) is large.

\item Added \texttt{lem:barany-grinberg} in \texttt{hodge-dec-6-handoff.tex} and used it to justify the Substep 4.3 claim that one can meet all
  cohomology pairing constraints to within \(<1/2\) simultaneously by rounding in fixed dimension (discrepancy bound depends only on
  \(b=\dim H^{2n-2p}\)).

\item Rewrote the proof of \texttt{prop:cohomology-match} in \texttt{hodge-dec-6-handoff.tex} to explicitly implement the
  B\'ar\'any--Grinberg rounding (replace the old ``LLL/continued fractions'' handwave). The proof now:
  (i) encodes rounding as \(N=\lfloor n\rfloor+\varepsilon\), (ii) bounds each sheet-piece pairing vector by \(O(h^{2(n-p)})\),
  (iii) applies discrepancy in dimension \(b\) to get \(<1/2\) simultaneous error.

\item Added \texttt{rem:param-tension} in \texttt{hodge-dec-6-handoff.tex}: makes explicit the fixed-\(m\) tension between needing \(h\to 0\) for small
  gluing error (template route gives \(\mathcal F(\partial T^{\mathrm{raw}})\lesssim m h\)) and the fact that the naive constant-mass sheet model has
  expected local counts \(N_Q\sim m h^{2p}\to 0\) as \(h\to 0\) for \(p>1\).

\item Added \texttt{lem:mass-tunable} + \texttt{rem:sliver} in \texttt{hodge-dec-6-handoff.tex}: in the flat model the mass of a translated plane slice
  through a cell is continuous down to \(0\), suggesting a fixed-\(m\) escape route via \textbf{many tiny ``sliver'' sheet pieces}
  (large \(N\) but small per-piece mass).

\item Added \texttt{conj:sliver-local} + \texttt{rem:sliver-close} in \texttt{hodge-dec-6-handoff.tex}: a precise quantitative local target stating what
  ``sliver microstructure'' would need in the \emph{holomorphic upgrade} (many calibrated pieces with tiny mass, whose weighted transverse measure
  approximates a smooth density in \(W_1\)).

\item Added \texttt{lem:sliver-stability} in \texttt{hodge-dec-6-handoff.tex}: shows that once a holomorphic sheet is a small-\(C^1\) graph over an affine
  calibrated plane slice, (i) its mass in the cell is comparable up to a \(1+O(\varepsilon^2)\) factor, and (ii) disjointness inside the cell
  persists under separation \(\gg \varepsilon h\). This reduces ``sliver masses'' in \texttt{conj:sliver-local} to the flat-model tunability plus
  sufficiently strong \(C^1\) approximation.

\item Added \texttt{rem:jet-separation} in \texttt{hodge-dec-6-handoff.tex}: notes that high powers \(L^k\) can separate jets at finitely many points,
  supporting the feasibility (at least at the ``local existence of many pieces'' level) of realizing many prescribed small-mass local plane pieces
  in the projective holomorphic upgrade.

\item Added \texttt{lem:plane-section-continuity} in \texttt{hodge-blocker.tex}: in the flat model, the intersection mass of a translated plane with a cell
  varies continuously from \(0\) to a maximum. This highlights a possible ``sliver microstructure'' mechanism: split fixed total mass into many
  tiny pieces to keep sheet counts large even at small \(h\).

\item Added \texttt{lem:sliver-ball-scaling} in \texttt{hodge-blocker.tex}: an explicit formula for the mass of a plane section of a ball, giving a clean
  toy scaling for how ``sliver'' masses decay as the plane approaches the boundary.

\item Added \texttt{conj:sliver-micro} in \texttt{ai-gap-prompts-and-proofs.tex}: a sharpened formulation of what fixed-\(m\) realization would need in a
  local chart---many tiny calibrated ``sliver pieces'' per cell whose transverse measures approximate a smooth density in \(W_1\) without mass blow-up.
\end{itemize}

\end{document}
