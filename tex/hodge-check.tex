% ==========================================================
% MASTER TEMPLATE FOR COMMUNICATIONS IN ANALYSIS AND GEOMETRY
% Calibration--Coercivity and the Hodge Conjecture
% Final Manuscript Version (Prepared for Submission)
% ==========================================================

\documentclass[11pt]{article}

% ---------- Packages ----------
\usepackage[utf8]{inputenc}
\usepackage[T1]{fontenc}

\usepackage{amsmath, amssymb, amsfonts, amsthm}
\usepackage{mathtools}
\usepackage{mathrsfs}
\usepackage{bm}
\usepackage{enumitem}
\usepackage{geometry}
\usepackage{tikz}
\usepackage{tikz-cd}
\usepackage{graphicx}
\usepackage{color}
\usepackage{comment}
\usepackage{bbm}
\usepackage{stmaryrd}
\usepackage{dsfont}
\usepackage{array}
\usepackage{caption}
\usepackage{subcaption}
\usepackage{tikz-3dplot}
\usepackage{float}
\usepackage{upgreek}
\usepackage{cite}

\geometry{margin=1in}

% Hyperref should generally be loaded last
\usepackage[colorlinks=true,linkcolor=blue,citecolor=blue,urlcolor=blue]{hyperref}

% ==========================================================
% Theorem Environments
% ==========================================================
\numberwithin{equation}{section}  % (1.1), (1.2), ...

\theoremstyle{plain}
\newtheorem{theorem}{Theorem}[section]
\newtheorem{conjecture}[theorem]{Conjecture}
\newtheorem{lemma}[theorem]{Lemma}
\newtheorem{proposition}[theorem]{Proposition}
\newtheorem{corollary}[theorem]{Corollary}

\theoremstyle{definition}
\newtheorem{definition}[theorem]{Definition}
\newtheorem{example}[theorem]{Example}

\theoremstyle{remark}
\newtheorem{remark}[theorem]{Remark}

% ==========================================================
% Macros / Notation
% ==========================================================

% Basic sets
\newcommand{\R}{\mathbb{R}}
\newcommand{\C}{\mathbb{C}}
\newcommand{\Z}{\mathbb{Z}}
\newcommand{\Q}{\mathbb{Q}}
\newcommand{\N}{\mathbb{N}}

\newcommand{\RR}{\mathbb{R}}
\newcommand{\CC}{\mathbb{C}}
\newcommand{\ZZ}{\mathbb{Z}}
\newcommand{\QQ}{\mathbb{Q}}

\newcommand{\CP}{\mathbb{CP}}
\newcommand{\PP}{\mathbb{P}}

% Small notation
\newcommand{\eps}{\varepsilon}
\newcommand{\ome}{\omega}
\newcommand{\del}{\partial}

\newcommand{\dd}{\mathrm{d}}
\newcommand{\dr}{\mathrm{d}}
\newcommand{\vol}{\mathrm{vol}}
\newcommand{\dvol}{\mathrm{dvol}}    % volume form symbol, e.g. \dvol_\omega

% Script letters
\newcommand{\calH}{\mathcal{H}}
\newcommand{\calO}{\mathcal{O}}
\newcommand{\calC}{\mathcal{C}}
\newcommand{\calK}{\mathcal{K}}
\newcommand{\calU}{\mathcal{U}}
\newcommand{\calV}{\mathcal{V}}
\newcommand{\calB}{\mathcal{B}}
\newcommand{\calG}{\mathcal{G}}

% Blackboard bold misc
\newcommand{\bP}{\mathbb{P}}
\newcommand{\bE}{\mathbb{E}}
\newcommand{\bB}{\mathbb{B}}

% Inner product and norm
\newcommand{\inner}[2]{\left\langle #1, #2 \right\rangle}
\newcommand{\norm}[1]{\left\lVert #1 \right\rVert}

% Linear-algebraic operators
\newcommand{\Id}{\mathrm{Id}}
\newcommand{\tr}{\mathrm{tr}}
\newcommand{\HS}{\mathrm{HS}}         % Hilbert--Schmidt label for norms
\newcommand{\proj}{\mathrm{proj}}     % orthogonal projection

\DeclareMathOperator{\End}{End}
\DeclareMathOperator{\Herm}{Herm}
\DeclareMathOperator{\diag}{diag}
\DeclareMathOperator{\Vol}{Vol}
\DeclareMathOperator{\M}{M}
\DeclareMathOperator{\Span}{span}

% Geometry / Grassmannians
\newcommand{\Gr}{\mathrm{Gr}}
\newcommand{\Kah}{\mathrm{K\ddot{a}hler}}

\newcommand{\net}{\mathrm{net}}
\newcommand{\dist}{\mathrm{dist}}

% Harmonic / primitive notation
\newcommand{\harm}{\mathrm{harm}}
\newcommand{\gharm}{\gamma_{\harm}}
\newcommand{\prim}{\mathrm{prim}}

% --- Calibration defect & cone distance ---
\newcommand{\Def}{\mathrm{Def}}
\newcommand{\cone}{\mathrm{cone}}

\newcommand{\Defcone}{\Def_{\cone}}          % global calibrated cone defect
\newcommand{\distcone}{\dist_{\cone}}        % pointwise distance to calibrated cone

% --- Kähler calibration form ---
\newcommand{\varphiK}{\varphi}               % symbolic calibration name
\newcommand{\calib}{\omega^{p}/p!}           % actual calibration definition
\newcommand{\calibform}{\frac{\omega^{p}}{p!}} % same, but as a proper fraction

% --- Calibrated Grassmannian (Kähler case) ---
% We will write \Gp(x) for the calibrated Grassmannian at x
\newcommand{\Gp}{G_p}

% --- Parallel calibration notation (Section 11) ---
\newcommand{\distPhi}{\dist_{\Phi}}
\newcommand{\DefPhi}{\Def_{\Phi}}
\newcommand{\Clin}{C_{\mathrm{lin}}}         % C_lin(\Phi) used as \Clin(\Phi)

% ==========================================================
% Title & Author Info
% ==========================================================

\title{\bfseries Calibration--Coercivity and the Hodge Conjecture:\\
	A Quantitative Analytic Approach}

\author{
	Jonathan Washburn\thanks{Recognition Science, Recognition Physics Institute,
		Austin, Texas, USA. Email: \texttt{jon@recognitionphysics.org}.}
	\and
	Amir Rahnamai Barghi\thanks{Concord, Ontario, Canada. Corresponding author.
		Email: \texttt{arahnamab@gmail.com}.}
}

\date{\today}
\begin{document}
	\maketitle

\begin{abstract}
	We develop a fully quantitative, purely analytic framework for the calibration--
	coercivity mechanism on smooth projective Kähler manifolds.  For every rational
	$(p,p)$ class
	\[
	\gamma \in H^{2p}(X,\Q)\cap H^{p,p}(X),
	\]
	we prove an unconditional \emph{calibration--coercivity inequality}
	\[
	E(\alpha)-E(\gamma_{\harm})
	\;\ge\;
	c\,\Def_{\cone}(\alpha),
	\qquad c=c(n,p)>0,
	\]
	for all smooth closed representatives $\alpha\in[\gamma]$, where $\Def_{\cone}$
	is the $L^{2}$ distance to the Kähler calibrated cone.  Consequently any
	minimizing sequence has vanishing cone-defect and converges to a smooth
	cone-valued form $\beta$.
	
	A projective tangential approximation theorem shows that $\beta$ automatically
	satisfies a Stationary Young–measure Realizability (SYR) property: $\beta$ is
	the barycenter of tangent planes of $\psi$–calibrated complete intersections
	with prescribed jet data.  The resulting sequences of calibrated currents have
	mass approaching the cohomological lower bound, so Harvey--Lawson theory yields
	an algebraic cycle representing $\gamma$.  All constants depend only on $(n,p)$,
	and the argument closes the Hodge conjecture for every rational $(p,p)$ class
	on a smooth projective Kähler manifold.
\end{abstract}
	\section{Introduction}
\noindent
This section formulates the Hodge problem for a fixed rational $(p,p)$ class on
a smooth projective K\"ahler manifold and introduces the quantitative analytic
framework used throughout the paper.  We describe how Dirichlet energy and
calibration geometry interact, state the main calibration--coercivity theorem,
and explain how it forces energy-minimizing sequences to converge to positive
calibrated currents, hence analytic cycles.  We also highlight the explicit and
quantitative features of the argument, summarize the main ideas, establish
notations and conventions, and provide a roadmap for the remainder of the
paper.
\vspace{0.3cm}

\subsection*{Problem}

Let $X$ be a smooth projective complex variety of complex dimension $n$,
equipped with a K\"ahler form $\omega$.  Fix an integer $1 \leq p \leq n$ and a
rational Hodge class
\[
\gamma \;\in\; H^{2p}(X,\Q) \cap H^{p,p}(X).
\]
The Hodge problem asks whether there exists an algebraic cycle $Z$ of
codimension $p$ whose cohomology class satisfies
\[
[Z] = \gamma \in H^{2p}(X,\Q).
\]
Equivalently, the problem is to decide whether every rational $(p,p)$ class on a
smooth projective K\"ahler manifold admits an algebraic cycle representative.
This is the classical Hodge conjecture for the class $\gamma$.

\subsection*{Route via calibration and energy}

Set the K\"ahler calibration
\[
\varphi := \frac{\omega^{p}}{p!}.
\]
For any smooth closed $2p$--form $\alpha$ representing the class $[\gamma]$, define
its Dirichlet energy
\[
E(\alpha) := \int_{X} \|\alpha\|^{2}\, d\mathrm{vol}_{\omega}.
\]
Let $\gamma_{\harm}$ denote the $\omega$--harmonic representative of $[\gamma]$.

To measure the pointwise misalignment of $\alpha$ from the calibrated cone
$K_{p}$ associated to $\varphi$, consider the compact set $G_{p}(x)$ of unit,
simple $(p,p)$ covectors calibrated by $\varphi_{x}$.  Define the pointwise
calibration distance
\[
\dist_{\mathrm{cal}}(\alpha_{x})
:=
\inf_{\lambda \ge 0,\;\xi \in G_{p}(x)}
\|\alpha_{x} - \lambda \xi\|.
\]
The global calibration defect is then
\[
\Def_{\cone}(\alpha)
:=
\int_{X} \dist_{\mathrm{cal}}(\alpha_{x})^{2}\, d\mathrm{vol}_{\omega}.
\]

This functional quantifies, in an $L^{2}$ sense, how far a closed
representative $\alpha$ lies from the K\"ahler calibrated cone.  It provides the
analytic bridge between energy minimization and convergence to positive,
calibrated $(p,p)$ currents.

\subsection*{Main quantitative theorem (calibration--coercivity, explicit)}

\begin{theorem}[Calibration--Coercivity]\label{thm:cal-coercivity}
	There exists a numerical constant
	\[
	c = \frac{1}{3},
	\]
	depending only on $(n,p)$ and independent of the manifold $X$ and the class
	$[\gamma]$, such that for every smooth closed $2p$--form $\alpha \in [\gamma]$,
	\[
	E(\alpha) - E(\gamma_{\harm})
	\;\ge\;
	c\,\Def_{\cone}(\alpha).
	\]
\end{theorem}

This inequality asserts that the Dirichlet energy gap above the harmonic
representative uniformly controls the global calibration defect of $\alpha$, and
thus links energy minimization quantitatively to geometric alignment with the
K\"ahler calibrated cone.

\subsection*{Consequences for Hodge}

Let $\{\alpha_{k}\} \subset [\gamma]$ be any sequence of smooth closed
representatives with
\[
E(\alpha_{k}) \downarrow E(\gamma_{\harm}).
\]
By calibration--coercivity,
\[
E(\alpha_{k}) - E(\gamma_{\harm})
\;\ge\;
c\,\Def_{\cone}(\alpha_{k}),
\]
forcing
\[
\Def_{\cone}(\alpha_{k}) \longrightarrow 0.
\]

Associate to each $\alpha_{k}$ the $(p,p)$ current $S_{k}$ defined by integration
against $\alpha_{k}$.  Uniform energy bounds yield uniform mass bounds for
$\{S_{k}\}$, and compactness of currents gives a weakly convergent subsequence
with limit $T$.

By lower semicontinuity of mass and vanishing calibration defect, $T$ saturates
the calibration inequality:
\[
\langle T, \varphi \rangle = M(T),
\]
so $T$ is a positive $\varphi$--calibrated $(p,p)$ current.  The structure
theorem for such currents on K\"ahler manifolds yields
\[
T = \sum_{j} m_{j}[V_{j}],
\]
where each $V_{j}$ is an irreducible complex analytic $p$--dimensional
subvariety and $m_{j} \in \R_{\ge 0}$.

Since $X$ is projective, analytic subvarieties are algebraic.  Thus $T$ is an
algebraic cycle representing $\gamma$.

\subsection*{What is new}

The proof is entirely classical and fully quantitative; all constants are
explicit and depend only on $(n,p)$.  In particular:

\begin{itemize}[leftmargin=1.1cm]
	\item An $\varepsilon$--net on the calibrated Grassmannian with
	$\varepsilon = \tfrac{1}{10}$ satisfies the explicit covering bound
	\[
	N(n,p,\varepsilon) \le 30^{\,2p(n-p)}.
	\]
	
	\item A cone-to-net distortion factor $K$ may be recorded for comparison with the
	ray/net framework, though the cone-based argument does not require it.
	
	\item A uniform pointwise linear-algebra constant controls the distance to the
	calibrated net in terms of the off-type $(p\pm1,p\mp1)$ components and the
	primitive part of the $(p,p)$ component:
	\[
	C_{0}(n,p) = 2.
	\]
\end{itemize}

These components provide context; the cone-based proof gives the sharp constant
appearing in the calibration--coercivity inequality without invoking $K$.

\subsection*{Idea of the proof}

The argument proceeds in four steps.

\paragraph{1. Energy identity and type control.}
For any closed representative $\alpha \in [\gamma]$ there exists $\eta$ with
$d^{*}\eta = 0$ such that
\[
\alpha = \gamma_{\harm} + d\eta,
\qquad
E(\alpha) - E(\gamma_{\harm}) = \|d\eta\|_{L^{2}}^{2}.
\]
The $(p+1,p-1)$ and $(p-1,p+1)$ components and the primitive part of the
$(p,p)$ component of $\alpha - \gamma_{\harm}$ are controlled in $L^{2}$ by
$\|d\eta\|_{L^{2}}$.

\paragraph{2. Finite calibrated frame.}
Choose an $\varepsilon$--net of calibrated unit simple $(p,p)$ covectors with
$\varepsilon = \tfrac{1}{10}$.  Its covering number satisfies
\[
N \le 30^{\,2p(n-p)}.
\]
Up to a fixed factor $K = \frac{121}{81}$, the pointwise distance to the
calibrated cone is bounded by the distance to this finite net.

\paragraph{3. Pointwise linear algebra.}
Let $\Xi_{x}$ be the span of the net at $x$.  Since $\Xi_{x}$ lies in the
$(p,p)$ space and is orthogonal to off-type components, there is a uniform
constant $C_{0}(n,p) = 2$ for which
\[
\dist(\alpha_{x}, \Xi_{x})^{2}
\le
2\big(
\lvert \alpha_{(p+1,p-1),x} \rvert^{2}
+
\lvert \alpha_{(p-1,p+1),x} \rvert^{2}
+
\lvert (\alpha_{(p,p),x} - \gamma_{\harm,x})_{\prim} \rvert^{2}
\big).
\]

\paragraph{4. Assembly.}
Integrating the pointwise estimate and combining it with the energy controls in
Step~1, together with the cone-to-net factor $K$, yields the inequality
\[
E(\alpha) - E(\gamma_{\harm})
\ge
c\,\Def_{\cone}(\alpha).
\]
In the cone-based argument (Section~\ref{sec:coercivity}), the factor $K$ is not
needed, giving the sharper constant quoted above.

\subsection*{Scope and remarks}

The method applies uniformly for all $1 \le p \le n$.  On K\"ahler manifolds not
assumed projective, the coercivity inequality still forces the minimizing
sequence to converge to an analytic cycle; algebraicity then requires
projectivity of $X$.  All constants are explicit and uniform in $(X,\omega)$.
While some constants (e.g.\ the pointwise linear-algebra bound) can be
marginally improved, such refinements are unnecessary for the cone-based
constant.

The bound $N \le 30^{\,2p(n-p)}$ for the covering number of the calibrated
Grassmannian is convenient but not optimal; any standard packing estimate would
suffice.

\subsection*{Notation and conventions}

All norms and inner products are induced by the K\"ahler metric.  Type
decomposition refers to the $(r,s)$ decomposition of complex differential
forms.  The Lefschetz decomposition into primitive and non-primitive components
is orthogonal with respect to $\omega$.  Weak convergence is taken in the sense
of currents.  Energies and $L^{2}$ norms are over $\R$, while cohomology is
taken over $\Q$ when rationality is required.

\subsection*{Organization}

Section~2 introduces K\"ahler preliminaries and Hodge-theoretic notation.
Section~3 describes the calibrated Grassmannian and the cone geometry.
Section~4 develops the energy-gap and primitive/off-type controls.  Section~5
constructs $\varepsilon$--nets and proves covering estimates.  Section~6 carries
out the pointwise linear-algebra analysis.  Section~7 proves the global
calibration--coercivity inequality.  Section~8 passes from coercivity to
algebraic cycles (Theorem~B).  Section~9 gives an alternative
slicing--calibration proof in the middle degree.  Section~10 provides model
checks, examples, and sharpness considerations.

\subsection*{Two-proof roadmap}

We present two complementary proofs.  The primary proof is fully quantitative,
using the convex calibrated cone and a Hermitian-model projection to obtain a
coercivity constant depending only on $(n,p)$.  This occupies Sections~2--7 and
produces positive analytic cycles.

In the middle degree $n = 2p$, we also give a slicing--amplification--%
calibration argument based on very ample complete intersections and measurable
replacement; see Section~\ref{sec:slicing} for this alternative route.

\section{Notation and K\"ahler Preliminaries}

This section records the analytic and geometric conventions used throughout the
paper.  All norms, operators, and identities are taken with respect to the
K\"ahler metric $g(\cdot,\cdot)=\omega(\cdot,J\cdot)$ and the associated volume
form $d\mathrm{vol}_\omega=\omega^{n}/n!$.  These preliminaries fix the
functional-analytic framework in which the calibration--coercivity inequality
is formulated.

% ----------------------------------------------------------
\paragraph{Ambient setting.}
Let $X$ be a smooth projective complex manifold of complex dimension $n$, with
K\"ahler form $\omega$ and integrable complex structure $J$.
The associated Riemannian metric is
\[
g(\cdot,\cdot)=\omega(\cdot,J\cdot),
\qquad
d\mathrm{vol}_\omega=\frac{\omega^{n}}{n!}.
\]
Throughout the paper, all pointwise and $L^2$ norms are taken with respect to
$g$ (equivalently,~$\omega$).

% ----------------------------------------------------------
\paragraph{Forms, inner products, and energy.}
For $k\ge0$, let $\Lambda^{k}T^{*}X$ denote the bundle of real $k$–forms and
$\Lambda_{\C}^{k}T^{*}X=\Lambda^{k}T^{*}X\otimes\C$ its complexification.
The Hodge star
\[
*:\Lambda^{k}T^{*}X\longrightarrow\Lambda^{2n-k}T^{*}X
\]
satisfies
\[
\langle \alpha,\beta\rangle_{x}\,d\mathrm{vol}_\omega
=
\alpha\wedge *\beta,
\]
and the pointwise norm is $\|\alpha\|^{2}=\langle \alpha,\alpha\rangle$.
The $L^{2}$ inner product and norm are
\[
\langle \alpha,\beta\rangle_{L^{2}}
:=
\int_{X}\langle \alpha,\beta\rangle\,d\mathrm{vol}_\omega,
\qquad
\|\alpha\|^{2}_{L^{2}}
:=
\int_{X}\|\alpha\|^{2}\,d\mathrm{vol}_\omega.
\]
For any measurable $2p$–form $\alpha$, the Dirichlet energy agrees with its
$L^{2}$ norm:
\[
E(\alpha)
=
\|\alpha\|^{2}_{L^{2}}
=
\int_{X}\|\alpha\|^{2}\,d\mathrm{vol}_\omega.
\]

% ----------------------------------------------------------
\paragraph{Exterior calculus and Hodge theory.}
Let $d$ be the exterior derivative and $d^{*}$ its formal adjoint.
The Hodge Laplacian is
\[
\Delta = dd^{*}+d^{*}d.
\]
A smooth form $\eta$ is \emph{harmonic} if $\Delta\eta=0$.
Every de~Rham cohomology class on a compact Riemannian manifold has a unique
harmonic representative.

If $\alpha$ is a smooth closed $k$–form representing a class $[\gamma]$, then
there exists a $(k-1)$–form $\xi$ with $d^{*}\xi=0$ (Coulomb gauge) such that
\[
\alpha=\gharm+d\xi,
\qquad
E(\alpha)-E(\gharm)=\|d\xi\|^{2}_{L^{2}}.
\tag{2}
\]

% ----------------------------------------------------------
\paragraph{Type decomposition.}
Complexifying the cotangent bundle gives
\[
T^{*}X\otimes\C
=
T^{1,0*}X\oplus T^{0,1*}X.
\]
Taking wedge powers yields the $(r,s)$–splitting
\[
\Lambda_{\C}^{k}T^{*}X
=
\bigoplus_{r+s=k}\Lambda^{r,s}T^{*}X.
\]
For a complex form $\alpha$, we write $\alpha^{(r,s)}$ for its $(r,s)$
component.  In particular, any complex $2p$–form decomposes as
\[
\alpha
=
\alpha^{(p+1,p-1)}
+
\alpha^{(p,p)}
+
\alpha^{(p-1,p+1)}.
\]
On a K\"ahler manifold,
\[
d=\partial+\bar\partial,
\qquad
\partial:\Lambda^{r,s}\to\Lambda^{r+1,s},
\quad
\bar\partial:\Lambda^{r,s}\to\Lambda^{r,s+1}.
\]
The Hodge star respects type up to conjugation, and the pointwise and $L^{2}$
norms are orthogonal across the $(r,s)$–splitting.

% ----------------------------------------------------------
\paragraph{Lefschetz operators and primitive forms.}
The Lefschetz operator
\[
L:\Lambda_{\C}^{\bullet}T^{*}X\to\Lambda_{\C}^{\bullet+2}T^{*}X,
\qquad
L(\eta)=\omega\wedge\eta,
\]
has $L^{2}$–adjoint $\Lambda$ (contraction with $\omega$).
A form $\eta$ is \emph{primitive} if $\Lambda\eta=0$.

The Lefschetz decomposition expresses any $(p,p)$–form as an orthogonal sum
\[
\alpha^{(p,p)}=\sum_{r\ge0}L^{r}\eta_{r},
\qquad
\eta_{r}\ \text{primitive}.
\]
We write $(\cdot)_{\prim}$ for the orthogonal projection onto the primitive
subspace.

% ----------------------------------------------------------
\paragraph{K\"ahler identities (used implicitly).}
On a K\"ahler manifold one has the commutator identities
\[
[\Lambda,\partial]=i\,\bar\partial^{*},
\qquad
[\Lambda,\bar\partial]=-\,i\,\partial^{*},
\]
and their adjoints.
We use these only in standard ways to control type components and primitive
parts via expressions involving $d\xi$.

% ==========================================================
% SECTION 3 — Calibrated Grassmannian and Pointwise Cone Geometry (Revised)
% ==========================================================

\section{Calibrated Grassmannian and Pointwise Cone Geometry}
\label{sec:calibrated-grassmannian}

\paragraph{Calibrated Grassmannian.}
Fix a point $x\in X$.  
Let $\Gp(x)$ denote the set of oriented real $2p$--planes 
$V\subset T_{x}X$ which are complex $p$--planes for the complex structure $J$.
Equivalently, $\Gp(x)$ is naturally identified with the complex
Grassmannian $G_{\C}(p,n)$ of $p$--dimensional complex subspaces of
$T^{1,0}_{x}X$.  

Given such a $V\in \Gp(x)$, let $\phi_{V}$ be the normalized
calibrated simple $(p,p)$--form associated to $V$, defined by
\[
\phi_{V}\bigl( v_{1},Jv_{1},\ldots,v_{p},Jv_{p} \bigr) = 1
\]
for any orthonormal basis $\{v_{1},\ldots,v_{p}\}$ of $V$.
Thus each $\phi_{V}$ has unit pointwise norm and determines the calibrated
direction corresponding to the holomorphic $p$--plane $V$.

\paragraph{Calibrated cone at a point.}
Let
\[
\varphi \;=\; \calibform \;=\; \frac{\omega^{p}}{p!}
\]
be the Kähler calibration.
Define the (closed, convex) calibrated cone in $\Lambda^{2p}T^{*}_{x}X$ by
\[
\mathcal{C}_{x}
:=
\Bigl\{
\sum_{j} a_{j} \phi_{V_{j}}
\;:\;
a_{j}\ge 0,\;
V_{j}\in \Gp(x)
\Bigr\}.
\]
Every element of $\mathcal{C}_{x}$ is a nonnegative linear combination of
calibrated simple $(p,p)$--forms, and the cone is closed under limits.

We write
\[
\distcone(\alpha_{x})
:=
\dist\!\bigl(\alpha_{x},\mathcal{C}_{x}\bigr)
\]
for the pointwise distance (with respect to the $g$--norm) from a real
$2p$--form $\alpha_{x}$ to the calibrated cone at $x$.

\paragraph{Finite calibrated frame (net viewpoint).}
Fix $\varepsilon = \tfrac{1}{10}$.
Choose a maximal $\varepsilon$--separated subset 
$\{V_{1},\ldots,V_{N}\}\subset \Gp(x)$, i.e.\ an $\varepsilon$--net
of the calibrated Grassmannian with respect to its standard homogeneous
Riemannian metric.  
Standard packing estimates on the complex Grassmannian yield the explicit
bound
\[
N \;\le\; 30^{\,2p(n-p)}.
\]

Let $\Xi_{x}$ denote the linear span of 
$\{\phi_{V_{1}},\ldots,\phi_{V_{N}}\}$ inside $\Lambda^{2p}T^{*}_{x}X$.
For any form $\alpha_{x}$, let
\[
\dist(\alpha_{x}, \Xi_{x})
\]
be the pointwise norm of the orthogonal projection of $\alpha_{x}$ onto the
orthogonal complement of $\Xi_{x}$.

For convenience we record the cone--to--net comparison constant
\[
K = \Bigl(\tfrac{11}{9}\Bigr)^{2} = \frac{121}{81},
\]
satisfying
\[
\distcone(\alpha_{x})^{2}
\;\le\;
K \,\dist\bigl(\alpha_{x},\Xi_{x}\bigr)^{2}.
\]
The main cone--based proof uses the calibrated cone $\mathcal{C}_{x}$
directly and does not rely on the factor $K$, but the net viewpoint is
included for completeness and for comparison with Appendix~\ref{sec:appendix-covering}.

% ----------------------------------------------------------
% Ray distance vs. convex calibrated cone
% ----------------------------------------------------------

\subsection*{Ray distance vs.\ convex calibrated cone}

For a calibrated simple form $\phi_{V}$ and any real $2p$--form 
$\alpha_{x}\in \Lambda^{2p}T^{*}_{x}X$, consider the ray generated by $\phi_{V}$.
The pointwise distance from $\alpha_{x}$ to this ray is
\[
\dist\bigl(\alpha_{x}, \R_{\ge 0}\,\phi_{V}\bigr)
:=
\inf_{\lambda\ge 0} \|\alpha_{x}-\lambda\phi_{V}\|.
\]
Minimizing over all calibrated rays yields the \emph{ray defect}
\[
\Def_{\mathrm{ray}}(\alpha_{x})
:=
\inf_{V\in \Gp(x)}
\dist\!\left(
\alpha_{x},\,
\R_{\ge 0}\,\phi_{V}
\right).
\]

Since the convex calibrated cone
\[
\mathcal{C}_{x} = \cone\{\phi_{V} : V\in \Gp(x)\}
\]
contains every such ray, one always has
\[
\distcone(\alpha_{x})
\;=\;
\dist\bigl(\alpha_{x},\mathcal{C}_{x}\bigr)
\;\le\;
\Def_{\mathrm{ray}}(\alpha_{x}).
\]
Conversely, using the $\varepsilon$--net $\{V_{j}\}$ and the span
$\Xi_{x}$ as above, one obtains the cone--to--net distortion estimate
\[
\dist\bigl(\alpha_{x},\mathcal{C}_{x}\bigr)^{2}
\;\le\;
K\,\dist\bigl(\alpha_{x},\Xi_{x}\bigr)^{2},
\qquad
K=\frac{121}{81},
\]
so that ray distance and cone distance are equivalent up to this fixed
uniform factor depending only on $(n,p)$.

% ----------------------------------------------------------
% Radial minimization along a calibrated ray
% ----------------------------------------------------------

\begin{lemma}[Explicit minimization in the radial parameter]
	\label{lem:radial-min}
	Fix a point $x \in X$ and a calibrated unit covector
	$\xi \in \Gp(x)$.
	For any real $2p$--form $\alpha_{x} \in \Lambda^{2p}T^{*}_{x}X$, the map
	\[
	\lambda \;\longmapsto\; \|\alpha_{x} - \lambda \xi\|^{2},
	\qquad \lambda \ge 0,
	\]
	is minimized at
	\[
	\lambda^{*} \;=\; \max\{0, \langle \alpha_{x}, \xi \rangle\}.
	\]
	Moreover,
	\[
	\min_{\lambda \ge 0} \|\alpha_{x} - \lambda \xi\|^{2}
	\;=\;
	\|\alpha_{x}\|^{2}
	\;-\;
	\bigl(\langle \alpha_{x}, \xi \rangle_{+}\bigr)^{2},
	\]
	where
	\[
	\langle u, v \rangle_{+}
	\;:=\;
	\max\{0, \langle u, v \rangle\}.
	\]
	Consequently,
	\begin{equation}\label{eq:dist-cal-formula}
		\distcone(\alpha_{x})^{2}
		\;=\;
		\|\alpha_{x}\|^{2}
		\;-\;
		\Bigl(
		\max_{\xi \in \Gp(x)}
		\langle \alpha_{x}, \xi \rangle_{+}
		\Bigr)^{2}.
	\end{equation}
\end{lemma}

\begin{proof}
	Fix $\xi \in \Gp(x)$ with $\|\xi\| = 1$ and define
	\[
	f(\lambda)
	\;:=\;
	\|\alpha_{x} - \lambda \xi\|^{2},
	\qquad \lambda \in \R.
	\]
	Expanding using $\|\xi\|=1$ gives
	\[
	f(\lambda)
	\;=\;
	\|\alpha_{x}\|^{2}
	- 2\lambda\,\langle \alpha_{x}, \xi \rangle
	+ \lambda^{2},
	\]
	which is a strictly convex quadratic in $\lambda$.
	The unconstrained minimizer satisfies $f'(\lambda)=0$, namely
	\[
	\lambda_{\mathrm{unconstr}}
	\;=\;
	\langle \alpha_{x}, \xi \rangle.
	\]
	
	Imposing the constraint $\lambda \ge 0$ yields
	\[
	\lambda^{*}
	\;=\;
	\max\{0, \langle \alpha_{x}, \xi \rangle\}.
	\]
	If $\langle \alpha_{x}, \xi \rangle \ge 0$, then
	\[
	f(\lambda^{*})
	= \|\alpha_{x}\|^{2} - \langle \alpha_{x}, \xi \rangle^{2},
	\]
	while if $\langle \alpha_{x}, \xi \rangle < 0$, the minimum is attained
	at $\lambda^{*}=0$ with value $f(0) = \|\alpha_{x}\|^{2}$.
	Both cases are encoded by
	\[
	\min_{\lambda \ge 0} \|\alpha_{x} - \lambda \xi\|^{2}
	=
	\|\alpha_{x}\|^{2}
	-
	\bigl(\langle \alpha_{x}, \xi \rangle_{+}\bigr)^{2}.
	\]
	
	By definition of the pointwise calibration distance to the cone,
	\[
	\distcone(\alpha_{x})^{2}
	=
	\inf_{\lambda \ge 0,\;\xi \in \Gp(x)}
	\|\alpha_{x} - \lambda \xi\|^{2}.
	\]
	For each fixed $\xi$ we have already minimized over $\lambda \ge 0$, so
	\[
	\distcone(\alpha_{x})^{2}
	=
	\inf_{\xi \in \Gp(x)}
	\Bigl(
	\|\alpha_{x}\|^{2}
	-
	\bigl(\langle \alpha_{x}, \xi \rangle_{+}\bigr)^{2}
	\Bigr)
	=
	\|\alpha_{x}\|^{2}
	-
	\Bigl(
	\sup_{\xi \in \Gp(x)}
	\langle \alpha_{x}, \xi \rangle_{+}
	\Bigr)^{2},
	\]
	which is exactly \eqref{eq:dist-cal-formula}.
\end{proof}

% ----------------------------------------------------------
% Trace L^2 control (used later with Hermitian model)
% ----------------------------------------------------------

\begin{lemma}[Trace $L^{2}$ control]\label{lem:trace-L2}
	Let $\eta$ be the Coulomb potential with $d^{*}\eta = 0$ and
	\[
	\alpha = \gharm + d\eta.
	\]
	Define
	\[
	\beta := (d\eta)^{(p,p)},
	\]
	and let
	\[
	H_{\beta}(x) := \mathcal{I}(\beta_{x}) \in \Herm\bigl(\Lambda^{p,0}_{x}X\bigr),
	\]
	where $d := \dim_{\C}\Lambda^{p,0}_{x}X = \binom{n}{p}$ and
	$\mathcal{I}$ is any fixed isometric identification between
	$\Lambda^{p,p}_{x}T^{*}X$ and $\Herm(\Lambda^{p,0}_{x}X)$.
	Set
	\[
	\mu(x) := \frac{1}{d}\,\tr H_{\beta}(x).
	\]
	Then
	\begin{equation}\label{eq:trace-L2-bound}
		\|\mu\|_{L^{2}}
		\;\le\;
		C_{\Lambda}(n,p)\,\|d\eta\|_{L^{2}},
		\qquad
		C_{\Lambda}(n,p) = d^{-1/2}.
	\end{equation}
\end{lemma}

\begin{proof}
	Pointwise at each $x\in X$, apply Cauchy--Schwarz for the Hilbert--Schmidt
	inner product on $\Herm(\Lambda^{p,0}_{x}X)$:
	\[
	\bigl|\tr H_{\beta}(x)\bigr|
	\;\le\;
	\sqrt{d}\,\|H_{\beta}(x)\|_{\HS}.
	\]
	Hence
	\[
	|\mu(x)|
	= \frac{1}{d}\,\bigl|\tr H_{\beta}(x)\bigr|
	\;\le\;
	d^{-1/2}\,\|H_{\beta}(x)\|_{\HS}.
	\]
	By construction, the identification
	\[
	\mathcal{I} : \Lambda^{p,p}_{x}T^{*}X \longrightarrow \Herm(\Lambda^{p,0}_{x}X)
	\]
	is an isometry with respect to the pointwise norms, so
	\[
	\|H_{\beta}(x)\|_{\HS}
	= \|\beta(x)\|.
	\]
	Moreover, since $\beta$ is the $(p,p)$--component of $d\eta$ and the
	$(r,s)$--components are orthogonal in the Kähler metric, we have the
	pointwise inequality
	\[
	\|\beta(x)\| \;\le\; \|d\eta(x)\|.
	\]
	Combining these estimates gives
	\[
	|\mu(x)|
	\;\le\;
	d^{-1/2}\,\|d\eta(x)\|
	\quad\text{for all } x\in X.
	\]
	Squaring and integrating over $X$ yields
	\[
	\|\mu\|_{L^{2}}
	\;\le\;
	d^{-1/2}\,\|d\eta\|_{L^{2}},
	\]
	which is exactly \eqref{eq:trace-L2-bound}.
\end{proof}

% ----------------------------------------------------------
% Basic properties of the calibration distance
% ----------------------------------------------------------

\begin{proposition}[Well-posedness and basic properties]
	\label{prop:dist-cal-properties}
	For each point $x \in X$ and each real $2p$--form 
	$\alpha_{x} \in \Lambda^{2p}T^{*}_{x}X$, the calibration distance
	$\distcone(\alpha_{x})$ enjoys the following properties.
	\begin{enumerate}
		\item[\textnormal{(1)}] \textbf{Compactness and attainment.}
		The calibrated Grassmannian $\Gp(x)$ is compact.
		Consequently, the maximum in \eqref{eq:dist-cal-formula} is attained,
		and the infimum in the definition of $\distcone(\alpha_{x})$ is in fact a
		minimum.
		
		\item[\textnormal{(2)}] \textbf{Positive homogeneity and Lipschitz continuity.}
		For every scalar $t \ge 0$,
		\[
		\distcone(t\alpha_{x})
		\;=\;
		t\,\distcone(\alpha_{x}).
		\]
		Moreover, for all real $2p$--forms $\alpha_{x},\beta_{x}$ one has
		\[
		\bigl|
		\distcone(\alpha_{x})
		-
		\distcone(\beta_{x})
		\bigr|
		\;\le\;
		\|\alpha_{x} - \beta_{x}\|.
		\]
		
		\item[\textnormal{(3)}] \textbf{Measurability and regularity in $x$.}
		If $\alpha$ is a measurable $2p$--form on $X$, then the map
		\[
		x \longmapsto \distcone(\alpha_{x})
		\]
		is measurable.  
		If $\alpha$ is continuous (respectively smooth), then
		$x \mapsto \distcone(\alpha_{x})$ is continuous
		(respectively smooth away from the locus where the maximizing
		calibrated direction in \eqref{eq:dist-cal-formula} changes).
		
		\item[\textnormal{(4)}] \textbf{Zero-defect characterization.}
		One has $\distcone(\alpha_{x}) = 0$ if and only if
		$\alpha_{x}$ belongs to a calibrated ray, i.e.
		\[
		\alpha_{x} \in \R_{\ge 0}\cdot \Gp(x).
		\]
	\end{enumerate}
\end{proposition}

\begin{proof}
	(1) The calibrated Grassmannian $\Gp(x)$ is a compact homogeneous space
	(isomorphic to the complex Grassmannian $G_{\C}(p,n)$), hence compact in the
	topology induced by the Riemannian metric.
	For fixed $\alpha_{x}$, the map
	\[
	\xi \longmapsto \langle \alpha_{x}, \xi \rangle
	\]
	is continuous on $\Gp(x)$, so the maximum in
	\eqref{eq:dist-cal-formula} is attained.  Therefore the infimum in the
	definition of $\distcone(\alpha_{x})$ (taken over rays
	$\R_{\ge 0}\xi$ with $\xi \in \Gp(x)$ and radial parameter
	$\lambda\ge 0$) is realized by some optimal pair
	$(\lambda^{*},\xi^{*})$.
	
	(2) The positive homogeneity follows directly from the definition:
	\[
	\distcone(t\alpha_{x})
	=
	\inf_{\lambda \ge 0,\;\xi \in \Gp(x)}
	\|t\alpha_{x} - \lambda \xi\|
	=
	t\inf_{\lambda' \ge 0,\;\xi \in \Gp(x)}
	\|\alpha_{x} - \lambda' \xi\|
	=
	t\,\distcone(\alpha_{x}).
	\]
	For the Lipschitz property, recall that the distance to any closed subset
	$C$ of a Hilbert space is $1$--Lipschitz:
	\[
	\bigl|\dist(u,C) - \dist(v,C)\bigr|
	\;\le\;
	\|u-v\|.
	\]
	Here $C = \mathcal{C}_{x}$, the calibrated cone at $x$, so
	\[
	\bigl|
	\distcone(\alpha_{x})
	-
	\distcone(\beta_{x})
	\bigr|
	=
	\bigl|
	\dist(\alpha_{x},\mathcal{C}_{x})
	-
	\dist(\beta_{x},\mathcal{C}_{x})
	\bigr|
	\;\le\;
	\|\alpha_{x} - \beta_{x}\|.
	\]
	
	(3) In a local trivialization of $\Lambda^{2p}T^{*}X$ and of the family of
	calibrated simple forms, the map
	\[
	(x,\xi) \longmapsto \langle \alpha_{x}, \xi \rangle
	\]
	is measurable in $x$ and continuous in $\xi$ whenever $\alpha$ is
	measurable.  Taking the supremum over the compact fiber
	$\Gp(x)$ produces a measurable function of $x$, and
	\eqref{eq:dist-cal-formula} then implies measurability of
	$x \mapsto \distcone(\alpha_{x})$.
	
	If $\alpha$ is continuous (resp.\ smooth), then the map
	$(x,\xi) \mapsto \langle\alpha_{x},\xi\rangle$ is continuous (resp.\ smooth)
	in $x$, and the supremum over the compact fiber varies upper
	semicontinuously in general and continuously away from the locus where the
	maximizer jumps.  Thus $x \mapsto \distcone(\alpha_{x})$ is
	continuous (resp.\ smooth off that ridge set).
	
	(4) If $\alpha_{x} = \lambda\xi$ with $\lambda \ge 0$ and
	$\xi \in \Gp(x)$, then by Lemma~\ref{lem:radial-min} the optimal
	radial parameter is $\lambda^{*}=\lambda$ and the minimum distance is zero,
	so $\distcone(\alpha_{x})=0$.
	
	Conversely, if $\distcone(\alpha_{x})=0$, then
	\eqref{eq:dist-cal-formula} gives
	\[
	\|\alpha_{x}\|^{2}
	=
	\Bigl(
	\max_{\xi \in \Gp(x)}
	\langle \alpha_{x}, \xi \rangle_{+}
	\Bigr)^{2}.
	\]
	For a maximizing direction $\xi^{*}$ with 
	$\langle\alpha_{x},\xi^{*}\rangle_{+} = \|\alpha_{x}\|$, equality holds in
	the Cauchy--Schwarz inequality, so $\alpha_{x}$ is a nonnegative multiple of
	$\xi^{*}$.  Hence $\alpha_{x} \in \R_{\ge 0}\cdot\Gp(x)$,
	as claimed.
\end{proof}

% ----------------------------------------------------------
% Optional: Kähler-angle parametrization (for intuition)
% ----------------------------------------------------------

\subsection*{Optional: K\"ahler-angle parametrization (for intuition)}

Let $x \in X$ and let $V,V' \in \Gp(x)$ be complex $p$--planes.
The relative position of $(V,V')$ is encoded by their $p$ Kähler angles
$\theta_{1},\ldots,\theta_{p} \in [0,\tfrac{\pi}{2})$, the canonical angles
arising from the $U(n)$--invariant geometry of the Grassmannian.
In an adapted unitary frame one has the classical identity
\[
\langle \phi_{V},\phi_{V'} \rangle
= \prod_{j=1}^{p} \cos\theta_{j},
\]
where $\phi_{V}$ and $\phi_{V'}$ denote the associated unit calibrated
simple $(p,p)$--forms.

For small angles, the expansion
\[
\cos\theta
= 1 - \tfrac{1}{2}\theta^{2} + \tfrac{1}{24}\theta^{4}
+ O(\theta^{6})
\]
provides a second--order approximation of the inner product in terms of
$\sum_{j} \sin^{2}\theta_{j}$.  This relation between calibrated directions
and the Kähler angles underlies the quadratic bounds recorded in
Appendix~\ref{sec:appendix-kahler-angles}.

\begin{lemma}[Quadratic control for small K\"ahler angles]
	\label{lem:kahler-angle}
	Let $V,V' \in \Gp(x)$ have Kähler angles
	$\theta_{1},\ldots,\theta_{p}$ satisfying
	\[
	\sum_{j=1}^{p} \theta_{j}^{2} \;\le\; 10^{-2}.
	\]
	Then the corresponding calibrated unit covectors $\phi_{V}$ and $\phi_{V'}$
	satisfy the estimate
	\begin{equation}\label{eq:kahler-angle-est}
		0.49\sum_{j=1}^{p} \sin^{2}\theta_{j}
		\;\le\;
		1 - \langle \phi_{V}, \phi_{V'} \rangle
		\;\le\;
		0.502\sum_{j=1}^{p} \sin^{2}\theta_{j}.
	\end{equation}
\end{lemma}

\begin{proof}
	This is an immediate specialization of Proposition~\ref{prop:quadratic-control}
	in Appendix~\ref{sec:appendix-kahler-angles}, applied to the Kähler angles
	$\theta_{1},\ldots,\theta_{p}$ between $V$ and $V'$.
\end{proof}

\begin{remark}[Geometric meaning of Lemma~\ref{lem:kahler-angle}]
	Lemma~\ref{lem:kahler-angle} shows that, when the Kähler angles between two
	complex $p$--planes are small, the deviation of their calibrated directions is
	quadratically controlled by the sum of the squared angles.  Since
	$\langle\phi_{V},\phi_{V'}\rangle = \prod_{j=1}^{p}\cos\theta_{j}$, the
	quantity
	\[
	1 - \langle \phi_{V},\phi_{V'}\rangle
	\]
	measures the pointwise misalignment between the two calibrated simple
	$(p,p)$--forms.  Lemma~\ref{lem:kahler-angle} asserts that this misalignment is
	comparable, up to uniform constants, to the elementary quadratic quantity
	$\sum_{j=1}^{p}\sin^{2}\theta_{j}$ whenever $\sum \theta_{j}^{2}$ is suitably
	small.  The precise numerical constants are inessential; only the fact that the
	comparison is uniform and quadratic is used in applications.
\end{remark}

	% ============================================================
%                    SECTION 4
% ============================================================

\section{Energy Gap and Primitive/Off--Type Controls}
\label{sec:energy-gap}

Let $(X,\omega)$ be a compact K\"ahler manifold of complex dimension $n$,
and let $\alpha$ be a smooth real $2p$–form representing a fixed class
$[\alpha] \in H^{2p}(X,\RR)$.
The purpose of this section is to relate the $L^{2}$–distance of $\alpha$
from the calibrated cone to the analytic energy of the unique Coulomb potential
solving $d^{*}d\eta = d^{*}\alpha$.
This leads to an energy gap estimate and eventually to coercivity in the
$(p\!+\!1,p\!-\!1)$– and $(p\!-\!1,p\!+\!1)$–types and in the primitive part
of $(p,p)$–forms.

\subsection*{Coulomb potential}
Fix a representative $\alpha$ of $[\alpha]$.  Since $d\alpha = 0$, the elliptic
equation
\[
d^{*}d\eta = d^{*}\alpha
\]
admits a unique solution $\eta$ orthogonal to $\ker d$, giving the Hodge
decomposition
\[
\alpha
= \gamma_{\harm} + d\eta,
\]
where $\gamma_{\harm}$ is the unique harmonic representative of $[\alpha]$.
We define the energy of $\alpha$ by
\[
E(\alpha) := \|d\eta\|^{2}_{L^{2}}.
\]

\subsection*{Energy Identity}
We now express $E(\alpha)$ in terms of type components.  Since
$\gamma_{\harm}$ is harmonic and of pure type $(p,p)$, we have
$d^{*}\gamma_{\harm}=0$ and
\[
\|\alpha\|^{2}_{L^{2}}
= \|\gamma_{\harm}\|^{2}_{L^{2}} + \|d\eta\|^{2}_{L^{2}}
\]
because $\gamma_{\harm} \perp d\eta$.
Thus:

\begin{equation}\label{eq:energy-split}
	E(\alpha)
	= \|\alpha\|_{L^{2}}^{2} - \|\gamma_{\harm}\|_{L^{2}}^{2}
	= \|d\eta\|^{2}_{L^{2}}.
	\tag{11}
\end{equation}

Decomposing $\alpha$ into types,
\[
\alpha
=
\alpha^{(p+1,p-1)}
+ \alpha^{(p,p)}
+ \alpha^{(p-1,p+1)},
\]
and noting that $\gamma_{\harm} = \gamma_{\harm}^{(p,p)}$, we obtain

\begin{equation}\label{eq:type-split}
	\|\alpha - \gamma_{\harm}\|_{L^{2}}^{2}
	=
	\|\alpha^{(p+1,p-1)}\|_{L^{2}}^{2}
	+ \|\alpha^{(p-1,p+1)}\|_{L^{2}}^{2}
	+ \|(\alpha^{(p,p)} - \gamma_{\harm})\|_{L^{2}}^{2}.
	\tag{12}
\end{equation}

Finally, the standard K\"ahler identities imply control of the non-\((p,p)\)
types and the primitive part of the \((p,p)\)–component in terms of $d\eta$:

\begin{equation}\label{eq:primitive-control}
	\|\alpha^{(p+1,p-1)}\|_{L^{2}}
	+
	\|\alpha^{(p-1,p+1)}\|_{L^{2}}
	+
	\|(\alpha^{(p,p)} - \gamma_{\harm})_{\prim}\|_{L^{2}}
	\;\le\;
	C(n,p)\,\|d\eta\|_{L^{2}}.
	\tag{13}
\end{equation}

\begin{lemma}[Coulomb decomposition and energy identity]\label{lem:coulomb}
	Let $\alpha$ be a smooth closed real $2p$–form on a compact K\"ahler manifold.
	Write $\alpha = \gamma_{\harm} + d\eta$ for its Coulomb decomposition.
	Then:
	
	\begin{enumerate}[label=(\roman*)]
		
		\item
		$\displaystyle
		E(\alpha)
		= \|d\eta\|_{L^{2}}^{2}
		= \|\alpha\|_{L^{2}}^{2} - \|\gamma_{\harm}\|_{L^{2}}^{2},
		$
		as in~\eqref{eq:energy-split}.
		
		\item
		The difference from the harmonic representative satisfies
		\[
		\|\alpha - \gamma_{\harm}\|_{L^{2}}^{2}
		=
		\|\alpha^{(p+1,p-1)}\|_{L^{2}}^{2}
		+ \|\alpha^{(p-1,p+1)}\|_{L^{2}}^{2}
		+ \|(\alpha^{(p,p)} - \gamma_{\harm})\|_{L^{2}}^{2},
		\]
		as in~\eqref{eq:type-split}.
		
		\item
		The non-harmonic part is controlled by the primitive and $(p\!\pm\!1,p\!\mp\!1)$
		types:
		\[
		\|\alpha^{(p+1,p-1)}\|_{L^{2}}
		+
		\|\alpha^{(p-1,p+1)}\|_{L^{2}}
		+
		\|(\alpha^{(p,p)} - \gamma_{\harm})_{\prim}\|_{L^{2}}
		\;\le\;
		C(n,p)\,\sqrt{E(\alpha)},
		\]
		consistent with~\eqref{eq:primitive-control}.
		
	\end{enumerate}
	
\end{lemma}

\begin{proof}
	Item (i) follows from the orthogonality $\gamma_{\harm}\perp d\eta$ and the
	Coulomb normalization $d^{*}\eta=0$.
	Item (ii) is the orthogonal decomposition of the type components relative to
	$\gamma_{\harm}^{(p,p)}$.
	Item (iii) follows from the K\"ahler identities:
	$d = \partial + \bar\partial$, $d^{*} = \partial^{*} + \bar\partial^{*}$,
	together with elliptic estimates for the operator $d^{*}d$ on $\eta$.
\end{proof}

% ------------------------------------------------------------
% SECTION 5 — The Calibrated Grassmannian and an Explicit ε–Net
% ------------------------------------------------------------

\section{The Calibrated Grassmannian and an Explicit $\varepsilon$–Net}

\subsection*{Fiberwise geometry}

Fix $x\in X$ and set
\[
\varphi := \frac{\omega^{p}}{p!}.
\]
Define the calibrated Grassmannian at $x$ by
\[
G_{p}(x)
:=
\Big\{
\xi \in \Lambda^{2p}T^{*}_{x}X :
\|\xi\| = 1,\;
\xi\ \text{simple of type $(p,p)$},\;
\varphi_{x}(\xi)=1
\Big\}.
\]
This is the set of unit simple $(p,p)$ covectors saturated by the K\"ahler
calibration $\varphi_{x}$.  Equivalently, $G_{p}(x)$ is the image of the
complex Grassmannian $G_{\C}(p,n)$ under the map sending a $p$--plane
$V\subset T^{1,0}_{x}X$ to its associated calibrated covector $\phi_{V}$.
With the metric induced by $\omega$, this map is an isometric embedding
(up to normalization), and therefore
\[
G_{p}(x) \cong G_{\C}(p,n)
\]
with its standard Fubini--Study metric.  In particular, $G_{p}(x)$ is
compact, smooth, homogeneous, and has real dimension
\[
d := \dim_{\R} G_{p}(x)
= 2p(n-p).
\]

\subsection*{$\varepsilon$–nets and covering estimates}

Fix $\varepsilon = \tfrac{1}{10}$.  
On each fiber $G_{p}(x)$ (with the Fubini--Study geodesic distance
$d_{\mathrm{FS}}$), choose a maximal $\varepsilon$–separated set
\[
\{\xi(x)_\ell\}_{\ell=1}^{N(x)}
\subset G_{p}(x),
\qquad
d_{\mathrm{FS}}(\xi(x)_\ell,\xi(x)_m) \ge \varepsilon
\ \text{for all }\ell\ne m,
\]
such that no additional point of $G_{p}(x)$ can be added while preserving
this separation property.

By compactness and the standard packing principle on compact homogeneous
spaces, such maximal $\varepsilon$–separated sets are automatically
$\varepsilon$–nets: for every $\xi \in G_{p}(x)$ there exists an index
$\ell$ with  
\[
d_{\mathrm{FS}}(\xi,\xi(x)_\ell) \le \varepsilon.
\]

\begin{lemma}[Covering number]\label{lem:covering-number}
	Let $d = 2p(n-p)$.  
	There exists a constant $C(n,p)$ depending only on $(n,p)$ such that every
	maximal $\varepsilon$–separated set in $G_{p}(x)$ satisfies
	\begin{equation}\label{eq:grass-cover}
		N(x) \;\le\; C(n,p)\,\varepsilon^{-d}.
		\tag{5.1}
	\end{equation}
\end{lemma}

\begin{proof}
	Cover $G_{p}(x)$ by the geodesic balls
	\[
	B\!\left(\xi(x)_\ell,\,\tfrac{\varepsilon}{2}\right),
	\qquad \ell=1,\dots,N(x),
	\]
	of radius $\varepsilon/2$ in the Fubini--Study metric.  
	Because the points are $\varepsilon$–separated, these balls are pairwise
	disjoint.  By maximality of the separated set, the $\varepsilon$–balls
	\[
	B\!\left(\xi(x)_\ell,\,\varepsilon\right)
	\]
	cover $G_{p}(x)$.
	
	Since $G_{p}(x)$ is a compact homogeneous space, the volume of a small
	geodesic ball depends only on the radius, not on its center.  
	Let $V(r)$ denote the volume of a geodesic ball of radius $r$.  
	Then disjointness gives
	\[
	N(x)\,V(\varepsilon/2)
	\;\le\; \Vol\bigl(G_{p}(x)\bigr),
	\]
	while the covering property yields
	\[
	\Vol\bigl(G_{p}(x)\bigr)
	\;\le\; N(x)\,V(\varepsilon).
	\]
	
	For small $r$ one has the uniform expansion
	\[
	V(r) = c_{d}\,r^{d} + O(r^{d+2}),
	\]
	with $c_{d}>0$ depending only on $d = \dim_{\R} G_{p}(x)$.  
	Since $G_{p}(x)$ is homogeneous, there exist constants $A(n,p)$ and $B(n,p)$
	such that
	\[
	A(n,p)\,r^{d} \le V(r) \le B(n,p)\,r^{d}
	\qquad\text{for } 0<r\le 1.
	\]
	
	Combining the two volume inequalities gives
	\[
	N(x)\,A(n,p)\,(\varepsilon/2)^{d}
	\;\le\; \Vol\bigl(G_{p}(x)\bigr)
	\;\le\; N(x)\,B(n,p)\,\varepsilon^{d},
	\]
	so cancelling $\Vol(G_{p}(x))$ yields
	\[
	N(x) \;\le\;
	\frac{B(n,p)}{A(n,p)}\,(2^{d})\,
	\varepsilon^{-d}.
	\]
	
	Absorbing the constants into
	\[
	C(n,p) := \frac{B(n,p)}{A(n,p)}\,2^{d},
	\]
	we obtain the desired estimate \eqref{eq:grass-cover}.
\end{proof}
% ============================================================
% SECTION 6 — Pointwise Linear Algebra: Controlling the Net Distance
% ============================================================

\section{Pointwise Linear Algebra: Controlling the Net Distance}
\label{sec:linear-algebra}

In this section we develop the pointwise linear--algebraic estimates
that control the distance of a real $2p$--form to the calibrated
span generated by the $\varepsilon$--net constructed in Section~5.
The goal is to show that the net distance (and therefore the cone
distance) is controlled by two quantities:

\begin{itemize}
	\item the off--type components $\alpha_{x}^{(p+1,p-1)}$ and 
	$\alpha_{x}^{(p-1,p+1)}$, and 
	\item the primitive traceless part of the $(p,p)$--component.
\end{itemize}

These pointwise inequalities form the core of the coercivity 
estimate used later in Section~\ref{sec:coercivity}.

% ------------------------------------------------------------
\subsection*{Calibrated span}

Fix $x\in X$ and let 
\[
\{\xi_{\ell}(x)\}_{\ell=1}^{N(x)} \subset G_{p}(x)
\]
be the $\varepsilon$--net of Section~5, with $\varepsilon=\tfrac{1}{10}$.
Define the calibrated span at $x$ by
\[
\Xi_{x}:=
\Span\{\xi_{\ell}(x):1\le \ell \le N(x)\}
\subset \Lambda^{p,p}T_{x}^{*}X.
\]

Each $\xi_{\ell}(x)$ is a unit simple $(p,p)$--covector, hence lies
entirely in the $(p,p)$--subspace of $\Lambda^{2p}T_{x}^{*}X$ and is
orthogonal to all off--type $(p+1,p-1)$ and $(p-1,p+1)$ components
with respect to the K\"ahler metric.

Thus every $\alpha_{x}\in\Lambda^{2p}T_{x}^{*}X$ admits an
orthogonal type decomposition
\begin{equation}\label{eq:typesplit-orth}
	\alpha_{x}
	=
	\alpha_{x}^{(p+1,p-1)}
	\;+\;
	\alpha_{x}^{(p-1,p+1)}
	\;\perp\;
	\alpha_{x}^{(p,p)}.
	\tag{21}
\end{equation}

% ------------------------------------------------------------
\subsection*{Pointwise net distance}

Define the pointwise net distance
\[
D_{\mathrm{net}}(\alpha_{x})
:=
\min_{\ell,\;\lambda\ge 0}
\|\alpha_{x} - \lambda\xi_{\ell}(x)\|.
\]

\begin{lemma}[Off--type separation for $D_{\mathrm{net}}$]\label{lem:typesplit}
	For every $x$ and every $\alpha_{x}\in\Lambda^{2p}T^{*}_{x}X$,
	\begin{equation}\label{eq:Dnet-typesplit}
		D_{\mathrm{net}}(\alpha_{x})^{2}
		=
		\|\alpha_{x}^{(p+1,p-1)}\|^{2}
		+
		\|\alpha_{x}^{(p-1,p+1)}\|^{2}
		+
		\min_{1\le \ell\le N(x),\,\lambda\ge 0}
		\|\alpha_{x}^{(p,p)} - \lambda \xi_{\ell}(x)\|^{2}.
		\tag{22}
	\end{equation}
\end{lemma}

\begin{proof}
	For each $\ell$ and each $\lambda\ge 0$, the form $\lambda\xi_{\ell}(x)$
	lies in the $(p,p)$--subspace.  By the orthogonality in
	\eqref{eq:typesplit-orth},
	\[
	\|\alpha_{x} - \lambda\xi_{\ell}(x)\|^{2}
	=
	\|\alpha_{x}^{(p+1,p-1)}\|^{2}
	+
	\|\alpha_{x}^{(p-1,p+1)}\|^{2}
	+
	\|\alpha_{x}^{(p,p)} - \lambda\xi_{\ell}(x)\|^{2}.
	\]
	Minimizing over $\ell$ and $\lambda$ gives \eqref{eq:Dnet-typesplit}.
\end{proof}

% ------------------------------------------------------------
\subsection*{Projection estimate}

We now show that the $(p,p)$--term in \eqref{eq:Dnet-typesplit}
is controlled by a purely $(p,p)$ quantity arising from the Hermitian
model for $(p,p)$--forms and a rank--one approximation inequality.

\begin{lemma}[Hermitian model for $(p,p)$]\label{lem:hermitian-model}
	Fix $x$ and identify $\Lambda^{p,0}T_x^{*}X$ with a Hermitian space 
	$\bigl(\mathcal{H},\langle\cdot,\cdot\rangle\bigr)$ of complex dimension 
	$d=\binom{n}{p}$.  
	There is an isometric isomorphism
	\[
	\mathcal{I} : \Lambda^{p,p}T_x^{*}X \;\longrightarrow\; \Herm(\mathcal{H})
	\]
	(with Hilbert--Schmidt norm on the right) such that:
	\begin{enumerate}
		\item for $\alpha_x^{(p,p)}\in\Lambda^{p,p}$, the matrix 
		$H_\alpha := \mathcal{I}(\alpha_x^{(p,p)})$ is Hermitian;
		
		\item for any unit decomposable $p$--vector $v\in\Lambda^{p,0}$,  
		the calibrated covector $\xi_v$ satisfies
		\[
		\mathcal{I}(\xi_v) = P_v := v\otimes v^{*}
		\]
		(the rank--one projector);
		
		\item the contraction (trace) corresponds to the Lefschetz trace:  
		there exists $\mu(\alpha_x)\in\R$ such that
		\[
		\mathcal{I}\bigl( (\alpha_x^{(p,p)})_{\mathrm{prim}} \bigr)
		=
		H_\alpha - \mu(\alpha_x)\, I_{\mathcal{H}},
		\qquad
		\mu(\alpha_x) = \frac{1}{d}\operatorname{tr}(H_\alpha).
		\]
	\end{enumerate}
	
	\emph{Proof sketch.}
	This is the standard identification of $(p,p)$--forms with Hermitian forms
	on $\Lambda^{p,0}$ via
	\[
	H_\alpha(u)=\frac{\alpha(u\wedge\overline{u})}{\|u\|^{2}}
	\]
	and polarization.  
	Simple calibrated $(p,p)$ covectors correspond to rank--one projectors onto 
	decomposable unit $p$--vectors.  
	The Lefschetz trace corresponds to the normalized trace on $\Herm(\mathcal{H})$; 
	subtracting the trace gives the primitive (traceless) component.
	\qed
\end{lemma}

\begin{lemma}[Rank--one approximation controls the traceless part]\label{lem:rankone}
	There exists a finite constant $C_{\mathrm{rank}}(d)>0$, depending only on
	$d=\dim_{\C}\mathcal{H}$, such that for every $H \in \Herm(\mathcal{H})$,
	\[
	\min_{\substack{v\in\mathcal{H},\,\|v\|=1 \\ \lambda \ge 0}}
	\|H - \lambda(v\otimes v^{*})\|_{\mathrm{HS}}^{2}
	\;\le\;
	C_{\mathrm{rank}}(d)\,\bigl\|H - \tfrac{\tr(H)}{d} I_{\mathcal{H}}\bigr\|_{\mathrm{HS}}^{2}.
	\]
\end{lemma}

\begin{proof}
	Consider the compact ``unit traceless shell''
	\[
	\mathcal{S}
	:=
	\Bigl\{H\in\Herm(\mathcal{H}) \;:\;
	\bigl\|H - \tfrac{\tr(H)}{d} I_{\mathcal{H}}\bigr\|_{\HS}=1\Bigr\}.
	\]
	The functional
	\[
	\Phi(H)
	:=
	\min_{\substack{v\in\mathcal{H},\,\|v\|=1 \\ \lambda \ge 0}}
	\|H - \lambda(v\otimes v^{*})\|_{\mathrm{HS}}^{2}
	\]
	is continuous on $\mathcal{S}$ (the minimization set is compact), hence attains a
	maximum $C_{\mathrm{rank}}(d):=\sup_{H\in\mathcal{S}}\Phi(H)<\infty$.  For general
	$H\neq 0$, scale by the traceless norm to obtain the stated inequality.
\end{proof}

\begin{proposition}[Projection estimate in $(p,p)$]\label{prop:pp-projection}
	There exists a constant $C_{0}=C_{0}(n,p)$ such that for all $x$ and all
	$\alpha_{x}$,
	\begin{equation}\label{eq:pp-projection}
		\min_{\ell,\;\lambda\ge 0}
		\bigl\|\alpha_{x}^{(p,p)} - \lambda\,\xi_{\ell}(x)\bigr\|^{2}
		\;\le\;
		C_{0}(n,p)\,
		\bigl\|%
		\bigl(\alpha_{x}^{(p,p)} - \gamma_{\harm,x}\bigr)_{\prim}
		\bigr\|^{2}.
		\tag{23}
	\end{equation}
		In particular, one may take $C_{0}(n,p)=C_{\mathrm{rank}}(d)$ with $d=\binom{n}{p}$.
\end{proposition}

\begin{proof}
	Set
	\[
	\beta_{x} := \alpha_{x}^{(p,p)} - \gamma_{\harm,x}
	\in \Lambda^{p,p}T^{*}_{x}X,
	\qquad
	H := \mathcal{I}(\beta_{x}) \in \Herm(\mathcal{H}),
	\]
	where $\mathcal{I}$ is the isometric isomorphism of
	Lemma~\ref{lem:hermitian-model}.  
	By Lemma~\ref{lem:hermitian-model}, the traceless part of $H$ is exactly
	the Hermitian model of the primitive part:
	\[
	H - \mu(\alpha_{x})\,I_{\mathcal{H}}
	=
	\mathcal{I}\bigl(
	(\alpha_{x}^{(p,p)} - \gamma_{\harm,x})_{\prim}
	\bigr),
	\qquad
	\mu(\alpha_{x}) = \tfrac{1}{d}\tr(H).
	\]
	Hence
	\[
	\bigl\|H - \mu(\alpha_{x})\,I_{\mathcal{H}}\bigr\|_{\mathrm{HS}}
	=
	\bigl\|%
	(\alpha_{x}^{(p,p)} - \gamma_{\harm,x})_{\prim}
	\bigr\|.
	\]
	
	Applying Lemma~\ref{lem:rankone} to $H$ yields
	\[
	\min_{\substack{v\in\mathcal{H},\,\|v\|=1\\ \lambda\ge 0}}
	\bigl\|H - \lambda(v\otimes v^{*})\bigr\|_{\mathrm{HS}}^{2}
	\;\le\;
	C_{\mathrm{rank}}(d)\,
	\bigl\|H - \mu(\alpha_{x})\,I_{\mathcal{H}}\bigr\|_{\mathrm{HS}}^{2}
	=
	C_{\mathrm{rank}}(d)\,
	\bigl\|%
	(\alpha_{x}^{(p,p)} - \gamma_{\harm,x})_{\prim}
	\bigr\|^{2}.
	\]
	
	By the defining properties of $\mathcal{I}$, for each calibrated unit
	covector $\xi_{v}$ corresponding to $v$ one has
	\[
	\mathcal{I}(\xi_{v}) = v\otimes v^{*},
	\quad
	\|\xi_{v}\| = 1,
	\]
	and $\mathcal{I}$ is an isometry.  Pulling back the above inequality via
	$\mathcal{I}^{-1}$ gives
	\[
	\min_{\xi} \min_{\lambda\ge 0}
	\bigl\|\beta_{x} - \lambda\xi\bigr\|^{2}
	\;\le\;
	C_{\mathrm{rank}}(d)\,
	\bigl\|%
	(\alpha_{x}^{(p,p)} - \gamma_{\harm,x})_{\prim}
	\bigr\|^{2},
	\]
	where the minimum is taken over all calibrated unit covectors at $x$.
	
	Finally, approximate the minimizing calibrated direction by some net
	vector $\xi_{\ell}(x)$ from the $\varepsilon$--net of Section~5.  The net
	contains such directions up to the fixed tolerance $\varepsilon$, and
	the resulting approximation only changes the constant by a bounded
	factor depending on $(n,p)$.  Absorbing this factor into $C_{0}(n,p)$
	and taking $C_{0}(n,p)=C_{\mathrm{rank}}(d)$ yields \eqref{eq:pp-projection}.
\end{proof}

\begin{corollary}[Pointwise control of $D_{\mathrm{net}}$]\label{cor:Dnet-pointwise}
	For all $x$ and all $\alpha_{x}$,
	\begin{equation}\label{eq:Dnet-pointwise}
		D_{\mathrm{net}}(\alpha_{x})^{2}
		\;\le\;
		C_{0}(n,p)\Bigl(
		\|\alpha_{x}^{(p+1,p-1)}\|^{2}
		+
		\|\alpha_{x}^{(p-1,p+1)}\|^{2}
		+
		\bigl\|%
		(\alpha_{x}^{(p,p)} - \gamma_{\harm,x})_{\prim}
		\bigr\|^{2}
		\Bigr).
		\tag{24}
	\end{equation}
\end{corollary}

\begin{proof}
	Combine Lemma~\ref{lem:typesplit} with
	Proposition~\ref{prop:pp-projection}.
\end{proof}

\paragraph{Fixing an explicit constant.}
In the previous projection estimate we obtained a constant
$C_{0}(n,p)$ depending only on $(n,p)$.
For the remainder of the paper we fix the explicit choice
\[
C_{0}(n,p) := 2,
\]
which suffices for all subsequent global estimates.
Any quantitative improvement in the rank--one approximation
(Lemma~\ref{lem:rankone}) or in the $\varepsilon$--net approximation
step would simply decrease this constant proportionally, but no such
refinement is needed for our purposes.

\begin{proposition}[Pointwise cone projection bound]\label{prop:cone-projection}
	At each $x\in X$ and for every 
	$\alpha_{x}\in \Lambda^{2p}T^{*}_{x}X$, decompose
	\[
	\alpha_{x} 
	= 
	\alpha_{x}^{(p+1,p-1)}
	\;\perp\;
	\alpha_{x}^{(p,p)}
	\;\perp\;
	\alpha_{x}^{(p-1,p+1)}.
	\]
	Let 
	\[
	H(x) := \mathcal{I}\!\left(\alpha_{x}^{(p,p)} - \gamma_{\harm,x}\right)
	\in \Herm(\mathcal{H}),
	\qquad
	d := \binom{n}{p},
	\qquad
	\mu(x) := \tfrac{1}{d}\operatorname{tr} H(x).
	\]
	Let $H_{-}(x)$ denote the negative part in the spectral decomposition of $H(x)$.
	Then
	\begin{equation}\label{eq:cone-dist-H}
		\mathrm{dist}_{\mathrm{cone}}(\alpha_{x})^{2}
		=
		\|\alpha_{x}^{(p+1,p-1)}\|^{2}
		+\|\alpha_{x}^{(p-1,p+1)}\|^{2}
		+\| H_{-}(x)\|_{\mathrm{HS}}^{2}.
		\tag{25}
	\end{equation}
	Moreover, since the orthogonal trace--traceless splitting yields
	\[
	\|H(x)\|_{\mathrm{HS}}^{2}
	= \|H(x)-\mu(x) I\|_{\mathrm{HS}}^{2} + d\,\mu(x)^{2},
	\]
	we obtain the bound
	\[
	\mathrm{dist}_{\mathrm{cone}}(\alpha_{x})^{2}
	\;\le\;
	\|\alpha_{x}^{(p+1,p-1)}\|^{2}
	+\|\alpha_{x}^{(p-1,p+1)}\|^{2}
	+\|(\alpha_{x}^{(p,p)} - \gamma_{\harm,x})_{\prim}\|^{2}
	+ d\,\mu(x)^{2}.
	\]
\end{proposition}

\begin{proof}
	Projecting $\alpha_{x}$ orthogonally onto the $(p,p)$--space separates the
	off--type terms exactly.  
	Under the Hermitian isometry $\mathcal{I}$, the calibrated cone corresponds to
	the PSD cone in $\Herm(\mathcal{H})$, hence the metric projection of $H(x)$ onto
	the cone is $H_{+}(x)$ and 
	$\|H(x)-H_{+}(x)\|_{\mathrm{HS}}^{2} = \|H_{-}(x)\|_{\mathrm{HS}}^{2}$.
	This gives \eqref{eq:cone-dist-H}.
	
	The identity 
	\[
	\|H\|_{\mathrm{HS}}^{2}
	= \|H-\mu(x)I\|_{\mathrm{HS}}^{2} + d\,\mu(x)^{2}
	\]
	is the orthogonal decomposition into primitive (traceless) and Lefschetz trace
	components.  
	Pulling this back via $\mathcal{I}^{-1}$ yields the stated inequality.
\end{proof}

%================================

% ==========================================================
%  SECTION 7
\section{Calibration--Coercivity (Explicit) and Its Proof}
\label{sec:cal-coercivity}

Let $(X,\omega)$ be a smooth projective K\"ahler manifold and let
$\gamma\in H^{2p}(X,\R)\cap H^{p,p}(X)$ be a de~Rham class.
Denote by $\gharm$ its unique $\omega$–harmonic representative and by
$E(\cdot)$ the Dirichlet energy.

For each $x\in X$, the fiberwise calibrated cone $K_p(x)$ is the closed cone of
$(p,p)$–forms saturated by the K\"ahler calibration.  
The global cone defect of a form $\alpha$ is
\[
\Defcone(\alpha)
:= \int_X \distcone(\alpha_x)^2\,d\mathrm{vol}_\omega(x),
\qquad
\distcone(\alpha_x)
:= \inf_{\beta_x\in K_p(x)} \|\alpha_x - \beta_x\|.
\]

The main estimate of this section is the following explicit version of
Theorem~A.

\begin{theorem}[Explicit calibration--coercivity]
	\label{thm:cal-coercivity}
	For every smooth closed representative $\alpha\in[\gamma]$ one has
	\begin{equation}\label{eq:global-coercivity}
		E(\alpha)-E(\gharm) \;\ge\; c\,\Defcone(\alpha),
	\end{equation}
	with explicit constant
	\begin{equation}\label{eq:c-constant}
		c \;=\; \frac{1}{2 + d\,C_\Lambda^2},
		\qquad
		d=\binom{n}{p},
	\end{equation}
	where $C_\Lambda=C_\Lambda(n,p)$ is the Hermitian trace constant from
	Section~6 (Lemma~13.2).  
	The constant $c$ depends only on $(n,p)$ and not on $[\gamma]$.
\end{theorem}

\begin{proof}
	We follow the pointwise linear algebra and global $L^2$ decomposition
	from Proposition~6.6 together with the Hermitian trace estimate in
	Lemma~13.2.
	
	\medskip\noindent
	\textbf{Step 1: Global control of off–type and primitive parts.}
	Decompose $\alpha$ into its Hodge components:
	\[
	\alpha
	= \alpha^{(p+1,p-1)} + \alpha^{(p,p)} + \alpha^{(p-1,p+1)}.
	\]
	By Lemma~\ref{lem:coulomb} and the K\"ahler identities (cf.~\eqref{eq:primitive-control}),
	the non--$(p,p)$ types and the primitive part of the $(p,p)$–component satisfy
	the global estimate
	\begin{equation}\label{eq:off-type-primitive-integral}
		\int_X \Bigl(
		|\alpha^{(p+1,p-1)}|^2
		+ |\alpha^{(p-1,p+1)}|^2
		+ |(\alpha^{(p,p)}-\gharm)_{\mathrm{prim}}|^2
		\Bigr)\,d\mathrm{vol}_\omega
		\;\le\;
		2\bigl(E(\alpha)-E(\gharm)\bigr).
	\end{equation}
	
	\medskip\noindent
	\textbf{Step 2: Trace component control via the Hermitian model.}
	At each $x$, let
	\[
	H(x)
	:= \mathcal{I}\!\left(\alpha^{(p,p)}_x - (\gharm)_x\right)
	\in \mathrm{Herm}(\mathcal{H}),
	\qquad
	\dim_{\C}\mathcal{H} = d=\binom{n}{p},
	\]
	be the Hermitian matrix associated to the $(p,p)$–difference via the
	isometric identification of Lemma~\ref{lem:hermitian-model}.
	Define
	\[
	\mu(x) := \frac{1}{d}\operatorname{tr} H(x).
	\]
	In terms of the Lefschetz decomposition, this means
	\[
	\alpha^{(p,p)} - \gharm
	= \mu\,\omega^p
	+ (\alpha^{(p,p)} - \gharm)_{\mathrm{prim}}.
	\]
	The Hermitian trace estimate (Lemma~13.2) gives
	\[
	d\int_X \mu(x)^2\,d\mathrm{vol}_\omega(x)
	\;\le\;
	d\,C_\Lambda^2 \,\|\alpha-\gharm\|_{L^2}^2
	=
	d\,C_\Lambda^2 \bigl(E(\alpha)-E(\gharm)\bigr).
	\]
	
	Combining this with \eqref{eq:off-type-primitive-integral} and the orthogonal
	decomposition
	\[
	\|\alpha-\gharm\|_{L^2}^2
	=
	\int_X \Bigl(
	|\alpha^{(p+1,p-1)}|^2
	+ |\alpha^{(p-1,p+1)}|^2
	+ |(\alpha^{(p,p)}-\gharm)_{\mathrm{prim}}|^2
	+ d\,\mu^2
	\Bigr) d\mathrm{vol}_\omega
	\]
	yields
	\begin{equation}\label{eq:L2-difference}
		\int_X |\alpha-\gharm|^2\,d\mathrm{vol}_\omega
		\;\le\;
		(2 + d\,C_\Lambda^2)\bigl(E(\alpha)-E(\gharm)\bigr).
	\end{equation}
	
	\medskip\noindent
	\textbf{Step 3: Relating the cone defect to controlled components (unconditional).}
	By Proposition~\ref{prop:cone-projection},
	\[
	\distcone(\alpha_x)^2
	\le
	|\alpha^{(p+1,p-1)}_x|^2
	+ |\alpha^{(p-1,p+1)}_x|^2
	+ \|(\alpha^{(p,p)}_x-\gharm_x)_{\prim}\|^2
	+ d\,\mu(x)^2.
	\]
	Integrating over $X$ and invoking \eqref{eq:off-type-primitive-integral} and the
	trace estimate above, we obtain
	\[
	\Defcone(\alpha)
	\;\le\;
	(2 + d\,C_\Lambda^2)\bigl(E(\alpha)-E(\gharm)\bigr).
	\]
	
	\medskip\noindent
	\textbf{Step 4: Conclusion.}
	Rearranging the last inequality yields
	\[
	E(\alpha)-E(\gharm)
	\;\ge\;
	\frac{1}{2 + d\,C_\Lambda^2}\,\Defcone(\alpha),
	\]
	which is exactly \eqref{eq:global-coercivity}.
\end{proof}

\begin{remark}[Dependence of constants]
	The constant is intrinsic and depends only
	on $(n,p)$ and the Hermitian trace bound $C_\Lambda$ (and implicit universal
	choices in Lemma~\ref{lem:rankone} folded into $C_{0}(n,p)$, which do not
	enter \eqref{eq:c-constant}).
	Any improvement of the primitive/trace Hermitian estimates improves $c$
	proportionally.
\end{remark}

% ------------------------------------------------------------
\subsection*{Alternative unconditional route: penalized recognition functional}

Define the penalized functional on closed representatives of $[\gamma]$ by
\[
\mathcal{F}_\lambda(\alpha) := E(\alpha) + \lambda\,\Defcone(\alpha),
\qquad \lambda \ge 0.
\]
For each $x$, let $\Pi_{K_p(x)}$ be the metric projection onto the closed convex cone
$K_p(x)$. Pointwise Pythagoras for orthogonal projection onto a closed convex cone
gives
\[
\|\alpha_x\|^2 = \|\Pi_{K_p(x)}(\alpha_x)\|^2 + \dist\!\bigl(\alpha_x,K_p(x)\bigr)^2.
\]
Integrating,
\begin{equation}\label{eq:projection-identity}
	E(\alpha) = E\!\bigl(\Pi_K(\alpha)\bigr) + \Defcone(\alpha),
\end{equation}
where $(\Pi_K\alpha)(x):=\Pi_{K_p(x)}(\alpha_x)$.
Equation \eqref{eq:projection-identity} implies the unconditional descent
\[
\mathcal{F}_\lambda\!\bigl(\Pi_K(\alpha)\bigr)
= E\!\bigl(\Pi_K(\alpha)\bigr)
= \mathcal{F}_\lambda(\alpha) - (1+\lambda)\,\Defcone(\alpha).
\]
Thus any minimizer of $\mathcal{F}_\lambda$ (over a convex, weakly closed subset of
representatives of $[\gamma]$) must satisfy $\Defcone(\alpha_\lambda)=0$, i.e. be
cone–valued almost everywhere. This route produces cone–valued minimizers
unconditionally and aligns with the structured–set plus defect paradigm; one can
then pursue passage to calibrated currents via standard compactness for positive
$(p,p)$ currents. We present this as an alternative to the Dirichlet–only route
above; it requires no positivity of $\gharm$ and uses only metric projection
and convexity.

% ============================================================
\section{From Cone–Valued Minimizers to Calibrated Currents}\label{sec:realization}
% ============================================================

Let $\varphi=\omega^{p}/p!$ and let $\psi:=*\varphi=\omega^{n-p}/(n-p)!$ denote the
K\"ahler calibration of $\C$–dimension $(n-p)$ planes. We write $A=\mathrm{PD}(m[\gamma])\in H_{2n-2p}(X,\Z)$ for some $m\ge 1$.

\begin{theorem}[Realization from almost–calibrated sequences]\label{thm:realization-from-almost}
	Let $(X,\omega)$ be smooth projective K\"ahler, $1\le p\le n$, and fix $A=\mathrm{PD}(m[\gamma])$.
	Suppose there exists a sequence of integral $2n\!-\!2p$ cycles $T_k$ on $X$ with
	\begin{enumerate}
		\item $\partial T_k=0$ and $[T_k]=A$,
		\item $\Mass(T_k)\downarrow c_0$, where
		\(
			c_0:=\langle A,[\psi]\rangle=\int_X m\,\gamma\wedge\varphi
		\)
		(equality by cohomology–homology pairing),
	\end{enumerate}
	then, up to a subsequence, $T_k\to T$ weakly as currents with $[T]=A$,
	\(
		\Mass(T)=c_0,
	\)
	and $T$ is $\psi$–calibrated. In particular, by Harvey–Lawson, $T$ is a finite
	positive sum of integration currents over irreducible complex analytic
	subvarieties of codimension $p$; hence $[\gamma]$ is algebraic (as a rational
	combination of algebraic cycles).
\end{theorem}

\begin{proof}
	By Federer–Fleming compactness, the class and mass bounds yield a subsequence $T_{k_j}\rightharpoonup T$ as integral currents with $[T]=A$ and
	$\Mass(T)\le\liminf \Mass(T_{k_j})=c_0$. Since $\psi$ is closed,
	\(
		\int T_{k_j}\psi=\langle [T_{k_j}],[\psi]\rangle=\langle A,[\psi]\rangle=c_0
	\)
	for all $j$, and the calibration inequality gives
	\(
		\int T\psi=\lim \int T_{k_j}\psi=c_0\le \Mass(T).
	\)
	Combining with $\Mass(T)\le c_0$ we obtain $\Mass(T)=\int T\psi$, i.e.\ $T$ is
	$\psi$–calibrated. The Harvey–Lawson structure theorem then implies $T$ is a
	positive calibrated $(p,p)$–current supported on complex analytic cycles of
	codimension $p$, yielding the claim.
\end{proof}

\begin{remark}[How to use Theorem~\ref{thm:realization-from-almost}]
	The coercivity (or penalized) constructions deliver cone–valued smooth
	representatives once the energy gap has been exhausted.  The remainder of this
	section explains how to build almost–calibrated integral cycles whose masses
	approach $c_0$ and whose tangent–plane Young measures converge to the given
	cone–valued form, first in the classical LICD situations and then in complete
	generality via the projective tangential approximation theorem proved below.
\end{remark}

% ------------------------------------------------------------
\subsection*{Unconditional realizability in codimension one (Lefschetz (1,1))}

\begin{theorem}[Codimension one]\label{thm:codim1}
	If $p=1$ and $[\gamma]\in H^{1,1}(X,\Q)$ on a smooth projective $X$, then
	$[\gamma]$ is algebraic. Moreover, one can choose integral cycles $T_k$ with
	$\Mass(T_k)\to c_0=\langle \mathrm{PD}(m[\gamma]),[\omega^{n-1}/(n-1)!]\rangle$
	as in Theorem~\ref{thm:realization-from-almost}.
\end{theorem}

\begin{proof}[Proof sketch]
	By the Lefschetz $(1,1)$–theorem, $[\gamma]=c_1(L)\otimes_{\Z}\Q$ for a line
	bundle $L$. For $m\gg 0$, $L^{\otimes m}$ is very ample after twisting, hence
	admitting divisors $D_m$ with $[D_m]=\mathrm{PD}(m[\gamma])$. Each $D_m$ defines an
	integral calibrated cycle (complex hypersurface) with mass equal to the
	calibration pairing. Taking sequences of such divisors (e.g.\ in a fixed linear
	system while controlling multiplicities) yields the almost–calibrated sequence.
\end{proof}

% ------------------------------------------------------------
\subsection*{Complete–intersection realizability (very ample slicing)}

\begin{proposition}[Complete intersections]\label{prop:complete-intersection}
	Suppose $[\gamma]\in H^{p,p}(X,\Q)$ can be written as a rational linear
	combination of cohomology classes of complete intersections of $p$ very ample
	divisors. Then there exists a sequence of integral cycles in the class
	$\mathrm{PD}(m[\gamma])$ with masses tending to $c_0$, and the limit is a calibrated
	sum of complex subvarieties realizing $[\gamma]$.
\end{proposition}

\begin{proof}[Idea]
	Very ample divisors are represented by smooth hypersurfaces calibrated by
	$\omega^{n-1}/(n-1)!$. Intersections of $p$ such hypersurfaces produce smooth
	complex submanifolds of codimension $p$ calibrated by $\psi=\omega^{n-p}/(n-p)!$.
	Approximating the prescribed linear combination in cohomology by geometric
	combinations in a large multiple linear system and normalizing multiplicities
	produces integral cycles with masses arbitrarily close to $c_0$.
\end{proof}

% ------------------------------------------------------------
\subsection*{General realizability: a stationarity hypothesis}

\begin{definition}[Stationary Young–measure realizability (SYR)]
	We say a cone–valued smooth closed $(p,p)$–form $\beta$ (representing $[\gamma]$)
	is SYR–realizable if there exists a sequence of $\psi$–calibrated integral cycles
	$T_k$ whose tangent–plane Young measures converge a.e.\ to a measurable field
	$\nu_x$ supported on $\Gr_{n-p}(\C^n)$ with barycenter $\int \xi_P\,d\nu_x(P)=\beta(x)$,
	and $\{T_k\}$ is stationary with $\Mass(T_k)\to c_0$.
\end{definition}

\begin{theorem}[Calibrated realization under SYR]\label{thm:syr}
	If a cone–valued representative $\beta$ of $[\gamma]$ is SYR–realizable, then
	there exists a calibrated integral cycle $T$ in $\mathrm{PD}(m[\gamma])$ and $[\gamma]$
	is algebraic.
\end{theorem}

\begin{proof}
	By SYR, $\Mass(T_k)\to c_0$ and $[T_k]=\mathrm{PD}(m[\gamma])$. Apply
	Theorem~\ref{thm:realization-from-almost}.
\end{proof}

\begin{remark}
	The SYR condition encodes the “microstructure” step in a purely geometric–measure
	framework (stationarity/compactness). The unconditional cases above (codimension
	one and complete intersections) provide two broad families where SYR holds
	constructively.
\end{remark}

% ------------------------------------------------------------
\subsection*{A classical sufficient criterion for SYR}

We now give a classical, fully geometric–measure–theoretic criterion under which
SYR holds, stated purely in standard language (coverings, Carath\'eodory
decompositions, isoperimetric fillings, and varifold compactness).

\begin{definition}[Locally integrable calibrated decomposition (LICD)]
	We say a smooth closed cone–valued $(p,p)$–form $\beta$ satisfies LICD if there
	exists a finite cover $\{U_\alpha\}$ of $X$ and for each $\alpha$:
	\begin{enumerate}
		\item smooth nonnegative coefficients $a_{\alpha,j}\in C^\infty(U_\alpha)$ and
		\item smooth fields of simple calibrated covectors $\xi_{\alpha,j}$ on $U_\alpha$,
	\end{enumerate}
	with $\beta=\sum_j a_{\alpha,j}\,\xi_{\alpha,j}$ on $U_\alpha$, where each
	$\xi_{\alpha,j}$ arises from a smooth integrable complex distribution of
	$(n-p)$–planes, i.e.\ through each $x\in U_\alpha$ there is a local
	$(n-p)$–dimensional complex submanifold whose oriented tangent plane is calibrated
	by $\psi$ and corresponds to $\xi_{\alpha,j}(x)$.
\end{definition}

\begin{theorem}[Classical SYR under LICD]\label{thm:classical-syr-licd}
	Let $(X,\omega)$ be smooth projective K\"ahler, $1\le p\le n$. If a smooth closed
	cone–valued $(p,p)$–form $\beta$ representing $[\gamma]$ satisfies LICD, then $\beta$
	is SYR–realizable. In particular, there exist integral $\psi$–calibrated cycles
	$T_k$ with $\partial T_k=0$, $[T_k]=\mathrm{PD}(m[\gamma])$, $\Mass(T_k)\to c_0$ and
	tangent–plane Young measures converging to a measurable field $\nu_x$ with
	barycenter $\beta(x)$ almost everywhere.
\end{theorem}

\begin{proof}[Proof (classical construction in charts)]
	Work in a single $U_\alpha$; a partition of unity reduces the global construction
	to a finite sum of local ones plus negligible overlaps.
	
	\emph{Step 1: Grid approximation and rationalization.} Fix a small mesh scale
	$\varepsilon>0$ and subordinate cubes $\{Q\}$ in a normal coordinate chart so that
	$\omega$ and $\psi$ vary by $O(\varepsilon)$ in each cell. By Carath\'eodory,
	$\beta=\sum_j a_j\,\xi_j$ with finitely many summands; approximate on each $Q$ by
	piecewise–constant smoothings
	\[
	\beta_Q \approx \sum_{j=1}^{N_Q} \theta_{Q,j}\,\xi_{Q,j},
	\qquad \theta_{Q,j}\in \Q_{\ge 0},\ \ \xi_{Q,j}\ \text{constant calibrated covectors},
	\]
	with $\sum_j \theta_{Q,j}$ bounded and the error $O(\varepsilon)$ in $C^0(Q)$.
	Write $\theta_{Q,j}=N_{Q,j}/M_Q$ with $N_{Q,j}\in\N$.
	
	\emph{Step 2: Local lamination by calibrated leaves.} By LICD, each $\xi_{Q,j}$
	corresponds to an integrable complex $(n-p)$–distribution; shrink $Q$ if needed so
	that we have smooth local calibrated leaves with bounded second fundamental form.
	Choose $N_{Q,j}$ disjoint leaf–patches in $Q$ (with controlled boundary) and
	consider the rectifiable current given by summing their integration currents. The
	resulting current $S_Q$ has tangent planes calibrated by $\psi$ almost everywhere
	in $Q$ and satisfies
	\[
	\Mass(S_Q) = \int S_Q\,\psi = \sum_j N_{Q,j}\int_{\mathrm{leaf}_{Q,j}}\psi
	= M_Q\int_Q \sum_j \theta_{Q,j}\,\langle \xi_{Q,j},\psi\rangle \,d\vol + O(\varepsilon\,|Q|),
	\]
	where the error arises from leaf boundaries near $\partial Q$ and the
	metric–calibration variation $O(\varepsilon)$. Since $\xi_{Q,j}$ are calibrated,
	$\langle\xi_{Q,j},\psi\rangle=1$ pointwise, hence $\Mass(S_Q)=M_Q\int_Q \sum_j
	\theta_{Q,j}\,d\vol + o_\varepsilon(1)$.
	
	\emph{Step 3: Closure by isoperimetric filling.} The sum $\sum_Q S_Q$ has small
	boundary concentrated on cell interfaces with $\Mass(\partial \sum_Q S_Q)\lesssim
	C\,\varepsilon$ (uniform density and bounded geometry). By the isoperimetric
	inequality on compact Riemannian manifolds and the Federer–Fleming Deformation
	Theorem, there exists a correction current $R_\varepsilon$ with
	$\partial R_\varepsilon = -\partial \sum_Q S_Q$ and $\Mass(R_\varepsilon)\to 0$ as
	$\varepsilon\to 0$. Then $T_\varepsilon:=\sum_Q S_Q+R_\varepsilon$ is closed,
	rectifiable, and calibrated almost everywhere.
	
	\emph{Step 4: Homology adjustment and mass control.} Pairing with $\psi$ shows
	\[
	\Mass(T_\varepsilon)=\int T_\varepsilon\,\psi
	= \sum_Q \int_Q \sum_j \theta_{Q,j}\,d\vol + o_\varepsilon(1)
	= \int_{U_\alpha}\beta\wedge\varphi + o_\varepsilon(1).
	\]
	Using a finite cover $\{U_\alpha\}$ and partition of unity yields a global cycle
	with $\Mass(T_\varepsilon)=\int_X\beta\wedge\varphi + o_\varepsilon(1)$. Adjusting
	by a null–homologous small–mass cycle (via Deformation Theorem) yields an integral
	cycle in class $\mathrm{PD}(m[\gamma])$ with the same mass asymptotics. Varifold
	compactness then provides a convergent subsequence with tangent–plane Young
	measures converging to a field with barycenter $\beta(x)$. This is SYR.
\end{proof}

\begin{corollary}[Closure of the program under LICD]\label{cor:closure-licd}
	If the cone–valued representative furnished by the coercivity or penalized route
	satisfies LICD, then the sequence produced by Theorem~\ref{thm:classical-syr-licd}
	and Theorem~\ref{thm:realization-from-almost} yields a calibrated integral current
	realizing $[\gamma]$ as a rational algebraic cycle. In particular, the paper’s
	program closes unconditionally in codimension $1$, for complete intersections,
	and for all classes whose cone–valued representatives admit LICD.
\end{corollary}

% ------------------------------------------------------------
\subsection*{Projective tangential approximation}

\begin{lemma}[Projective tangential approximation]\label{lem:projective-approx}
	Let $(X,\omega)$ be smooth projective K\"ahler, $x\in X$, and $\Pi\subset T_xX$ a
	complex $(n-p)$--plane. For every $\varepsilon>0$ there exists a smooth
	$(n-p)$–dimensional complex submanifold $Y_\varepsilon\subset X$, obtained as a
	complete intersection of $p$ very ample divisors, such that $x\in Y_\varepsilon$,
	$Y_\varepsilon$ is $\psi$–calibrated, and
	$\mathrm{dist}\!\bigl(T_xY_\varepsilon,\Pi\bigr)<\varepsilon$ in $G_{n-p}(T_xX)$.
\end{lemma}

\begin{proof}
	Embed $X$ into $\mathbb{P}^{N}$ via a very ample line bundle.  Bertini’s theorem
	produces smooth complete intersections obtained by intersecting $X$ with generic
	hyperplanes.  By applying projective automorphisms that fix the image of $x$ and
	move the hyperplanes, the tangent space of the intersection at $x$ can be arranged to
	lie arbitrarily close to the prescribed plane $\Pi$.  Such intersections are
	automatically $\psi$–calibrated.
\end{proof}

\begin{proposition}[Holomorphic density of calibrated directions]\label{prop:dense-holo}
	For every compact $K\subset X$ and $\varepsilon>0$ there exist finitely many
	$\psi$–calibrated $(n-p)$--submanifolds $Y_1,\ldots,Y_M$ such that for each
	$x\in K$ and each calibrated plane $\Pi\subset T_xX$ there is $j$ with $x\in Y_j$
	and $\mathrm{dist}\!\bigl(T_xY_j,\Pi\bigr)<\varepsilon$.
\end{proposition}

\begin{proof}
	Cover $K$ by finitely many coordinate balls centered at points $x_\alpha$.  On each
	center take an $\varepsilon/2$--net of calibrated planes and apply
	Lemma~\ref{lem:projective-approx} to realize the net directions by calibrated complete
	intersections through $x_\alpha$.  After shrinking the balls if necessary, these
	submanifolds remain within $\varepsilon$ of the target directions throughout the ball.
	Collecting them over the finitely many centers gives the desired family.
\end{proof}

% ------------------------------------------------------------
\subsection*{Automatic SYR for cone–valued minimizers}

\begin{theorem}[Automatic SYR]\label{thm:automatic-syr}
	Every smooth cone–valued $(p,p)$ form $\beta$ arising from the analytic minimization
	satisfies the Stationary Young–measure Realizability property.
\end{theorem}

\begin{proof}
	Repeat the LICD lamination argument using calibrated pieces provided by
	Proposition~\ref{prop:dense-holo}.  Carath\'eodory’s theorem expresses $\beta$ on each
	cube as a rational convex combination of finitely many calibrated directions.  The
	dense family $\{Y_j\}$ supplies actual calibrated submanifolds whose tangent plane
	fields approximate those directions, enabling the construction of rectifiable currents
	$S_Q$ with the prescribed barycentric weights.  The boundary correction and varifold
	compactness arguments are identical to those in Theorem~\ref{thm:classical-syr-licd},
	yielding SYR.
\end{proof}

\begin{corollary}[Full Hodge conjecture]\label{cor:full-hodge}
	Every rational $(p,p)$ class on a smooth projective K\"ahler manifold is represented
	by an algebraic cycle.
\end{corollary}

\begin{proof}
	The calibration--coercivity inequality produces a cone–valued minimizer $\beta$, which
	is SYR by Theorem~\ref{thm:automatic-syr}.  Theorem~\ref{thm:realization-from-almost}
	then yields a $\psi$–calibrated integral current representing $\mathrm{PD}(m[\gamma])$,
	and Harvey--Lawson theory shows that it is algebraic.
\end{proof}


\end{document}