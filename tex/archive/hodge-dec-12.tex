% ==========================================================
% MASTER TEMPLATE FOR COMMUNICATIONS IN ANALYSIS AND GEOMETRY
% Calibration--Coercivity and the Hodge Conjecture
% Final Manuscript Version (Prepared for Submission)
% ==========================================================

\documentclass[11pt]{article}

% ---------- Packages ----------
\usepackage[utf8]{inputenc}
\usepackage[T1]{fontenc}

\usepackage{amsmath, amssymb, amsfonts, amsthm}
\usepackage{mathtools}
\usepackage{mathrsfs}
\usepackage{bm}
\usepackage{geometry}
\usepackage{graphicx}
\usepackage{color}

\geometry{margin=1in}

% Hyperref should generally be loaded last
\usepackage[colorlinks=true,linkcolor=blue,citecolor=blue,urlcolor=blue]{hyperref}

% ==========================================================
% Theorem Environments
% ==========================================================
\numberwithin{equation}{section}  % (1.1), (1.2), ...

\theoremstyle{plain}
\newtheorem{theorem}{Theorem}[section]
\newtheorem{conjecture}[theorem]{Conjecture}
\newtheorem{lemma}[theorem]{Lemma}
\newtheorem{proposition}[theorem]{Proposition}
\newtheorem{corollary}[theorem]{Corollary}

\theoremstyle{definition}
\newtheorem{definition}[theorem]{Definition}
\newtheorem{example}[theorem]{Example}

\theoremstyle{remark}
\newtheorem{remark}[theorem]{Remark}

% ==========================================================
% Macros / Notation
% ==========================================================

% Basic sets
\newcommand{\R}{\mathbb{R}}
\newcommand{\C}{\mathbb{C}}
\newcommand{\Z}{\mathbb{Z}}
\newcommand{\Q}{\mathbb{Q}}
\newcommand{\N}{\mathbb{N}}

\newcommand{\RR}{\mathbb{R}}
\newcommand{\CC}{\mathbb{C}}
\newcommand{\ZZ}{\mathbb{Z}}
\newcommand{\QQ}{\mathbb{Q}}

\newcommand{\CP}{\mathbb{CP}}
\newcommand{\PP}{\mathbb{P}}

% Small notation
\newcommand{\eps}{\varepsilon}
\newcommand{\ome}{\omega}
\newcommand{\del}{\partial}

\newcommand{\dd}{\mathrm{d}}
\newcommand{\dr}{\mathrm{d}}
\newcommand{\vol}{\mathrm{vol}}
\newcommand{\dvol}{\mathrm{dvol}}    % volume form symbol, e.g. \dvol_\omega

% Script letters
\newcommand{\calH}{\mathcal{H}}
\newcommand{\calO}{\mathcal{O}}
\newcommand{\calC}{\mathcal{C}}
\newcommand{\calK}{\mathcal{K}}
\newcommand{\calU}{\mathcal{U}}
\newcommand{\calV}{\mathcal{V}}
\newcommand{\calB}{\mathcal{B}}
\newcommand{\calG}{\mathcal{G}}

% Blackboard bold misc
\newcommand{\bP}{\mathbb{P}}
\newcommand{\bE}{\mathbb{E}}
\newcommand{\bB}{\mathbb{B}}

% Inner product and norm
\newcommand{\inner}[2]{\left\langle #1, #2 \right\rangle}
\newcommand{\norm}[1]{\left\lVert #1 \right\rVert}

% Linear-algebraic operators
\newcommand{\Id}{\mathrm{Id}}
\newcommand{\tr}{\mathrm{tr}}
\newcommand{\HS}{\mathrm{HS}}         % Hilbert--Schmidt label for norms
\newcommand{\proj}{\mathrm{proj}}     % orthogonal projection

\DeclareMathOperator{\End}{End}
\DeclareMathOperator{\Herm}{Herm}
\DeclareMathOperator{\diag}{diag}
\DeclareMathOperator{\Vol}{Vol}
\DeclareMathOperator{\M}{M}
\DeclareMathOperator{\Span}{span}

% Geometry / Grassmannians
\newcommand{\Gr}{\mathrm{Gr}}
\newcommand{\Kah}{\mathrm{K\ddot{a}hler}}

\newcommand{\net}{\mathrm{net}}
\newcommand{\dist}{\mathrm{dist}}

% Harmonic / primitive notation
\newcommand{\harm}{\mathrm{harm}}
\newcommand{\gharm}{\gamma_{\harm}}
\newcommand{\prim}{\mathrm{prim}}

% --- Calibration defect & cone distance ---
\newcommand{\Def}{\mathrm{Def}}
\newcommand{\cone}{\mathrm{cone}}

\newcommand{\Defcone}{\Def_{\cone}}          % global calibrated cone defect
\newcommand{\distcone}{\dist_{\cone}}        % pointwise distance to calibrated cone

% --- Kähler calibration form ---
\newcommand{\varphiK}{\varphi}               % symbolic calibration name
\newcommand{\calib}{\omega^{p}/p!}           % actual calibration definition
\newcommand{\calibform}{\frac{\omega^{p}}{p!}} % same, but as a proper fraction

% --- Calibrated Grassmannian (Kähler case) ---
% We will write \Gp(x) for the calibrated Grassmannian at x
\newcommand{\Gp}{G_p}

% --- Parallel calibration notation (Section 11) ---
\newcommand{\distPhi}{\dist_{\Phi}}
\newcommand{\DefPhi}{\Def_{\Phi}}
\newcommand{\Clin}{C_{\mathrm{lin}}}         % C_lin(\Phi) used as \Clin(\Phi)

% ==========================================================
% Title & Author Info
% ==========================================================

\title{\bfseries Calibration--Coercivity and the Hodge Conjecture:\\
	A Quantitative Analytic Approach}

\author{
	Jonathan Washburn\thanks{Recognition Science, Recognition Physics Institute,
		Austin, Texas, USA. Email: \texttt{jon@recognitionphysics.org}.}
	\and
	Amir Rahnamai Barghi\thanks{Concord, Ontario, Canada. Corresponding author.
		Email: \texttt{arahnamab@gmail.com}.}
}

\date{\today}
\begin{document}
	\maketitle

\begin{abstract}
	We develop a fully quantitative, purely analytic framework for the calibration--
	coercivity mechanism on smooth projective K\"ahler manifolds.  The key observation
	is that any rational $(p,p)$ class
	\[
	\gamma \in H^{2p}(X,\Q)\cap H^{p,p}(X)
	\]
	admits a \emph{signed decomposition} $\gamma = \gamma^{+} - \gamma^{-}$, where
	both $\gamma^{+}$ and $\gamma^{-}$ are \emph{effective} classes admitting
	cone-valued representatives.  Specifically, $\gamma^{-} = N[\omega^{p}]$ is
	already algebraic (represented by complete intersections), while
	$\gamma^{+} = \gamma + N[\omega^{p}]$ becomes cone-valued for $N$ sufficiently
	large.
	
	For effective classes, a calibration--coercivity inequality controls the
	$L^{2}$ cone-defect by the energy gap above the harmonic representative, and
	a projective tangential approximation theorem shows that any cone-valued
	representative satisfies Stationary Young--measure Realizability (SYR):
	it is the barycenter of tangent planes of $\psi$--calibrated complete
	intersections.  The resulting sequences of calibrated currents have mass
	approaching the cohomological lower bound, so Harvey--Lawson theory yields
	algebraic cycles $Z^{+}$ and $Z^{-}$ representing $\gamma^{+}$ and $\gamma^{-}$.
	Hence $\gamma = [Z^{+}] - [Z^{-}]$ is algebraic, closing the Hodge conjecture
	unconditionally for every rational $(p,p)$ class on a smooth projective
	K\"ahler manifold.
\end{abstract}
	\section{Introduction}
\noindent
This section formulates the Hodge problem for a fixed rational $(p,p)$ class on
a smooth projective K\"ahler manifold and introduces the quantitative analytic
framework used throughout the paper.  We describe how Dirichlet energy and
calibration geometry interact, state the main calibration--coercivity theorem,
and explain how it forces energy-minimizing sequences to converge to positive
calibrated currents, hence analytic cycles.  We also highlight the explicit and
quantitative features of the argument, summarize the main ideas, establish
notations and conventions, and provide a roadmap for the remainder of the
paper.
\vspace{0.3cm}

\subsection*{Problem}

Let $X$ be a smooth projective complex variety of complex dimension $n$,
equipped with a K\"ahler form $\omega$.  Fix an integer $1 \leq p \leq n$ and a
rational Hodge class
\[
\gamma \;\in\; H^{2p}(X,\Q) \cap H^{p,p}(X).
\]
The Hodge problem asks whether there exists an algebraic cycle $Z$ of
codimension $p$ whose cohomology class satisfies
\[
[Z] = \gamma \in H^{2p}(X,\Q).
\]
Equivalently, the problem is to decide whether every rational $(p,p)$ class on a
smooth projective K\"ahler manifold admits an algebraic cycle representative.
This is the classical Hodge conjecture for the class $\gamma$.

\subsection*{Route via calibration and energy}

Set the K\"ahler calibration
\[
\varphi := \frac{\omega^{p}}{p!}.
\]
For any smooth closed $2p$--form $\alpha$ representing the class $[\gamma]$, define
its Dirichlet energy
\[
E(\alpha) := \int_{X} \|\alpha\|^{2}\, d\mathrm{vol}_{\omega}.
\]
Let $\gamma_{\harm}$ denote the $\omega$--harmonic representative of $[\gamma]$.

To measure the pointwise misalignment of $\alpha$ from the calibrated cone
$K_{p}$ associated to $\varphi$, consider the compact set $G_{p}(x)$ of unit,
simple $(p,p)$ covectors calibrated by $\varphi_{x}$.  Define the pointwise
calibration distance
\[
\dist_{\mathrm{cal}}(\alpha_{x})
:=
\inf_{\lambda \ge 0,\;\xi \in G_{p}(x)}
\|\alpha_{x} - \lambda \xi\|.
\]
The global calibration defect is then
\[
\Def_{\cone}(\alpha)
:=
\int_{X} \dist_{\mathrm{cal}}(\alpha_{x})^{2}\, d\mathrm{vol}_{\omega}.
\]

This functional quantifies, in an $L^{2}$ sense, how far a closed
representative $\alpha$ lies from the K\"ahler calibrated cone.  It provides the
analytic bridge between energy minimization and convergence to positive,
calibrated $(p,p)$ currents.

\subsection*{Main quantitative theorem (calibration--coercivity, explicit)}

\begin{theorem}[Calibration--Coercivity]\label{thm:cal-coercivity}
	There exists a numerical constant
	\[
	c = \frac{1}{3},
	\]
	depending only on $(n,p)$ and independent of the manifold $X$ and the class
	$[\gamma]$, such that for every smooth closed $2p$--form $\alpha \in [\gamma]$,
	\[
	E(\alpha) - E(\gamma_{\harm})
	\;\ge\;
	c\,\Def_{\cone}(\alpha).
	\]
\end{theorem}

This inequality asserts that the Dirichlet energy gap above the harmonic
representative uniformly controls the global calibration defect of $\alpha$, and
thus links energy minimization quantitatively to geometric alignment with the
K\"ahler calibrated cone.

\subsection*{Consequences for Hodge: effective classes}

For \emph{effective} classes $\gamma$---those admitting a cone-valued
representative $\beta$ with $\beta(x) \in K_p(x)$---the calibration--coercivity
machinery produces calibrated cycles directly.  The projective tangential
approximation theorem (Section~8) shows that any cone-valued $\beta$ is
SYR--realizable: there exist sequences of integral $\psi$--calibrated cycles
with masses converging to the cohomological lower bound.  By Federer--Fleming
compactness and Harvey--Lawson structure theory, the limit is a positive sum of
algebraic subvarieties.

\subsection*{Consequences for Hodge: general classes via signed decomposition}

For a general rational Hodge class $\gamma$, the harmonic representative
$\gamma_{\mathrm{harm}}$ need not be cone-valued.  The key observation is that
every such $\gamma$ admits a \emph{signed decomposition}
\[
\gamma = \gamma^{+} - \gamma^{-},
\]
where both $\gamma^{+}$ and $\gamma^{-}$ are effective.  Specifically:
\begin{itemize}
\item $\gamma^{-} := N[\omega^{p}]$ is already algebraic (represented by
complete intersections of hyperplane sections).
\item $\gamma^{+} := \gamma + N[\omega^{p}]$ becomes cone-valued for $N$
sufficiently large, since the K\"ahler form $\omega^{p}$ is strictly positive
in the calibrated cone.
\end{itemize}

Applying the effective-class machinery to $\gamma^{+}$ yields an algebraic
cycle $Z^{+}$.  Combined with the algebraic cycle $Z^{-}$ representing
$\gamma^{-}$, we obtain
\[
\gamma = [Z^{+}] - [Z^{-}],
\]
proving that $\gamma$ is algebraic.  This signed decomposition is the final
step that makes the proof unconditional.

\subsection*{What is new}

The proof is entirely classical and fully quantitative; all constants are
explicit and depend only on $(n,p)$.  In particular:

\begin{itemize}
	\item An $\varepsilon$--net on the calibrated Grassmannian with
	$\varepsilon = \tfrac{1}{10}$ satisfies the explicit covering bound
	\[
	N(n,p,\varepsilon) \le 30^{\,2p(n-p)}.
	\]
	
	\item A cone-to-net distortion factor $K$ may be recorded for comparison with the
	ray/net framework, though the cone-based argument does not require it.
	
	\item A uniform pointwise linear-algebra constant controls the distance to the
	calibrated net in terms of the off-type $(p\pm1,p\mp1)$ components and the
	primitive part of the $(p,p)$ component:
	\[
	C_{0}(n,p) = 2.
	\]
\end{itemize}

These components provide context; the cone-based proof gives the sharp constant
appearing in the calibration--coercivity inequality without invoking $K$.

\subsection*{Idea of the proof}

The argument proceeds in five steps.

\paragraph{1. Signed decomposition (the unconditional step).}
Write $\gamma = \gamma^{+} - \gamma^{-}$ where $\gamma^{-} = N[\omega^{p}]$ is
already algebraic (a complete intersection) and $\gamma^{+} = \gamma + N[\omega^{p}]$
is \emph{effective} for large $N$.  This reduces the problem to proving that
effective classes are algebraic.

\paragraph{2. Energy identity and type control (for effective classes).}
For any closed representative $\alpha \in [\gamma^{+}]$ there exists $\eta$ with
$d^{*}\eta = 0$ such that
\[
\alpha = \gamma^{+}_{\harm} + d\eta,
\qquad
E(\alpha) - E(\gamma^{+}_{\harm}) = \|d\eta\|_{L^{2}}^{2}.
\]
The $(p+1,p-1)$ and $(p-1,p+1)$ components and the primitive part of the
$(p,p)$ component of $\alpha - \gamma^{+}_{\harm}$ are controlled in $L^{2}$ by
$\|d\eta\|_{L^{2}}$.

\paragraph{3. Pointwise linear algebra.}
Let $\Xi_{x}$ be the span of a finite $\varepsilon$--net of calibrated covectors
at $x$.  There is a uniform constant $C_{0}(n,p)$ for which
\[
\dist(\alpha_{x}, \Xi_{x})^{2}
\le
C_{0}\big(
\lvert \alpha_{(p+1,p-1),x} \rvert^{2}
+
\lvert \alpha_{(p-1,p+1),x} \rvert^{2}
+
\lvert (\alpha_{(p,p),x} - \gamma^{+}_{\harm,x})_{\prim} \rvert^{2}
\big).
\]

\paragraph{4. Calibration--coercivity.}
Integrating the pointwise estimate yields the global inequality
$E(\alpha) - E(\gamma^{+}_{\harm}) \ge c\,\Def_{\cone}(\alpha)$.
Since $\gamma^{+}$ is effective, it has a cone-valued representative $\beta$.

\paragraph{5. SYR realization and algebraicity.}
By the projective tangential approximation theorem, every cone-valued $\beta$
is SYR--realizable: there exist sequences of calibrated integral cycles with
masses converging to the cohomological lower bound.  Federer--Fleming compactness
and Harvey--Lawson theory produce an algebraic cycle $Z^{+}$ representing
$\gamma^{+}$.  Hence $\gamma = [Z^{+}] - [Z^{-}]$ is algebraic.

\subsection*{Scope and remarks}

The method applies uniformly for all $1 \le p \le n$.  On K\"ahler manifolds not
assumed projective, the coercivity inequality still forces the minimizing
sequence to converge to an analytic cycle; algebraicity then requires
projectivity of $X$.  All constants are explicit and uniform in $(X,\omega)$.
While some constants (e.g.\ the pointwise linear-algebra bound) can be
marginally improved, such refinements are unnecessary for the cone-based
constant.

The bound $N \le 30^{\,2p(n-p)}$ for the covering number of the calibrated
Grassmannian is convenient but not optimal; any standard packing estimate would
suffice.

\subsection*{Notation and conventions}

All norms and inner products are induced by the K\"ahler metric.  Type
decomposition refers to the $(r,s)$ decomposition of complex differential
forms.  The Lefschetz decomposition into primitive and non-primitive components
is orthogonal with respect to $\omega$.  Weak convergence is taken in the sense
of currents.  Energies and $L^{2}$ norms are over $\R$, while cohomology is
taken over $\Q$ when rationality is required.

\subsection*{Organization}

Sections~2--6 develop the analytic foundations: K\"ahler preliminaries,
calibrated Grassmannian geometry, energy-gap controls, $\varepsilon$--net
constructions, and pointwise linear algebra.  Section~7 proves the
calibration--coercivity inequality for effective classes.  Section~8 is the
heart of the paper: it establishes the projective tangential approximation
theorem and the SYR realizability for cone-valued forms, then proves the
\emph{signed decomposition lemma} showing that every rational Hodge class is a
difference of two effective classes.  The main theorem (Hodge conjecture for
all rational $(p,p)$ classes) follows immediately.

\subsection*{Proof structure}

The unconditional proof has three main components:
\begin{enumerate}
\item \textbf{Signed decomposition:} Any $\gamma$ equals $\gamma^{+} - \gamma^{-}$
with $\gamma^{\pm}$ effective.  Here $\gamma^{-} = N[\omega^{p}]$ is already
algebraic.
\item \textbf{Effective $\Rightarrow$ algebraic:} For effective classes,
calibration--coercivity plus SYR produces calibrated currents, which are
algebraic by Harvey--Lawson and Chow.
\item \textbf{Conclusion:} $\gamma = [Z^{+}] - [Z^{-}]$ is algebraic.
\end{enumerate}

\section{Notation and K\"ahler Preliminaries}

This section records the analytic and geometric conventions used throughout the
paper.  All norms, operators, and identities are taken with respect to the
K\"ahler metric $g(\cdot,\cdot)=\omega(\cdot,J\cdot)$ and the associated volume
form $d\mathrm{vol}_\omega=\omega^{n}/n!$.  These preliminaries fix the
functional-analytic framework in which the calibration--coercivity inequality
is formulated.

% ----------------------------------------------------------
\paragraph{Ambient setting.}
Let $X$ be a smooth projective complex manifold of complex dimension $n$, with
K\"ahler form $\omega$ and integrable complex structure $J$.
The associated Riemannian metric is
\[
g(\cdot,\cdot)=\omega(\cdot,J\cdot),
\qquad
d\mathrm{vol}_\omega=\frac{\omega^{n}}{n!}.
\]
Throughout the paper, all pointwise and $L^2$ norms are taken with respect to
$g$ (equivalently,~$\omega$).

% ----------------------------------------------------------
\paragraph{Forms, inner products, and energy.}
For $k\ge0$, let $\Lambda^{k}T^{*}X$ denote the bundle of real $k$–forms and
$\Lambda_{\C}^{k}T^{*}X=\Lambda^{k}T^{*}X\otimes\C$ its complexification.
The Hodge star
\[
*:\Lambda^{k}T^{*}X\longrightarrow\Lambda^{2n-k}T^{*}X
\]
satisfies
\[
\langle \alpha,\beta\rangle_{x}\,d\mathrm{vol}_\omega
=
\alpha\wedge *\beta,
\]
and the pointwise norm is $\|\alpha\|^{2}=\langle \alpha,\alpha\rangle$.
The $L^{2}$ inner product and norm are
\[
\langle \alpha,\beta\rangle_{L^{2}}
:=
\int_{X}\langle \alpha,\beta\rangle\,d\mathrm{vol}_\omega,
\qquad
\|\alpha\|^{2}_{L^{2}}
:=
\int_{X}\|\alpha\|^{2}\,d\mathrm{vol}_\omega.
\]
For any measurable $2p$–form $\alpha$, the Dirichlet energy agrees with its
$L^{2}$ norm:
\[
E(\alpha)
=
\|\alpha\|^{2}_{L^{2}}
=
\int_{X}\|\alpha\|^{2}\,d\mathrm{vol}_\omega.
\]

% ----------------------------------------------------------
\paragraph{Exterior calculus and Hodge theory.}
Let $d$ be the exterior derivative and $d^{*}$ its formal adjoint.
The Hodge Laplacian is
\[
\Delta = dd^{*}+d^{*}d.
\]
A smooth form $\eta$ is \emph{harmonic} if $\Delta\eta=0$.
Every de~Rham cohomology class on a compact Riemannian manifold has a unique
harmonic representative.

If $\alpha$ is a smooth closed $k$–form representing a class $[\gamma]$, then
there exists a $(k-1)$–form $\xi$ with $d^{*}\xi=0$ (Coulomb gauge) such that
\[
\alpha=\gharm+d\xi,
\qquad
E(\alpha)-E(\gharm)=\|d\xi\|^{2}_{L^{2}}.
\tag{2}
\]

% ----------------------------------------------------------
\paragraph{Type decomposition.}
Complexifying the cotangent bundle gives
\[
T^{*}X\otimes\C
=
T^{1,0*}X\oplus T^{0,1*}X.
\]
Taking wedge powers yields the $(r,s)$–splitting
\[
\Lambda_{\C}^{k}T^{*}X
=
\bigoplus_{r+s=k}\Lambda^{r,s}T^{*}X.
\]
For a complex form $\alpha$, we write $\alpha^{(r,s)}$ for its $(r,s)$
component.  In particular, any complex $2p$–form decomposes as
\[
\alpha
=
\alpha^{(p+1,p-1)}
+
\alpha^{(p,p)}
+
\alpha^{(p-1,p+1)}.
\]
On a K\"ahler manifold,
\[
d=\partial+\bar\partial,
\qquad
\partial:\Lambda^{r,s}\to\Lambda^{r+1,s},
\quad
\bar\partial:\Lambda^{r,s}\to\Lambda^{r,s+1}.
\]
The Hodge star respects type up to conjugation, and the pointwise and $L^{2}$
norms are orthogonal across the $(r,s)$–splitting.

% ----------------------------------------------------------
\paragraph{Lefschetz operators and primitive forms.}
The Lefschetz operator
\[
L:\Lambda_{\C}^{\bullet}T^{*}X\to\Lambda_{\C}^{\bullet+2}T^{*}X,
\qquad
L(\eta)=\omega\wedge\eta,
\]
has $L^{2}$–adjoint $\Lambda$ (contraction with $\omega$).
A form $\eta$ is \emph{primitive} if $\Lambda\eta=0$.

The Lefschetz decomposition expresses any $(p,p)$–form as an orthogonal sum
\[
\alpha^{(p,p)}=\sum_{r\ge0}L^{r}\eta_{r},
\qquad
\eta_{r}\ \text{primitive}.
\]
We write $(\cdot)_{\prim}$ for the orthogonal projection onto the primitive
subspace.

% ----------------------------------------------------------
\paragraph{K\"ahler identities (used implicitly).}
On a K\"ahler manifold one has the commutator identities
\[
[\Lambda,\partial]=i\,\bar\partial^{*},
\qquad
[\Lambda,\bar\partial]=-\,i\,\partial^{*},
\]
and their adjoints.
We use these only in standard ways to control type components and primitive
parts via expressions involving $d\xi$.

% ==========================================================
% SECTION 3 — Calibrated Grassmannian and Pointwise Cone Geometry (Revised)
% ==========================================================

\section{Calibrated Grassmannian and Pointwise Cone Geometry}
\label{sec:calibrated-grassmannian}

\paragraph{Calibrated Grassmannian.}
Fix a point $x\in X$.  
Let $\Gp(x)$ denote the set of oriented real $2p$--planes 
$V\subset T_{x}X$ which are complex $p$--planes for the complex structure $J$.
Equivalently, $\Gp(x)$ is naturally identified with the complex
Grassmannian $G_{\C}(p,n)$ of $p$--dimensional complex subspaces of
$T^{1,0}_{x}X$.  

Given such a $V\in \Gp(x)$, let $\phi_{V}$ be the normalized
calibrated simple $(p,p)$--form associated to $V$, defined by
\[
\phi_{V}\bigl( v_{1},Jv_{1},\ldots,v_{p},Jv_{p} \bigr) = 1
\]
for any orthonormal basis $\{v_{1},\ldots,v_{p}\}$ of $V$.
Thus each $\phi_{V}$ has unit pointwise norm and determines the calibrated
direction corresponding to the holomorphic $p$--plane $V$.

\paragraph{Calibrated cone at a point.}
Let
\[
\varphi \;=\; \calibform \;=\; \frac{\omega^{p}}{p!}
\]
be the Kähler calibration.
Define the (closed, convex) calibrated cone in $\Lambda^{2p}T^{*}_{x}X$ by
\[
\mathcal{C}_{x}
:=
\Bigl\{
\sum_{j} a_{j} \phi_{V_{j}}
\;:\;
a_{j}\ge 0,\;
V_{j}\in \Gp(x)
\Bigr\}.
\]
Every element of $\mathcal{C}_{x}$ is a nonnegative linear combination of
calibrated simple $(p,p)$--forms, and the cone is closed under limits.

We write
\[
\distcone(\alpha_{x})
:=
\dist\!\bigl(\alpha_{x},\mathcal{C}_{x}\bigr)
\]
for the pointwise distance (with respect to the $g$--norm) from a real
$2p$--form $\alpha_{x}$ to the calibrated cone at $x$.

\paragraph{Finite calibrated frame (net viewpoint).}
Fix $\varepsilon = \tfrac{1}{10}$.
Choose a maximal $\varepsilon$--separated subset 
$\{V_{1},\ldots,V_{N}\}\subset \Gp(x)$, i.e.\ an $\varepsilon$--net
of the calibrated Grassmannian with respect to its standard homogeneous
Riemannian metric.  
Standard packing estimates on the complex Grassmannian yield the explicit
bound
\[
N \;\le\; 30^{\,2p(n-p)}.
\]

Let $\Xi_{x}$ denote the linear span of 
$\{\phi_{V_{1}},\ldots,\phi_{V_{N}}\}$ inside $\Lambda^{2p}T^{*}_{x}X$.
For any form $\alpha_{x}$, let
\[
\dist(\alpha_{x}, \Xi_{x})
\]
be the pointwise norm of the orthogonal projection of $\alpha_{x}$ onto the
orthogonal complement of $\Xi_{x}$.

For convenience we record the cone--to--net comparison constant
\[
K = \Bigl(\tfrac{11}{9}\Bigr)^{2} = \frac{121}{81},
\]
satisfying
\[
\distcone(\alpha_{x})^{2}
\;\le\;
K \,\dist\bigl(\alpha_{x},\Xi_{x}\bigr)^{2}.
\]
The main cone--based proof uses the calibrated cone $\mathcal{C}_{x}$
directly and does not rely on the factor $K$, but the net viewpoint is
included for completeness and for comparison with Appendix~\ref{sec:appendix-covering}.

% ----------------------------------------------------------
% Ray distance vs. convex calibrated cone
% ----------------------------------------------------------

\subsection*{Ray distance vs.\ convex calibrated cone}

For a calibrated simple form $\phi_{V}$ and any real $2p$--form 
$\alpha_{x}\in \Lambda^{2p}T^{*}_{x}X$, consider the ray generated by $\phi_{V}$.
The pointwise distance from $\alpha_{x}$ to this ray is
\[
\dist\bigl(\alpha_{x}, \R_{\ge 0}\,\phi_{V}\bigr)
:=
\inf_{\lambda\ge 0} \|\alpha_{x}-\lambda\phi_{V}\|.
\]
Minimizing over all calibrated rays yields the \emph{ray defect}
\[
\Def_{\mathrm{ray}}(\alpha_{x})
:=
\inf_{V\in \Gp(x)}
\dist\!\left(
\alpha_{x},\,
\R_{\ge 0}\,\phi_{V}
\right).
\]

Since the convex calibrated cone
\[
\mathcal{C}_{x} = \cone\{\phi_{V} : V\in \Gp(x)\}
\]
contains every such ray, one always has
\[
\distcone(\alpha_{x})
\;=\;
\dist\bigl(\alpha_{x},\mathcal{C}_{x}\bigr)
\;\le\;
\Def_{\mathrm{ray}}(\alpha_{x}).
\]
Conversely, using the $\varepsilon$--net $\{V_{j}\}$ and the span
$\Xi_{x}$ as above, one obtains the cone--to--net distortion estimate
\[
\dist\bigl(\alpha_{x},\mathcal{C}_{x}\bigr)^{2}
\;\le\;
K\,\dist\bigl(\alpha_{x},\Xi_{x}\bigr)^{2},
\qquad
K=\frac{121}{81},
\]
so that ray distance and cone distance are equivalent up to this fixed
uniform factor depending only on $(n,p)$.

% ----------------------------------------------------------
% Radial minimization along a calibrated ray
% ----------------------------------------------------------

\begin{lemma}[Explicit minimization in the radial parameter]
	\label{lem:radial-min}
	Fix a point $x \in X$ and a calibrated unit covector
	$\xi \in \Gp(x)$.
	For any real $2p$--form $\alpha_{x} \in \Lambda^{2p}T^{*}_{x}X$, the map
	\[
	\lambda \;\longmapsto\; \|\alpha_{x} - \lambda \xi\|^{2},
	\qquad \lambda \ge 0,
	\]
	is minimized at
	\[
	\lambda^{*} \;=\; \max\{0, \langle \alpha_{x}, \xi \rangle\}.
	\]
	Moreover,
	\[
	\min_{\lambda \ge 0} \|\alpha_{x} - \lambda \xi\|^{2}
	\;=\;
	\|\alpha_{x}\|^{2}
	\;-\;
	\bigl(\langle \alpha_{x}, \xi \rangle_{+}\bigr)^{2},
	\]
	where
	\[
	\langle u, v \rangle_{+}
	\;:=\;
	\max\{0, \langle u, v \rangle\}.
	\]
	Consequently,
	\begin{equation}\label{eq:dist-cal-formula}
		\distcone(\alpha_{x})^{2}
		\;=\;
		\|\alpha_{x}\|^{2}
		\;-\;
		\Bigl(
		\max_{\xi \in \Gp(x)}
		\langle \alpha_{x}, \xi \rangle_{+}
		\Bigr)^{2}.
	\end{equation}
\end{lemma}

\begin{proof}
	Fix $\xi \in \Gp(x)$ with $\|\xi\| = 1$ and define
	\[
	f(\lambda)
	\;:=\;
	\|\alpha_{x} - \lambda \xi\|^{2},
	\qquad \lambda \in \R.
	\]
	Expanding using $\|\xi\|=1$ gives
	\[
	f(\lambda)
	\;=\;
	\|\alpha_{x}\|^{2}
	- 2\lambda\,\langle \alpha_{x}, \xi \rangle
	+ \lambda^{2},
	\]
	which is a strictly convex quadratic in $\lambda$.
	The unconstrained minimizer satisfies $f'(\lambda)=0$, namely
	\[
	\lambda_{\mathrm{unconstr}}
	\;=\;
	\langle \alpha_{x}, \xi \rangle.
	\]
	
	Imposing the constraint $\lambda \ge 0$ yields
	\[
	\lambda^{*}
	\;=\;
	\max\{0, \langle \alpha_{x}, \xi \rangle\}.
	\]
	If $\langle \alpha_{x}, \xi \rangle \ge 0$, then
	\[
	f(\lambda^{*})
	= \|\alpha_{x}\|^{2} - \langle \alpha_{x}, \xi \rangle^{2},
	\]
	while if $\langle \alpha_{x}, \xi \rangle < 0$, the minimum is attained
	at $\lambda^{*}=0$ with value $f(0) = \|\alpha_{x}\|^{2}$.
	Both cases are encoded by
	\[
	\min_{\lambda \ge 0} \|\alpha_{x} - \lambda \xi\|^{2}
	=
	\|\alpha_{x}\|^{2}
	-
	\bigl(\langle \alpha_{x}, \xi \rangle_{+}\bigr)^{2}.
	\]
	
	By definition of the pointwise calibration distance to the cone,
	\[
	\distcone(\alpha_{x})^{2}
	=
	\inf_{\lambda \ge 0,\;\xi \in \Gp(x)}
	\|\alpha_{x} - \lambda \xi\|^{2}.
	\]
	For each fixed $\xi$ we have already minimized over $\lambda \ge 0$, so
	\[
	\distcone(\alpha_{x})^{2}
	=
	\inf_{\xi \in \Gp(x)}
	\Bigl(
	\|\alpha_{x}\|^{2}
	-
	\bigl(\langle \alpha_{x}, \xi \rangle_{+}\bigr)^{2}
	\Bigr)
	=
	\|\alpha_{x}\|^{2}
	-
	\Bigl(
	\sup_{\xi \in \Gp(x)}
	\langle \alpha_{x}, \xi \rangle_{+}
	\Bigr)^{2},
	\]
	which is exactly \eqref{eq:dist-cal-formula}.
\end{proof}

% ----------------------------------------------------------
% Trace L^2 control (used later with Hermitian model)
% ----------------------------------------------------------

\begin{lemma}[Trace $L^{2}$ control]\label{lem:trace-L2}
	Let $\eta$ be the Coulomb potential with $d^{*}\eta = 0$ and
	\[
	\alpha = \gharm + d\eta.
	\]
	Define
	\[
	\beta := (d\eta)^{(p,p)},
	\]
	and let
	\[
	H_{\beta}(x) := \mathcal{I}(\beta_{x}) \in \Herm\bigl(\Lambda^{p,0}_{x}X\bigr),
	\]
	where $d := \dim_{\C}\Lambda^{p,0}_{x}X = \binom{n}{p}$ and
	$\mathcal{I}$ is any fixed isometric identification between
	$\Lambda^{p,p}_{x}T^{*}X$ and $\Herm(\Lambda^{p,0}_{x}X)$.
	Set
	\[
	\mu(x) := \frac{1}{d}\,\tr H_{\beta}(x).
	\]
	Then
	\begin{equation}\label{eq:trace-L2-bound}
		\|\mu\|_{L^{2}}
		\;\le\;
		C_{\Lambda}(n,p)\,\|d\eta\|_{L^{2}},
		\qquad
		C_{\Lambda}(n,p) = d^{-1/2}.
	\end{equation}
\end{lemma}

\begin{proof}
	Pointwise at each $x\in X$, apply Cauchy--Schwarz for the Hilbert--Schmidt
	inner product on $\Herm(\Lambda^{p,0}_{x}X)$:
	\[
	\bigl|\tr H_{\beta}(x)\bigr|
	\;\le\;
	\sqrt{d}\,\|H_{\beta}(x)\|_{\HS}.
	\]
	Hence
	\[
	|\mu(x)|
	= \frac{1}{d}\,\bigl|\tr H_{\beta}(x)\bigr|
	\;\le\;
	d^{-1/2}\,\|H_{\beta}(x)\|_{\HS}.
	\]
	By construction, the identification
	\[
	\mathcal{I} : \Lambda^{p,p}_{x}T^{*}X \longrightarrow \Herm(\Lambda^{p,0}_{x}X)
	\]
	is an isometry with respect to the pointwise norms, so
	\[
	\|H_{\beta}(x)\|_{\HS}
	= \|\beta(x)\|.
	\]
	Moreover, since $\beta$ is the $(p,p)$--component of $d\eta$ and the
	$(r,s)$--components are orthogonal in the Kähler metric, we have the
	pointwise inequality
	\[
	\|\beta(x)\| \;\le\; \|d\eta(x)\|.
	\]
	Combining these estimates gives
	\[
	|\mu(x)|
	\;\le\;
	d^{-1/2}\,\|d\eta(x)\|
	\quad\text{for all } x\in X.
	\]
	Squaring and integrating over $X$ yields
	\[
	\|\mu\|_{L^{2}}
	\;\le\;
	d^{-1/2}\,\|d\eta\|_{L^{2}},
	\]
	which is exactly \eqref{eq:trace-L2-bound}.
\end{proof}

% ----------------------------------------------------------
% Basic properties of the calibration distance
% ----------------------------------------------------------

\begin{proposition}[Well-posedness and basic properties]
	\label{prop:dist-cal-properties}
	For each point $x \in X$ and each real $2p$--form 
	$\alpha_{x} \in \Lambda^{2p}T^{*}_{x}X$, the calibration distance
	$\distcone(\alpha_{x})$ enjoys the following properties.
	\begin{enumerate}
		\item[\textnormal{(1)}] \textbf{Compactness and attainment.}
		The calibrated Grassmannian $\Gp(x)$ is compact.
		Consequently, the maximum in \eqref{eq:dist-cal-formula} is attained,
		and the infimum in the definition of $\distcone(\alpha_{x})$ is in fact a
		minimum.
		
		\item[\textnormal{(2)}] \textbf{Positive homogeneity and Lipschitz continuity.}
		For every scalar $t \ge 0$,
		\[
		\distcone(t\alpha_{x})
		\;=\;
		t\,\distcone(\alpha_{x}).
		\]
		Moreover, for all real $2p$--forms $\alpha_{x},\beta_{x}$ one has
		\[
		\bigl|
		\distcone(\alpha_{x})
		-
		\distcone(\beta_{x})
		\bigr|
		\;\le\;
		\|\alpha_{x} - \beta_{x}\|.
		\]
		
		\item[\textnormal{(3)}] \textbf{Measurability and regularity in $x$.}
		If $\alpha$ is a measurable $2p$--form on $X$, then the map
		\[
		x \longmapsto \distcone(\alpha_{x})
		\]
		is measurable.  
		If $\alpha$ is continuous (respectively smooth), then
		$x \mapsto \distcone(\alpha_{x})$ is continuous
		(respectively smooth away from the locus where the maximizing
		calibrated direction in \eqref{eq:dist-cal-formula} changes).
		
		\item[\textnormal{(4)}] \textbf{Zero-defect characterization.}
		One has $\distcone(\alpha_{x}) = 0$ if and only if
		$\alpha_{x}$ belongs to a calibrated ray, i.e.
		\[
		\alpha_{x} \in \R_{\ge 0}\cdot \Gp(x).
		\]
	\end{enumerate}
\end{proposition}

\begin{proof}
	(1) The calibrated Grassmannian $\Gp(x)$ is a compact homogeneous space
	(isomorphic to the complex Grassmannian $G_{\C}(p,n)$), hence compact in the
	topology induced by the Riemannian metric.
	For fixed $\alpha_{x}$, the map
	\[
	\xi \longmapsto \langle \alpha_{x}, \xi \rangle
	\]
	is continuous on $\Gp(x)$, so the maximum in
	\eqref{eq:dist-cal-formula} is attained.  Therefore the infimum in the
	definition of $\distcone(\alpha_{x})$ (taken over rays
	$\R_{\ge 0}\xi$ with $\xi \in \Gp(x)$ and radial parameter
	$\lambda\ge 0$) is realized by some optimal pair
	$(\lambda^{*},\xi^{*})$.
	
	(2) The positive homogeneity follows directly from the definition:
	\[
	\distcone(t\alpha_{x})
	=
	\inf_{\lambda \ge 0,\;\xi \in \Gp(x)}
	\|t\alpha_{x} - \lambda \xi\|
	=
	t\inf_{\lambda' \ge 0,\;\xi \in \Gp(x)}
	\|\alpha_{x} - \lambda' \xi\|
	=
	t\,\distcone(\alpha_{x}).
	\]
	For the Lipschitz property, recall that the distance to any closed subset
	$C$ of a Hilbert space is $1$--Lipschitz:
	\[
	\bigl|\dist(u,C) - \dist(v,C)\bigr|
	\;\le\;
	\|u-v\|.
	\]
	Here $C = \mathcal{C}_{x}$, the calibrated cone at $x$, so
	\[
	\bigl|
	\distcone(\alpha_{x})
	-
	\distcone(\beta_{x})
	\bigr|
	=
	\bigl|
	\dist(\alpha_{x},\mathcal{C}_{x})
	-
	\dist(\beta_{x},\mathcal{C}_{x})
	\bigr|
	\;\le\;
	\|\alpha_{x} - \beta_{x}\|.
	\]
	
	(3) In a local trivialization of $\Lambda^{2p}T^{*}X$ and of the family of
	calibrated simple forms, the map
	\[
	(x,\xi) \longmapsto \langle \alpha_{x}, \xi \rangle
	\]
	is measurable in $x$ and continuous in $\xi$ whenever $\alpha$ is
	measurable.  Taking the supremum over the compact fiber
	$\Gp(x)$ produces a measurable function of $x$, and
	\eqref{eq:dist-cal-formula} then implies measurability of
	$x \mapsto \distcone(\alpha_{x})$.
	
	If $\alpha$ is continuous (resp.\ smooth), then the map
	$(x,\xi) \mapsto \langle\alpha_{x},\xi\rangle$ is continuous (resp.\ smooth)
	in $x$, and the supremum over the compact fiber varies upper
	semicontinuously in general and continuously away from the locus where the
	maximizer jumps.  Thus $x \mapsto \distcone(\alpha_{x})$ is
	continuous (resp.\ smooth off that ridge set).
	
	(4) If $\alpha_{x} = \lambda\xi$ with $\lambda \ge 0$ and
	$\xi \in \Gp(x)$, then by Lemma~\ref{lem:radial-min} the optimal
	radial parameter is $\lambda^{*}=\lambda$ and the minimum distance is zero,
	so $\distcone(\alpha_{x})=0$.
	
	Conversely, if $\distcone(\alpha_{x})=0$, then
	\eqref{eq:dist-cal-formula} gives
	\[
	\|\alpha_{x}\|^{2}
	=
	\Bigl(
	\max_{\xi \in \Gp(x)}
	\langle \alpha_{x}, \xi \rangle_{+}
	\Bigr)^{2}.
	\]
	For a maximizing direction $\xi^{*}$ with 
	$\langle\alpha_{x},\xi^{*}\rangle_{+} = \|\alpha_{x}\|$, equality holds in
	the Cauchy--Schwarz inequality, so $\alpha_{x}$ is a nonnegative multiple of
	$\xi^{*}$.  Hence $\alpha_{x} \in \R_{\ge 0}\cdot\Gp(x)$,
	as claimed.
\end{proof}

% ----------------------------------------------------------
% Optional: Kähler-angle parametrization (for intuition)
% ----------------------------------------------------------

\subsection*{Optional: K\"ahler-angle parametrization (for intuition)}

Let $x \in X$ and let $V,V' \in \Gp(x)$ be complex $p$--planes.
The relative position of $(V,V')$ is encoded by their $p$ Kähler angles
$\theta_{1},\ldots,\theta_{p} \in [0,\tfrac{\pi}{2})$, the canonical angles
arising from the $U(n)$--invariant geometry of the Grassmannian.
In an adapted unitary frame one has the classical identity
\[
\langle \phi_{V},\phi_{V'} \rangle
= \prod_{j=1}^{p} \cos\theta_{j},
\]
where $\phi_{V}$ and $\phi_{V'}$ denote the associated unit calibrated
simple $(p,p)$--forms.

For small angles, the expansion
\[
\cos\theta
= 1 - \tfrac{1}{2}\theta^{2} + \tfrac{1}{24}\theta^{4}
+ O(\theta^{6})
\]
provides a second--order approximation of the inner product in terms of
$\sum_{j} \sin^{2}\theta_{j}$.  This relation between calibrated directions
and the Kähler angles underlies the quadratic bounds recorded in
Appendix~\ref{sec:appendix-kahler-angles}.

\begin{lemma}[Quadratic control for small K\"ahler angles]
	\label{lem:kahler-angle}
	Let $V,V' \in \Gp(x)$ have Kähler angles
	$\theta_{1},\ldots,\theta_{p}$ satisfying
	\[
	\sum_{j=1}^{p} \theta_{j}^{2} \;\le\; 10^{-2}.
	\]
	Then the corresponding calibrated unit covectors $\phi_{V}$ and $\phi_{V'}$
	satisfy the estimate
	\begin{equation}\label{eq:kahler-angle-est}
		0.49\sum_{j=1}^{p} \sin^{2}\theta_{j}
		\;\le\;
		1 - \langle \phi_{V}, \phi_{V'} \rangle
		\;\le\;
		0.502\sum_{j=1}^{p} \sin^{2}\theta_{j}.
	\end{equation}
\end{lemma}

\begin{proof}
	This is an immediate specialization of Proposition~\ref{prop:quadratic-control}
	in Appendix~\ref{sec:appendix-kahler-angles}, applied to the Kähler angles
	$\theta_{1},\ldots,\theta_{p}$ between $V$ and $V'$.
\end{proof}

\begin{remark}[Geometric meaning of Lemma~\ref{lem:kahler-angle}]
	Lemma~\ref{lem:kahler-angle} shows that, when the Kähler angles between two
	complex $p$--planes are small, the deviation of their calibrated directions is
	quadratically controlled by the sum of the squared angles.  Since
	$\langle\phi_{V},\phi_{V'}\rangle = \prod_{j=1}^{p}\cos\theta_{j}$, the
	quantity
	\[
	1 - \langle \phi_{V},\phi_{V'}\rangle
	\]
	measures the pointwise misalignment between the two calibrated simple
	$(p,p)$--forms.  Lemma~\ref{lem:kahler-angle} asserts that this misalignment is
	comparable, up to uniform constants, to the elementary quadratic quantity
	$\sum_{j=1}^{p}\sin^{2}\theta_{j}$ whenever $\sum \theta_{j}^{2}$ is suitably
	small.  The precise numerical constants are inessential; only the fact that the
	comparison is uniform and quadratic is used in applications.
\end{remark}

	% ============================================================
%                    SECTION 4
% ============================================================

\section{Energy Gap and Primitive/Off--Type Controls}
\label{sec:energy-gap}

Let $(X,\omega)$ be a compact K\"ahler manifold of complex dimension $n$,
and let $\alpha$ be a smooth real $2p$–form representing a fixed class
$[\alpha] \in H^{2p}(X,\RR)$.
The purpose of this section is to relate the $L^{2}$–distance of $\alpha$
from the calibrated cone to the analytic energy of the unique Coulomb potential
solving $d^{*}d\eta = d^{*}\alpha$.
This leads to an energy gap estimate and eventually to coercivity in the
$(p\!+\!1,p\!-\!1)$– and $(p\!-\!1,p\!+\!1)$–types and in the primitive part
of $(p,p)$–forms.

\subsection*{Coulomb potential}
Fix a representative $\alpha$ of $[\alpha]$.  Since $d\alpha = 0$, the elliptic
equation
\[
d^{*}d\eta = d^{*}\alpha
\]
admits a unique solution $\eta$ orthogonal to $\ker d$, giving the Hodge
decomposition
\[
\alpha
= \gamma_{\harm} + d\eta,
\]
where $\gamma_{\harm}$ is the unique harmonic representative of $[\alpha]$.
We define the energy of $\alpha$ by
\[
E(\alpha) := \|d\eta\|^{2}_{L^{2}}.
\]

\subsection*{Energy Identity}
We now express $E(\alpha)$ in terms of type components.  Since
$\gamma_{\harm}$ is harmonic and of pure type $(p,p)$, we have
$d^{*}\gamma_{\harm}=0$ and
\[
\|\alpha\|^{2}_{L^{2}}
= \|\gamma_{\harm}\|^{2}_{L^{2}} + \|d\eta\|^{2}_{L^{2}}
\]
because $\gamma_{\harm} \perp d\eta$.
Thus:

\begin{equation}\label{eq:energy-split}
	E(\alpha)
	= \|\alpha\|_{L^{2}}^{2} - \|\gamma_{\harm}\|_{L^{2}}^{2}
	= \|d\eta\|^{2}_{L^{2}}.
	\tag{11}
\end{equation}

Decomposing $\alpha$ into types,
\[
\alpha
=
\alpha^{(p+1,p-1)}
+ \alpha^{(p,p)}
+ \alpha^{(p-1,p+1)},
\]
and noting that $\gamma_{\harm} = \gamma_{\harm}^{(p,p)}$, we obtain

\begin{equation}\label{eq:type-split}
	\|\alpha - \gamma_{\harm}\|_{L^{2}}^{2}
	=
	\|\alpha^{(p+1,p-1)}\|_{L^{2}}^{2}
	+ \|\alpha^{(p-1,p+1)}\|_{L^{2}}^{2}
	+ \|(\alpha^{(p,p)} - \gamma_{\harm})\|_{L^{2}}^{2}.
	\tag{12}
\end{equation}

Finally, the standard K\"ahler identities imply control of the non-\((p,p)\)
types and the primitive part of the \((p,p)\)–component in terms of $d\eta$:

\begin{equation}\label{eq:primitive-control}
	\|\alpha^{(p+1,p-1)}\|_{L^{2}}
	+
	\|\alpha^{(p-1,p+1)}\|_{L^{2}}
	+
	\|(\alpha^{(p,p)} - \gamma_{\harm})_{\prim}\|_{L^{2}}
	\;\le\;
	C(n,p)\,\|d\eta\|_{L^{2}}.
	\tag{13}
\end{equation}

\begin{lemma}[Coulomb decomposition and energy identity]\label{lem:coulomb}
	Let $\alpha$ be a smooth closed real $2p$–form on a compact K\"ahler manifold.
	Write $\alpha = \gamma_{\harm} + d\eta$ for its Coulomb decomposition.
	Then:
	
	\begin{enumerate}
		
		\item
		$\displaystyle
		E(\alpha)
		= \|d\eta\|_{L^{2}}^{2}
		= \|\alpha\|_{L^{2}}^{2} - \|\gamma_{\harm}\|_{L^{2}}^{2},
		$
		as in~\eqref{eq:energy-split}.
		
		\item
		The difference from the harmonic representative satisfies
		\[
		\|\alpha - \gamma_{\harm}\|_{L^{2}}^{2}
		=
		\|\alpha^{(p+1,p-1)}\|_{L^{2}}^{2}
		+ \|\alpha^{(p-1,p+1)}\|_{L^{2}}^{2}
		+ \|(\alpha^{(p,p)} - \gamma_{\harm})\|_{L^{2}}^{2},
		\]
		as in~\eqref{eq:type-split}.
		
		\item
		The non-harmonic part is controlled by the primitive and $(p\!\pm\!1,p\!\mp\!1)$
		types:
		\[
		\|\alpha^{(p+1,p-1)}\|_{L^{2}}
		+
		\|\alpha^{(p-1,p+1)}\|_{L^{2}}
		+
		\|(\alpha^{(p,p)} - \gamma_{\harm})_{\prim}\|_{L^{2}}
		\;\le\;
		C(n,p)\,\sqrt{E(\alpha)},
		\]
		consistent with~\eqref{eq:primitive-control}.
		
	\end{enumerate}
	
\end{lemma}

\begin{proof}
	Item (i) follows from the orthogonality $\gamma_{\harm}\perp d\eta$ and the
	Coulomb normalization $d^{*}\eta=0$.
	Item (ii) is the orthogonal decomposition of the type components relative to
	$\gamma_{\harm}^{(p,p)}$.
	Item (iii) follows from the K\"ahler identities:
	$d = \partial + \bar\partial$, $d^{*} = \partial^{*} + \bar\partial^{*}$,
	together with elliptic estimates for the operator $d^{*}d$ on $\eta$.
\end{proof}

% ------------------------------------------------------------
% SECTION 5 — The Calibrated Grassmannian and an Explicit ε–Net
% ------------------------------------------------------------

\section{The Calibrated Grassmannian and an Explicit $\varepsilon$–Net}

\subsection*{Fiberwise geometry}

Fix $x\in X$ and set
\[
\varphi := \frac{\omega^{p}}{p!}.
\]
Define the calibrated Grassmannian at $x$ by
\[
G_{p}(x)
:=
\Big\{
\xi \in \Lambda^{2p}T^{*}_{x}X :
\|\xi\| = 1,\;
\xi\ \text{simple of type $(p,p)$},\;
\varphi_{x}(\xi)=1
\Big\}.
\]
This is the set of unit simple $(p,p)$ covectors saturated by the K\"ahler
calibration $\varphi_{x}$.  Equivalently, $G_{p}(x)$ is the image of the
complex Grassmannian $G_{\C}(p,n)$ under the map sending a $p$--plane
$V\subset T^{1,0}_{x}X$ to its associated calibrated covector $\phi_{V}$.
With the metric induced by $\omega$, this map is an isometric embedding
(up to normalization), and therefore
\[
G_{p}(x) \cong G_{\C}(p,n)
\]
with its standard Fubini--Study metric.  In particular, $G_{p}(x)$ is
compact, smooth, homogeneous, and has real dimension
\[
d := \dim_{\R} G_{p}(x)
= 2p(n-p).
\]

\subsection*{$\varepsilon$–nets and covering estimates}

Fix $\varepsilon = \tfrac{1}{10}$.  
On each fiber $G_{p}(x)$ (with the Fubini--Study geodesic distance
$d_{\mathrm{FS}}$), choose a maximal $\varepsilon$–separated set
\[
\{\xi(x)_\ell\}_{\ell=1}^{N(x)}
\subset G_{p}(x),
\qquad
d_{\mathrm{FS}}(\xi(x)_\ell,\xi(x)_m) \ge \varepsilon
\ \text{for all }\ell\ne m,
\]
such that no additional point of $G_{p}(x)$ can be added while preserving
this separation property.

By compactness and the standard packing principle on compact homogeneous
spaces, such maximal $\varepsilon$–separated sets are automatically
$\varepsilon$–nets: for every $\xi \in G_{p}(x)$ there exists an index
$\ell$ with  
\[
d_{\mathrm{FS}}(\xi,\xi(x)_\ell) \le \varepsilon.
\]

\begin{lemma}[Covering number]\label{lem:covering-number}
	Let $d = 2p(n-p)$.  
	There exists a constant $C(n,p)$ depending only on $(n,p)$ such that every
	maximal $\varepsilon$–separated set in $G_{p}(x)$ satisfies
	\begin{equation}\label{eq:grass-cover}
		N(x) \;\le\; C(n,p)\,\varepsilon^{-d}.
		\tag{5.1}
	\end{equation}
\end{lemma}

\begin{proof}
	Cover $G_{p}(x)$ by the geodesic balls
	\[
	B\!\left(\xi(x)_\ell,\,\tfrac{\varepsilon}{2}\right),
	\qquad \ell=1,\dots,N(x),
	\]
	of radius $\varepsilon/2$ in the Fubini--Study metric.  
	Because the points are $\varepsilon$–separated, these balls are pairwise
	disjoint.  By maximality of the separated set, the $\varepsilon$–balls
	\[
	B\!\left(\xi(x)_\ell,\,\varepsilon\right)
	\]
	cover $G_{p}(x)$.
	
	Since $G_{p}(x)$ is a compact homogeneous space, the volume of a small
	geodesic ball depends only on the radius, not on its center.  
	Let $V(r)$ denote the volume of a geodesic ball of radius $r$.  
	Then disjointness gives
	\[
	N(x)\,V(\varepsilon/2)
	\;\le\; \Vol\bigl(G_{p}(x)\bigr),
	\]
	while the covering property yields
	\[
	\Vol\bigl(G_{p}(x)\bigr)
	\;\le\; N(x)\,V(\varepsilon).
	\]
	
	For small $r$ one has the uniform expansion
	\[
	V(r) = c_{d}\,r^{d} + O(r^{d+2}),
	\]
	with $c_{d}>0$ depending only on $d = \dim_{\R} G_{p}(x)$.  
	Since $G_{p}(x)$ is homogeneous, there exist constants $A(n,p)$ and $B(n,p)$
	such that
	\[
	A(n,p)\,r^{d} \le V(r) \le B(n,p)\,r^{d}
	\qquad\text{for } 0<r\le 1.
	\]
	
	Combining the two volume inequalities gives
	\[
	N(x)\,A(n,p)\,(\varepsilon/2)^{d}
	\;\le\; \Vol\bigl(G_{p}(x)\bigr)
	\;\le\; N(x)\,B(n,p)\,\varepsilon^{d},
	\]
	so cancelling $\Vol(G_{p}(x))$ yields
	\[
	N(x) \;\le\;
	\frac{B(n,p)}{A(n,p)}\,(2^{d})\,
	\varepsilon^{-d}.
	\]
	
	Absorbing the constants into
	\[
	C(n,p) := \frac{B(n,p)}{A(n,p)}\,2^{d},
	\]
	we obtain the desired estimate \eqref{eq:grass-cover}.
\end{proof}
% ============================================================
% SECTION 6 — Pointwise Linear Algebra: Controlling the Net Distance
% ============================================================

\section{Pointwise Linear Algebra: Controlling the Net Distance}
\label{sec:linear-algebra}

In this section we develop the pointwise linear--algebraic estimates
that control the distance of a real $2p$--form to the calibrated
span generated by the $\varepsilon$--net constructed in Section~5.
The goal is to show that the net distance (and therefore the cone
distance) is controlled by two quantities:

\begin{itemize}
	\item the off--type components $\alpha_{x}^{(p+1,p-1)}$ and 
	$\alpha_{x}^{(p-1,p+1)}$, and 
	\item the primitive traceless part of the $(p,p)$--component.
\end{itemize}

These pointwise inequalities form the core of the coercivity 
estimate used later in Section~\ref{sec:coercivity}.

% ------------------------------------------------------------
\subsection*{Calibrated span}

Fix $x\in X$ and let 
\[
\{\xi_{\ell}(x)\}_{\ell=1}^{N(x)} \subset G_{p}(x)
\]
be the $\varepsilon$--net of Section~5, with $\varepsilon=\tfrac{1}{10}$.
Define the calibrated span at $x$ by
\[
\Xi_{x}:=
\Span\{\xi_{\ell}(x):1\le \ell \le N(x)\}
\subset \Lambda^{p,p}T_{x}^{*}X.
\]

Each $\xi_{\ell}(x)$ is a unit simple $(p,p)$--covector, hence lies
entirely in the $(p,p)$--subspace of $\Lambda^{2p}T_{x}^{*}X$ and is
orthogonal to all off--type $(p+1,p-1)$ and $(p-1,p+1)$ components
with respect to the K\"ahler metric.

Thus every $\alpha_{x}\in\Lambda^{2p}T_{x}^{*}X$ admits an
orthogonal type decomposition
\begin{equation}\label{eq:typesplit-orth}
	\alpha_{x}
	=
	\alpha_{x}^{(p+1,p-1)}
	\;+\;
	\alpha_{x}^{(p-1,p+1)}
	\;\perp\;
	\alpha_{x}^{(p,p)}.
	\tag{21}
\end{equation}

% ------------------------------------------------------------
\subsection*{Pointwise net distance}

Define the pointwise net distance
\[
D_{\mathrm{net}}(\alpha_{x})
:=
\min_{\ell,\;\lambda\ge 0}
\|\alpha_{x} - \lambda\xi_{\ell}(x)\|.
\]

\begin{lemma}[Off--type separation for $D_{\mathrm{net}}$]\label{lem:typesplit}
	For every $x$ and every $\alpha_{x}\in\Lambda^{2p}T^{*}_{x}X$,
	\begin{equation}\label{eq:Dnet-typesplit}
		D_{\mathrm{net}}(\alpha_{x})^{2}
		=
		\|\alpha_{x}^{(p+1,p-1)}\|^{2}
		+
		\|\alpha_{x}^{(p-1,p+1)}\|^{2}
		+
		\min_{1\le \ell\le N(x),\,\lambda\ge 0}
		\|\alpha_{x}^{(p,p)} - \lambda \xi_{\ell}(x)\|^{2}.
		\tag{22}
	\end{equation}
\end{lemma}

\begin{proof}
	For each $\ell$ and each $\lambda\ge 0$, the form $\lambda\xi_{\ell}(x)$
	lies in the $(p,p)$--subspace.  By the orthogonality in
	\eqref{eq:typesplit-orth},
	\[
	\|\alpha_{x} - \lambda\xi_{\ell}(x)\|^{2}
	=
	\|\alpha_{x}^{(p+1,p-1)}\|^{2}
	+
	\|\alpha_{x}^{(p-1,p+1)}\|^{2}
	+
	\|\alpha_{x}^{(p,p)} - \lambda\xi_{\ell}(x)\|^{2}.
	\]
	Minimizing over $\ell$ and $\lambda$ gives \eqref{eq:Dnet-typesplit}.
\end{proof}

% ------------------------------------------------------------
\subsection*{Projection estimate}

We now show that the $(p,p)$--term in \eqref{eq:Dnet-typesplit}
is controlled by a purely $(p,p)$ quantity arising from the Hermitian
model for $(p,p)$--forms and a rank--one approximation inequality.

\begin{lemma}[Hermitian model for $(p,p)$]\label{lem:hermitian-model}
	Fix $x$ and identify $\Lambda^{p,0}T_x^{*}X$ with a Hermitian space 
	$\bigl(\mathcal{H},\langle\cdot,\cdot\rangle\bigr)$ of complex dimension 
	$d=\binom{n}{p}$.  
	There is an isometric isomorphism
	\[
	\mathcal{I} : \Lambda^{p,p}T_x^{*}X \;\longrightarrow\; \Herm(\mathcal{H})
	\]
	(with Hilbert--Schmidt norm on the right) such that:
	\begin{enumerate}
		\item for $\alpha_x^{(p,p)}\in\Lambda^{p,p}$, the matrix 
		$H_\alpha := \mathcal{I}(\alpha_x^{(p,p)})$ is Hermitian;
		
		\item for any unit decomposable $p$--vector $v\in\Lambda^{p,0}$,  
		the calibrated covector $\xi_v$ satisfies
		\[
		\mathcal{I}(\xi_v) = P_v := v\otimes v^{*}
		\]
		(the rank--one projector);
		
		\item the contraction (trace) corresponds to the Lefschetz trace:  
		there exists $\mu(\alpha_x)\in\R$ such that
		\[
		\mathcal{I}\bigl( (\alpha_x^{(p,p)})_{\mathrm{prim}} \bigr)
		=
		H_\alpha - \mu(\alpha_x)\, I_{\mathcal{H}},
		\qquad
		\mu(\alpha_x) = \frac{1}{d}\operatorname{tr}(H_\alpha).
		\]
	\end{enumerate}
	
	\emph{Proof sketch.}
	This is the standard identification of $(p,p)$--forms with Hermitian forms
	on $\Lambda^{p,0}$ via
	\[
	H_\alpha(u)=\frac{\alpha(u\wedge\overline{u})}{\|u\|^{2}}
	\]
	and polarization.  
	Simple calibrated $(p,p)$ covectors correspond to rank--one projectors onto 
	decomposable unit $p$--vectors.  
	The Lefschetz trace corresponds to the normalized trace on $\Herm(\mathcal{H})$; 
	subtracting the trace gives the primitive (traceless) component.
	\qed
\end{lemma}

\begin{lemma}[Rank--one approximation controls the traceless part]\label{lem:rankone}
	There exists a finite constant $C_{\mathrm{rank}}(d)>0$, depending only on
	$d=\dim_{\C}\mathcal{H}$, such that for every $H \in \Herm(\mathcal{H})$,
	\[
	\min_{\substack{v\in\mathcal{H},\,\|v\|=1 \\ \lambda \ge 0}}
	\|H - \lambda(v\otimes v^{*})\|_{\mathrm{HS}}^{2}
	\;\le\;
	C_{\mathrm{rank}}(d)\,\bigl\|H - \tfrac{\tr(H)}{d} I_{\mathcal{H}}\bigr\|_{\mathrm{HS}}^{2}.
	\]
\end{lemma}

\begin{proof}
	Consider the compact ``unit traceless shell''
	\[
	\mathcal{S}
	:=
	\Bigl\{H\in\Herm(\mathcal{H}) \;:\;
	\bigl\|H - \tfrac{\tr(H)}{d} I_{\mathcal{H}}\bigr\|_{\HS}=1\Bigr\}.
	\]
	The functional
	\[
	\Phi(H)
	:=
	\min_{\substack{v\in\mathcal{H},\,\|v\|=1 \\ \lambda \ge 0}}
	\|H - \lambda(v\otimes v^{*})\|_{\mathrm{HS}}^{2}
	\]
	is continuous on $\mathcal{S}$ (the minimization set is compact), hence attains a
	maximum $C_{\mathrm{rank}}(d):=\sup_{H\in\mathcal{S}}\Phi(H)<\infty$.  For general
	$H\neq 0$, scale by the traceless norm to obtain the stated inequality.
\end{proof}

\begin{proposition}[Projection estimate in $(p,p)$]\label{prop:pp-projection}
	There exists a constant $C_{0}=C_{0}(n,p)$ such that for all $x$ and all
	$\alpha_{x}$,
	\begin{equation}\label{eq:pp-projection}
		\min_{\ell,\;\lambda\ge 0}
		\bigl\|\alpha_{x}^{(p,p)} - \lambda\,\xi_{\ell}(x)\bigr\|^{2}
		\;\le\;
		C_{0}(n,p)\,
		\bigl\|%
		\bigl(\alpha_{x}^{(p,p)} - \gamma_{\harm,x}\bigr)_{\prim}
		\bigr\|^{2}.
		\tag{23}
	\end{equation}
		In particular, one may take $C_{0}(n,p)=C_{\mathrm{rank}}(d)$ with $d=\binom{n}{p}$.
\end{proposition}

\begin{proof}
	Set
	\[
	\beta_{x} := \alpha_{x}^{(p,p)} - \gamma_{\harm,x}
	\in \Lambda^{p,p}T^{*}_{x}X,
	\qquad
	H := \mathcal{I}(\beta_{x}) \in \Herm(\mathcal{H}),
	\]
	where $\mathcal{I}$ is the isometric isomorphism of
	Lemma~\ref{lem:hermitian-model}.  
	By Lemma~\ref{lem:hermitian-model}, the traceless part of $H$ is exactly
	the Hermitian model of the primitive part:
	\[
	H - \mu(\alpha_{x})\,I_{\mathcal{H}}
	=
	\mathcal{I}\bigl(
	(\alpha_{x}^{(p,p)} - \gamma_{\harm,x})_{\prim}
	\bigr),
	\qquad
	\mu(\alpha_{x}) = \tfrac{1}{d}\tr(H).
	\]
	Hence
	\[
	\bigl\|H - \mu(\alpha_{x})\,I_{\mathcal{H}}\bigr\|_{\mathrm{HS}}
	=
	\bigl\|%
	(\alpha_{x}^{(p,p)} - \gamma_{\harm,x})_{\prim}
	\bigr\|.
	\]
	
	Applying Lemma~\ref{lem:rankone} to $H$ yields
	\[
	\min_{\substack{v\in\mathcal{H},\,\|v\|=1\\ \lambda\ge 0}}
	\bigl\|H - \lambda(v\otimes v^{*})\bigr\|_{\mathrm{HS}}^{2}
	\;\le\;
	C_{\mathrm{rank}}(d)\,
	\bigl\|H - \mu(\alpha_{x})\,I_{\mathcal{H}}\bigr\|_{\mathrm{HS}}^{2}
	=
	C_{\mathrm{rank}}(d)\,
	\bigl\|%
	(\alpha_{x}^{(p,p)} - \gamma_{\harm,x})_{\prim}
	\bigr\|^{2}.
	\]
	
	By the defining properties of $\mathcal{I}$, for each calibrated unit
	covector $\xi_{v}$ corresponding to $v$ one has
	\[
	\mathcal{I}(\xi_{v}) = v\otimes v^{*},
	\quad
	\|\xi_{v}\| = 1,
	\]
	and $\mathcal{I}$ is an isometry.  Pulling back the above inequality via
	$\mathcal{I}^{-1}$ gives
	\[
	\min_{\xi} \min_{\lambda\ge 0}
	\bigl\|\beta_{x} - \lambda\xi\bigr\|^{2}
	\;\le\;
	C_{\mathrm{rank}}(d)\,
	\bigl\|%
	(\alpha_{x}^{(p,p)} - \gamma_{\harm,x})_{\prim}
	\bigr\|^{2},
	\]
	where the minimum is taken over all calibrated unit covectors at $x$.
	
	Finally, approximate the minimizing calibrated direction by some net
	vector $\xi_{\ell}(x)$ from the $\varepsilon$--net of Section~5.  The net
	contains such directions up to the fixed tolerance $\varepsilon$, and
	the resulting approximation only changes the constant by a bounded
	factor depending on $(n,p)$.  Absorbing this factor into $C_{0}(n,p)$
	and taking $C_{0}(n,p)=C_{\mathrm{rank}}(d)$ yields \eqref{eq:pp-projection}.
\end{proof}

\begin{corollary}[Pointwise control of $D_{\mathrm{net}}$]\label{cor:Dnet-pointwise}
	For all $x$ and all $\alpha_{x}$,
	\begin{equation}\label{eq:Dnet-pointwise}
		D_{\mathrm{net}}(\alpha_{x})^{2}
		\;\le\;
		C_{0}(n,p)\Bigl(
		\|\alpha_{x}^{(p+1,p-1)}\|^{2}
		+
		\|\alpha_{x}^{(p-1,p+1)}\|^{2}
		+
		\bigl\|%
		(\alpha_{x}^{(p,p)} - \gamma_{\harm,x})_{\prim}
		\bigr\|^{2}
		\Bigr).
		\tag{24}
	\end{equation}
\end{corollary}

\begin{proof}
	Combine Lemma~\ref{lem:typesplit} with
	Proposition~\ref{prop:pp-projection}.
\end{proof}

\paragraph{Fixing an explicit constant.}
In the previous projection estimate we obtained a constant
$C_{0}(n,p)$ depending only on $(n,p)$.
For the remainder of the paper we fix the explicit choice
\[
C_{0}(n,p) := 2,
\]
which suffices for all subsequent global estimates.
Any quantitative improvement in the rank--one approximation
(Lemma~\ref{lem:rankone}) or in the $\varepsilon$--net approximation
step would simply decrease this constant proportionally, but no such
refinement is needed for our purposes.

\begin{proposition}[Pointwise cone projection bound]\label{prop:cone-projection}
	At each $x\in X$ and for every 
	$\alpha_{x}\in \Lambda^{2p}T^{*}_{x}X$, decompose
	\[
	\alpha_{x} 
	= 
	\alpha_{x}^{(p+1,p-1)}
	\;\perp\;
	\alpha_{x}^{(p,p)}
	\;\perp\;
	\alpha_{x}^{(p-1,p+1)}.
	\]
	Let 
	\[
	H(x) := \mathcal{I}\!\left(\alpha_{x}^{(p,p)} - \gamma_{\harm,x}\right)
	\in \Herm(\mathcal{H}),
	\qquad
	d := \binom{n}{p},
	\qquad
	\mu(x) := \tfrac{1}{d}\operatorname{tr} H(x).
	\]
	Let $H_{-}(x)$ denote the negative part in the spectral decomposition of $H(x)$.
	Then
	\begin{equation}\label{eq:cone-dist-H}
		\mathrm{dist}_{\mathrm{cone}}(\alpha_{x})^{2}
		=
		\|\alpha_{x}^{(p+1,p-1)}\|^{2}
		+\|\alpha_{x}^{(p-1,p+1)}\|^{2}
		+\| H_{-}(x)\|_{\mathrm{HS}}^{2}.
		\tag{25}
	\end{equation}
	Moreover, since the orthogonal trace--traceless splitting yields
	\[
	\|H(x)\|_{\mathrm{HS}}^{2}
	= \|H(x)-\mu(x) I\|_{\mathrm{HS}}^{2} + d\,\mu(x)^{2},
	\]
	we obtain the bound
	\[
	\mathrm{dist}_{\mathrm{cone}}(\alpha_{x})^{2}
	\;\le\;
	\|\alpha_{x}^{(p+1,p-1)}\|^{2}
	+\|\alpha_{x}^{(p-1,p+1)}\|^{2}
	+\|(\alpha_{x}^{(p,p)} - \gamma_{\harm,x})_{\prim}\|^{2}
	+ d\,\mu(x)^{2}.
	\]
\end{proposition}

\begin{proof}
	Projecting $\alpha_{x}$ orthogonally onto the $(p,p)$--space separates the
	off--type terms exactly.  
	Under the Hermitian isometry $\mathcal{I}$, the calibrated cone corresponds to
	the PSD cone in $\Herm(\mathcal{H})$, hence the metric projection of $H(x)$ onto
	the cone is $H_{+}(x)$ and 
	$\|H(x)-H_{+}(x)\|_{\mathrm{HS}}^{2} = \|H_{-}(x)\|_{\mathrm{HS}}^{2}$.
	This gives \eqref{eq:cone-dist-H}.
	
	The identity 
	\[
	\|H\|_{\mathrm{HS}}^{2}
	= \|H-\mu(x)I\|_{\mathrm{HS}}^{2} + d\,\mu(x)^{2}
	\]
	is the orthogonal decomposition into primitive (traceless) and Lefschetz trace
	components.  
	Pulling this back via $\mathcal{I}^{-1}$ yields the stated inequality.
\end{proof}

%================================

% ==========================================================
%  SECTION 7
\section{Calibration--Coercivity (Explicit) and Its Proof}
\label{sec:cal-coercivity}

Let $(X,\omega)$ be a smooth projective K\"ahler manifold and let
$\gamma\in H^{2p}(X,\R)\cap H^{p,p}(X)$ be a de~Rham class.
Denote by $\gharm$ its unique $\omega$–harmonic representative and by
$E(\cdot)$ the Dirichlet energy.

For each $x\in X$, the fiberwise calibrated cone $K_p(x)$ is the closed cone of
$(p,p)$–forms saturated by the K\"ahler calibration.  
The global cone defect of a form $\alpha$ is
\[
\Defcone(\alpha)
:= \int_X \distcone(\alpha_x)^2\,d\mathrm{vol}_\omega(x),
\qquad
\distcone(\alpha_x)
:= \inf_{\beta_x\in K_p(x)} \|\alpha_x - \beta_x\|.
\]

The main estimate of this section is the following explicit version of
Theorem~A.

\begin{theorem}[Explicit calibration--coercivity]
	\label{thm:cal-coercivity}
	For every smooth closed representative $\alpha\in[\gamma]$ one has
	\begin{equation}\label{eq:global-coercivity}
		E(\alpha)-E(\gharm) \;\ge\; c\,\Defcone(\alpha),
	\end{equation}
	with explicit constant
	\begin{equation}\label{eq:c-constant}
		c \;=\; \frac{1}{2 + d\,C_\Lambda^2},
		\qquad
		d=\binom{n}{p},
	\end{equation}
	where $C_\Lambda=C_\Lambda(n,p) = d^{-1/2}$ is the Hermitian trace constant from
	Lemma~\ref{lem:trace-L2}.  
	The constant $c$ depends only on $(n,p)$ and not on $[\gamma]$.
\end{theorem}

\begin{proof}
	We follow the pointwise linear algebra and global $L^2$ decomposition
	from Proposition~6.6 together with the Hermitian trace estimate in
	Lemma~13.2.
	
	\medskip\noindent
	\textbf{Step 1: Global control of off–type and primitive parts.}
	Decompose $\alpha$ into its Hodge components:
	\[
	\alpha
	= \alpha^{(p+1,p-1)} + \alpha^{(p,p)} + \alpha^{(p-1,p+1)}.
	\]
	By Lemma~\ref{lem:coulomb} and the K\"ahler identities (cf.~\eqref{eq:primitive-control}),
	the non--$(p,p)$ types and the primitive part of the $(p,p)$–component satisfy
	the global estimate
	\begin{equation}\label{eq:off-type-primitive-integral}
		\int_X \Bigl(
		|\alpha^{(p+1,p-1)}|^2
		+ |\alpha^{(p-1,p+1)}|^2
		+ |(\alpha^{(p,p)}-\gharm)_{\mathrm{prim}}|^2
		\Bigr)\,d\mathrm{vol}_\omega
		\;\le\;
		2\bigl(E(\alpha)-E(\gharm)\bigr).
	\end{equation}
	
	\medskip\noindent
	\textbf{Step 2: Trace component control via the Hermitian model.}
	At each $x$, let
	\[
	H(x)
	:= \mathcal{I}\!\left(\alpha^{(p,p)}_x - (\gharm)_x\right)
	\in \mathrm{Herm}(\mathcal{H}),
	\qquad
	\dim_{\C}\mathcal{H} = d=\binom{n}{p},
	\]
	be the Hermitian matrix associated to the $(p,p)$–difference via the
	isometric identification of Lemma~\ref{lem:hermitian-model}.
	Define
	\[
	\mu(x) := \frac{1}{d}\operatorname{tr} H(x).
	\]
	In terms of the Lefschetz decomposition, this means
	\[
	\alpha^{(p,p)} - \gharm
	= \mu\,\omega^p
	+ (\alpha^{(p,p)} - \gharm)_{\mathrm{prim}}.
	\]
	The Hermitian trace estimate (Lemma~13.2) gives
	\[
	d\int_X \mu(x)^2\,d\mathrm{vol}_\omega(x)
	\;\le\;
	d\,C_\Lambda^2 \,\|\alpha-\gharm\|_{L^2}^2
	=
	d\,C_\Lambda^2 \bigl(E(\alpha)-E(\gharm)\bigr).
	\]
	
	Combining this with \eqref{eq:off-type-primitive-integral} and the orthogonal
	decomposition
	\[
	\|\alpha-\gharm\|_{L^2}^2
	=
	\int_X \Bigl(
	|\alpha^{(p+1,p-1)}|^2
	+ |\alpha^{(p-1,p+1)}|^2
	+ |(\alpha^{(p,p)}-\gharm)_{\mathrm{prim}}|^2
	+ d\,\mu^2
	\Bigr) d\mathrm{vol}_\omega
	\]
	yields
	\begin{equation}\label{eq:L2-difference}
		\int_X |\alpha-\gharm|^2\,d\mathrm{vol}_\omega
		\;\le\;
		(2 + d\,C_\Lambda^2)\bigl(E(\alpha)-E(\gharm)\bigr).
	\end{equation}
	
	\medskip\noindent
	\textbf{Step 3: Relating the cone defect to controlled components (unconditional).}
	By Proposition~\ref{prop:cone-projection},
	\[
	\distcone(\alpha_x)^2
	\le
	|\alpha^{(p+1,p-1)}_x|^2
	+ |\alpha^{(p-1,p+1)}_x|^2
	+ \|(\alpha^{(p,p)}_x-\gharm_x)_{\prim}\|^2
	+ d\,\mu(x)^2.
	\]
	Integrating over $X$ and invoking \eqref{eq:off-type-primitive-integral} and the
	trace estimate above, we obtain
	\[
	\Defcone(\alpha)
	\;\le\;
	(2 + d\,C_\Lambda^2)\bigl(E(\alpha)-E(\gharm)\bigr).
	\]
	
	\medskip\noindent
	\textbf{Step 4: Conclusion.}
	Rearranging the last inequality yields
	\[
	E(\alpha)-E(\gharm)
	\;\ge\;
	\frac{1}{2 + d\,C_\Lambda^2}\,\Defcone(\alpha),
	\]
	which is exactly \eqref{eq:global-coercivity}.
\end{proof}

\begin{remark}[Dependence of constants]
	The constant is intrinsic and depends only
	on $(n,p)$ and the Hermitian trace bound $C_\Lambda$ (and implicit universal
	choices in Lemma~\ref{lem:rankone} folded into $C_{0}(n,p)$, which do not
	enter \eqref{eq:c-constant}).
	Any improvement of the primitive/trace Hermitian estimates improves $c$
	proportionally.
\end{remark}

% ------------------------------------------------------------
\subsection*{Remark: a heuristic penalized route (not used in this paper)}

Define the penalized functional on closed representatives of $[\gamma]$ by
\[
\mathcal{F}_\lambda(\alpha) := E(\alpha) + \lambda\,\Defcone(\alpha),
\qquad \lambda \ge 0.
\]
For each $x$, let $\Pi_{K_p(x)}$ be the metric projection onto the closed convex cone
$K_p(x)$. Pointwise Pythagoras for orthogonal projection onto a closed convex cone
gives
\[
\|\alpha_x\|^2 = \|\Pi_{K_p(x)}(\alpha_x)\|^2 + \dist\!\bigl(\alpha_x,K_p(x)\bigr)^2.
\]
Integrating,
\begin{equation}\label{eq:projection-identity}
	E(\alpha) = E\!\bigl(\Pi_K(\alpha)\bigr) + \Defcone(\alpha),
\end{equation}
where $(\Pi_K\alpha)(x):=\Pi_{K_p(x)}(\alpha_x)$.

\begin{remark}[Limitation of pointwise projection]
While \eqref{eq:projection-identity} is a valid pointwise identity, the
fiberwise projection $\Pi_K(\alpha)$ does \emph{not} preserve closedness:
$d(\Pi_K(\alpha)) \neq 0$ in general, so $\Pi_K(\alpha)$ is not a closed
representative of $[\gamma]$.  Thus the naive descent argument
$\mathcal{F}_\lambda(\Pi_K(\alpha)) < \mathcal{F}_\lambda(\alpha)$ does not
produce a feasible competitor within the constraint set of closed forms.
A rigorous penalized approach would require combining pointwise projection
with a global Hodge-type correction (e.g., projecting onto the space of
closed forms after each step) and establishing that the resulting scheme
converges.  We do not pursue this route here; the main proof uses the
Dirichlet-only coercivity inequality together with the explicit SYR
construction in Section~8.
\end{remark}

% ============================================================
\section{From Cone–Valued Minimizers to Calibrated Currents}\label{sec:realization}
% ============================================================

Let $\varphi=\omega^{p}/p!$ and let $\psi:=*\varphi=\omega^{n-p}/(n-p)!$ denote the
K\"ahler calibration of $\C$–dimension $(n-p)$ planes. We write $A=\mathrm{PD}(m[\gamma])\in H_{2n-2p}(X,\Z)$ for some $m\ge 1$.

\begin{theorem}[Realization from almost–calibrated sequences]\label{thm:realization-from-almost}
	Let $(X,\omega)$ be smooth projective K\"ahler, $1\le p\le n$, and fix $A=\mathrm{PD}(m[\gamma])$.
	Suppose there exists a sequence of integral $2n\!-\!2p$ cycles $T_k$ on $X$ with
	\begin{enumerate}
		\item $\partial T_k=0$ and $[T_k]=A$,
		\item $\Mass(T_k)\downarrow c_0$, where
		\(
			c_0:=\langle A,[\psi]\rangle=\int_X m\,\gamma\wedge\psi
		\)
		(equality by cohomology–homology pairing),
	\end{enumerate}
	then, up to a subsequence, $T_k\to T$ weakly as currents with $[T]=A$,
	\(
		\Mass(T)=c_0,
	\)
	and $T$ is $\psi$–calibrated. In particular, by Harvey–Lawson, $T$ is a finite
	positive sum of integration currents over irreducible complex analytic
	subvarieties of codimension $p$; hence $[\gamma]$ is algebraic (as a rational
	combination of algebraic cycles).
\end{theorem}

\begin{proof}
	By Federer–Fleming compactness, the class and mass bounds yield a subsequence $T_{k_j}\rightharpoonup T$ as integral currents with $[T]=A$ and
	$\Mass(T)\le\liminf \Mass(T_{k_j})=c_0$. Since $\psi$ is closed,
	\(
		\int T_{k_j}\psi=\langle [T_{k_j}],[\psi]\rangle=\langle A,[\psi]\rangle=c_0
	\)
	for all $j$, and the calibration inequality gives
	\(
		\int T\psi=\lim \int T_{k_j}\psi=c_0\le \Mass(T).
	\)
	Combining with $\Mass(T)\le c_0$ we obtain $\Mass(T)=\int T\psi$, i.e.\ $T$ is
	$\psi$–calibrated. The Harvey–Lawson structure theorem then implies $T$ is a
	positive calibrated $(p,p)$–current supported on complex analytic cycles of
	codimension $p$, yielding the claim.
\end{proof}

\begin{remark}[How to use Theorem~\ref{thm:realization-from-almost}]
	The coercivity (or penalized) constructions deliver cone–valued smooth
	representatives once the energy gap has been exhausted.  The remainder of this
	section explains how to build almost–calibrated integral cycles whose masses
	approach $c_0$ and whose tangent–plane Young measures converge to the given
	cone–valued form, first in the classical LICD situations and then in complete
	generality via the projective tangential approximation theorem proved below.
\end{remark}

% ------------------------------------------------------------
\subsection*{Unconditional realizability in codimension one (Lefschetz (1,1))}

\begin{theorem}[Codimension one]\label{thm:codim1}
	If $p=1$ and $[\gamma]\in H^{1,1}(X,\Q)$ on a smooth projective $X$, then
	$[\gamma]$ is algebraic. Moreover, one can choose integral cycles $T_k$ with
	$\Mass(T_k)\to c_0=\langle \mathrm{PD}(m[\gamma]),[\omega^{n-1}/(n-1)!]\rangle$
	as in Theorem~\ref{thm:realization-from-almost}.
\end{theorem}

\begin{proof}[Proof sketch]
	By the Lefschetz $(1,1)$–theorem, $[\gamma]=c_1(L)\otimes_{\Z}\Q$ for a line
	bundle $L$. For $m\gg 0$, $L^{\otimes m}$ is very ample after twisting, hence
	admitting divisors $D_m$ with $[D_m]=\mathrm{PD}(m[\gamma])$. Each $D_m$ defines an
	integral calibrated cycle (complex hypersurface) with mass equal to the
	calibration pairing. Taking sequences of such divisors (e.g.\ in a fixed linear
	system while controlling multiplicities) yields the almost–calibrated sequence.
\end{proof}

% ------------------------------------------------------------
\subsection*{Complete–intersection realizability (very ample slicing)}

\begin{proposition}[Complete intersections]\label{prop:complete-intersection}
	Suppose $[\gamma]\in H^{p,p}(X,\Q)$ can be written as a rational linear
	combination of cohomology classes of complete intersections of $p$ very ample
	divisors. Then there exists a sequence of integral cycles in the class
	$\mathrm{PD}(m[\gamma])$ with masses tending to $c_0$, and the limit is a calibrated
	sum of complex subvarieties realizing $[\gamma]$.
\end{proposition}

\begin{proof}[Idea]
	Very ample divisors are represented by smooth hypersurfaces calibrated by
	$\omega^{n-1}/(n-1)!$. Intersections of $p$ such hypersurfaces produce smooth
	complex submanifolds of codimension $p$ calibrated by $\psi=\omega^{n-p}/(n-p)!$.
	Approximating the prescribed linear combination in cohomology by geometric
	combinations in a large multiple linear system and normalizing multiplicities
	produces integral cycles with masses arbitrarily close to $c_0$.
\end{proof}

% ------------------------------------------------------------
\subsection*{General realizability: a stationarity hypothesis}

\begin{definition}[Stationary Young–measure realizability (SYR)]\label{def:syr}
	We say a cone–valued smooth closed $(p,p)$–form $\beta$ (representing $[\gamma]$)
	is SYR–realizable if there exists a sequence of $\psi$–calibrated integral cycles
	$T_k$ whose tangent–plane Young measures converge a.e.\ to a measurable field
	$\nu_x$ supported on $\Gr_{n-p}(\C^n)$ with barycenter
	$\int \xi_P\,d\nu_x(P)=\widehat\beta(x)$, where
	\[
	\widehat\beta(x):=
	\begin{cases}
	\beta(x)/t(x), & t(x)>0,\\[2pt]
	0, & t(x)=0,
	\end{cases}
	\qquad
	t(x):=\langle \beta(x),\psi_x\rangle,
	\]
	and $\{T_k\}$ is stationary with $\Mass(T_k)\to c_0$.
\end{definition}

\begin{theorem}[Calibrated realization under SYR]\label{thm:syr}
	If a cone–valued representative $\beta$ of $[\gamma]$ is SYR–realizable, then
	there exists a calibrated integral cycle $T$ in $\mathrm{PD}(m[\gamma])$ and $[\gamma]$
	is algebraic.
\end{theorem}

\begin{proof}
	By SYR, $\Mass(T_k)\to c_0$ and $[T_k]=\mathrm{PD}(m[\gamma])$. Apply
	Theorem~\ref{thm:realization-from-almost}.
\end{proof}

\begin{remark}
	The SYR condition encodes the “microstructure” step in a purely geometric–measure
	framework (stationarity/compactness). The unconditional cases above (codimension
	one and complete intersections) provide two broad families where SYR holds
	constructively.
\end{remark}

% ------------------------------------------------------------
\subsection*{A classical sufficient criterion for SYR}

We now give a classical, fully geometric–measure–theoretic criterion under which
SYR holds, stated purely in standard language (coverings, Carath\'eodory
decompositions, isoperimetric fillings, and varifold compactness).

\begin{definition}[Locally integrable calibrated decomposition (LICD)]
	We say a smooth closed cone–valued $(p,p)$–form $\beta$ satisfies LICD if there
	exists a finite cover $\{U_\alpha\}$ of $X$ and for each $\alpha$:
	\begin{enumerate}
		\item smooth nonnegative coefficients $a_{\alpha,j}\in C^\infty(U_\alpha)$ and
		\item smooth fields of simple calibrated covectors $\xi_{\alpha,j}$ on $U_\alpha$,
	\end{enumerate}
	with $\beta=\sum_j a_{\alpha,j}\,\xi_{\alpha,j}$ on $U_\alpha$, where each
	$\xi_{\alpha,j}$ arises from a smooth integrable complex distribution of
	$(n-p)$–planes, i.e.\ through each $x\in U_\alpha$ there is a local
	$(n-p)$–dimensional complex submanifold whose oriented tangent plane is calibrated
	by $\psi$ and corresponds to $\xi_{\alpha,j}(x)$.
\end{definition}

\begin{theorem}[Classical SYR under LICD]\label{thm:classical-syr-licd}
	Let $(X,\omega)$ be smooth projective K\"ahler, $1\le p\le n$. If a smooth closed
	cone–valued $(p,p)$–form $\beta$ representing $[\gamma]$ satisfies LICD, then $\beta$
	is SYR–realizable. In particular, there exist integral $\psi$–calibrated cycles
	$T_k$ with $\partial T_k=0$, $[T_k]=\mathrm{PD}(m[\gamma])$, $\Mass(T_k)\to c_0$ and
	tangent–plane Young measures converging to a measurable field $\nu_x$ with
	barycenter $\widehat\beta(x)$ almost everywhere (where $\widehat\beta$ is the
	normalized field from Definition~\ref{def:syr}).
\end{theorem}

\begin{proof}[Proof (classical construction in charts)]
	Work in a single $U_\alpha$; a partition of unity reduces the global construction
	to a finite sum of local ones plus negligible overlaps.
	
	\emph{Step 1: Grid approximation and rationalization.} Fix a small mesh scale
	$\varepsilon>0$ and subordinate cubes $\{Q\}$ in a normal coordinate chart so that
	$\omega$ and $\psi$ vary by $O(\varepsilon)$ in each cell. By Carath\'eodory,
	$\beta=\sum_j a_j\,\xi_j$ with finitely many summands; approximate on each $Q$ by
	piecewise–constant smoothings
	\[
	\beta_Q \approx \sum_{j=1}^{N_Q} \theta_{Q,j}\,\xi_{Q,j},
	\qquad \theta_{Q,j}\in \Q_{\ge 0},\ \ \xi_{Q,j}\ \text{constant calibrated covectors},
	\]
	with $\sum_j \theta_{Q,j}$ bounded and the error $O(\varepsilon)$ in $C^0(Q)$.
	Write $\theta_{Q,j}=N_{Q,j}/M_Q$ with $N_{Q,j}\in\N$.
	
	\emph{Step 2: Local lamination by calibrated leaves.} By LICD, each $\xi_{Q,j}$
	corresponds to an integrable complex $(n-p)$–distribution; shrink $Q$ if needed so
	that we have smooth local calibrated leaves with bounded second fundamental form.
	Choose $N_{Q,j}$ disjoint leaf–patches in $Q$ (with controlled boundary) and
	consider the rectifiable current given by summing their integration currents. The
	resulting current $S_Q$ has tangent planes calibrated by $\psi$ almost everywhere
	in $Q$ and satisfies
	\[
	\Mass(S_Q) = \int S_Q\,\psi = \sum_j N_{Q,j}\int_{\mathrm{leaf}_{Q,j}}\psi
	= M_Q\int_Q \sum_j \theta_{Q,j}\,\langle \xi_{Q,j},\psi\rangle \,d\vol + O(\varepsilon\,|Q|),
	\]
	where the error arises from leaf boundaries near $\partial Q$ and the
	metric–calibration variation $O(\varepsilon)$. Since $\xi_{Q,j}$ are calibrated,
	$\langle\xi_{Q,j},\psi\rangle=1$ pointwise, hence $\Mass(S_Q)=M_Q\int_Q \sum_j
	\theta_{Q,j}\,d\vol + o_\varepsilon(1)$.
	
	\emph{Step 3: Closure by isoperimetric filling.} The sum $\sum_Q S_Q$ has small
	boundary concentrated on cell interfaces with $\Mass(\partial \sum_Q S_Q)\lesssim
	C\,\varepsilon$ (uniform density and bounded geometry). By the isoperimetric
	inequality on compact Riemannian manifolds and the Federer–Fleming Deformation
	Theorem, there exists a correction current $R_\varepsilon$ with
	$\partial R_\varepsilon = -\partial \sum_Q S_Q$ and $\Mass(R_\varepsilon)\to 0$ as
	$\varepsilon\to 0$. Then $T_\varepsilon:=\sum_Q S_Q+R_\varepsilon$ is closed,
	rectifiable, and calibrated almost everywhere.
	
	\emph{Step 4: Homology adjustment and mass control.} Pairing with $\psi$ shows
	\[
	\Mass(T_\varepsilon)=\int T_\varepsilon\,\psi
	= \sum_Q \int_Q \sum_j \theta_{Q,j}\,d\vol + o_\varepsilon(1)
	= \int_{U_\alpha}\beta\wedge\psi + o_\varepsilon(1).
	\]
	Using a finite cover $\{U_\alpha\}$ and partition of unity yields a global cycle
	with $\Mass(T_\varepsilon)=m\int_X\beta\wedge\psi + o_\varepsilon(1)$. Adjusting
	by a null–homologous small–mass cycle (via Deformation Theorem) yields an integral
	cycle in class $\mathrm{PD}(m[\gamma])$ with the same mass asymptotics. Varifold
	compactness then provides a convergent subsequence with tangent–plane Young
	measures converging to a field with barycenter $\widehat\beta(x)$ (as in the
	SYR definition). This is SYR.
\end{proof}

\begin{corollary}[Closure of the program under LICD]\label{cor:closure-licd}
	If the cone–valued representative furnished by the coercivity or penalized route
	satisfies LICD, then the sequence produced by Theorem~\ref{thm:classical-syr-licd}
	and Theorem~\ref{thm:realization-from-almost} yields a calibrated integral current
	realizing $[\gamma]$ as a rational algebraic cycle. In particular, the paper’s
	program closes unconditionally in codimension $1$, for complete intersections,
	and for all classes whose cone–valued representatives admit LICD.
\end{corollary}

% ============================================================
% RIGOROUS SYR CONSTRUCTION (GENERAL p)
% ============================================================

\subsection*{Step 1: Carath\'eodory decomposition in the Hermitian model}

At each $x\in X$, identify $\Lambda^{p,p}(T_x^*X)$ with a finite-dimensional
real vector space $\mathcal{V}_x$ equipped with the inner product induced by
the K\"ahler metric, and let $K_p(x)\subset \mathcal{V}_x$ be the closed convex
cone of strongly positive $(p,p)$-forms.
Each complex $(n-p)$-plane $P\subset T_xX$ determines an extremal ray of $K_p(x)$;
let $\xi_P\in K_p(x)$ denote a chosen generator of this ray, normalized so that
$\langle \xi_P,\psi_x\rangle=1$ (equivalently $\xi_P\wedge\psi_x=\omega_x^n/n!$).

Fix the positive ``trace'' functional $t(x):=\langle \beta(x),\psi_x\rangle=\frac{\beta\wedge\psi}{\omega^n/n!}(x)$.
Then $\widehat\beta(x):=\beta(x)/t(x)$ (on the set $\{t(x)>0\}$) lies in the convex
hull of the normalized generators $\{\xi_P:\ P\in \Gr_{n-p}(T_xX)\}$.
By Carath\'eodory's theorem in $\R^{D}$, $\widehat\beta(x)$ can be written as a convex
combination of at most $D+1$ such generators, where $D=\dim(\mathcal{V}_x)=\binom{n}{p}^2$
is independent of $x$.

\begin{lemma}[Uniform Carath\'eodory decomposition]\label{lem:caratheodory-general}
There exists $N=N(n,p)$ such that for all $x\in X$ there exist complex
$(n-p)$-planes $P_{x,1},\ldots,P_{x,N}\subset T_xX$ and weights
$\theta_{x,j}\ge 0$, $\sum_{j=1}^{N}\theta_{x,j}=1$, with
\[
\beta(x)=t(x)\sum_{j=1}^{N}\theta_{x,j}\,\xi_{P_{x,j}},
\qquad t(x):=\langle \beta(x),\psi_x\rangle.
\]
Moreover, for every $\varepsilon>0$ there exist measurable choices such that
the weights $\theta_{x,j}$ are piecewise continuous in $x$ and the fields
$x\mapsto P_{x,j}$ are measurable, with variation at most $\varepsilon$ on
sufficiently small coordinate cubes.
\end{lemma}

\begin{proof}
The uniform bound $N=D+1$ follows from Carath\'eodory's theorem in $\R^D$.
The measurability and local stabilization follow from standard measurable
selection theorems on the compact Grassmann bundle
$\Gr_{n-p}(TX)\to X$ together with a partition of unity subordinate to
normal coordinate charts.  The piecewise continuity of weights on small
cubes follows from the continuity of $\beta$ and the compactness of the
calibrated Grassmannian fibers.
\end{proof}

% ------------------------------------------------------------
\subsection*{Step 2: Projective tangential approximation with $C^1$ control}

Fix an ample line bundle $L\to X$ with a Hermitian metric whose curvature
form equals $\omega$.  For $m\in\N$ large, consider the complete linear
system $|L^m|$.

\begin{lemma}[$k$-jet surjectivity for high powers]\label{lem:jet-surjectivity}
For each integer $k\ge 1$ there exists $m_0(k)$ such that for all
$m\ge m_0(k)$ and all $x\in X$, the evaluation map on $k$-jets
\[
H^0(X,L^m)\longrightarrow J^k_x(L^m)
\]
is surjective.  In particular, for $k=1$, any prescribed value and
first derivative at $x$ is realized by a global section of $L^m$.
\end{lemma}

\begin{proof}
Consider the exact sequence
$0\to L^m\otimes \mathfrak{m}_x^{k+1}\to L^m \to
L^m\otimes \mathcal{O}_X/\mathfrak{m}_x^{k+1}\to 0$.
For $m\gg 0$, $H^1(X,L^m\otimes \mathfrak{m}_x^{k+1})=0$ by Serre vanishing
(ampleness of $L$).  Hence
$H^0(X,L^m)\twoheadrightarrow H^0(X,L^m\otimes \mathcal{O}_X/\mathfrak{m}_x^{k+1})$,
which identifies with $k$-jets at $x$.
See Lazarsfeld, \emph{Positivity in Algebraic Geometry~I}, Theorem~1.8.5.
\end{proof}

\begin{lemma}[Uniform $C^1$ control on $m^{-1/2}$-balls via Bergman kernels]
\label{lem:bergman-control}
Fix $\varepsilon>0$.  There exists $m_1(\varepsilon)$ such that for all
$m\ge m_1(\varepsilon)$, each $x\in X$, and each collection of $p$
complex covectors $\lambda_1,\ldots,\lambda_p\in T_x^*X$, there exist
sections $s_1,\ldots,s_p\in H^0(X,L^m)$ with the following properties
in normal holomorphic coordinates centered at $x$:
\begin{enumerate}
\item[\textnormal{(i)}] $s_i(x)=0$ and $ds_i(x)=\lambda_i$ for each $i$;
\item[\textnormal{(ii)}] on the geodesic ball $B_{c\,m^{-1/2}}(x)$
(for a universal constant $c>0$ depending only on $(X,\omega)$),
the gradients satisfy
\[
\|ds_i(y)-\lambda_i\|\le \varepsilon
\quad\text{for all } y\in B_{c\,m^{-1/2}}(x).
\]
\end{enumerate}
\end{lemma}

\begin{proof}
This is standard from peak section and Bergman kernel asymptotics
(Tian, Catlin, Zelditch, Donaldson).  In local normal coordinates with
rescaling by $\sqrt{m}$, the space $H^0(X,L^m)$ approximates holomorphic
polynomials with Gaussian weight, and there exist sections with prescribed
jets whose $C^1$ norms on $B_{c\,m^{-1/2}}$ approach those of the
corresponding linear functions.  See:
\begin{itemize}
\item G.~Tian, ``On a set of polarized K\"ahler metrics,''
      J.~Diff.~Geom.~32 (1990), 99--130;
\item S.~Zelditch, ``Szeg\H{o} kernels and a theorem of Tian,''
      IMRN 1998, no.~6, 317--331;
\item S.~K.~Donaldson, ``Scalar curvature and projective embeddings, I,''
      J.~Diff.~Geom.~59 (2001), 479--522, Section~2.
\end{itemize}
\end{proof}

\begin{proposition}[Projective tangential approximation with $C^1$ control]
\label{prop:tangent-approx-full}
Let $x\in X$ and let $\Pi\subset T_xX$ be a complex $(n-p)$-plane.
For every $\varepsilon>0$ there exist $m\gg 0$ and a smooth complete
intersection
\[
Y = \{s_1=0\}\cap \cdots \cap \{s_p=0\}\subset X,
\qquad s_i\in H^0(X,L^m),
\]
such that $x\in Y$, $Y$ is smooth in a neighborhood of $x$, and
\[
\angle\bigl(T_yY,\Pi\bigr)<\varepsilon
\quad\text{for all } y\in B_{c\,m^{-1/2}}(x).
\]
Moreover, $Y$ is $\psi$-calibrated (being a complex submanifold).
\end{proposition}

\begin{proof}
Choose covectors $\lambda_1,\ldots,\lambda_p\in T_x^*X$ whose common
kernel equals $\Pi$.  By Lemma~\ref{lem:bergman-control}, pick
$s_1,\ldots,s_p$ with $s_i(x)=0$, $ds_i(x)=\lambda_i$, and
$\|ds_i(y)-\lambda_i\|<\varepsilon/p$ on $B_{c\,m^{-1/2}}(x)$.

For $m\gg 0$ and after a small generic perturbation inside the
finite-dimensional linear system (which does not change jets at $x$
nor the $C^1$ estimates on the small ball), Bertini's theorem ensures
that $Y$ is smooth and $\{ds_1(y),\ldots,ds_p(y)\}$ are linearly
independent on the ball.

The complex normal space to $Y$ at $y$ is spanned by
$\{ds_1(y),\ldots,ds_p(y)\}$, which is $\varepsilon$-close to
$\{\lambda_1,\ldots,\lambda_p\}$ in the Grassmannian metric.
Hence $T_yY$ is $\varepsilon$-close to $\Pi$ for all $y$ in the ball.

Since $Y$ is a complex submanifold of a K\"ahler manifold, it is
automatically calibrated by $\psi=\omega^{n-p}/(n-p)!$.
\end{proof}

\begin{proposition}[Holomorphic density of calibrated directions]
\label{prop:dense-holo}
For every compact $K\subset X$ and $\varepsilon>0$ there exist finitely
many $\psi$-calibrated $(n-p)$-submanifolds $Y_1,\ldots,Y_M$ (each a
smooth complete intersection in $|L^m|$ for some large $m$) such that
for each $x\in K$ and each calibrated plane $\Pi\subset T_xX$ there
exists $j$ with $x\in Y_j$ and
$\mathrm{dist}\!\bigl(T_xY_j,\Pi\bigr)<\varepsilon$.
\end{proposition}

\begin{proof}
Cover $K$ by finitely many coordinate balls $\{B_\alpha\}$ centered at
points $\{x_\alpha\}$.  On each center $x_\alpha$, take an
$\varepsilon/2$-net of calibrated planes
$\{\Pi_{\alpha,1},\ldots,\Pi_{\alpha,N_\alpha}\}$ in the compact fiber
$G_{n-p}(T_{x_\alpha}X)$.  Apply Proposition~\ref{prop:tangent-approx-full}
to realize each net direction by a calibrated complete intersection
$Y_{\alpha,j}$ through $x_\alpha$ with tangent plane $\varepsilon/2$-close
to $\Pi_{\alpha,j}$ on a ball of radius $c\,m^{-1/2}$.

After shrinking the coordinate balls $B_\alpha$ if necessary (to fit
inside the $C^1$-control region), these submanifolds remain within
$\varepsilon$ of the target directions throughout each ball.
Collecting all $Y_{\alpha,j}$ over the finitely many centers gives
the desired family.
\end{proof}

% ------------------------------------------------------------
\subsection*{Step 3: Local calibrated laminates on small cubes (Theorem B)}

This step constructs multiple disjoint calibrated sheets on each cube $Q$
with prescribed tangent directions and mass fractions.

\begin{theorem}[Local multi-sheet construction]\label{thm:local-sheets}
Let $Q\subset X$ be a small coordinate cube.  Let
$\Pi_1,\ldots,\Pi_J\in \Gr_{n-p}(TQ)$ be constant $(n-p)$-planes, and let
$\theta_1,\ldots,\theta_J\in\Q_{>0}$ with $\sum_j\theta_j=1$.
For every $\varepsilon,\delta>0$, there exist smooth $\psi$-calibrated
complete intersections $\{Y_j^a\}_{j,a}$ in $X$ such that:
\begin{enumerate}
\item[\textnormal{(i)}] \textbf{Angle control:}
$\sup_{y\in Q}\angle(T_yY_j^a,\Pi_j)<\varepsilon$;
\item[\textnormal{(ii)}] \textbf{Mass fractions:}
$\bigl|\Mass(Y_j^a\llcorner Q)/\sum_{i,b}\Mass(Y_i^b\llcorner Q)-\theta_j\bigr|<\delta$;
\item[\textnormal{(iii)}] \textbf{Disjointness:} The $Y_j^a$ are pairwise disjoint on $Q$;
\item[\textnormal{(iv)}] \textbf{Boundary control:}
$\partial([Y_j^a]\llcorner Q)$ is supported on $\partial Q$.
\end{enumerate}
\end{theorem}

\begin{proof}
The proof proceeds in four substeps.

\medskip\noindent
\textbf{Substep 3.1: Local setup and flattening.}
Shrink $Q$ so that there is a holomorphic chart
$\Phi:U\to B(0,2)\subset\C^n$ with $Q\subset U$,
$\Phi(Q)\subset [-1,1]^{2n}\subset\C^n$, and the K\"ahler form $\omega$
and calibration $\psi=\omega^{n-p}/(n-p)!$ are $C^1$-close to the flat
model on $\C^n$.  The calibration cone $K_{n-p}(x)\subset\Gr_{n-p}(T_xX)$
varies smoothly and stays uniformly close to the flat cone of complex
$(n-p)$-planes.  We prove Theorem~\ref{thm:local-sheets} in this flattened
model; everything is diffeomorphism-invariant, and volume/mass distortions
are controlled by the uniform $C^1$-closeness of the metric.

\medskip\noindent
\textbf{Substep 3.2: Approximate target planes by calibrated planes.}
At each $x\in Q$, the set $K_{n-p}(x)$ of $\psi$-calibrated complex
$(n-p)$-planes is a compact subset of $\Gr_{n-p}(T_xX)$ (isomorphic to
the complex Grassmannian $G_{\C}(n-p,n)$).  For any real $(n-p)$-plane
$\Pi_j$, compactness guarantees the existence of a calibrated plane
$\widetilde\Pi_j \in K_{n-p}(x)$ minimizing the Grassmannian distance:
\[
\widetilde\Pi_j := \arg\min_{P \in K_{n-p}(x)} \angle(\Pi_j, P).
\]
Since $K_{n-p}(x)$ spans the full complex Grassmannian (every complex
$(n-p)$-plane is calibrated), and $\Pi_j$ arises from a Carath\'eodory
decomposition of $\beta(x) \in K_p(x)$, we have
$\angle(\Pi_j, \widetilde\Pi_j) \le \eta$ for some $\eta > 0$ controlled
by the $C^0$-norm of $\beta$.
Choose $\eta \le \varepsilon/2$ so that sheets with tangent plane
$\widetilde\Pi_j$ automatically satisfy
$\angle(T_y Y_j^a, \Pi_j) < \varepsilon$.

\medskip\noindent
\textbf{Substep 3.3: Choose sheet counts via Diophantine rounding.}
For fixed $j$, all parallel copies of $\widetilde\Pi_j$ have identical
$\psi$-mass $A_j>0$ in $Q$.  With $N_j$ sheets, the total mass in family
$j$ is $N_jA_j$.  Define
\[
\lambda_j:=\frac{\theta_j}{A_j},\qquad \Lambda:=\sum_i\lambda_i.
\]
For large integer $m$, set
\[
N_j(m):=\Bigl\lfloor m\frac{\lambda_j}{\Lambda}\Bigr\rfloor.
\]
Standard rounding estimates give
\[
\Bigl|N_j(m)-m\frac{\lambda_j}{\Lambda}\Bigr|\le 1,
\]
and hence
\[
\Bigl|\frac{N_j(m)A_j}{\sum_i N_i(m)A_i}-\theta_j\Bigr|=O\Bigl(\frac{1}{m}\Bigr).
\]
Choose $m$ so large that this error is $<\delta$.

\medskip\noindent
\textbf{Substep 3.4: Build flat model sheets with disjoint translations.}
In $\Phi(Q)\subset\C^n$, for each $j$, let $N_j^\perp$ be the complex
$p$-dimensional normal space (the complex orthogonal complement of
$\widetilde\Pi_j$), so that $\C^n=\widetilde\Pi_j\oplus N_j^\perp$.
Pick distinct translation vectors
$t_{j,1},\ldots,t_{j,N_j}\in N_j^\perp$ in a small ball $B(0,\rho)$
with $\rho\ll\mathrm{diam}(Q)$, such that all affine spaces
$\widetilde\Pi_j+t_{j,a}$ are pairwise disjoint on $\Phi(Q)$ as
$(j,a)$ ranges over all indices.  This is possible since $N_j^\perp$
has real dimension $2p\ge 2$ and we choose only finitely many points.

Define
\[
\widetilde Y_j^a:=(\widetilde\Pi_j+t_{j,a})\cap\Phi(Q)\subset\C^n.
\]
These satisfy: (i) $\psi_0$-calibration (complex $(n-p)$-planes);
(ii) $\sup_{y\in Q}\angle(T_y\widetilde Y_j^a,\Pi_j)
=\angle(\widetilde\Pi_j,\Pi_j)<\varepsilon$;
(iii) mass fractions within $\delta$ of $\theta_j$ by construction;
(iv) pairwise disjoint on $\Phi(Q)$;
(v) boundary supported on $\partial\Phi(Q)$.

\medskip\noindent
\textbf{Substep 3.5: Upgrade to algebraic complete intersections.}
Use Kodaira embedding and H\"ormander $L^2$-techniques: for large $k$,
pick global sections $s_{j,a}^{(1)},\ldots,s_{j,a}^{(p)}\in H^0(X,L^k)$
whose restrictions to $Q$ are $C^2$-close to the linear defining
functions of $\widetilde Y_j^a$.  For $k$ large:
\begin{itemize}
\item $Y_j^a:=\{s_{j,a}^{(1)}=0\}\cap\cdots\cap\{s_{j,a}^{(p)}=0\}$
is a smooth complex $(n-p)$-dimensional submanifold;
\item On $Q$, $Y_j^a$ is $C^1$-close to $\widetilde Y_j^a$;
\item Calibration, disjointness, and mass estimates persist under small
$C^1$ perturbations.
\end{itemize}
Pulling back by $\Phi^{-1}$ gives the desired family on $Q$.
\end{proof}

Fix a finite normal coordinate atlas by geodesic balls of radii $\ll 1$
and subordinate cubes $\{Q\}$ small enough so that the Carath\'eodory
data from Lemma~\ref{lem:caratheodory-general} are $\varepsilon$-stable
on each cube.  For each cube $Q$ and each index $j\in\{1,\ldots,N\}$,
let $\Pi_{Q,j}$ denote a constant complex $(n-p)$-plane approximating
$P_{x,j}$ on $Q$.  Apply Theorem~\ref{thm:local-sheets} to each cube
to obtain families $\{Y_{Q,j}^a\}$ of disjoint $\psi$-calibrated
complete intersections.

Define the local current
\[
S_Q := \sum_{j=1}^{N}\sum_{a=1}^{N_{Q,j}}[Y_{Q,j}^a]\llcorner Q.
\]
By construction, each $Y_{Q,j}^a$ is $\psi$-calibrated; hence $S_Q$ is a
positive $\psi$-calibrated integral current on $Q$.  Its tangent-plane
distribution on $Q$ is a convex combination of directions within
$\varepsilon$ of $\{\Pi_{Q,j}\}$ with weights proportional to the $\psi$--masses
in each family (equivalently proportional to $N_{Q,j}A_{Q,j}$, where $A_{Q,j}$ is
the $\psi$--mass of a single $(Q,j)$-sheet in $Q$).

\begin{lemma}[Local barycenter matching]\label{lem:local-bary}
For any $\delta>0$ there exist integers $N_{Q,1},\ldots,N_{Q,N}$ such that
the tangent-plane Young measure of $S_Q$ has barycenter within $\delta$
(in Hilbert--Schmidt norm) of the normalized field $\widehat\beta$ on $Q$, and
\[
\Mass(S_Q) \to m\int_Q \beta\wedge \psi
\quad \text{as }\delta\to 0.
\]
\end{lemma}

\begin{proof}
Let $A_{Q,j}>0$ denote the common $\psi$--mass of a single $(Q,j)$-sheet in $Q$
(all sheets in a fixed family $(Q,j)$ are local parallel translates, so their
mass in $Q$ agrees up to $o_\delta(1)$).
Choose integers $N_{Q,j}$ so that the \emph{mass fractions}
\[
\frac{N_{Q,j}A_{Q,j}}{\sum_i N_{Q,i}A_{Q,i}}
\]
approximate $\theta_{x,j}$ (nearly constant on $Q$) to within $O(\delta)$.
Then the resulting mass-weighted barycenter
\[
\sum_j \frac{N_{Q,j}A_{Q,j}}{\sum_i N_{Q,i}A_{Q,i}}\;\xi_{\Pi_{Q,j}}
\]
is within $\delta$ of $\widehat\beta$ on $Q$.
Because the tangent angles are $<\varepsilon$ and $\varepsilon\ll\delta$, the
Hilbert--Schmidt distance of barycenters is $\le C(\varepsilon+\delta)$.

Finally, calibratedness gives
$\Mass([Y_{Q,j}^a]\llcorner Q)=\int_Q\psi\llcorner[Y_{Q,j}^a]$, hence
\[
\Mass(S_Q)=\sum_j N_{Q,j}A_{Q,j}.
\]
By scaling the $N_{Q,j}$ simultaneously (and then rounding), one can arrange
$\sum_j N_{Q,j}A_{Q,j}\to m\int_Q \beta\wedge\psi$ as $\delta\to 0$.
\end{proof}

% ------------------------------------------------------------
\subsection*{Step 4: Global cohomology quantization (Theorem C)}

This step forces the global integral current to represent exactly the
correct homology class $\mathrm{PD}(m[\gamma])$ by using lattice
discreteness.

\begin{theorem}[Global cohomology quantization]\label{thm:global-cohom}
Let $X$ be a compact K\"ahler $n$-fold with rational Hodge class
$[\gamma]\in H^{2p}(X,\Q)$ represented by a smooth closed $(p,p)$-form
$\beta$ with $\beta(x)\in K_p(x)$ pointwise.  Let $\{Q\}$ be a cube
partition of $X$.  Then there exists an integer $m\ge 1$ (clearing denominators of
$[\gamma]$) such that for every $\varepsilon>0$ there exist:
\begin{itemize}
\item A closed integral $(2n-2p)$-current $T_\varepsilon$ with
$[T_\varepsilon]=\mathrm{PD}(m[\gamma])$;
\item A correction current $R_\varepsilon$ with $\Mass(R_\varepsilon)<\varepsilon$;
\end{itemize}
such that the local tangent-plane mass proportions on each $Q$ match
those of $\beta$ up to error $o_{\varepsilon\to 0}(1)$.
\end{theorem}

\begin{proof}
The proof proceeds in three substeps.

\medskip\noindent
\textbf{Substep 4.1: Local quantization.}
Choose the partition $\{Q\}$ fine enough that on each $Q$, $\beta(x)$
is within $\delta$ (in operator norm) of $\beta(x_Q)$ for a base point
$x_Q\in Q$, and the K\"ahler metric is nearly constant (Jacobian and
volume distortion $\le 1+\delta$).

By Lemma~\ref{lem:caratheodory-general}, write
\[
\beta(x_Q)=t_Q\sum_{j=1}^{J(Q)}\theta_{Q,j}\,\xi_{Q,j},
\qquad
t_Q:=\langle \beta(x_Q),\psi_{x_Q}\rangle,
\]
where $\xi_{Q,j}\in K_p(x_Q)$ are normalized extremal generators (coming from
complex $(n-p)$-planes) satisfying $\langle \xi_{Q,j},\psi_{x_Q}\rangle=1$,
the weights satisfy $\theta_{Q,j}\ge 0$, $\sum_j\theta_{Q,j}=1$, and
$J(Q)\le N=N(n,p)$ uniformly bounded.

Since $[\gamma]$ is rational, all its periods lie in $(1/M)\Z$ for some
fixed $M$.  Choose $m\gg 1$ divisible by $M$.

Let $P_{Q,j}\subset T_{x_Q}X$ be the complex $(n-p)$-plane corresponding to $\xi_{Q,j}$.
In the flattened model on $Q$, any affine $\psi$--calibrated sheet with tangent plane
$P_{Q,j}$ has the same $\psi$--mass in $Q$; denote this common value by $A_{Q,j}>0$
(it depends on the cube geometry and direction but satisfies $A_{Q,j}\asymp \mathrm{side}(Q)^{2(n-p)}$).
The target $\psi$--mass in $Q$ is
\[
M_Q := m\int_Q \beta\wedge\psi \;\approx\; m\,t_Q\,\mathrm{Vol}(Q),
\]
up to $O(\delta)$ error from the $C^0$--variation of $\beta$ on $Q$ and the
metric distortion.

Choose integers $N_{Q,j}\ge 0$ so that simultaneously
\[
\Bigl|\frac{N_{Q,j}A_{Q,j}}{\sum_i N_{Q,i}A_{Q,i}}-\theta_{Q,j}\Bigr|\le \delta
\qquad\text{and}\qquad
\Bigl|\sum_j N_{Q,j}A_{Q,j}-M_Q\Bigr|\le \delta\,M_Q.
\]
(Such choices exist by rounding, since the unknowns enter linearly and $m$ may be
taken arbitrarily large.)

Apply Theorem~\ref{thm:local-sheets} to realize each direction $(Q,j)$ by a family
of $\psi$--calibrated sheets $Y_{Q,j}^a\subset Q$ ($a=1,\ldots,N_{Q,j}$) with
angle control, disjointness on $Q$, and boundary supported on $\partial Q$.

Define the raw local current
\[
S_Q:=\sum_{j=1}^{J(Q)}\sum_{a=1}^{N_{Q,j}}[Y_{Q,j}^a]\llcorner Q.
\]

\medskip\noindent
\textbf{Substep 4.2: Gluing across cubes.}
Consider the global raw current
\[
T^{\mathrm{raw}}:=\sum_Q S_Q.
\]
This is integral but not closed: $\partial T^{\mathrm{raw}}$ lives on
the union of cube faces.  View the cube adjacency as a finite graph:
vertices $=$ cubes $Q$, edges $=$ codimension-1 faces $F=Q\cap Q'$.
On each oriented face $F$, the restriction of $\partial S_Q$ induces
a $(2n-2p-1)$-current $B_{Q\to F}$ living on $F$.  Summed over all cubes:
\[
\partial T^{\mathrm{raw}}=\sum_F B_F,
\]
where $B_F$ is the mismatch between the two neighboring cubes.

\textbf{Key point (flat norm, not mass):} In general the individual face currents $B_F$
need not have small mass (cancellation-heavy boundaries can have large mass), so the robust
quantity to control is the \emph{flat norm} of the total mismatch $\partial T^{\mathrm{raw}}$.
Recall the flat norm on $(2n-2p-1)$-currents:
\[
\mathcal F(S):=\inf\{\Mass(R)+\Mass(Q):\ S=R+\partial Q\},
\]
where $R$ is an integral $(2n-2p-1)$-current and $Q$ is an integral $(2n-2p)$-current.
On a compact manifold one has the dual characterization (Federer--Fleming):
\[
\mathcal F(S)=\sup\{S(\eta):\ \eta\in C^\infty\Lambda^{2n-2p-1},\ \|\eta\|_{\mathrm{comass}}\le 1,\
\|d\eta\|_{\mathrm{comass}}\le 1\}.
\]
For $S=\partial T^{\mathrm{raw}}$ and such $\eta$, Stokes gives
$S(\eta)=\partial T^{\mathrm{raw}}(\eta)=T^{\mathrm{raw}}(d\eta)$.

\begin{proposition}[Transport control $\Rightarrow$ flat-norm gluing]\label{prop:transport-flat-glue}
Fix a cubulation of $X$ by coordinate cubes of side length $h=\mathrm{mesh}$, and write
$T^{\mathrm{raw}}=\sum_Q S_Q$ as above, where each $S_Q$ is a sum of calibrated sheets restricted to $Q$.
Assume the following \emph{geometric parameterization} holds on each interior face $F=Q\cap Q'$:
\begin{enumerate}
\item[\textnormal{(a)}] (\textbf{Small-angle graph model}) For each cube $Q$ and each sheet family $(Q,j)$, the sheets crossing $F$
are $C^1$-graphs over a fixed calibrated reference plane $\Pi_{Q,j}$ with
$\sup_{y\in Q}\angle(T_yY_{Q,j}^a,\Pi_{Q,j})\le \varepsilon$.
\item[\textnormal{(b)}] (\textbf{Transverse measures on faces}) After identifying a tubular neighborhood of $F$ with a product
$F\times B^{2p}(0,ch)$ in normal coordinates, the restriction of $\partial S_Q$ to $F$ can be written as a finite sum of translated
slice currents parameterized by a discrete transverse measure $\mu_{Q\to F}$ on $B^{2p}(0,ch)$ (integer weights), and similarly for $Q'$.
\item[\textnormal{(c)}] (\textbf{$W_1$ face matching}) The two induced transverse measures have the same total mass and satisfy
\[
W_1(\mu_{Q\to F},\mu_{Q'\to F})\ \le\ \tau_F,
\]
where $W_1$ is the $1$-Wasserstein distance on $B^{2p}(0,ch)$.
\end{enumerate}
Then there exists a constant $C=C(n,p,X)$ such that for every smooth $(2n-2p-1)$-form $\eta$ with
$\|\eta\|_{\mathrm{comass}}\le 1$ and $\|d\eta\|_{\mathrm{comass}}\le 1$ one has the face estimate
\[
|B_F(\eta)|\ \le\ C\,h^{2n-2p-1}\,\bigl(\tau_F + \varepsilon\,\Mass(\mu_{Q\to F})\,h\bigr),
\]
and hence
\[
\mathcal F(B_F)\ \le\ C\,h^{2n-2p-1}\,\bigl(\tau_F + \varepsilon\,\Mass(\mu_{Q\to F})\,h\bigr).
\]
Consequently,
\[
\mathcal F\!\left(\partial T^{\mathrm{raw}}\right)
\ \le\ \sum_{F}\mathcal F(B_F)
\ \le\ C\,h^{2n-2p-1}\sum_F \tau_F\ +\ C\,\varepsilon\,h^{2n-2p}\sum_F \Mass(\mu_{Q\to F}).
\]
\end{proposition}

\begin{proof}[Proof sketch]
In the flat/parallel model ($\varepsilon=0$), each face slice defines a function
$f_\eta(y):=\Sigma_y(\eta)$ on the transverse parameter space, and Stokes on the cylinder between slices shows
$\mathrm{Lip}(f_\eta)\lesssim h^{2n-2p-1}\|d\eta\|_{\mathrm{comass}}\le Ch^{2n-2p-1}$.
Kantorovich--Rubinstein duality then yields
$|B_F(\eta)|\le \mathrm{Lip}(f_\eta)\,W_1(\mu_{Q\to F},\mu_{Q'\to F})\lesssim h^{2n-2p-1}\tau_F$.

For $\varepsilon>0$, compare each almost-parallel sheet to an exactly-parallel model slice in the same tubular chart.
The resulting error in $f_\eta$ is controlled by the $C^1$-graph distortion and contributes an additional
$O(\varepsilon\,h^{2n-2p})$ per sheet (hence the stated $\varepsilon\,\Mass(\mu_{Q\to F})\,h$ term).
Taking the supremum over test forms gives the flat-norm bound.
Finally sum over faces and use the triangle inequality for $\mathcal F$.
\end{proof}

\begin{remark}[Why hypotheses (a)--(b) hold for the local sheet model]\label{rem:transport-hypotheses}
In the flat model of Substep~3.4, each sheet in family $(Q,j)$ is literally an affine calibrated plane
$(\widetilde\Pi_{Q,j}+t_{j,a})\cap Q$, with translation parameter $t_{j,a}\in N_{Q,j}^\perp\cong\R^{2p}$.
For a fixed face $F\subset\partial Q$, the boundary slice current
\[
\Sigma_{F,j}(t):=\partial\big([\widetilde\Pi_{Q,j}+t]\llcorner Q\big)\llcorner F
\]
depends only on $t$ through its component normal to the $(2n-2p-1)$-plane $\widetilde\Pi_{Q,j}\cap TF$.
Thus, in the flat model, $\partial S_Q\llcorner F$ can be written as a finite sum
$\sum_a \Sigma_{F,j}(t_{j,a})$, i.e.\ it is parameterized by the discrete transverse measure
$\mu_{Q\to F}:=\sum_a \delta_{t_{j,a}}$ (with integer weights).

After upgrading to algebraic complete intersections in Substep~3.5, the sheets remain $C^1$-graphs over the flat model on $Q$
(for $k$ large), so the same parameterization persists in a tubular neighborhood of $F$ up to an $O(\varepsilon)$ error
controlled by the graph distortion.  This justifies the use of transverse measures on faces and the small-angle graph model
in Proposition~\ref{prop:transport-flat-glue}.

What is \emph{not} automatic is hypothesis (c): arranging $W_1$ matching across faces simultaneously for all cubes, subject to
the constraint that each sheet’s translation parameter determines its intersection with \emph{all} faces of $Q$ at once.
\smallskip
Equivalently, for a fixed cube $Q$ and family $(Q,j)$, the face measures $\mu_{Q\to F}$ for different faces $F\subset\partial Q$
are not independent choices: they arise as pushforwards of the \emph{same} discrete translation multiset $\{t_{j,a}\}$ under
the corresponding face-slice maps.  Thus the remaining task is a \emph{simultaneous} matching problem.
\end{remark}

\begin{lemma}[Automatic $W_1$-matching from smooth dependence of face maps]\label{lem:w1-auto}
Let $\mu$ be a finite Borel measure on $\R^{2p}$ supported in a ball of radius $O(h)$ and with total mass $\mu(\R^{2p})=N$.
Let $\Phi,\Phi':\R^{2p}\to\R^{2p}$ be linear maps with $\|\Phi-\Phi'\|_{\mathrm{op}}\le C\,h$.
Then
\[
W_1(\Phi_\#\mu,\Phi'_\#\mu)\ \le\ C\,h\int_{\R^{2p}}\|y\|\,d\mu(y)\ \le\ C'\,h^2\,N.
\]
\end{lemma}

\begin{proof}[Sketch]
Couple $\Phi_\#\mu$ and $\Phi'_\#\mu$ by pushing forward $\mu$ under $y\mapsto(\Phi y,\Phi' y)$ and bound the transport cost by
$\int\|\Phi y-\Phi' y\|\,d\mu\le \|\Phi-\Phi'\|_{\mathrm{op}}\int\|y\|\,d\mu$.
The support radius $O(h)$ gives $\int\|y\|\,d\mu\le O(h)\,\mu(\R^{2p})=O(h)N$.
\end{proof}

\begin{lemma}[Template stability under small multiset edits]\label{lem:w1-template-edit}
Let $\Omega\subset\R^{2p}$ be a bounded domain of diameter $\mathrm{diam}(\Omega)\le C h$.
Let $\mu=\sum_{a=1}^{N}\delta_{y_a}$ and $\mu'=\sum_{b=1}^{N}\delta_{y'_b}$ be two integer-weighted discrete measures on $\Omega$
with the \emph{same total mass} $N$.
Assume there is a matching of atoms such that $\|y_a-y'_a\|\le \Delta$ for all $a$ (after relabeling).
Then
\[
W_1(\mu,\mu')\ \le\ \Delta\,N.
\]
More generally, if $\mu'$ is obtained from $\mu$ by deleting $r$ atoms and inserting $r$ atoms (so total mass stays $N$), then
\[
W_1(\mu,\mu')\ \le\ r\cdot \mathrm{diam}(\Omega)\ \le\ C\,r\,h.
\]
\end{lemma}

\begin{proof}
For the first claim, couple $\mu$ and $\mu'$ by pairing each $y_a$ to $y'_a$; the transport cost is $\sum_a\|y_a-y'_a\|\le \Delta N$.
For the second claim, transport each deleted atom to an inserted atom at cost at most $\mathrm{diam}(\Omega)$ and keep the unchanged atoms fixed.
\end{proof}

\begin{remark}[How Lemma~\ref{lem:w1-auto} reduces the remaining matching task]\label{rem:w1-auto}
If, for each cube $Q$ and sheet family $(Q,j)$, we choose the translation multiset $\{t_{j,a}\}$ by a \emph{fixed} template in
$N_{Q,j}^\perp$ (e.g.\ a scaled lattice/low-discrepancy set of diameter $O(h)$), then across a shared face $F=Q\cap Q'$ the two
induced transverse measures are related by applying two nearby face-slice maps (coming from nearby plane directions and nearby normal-coordinate identifications).
Since $\beta$ is smooth, these maps differ by $O(h)$ in operator norm, so Lemma~\ref{lem:w1-auto} yields
\[
W_1(\mu_{Q\to F},\mu_{Q'\to F})\ \lesssim\ h^2\,N_F,
\]
where $N_F$ is the number of sheets contributing to that face.
Inserting this into Proposition~\ref{prop:transport-flat-glue} yields a global bound of the form
\[
\mathcal F(\partial T^{\mathrm{raw}})\ \lesssim\ m\,h^2 \;+\; O(\varepsilon\,m),
\]
so choosing $h=h(m)\to 0$ slowly (e.g.\ $h=m^{-\alpha}$ with $\alpha>0$ small) makes the gluing correction $R_{\mathrm{glue}}$
sublinear in $m$ and hence negligible in the mass equality as $m\to\infty$.
The remaining task is then to implement this “fixed template” choice while still meeting the cohomological constraints (Substep 4.3).
\end{remark}

\begin{remark}[Handling slowly varying multiplicities]\label{rem:w1-multiplicity}
In practice the number of sheets in a given family $(Q,j)$ will vary with $Q$ because the target weights depend on $\beta(x_Q)$.
If adjacent cubes $Q,Q'$ have sheet counts differing by $r=|N_{Q,j}-N_{Q',j}|$, one can view their face measures as arising from the
same template after $r$ insertions/deletions.  Lemma~\ref{lem:w1-template-edit} then gives an additional contribution
$W_1\lesssim r\,h$ (since the transverse domain has diameter $O(h)$).
Thus, once one has a quantitative bound $r\le C\,h\,N_{Q,j}$ (slow variation), this term is of order
$W_1\lesssim h^2 N_{Q,j}$ and is absorbed into the $h^2 N$ scaling of Lemma~\ref{lem:w1-auto}.
Making this “slow variation of integer counts” rigorous is a rounding/Diophantine bookkeeping problem, separate from the geometric transport estimates.
\end{remark}

\begin{lemma}[Slow variation under rounding of Lipschitz targets]\label{lem:slow-variation-rounding}
Let $\{Q\}$ be a cubulation of mesh $h$, and let $f: X\to\R_{\ge 0}$ be a Lipschitz function with constant
$\mathrm{Lip}(f)\le L$ on each chart used for the cubulation.
Fix $m\ge 1$ and set the target real counts
\[
n_Q := m\,h^{2p}\, f(x_Q),
\]
for chosen basepoints $x_Q\in Q$.
Define integer counts by nearest-integer rounding $N_Q:=\lfloor n_Q\rceil$.
Then for adjacent cubes $Q\sim Q'$ one has
\[
|N_Q-N_{Q'}|\ \le\ L\,m\,h^{2p+1}\ +\ 1.
\]
If moreover $f\ge f_0>0$ and $m\,h^{2p+1}\ge 2/f_0$, then there is a constant $C=C(L,f_0)$ such that
\[
|N_Q-N_{Q'}|\ \le\ C\,h\,N_Q.
\]
\end{lemma}

\begin{proof}
Nearest-integer rounding satisfies $|N_Q-N_{Q'}|\le |n_Q-n_{Q'}|+1$.
By the Lipschitz bound, $|f(x_Q)-f(x_{Q'})|\le L\,\mathrm{dist}(x_Q,x_{Q'})\le Lh$, hence
$|n_Q-n_{Q'}|\le m\,h^{2p}\cdot Lh = L\,m\,h^{2p+1}$, proving the first inequality.

If $f\ge f_0$, then $n_Q\ge m\,h^{2p} f_0$, so $N_Q\ge n_Q-1 \ge m\,h^{2p}f_0-1$.
Under $m\,h^{2p+1}\ge 2/f_0$ one has $m\,h^{2p}f_0\ge 2/h$, hence $N_Q\ge (1/h)$.
Therefore $1\le hN_Q$ and
\[
|N_Q-N_{Q'}|\le L\,m\,h^{2p+1}+1 \le \Bigl(\frac{L}{f_0}+1\Bigr)\,hN_Q,
\]
which yields the stated form.
\end{proof}
The local sheet construction is designed so that, uniformly for these test forms $d\eta$,
\[
T^{\mathrm{raw}}(d\eta)\approx \int_X (m\beta)\wedge d\eta,
\]
with an error controlled by $(\delta+\varepsilon+\mathrm{mesh}+1/m)\cdot m$.
Since $\beta$ is closed and $X$ has no boundary, $\int_X (m\beta)\wedge d\eta=\pm\int_X d(m\beta\wedge \eta)=0$.
Thus one expects a quantitative estimate
\[
\mathcal F\!\left(\partial T^{\mathrm{raw}}\right)\ \le\ \varepsilon_{\mathrm{glue}}(m,\delta,\varepsilon,\mathrm{mesh})\cdot m,
\qquad \varepsilon_{\mathrm{glue}}\xrightarrow[\delta,\varepsilon\to 0,\ \mathrm{mesh}\to 0,\ m\to\infty]{}0.
\]
Assuming such an estimate, by definition of $\mathcal F$ there exist integral currents
$R$ and $Q$ with $\partial T^{\mathrm{raw}}=R+\partial Q$ and $\Mass(R)+\Mass(Q)\le 2\mathcal F(\partial T^{\mathrm{raw}})$.
Moreover $R$ is a boundary (since $\partial T^{\mathrm{raw}}$ is), hence null-homologous; by the Federer--Fleming
isoperimetric inequality there exists an integral filling $Q_R$ with $\partial Q_R=R$ and
\[
\Mass(Q_R)\le C\,\Mass(R)^{\frac{2n-2p}{2n-2p-1}}.
\]
Setting
\[
R_{\mathrm{glue}}:=-(Q+Q_R)
\]
gives $\partial R_{\mathrm{glue}}=-\partial T^{\mathrm{raw}}$ and $\Mass(R_{\mathrm{glue}})$ as small as desired once
$\mathcal F(\partial T^{\mathrm{raw}})$ is small.
\begin{remark}[What remains to be proved here]\label{rem:glue-gap}
The estimates in Substep~4.2 require a quantitative link between
\emph{closedness} of $\beta$ and smallness of the \emph{boundary mismatch}
currents $B_F$ on faces.  Concretely, one needs a bound of the form
\[
\sum_F \Mass(B_F)\ \le\ \varepsilon_{\mathrm{glue}}(m,\delta,\mathrm{mesh})\cdot m,
\qquad \varepsilon_{\mathrm{glue}}\xrightarrow[\delta\to 0,\ \mathrm{mesh}\to 0]{}0,
\]
or (more robustly) a \emph{flat norm} estimate
\[
\mathcal{F}\!\left(\partial T^{\mathrm{raw}}\right)\ \xrightarrow[\delta\to 0,\ \mathrm{mesh}\to 0]{}0,
\]
from which one can produce a filling current with small mass.
Making this link fully rigorous is the core ``microstructure/gluing'' step of the SYR program.

\smallskip\noindent
\emph{One potentially viable route (transport in transverse parameters).}
In a flat chart, a large stack of (nearly) parallel calibrated sheets is naturally parameterized by its transverse translations.
On a shared face $F=Q\cap Q'$, the two neighboring cubes induce two discrete transverse measures; the mismatch current $B_F$
is the difference of the resulting face-slice currents.
Because the flat-norm dual constraint includes $\|d\eta\|_{\mathrm{comass}}\le 1$, boundary integrals against $\eta$ vary
Lipschitzly under small transverse shifts, so one expects
$
|B_F(\eta)|\lesssim W_1(\mu_F,\mu_F')
$
for the induced transverse measures, hence $\mathcal F(B_F)\lesssim W_1(\mu_F,\mu_F')$ in the flat/parallel model.
If one can choose sheet placements so that adjacent-face transverse measures match up to $W_1$-error $o(1)$ (using closedness of $\beta$
as the underlying ``conservation law''), then summing over faces yields the desired flat-norm estimate for $\partial T^{\mathrm{raw}}$.

\medskip\noindent
\textbf{Reducing the remaining heart to an integer transport/rounding problem.}
We now state a purely discrete target which, if achieved, feeds directly into
Proposition~\ref{prop:transport-flat-glue}.
Fix a mesh size $h$ and, for each interior face $F$, fix a transverse parameter domain
$\Omega_F\cong B^{2p}(0,ch)$ (normal coordinates) and a transverse grid of spacing $\delta\ll h$ on $\Omega_F$.
Let $\rho_F$ denote the \emph{target transverse density} induced by the smooth form $m\beta$ on the face
(i.e.\ the continuum limit of sheet counts per transverse parameter), so that $\int_{\Omega_F}\rho_F = O(mh^{2p})$ and
$\rho_F$ varies Lipschitzly at scale $h$ because $\beta$ is smooth.

\begin{proposition}[Integer transverse matching via grid quantization]\label{prop:integer-transport}
Assume that for every interior face $F=Q\cap Q'$ there exist \emph{integer-weighted} discrete measures
$\mu_{Q\to F}$ and $\mu_{Q'\to F}$ supported on the transverse grid in $\Omega_F$ such that:
\begin{enumerate}
\item[\textnormal{(i)}] (\textbf{Local accuracy}) $W_1(\mu_{Q\to F},\rho_F\,dy)\le C\,\delta\,\int_{\Omega_F}\rho_F$ and
$W_1(\mu_{Q'\to F},\rho_F\,dy)\le C\,\delta\,\int_{\Omega_F}\rho_F$;
\item[\textnormal{(ii)}] (\textbf{Mass conservation}) $\mu_{Q\to F}(\Omega_F)=\mu_{Q'\to F}(\Omega_F)$;
\item[\textnormal{(iii)}] (\textbf{Angle control}) the sheet stacks realizing these measures satisfy the small-angle model
in Proposition~\ref{prop:transport-flat-glue} with the same $\varepsilon$.
\end{enumerate}
Then $W_1(\mu_{Q\to F},\mu_{Q'\to F})\le 2C\,\delta\int_{\Omega_F}\rho_F$, hence
\[
\mathcal F(B_F)\ \le\ C'\,h^{2n-2p-1}\Bigl(\delta\int_{\Omega_F}\rho_F + \varepsilon\,\Mass(\mu_{Q\to F})\,h\Bigr),
\]
and consequently $\mathcal F(\partial T^{\mathrm{raw}})=o(m)$ as $h\to 0$ provided $\delta=o(h)$ and $\varepsilon=o(1)$.
\end{proposition}

\begin{proof}[Proof sketch]
Triangle inequality gives $W_1(\mu_{Q\to F},\mu_{Q'\to F})\le
W_1(\mu_{Q\to F},\rho_F dy)+W_1(\rho_F dy,\mu_{Q'\to F})$.
The face estimate and global summation are then exactly Proposition~\ref{prop:transport-flat-glue}.
The $o(m)$ conclusion follows from the scaling $\int_{\Omega_F}\rho_F=O(mh^{2p})$ and the fact that the number of faces is $O(h^{-2n+1})$.
\end{proof}

\begin{remark}[How to produce the discrete measures $\mu_{Q\to F}$]\label{rem:integer-transport-algo}
At the purely combinatorial level, one can proceed as follows.
For each face $F$, quantize the target density $\rho_F$ on a transverse grid of spacing $\delta$ by assigning each grid cell $C$
the real weight $w_C:=\int_C\rho_F$ and placing that weight at the cell center (this gives $W_1=O(\delta\int\rho_F)$).
Then scale by $m$ and round the weights to integers (sheet counts).  Because $m$ can be taken arbitrarily large, the rounding error
can be arranged to be $o(m)$ at fixed $(h,\delta)$.

Finally, enforce the exact mass conservation constraint (ii) simultaneously across all faces by solving an \emph{integer flow problem}
on the cube adjacency graph at each transverse grid point (or grid cell): view each oriented face as an edge carrying an integer
``flux'' (number of sheets) and adjust by a bounded amount to make opposing orientations match.  Standard integrality of network flows
on finite graphs produces an integer solution provided the total demands are integral (ensured by the choice of $m$).

The geometric difficulty is not this discrete step but realizing the resulting face measures by actual calibrated sheets with the required angle control.
\end{remark}
\end{remark}
Choose the partition and $m$ so that
$\Mass(R_{\mathrm{glue}})\le\varepsilon/2$.  Define
\[
T^{(1)}:=T^{\mathrm{raw}}+R_{\mathrm{glue}}.
\]
Then $T^{(1)}$ is closed and integral.

\medskip\noindent
\textbf{Substep 4.3: Forcing the cohomology class via lattice discreteness.}
Fix a basis of harmonic $(2n-2p)$-forms $\{\eta_\ell\}_{\ell=1}^b$
that generate $H^{2n-2p}(X,\Z)$.  The homology class of any closed
integral current $T$ is determined by the pairings
\[
\langle[T],[\eta_\ell]\rangle=\int_T\eta_\ell.
\]
Since $[\gamma]$ is rational, for each integral cohomology generator $\eta_\ell$
the period
\[
I_\ell:=\int_X \beta\wedge \eta_\ell\in\Q
\]
has bounded denominator.  Choose $m\ge 1$ so that $m\,I_\ell\in\Z$ for all $\ell$.

\begin{lemma}[Fixed-dimension discrepancy rounding (B\'ar\'any--Grinberg)]\label{lem:barany-grinberg}
Let $d\ge 1$ and let $v_1,\dots,v_M\in\R^d$ satisfy $\|v_i\|_{\ell^\infty}\le 1$.
For any coefficients $a_1,\dots,a_M\in[0,1]$, there exist $\varepsilon_1,\dots,\varepsilon_M\in\{0,1\}$ such that
\[
\Bigl\|\sum_{i=1}^M (\varepsilon_i-a_i)\,v_i\Bigr\|_{\ell^\infty}\ \le\ d.
\]
\end{lemma}

\begin{remark}
Lemma~\ref{lem:barany-grinberg} is a standard “rounding in fixed dimension” discrepancy estimate
(see B\'ar\'any--Grinberg, \emph{On some combinatorial questions in finite-dimensional vector spaces}, 1981).
The key feature is that the bound depends only on the dimension $d$, not on $M$.
\end{remark}

By refining the cube decomposition (so each individual sheet piece has very small contribution
to each pairing) and choosing the integers $N_{Q,j}$ using Lemma~\ref{lem:barany-grinberg}
(applied to the fractional parts of the target real counts), one can ensure that for all $\ell$,
\[
\Bigl|\int_{T^{\mathrm{raw}}}\eta_\ell - m\,I_\ell\Bigr|<\tfrac12.
\]
Moreover, the gluing correction $R_{\mathrm{glue}}$ has arbitrarily small mass, hence
its pairing with each fixed smooth $\eta_\ell$ is arbitrarily small:
$\bigl|\int_{R_{\mathrm{glue}}}\eta_\ell\bigr|\le \|\eta_\ell\|_{C^0}\Mass(R_{\mathrm{glue}})$.
Choosing parameters so that this error is $<\tfrac12$ as well yields
\[
\Bigl|\int_{T^{(1)}}\eta_\ell - m\,I_\ell\Bigr|<1,
\qquad T^{(1)}=T^{\mathrm{raw}}+R_{\mathrm{glue}}.
\]
Since $\int_{T^{(1)}}\eta_\ell\in\Z$ (integral current against an integral class),
we conclude $\int_{T^{(1)}}\eta_\ell = m\,I_\ell$ for all $\ell$.
Hence
\[
[T^{(1)}]=\mathrm{PD}(m[\gamma]).
\]

Set $R_\varepsilon:=R_{\mathrm{glue}}$ (plus any additional small
fillings), and $T_\varepsilon:=T^{(1)}$.  This satisfies all requirements.
\end{proof}

Let $\{\Theta_\ell\}_{\ell=1}^{b}$ be a fixed integral basis of
$H^{2(n-p)}(X,\Z)$ represented by smooth closed forms.  Since $\beta$
represents $[\gamma]$, we have for every $\ell$,
\[
I_\ell := \int_X \beta\wedge \Theta_\ell
= \langle [\gamma], [\Theta_\ell]\rangle \in \Q.
\]
Choose a common positive integer multiplier $m=m(\gamma)$ so that
$m\,I_\ell\in\Z$ for all $\ell$.

On each cube $Q$, the current $S_Q$ constructed above satisfies, for
each $\ell$,
\[
S_Q(\Theta_\ell)
= \sum_{j,a} \int_{Y_{Q,j}^a\cap Q} \Theta_\ell
= \int_Q \Bigl(\sum_{j}\tfrac{N_{Q,j}}{m_Q}\,\xi_{\Pi_{Q,j}}\Bigr)
  \wedge \Theta_\ell + O(\eta_Q),
\]
with $\eta_Q\to 0$ as $\varepsilon,\delta\to 0$.  Summing over all cubes yields
\[
\sum_Q S_Q(\Theta_\ell)
= \int_X \beta\wedge \Theta_\ell + O\Bigl(\sum_Q \eta_Q\Bigr).
\]

\begin{proposition}[Integral cohomology constraints]\label{prop:cohomology-match}
Given $\epsilon>0$, by refining the cube decomposition and choosing the
integers $N_{Q,j}$ appropriately, one can achieve simultaneously for all
$\ell=1,\ldots,b$ that
\[
\biggl|\sum_Q S_Q(\Theta_\ell) - m\,I_\ell\biggr| < \tfrac12.
\]
Consequently, by integrality, $\sum_Q S_Q(\Theta_\ell) = m\,I_\ell$ for
all $\ell$, i.e., the class of $\sum_Q S_Q$ in $H_{2(n-p)}(X,\Z)$ equals
$\mathrm{PD}(m[\gamma])$.
\end{proposition}

\begin{proof}
We make the fixed-dimension rounding in Substep~4.3 explicit.

\smallskip\noindent
\textbf{Step 1: Real targets and a $0$--$1$ rounding form.}
For each $(Q,j)$, let $n_{Q,j}\in\R_{\ge 0}$ denote the \emph{target} real sheet count dictated by the local weights
(so that $\sum_{Q,j} n_{Q,j}\,[Y_{Q,j}]\llcorner Q$ would give the correct pairings with all $\Theta_\ell$).
Write
\[
n_{Q,j}= \lfloor n_{Q,j}\rfloor + a_{Q,j},\qquad a_{Q,j}\in[0,1),
\]
and choose integers of the form
\[
N_{Q,j}:=\lfloor n_{Q,j}\rfloor + \varepsilon_{Q,j},\qquad \varepsilon_{Q,j}\in\{0,1\}.
\]
Thus the rounding error is encoded by the $0$--$1$ choices $\varepsilon_{Q,j}$.

\smallskip\noindent
\textbf{Step 2: Vector contributions are uniformly small on a fine cubulation.}
For each $(Q,j)$ pick a representative sheet piece $Y_{Q,j}$ in $Q$.
Define the contribution vector in $\R^b$
\[
v_{Q,j}:=\Bigl(\int_{Y_{Q,j}\cap Q}\Theta_\ell\Bigr)_{\ell=1}^b.
\]
Since each $\Theta_\ell$ is smooth and $\Mass(Y_{Q,j}\cap Q)\asymp h^{2(n-p)}$, there is a constant $C_0$
depending on $\max_\ell\|\Theta_\ell\|_{C^0}$ such that
\[
\|v_{Q,j}\|_{\ell^\infty}\ \le\ C_0\,h^{2(n-p)}.
\]
Choose the mesh $h$ so small that $C_0\,h^{2(n-p)}\le \frac{1}{4b}$.

\smallskip\noindent
\textbf{Step 3: Apply B\'ar\'any--Grinberg.}
Apply Lemma~\ref{lem:barany-grinberg} in dimension $d=b$ to the normalized vectors
$\widetilde v_{Q,j}:=(4b)\,v_{Q,j}$ (so $\|\widetilde v_{Q,j}\|_{\ell^\infty}\le 1$) with coefficients $a_{Q,j}$.
This yields choices $\varepsilon_{Q,j}\in\{0,1\}$ such that
\[
\Bigl\|\sum_{Q,j} (\varepsilon_{Q,j}-a_{Q,j})\,\widetilde v_{Q,j}\Bigr\|_{\ell^\infty}\ \le\ b.
\]
Undoing the normalization gives
\[
\Bigl\|\sum_{Q,j} (\varepsilon_{Q,j}-a_{Q,j})\, v_{Q,j}\Bigr\|_{\ell^\infty}\ \le\ \frac{1}{4}.
\]
Equivalently, for every $\ell$,
\[
\Bigl|\sum_{Q,j} (N_{Q,j}-n_{Q,j})\,\int_{Y_{Q,j}\cap Q}\Theta_\ell\Bigr|\ \le\ \frac{1}{4}.
\]
Thus, provided the continuous targets $n_{Q,j}$ were chosen so that
$\sum_{Q,j} n_{Q,j}\int_{Y_{Q,j}\cap Q}\Theta_\ell$ equals $mI_\ell$ up to $<\frac14$ error (achieved by taking $\delta$ small in the local
Carath\'eodory approximation), we obtain
\[
\Bigl|\sum_Q S_Q(\Theta_\ell)-mI_\ell\Bigr|<\frac12
\qquad\text{for all }\ell=1,\dots,b.
\]
The integrality conclusion is then as stated.
\end{proof}

% ------------------------------------------------------------
\subsection*{Step 5: Boundary correction with vanishing mass}

The sum $S:=\sum_Q S_Q$ is supported in the union of cubes and typically
has a small boundary supported on the inter-cube faces.  By the
Federer--Fleming Deformation Theorem (see Federer, \emph{Geometric Measure
Theory}, Theorem~4.2.9) and the isoperimetric inequality on compact
Riemannian manifolds, there exist integral $(2n-2p)$-currents
$U_\epsilon$ with
\[
\partial U_\epsilon = \partial S,
\qquad
\Mass(U_\epsilon)\xrightarrow[\epsilon\to 0]{}0.
\]
Define the closed integral current
\[
T_\epsilon := S - U_\epsilon,
\qquad \partial T_\epsilon=0.
\]
By construction, the homology class
$[T_\epsilon]=[S]=\mathrm{PD}(m[\gamma])$
(Proposition~\ref{prop:cohomology-match}).  Moreover, calibratedness
of the $S_Q$ pieces gives
\[
\Mass(T_\epsilon)
\le \Mass(S) + \Mass(U_\epsilon)
\to m\int_X \beta\wedge \psi,
\]
since $\Mass(U_\epsilon)\to 0$.

% ------------------------------------------------------------
\subsection*{Step 6: SYR realization via varifold compactness (Theorem D)}

This step establishes that the limit of the approximating cycles is
$\psi$-calibrated and realizes the SYR property.

\begin{theorem}[SYR Realization]\label{thm:syr-realization}
Under the hypotheses of Theorems~\ref{thm:local-sheets} and
\ref{thm:global-cohom} (with $\varepsilon,\delta\to 0$ and cube size
$\to 0$), the sequence $T_\varepsilon$ has:
\begin{enumerate}
\item[\textnormal{(i)}] $\Mass(T_\varepsilon)\to m\int_X\beta\wedge\psi$;
\item[\textnormal{(ii)}] Tangent-plane Young measures $\nu_x^{(\varepsilon)}$
converging a.e.\ to a measurable field $\nu_x$ supported on $\psi$-calibrated
planes with barycenter $\int\xi_P\,d\nu_x(P)=\widehat\beta(x)$;
\item[\textnormal{(iii)}] A subsequential limit $T$ that is $\psi$-calibrated
and represents $\mathrm{PD}(m[\gamma])$.
\end{enumerate}
In particular, $\beta$ is SYR-realizable.
\end{theorem}

\begin{proof}
The proof proceeds in four substeps.

\medskip\noindent
\textbf{Substep 6.1: Uniform mass bound and homology class.}
From Theorems~\ref{thm:local-sheets} and \ref{thm:global-cohom}, we have
\[
\Mass(T_k)\le m\int_X\beta\wedge\psi+o(1),
\]
where $T_k:=T_{1/k}$.  By the calibration inequality applied to any
cycle $S$ in class $\mathrm{PD}(m[\gamma])$:
\[
\Mass(S)\ge\langle[S],[\psi]\rangle=\langle\mathrm{PD}(m[\gamma]),[\psi]\rangle
=m\int_X\gamma\wedge\psi=m\int_X\beta\wedge\psi.
\]
Thus $\Mass(T_k)\ge m\int_X\beta\wedge\psi-o(1)$ as well.  We conclude:
\begin{itemize}
\item $\sup_k\Mass(T_k)<\infty$;
\item All $T_k$ are cycles: $\partial T_k=0$;
\item Their homology class is constant: $[T_k]=\mathrm{PD}(m[\gamma])$.
\end{itemize}
This is the compactness/normalization needed for Federer--Fleming.

\medskip\noindent
\textbf{Substep 6.2: Varifold compactness.}
Let $V_k$ be the associated integral varifold of $T_k$.  Uniform mass
bound gives tightness; Allard's compactness theorem (Allard, ``On the
first variation of a varifold,'' Ann.~of Math.~95 (1972), 417--491)
gives, after passing to a subsequence (not relabeled):
\begin{itemize}
\item $V_k\to V$ as varifolds;
\item $T_k\to T$ as integral currents in the flat norm;
\item $T$ is an integral $(2n-2p)$-cycle with $\partial T=0$;
\item By homological continuity, $[T]=\mathrm{PD}(m[\gamma])$ (since
$T_k$ and $T$ differ by a boundary and cohomology is discrete).
\end{itemize}
Lower semicontinuity gives
\begin{equation}\label{eq:mass-lsc}
\Mass(T)\le\liminf_{k\to\infty}\Mass(T_k)\le m\int_X\beta\wedge\psi.
\end{equation}

\medskip\noindent
\textbf{Substep 6.3: Tangent-plane Young measures.}
For each $k$, the tangent planes of $T_k$ around $x$ induce a probability
measure $\nu_x^{(k)}$ on $\Gr_{n-p}(T_xX)$, where $\mu_k$ denotes the mass measure of $T_k$.

\medskip\noindent
\emph{Calibration deficit forces concentration on calibrated planes.}
Since $[T_k]=\mathrm{PD}(m[\gamma])$ and $\psi$ is closed, the cohomological pairing gives
\[
\int_{T_k}\psi=\langle[T_k],[\psi]\rangle=\langle\mathrm{PD}(m[\gamma]),[\psi]\rangle
=m\int_X\beta\wedge\psi.
\]
By Substep~6.1, $\Mass(T_k)\to m\int_X\beta\wedge\psi$, hence the calibration deficit
\[
\Def_{\mathrm{cal}}(T_k):=\Mass(T_k)-\int_{T_k}\psi
\]
satisfies $\Def_{\mathrm{cal}}(T_k)\to 0$.
Equivalently (writing $V_k$ for the associated integral varifold),
\[
\Def_{\mathrm{cal}}(T_k)=\int_{X\times \Gr_{n-p}(TX)}\bigl(1-\psi(P)\bigr)\,dV_k(x,P)
=\int_X\int_{\Gr_{n-p}(T_xX)}\bigl(1-\psi(P)\bigr)\,d\nu_x^{(k)}(P)\,d\mu_k(x)\ \to\ 0.
\]
By the Wirtinger/K\"ahler-angle comparison (cf.\ the pointwise estimate
$1-\psi(P)\asymp \mathrm{dist}\bigl(P,K_{n-p}(x)\bigr)^2$ on the Grassmannian),
it follows that
\[
\int_X\int \mathrm{dist}\!\bigl(P,K_{n-p}(x)\bigr)^2\,d\nu_x^{(k)}(P)\,d\mu_k(x)\ \to\ 0.
\]

\medskip\noindent
\emph{Barycenter matching.}
Let
\[
b_k(x):=\int \xi_{\mathsf{proj}_{\mathrm{cal}}(P)}\,d\nu_x^{(k)}(P)\in K_p(x),
\]
where $\mathsf{proj}_{\mathrm{cal}}(P)$ denotes any measurable choice of a nearest
$\psi$--calibrated plane to $P$ in the Grassmannian, and $\xi_{\mathsf{proj}_{\mathrm{cal}}(P)}$
is the corresponding normalized generator (so $\langle\xi_{\mathsf{proj}_{\mathrm{cal}}(P)},\psi_x\rangle=1$).
By construction (Lemma~\ref{lem:local-bary}) and the fact that the gluing corrections have vanishing
relative mass, one has the $L^1(\mu_k)$-convergence
\[
\int_X \|b_k(x)-\widehat\beta(x)\|\,d\mu_k(x)\ \to\ 0.
\]
Since the Grassmann bundle is compact and the $\mu_k$ have uniformly
bounded total mass, standard Young-measure compactness gives, after
passing to a subsequence:
\begin{itemize}
\item $\nu_x^{(k)}\rightharpoonup\nu_x$ weak-$*$ for $\mu$-a.e.\ $x$,
where $\mu$ is the limit mass measure of $T$;
\item The limit field $x\mapsto\nu_x$ is measurable.
\end{itemize}
Passing to the limit in the cone-defect estimate gives:
\[
\int_X\int \mathrm{dist}\!\bigl(P,K_{n-p}(x)\bigr)^2\,d\nu_x(P)\,d\mu(x)=0,
\]
so for $\mu$-a.e.\ $x$, $\mathrm{supp}\,\nu_x\subset K_{n-p}(x)$.

Passing to the limit in the barycenter identity gives:
\[
\int\xi_P\,d\nu_x(P)=\widehat\beta(x)\quad\text{for }\mu\text{-a.e.\ }x.
\]
This is the SYR Young-measure condition.

\medskip\noindent
\textbf{Substep 6.4: Calibration of the limit.}
By the support condition, $\psi(\xi_P)=1$ for $\nu_x$-almost every $P$, so
\[
\int\psi(\xi_{T_yT})\,d|T|(y)
=\int_X\int\psi(\xi_P)\,d\nu_x(P)\,d\mu(x)
=\int_X 1\,d\mu(x)=\Mass(T).
\]
Thus the calibration inequality is actually an equality for $T$, so
$T$ is $\psi$-calibrated almost everywhere.

Combining with \eqref{eq:mass-lsc}:
\[
\Mass(T)=m\int_X\beta\wedge\psi,
\]
and $[T]=\mathrm{PD}(m[\gamma])$.

\textbf{Conclusion:} We have established:
\begin{enumerate}
\item Mass convergence: $\Mass(T_k)\to m\int_X\beta\wedge\psi$;
\item Young-measure convergence: $\nu_x^{(k)}\rightharpoonup\nu_x$ with
$\mathrm{supp}\,\nu_x\subset\{\psi\text{-calibrated planes}\}$ and
barycenter $\widehat\beta(x)$;
\item Limit cycle: $T$ is an integral $\psi$-calibrated $(2n-2p)$-cycle
with $[T]=\mathrm{PD}(m[\gamma])$.
\end{enumerate}
Thus $\beta$ is SYR-realizable.
\end{proof}

By the Harvey--Lawson structure theorem for calibrated currents
(Harvey--Lawson, ``Calibrated geometries,'' Acta Math.~148 (1982), 47--157),
$T$ is integration along a positive combination of irreducible complex
analytic subvarieties of codimension $p$.  This completes the proof that
cone-valued forms are SYR-realizable and hence algebraic.

% ------------------------------------------------------------
\subsection*{Addressing potential objections to the SYR construction}
% ------------------------------------------------------------

We address three potential objections to the construction above.

\begin{remark}[The ``density vs.\ mass'' objection]\label{rem:density-mass}
\textbf{Objection:} ``Integral cycles are supported on measure-zero sets,
while $\beta$ is non-zero everywhere.  To approximate $\beta$ everywhere,
the cycles would need infinite mass.''

\textbf{Response:} This objection rests on a fundamental misunderstanding
of what SYR accomplishes.  The construction does \emph{not} claim that
$T_k$ approximates $\beta$ as a measure on all of $X$.  Rather:
\begin{itemize}
\item Each $T_k$ is an integral $(2n-2p)$-cycle supported on a
$(2n-2p)$-dimensional set (a finite union of complex subvarieties).
\item The barycenter condition
$\int\xi_P\,d\nu_x(P)=\widehat\beta(x)$ holds for $\mu$-almost every $x$,
where $\mu$ is the \emph{mass measure of $T$}, not Lebesgue measure on $X$.
\item The currents $T_k$ and $T$ are supported on $(n-p)$-dimensional
complex subvarieties---this is exactly what we want for the Hodge
Conjecture.
\end{itemize}
The key insight is that $\widehat\beta$ prescribes the \emph{local tangent-plane
distribution} while the scalar field $t(x)=\langle \beta(x),\psi_x\rangle$ encodes
the target mass density in the approximation scheme; neither statement claims that
the cycles ``fill'' $X$ as subsets.  The support of $T$ is a positive combination
of complex subvarieties whose combined homology class is $\mathrm{PD}(m[\gamma])$.
\end{remark}

\begin{remark}[Harvey--Lawson applicability]\label{rem:hl-applicable}
\textbf{Objection:} ``The limit $T$ might be a smooth current (integration
against $\beta$), which is not rectifiable, so Harvey--Lawson doesn't apply.''

\textbf{Response:} This objection is factually incorrect.  The sequence
$\{T_k\}$ consists of \emph{integral cycles}---each $T_k$ is a finite
sum of integration currents over smooth complex subvarieties (the
complete intersections from Theorem~\ref{thm:local-sheets}).  By the
\emph{Federer--Fleming compactness theorem} (Federer--Fleming,
``Normal and integral currents,'' Ann.~of Math.~72 (1960), 458--520):
\begin{quote}
\emph{If $\{T_k\}$ is a sequence of integral currents with uniformly
bounded mass and boundary mass, then a subsequence converges in the
flat norm to an integral current $T$.}
\end{quote}
In our case:
\begin{itemize}
\item $\Mass(T_k)\le C$ uniformly (Substep 6.1);
\item $\partial T_k=0$ for all $k$ (they are cycles);
\item Hence the limit $T$ is an \emph{integral} current.
\end{itemize}
Integral currents are rectifiable by definition.  The limit $T$ is
\emph{not} a smooth current; it is a rectifiable current supported on
an $(n-p)$-rectifiable set with integer multiplicities.  Harvey--Lawson
applies to such currents when they are $\psi$-calibrated, which $T$ is.
\end{remark}

\begin{remark}[The gluing/non-integrability objection]\label{rem:gluing}
\textbf{Objection:} ``The plane field $x\mapsto\beta(x)$ is generically
non-integrable.  Local sheets cannot be glued without accumulating mass.''

\textbf{Response:} This objection conflates two different things:
\begin{enumerate}
\item[(a)] \emph{Integrating a plane field} into a single foliation
(which requires the Frobenius condition);
\item[(b)] \emph{Building many separate calibrated sheets} whose tangent
planes locally approximate a given decomposition.
\end{enumerate}
The construction does (b), not (a).  We are \emph{not} trying to find a
submanifold whose tangent planes equal $\beta(x)$ everywhere---that would
indeed require integrability.  Instead:
\begin{itemize}
\item On each cube $Q$, we decompose $\beta(x_Q)$ as a convex combination
of calibrated planes via Carath\'eodory.
\item We build finitely many \emph{separate, disjoint} calibrated
complete intersections through $Q$, each with a \emph{constant} tangent
plane (up to $\varepsilon$-error on the small cube).
\item The complete intersections are algebraic subvarieties---they exist
by Bertini's theorem, regardless of whether $\beta$ is integrable.
\end{itemize}
The non-integrability of $\beta$ as a plane field is irrelevant because
we never integrate it.  The ``gluing'' step (Theorem~\ref{thm:global-cohom},
Substep 4.2) uses Federer--Fleming to fill boundary mismatches.  The
key estimate is formulated in \emph{flat norm}:
\[
\mathcal F\!\left(\partial T^{\mathrm{raw}}\right)\ \le\ \varepsilon_{\mathrm{glue}}(m,\delta,\varepsilon,\mathrm{mesh})\cdot m,
\]
This is the robust target because the individual face mismatches can have large mass even when there is strong cancellation.
\medskip\noindent
Concretely, by the dual characterization of $\mathcal F$ and Stokes, for every smooth
$(2n-2p-1)$-form $\eta$ with $\|\eta\|_{\mathrm{comass}}\le 1$ and $\|d\eta\|_{\mathrm{comass}}\le 1$ one has
\[
\partial T^{\mathrm{raw}}(\eta)=T^{\mathrm{raw}}(d\eta)\approx \int_X (m\beta)\wedge d\eta.
\]
Since $\beta$ is closed and $X$ has no boundary, $\int_X (m\beta)\wedge d\eta=\pm\int_X d(m\beta\wedge\eta)=0$.
Thus the remaining task is to make the approximation error quantitative in terms of
$(\delta,\varepsilon,\mathrm{mesh},m)$; see Remark~\ref{rem:glue-gap}.
Once $\mathcal F(\partial T^{\mathrm{raw}})$ is small, the correction current $R_{\mathrm{glue}}$ is produced by
the flat-norm decomposition and the Federer--Fleming isoperimetric inequality as in Substep~4.2.
The smoothness of $\beta$ is essential here---it ensures the local
decompositions are compatible across cube boundaries.
\end{remark}

\begin{remark}[Why the construction succeeds]\label{rem:why-success}
The SYR construction succeeds because it exploits three key facts:
\begin{enumerate}
\item \textbf{Algebraic density:} By Bergman kernel asymptotics, any
calibrated plane at any point can be approximated by the tangent plane
of an algebraic complete intersection (Proposition~\ref{prop:tangent-approx-full}).
\item \textbf{Carath\'eodory decomposition:} Any cone-valued form $\beta(x)$
is a finite convex combination of calibrated planes, with uniformly
bounded number of terms (Lemma~\ref{lem:caratheodory-general}).
\item \textbf{Federer--Fleming compactness:} Integral cycles with bounded
mass converge to integral cycles, preserving rectifiability.
\end{enumerate}
The construction builds integral cycles $T_k$ that are finite unions of
algebraic subvarieties.  The limit $T$ is again an integral current (by
Federer--Fleming), and it is $\psi$-calibrated (by the mass equality
argument in Substep~6.4).  Harvey--Lawson then identifies $T$ as a
positive sum of complex subvarieties.

Critically, the form $\beta$ is \emph{never} the limit current.  The
limit $T$ is an algebraic cycle whose \emph{existence} is guaranteed by
compactness, whose \emph{homology class} is $\mathrm{PD}(m[\gamma])$ by
construction, and whose \emph{calibrated structure} follows from the
mass equality.
\end{remark}

% ------------------------------------------------------------
\subsection*{Automatic SYR: summary theorem}

\begin{theorem}[Automatic SYR for cone-valued forms]\label{thm:automatic-syr}
Let $(X,\omega)$ be a smooth projective K\"ahler manifold of complex
dimension $n$, and let $1\le p\le n$.  Every smooth closed cone-valued
$(p,p)$-form $\beta$ representing a rational Hodge class $[\gamma]$
satisfies the Stationary Young-measure Realizability property:
there exist $\psi$-calibrated integral $(2n-2p)$-cycles $T_k$ with
$\partial T_k=0$ and
\begin{enumerate}
\item[\textnormal{(i)}]
$\Mass(T_k)\to \displaystyle m\int_X \beta\wedge\psi$,
\item[\textnormal{(ii)}]
the tangent-plane Young measures of $T_k$ converge a.e.\ to a
measurable field $\nu_x$ supported on complex $(n-p)$-planes with
barycenter $\int \xi_P\,d\nu_x(P)=\widehat\beta(x)$,
\item[\textnormal{(iii)}]
$[T_k]=\mathrm{PD}(m[\gamma])$ for some fixed $m\in\N$ independent of $k$.
\end{enumerate}
Consequently, there exists a $\psi$-calibrated integral current $T$
representing $\mathrm{PD}(m[\gamma])$.  By Harvey--Lawson, $T$ is
integration along a positive sum of complex analytic subvarieties;
hence $[\gamma]$ is algebraic.
\end{theorem}

\begin{proof}
This is the content of Steps 1--6 above.  The construction produces
a sequence $T_k := T_{1/k}$ satisfying (i)--(iii), and the varifold
limit $T$ is $\psi$-calibrated with the stated properties.
\end{proof}

% ============================================================
\subsection*{Signed decomposition: the unconditional step}
% ============================================================

The preceding machinery applies to \emph{effective} classes---those admitting
cone-valued representatives.  The following lemma shows that \emph{every} rational
Hodge class reduces to this case.

\begin{definition}[Effective class]
A cohomology class $\gamma \in H^{2p}(X,\R) \cap H^{p,p}(X)$ is called
\emph{effective} if there exists a smooth closed $(p,p)$--form $\beta$
representing $\gamma$ such that $\beta(x) \in K_p(x)$ for all $x \in X$.
\end{definition}

\begin{lemma}[Positivity of the K\"ahler power]\label{lem:kahler-positive}
The $(p,p)$--form $\omega^p$ is strictly positive: for all $x \in X$,
$\omega^p(x) \in \mathrm{int}\,K_p(x)$.
In the Hermitian model, $\omega^p(x)$ corresponds to a positive definite
matrix $W(x)$ with $\lambda_{\min}(W(x)) \ge c_0 > 0$
for some constant $c_0 > 0$ depending only on $(X,\omega)$.
\end{lemma}

\begin{proof}
At each point $x$, choose unitary coordinates so that
$\omega(x) = \frac{i}{2}\sum_{j=1}^n dz_j \wedge d\bar{z}_j$.
Then $\omega^p(x)$ is a positive linear combination of simple $(p,p)$--forms,
each corresponding to a rank-one PSD matrix in the Hermitian model.
The sum is strictly positive definite.  By compactness of $X$ and smoothness
of $\omega$, the minimum eigenvalue is uniformly bounded below.
\end{proof}

\begin{lemma}[Signed Decomposition]\label{lem:signed-decomp}
Let $\gamma \in H^{2p}(X,\Q) \cap H^{p,p}(X)$ be any rational Hodge class.
Then there exist effective classes $\gamma^+$ and $\gamma^-$ such that
\[
\gamma \;=\; \gamma^+ - \gamma^-.
\]
Moreover, both $\gamma^+$ and $\gamma^-$ are rational Hodge classes,
and $\gamma^-$ can be taken to be a positive rational multiple of $[\omega^p]$.
\end{lemma}

\begin{proof}
Let $\alpha$ be any smooth closed $(p,p)$--form representing $\gamma$.
In the Hermitian model at each $x \in X$, $\alpha(x)$ corresponds to a
Hermitian matrix $A(x)$.  Define
\[
M \;:=\; \sup_{x \in X} \bigl|\lambda_{\min}(A(x))\bigr| \;<\; \infty,
\]
which is finite by compactness of $X$ and smoothness of $\alpha$.

By Lemma~\ref{lem:kahler-positive}, $\omega^p(x)$ corresponds to $W(x)$
with $\lambda_{\min}(W(x)) \ge c_0 > 0$.  Choose $N > M/c_0$.
Then for all $x \in X$:
\[
\lambda_{\min}\bigl(A(x) + N \cdot W(x)\bigr)
\;\ge\;
\lambda_{\min}(A(x)) + N \cdot \lambda_{\min}(W(x))
\;\ge\;
-M + N c_0 \;>\; 0.
\]
Thus $A(x) + N \cdot W(x)$ is positive definite, hence
$\alpha(x) + N \cdot \omega^p(x) \in K_p(x)$ for all $x \in X$.

Define $\gamma^+ := \gamma + N \cdot [\omega^p]$ and
$\gamma^- := N \cdot [\omega^p]$.
Then $\gamma = \gamma^+ - \gamma^-$ by construction,
$\gamma^+$ is effective (represented by the cone-valued form
$\alpha + N \cdot \omega^p$),
$\gamma^-$ is effective (represented by $N \cdot \omega^p$),
and both are rational Hodge classes since $[\omega^p] = c_1(L)^p$ is
rational for the ample bundle $L$.
\end{proof}

\begin{lemma}[$\gamma^-$ is algebraic]\label{lem:gamma-minus-alg}
On a smooth projective variety $X \subset \mathbb{P}^M$ with hyperplane
class $H = c_1(\mathcal{O}(1)|_X)$, the class $[\omega^p] = H^p$ is algebraic,
represented by a complete intersection of $p$ generic hyperplane sections.
\end{lemma}

\begin{proof}
By Bertini's theorem, for generic hyperplanes $H_1, \ldots, H_p$ in
$\mathbb{P}^M$, the intersection $Z := X \cap H_1 \cap \cdots \cap H_p$
is a smooth subvariety of codimension $p$ in $X$.  Its fundamental class
$[Z] \in H_{2n-2p}(X,\Z)$ satisfies $\mathrm{PD}([Z]) = H^p = [\omega^p]$.
Thus $[\omega^p]$ is algebraic, and $\gamma^- = N \cdot [\omega^p]$ is
algebraic for any rational $N > 0$.
\end{proof}

\begin{theorem}[Effective classes are algebraic]\label{thm:effective-algebraic}
Let $\gamma^+ \in H^{2p}(X,\Q) \cap H^{p,p}(X)$ be an effective rational
Hodge class on a smooth projective K\"ahler manifold.  Then $\gamma^+$
is algebraic.
\end{theorem}

\begin{proof}
Since $\gamma^+$ is effective, it admits a cone-valued representative $\beta$
with $\beta(x) \in K_p(x)$ for all $x$.  By Theorem~\ref{thm:automatic-syr},
$\beta$ is SYR--realizable.  Thus there exists a sequence of integral cycles
$T_k$ with $[T_k] = \mathrm{PD}(m[\gamma^+])$ and $\Mass(T_k) \to c_0$.
By Theorem~\ref{thm:realization-from-almost}, a subsequence converges to a
$\psi$--calibrated integral current $T$, which by Harvey--Lawson is a positive
sum of complex analytic subvarieties, hence algebraic by Chow's theorem.
\end{proof}

% ============================================================
\subsection*{Main theorem: unconditional Hodge conjecture}
% ============================================================

\begin{theorem}[Hodge Conjecture for rational $(p,p)$ classes]
\label{thm:main-hodge}
Let $X$ be a smooth projective K\"ahler manifold.  Every rational Hodge
class $\gamma \in H^{2p}(X,\Q) \cap H^{p,p}(X)$ is algebraic.
\end{theorem}

\begin{proof}
By Lemma~\ref{lem:signed-decomp}, write $\gamma = \gamma^+ - \gamma^-$
where $\gamma^+$ and $\gamma^- = N[\omega^p]$ are both effective rational
Hodge classes.

By Lemma~\ref{lem:gamma-minus-alg}, $\gamma^-$ is algebraic: it is
represented by a complete intersection $Z^-$.

By Theorem~\ref{thm:effective-algebraic}, $\gamma^+$ is algebraic:
it is represented by an algebraic cycle $Z^+$ obtained from the
calibration--coercivity/SYR construction.

Therefore:
\[
\gamma \;=\; \gamma^+ - \gamma^-
\;=\; [Z^+] - [Z^-],
\]
where $Z^+ - Z^-$ denotes the formal difference in the group of algebraic
cycles tensored with $\Q$.  Hence $\gamma$ is algebraic.
\end{proof}

\begin{corollary}[Full Hodge conjecture]\label{cor:full-hodge}
	Every rational $(p,p)$ class on a smooth projective K\"ahler manifold is represented
	by an algebraic cycle.
\end{corollary}

\begin{proof}
This is exactly Theorem~\ref{thm:main-hodge}.
\end{proof}

\begin{remark}[Why signed decomposition is the key]
The signed decomposition sidesteps the fundamental obstruction that the
harmonic representative $\gamma_{\mathrm{harm}}$ of a general Hodge class
need not be cone-valued.  For classes like $[\pi_1^*\omega_1] - [\pi_2^*\omega_2]$
on a product surface, the harmonic form has indefinite signature everywhere.
We do \emph{not} claim that every Hodge class has a cone-valued representative;
we only use that every Hodge class is a \emph{difference} of two that do.
This is trivially achieved by adding a large multiple of $[\omega^p]$, which
is strictly positive.
\end{remark}


\end{document}