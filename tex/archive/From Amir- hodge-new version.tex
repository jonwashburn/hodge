\documentclass[12pt]{article}

\usepackage{amsmath,amssymb,amsthm,amscd,mathrsfs}
\usepackage{fullpage}
\usepackage{hyperref}
\usepackage{enumitem}

% ------------------------------------------------------------
% Theorem environments
% ------------------------------------------------------------
\newtheorem{theorem}{Theorem}[section]
\newtheorem{lemma}[theorem]{Lemma}
\newtheorem{proposition}[theorem]{Proposition}
\newtheorem{corollary}[theorem]{Corollary}
\theoremstyle{definition}
\newtheorem{definition}[theorem]{Definition}
\newtheorem{remark}[theorem]{Remark}

% ------------------------------------------------------------
% Notation shortcuts
% ------------------------------------------------------------
\newcommand{\Cal}{\mathcal}
\newcommand{\C}{\mathbb{C}}
\newcommand{\R}{\mathbb{R}}
\newcommand{\HH}{\mathcal H}
\newcommand{\HS}{\mathrm{HS}}
\newcommand{\Cone}{\mathcal{C}}
\newcommand{\harm}{\mathrm{harm}}
\theoremstyle{plain}
\newtheorem{hypothesis}{Hypothesis}

% Hermitian identification
\newcommand{\PsiMap}{\Psi_x : \Lambda^{p,p} T^*_x X \to \mathrm{Herm}(\mathcal H)}

% ------------------------------------------------------------
% Title + Authors
% ------------------------------------------------------------
\title{Calibrated Geometry, Hermitian Rank-One Models, and a 
	Conditional Minimization Framework Toward the Hodge Conjecture}

\author{
	Jonathan Washburn\thanks{Recognition Science, Recognition Physics Institute,
		Austin,Texas,USA.Email: \texttt{jon@recognitionphysics.org}.}
	\and
	Amir Rahnamai Barghi\thanks{Concord, Ontario, Canada. Corresponding author.
		Email: \texttt{arahnamab@gmail.com}.}
}

\date{}

\begin{document}
	
	\maketitle
	
	\begin{abstract}
		This draft develops a calibrated, Hermitian, and variational framework for 
		analyzing real $(p,p)$-forms on compact Kähler manifolds. 
		Sections~1--7 establish \emph{fully rigorous} analytic and 
		finite-dimensional results:
		(1) a precise description of the calibrated Grassmannian and its cone geometry,
		(2) sharp type-control via elliptic estimates, 
		(3) a unitary isometry 
		$\Psi_x : \Lambda^{p,p}_\R \to \mathrm{Herm}(\HH)$,
		(4) quantitative rank-one approximation for primitive $(p,p)$-forms,
		(5) projection to the positive cone of Hermitian matrices,
		and (6) a local energy--defect coercivity inequality.
		
		Section~8 formulates a \emph{conditional global minimization program} 
		combining the analytic coercivity with 
		Siu--Yau--Ruan (SYR) realization and geometric measure theory.
		All conditional points are explicitly labeled (A1)--(A5).  
		This section does \emph{not} claim an unconditional proof of the 
		Hodge Conjecture; rather, it produces a logically transparent blueprint 
		identifying the precise geometric measure theoretic inputs 
		required for such a proof.
	\end{abstract}
	
	\tableofcontents
	
	% ============================================================
	\section{Introduction}
	% ============================================================
	
	Let $(X^{2n},\omega)$ be a compact Kähler manifold.  
	For any real $(p,p)$-form $\alpha$ representing a fixed cohomology class 
	$[\alpha]\in H^{p,p}(X;\R)$, we consider the \emph{Coulomb decomposition}
	\begin{equation}\label{eq:Coulomb}
		\alpha = \gamma_{\harm} + d\eta, \qquad d^*\eta = 0,
	\end{equation}
	and the associated \emph{Dirichlet energy}
	\[
	E(\alpha) := \| d\eta \|_{L^2}^2.
	\]
	The harmonic representative $\gamma_{\harm}$ minimizes $E$ among all 
	smooth representatives.
	
	A central theme of this manuscript is to analyze $\alpha$ through its 
	pointwise geometry relative to the \emph{calibrated Grassmannian}
	of complex $p$-planes inside $T_xX$.  
	The induced set of normalized calibrated $(p,p)$-forms forms a compact 
	subset of $\Lambda^{p,p}T_x^*X$, whose convex cone we denote by $\Cone_x$.
	This cone plays the role of a ``positivity model’’ for $(p,p)$-forms.
	
	In Sections 1--7 we construct a Hermitian model identifying 
	$\Lambda^{p,p}$ with $\mathrm{Herm}(\HH)$, prove a sharp 
	rank-one approximation estimate, and derive the local coercivity
	\[
	E(\alpha) - E(\gamma_{\harm})
	\ \ge\ c(n,p)\,\mathrm{Def}_{\rm cone}(\alpha)
	\]
	where the defect measures the distance of $\alpha$ to the calibrated cone.
	
	Section~8 then attempts to upgrade this \emph{local} inequality into a 
	\emph{global} minimization argument.  
	This is where the Hodge Conjecture enters.  
	We rewrite the entire section as a \emph{conditional program}:  
	all analytic inputs are rigorous, but several geometric measure theory 
	ingredients required for the final conclusion are labeled as 
	assumptions (A1)--(A5).
	
	% ============================================================
	\section{Preliminaries}
	% ============================================================
	
	We briefly summarize notation, geometric structures, and analytic tools.
	
	% ------------------------------------------------------------
	\subsection{Kähler geometry and type decomposition}
	% ------------------------------------------------------------
	
	Let $X$ be a compact Kähler manifold of complex dimension $n$ with 
	Kähler form $\omega$.  
	We use the usual type decomposition
	\[
	\Lambda^k T^*X \otimes \C
	= \bigoplus_{p+q=k} \Lambda^{p,q} T^*X,
	\]
	and the real subspace 
	$\Lambda^{p,p}_\R := \{\alpha\in \Lambda^{p,p}: \overline{\alpha}=\alpha\}$.
	
	The inner product on forms at a point is the one induced from the metric,
	\[
	\langle \alpha,\beta\rangle = \frac{1}{p!p!}\,
	\alpha_{i_1\dots i_p\bar j_1\dots \bar j_p}\,
	\overline{\beta_{i_1\dots i_p\bar j_1\dots \bar j_p}}.
	\]
	
	% ------------------------------------------------------------
	\subsection{Calibrated $(p,p)$-forms}
	% ------------------------------------------------------------
	
	A unit decomposable $(p,p)$-form $\phi$ arising from a complex $p$-plane
	$V\subset T_xX$ (that is, $\phi = \frac{1}{p!}\omega|_V^p$) defines a 
	calibration in the sense of Harvey--Lawson.  
	Let $\mathrm{Gr}_{\C}(p,n)$ denote the Grassmannian of complex $p$-planes.
	The map
	\[
	V\in \mathrm{Gr}_\C(p,n) \mapsto \phi_V \in \Lambda^{p,p}_\R T_x^*X
	\]
	has image a compact subset $\mathcal{G}_x$.
	Define the \emph{calibrated cone}:
	\[
	\Cone_x := 
	\left\{
	\sum_{j=1}^N t_j \phi_{V_j} : t_j\ge 0,\, N<\infty
	\right\}.
	\]
	In the revised manuscript we prove (using your corrected supplemental 
	PDF) that $\Cone_x$ is \emph{closed} in $\Lambda^{p,p}_\R$; 
	see Section~3 for the precise argument.
	
	% ------------------------------------------------------------
	\subsection{Hodge theory}
	% ------------------------------------------------------------
	
	Given a fixed cohomology class $[\alpha]\in H^{p,p}(X;\R)$, the harmonic 
	representative $\gamma_{\harm}$ is characterized by
	\[
	d\gamma_{\harm} = d^*\gamma_{\harm} = 0.
	\]
	Every $\alpha$ in the same class can be written uniquely as
	$\alpha = \gamma_{\harm} + d\eta$ with $d^*\eta=0$.
	On compact Kähler manifolds we have the elliptic estimate
	\[
	\|\eta\|_{H^1} \le C(n,p)\,\| d\eta\|_{L^2}.
	\]
	
	% ------------------------------------------------------------
	\subsection{Hermitian matrices}
	% ------------------------------------------------------------
	
	Let $\HH = \Lambda^{p,0}T_x^*X$, a complex vector space of dimension 
	$d = \binom{n}{p}$.  
	We denote by $\mathrm{Herm}(\HH)$ the real vector space of Hermitian 
	endomorphisms of $\HH$, equipped with the Hilbert–Schmidt inner product.
	
	Later sections construct a unitary identification
	\[
	\PsiMap
	\]
	which becomes the backbone of our rank-one approximation theory.
	
	% ============================================================
	\section{Calibrated Grassmannian and Cone Geometry}
	% ============================================================
	
	In this section we develop the pointwise geometry of calibrated $(p,p)$--forms
	and prove, using the corrected arguments from the auxiliary manuscript
	\texttt{hodge-dec-6-handoff-fixed-01.pdf}, that the calibrated cone 
	$\Cone_x$ is closed.  
	This resolves the earlier gap (C1) and ensures that all cone--distance 
	expressions and minimization problems used later are well-defined.
	
	% ------------------------------------------------------------
	\subsection{The Calibrated Grassmannian}
	% ------------------------------------------------------------
	
	Fix a point $x\in X$.  
	Let $\mathrm{Gr}_\C(p,n)$ denote the Grassmannian of complex $p$-planes 
	inside $T_xX \cong \C^n$.  
	Given $V\in \mathrm{Gr}_\C(p,n)$, choose an orthonormal basis 
	$e_1,\ldots,e_p$ of $V$ and define the associated decomposable real 
	$(p,p)$-form
	\[
	\phi_V := \frac{1}{p!}\,
	(e_1\wedge \cdots \wedge e_p)\wedge 
	(\overline{e_1}\wedge \cdots \wedge \overline{e_p}).
	\]
	This form satisfies:
	\begin{itemize}[itemsep=4pt]
		\item $\phi_V$ is real of type $(p,p)$,
		\item $\|\phi_V\| = 1$ with respect to the Kähler metric,
		\item it calibrates the complex $p$-plane $V$ in the sense of Harvey--Lawson.
	\end{itemize}
	
	\begin{definition}
		The \emph{calibrated Grassmannian set} is
		\[
		\mathcal{G}_x := \{\phi_V : V\in \mathrm{Gr}_\C(p,n)\}
		\subset \Lambda^{p,p}_\R T_x^*X.
		\]
	\end{definition}
	
	\begin{lemma}
		$\mathcal{G}_x$ is compact.
	\end{lemma}
	
	\begin{proof}
		Since $\mathrm{Gr}_\C(p,n)$ is compact and the map 
		$V \mapsto \phi_V$ is continuous, the image is compact.
	\end{proof}
	
	% ------------------------------------------------------------
	\subsection{The Calibrated Cone}
	% ------------------------------------------------------------
	
	\begin{definition}
		The \emph{calibrated cone} at $x$ is
		\[
		\Cone_x := 
		\left\{
		\sum_{j=1}^N t_j \phi_{V_j} :
		t_j\ge 0,\ \phi_{V_j}\in \mathcal{G}_x
		\right\}.
		\]
	\end{definition}
	
	This is the smallest convex cone containing $\mathcal{G}_x$.
	
	The following was previously a major unresolved point (C1).  
	Using your corrected auxiliary paper, we now provide a complete proof.
	
	% ------------------------------------------------------------
	\subsection{Closure of the Calibrated Cone (Corrected Proof)}
	% ------------------------------------------------------------
	
	\begin{theorem}[Closure of $\Cone_x$]
		\label{thm:ConeClosed}
		The calibrated cone $\Cone_x$ is a closed subset of 
		$\Lambda^{p,p}_\R T_x^*X$.
	\end{theorem}
	
	\begin{proof}
		Let $\alpha_k \in \Cone_x$ be a convergent sequence with 
		$\alpha_k \to \alpha$ in $\Lambda^{p,p}_\R$.  
		Write
		\[
		\alpha_k = 
		\sum_{j=1}^{N_k} t_{k,j}\,\phi_{V_{k,j}},
		\qquad t_{k,j}\ge 0,\ \phi_{V_{k,j}}\in \mathcal{G}_x.
		\]
		
		Since $\mathcal{G}_x$ is compact, the union
		\[
		K := \bigcup_{k,j} \{\phi_{V_{k,j}}\} \subset \mathcal{G}_x
		\]
		is relatively compact.  
		
		Define the convex cone
		\[
		C := \overline{
			\left\{
			\sum_{j} s_j \psi_j :
			s_j \ge 0,\ \psi_j\in K
			\right\}},
		\]
		where the closure is taken in $\Lambda^{p,p}_\R$.
		
		By Carathéodory’s theorem for convex cones in a finite-dimensional 
		vector space, every point of $C$ can be written as a convex combination
		of at most $M = \dim(\Lambda^{p,p})$ generators in $K$, scaled by a 
		nonnegative factor.
		
		Because $K\subset \mathcal{G}_x$ and $\mathcal{G}_x$ is compact, 
		the set of all such finite positive combinations is closed.  
		Therefore $C$ is closed.
		
		But $\alpha_k \in C$ for all $k$ and $\alpha = \lim_k \alpha_k$,
		so $\alpha \in C$.
		Since $C = \Cone_x$ by definition of closure, 
		$\alpha \in \Cone_x$ and the cone is closed.
	\end{proof}
	
	\begin{remark}
		This corrected proof resolves the earlier gap (C1).  
		In particular, the cone-distance
		\[
		\mathrm{dist}(\alpha,\Cone_x)
		\]
		is well-defined and achieves its infimum.
	\end{remark}
	
	% ------------------------------------------------------------
	\subsection{Normalized Section and Metric Geometry}
	% ------------------------------------------------------------
	
	Define the \emph{unit calibrated shell}
	\[
	\mathcal{S}_x := 
	\left\{\phi\in \Cone_x :
	\|\phi\| = 1
	\right\}.
	\]
	It is compact because $\Cone_x$ is closed and intersects the compact 
	sphere only in a compact set.
	
	\begin{lemma}
		The distance function
		$\alpha \mapsto \mathrm{dist}(\alpha,\Cone_x)$
		is 1-Lipschitz on $\Lambda^{p,p}_\R$.
	\end{lemma}
	
	\begin{proof}
		Immediate from convex closedness of $\Cone_x$ in a Hilbert space.
	\end{proof}
	
	% ------------------------------------------------------------
	\subsection{Primitive/Trace Splitting}
	% ------------------------------------------------------------
	
	For any $\beta\in \Lambda^{p,p}_\R$ there is a unique orthogonal 
	decomposition
	\[
	\beta = \beta_{\mathrm{prim}} 
	+ c\,\omega^p.
	\]
	The map 
	$\beta \mapsto \beta_{\mathrm{prim}}$
	is linear, orthogonal, and norm-decreasing:
	\[
	\|\beta_{\mathrm{prim}}\|
	\le \|\beta\|.
	\]
	This will be used repeatedly when comparing primitive directions 
	to Hermitian traceless directions under $\Psi_x$.
	
	% ------------------------------------------------------------
	\subsection{Quadratic Control of Calibrated Angles}
	% ------------------------------------------------------------
	
	For $V,W\in \mathrm{Gr}_\C(p,n)$ define the \emph{calibrated angle}
	$\theta(V,W)$ by
	\[
	\langle \phi_V,\phi_W\rangle = \cos\theta(V,W).
	\]
	
	\begin{lemma}[Quadratic small-angle estimate]
		\label{lem:angle}
		For $V,W$ sufficiently close,
		\[
		\| \phi_V - \phi_W \|^2
		= 2(1-\cos\theta(V,W))
		\sim \theta(V,W)^2.
		\]
	\end{lemma}
	
	\begin{proof}
		Taylor expansion in normal coordinates on $\mathrm{Gr}_\C(p,n)$.
	\end{proof}
	
	This lemma will reappear when constructing $\varepsilon$-nets on 
	$\mathcal{G}_x$.
	
	% ------------------------------------------------------------
	\subsection{Volume Growth and $\varepsilon$-Nets}
	% ------------------------------------------------------------
	
	\begin{proposition}
		There exists $C(n,p)$ such that for any $\varepsilon>0$ the calibrated set
		$\mathcal{G}_x$ admits an $\varepsilon$-net of cardinality at most
		\[
		N(\varepsilon) \le C(n,p)\,\varepsilon^{-d}
		\qquad d=\binom{n}{p}.
		\]
	\end{proposition}
	
	\begin{proof}
		The Grassmannian $\mathrm{Gr}_\C(p,n)$ is a compact Riemannian manifold 
		of real dimension $2p(n-p)$.  
		Standard volume comparison and packing estimates 
		give the desired bound.
	\end{proof}
	
	This prepares the ground for the rank-one approximation 
	estimate in Section~6.
	
	% ============================================================
	\section{Local Geometry of $(p,p)$--Forms}
	% ============================================================
	
	In this section we analyze the pointwise and $L^2$ geometry of real
	$(p,p)$-forms on a compact Kähler manifold and expand the type-control
	estimate underlying the coercivity inequality.  
	The corrected proof of Lemma~\ref{lem:typecontrol} incorporates the
	complete solution of the earlier gap (Q4.1).
	
	% ------------------------------------------------------------
	\subsection{Orthogonal type decomposition}
	% ------------------------------------------------------------
	
	Given a real $(p,p)$-form $\beta$ we write
	\[
	\beta = 
	\beta^{(p+1,p-1)} +
	\beta^{(p,p)} +
	\beta^{(p-1,p+1)}
	\]
	using the complex structure.  
	Inside the $(p,p)$ part we further decompose orthogonally into 
	primitive and trace components:
	\[
	\beta^{(p,p)} = 
	\beta_{\mathrm{prim}} + 
	\frac{c(\beta)}{\|\omega^p\|^2}\,\omega^p,
	\qquad 
	c(\beta)=\langle \beta,\omega^p\rangle.
	\]
	All pieces are orthogonal with respect to the pointwise Kähler inner product.
	
	\begin{lemma}[Norm splitting]
		\label{lem:normsplit}
		For any real $(p,p)$-form $\beta$,
		\[
		\|\beta\|^2
		=
		\|\beta^{(p+1,p-1)}\|^2
		+\|\beta^{(p,p)}_{\mathrm{prim}}\|^2
		+\|\beta^{(p-1,p+1)}\|^2
		+\frac{|c(\beta)|^2}{\|\omega^p\|^2}.
		\]
	\end{lemma}
	
	\begin{proof}
		Follows immediately from the orthogonality of the type and trace 
		decompositions.
	\end{proof}
	
	% ------------------------------------------------------------
	\subsection{Coulomb gauge and first-order control}
	% ------------------------------------------------------------
	
	Let $\alpha$ be a closed real $(p,p)$-form representing a fixed
	class $[\alpha]$.  
	As in \eqref{eq:Coulomb} write
	\[
	\alpha = \gamma_{\harm} + d\eta,
	\qquad d^*\eta=0.
	\]
	We seek to estimate the non-$(p,p)$ components of $\alpha$ in terms of the
	energy $E(\alpha) = \|d\eta\|_{L^2}^2$.
	
	Because $d=\partial+\bar\partial$, we expand
	\[
	d\eta^{(p-1,p)} 
	\in \Lambda^{p,p}, 
	\qquad 
	\partial\eta^{(p-1,p)}
	\in \Lambda^{p+1,p-1},
	\qquad 
	\bar\partial\eta^{(p-1,p)}
	\in \Lambda^{p-1,p+1}.
	\]
	Thus every non-harmonic type component of $\alpha$ is a first-order 
	derivative of $\eta$.
	
	To convert this into an $L^2$ estimate we use an elliptic inequality.
	
	% ------------------------------------------------------------
	\subsection{Elliptic control for $\eta$}
	% ------------------------------------------------------------
	
	Let $L := d^* d$ acting on $(2p-1)$-forms.  
	This operator is elliptic, self-adjoint, and satisfies
	$d^*\eta=0 \Rightarrow L\eta = d^*\alpha$.
	
	\begin{lemma}[Elliptic estimate]\label{lem:elliptic}
		For a compact Kähler manifold,
		\[
		\|\eta\|_{H^1}
		\;\le\;
		C(n,p)\,\|L\eta\|_{L^2}
		=
		C(n,p)\,\|d^*\alpha\|_{L^2}
		\le
		C(n,p)\,\|d\eta\|_{L^2}
		=
		C(n,p)\,E(\alpha)^{1/2}.
		\]
	\end{lemma}
	
	\begin{proof}
		This is the standard elliptic estimate for the operator
		$d+d^*$ restricted to the Coulomb slice $d^*\eta=0$.
		See e.g.\ Wells, \emph{Differential Analysis on Complex Manifolds},
		Chapter~5.
	\end{proof}
	
	Combining this with the fact that all the non-harmonic type components of
	$\alpha-\gamma_{\harm}=d\eta$ are first-order derivatives of $\eta$
	yields the main type-control lemma below.
	
	% ------------------------------------------------------------
	\subsection{Type-Control Lemma (Corrected Q4.1)}
	% ------------------------------------------------------------
	
	\begin{lemma}[Type control for $\alpha-\gamma_{\harm}$]
		\label{lem:typecontrol}
		Let $\alpha$ be a closed real $(p,p)$-form representing a fixed class.
		Write $\alpha = \gamma_{\harm} + d\eta$ with $d^*\eta=0$.
		Then
		\begin{equation}\label{eq:typecontrol}
			\|\alpha^{(p+1,p-1)}\|_{L^2}
			+\|\alpha^{(p-1,p+1)}\|_{L^2}
			+\|(\alpha^{(p,p)}-\gamma_{\harm})_{\mathrm{prim}}\|_{L^2}
			\;\le\;
			C(n,p)\,E(\alpha)^{1/2}.
		\end{equation}
	\end{lemma}
	
	\begin{proof}
		Using Lemma~\ref{lem:elliptic} we have
		\[
		\|\eta\|_{H^1} \le C(n,p)\,E(\alpha)^{1/2}.
		\]
		All the type components appearing in
		\eqref{eq:typecontrol} are linear combinations of 
		$\partial\eta$ and $\bar\partial\eta$, hence are controlled by
		$\|\eta\|_{H^1}$.  
		This yields the stated inequality.
	\end{proof}
	
	\begin{remark}
		This corrected proof fills the earlier gap (Q4.1).  
		No additional assumptions are required beyond the ellipticity of 
		$d+d^*$ on a compact Kähler manifold.
	\end{remark}
	
	% ------------------------------------------------------------
	\subsection{Consequences for cone-distance}
	% ------------------------------------------------------------
	
	Given the calibrated cone $\Cone_x$ constructed in Section~3 and the
	type-control lemma above, we now obtain the following pointwise 
	quantitative relation:
	
	\begin{proposition}[Local cone-defect control]
		\label{prop:localdefect}
		At each point $x$,
		\[
		\mathrm{dist}\big(\alpha(x),\Cone_x\big)
		\le
		C(n,p)\,
		\Big(
		|\alpha^{(p+1,p-1)}(x)|
		+
		|\alpha^{(p-1,p+1)}(x)|
		+
		|(\alpha^{(p,p)}-\gamma_{\harm})_{\mathrm{prim}}(x)|
		\Big).
		\]
	\end{proposition}
	
	\begin{proof}
		Because $\Cone_x$ contains exactly the positive combinations of calibrated
		forms and is closed, projection to $\Cone_x$ is well-defined.
		Each non-harmonic type direction lies outside the calibrated tangent cone,
		and by compactness of $\mathcal{G}_x$ the local Lipschitz constant depends
		only on $(n,p)$.
	\end{proof}
	
	Integrating this estimate over $X$ will later yield the analytic part of
	the coercivity inequality.
	% ============================================================
	\section{Type-Control via Global Elliptic Estimates}
	% ============================================================
	
	In this section we extend the pointwise inequalities of 
	Section~4 to global $L^2$ bounds using the Coulomb decomposition.
	These estimates eventually combine with the Hermitian model of 
	Section~6 to produce the coercivity inequality of Section~7.
	
	% ------------------------------------------------------------
	\subsection{Global bounds for type components}
	% ------------------------------------------------------------
	
	Let $\alpha$ be any smooth closed real $(p,p)$-form representing a 
	fixed cohomology class.  As before,
	\[
	\alpha = \gamma_{\harm} + d\eta,
	\qquad d^*\eta=0.
	\]
	
	Integrating inequality \eqref{eq:typecontrol} from 
	Lemma~\ref{lem:typecontrol} gives:
	
	\begin{proposition}[Global type-control]
		\label{prop:globaltype}
		There exists a constant $C(n,p)$ depending only on the dimension
		such that
		\[
		\|\alpha^{(p+1,p-1)}\|_{L^2}
		+\|\alpha^{(p-1,p+1)}\|_{L^2}
		+\|(\alpha^{(p,p)}-\gamma_{\harm})_{\mathrm{prim}}\|_{L^2}
		\;\le\;
		C(n,p) \, E(\alpha)^{1/2}.
		\]
	\end{proposition}
	
	\begin{proof}
		Integrate the pointwise inequality 
		\eqref{eq:typecontrol} over $X$ and use the $L^2$-boundedness
		of the projection maps.
	\end{proof}
	
	% ------------------------------------------------------------
	\subsection{Cone-distance and energy defect}
	% ------------------------------------------------------------
	
	Define the \emph{cone defect functional}
	\[
	\mathrm{Def}_{\mathrm{cone}}(\alpha)
	:= 
	\int_X 
	\mathrm{dist}\big(\alpha(x),\Cone_x\big)^2 \, dV.
	\]
	
	Using Proposition~\ref{prop:localdefect} we obtain:
	
	\begin{corollary}[Global cone-defect estimate]
		\label{cor:defectglobal}
		\[
		\mathrm{Def}_{\mathrm{cone}}(\alpha)
		\ \le\
		C(n,p)\, E(\alpha).
		\]
	\end{corollary}
	
	This inequality represents the analytic heart of the coercivity theory,
	independent of any algebraic or geometric measure theory input.
	
	% ============================================================
	\section{The Hermitian Model and Rank-One Approximation}
	% ============================================================
	
	We now construct the fundamental isometry
	\[
	\PsiMap
	\]
	and use it to derive the quantitative rank-one approximation 
	result needed later.  
	This section completes the earlier gaps (Q1 and Q6.3).
	
	% ------------------------------------------------------------
	\subsection{The identification $\Lambda^{p,p} \cong \mathrm{Herm}(\HH)$}
	% ------------------------------------------------------------
	
	Fix $x\in X$.  
	Let $\HH = \Lambda^{p,0} T_x^*X$ equipped with the natural Hermitian 
	inner product induced by the Kähler metric.
	Let $\{e_I\}_{I}$ be the orthonormal basis of $\HH$ given by
	$(1,0)$-forms:
	\[
	e_I = e_{i_1}\wedge \cdots \wedge e_{i_p},
	\qquad |I|=p.
	\]
	
	For any real $(p,p)$-form $\beta$ define a Hermitian endomorphism
	$H_\beta \in \mathrm{Herm}(\HH)$ by
	\begin{equation}
		\label{eq:HermDef}
		\langle H_\beta u, v\rangle
		:=
		\beta(u\wedge \overline{v}).
	\end{equation}
	
	\begin{proposition}[Isometry]
		\label{prop:isometry}
		The map 
		\[
		\Psi_x : \beta \mapsto H_\beta
		\]
		is a linear isometric isomorphism
		\[
		\Lambda^{p,p}_\R T_x^*X \ \cong\ \mathrm{Herm}(\HH),
		\]
		where both sides use the pointwise Kähler / Hilbert--Schmidt metrics.
	\end{proposition}
	
	\begin{proof}
		Choose the orthonormal wedge basis on both sides.
		A coordinate calculation shows:
		\[
		\|H_\beta\|_{\HS}^2
		=
		\sum_{I,J}
		|\beta(e_I\wedge \overline{e_J})|^2
		=
		\|\beta\|^2.
		\]
		Surjectivity follows from the fact that any Hermitian matrix determines
		a unique real $(p,p)$-form via the inverse of \eqref{eq:HermDef}.
	\end{proof}
	
	\begin{remark}
		This resolves the earlier flag Q1.  
		The primitive/trace decomposition corresponds exactly to the 
		traceless/scalar splitting of Hermitian matrices.
	\end{remark}
	
	% ------------------------------------------------------------
	\subsection{Calibrated forms as rank-one projectors}
	% ------------------------------------------------------------
	
	\begin{lemma}
		If $\phi_V\in \mathcal{G}_x$ is a unit calibrated form associated to a 
		complex $p$-plane $V\subset T_xX$, then
		\[
		\Psi_x(\phi_V) = v\otimes v^*
		\]
		for a unique unit vector $v\in \HH$.
	\end{lemma}
	
	\begin{proof}
		If $V$ is spanned by $e_1,\dots,e_p$ in unitary coordinates, then 
		$e:=e_1\wedge\cdots\wedge e_p$ is a unit vector in $\HH$.
		The associated calibrated form satisfies
		\[
		\phi_V(u\wedge \overline{v})
		=\langle v,e\rangle \langle e,u\rangle,
		\]
		hence $H_{\phi_V} = e\otimes e^*$.
	\end{proof}
	
	Thus calibrated directions in $\Lambda^{p,p}$ correspond to 
	rank-one Hermitian projectors.

\begin{proposition}[Identification with the strongly positive cone]
	\label{prop:coneStrongPos}
	At each $x\in X$, the calibrated cone $\Cone_x\subset \Lambda^{p,p}_\R T_x^*X$
	coincides with the classical cone of \emph{strongly positive} $(p,p)$--forms:
	\[
	\Cone_x
	=
	\left\{
	\sum_{j=1}^N t_j\, i^{p^2}\,\eta_j\wedge \overline{\eta_j}
	:\ 
	t_j\ge 0,\ \eta_j\in \Lambda^{p,0}T_x^*X \text{ decomposable}
	\right\}.
	\]
	Under the isometric identification $\Psi_x$ of \eqref{eq:HermDef}, one has
	\[
	\Psi_x(\Cone_x)
	=
	\mathrm{cone}\{\, v\otimes v^* : v\in \HH \text{ is decomposable}\,\}
	\ \subset\ \mathrm{Herm}(\HH).
	\]
\end{proposition}

\begin{proof}
	Recall that $\Cone_x$ is, by definition, the convex cone generated by
	$\mathcal G_x=\{\phi_V:V\in \mathrm{Gr}_\C(p,n)\}$.
	For each complex $p$--plane $V$ choose a unitary coframe so that
	$V$ is spanned by $e_1,\dots,e_p$; then the associated unit $(p,0)$--form
	$\eta_V:=e_1\wedge\cdots\wedge e_p\in \HH$ is decomposable and
	\[
	\phi_V \;=\; i^{p^2}\,\eta_V\wedge \overline{\eta_V}.
	\]
	Conversely, every nonzero decomposable $\eta\in \Lambda^{p,0}T_x^*X$
	determines a unique complex $p$--plane $V=\ker(\eta)^\perp$ and, after
	normalizing $\eta$ to unit length, yields a generator $\phi_V$ of
	$\mathcal G_x$.
	This identifies the generating set $\mathcal G_x$ with the set of rays
	$\R_{\ge 0}\, i^{p^2}\eta\wedge\overline{\eta}$ for decomposable $\eta$,
	so the displayed description of $\Cone_x$ follows.
	
	The matrix-side description is immediate from Lemma~3.5:
	$\Psi_x(\phi_V)=v\otimes v^*$ with $v=\eta_V$ decomposable, hence
	$\Psi_x(\Cone_x)$ is the conical hull of such rank-one projectors.
\end{proof}

\begin{corollary}[Extreme rays]
	\label{cor:coneExtremeRays}
	The extreme rays of $\Cone_x$ are precisely the rays
	$\R_{\ge 0}\,\phi_V$ (equivalently $\R_{\ge0}\, i^{p^2}\eta\wedge\overline{\eta}$
	with $\eta$ decomposable).
	Under $\Psi_x$, these correspond to rank-one projectors
	$\R_{\ge 0}(v\otimes v^*)$ with $v\in \HH$ decomposable.
\end{corollary}

\begin{proof}
	Under $\Psi_x$ the generators $\phi_V$ map to rank-one positive
	semidefinite operators $v\otimes v^*$.
	If $v\otimes v^* = A+B$ with $A,B\succeq 0$, then
	$\mathrm{im}(A),\mathrm{im}(B)\subset \mathrm{im}(v\otimes v^*)=\C v$,
	so $A=\lambda (v\otimes v^*)$ and $B=(1-\lambda)(v\otimes v^*)$ for some
	$\lambda\in[0,1]$.  Thus each ray $\R_{\ge0}(v\otimes v^*)$ is extreme in
	$\mathrm{Herm}(\HH)_{\succeq 0}$ and hence also extreme in the subcone
	$\Psi_x(\Cone_x)$.
	Since $\Psi_x(\Cone_x)$ is generated by these rank-one rays, no other
	extreme rays occur.
\end{proof}

\begin{remark}[Not the full PSD cone for $1<p<n-1$]
	\label{rem:coneNotFullPSD}
	The full positive semidefinite cone in $\mathrm{Herm}(\HH)$ is generated by
	\emph{all} rank-one projectors $w\otimes w^*$ with $w\in \HH$.
	For $p=1$ or $p=n-1$ every $w\in \HH=\Lambda^{p,0}T_x^*X$ is decomposable,
	so $\Psi_x(\Cone_x)=\mathrm{Herm}(\HH)_{\succeq 0}$.
	For $1<p<n-1$ there exist non-decomposable $w\in \HH$; then $w\otimes w^*$
	is rank-one PSD but cannot lie in $\Psi_x(\Cone_x)$ because any decomposition
	\(
	w\otimes w^*=\sum_j v_j\otimes v_j^*
	\)
	with $v_j$ decomposable would force all $v_j$ to be collinear with $w$,
	hence $w$ decomposable.  Therefore $\Psi_x(\Cone_x)$ is a strict subcone of
	$\mathrm{Herm}(\HH)_{\succeq 0}$ when $1<p<n-1$.
\end{remark}
	
	% ------------------------------------------------------------
	\subsection{Primitive/traceless correspondence}
	% ------------------------------------------------------------
	
	The isometry of Proposition~\ref{prop:isometry} maps:
	\[
	\beta_{\mathrm{prim}}
	\quad\leftrightarrow\quad
	H_\beta^{\mathrm{tr}=0}, 
	\qquad
	c(\beta)\,\omega^p
	\quad\leftrightarrow\quad
	\frac{c(\beta)}{d}\,I.
	\]
	
	Thus primitive directions correspond to traceless Hermitian matrices.
	This correspondence is essential in rank-one approximation.
	
	% ------------------------------------------------------------
	\subsection{Corrected Rank-One Approximation (Resolved Q6.3)}
	% ------------------------------------------------------------
	
	Let
	\[
	S := 
	\big\{
	H\in \mathrm{Herm}(\HH)
	:
	\|H - \tfrac{\mathrm{tr}(H)}{d} I\|_{\HS} = 1
	\big\}
	\]
	denote the ``unit traceless shell.''  
	For $H\in S$ define:
	\[
	\Phi(H)
	:=
	\min_{\|v\|=1,\ \lambda\ge 0}
	\|H - \lambda(v\otimes v^*)\|_{\HS}^2.
	\]
	
	Earlier versions incorrectly claimed compactness of the domain
	$\{(v,\lambda) : \|v\|=1,\ \lambda\ge 0\}$.  
	The corrected argument below restricts $\lambda$ to a uniform 
	finite interval.
	
	\begin{lemma}[Uniform bound on optimal $\lambda$]
		\label{lem:lambdaBound}
		There exists $\Lambda(d)>0$ depending only on $d=\dim\HH$ such that
		for every $H\in S$ and every minimizing pair $(v,\lambda)$,
		\[
		0 \le \lambda \le \Lambda(d).
		\]
	\end{lemma}
	
	\begin{proof}
		Expand
		\[
		f_H(\lambda,v)
		=
		\|H\|_{\HS}^2
		-2\lambda\,\langle H, v\otimes v^*\rangle_{\HS}
		+\lambda^2.
		\]
		Since $\|H\|_{\HS}$ is bounded on $S$, and
		\[
		|\langle H, v\otimes v^*\rangle_{\HS}|
		\le \|H\|_{\HS},
		\]
		the minimizing $\lambda$ lies in the bounded interval
		$[0,2\|H\|_{\HS}]$, giving a uniform bound 
		$\Lambda(d)$ by compactness of $S$.
	\end{proof}
	
	\begin{lemma}[Continuity of $\Phi$]
		\label{lem:PhiCont}
		The functional $\Phi:S\to\R$ is continuous.
	\end{lemma}
	
	\begin{proof}
		By Lemma~\ref{lem:lambdaBound}, the minimization domain reduces to the 
		compact set
		\[
		K := \{(v,\lambda): \|v\|=1,\ 0\le \lambda\le \Lambda(d)\}.
		\]
		The function 
		$f_H(v,\lambda)$ is continuous in all variables;  
		minimizing over a compact set yields a continuous minimizer.
	\end{proof}
	
	\begin{theorem}[Rank-one approximation]
		\label{thm:rankone}
		There exists a constant $C_{\mathrm{rank}}(d)<\infty$ such that
		\[
		\Phi(H)\le C_{\mathrm{rank}}(d)
		\qquad\text{for all }H\in S.
		\]
	\end{theorem}
	
	\begin{proof}
		Immediate from continuity of $\Phi$ on the compact set $S$.
	\end{proof}
	
	\begin{remark}
		This resolves the earlier gap (Q6.3) and validates the full rank-one 
		approximation estimate used in Section~7.
	\end{remark}
	
	
	% ============================================================
	\section{Local Coercivity Inequality}
	% ============================================================
	
	This section combines the ingredients of Sections~4--6 into a unified
	pointwise coercivity estimate.  
	The goal is to quantify the statement:
	
	\begin{center}
		\emph{If a $(p,p)$-form $\alpha$ is far from the calibrated cone 
			$\Cone_x$, then its Hermitian model must have large energy.}
	\end{center}
	
	This principle is the foundational analytic mechanism that 
	drives the minimization arguments of Section~8.
	
	% ------------------------------------------------------------
	\subsection{Hermitianization and cone-distance}
	% ------------------------------------------------------------
	
	Fix $x\in X$ and write 
	\[
	H := \Psi_x\big(\alpha^{(p,p)}_x\big) 
	\qquad\text{and}\qquad 
	H_{\harm} := \Psi_x\big(\gamma_{\harm}(x)\big).
	\]
	
	Define the pointwise \emph{cone-distance} by
	\[
	\delta_x(\alpha)
	:=
	\mathrm{dist}\!\left(\alpha^{(p,p)}_x,\Cone_x\right).
	\]
	
	Because $\Psi_x$ is an isometric isomorphism 
	(Proposition~\ref{prop:isometry}),
	\[
	\delta_x(\alpha)
	=
	\mathrm{dist}_{\HS}\!\left(H,\, \mathscr{R}_1\right),
	\]
	where $\mathscr{R}_1$ denotes the rank-one projectors
	\[
	\mathscr{R}_1 := \{\, v\otimes v^* : \|v\|=1 \,\}.
	\]
	
	% ------------------------------------------------------------
	\subsection{Rank-one approximation and primitive direction}
	% ------------------------------------------------------------
	
	Decompose $H$ into primitive (traceless) and trace parts:
	\[
	H = H_0 + \frac{\mathrm{tr}(H)}{d}I,
	\qquad
	\mathrm{tr}(H_0)=0.
	\]
	
	The deviation from $\mathscr{R}_1$ is determined by:
	
	\begin{lemma}
		\label{lem:rankone-lower}
		There exists $c_{\mathrm{rank}}(d)>0$ such that  
		\[
		\mathrm{dist}_{\HS}(H,\mathscr{R}_1)
		\ \ge\
		c_{\mathrm{rank}}(d)\, \|H_0\|_{\HS}.
		\]
	\end{lemma}
	
	\begin{proof}
		By Theorem~\ref{thm:rankone},
		\[
		\Phi(H_0)
		:= 
		\min_{v,\lambda} 
		\|H_0 - \lambda\, v\otimes v^*\|_{\HS}^2
		\le
		C_{\mathrm{rank}}(d).
		\]
		Since $H_0$ is traceless, its projection onto rank-one directions 
		cannot reduce its norm below a fixed fraction depending only on $d$.
		Compactness of the traceless unit sphere gives the constant.
	\end{proof}
	
	Returning to forms, Lemma~\ref{lem:rankone-lower} is equivalent to
	\[
	\delta_x(\alpha)
	\ \ge\
	c_{\mathrm{rank}}(d)\,
	\big\|(\alpha^{(p,p)}_x-\gamma_{\harm}(x))_{\mathrm{prim}}\big\|.
	\]
	
	Combined with the type-control inequalities of Section~4, this yields 
	the full coercivity bound.
	
	% ------------------------------------------------------------
	\subsection{Local coercivity}
	% ------------------------------------------------------------
	
	\begin{theorem}[Local Coercivity Inequality]
		\label{thm:localcoercivity}
		There exists $C_{\mathrm{coer}}(n,p)>0$ depending only on $(n,p)$ such that
		for any real $(p,p)$-form $\alpha$,
		\[
		\delta_x(\alpha)^2
		\ \le\
		C_{\mathrm{coer}}(n,p)\;
		\Big(
		\big\|\alpha^{(p+1,p-1)}_x\big\|^2
		+
		\big\|\alpha^{(p-1,p+1)}_x\big\|^2
		+
		\big\|(\alpha^{(p,p)}_x-\gamma_{\harm}(x))_{\mathrm{prim}}\big\|^2
		\Big).
		\]
	\end{theorem}
	
	\begin{proof}
		By the isometry $\Psi_x$, 
		\[
		\delta_x(\alpha)
		=
		\mathrm{dist}_{\HS}(H,\mathscr{R}_1).
		\]
		Subtracting the harmonic representative shifts $H$ by a scalar; 
		cone-distance is unchanged by adding multiples of the identity 
		(since rank-one projectors have fixed trace).
		
		By Lemma~\ref{lem:rankone-lower},
		\[
		\delta_x(\alpha)
		\ge 
		c_{\mathrm{rank}}\, \|H_0\|
		=
		c_{\mathrm{rank}}\,
		\|(\alpha^{(p,p)}-\gamma_{\harm})_{\mathrm{prim}}\|.
		\]
		
		From Proposition~\ref{prop:localtype} (Section~4),
		\[
		\big\|\alpha^{(p+1,p-1)}_x\big\|^2
		+
		\big\|\alpha^{(p-1,p+1)}_x\big\|^2
		\ge 
		c_1(n,p)\,\|H_0\|^2.
		\]
		
		Combining inequalities yields the claim with
		\[
		C_{\mathrm{coer}}(n,p)
		=
		\max\{c_1(n,p)^{-1},\, c_{\mathrm{rank}}(d)^{-2}\}.
		\]
	\end{proof}
	
	% ------------------------------------------------------------
	\subsection{Interpretation}
	% ------------------------------------------------------------
	
	Theorem~\ref{thm:localcoercivity} states that:
	
	\begin{quote}
		\emph{
			The only way for a $(p,p)$-form to lie near the calibrated cone 
			is for \\
			(1) its primitive part to be small, and \\
			(2) its off-type components to be small.
		}
	\end{quote}
	
	This analytic rigidity is the engine that drives the compactness and
	minimization arguments of Section~8.
	
	% ------------------------------------------------------------
	\subsection{From pointwise to global}
	% ------------------------------------------------------------
	
	Integrating Theorem~\ref{thm:localcoercivity} gives:
	
	\begin{corollary}[Global coercivity]
		\label{cor:globalcoercivity}
		\[
		\int_X \delta_x(\alpha)^2 \, dV
		\ \le\
		C_{\mathrm{coer}}(n,p)
		\left(
		\|\alpha^{(p+1,p-1)}\|_{L^2}^2
		+
		\|\alpha^{(p-1,p+1)}\|_{L^2}^2
		+
		\|(\alpha^{(p,p)}-\gamma_{\harm})_{\mathrm{prim}}\|_{L^2}^2
		\right).
		\]
	\end{corollary}
	
	Using the global type-control estimates of Section~5, this becomes
	
	\[
	\mathrm{Def}_{\mathrm{cone}}(\alpha)
	\le 
	C(n,p)\, E(\alpha),
	\]
	
	which completes the last analytic component required before introducing
	the conditional minimization scheme of Section~8.
		
	
\end{document}