\documentclass[11pt]{article}

\usepackage{amsmath,amssymb,amsthm,hyperref}

\newtheorem{theorem}{Theorem}
\newtheorem{proposition}{Proposition}
\newtheorem{lemma}{Lemma}
\newtheorem{remark}{Remark}

\newcommand{\F}{\mathbf{F}} % flat norm (as in manuscript)
\newcommand{\Mass}{\mathbf{M}}
\newcommand{\eps}{\varepsilon}

\begin{document}
	
	\begin{center}
		{\Large \textbf{Referee Note on Theorem 8.46 (Global prefix-template activation / mass matching)}}\\[2mm]
		{\large Amir Rahnamai Barghi (referee)}\\
	\end{center}
	
	\section*{0. Executive summary (status of the ``global activation'' gate)}
	In the current Dec 2025 draft (\texttt{hodge-SAVE-dec-12-handoff.tex}), the result corresponding to ``Theorem 8.46'' is the theorem labeled
	\texttt{thm:sliver-mass-matching-on-template}.  It is intentionally stated as a bookkeeping reduction: assuming hypotheses (i)--(iv),
	it yields the per-face and global flat-norm bounds needed for the gluing step.
	
	\medskip\noindent
	\textbf{Update (Dec 2025): the manuscript now proves hypotheses (i)--(iv) in the corner-exit vertex-template route.}
	The certification points are:
	\begin{itemize}
		\item \textbf{(i)--(ii)} many pieces and slow variation (including stability under $0$--$1$ discrepancy rounding) are proved in
		\texttt{lem:slow-variation-rounding} and \texttt{lem:slow-variation-discrepancy}.
		\item \textbf{(iii)--(iv)} fixed-template local holomorphic realizability and the $O(h)$ face-edit regime are verified for corner-exit vertex templates in
		\texttt{cor:corner-exit-iii-iv}, using \texttt{prop:holomorphic-corner-exit-L1},
		\texttt{prop:vertex-template-mass-matching}, and \texttt{prop:vertex-template-face-edits} /
		\texttt{prop:checkerboard-face-oh-edit}.
		\item The all-direction packaged execution (weights, rounding, cohomology constraints, and holomorphic realization) is recorded in
		\texttt{prop:global-coherence-all-labels}; see also the in-place status remark \texttt{rem:activation-hypotheses-status}.
	\end{itemize}
	Therefore the ``global activation'' gate is no longer a conditional gap; what remains is expository (keeping these pointers visible at the theorem’s point of use).
	
	\section*{1. The statement under review (as used later)}
	Theorem 8.46 fixes:
	\begin{itemize}
		\item a mesh-$h$ decomposition into smooth uniformly convex cells (rounded cubes),
		\item a direction label $j$ and paired calibrated reference planes across neighbors,
		\item an ordered master template of transverse atoms $\{y_a\}_{a\ge 1}\subset B_{c_0 h}(0)\subset \mathbb{R}^{2p}$,
		\item for each cell $Q$, an integer $N_Q\ge 0$ (desired integer sheet count for this family), and a target matching mass budget $M_Q\ge 0$ (from the smooth form $m\beta$).
	\end{itemize}
	It then assumes:
	\begin{enumerate}
		\item (Many pieces) $N_Q\gtrsim h^{-1}$ on the region where $M_Q$ is not negligible;
		\item (Slow variation) $|N_Q-N_{Q'}|\le C h \min\{N_Q,N_{Q'}\}$ for adjacent $Q\sim Q'$;
		\item (Local realizability on a fixed template) for each $Q$ there exist disjoint $\psi$-calibrated holomorphic pieces
		$Y^{1},\dots,Y^{N_Q}$ in $Q$ whose transverse parameters are the prefix $\{y_a\}_{a\le N_Q}$ and whose \emph{total} mass satisfies
		\[
		\sum_{a=1}^{N_Q} \big([Y^a]\llcorner Q\big)= M_Q + o(M_Q)
		\qquad (h\to 0,\ \text{uniformly in }Q);
		\]
		\item (O($h$) edit regime on faces) for every interior interface $F=Q\cap Q'$ the ``unmatched part'' satisfies the $O(h)$-fraction hypothesis of Proposition 8.45.
	\end{enumerate}
	The conclusion is that $\partial T^{raw}$ satisfies the per-face flat-norm mismatch bound of Proposition 8.45, hence
	\[
	\F(\partial T^{raw}) \ \lesssim\ h^2 \sum_{Q}\sum_{a\in S(Q)} m_{Q,a}^{\frac{k-1}{k}} + O(\eps m),
	\qquad k:=2n-2p,
	\]
	where $m_{Q,a}:=\big([Y^{Q,a}]\llcorner Q\big)$ and $S(Q)$ indexes the pieces meeting the interface.
	Finally, the manuscript asserts that in the parameter regime recorded in Remark 8.36 (e.g.\ $p\le n/2$ as stated there) one gets $\F(\partial T^{raw})=o(m)$.
	
	\section*{2. Referee update (Dec 2025): the conditional bookkeeping gate is discharged}
	The bookkeeping implication ``(i)--(iv) $\Rightarrow$ global bound'' is exactly the role of the theorem labeled
	\texttt{thm:sliver-mass-matching-on-template} in the manuscript (it is a reduction from per-cell activation + face-edit control to the global flat-norm bound).
	
	\medskip\noindent
	In the Dec 2025 draft, the manuscript also supplies the missing certification of (i)--(iv) for the actual corner-exit construction:
	\begin{itemize}
		\item \textbf{(i)--(ii)} are proved by rounding Lipschitz target counts and controlling neighbor variation; see
		\texttt{lem:slow-variation-rounding} and \texttt{lem:slow-variation-discrepancy}.
		\item \textbf{(iii)--(iv)} are proved for the holomorphic corner-exit vertex-template activation by \texttt{cor:corner-exit-iii-iv}
		(and made uniform over the direction net and all labels in \texttt{prop:global-coherence-all-labels}).
		\item The scaling regime and its relation to the full Hodge statement is recorded in \texttt{rem:weighted-scaling} together with the reduction
		\texttt{rem:lefschetz-reduction}.
	\end{itemize}
	The remaining issue is therefore presentation: a referee should be pointed to these labels immediately when the bookkeeping theorem is invoked;
	the manuscript now includes an in-place status pointer \texttt{rem:activation-hypotheses-status}.
	
	\section*{3. Audit of the four assumptions}
	\subsection*{3.1 Assumption (i): ``many pieces''}
	This is a quantitative lower bound on $N_Q$ whenever $M_Q$ is not negligible.
	It is \emph{not} automatic from the definitions unless the manuscript:
	\begin{itemize}
		\item defines $N_Q$ explicitly as a function of $M_Q$ and $h$ (e.g.\ $N_Q\sim M_Q/h$ in the ``sliver'' model), and
		\item proves a uniform lower bound on $M_Q$ on the active region (or explicitly restricts to the active region).
	\end{itemize}
	\textbf{Update (Dec 2025 draft):} this is now proved in the manuscript via rounding of Lipschitz targets:
	one defines target real counts $n_Q:=m\,h^{2p}\,f(x_Q)$ and takes $N_Q:=\lfloor n_Q\rceil$
	(\texttt{Lemma~lem:slow-variation-rounding}).  On the active region where $f\ge f_0>0$ (equivalently, where the family’s budget $M_Q$ is not negligible),
	the same lemma yields $N_Q\gtrsim h^{-1}$ once $m\,h^{2p+1}$ is taken large enough.
	
	\subsection*{3.2 Assumption (ii): slow variation of $N_Q$ across neighbors}
	The text says $N_Q$ is ``derived from Lipschitz target weights''.
	To make (ii) rigorous, one needs a chain:
	\[
	\text{(smooth/Lipschitz density of target weights)} \quad\Longrightarrow\quad
	\text{(neighbor budgets differ by $O(h)$)}\quad\Longrightarrow\quad
	\text{(integer counts differ by $O(h)$ relative)}.
	\]
	This typically requires an explicit rounding stability lemma: if $N_Q$ is obtained by rounding a smooth real-valued profile $\nu_Q$, then
	$|\nu_Q-\nu_{Q'}|\lesssim h\,\nu_Q$ should imply $|N_Q-N_{Q'}|\lesssim h\,N_Q$.
	\textbf{Update (Dec 2025 draft):} this is now proved in \texttt{lem:slow-variation-rounding} (nearest-integer rounding), and the stability under $0$--$1$
	discrepancy rounding (the form used to satisfy finitely many global constraints) is proved in \texttt{lem:slow-variation-discrepancy}.
	
	\subsection*{3.3 Assumption (iii): local realizability on a fixed ordered template}
	This is the \textbf{hardest} assumption and appears to contain the substantive analytic geometry:
	for \emph{each} cell $Q$ one must realize the \emph{same ordered list} $\{y_a\}$ (up to the prefix length $N_Q$) by disjoint
	$\psi$-calibrated holomorphic pieces, and match the \emph{cellwise} mass budget $M_Q$ with error $o(M_Q)$ uniformly in $Q$.
	
	Even if the manuscript has local existence theorems (e.g.\ complete intersections or local graphs over calibrated planes),
	assumption (iii) requires additional uniformity:
	\begin{itemize}
		\item uniform control at the Bergman scale $h\sim m^{-1/2}$,
		\item disjointness of \emph{all} pieces in the prefix (not just pairwise existence),
		\item a mechanism ensuring the sum of masses matches $M_Q$ up to $o(M_Q)$ \emph{uniformly across all cells}.
	\end{itemize}
	\textbf{Update (Dec 2025 draft):} this is now supplied by the holomorphic corner-exit route:
	\begin{itemize}
		\item \texttt{prop:holomorphic-corner-exit-L1} constructs holomorphic corner-exit slivers from a corner-exit translation template with uniform $C^1$ single-sheet control,
		and \texttt{rem:vertex-star-coherence} explains how the \emph{same indexed template} is realized coherently on each vertex star.
		\item \texttt{prop:vertex-template-mass-matching} chooses prefix lengths to match local mass budgets with uniform $o(M_Q)$ error.
		\item The verification of hypothesis (iii) at the level of the bookkeeping theorem is summarized in \texttt{cor:corner-exit-iii-iv} and packaged across all labels in
		\texttt{prop:global-coherence-all-labels}.
	\end{itemize}
	
	\subsection*{3.4 Assumption (iv): the $O(h)$ edit regime on faces}
	As stated, (iv) is \emph{not} a mere corollary of (ii) unless one can convert a bound on the \emph{count}
	$|N_Q-N_{Q'}|$ into a bound on the \emph{boundary mass fraction} contributed by unmatched pieces.
	
	A correct sufficient condition has the following form.
	
	\begin{lemma}[A sufficient condition for the $O(h)$ face-edit regime]
		\label{lem:iv-sufficient}
		Fix an interior interface $F=Q\cap Q'$. Assume:
		\begin{enumerate}
			\item[(a)] the pieces in $Q$ and $Q'$ are indexed by the same ordered template $\{y_a\}$ and the matched pieces are those with
			$a\le N_{\min}:=\min\{N_Q,N_{Q'}\}$;
			\item[(b)] there are nonnegative ``face-boundary weights'' $b_{Q,a}(F)$ and $b_{Q',a}(F)$ such that the total boundary mass across $F$ satisfies
			\[
			\Mass\big(\partial([Y^{Q,a}]\llcorner Q)\llcorner F\big)\ \le\ b_{Q,a}(F),
			\qquad
			\Mass\big(\partial([Y^{Q',a}]\llcorner Q')\llcorner F\big)\ \le\ b_{Q',a}(F);
			\]
			\item[(c)] the template ordering is \emph{mass-compatible} in the sense that the boundary weights of the ``tail'' are controlled by the boundary weights of the ``prefix'':
			there is a constant $C_*$ such that
			\[
			\sum_{a=N_{\min}+1}^{N_{\max}} b_{Q,a}(F)\ \le\ C_*\,\frac{N_{\max}-N_{\min}}{N_{\min}}\,
			\sum_{a=1}^{N_{\min}} b_{Q,a}(F),
			\]
			and similarly on the $Q'$ side (here $N_{\max}:=\max\{N_Q,N_{Q'}\}$).
		\end{enumerate}
		Then (ii) implies the $O(h)$-fraction hypothesis in Proposition 8.45, i.e.\ the unmatched boundary contribution across $F$ is
		$\le C h$ times the total boundary contribution, for some $C$ depending only on $C_*$ and the constant in (ii).
	\end{lemma}
	
	\begin{proof}
		Assume wlog $N_{\max}=N_Q\ge N_{Q'}=N_{\min}$.
		The unmatched boundary contribution on the $Q$ side is supported on the indices $a\in\{N_{\min}+1,\dots,N_{\max}\}$, hence
		\[
		\text{Unmatched}(F)\ \le\ \sum_{a=N_{\min}+1}^{N_{\max}} b_{Q,a}(F)
		\ \le\ C_*\,\frac{N_{\max}-N_{\min}}{N_{\min}}\,
		\sum_{a=1}^{N_{\min}} b_{Q,a}(F).
		\]
		By (ii),
		\(
		\frac{N_{\max}-N_{\min}}{N_{\min}}\le C h.
		\)
		Therefore
		\[
		\text{Unmatched}(F)\ \le\ (C_*C)\,h \sum_{a=1}^{N_{\min}} b_{Q,a}(F)
		\ \le\ (C_*C)\,h \cdot \text{Total}(F),
		\]
		where $\text{Total}(F)$ denotes the total boundary contribution across $F$ from all pieces on both sides.
		This is exactly the $O(h)$-fraction form required to invoke Proposition 8.45.
	\end{proof}
	
	\noindent
	\textbf{Update (Dec 2025 draft):} the manuscript now does exactly this in its corner-exit vertex-template route.
	The needed ``no heavy tail'' / mass-compatibility on faces (so tail pieces cannot dominate the prefix on a given interface) is built into the corner-exit simplex
	geometry (deterministic face incidence plus equal/comparable per-piece slice masses).
	The resulting $O(h)$ face-edit regime is proved in \texttt{prop:vertex-template-face-edits} (and alternatively in the single-master-template formulation
	\texttt{prop:checkerboard-face-oh-edit}), with the abstract tail-vs-prefix reduction recorded as \texttt{lem:oh-face-edit-regime} in the manuscript and summarized in
	\texttt{cor:corner-exit-iii-iv}.
	
	\section*{4. The parameter restriction: where $p\le n/2$ (or $p<(n+1)/2$) enters}
	The global estimate in Theorem 8.46 is of the form
	\[
	\F(\partial T^{raw}) \ \lesssim\ h^2 \sum_{Q}\sum_{a} m_{Q,a}^{\frac{k-1}{k}} + O(\eps m),
	\qquad k=2n-2p.
	\]
	Remark 8.36 then uses concavity/H\"older-type bounds to estimate $\sum_a m_{Q,a}^{(k-1)/k}$ by a power of $\sum_a m_{Q,a}=M_Q$,
	introducing the exponent $(k-1)/k$ and the condition that the resulting scaling be sublinear in $m$.
	In the manuscript's own discussion (see Remark 8.36), the ratio
	\(
	\F(\partial T^{raw})/m
	\)
	tends to $0$ only under a condition of the form $k>n-1$, equivalently
	\[
	2n-2p\ >\ n-1 \quad\Longleftrightarrow\quad p\ <\ \frac{n+1}{2},
	\]
	and the text further highlights the regime $p\le n/2$.
	
	\medskip\noindent
	\textbf{Update (Dec 2025 draft):} the scaling computation is made explicit in the manuscript as \texttt{rem:weighted-scaling}, yielding
	$\F(\partial T^{raw})/m\to 0$ at Bergman scale whenever $p<\frac{n+1}{2}$.
	The manuscript also includes the standard projective Hard Lefschetz reduction \texttt{rem:lefschetz-reduction}, which reduces the Hodge conjecture to
	$p\le \frac{n}{2}$.  Since $p\le \frac{n}{2}\Rightarrow p<\frac{n+1}{2}$, the scaling condition is compatible with the reduced range needed for the full statement.
	
	\medskip\noindent
	(If one insisted on treating $p>\frac{n}{2}$ directly \emph{without} reduction, then the manuscript would need one of the following:)
	\begin{itemize}
		\item produce a \emph{stronger} per-face mismatch estimate than the $h^2$-based bound (e.g.\ an extra $h^\delta$ gain),
		\item change the scaling choice $h\sim m^{-1/2}$ in a way compatible with the holomorphic control scale,
		\item or introduce a different global gluing mechanism that avoids the concavity bottleneck.
	\end{itemize}
	In the Dec 2025 draft the range $p\le n/2$ is the one used for unconditional closure, and it lies in the scaling regime above.
	
	\section*{5. Checklist (now present in the Dec 2025 draft)}
	The referee ``action items'' above are now implemented in the manuscript with explicit labels:
	\begin{enumerate}
		\item Definition of $N_Q$ / neighbor control: \texttt{lem:slow-variation-rounding} and \texttt{lem:slow-variation-discrepancy}.
		\item Fixed-template holomorphic realizability and uniform mass matching: \texttt{prop:holomorphic-corner-exit-L1},
		\texttt{rem:vertex-star-coherence}, and \texttt{prop:vertex-template-mass-matching}, summarized in \texttt{cor:corner-exit-iii-iv}.
		\item Rigorous derivation of the $O(h)$ face-edit regime from boundary-mass control (not just counts): \texttt{lem:oh-face-edit-regime} together with
		\texttt{prop:vertex-template-face-edits} / \texttt{prop:checkerboard-face-oh-edit}.
		\item Exact scaling range and its relation to the full Hodge statement: \texttt{rem:weighted-scaling} and the reduction \texttt{rem:lefschetz-reduction}.
	\end{enumerate}
	
	\bigskip
	\noindent\textbf{Bottom line (referee update):} in the Dec 2025 draft, the bookkeeping theorem is paired with an explicit certification
	of (i)--(iv) for the corner-exit vertex-template construction (see especially \texttt{prop:global-coherence-all-labels} and
	\texttt{rem:activation-hypotheses-status}).  The ``activation gate'' is therefore no longer a conditional gap.
	
\end{document}