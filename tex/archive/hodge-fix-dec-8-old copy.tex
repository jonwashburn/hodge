\documentclass[12pt]{article}

\usepackage{amsmath,amssymb,amsthm}

\title{Calibrated Cones, Positive Semidefinite Cones,\\
and a Quantitative Replacement for Calibration--Coercivity}

\author{Jonathan Washburn\\
Recognition Science, Recognition Physics Institute\\
\texttt{jon@recognitionphysics.org}\\
Austin, Texas, USA}

\date{}

\theoremstyle{plain}
\newtheorem{theorem}{Theorem}[section]
\newtheorem{proposition}[theorem]{Proposition}
\newtheorem{lemma}[theorem]{Lemma}
\newtheorem{corollary}[theorem]{Corollary}

\theoremstyle{definition}
\newtheorem{definition}[theorem]{Definition}

\theoremstyle{remark}
\newtheorem{remark}[theorem]{Remark}

\begin{document}

\maketitle

\begin{abstract}
In the K\"ahler calibration setting, the cone generated by simple calibrated
$(p,p)$--forms at a point is often informally identified with a positive
semidefinite cone in a Hermitian model.  In middle codimension $1<p<n-1$,
this identification is \emph{false}.  We isolate the precise linear--algebra
issue, prove that the calibrated cone is a proper subcone of the positive
semidefinite cone, and then establish a quantitative approximation result:
distance to the calibrated cone is controlled by a natural ``error norm'' built
from off--type and primitive components, with a constant depending only on
$(n,p)$.  Finally, we explain why this replacement suffices for the
calibration--coercivity step in the Hodge program.
\end{abstract}

\section{Context and the issue}

Let $(X,\omega)$ be a K\"ahler manifold of complex dimension $n$ and fix
$1\le p\le n$.  At a point $x\in X$, let
$T_xX\simeq\mathbb C^n$ be a Hermitian vector space, and consider the real
vector space
\[
V := \Lambda^{2p} T_x^*X
\]
of real $2p$--forms at $x$, endowed with the inner product induced by the
K\"ahler metric.

The K\"ahler calibration form of dimension $(n-p)$ is
\[
\psi := \frac{\omega^{n-p}}{(n-p)!}.
\]
A real oriented $2(n-p)$--plane $W\subset T_xX$ is \emph{calibrated} by
$\psi$ if
\[
\psi|_W = \mathrm{vol}_W.
\]
Equivalently, $W$ is a complex $(n-p)$--plane for the complex structure $J$
and the K\"ahler metric.

Each such calibrated $(n-p)$--plane $W$ has an associated simple $(p,p)$--form
$\xi_W\in V$ (the dual calibration).  The \emph{calibrated cone} is the
closed convex cone
\begin{equation}\label{eq:calibrated-cone}
\mathcal C_x
:=
\left\{
\sum_{j=1}^N a_j \,\xi_{W_j} \;:\;
N\in\mathbb N,\; a_j\ge 0,\; W_j\text{ calibrated $(n-p)$--planes}
\right\}
\subset V.
\end{equation}
By construction, $\mathcal C_x$ consists exactly of finite nonnegative
combinations of simple calibrated $(p,p)$--forms, and its closure
contains all barycenter forms arising as averages of tangent planes of
$\psi$--calibrated cycles at $x$.

In the Hermitian model, one identifies $\Lambda^{p,p}T_x^*X$ with a space
of Hermitian operators on $\Lambda^{p,0}T_x^*X$, and each simple calibrated
$(p,p)$--form $\xi_W$ is mapped to a rank--one positive semidefinite (PSD)
projector.  A tempting but \emph{incorrect} slogan is:

\medskip
\emph{``Under this identification, the calibrated cone is the full PSD cone.''}
\medskip

This is true for $p=1$ (codimension one) and $p=n-1$ (dual codimension one),
but it fails in middle codimension $1<p<n-1$.  The goal of this note is to:

\begin{itemize}
\item Isolate and prove the precise linear--algebra statement showing
that the calibrated cone is a \emph{proper} subcone of the PSD cone when
$1<p<n-1$.
\item Provide a valid replacement: a quantitative approximation result that
controls distance to the calibrated cone by an error norm involving off--type
and primitive components.
\item Explain why this weaker but correct statement suffices for the
calibration--coercivity inequality used in the Hodge program.
\end{itemize}

\section{Hermitian model and decomposable vectors}

Fix a Hermitian vector space $E\simeq\mathbb C^n$ with inner product
$\langle\cdot,\cdot\rangle$, and consider the complex space
\[
\mathcal H := \Lambda^{p,0} E^*,
\]
with the induced inner product.  Let $\Herm(\mathcal H)$ denote the real
vector space of Hermitian operators on $\mathcal H$, endowed with the
Hilbert--Schmidt norm
\[
\|A\|_{\mathrm{HS}}^2 := \mathrm{tr}(A^*A).
\]

A vector $v\in\mathcal H$ is \emph{decomposable} if it can be written as
a wedge of $p$ covectors:
\[
v = \alpha_1\wedge\cdots\wedge\alpha_p
\]
for some $\alpha_i\in E^*$.  The set of decomposable lines in
$\mathbb P(\mathcal H)$ is the image of the Grassmannian $G_{\mathbb C}(p,n)$
under the Pl\"ucker embedding; it is a proper algebraic subvariety of
$\mathbb P(\mathcal H)$ unless $p=1$ or $p=n-1$.

Each nonzero $v\in\mathcal H$ determines a rank--one projector
\[
P_v := v\otimes v^* \in \Herm(\mathcal H),
\]
characterized by $P_v(w) = \langle w,v\rangle v$.  If $\|v\|=1$ then $P_v$
has eigenvalues $1$ on $\mathbb C v$ and $0$ on its orthogonal complement.

Let
\[
\mathrm{PSD}(\mathcal H) := \{A\in\Herm(\mathcal H) : A\ge 0\}
\]
denote the cone of positive semidefinite Hermitian operators.  The
following structure is standard.

\begin{proposition}[Extreme rays of the PSD cone]
\label{prop:extreme-rays}
Let $\mathcal H$ be a finite--dimensional Hermitian space.  Then:
\begin{enumerate}
\item Every $A\in\mathrm{PSD}(\mathcal H)$ can be written as a finite sum
\[ 
A = \sum_{i} \lambda_i P_{v_i},
\]
with $\lambda_i\ge 0$ and $\|v_i\|=1$.
\item The extreme rays of $\mathrm{PSD}(\mathcal H)$ are exactly the rays
$\mathbb R_{\ge 0}P_v$ for nonzero $v\in\mathcal H$.
\end{enumerate}
\end{proposition}

\begin{proof}
The spectral theorem gives an orthonormal eigenbasis $\{e_i\}$ of $\mathcal H$
with nonnegative eigenvalues $\lambda_i$ for any $A\ge 0$, so
\[
A = \sum_i \lambda_i (e_i\otimes e_i^*) = \sum_i \lambda_i P_{e_i}.
\]
This proves (1).

For (2), suppose $P_v=Y+Z$ with $Y,Z\in\mathrm{PSD}(\mathcal H)$.  For any
$w$ orthogonal to $v$ one has
\[
0 = \langle P_v w, w\rangle
  = \langle Yw,w\rangle + \langle Zw,w\rangle.
\]
Both terms are nonnegative, hence vanish.  Thus $Y$ and $Z$ vanish on
$\{v\}^\perp$ and have range in $\mathbb C v$.  On this one--dimensional
subspace they are just nonnegative scalars times $P_v$, so $Y=c_1P_v$,
$Z=c_2P_v$, and the ray $\mathbb R_{\ge 0}P_v$ is extreme.

Conversely, if $\mathrm{rank}(A)\ge 2$, write $A=\sum_i\lambda_iP_{e_i}$ with
at least two positive eigenvalues, say $\lambda_1,\lambda_2>0$.  Then
\[
A = \bigl((\lambda_1+\varepsilon)P_{e_1} + \sum_{i\ge 2} \lambda_i P_{e_i}\bigr)
  + \bigl((\lambda_1-\varepsilon)P_{e_1}\bigr),
\]
for small $\varepsilon>0$, expresses $A$ as a nontrivial sum of PSD operators
which are not proportional to $A$.  Hence the ray $\mathbb R_{\ge 0}A$ is not
extreme.
\end{proof}

\section{Calibrated cone vs.\ PSD cone in middle codimension}

In the K\"ahler setting, there is a canonical real--linear isometry
\[
\mathcal I : \Lambda^{p,p} T_x^*X \longrightarrow \Herm(\mathcal H)
\]
with the properties:
\begin{itemize}
\item For any simple calibrated $(p,p)$--form $\xi_W$ associated to a
calibrated $(n-p)$--plane $W$, one has $\mathcal I(\xi_W) = P_v$, where
$v\in\mathcal H$ is a decomposable unit vector representing the corresponding
complex $p$--plane in the Pl\"ucker embedding.
\item The K\"ahler form $\omega^p$ is mapped to a positive definite operator
proportional to the identity on $\mathcal H$.
\end{itemize}
We do not need the explicit construction of $\mathcal I$ here, only these
formal properties.

Define the \emph{Hermitian calibrated cone}
\[
\mathcal C^{\mathrm{Herm}}
:=
\mathrm{cone}\bigl\{P_v : v\in\mathcal H,\ v\ \text{decomposable},\ \|v\|=1\bigr\}
\subset\Herm(\mathcal H).
\]
This is precisely the image $\mathcal I(\mathcal C_x)$ of the calibrated cone
\eqref{eq:calibrated-cone} restricted to $(p,p)$--forms.

A natural question is whether $\mathcal C^{\mathrm{Herm}}$ coincides with the
full PSD cone $\mathrm{PSD}(\mathcal H)$.  The following proposition shows
that this fails in middle codimension.

\begin{proposition}[Calibrated cone is a proper subcone of PSD in middle codimension]
\label{prop:proper-subcone}
Let $\mathcal H=\Lambda^{p,0}\mathbb C^n$ with $1<p<n-1$.  Let $S$ denote
the set of unit decomposable vectors in $\mathcal H$, and define
\[
\mathcal C^{\mathrm{Herm}}
=
\mathrm{cone}\{P_v : v\in S\}.
\]
Then:
\begin{enumerate}
\item $\mathcal C^{\mathrm{Herm}}\subset \mathrm{PSD}(\mathcal H)$.
\item $\mathcal C^{\mathrm{Herm}}$ is a \emph{proper} subcone of
$\mathrm{PSD}(\mathcal H)$, i.e.
\[
\mathcal C^{\mathrm{Herm}} \subsetneq \mathrm{PSD}(\mathcal H).
\]
\item Equality $\mathcal C^{\mathrm{Herm}} = \mathrm{PSD}(\mathcal H)$ can
occur only when $p=1$ or $p=n-1$.
\end{enumerate}
\end{proposition}

\begin{proof}
Each generator $P_v$ with $v$ decomposable is rank--one, positive semidefinite,
so any nonnegative linear combination of such projectors is also PSD.  This
proves $\mathcal C^{\mathrm{Herm}}\subset \mathrm{PSD}(\mathcal H)$.

For the second assertion, recall that the set of decomposable lines in
$\mathbb P(\mathcal H)$ is the image of the Grassmannian $G_{\mathbb C}(p,n)$
under the Pl\"ucker embedding.  For $1<p<n-1$, the Pl\"ucker relations
cut out a nontrivial algebraic subvariety, so the decomposable locus is a
proper subset of $\mathbb P(\mathcal H)$.  Equivalently, there exist unit
vectors $w\in\mathcal H$ which are \emph{not} decomposable.

By Proposition~\ref{prop:extreme-rays}, every rank--one projector $P_w$ with
$\|w\|=1$ spans an extreme ray of the full PSD cone $\mathrm{PSD}(\mathcal H)$.
If
\[
\mathcal C^{\mathrm{Herm}} = \mathrm{PSD}(\mathcal H),
\]
then every such extreme ray must be generated by some $P_v$ with $v\in S$.
In particular, for each unit $w$ there must exist a unit decomposable $v$ such
that
\[
P_w = P_v.
\]
This implies that $w$ and $v$ differ by a complex scalar of unit modulus, so
$w$ is decomposable.  Thus every unit vector in $\mathcal H$ would be
decomposable, contradicting the existence of nondecomposable $w$ when
$1<p<n-1$.

The only cases where every $p$--vector is decomposable are $p=1$ and
$p=n-1$, where $\Lambda^{p,0}\mathbb C^n$ is naturally identified with
$\mathbb C^n$ or its dual.  This proves (3).
\end{proof}

\begin{remark}
In codimension one ($p=1$) and its dual ($p=n-1$), the Hermitian calibrated
cone does coincide with the PSD cone.  In particular, the codimension one
case recovers the classical identification of positive $(1,1)$--forms with
positive semidefinite Hermitian matrices.
\end{remark}

\section{A quantitative approximation to the calibrated cone}

The failure of equality
$\mathcal C^{\mathrm{Herm}} = \mathrm{PSD}(\mathcal H)$ in middle codimension
means one cannot identify the metric projection onto the calibrated cone with
the positive part of a Hermitian operator.  However, for the purposes of a
calibration--coercivity inequality, one does not need such an identification.
A \emph{quantitative} estimate of the form
\[
\mathrm{dist}_{\mathcal C_x}(\alpha_x)
\le
K(n,p)\,\Phi(\alpha_x)
\]
for a suitable norm $\Phi$ suffices.

We now state and prove such an estimate at a single point, in an abstract
finite--dimensional setting.  The key inputs are:
\begin{itemize}
\item the finite--dimensionality of $V=\Lambda^{2p}T_x^*X$,
\item the existence of an orthogonal decomposition into off--type,
primitive, and trace pieces, and
\item the fact that the calibrated cone $\mathcal C_x$ is a closed convex
cone with nonempty interior.
\end{itemize}

\subsection{Error norm from Hodge decomposition}

At a point $x$, write a real $2p$--form $\alpha\in V$ in Hodge components:
\[
\alpha
=
\alpha^{(p+1,p-1)} + \alpha^{(p,p)} + \alpha^{(p-1,p+1)}.
\]
Within the $(p,p)$--component, use the Lefschetz decomposition
\[
\alpha^{(p,p)}
=
\mu(\alpha)\,\omega^p
+
\bigl(\alpha^{(p,p)}\bigr)_{\mathrm{prim}},
\]
where $\mu(\alpha)\in\mathbb R$ is the trace coefficient and
$(\alpha^{(p,p)})_{\mathrm{prim}}$ is primitive.  These components are
mutually orthogonal with respect to the K\"ahler inner product on forms.

Define the \emph{error norm} $\Phi:V\to[0,\infty)$ by
\begin{equation}\label{eq:phi-def}
\Phi(\alpha)^2
:=
\bigl|\alpha^{(p+1,p-1)}\bigr|^2
+
\bigl|\alpha^{(p-1,p+1)}\bigr|^2
+
\bigl|\bigl(\alpha^{(p,p)}\bigr)_{\mathrm{prim}}\bigr|^2
+
\mu(\alpha)^2.
\end{equation}

\begin{lemma}[$\Phi$ is a norm]
\label{lem:phi-norm}
The function $\Phi$ defined by \eqref{eq:phi-def} is a norm on $V$.
\end{lemma}

\begin{proof}
Homogeneity and the triangle inequality follow from the Euclidean structure
on $V$ and the orthogonality of the four components being squared.  If
$\Phi(\alpha)=0$, then all four summands vanish:
\[
\alpha^{(p+1,p-1)} = \alpha^{(p-1,p+1)} = 0,\quad
(\alpha^{(p,p)})_{\mathrm{prim}}=0,\quad
\mu(\alpha)=0.
\]
Hence $\alpha^{(p,p)}=0$ and $\alpha=0$.  Thus $\Phi$ is a genuine norm.
\end{proof}

We emphasize that $\Phi$ measures the failure of $\alpha$ to be a pure
multiple of $\omega^p$ of type $(p,p)$; it vanishes exactly on the line
spanned by $\omega^p$.

\subsection{Distance to the cone and compactness}

Let $\|\cdot\|$ denote the K\"ahler pointwise norm on $V$.  For a closed
convex cone $\mathcal C_x\subset V$, define the Euclidean distance
\[
\mathrm{dist}_{\mathcal C_x}(\alpha)
:=
\inf_{\beta\in\mathcal C_x} \|\alpha-\beta\|,
\qquad \alpha\in V.
\]
The following lemma is purely finite--dimensional convex geometry.

\begin{lemma}[Abstract quantitative approximation]
\label{lem:abstract-quantitative}
Let $V$ be a finite--dimensional real normed space with norm $\|\cdot\|$,
let $\Phi$ be any norm on $V$, and let $\mathcal C\subset V$ be a closed
convex cone.  Define
\[
R(\alpha)
:=
\frac{\mathrm{dist}_{\mathcal C}(\alpha)}{\Phi(\alpha)}
\]
for $\alpha\neq 0$.  Then $R$ attains a finite supremum on the $\Phi$--unit
sphere, and there exists a constant $K<\infty$ such that
\[
\mathrm{dist}_{\mathcal C}(\alpha)
\le
K\,\Phi(\alpha)
\quad\text{for all }\alpha\in V.
\]
\end{lemma}

\begin{proof}
Let
\[
S := \{\alpha\in V : \Phi(\alpha)=1\}
\]
be the unit sphere of $\Phi$.  Since $\Phi$ is a norm on a finite--dimensional
space, $S$ is compact and nonempty.  The distance function
$\alpha\mapsto\mathrm{dist}_{\mathcal C}(\alpha)$ is continuous on $V$, so
its restriction to $S$ is continuous.  Therefore
\[
K := \sup_{\alpha\in S} \mathrm{dist}_{\mathcal C}(\alpha)
\]
is finite.

Now let $\alpha\in V$ be arbitrary.  If $\alpha=0$ the inequality is
trivial.  If $\alpha\neq 0$, set $\tilde\alpha := \alpha/\Phi(\alpha)$.
Then $\tilde\alpha\in S$, so $\mathrm{dist}_{\mathcal C}(\tilde\alpha)\le K$.
Because $\mathcal C$ is a cone,
\[
\mathrm{dist}_{\mathcal C}(\alpha)
=
\mathrm{dist}_{\mathcal C}\bigl(\Phi(\alpha)\,\tilde\alpha\bigr)
=
\Phi(\alpha)\,\mathrm{dist}_{\mathcal C}(\tilde\alpha)
\le
\Phi(\alpha)\,K.
\]
This is the desired inequality.
\end{proof}

Applying this to the calibrated cone $\mathcal C_x$ and the error norm
$\Phi$ defined by \eqref{eq:phi-def} gives the pointwise estimate we need.

\begin{theorem}[Quantitative approximation to the calibrated cone]
\label{thm:quantitative-calibrated}
Let $x\in X$ and let $\mathcal C_x\subset V=\Lambda^{2p}T_x^*X$ be the
calibrated cone \eqref{eq:calibrated-cone}.  Let $\Phi$ be the error norm
\eqref{eq:phi-def}.  Then there exists a finite constant $K_x$ such that
for all $\alpha\in V$,
\[
\mathrm{dist}_{\mathcal C_x}(\alpha)
\le
K_x\,\Phi(\alpha),
\]
i.e.
\[
\mathrm{dist}_{\mathcal C_x}(\alpha)^2
\le
K_x^2\Bigl(
\bigl|\alpha^{(p+1,p-1)}\bigr|^2
+
\bigl|\alpha^{(p-1,p+1)}\bigr|^2
+
\bigl|\bigl(\alpha^{(p,p)}\bigr)_{\mathrm{prim}}\bigr|^2
+
\mu(\alpha)^2
\Bigr).
\]
\end{theorem}

\begin{proof}
Apply Lemma~\ref{lem:abstract-quantitative} with $V=\Lambda^{2p}T_x^*X$,
$\mathcal C=\mathcal C_x$, and $\Phi$ as in \eqref{eq:phi-def}.
\end{proof}

In fact, the constant $K_x$ can be chosen uniformly in $x$.  The K\"ahler
structure identifies each tangent space $T_xX$ unitarily with a fixed model
$\mathbb C^n$, and under such unitary changes of frame both:
\begin{itemize}
\item the calibrated cone $\mathcal C_x$, and
\item the decomposition into off--type, primitive, and trace parts,
\end{itemize}
are invariant.  Thus the ratio
\[
\frac{\mathrm{dist}_{\mathcal C_x}(\alpha)}{\Phi(\alpha)}
\]
depends only on the direction of $\alpha$ and on $(n,p)$, not on $x$.
Therefore the supremum over the $\Phi$--unit sphere is a single constant
$K(n,p)$, independent of $x$.

\begin{corollary}[Uniform quantitative approximation]
\label{cor:uniform-K}
There exists a constant $K(n,p)$ such that for all $x\in X$ and all
$\alpha_x\in\Lambda^{2p}T_x^*X$,
\[
\mathrm{dist}_{\mathcal C_x}(\alpha_x)^2
\le
K(n,p)^2\Bigl(
\bigl|\alpha_x^{(p+1,p-1)}\bigr|^2
+
\bigl|\alpha_x^{(p-1,p+1)}\bigr|^2
+
\bigl|\bigl(\alpha_x^{(p,p)}\bigr)_{\mathrm{prim}}\bigr|^2
+
\mu(\alpha_x)^2
\Bigr).
\]
\end{corollary}

\begin{proof}
Fix a model Hermitian space $E\simeq\mathbb C^n$ and identify each
$T_xX$ unitarily with $E$.  Under this identification, the calibrated
cones and the error norm $\Phi$ are invariant, so the constant $K_x$ in
Theorem~\ref{thm:quantitative-calibrated} is independent of $x$.
\end{proof}

\section{Why this is sufficient for calibration--coercivity}

The global calibration--coercivity inequality used in the Hodge program
has the schematic form
\begin{equation}\label{eq:coercivity-target}
E(\alpha)-E(\gamma_{\mathrm{harm}})
\;\ge\;
c(n,p)\,\Def_{\mathrm{cone}}(\alpha),
\end{equation}
where:
\begin{itemize}
\item $\alpha$ is a smooth closed representative of a fixed Hodge class
$[\gamma]$;
\item $\gamma_{\mathrm{harm}}$ is the harmonic representative of $[\gamma]$;
\item $E(\cdot)$ is the Dirichlet energy $\int_X|\cdot|^2$ of the covariant
derivative (up to standard normalization);
\item $\Def_{\mathrm{cone}}(\alpha)$ is the global cone defect
\[
\Def_{\mathrm{cone}}(\alpha)
:=
\int_X \mathrm{dist}_{\mathcal C_x}(\alpha_x)^2\,\mathrm{vol}_\omega(x),
\]
where $\mathcal C_x$ is the calibrated cone at $x$.
\end{itemize}

The analytic part of the Hodge theory (standard K\"ahler identities and
Bochner formulas) provides a global inequality of the form
\begin{equation}\label{eq:analytic-gap}
\int_X\Bigl(
\bigl|\alpha^{(p+1,p-1)}\bigr|^2
+
\bigl|\alpha^{(p-1,p+1)}\bigr|^2
+
\bigl|\bigl(\alpha^{(p,p)}-\gamma_{\mathrm{harm}}^{(p,p)}\bigr)_{\mathrm{prim}}\bigr|^2
+
\mu(\alpha-\gamma_{\mathrm{harm}})^2
\Bigr)
\le
A(n,p)\,\bigl(E(\alpha)-E(\gamma_{\mathrm{harm}})\bigr),
\end{equation}
for some constant $A(n,p)$.  This is the ``Step~1'' estimate in the usual
calibration--coercivity argument; it does not involve the cone at all.

The cone geometry enters only when we relate the pointwise distance to the
calibrated cone to these components.  In earlier drafts, this was done via
an incorrect identification of $\mathcal C^{\mathrm{Herm}}$ with the full
PSD cone, leading to a spectral formula for the distance.  As
Proposition~\ref{prop:proper-subcone} shows, this identification fails in
middle codimension.

However, the quantitative approximation result of
Corollary~\ref{cor:uniform-K} is exactly what is needed.  For each $x$,
apply Corollary~\ref{cor:uniform-K} to the difference
\[
\delta_x := \alpha_x-\gamma_{\mathrm{harm},x}.
\]
Integrating over $X$ yields
\[
\Def_{\mathrm{cone}}(\alpha)
=
\int_X \mathrm{dist}_{\mathcal C_x}(\delta_x)^2
\le
K(n,p)^2\int_X \Phi(\delta_x)^2
\]
with $\Phi$ as in \eqref{eq:phi-def}.  The right--hand side is precisely
the integrand appearing in \eqref{eq:analytic-gap}.  Combining the two
gives
\[
\Def_{\mathrm{cone}}(\alpha)
\le
K(n,p)^2\,A(n,p)\,\bigl(E(\alpha)-E(\gamma_{\mathrm{harm}})\bigr),
\]
which we can rewrite as
\[
E(\alpha)-E(\gamma_{\mathrm{harm}})
\;\ge\;
\frac{1}{K(n,p)^2\,A(n,p)}\,\Def_{\mathrm{cone}}(\alpha).
\]
This is the desired calibration--coercivity inequality
\eqref{eq:coercivity-target} with
\[
c(n,p) = \frac{1}{K(n,p)^2\,A(n,p)}.
\]

\begin{remark}
The constant $c(n,p)$ obtained this way is not explicit; it is the product
of the analytic constant $A(n,p)$ from the K\"ahler identities and the
geometric constant $K(n,p)$ from the calibrated cone geometry.  For the
purposes of the Hodge program, only the \emph{existence} of a positive
constant depending on $(n,p)$ is needed.  No step in the subsequent SYR
construction or algebraicity argument uses the precise value of $c(n,p)$.
\end{remark}

\section{Summary}

In middle codimension $1<p<n-1$, the Hermitian image of the calibrated cone
is a proper $U(n)$--invariant subcone of the positive semidefinite cone
on $\Lambda^{p,0}\mathbb C^n$.  The extreme--ray structure of the PSD cone
and the existence of nondecomposable $p$--vectors rule out any equality
between these cones.

Despite this, the calibration--coercivity step in the Hodge program does
not require an exact identification with the PSD cone.  A finite--dimensional
compactness argument shows that the distance to the calibrated cone is
\emph{quantitatively} controlled by an error norm built from off--type,
primitive, and trace components.  This, combined with the standard analytic
estimate controlling those components by the Dirichlet energy gap, yields
a full calibration--coercivity inequality with a constant depending only on
$(n,p)$.

Thus the flawed PSD--identification can be removed and replaced by a valid
quantitative approximation statement, without weakening the global
consequences needed for the Hodge program.
\end{document}
