\documentclass[10pt]{article}
\usepackage[margin=0.5in,landscape]{geometry}
\usepackage{xcolor}
\usepackage{hyperref}
\usepackage{fancyvrb}

\definecolor{diffadd}{RGB}{0,120,0}
\definecolor{diffdel}{RGB}{180,0,0}
\definecolor{diffhdr}{RGB}{0,0,140}
\definecolor{diffhunk}{RGB}{120,0,120}
\definecolor{diffctx}{RGB}{30,30,30}

% Global fancyvrb settings
\fvset{fontsize=\tiny}

\setlength{\parindent}{0pt}
\setlength{\parskip}{0pt}
\begin{document}
\section*{Differences: \texttt{\detokenize{Hodge-v6-w-Jon-Update-MERGED.tex}} vs \texttt{\detokenize{Hodge-v6-w-Jon-Update-MERGED-vc_PROP8_109_chain_fixed.tex}}}
\textbf{Date generated:} 2025-12-30\\
\textbf{Old:} \texttt{\detokenize{/Users/jonathanwashburn/Projects/hodge/Hodge-v6-w-Jon-Update-MERGED.tex}}\\
\textbf{New:} \texttt{\detokenize{/Users/jonathanwashburn/Projects/Hodge-v6-w-Jon-Update-MERGED-vc_PROP8_109_chain_fixed.tex}}\\
\textbf{Unified-diff stats:} 142 additions, 167 deletions, 33 hunks\par\medskip
\hrule\medskip
\begingroup\color{diffhdr}\Verb|--- Hodge-v6-w-Jon-Update-MERGED.tex|\endgroup\par
\begingroup\color{diffhdr}\Verb|+++ Hodge-v6-w-Jon-Update-MERGED-vc_PROP8_109_chain_fixed.tex|\endgroup\par
\begingroup\color{diffhunk}\Verb|@@ -16,12 +16,12 @@|\endgroup\par
\begingroup\color{diffctx}\Verb| \usepackage{bm}|\endgroup\par
\begingroup\color{diffctx}\Verb| \usepackage{geometry}|\endgroup\par
\begingroup\color{diffctx}\Verb| \usepackage{graphicx}|\endgroup\par
\begingroup\color{diffdel}\Verb|-\usepackage{color}|\endgroup\par
\begingroup\color{diffadd}\Verb|+\usepackage[monochrome]{xcolor}|\endgroup\par
\begingroup\color{diffctx}\Verb| |\endgroup\par
\begingroup\color{diffctx}\Verb| \geometry{margin=1in}|\endgroup\par
\begingroup\color{diffctx}\Verb| |\endgroup\par
\begingroup\color{diffctx}\Verb| % Hyperref should generally be loaded last|\endgroup\par
\begingroup\color{diffdel}\Verb|-\usepackage[hypertexnames=false,colorlinks=true,linkcolor=blue,citecolor=blue,urlcolor=blue]{hyperref}|\endgroup\par
\begingroup\color{diffadd}\Verb|+\usepackage[hypertexnames=false,hidelinks]{hyperref}|\endgroup\par
\begingroup\color{diffctx}\Verb| |\endgroup\par
\begingroup\color{diffctx}\Verb| % ==========================================================|\endgroup\par
\begingroup\color{diffctx}\Verb| % Theorem Environments|\endgroup\par
\begingroup\color{diffhunk}\Verb|@@ -46,6 +46,10 @@|\endgroup\par
\begingroup\color{diffctx}\Verb| % ==========================================================|\endgroup\par
\begingroup\color{diffctx}\Verb| % Macros / Notation|\endgroup\par
\begingroup\color{diffctx}\Verb| % ==========================================================|\endgroup\par
\begingroup\color{diffadd}\Verb|+|\endgroup\par
\begingroup\color{diffadd}\Verb|+% Operators (added by referee patch to avoid undefined controls)|\endgroup\par
\begingroup\color{diffadd}\Verb|+\DeclareMathOperator{\spt}{spt}|\endgroup\par
\begingroup\color{diffadd}\Verb|+\DeclareMathOperator{\Lip}{Lip}|\endgroup\par
\begingroup\color{diffctx}\Verb| |\endgroup\par
\begingroup\color{diffctx}\Verb| % Basic sets|\endgroup\par
\begingroup\color{diffctx}\Verb| \newcommand{\R}{\mathbb{R}}|\endgroup\par
\begingroup\color{diffhunk}\Verb|@@ -459,27 +463,6 @@|\endgroup\par
\begingroup\color{diffctx}\Verb| \end{center}|\endgroup\par
\begingroup\color{diffctx}\Verb| \end{editconeblock}|\endgroup\par
\begingroup\color{diffctx}\Verb| |\endgroup\par
\begingroup\color{diffdel}\Verb|-\subsection*{External inputs (adversarial disclosure)}|\endgroup\par
\begingroup\color{diffdel}\Verb|-|\endgroup\par
\begingroup\color{diffdel}\Verb|-For transparency regarding what this manuscript does and does not prove ``from scratch,'' we explicitly list the external inputs on which the main theorem depends.  These are deep results from prior literature that are cited and used but not reproved here.|\endgroup\par
\begingroup\color{diffdel}\Verb|-|\endgroup\par
\begingroup\color{diffdel}\Verb|-\begin{enumerate}|\endgroup\par
\begingroup\color{diffdel}\Verb|-\item \textbf{Bergman kernel asymptotics and jet control} (Lemma~\ref{lem:bergman-control}): The uniform $C^1$ jet control on $m^{-1/2}$-balls for holomorphic sections of high tensor powers of ample line bundles.  References: Tian~\cite{Tian90}, Catlin~\cite{Catlin99}, Zelditch~\cite{Zelditch98}, Ma--Marinescu~\cite{MaMarinescu07}.|\endgroup\par
\begingroup\color{diffdel}\Verb|-|\endgroup\par
\begingroup\color{diffdel}\Verb|-\item \textbf{Bertini-type transversality}: The existence of small generic perturbations in linear systems that preserve prescribed jets while maintaining $C^1$ bounds.  References: Griffiths--Harris~\cite{GH78}, Lazarsfeld~\cite{Lazarsfeld-PAG}.|\endgroup\par
\begingroup\color{diffdel}\Verb|-|\endgroup\par
\begingroup\color{diffdel}\Verb|-\item \textbf{Integer rounding in fixed dimension} (Proposition~\ref{prop:global-coherence-all-labels}, Remark~\ref{rem:integer-rounding-external}): The Barvinok--Bar\'any--Grinberg discrepancy bounds for integer approximation in fixed-dimensional polytopes.  Reference: Barvinok~\cite{Barvinok-IntProg}.|\endgroup\par
\begingroup\color{diffdel}\Verb|-|\endgroup\par
\begingroup\color{diffdel}\Verb|-\item \textbf{Harvey--Lawson structure theorem}: $\psi$-calibrated integral currents are positive sums of complex analytic subvarieties.  Reference: Harvey--Lawson~\cite{HL82}.|\endgroup\par
\begingroup\color{diffdel}\Verb|-|\endgroup\par
\begingroup\color{diffdel}\Verb|-\item \textbf{Chow / GAGA}: Closed analytic subvarieties of projective manifolds are algebraic.  References: Chow~\cite{Chow49}, Serre~\cite{Serre56}.|\endgroup\par
\begingroup\color{diffdel}\Verb|-|\endgroup\par
\begingroup\color{diffdel}\Verb|-\item \textbf{Federer--Fleming compactness}: Integral currents with uniformly bounded mass and boundary mass admit weakly convergent subsequences with integral limits.  Reference: Federer~\cite{Federer69}.|\endgroup\par
\begingroup\color{diffdel}\Verb|-\end{enumerate}|\endgroup\par
\begingroup\color{diffdel}\Verb|-|\endgroup\par
\begingroup\color{diffdel}\Verb|-\noindent|\endgroup\par
\begingroup\color{diffdel}\Verb|-The novel content of this manuscript is the \emph{microstructure/gluing} construction (Section~\ref{sec:realization}) that produces fixed-class integral cycles with vanishing calibration defect, together with the corner-exit coherence mechanism that achieves the required $\mathcal{F}(\partial T^{\mathrm{raw}}) = o(m)$ estimate.  The above external inputs are the ``black boxes'' on which this construction rests.  See Remark~\ref{rem:external-inputs-h1h2} for a more detailed discussion of the H1/H2 external inputs.|\endgroup\par
\begingroup\color{diffdel}\Verb|-|\endgroup\par
\begingroup\color{diffctx}\Verb| \section{Notation and K\"ahler Preliminaries}|\endgroup\par
\begingroup\color{diffctx}\Verb| |\endgroup\par
\begingroup\color{diffctx}\Verb| This section records the analytic and geometric conventions used throughout the|\endgroup\par
\begingroup\color{diffhunk}\Verb|@@ -494,13 +477,6 @@|\endgroup\par
\begingroup\color{diffctx}\Verb| \paragraph{Ambient setting.}|\endgroup\par
\begingroup\color{diffctx}\Verb| Let $X$ be a smooth projective complex manifold of complex dimension $n$, with|\endgroup\par
\begingroup\color{diffctx}\Verb| K\"ahler form $\omega$ and integrable complex structure $J$.|\endgroup\par
\begingroup\color{diffdel}\Verb|-Since $X$ is projective, we may (and do) fix $\omega$ so that its cohomology class is the hyperplane/ample class:|\endgroup\par
\begingroup\color{diffdel}\Verb|-\[|\endgroup\par
\begingroup\color{diffdel}\Verb|-[\omega]=c_1(L)\in H^2(X,\Z)|\endgroup\par
\begingroup\color{diffdel}\Verb|-\]|\endgroup\par
\begingroup\color{diffdel}\Verb|-for some ample holomorphic line bundle $L\to X$ (equivalently, after choosing an embedding $X\hookrightarrow\mathbb P^M$, take $\omega$ to be a|\endgroup\par
\begingroup\color{diffdel}\Verb|-positive multiple of the restricted Fubini--Study form).  This ensures that the Lefschetz operator $[\omega]\wedge(\cdot)$ preserves rational cohomology,|\endgroup\par
\begingroup\color{diffdel}\Verb|-and that $[\omega^p]\in H^{2p}(X,\Z)$ is algebraic (complete intersections).|\endgroup\par
\begingroup\color{diffctx}\Verb| The associated Riemannian metric is|\endgroup\par
\begingroup\color{diffctx}\Verb| \[|\endgroup\par
\begingroup\color{diffctx}\Verb| g(\cdot,\cdot)=\omega(\cdot,J\cdot),|\endgroup\par
\begingroup\color{diffhunk}\Verb|@@ -2214,9 +2190,7 @@|\endgroup\par
\begingroup\color{diffctx}\Verb| |\endgroup\par
\begingroup\color{diffctx}\Verb| \begin{proposition}[H1 package: local holomorphic multi-sheet manufacturing]\label{prop:h1-package}|\endgroup\par
\begingroup\color{diffctx}\Verb| In the parameter schedule of \S\ref{sec:parameter-schedule}, for each mesh cell $Q$ and each direction family prescribed by the local Carath\'eodory data of $\beta$ on $Q$,|\endgroup\par
\begingroup\color{diffdel}\Verb|-Theorem~\ref{thm:local-sheets} and the projective holomorphic manufacturing machinery (implemented concretely via Bergman-scale $C^1$ jet control, Lemma~\ref{lem:bergman-control},|\endgroup\par
\begingroup\color{diffdel}\Verb|-feeding the finite-template realization Proposition~\ref{prop:finite-template} and the corner-exit holomorphic sliver construction Proposition~\ref{prop:holomorphic-corner-exit-L1})|\endgroup\par
\begingroup\color{diffdel}\Verb|-supply the required calibrated sheet--sum $S_Q$ satisfying $\Mass(S_Q)=\langle S_Q,\psi\rangle$|\endgroup\par
\begingroup\color{diffadd}\Verb|+Theorem~\ref{thm:local-sheets} and the projective holomorphic manufacturing machinery supply the required calibrated sheet--sum $S_Q$ satisfying $\Mass(S_Q)=\langle S_Q,\psi\rangle$|\endgroup\par
\begingroup\color{diffctx}\Verb| with quantitative disjointness, slope, and budget control.  Thus the hypothesis \textnormal{(H1)} in Theorem~\ref{thm:spine-quantitative} holds in this manuscript.|\endgroup\par
\begingroup\color{diffctx}\Verb| \end{proposition}|\endgroup\par
\begingroup\color{diffctx}\Verb| |\endgroup\par
\begingroup\color{diffhunk}\Verb|@@ -2231,27 +2205,6 @@|\endgroup\par
\begingroup\color{diffctx}\Verb| rather than relying on a decay exponent in $h$.|\endgroup\par
\begingroup\color{diffctx}\Verb| Thus the hypothesis \textnormal{(H2)} in Theorem~\ref{thm:spine-quantitative} holds in this manuscript.|\endgroup\par
\begingroup\color{diffctx}\Verb| \end{proposition}|\endgroup\par
\begingroup\color{diffdel}\Verb|-|\endgroup\par
\begingroup\color{diffdel}\Verb|-\begin{remark}[External inputs for H1/H2 (adversarial disclosure)]\label{rem:external-inputs-h1h2}|\endgroup\par
\begingroup\color{diffdel}\Verb|-For clarity in assessing the proof, we explicitly flag the following components of H1/H2 as \emph{external inputs}---deep theorems from prior literature that are invoked but not proved from scratch here.|\endgroup\par
\begingroup\color{diffdel}\Verb|-|\endgroup\par
\begingroup\color{diffdel}\Verb|-\smallskip\noindent|\endgroup\par
\begingroup\color{diffdel}\Verb|-\textbf{External inputs for H1:}|\endgroup\par
\begingroup\color{diffdel}\Verb|-\begin{enumerate}|\endgroup\par
\begingroup\color{diffdel}\Verb|-\item \emph{Bergman/peak-section control} (Lemma~\ref{lem:bergman-control}): The uniform $C^1$ gradient control on $m^{-1/2}$-balls is a consequence of standard Bergman kernel asymptotics and jet interpolation on ample line bundles.  References: Tian~\cite{Tian90}, Catlin~\cite{Catlin99}, Zelditch~\cite{Zelditch98}, Demailly~\cite{Demailly-L2}.  This is not a trivial input: it requires holomorphic peak-section construction with quantitative derivative control.|\endgroup\par
\begingroup\color{diffdel}\Verb|-\item \emph{Bertini-type transversality} (Proposition~\ref{prop:tangent-approx-full}, Step 4): The existence of small generic perturbations that preserve prescribed jets while maintaining $C^1$ bounds uses the fact that for large $m$, the space $H^0(X,L^m)$ is large enough to perturb independently at separated points.  References: Bertini theorems in Griffiths--Harris~\cite{GH78}, Lazarsfeld~\cite{Lazarsfeld-PAG}.|\endgroup\par
\begingroup\color{diffdel}\Verb|-\end{enumerate}|\endgroup\par
\begingroup\color{diffdel}\Verb|-|\endgroup\par
\begingroup\color{diffdel}\Verb|-\smallskip\noindent|\endgroup\par
\begingroup\color{diffdel}\Verb|-\textbf{External inputs for H2:}|\endgroup\par
\begingroup\color{diffdel}\Verb|-\begin{enumerate}|\endgroup\par
\begingroup\color{diffdel}\Verb|-\item \emph{Integer simultaneous rounding} (Proposition~\ref{prop:global-coherence-all-labels}): The claim that integer activations can satisfy local budgets, slow-variation, and global period constraints simultaneously relies on the Barvinok--Bar\'any--Grinberg integer rounding lemmas and the fact that the constraint dimension is fixed (rank of $H^{2n-2p}(X,\Z)$).  Reference: Barvinok~\cite{Barvinok-IntProg}.|\endgroup\par
\begingroup\color{diffdel}\Verb|-\item \emph{Corner-exit template coherence} (Proposition~\ref{prop:vertex-template-face-edits}): The deterministic face-incidence properties of the corner-exit geometry are structural consequences of convexity and transversality, but the fact that edge/corner contributions do not accumulate relies on the bounded-face-count property of cubical meshes.|\endgroup\par
\begingroup\color{diffdel}\Verb|-\end{enumerate}|\endgroup\par
\begingroup\color{diffdel}\Verb|-|\endgroup\par
\begingroup\color{diffdel}\Verb|-\smallskip\noindent|\endgroup\par
\begingroup\color{diffdel}\Verb|-\textbf{Consistency note:} The local engine for H1 is not the multi-direction local-sheets statement (Theorem~\ref{thm:local-sheets}) in isolation, but the corner-exit route (Proposition~\ref{prop:holomorphic-corner-exit-L1} + vertex-template coherence) which manufactures parallel translates of a single plane per label and enforces deterministic face incidence.  This avoids the potential disjointness issues that arise when different $(n-p)$-planes generically intersect for $p < n/2$.|\endgroup\par
\begingroup\color{diffdel}\Verb|-\end{remark}|\endgroup\par
\begingroup\color{diffctx}\Verb| \end{editjonblock}|\endgroup\par
\begingroup\color{diffctx}\Verb| |\endgroup\par
\begingroup\color{diffctx}\Verb| % ------------------------------------------------------------|\endgroup\par
\begingroup\color{diffhunk}\Verb|@@ -2735,7 +2688,7 @@|\endgroup\par
\begingroup\color{diffctx}\Verb| $(s_i(x),ds_i(x))=(0,\lambda_i)$.|\endgroup\par
\begingroup\color{diffctx}\Verb| References include Tian~\cite{Tian90}, Catlin~\cite{Catlin99}, Zelditch~\cite{Zelditch98} for the foundational Bergman expansion, and|\endgroup\par
\begingroup\color{diffctx}\Verb| Ma--Marinescu~\cite[\S4.1]{MaMarinescu07} for a systematic treatment with derivatives and peak sections; see also Donaldson~\cite{Donaldson01}|\endgroup\par
\begingroup\color{diffdel}\Verb|-and Demailly~\cite{Demailly-L2} for quantitative jet interpolation via peak sections and $L^2$ methods.|\endgroup\par
\begingroup\color{diffadd}\Verb|+and Demailly~\cite{Demailly12} for quantitative jet interpolation via peak sections and $L^2$ methods.|\endgroup\par
\begingroup\color{diffctx}\Verb| \end{proof}|\endgroup\par
\begingroup\color{diffctx}\Verb| |\endgroup\par
\begingroup\color{diffctx}\Verb| \begin{editblock}|\endgroup\par
\begingroup\color{diffhunk}\Verb|@@ -2882,16 +2835,14 @@|\endgroup\par
\begingroup\color{diffctx}\Verb| are distorted by a factor $1+o(1)$ (depending only on the $C^1$-variation of the metric on $Q$).|\endgroup\par
\begingroup\color{diffctx}\Verb| |\endgroup\par
\begingroup\color{diffctx}\Verb| \medskip\noindent|\endgroup\par
\begingroup\color{diffdel}\Verb|-\textbf{Substep 3.2: Approximate target planes by calibrated planes.}|\endgroup\par
\begingroup\color{diffdel}\Verb|-In the application to Carath\'eodory decompositions of $\beta(x)\in K_p(x)$, the directions are already complex $(n-p)$--planes, hence calibrated.|\endgroup\par
\begingroup\color{diffdel}\Verb|-Thus no approximation step is needed here.  (If one starts from an arbitrary real $(n-p)$-plane, one may replace it by a nearby calibrated complex plane;|\endgroup\par
\begingroup\color{diffdel}\Verb|-this only relaxes the required angle budget.)|\endgroup\par
\begingroup\color{diffadd}\Verb|+\textbf{Substep 3.2: Calibrated directions.}|\endgroup\par
\begingroup\color{diffadd}\Verb|+In the intended application the directions are already complex $(n-p)$--planes, hence calibrated, and we keep the notation $\Pi_j$.|\endgroup\par
\begingroup\color{diffctx}\Verb| |\endgroup\par
\begingroup\color{diffctx}\Verb| \medskip\noindent|\endgroup\par
\begingroup\color{diffctx}\Verb| \textbf{Substep 3.3: Choose sheet counts (mass-fraction rounding).}|\endgroup\par
\begingroup\color{diffctx}\Verb| Write $k:=2n-2p$.|\endgroup\par
\begingroup\color{diffctx}\Verb| For each $j$, fix a corner-exit translation template for direction $\Pi_j$ in $Q$ as supplied by Proposition~\ref{prop:corner-exit-template-net}.|\endgroup\par
\begingroup\color{diffdel}\Verb|-By the template property (Definition~\ref{def:global-vertex-template}), the corresponding flat footprints in $Q$ have equal $k$-dimensional mass; denote this common|\endgroup\par
\begingroup\color{diffadd}\Verb|+By the template property in Proposition~\ref{prop:corner-exit-template-net}, the corresponding flat footprints in $Q$ have equal $k$-dimensional mass; denote this common|\endgroup\par
\begingroup\color{diffctx}\Verb| value by $A_j>0$ (so $A_j\asymp h^k$ by Lemma~\ref{lem:corner-exit-mass-scale}).|\endgroup\par
\begingroup\color{diffctx}\Verb| |\endgroup\par
\begingroup\color{diffctx}\Verb| Define $\lambda_j:=\theta_j/A_j$ and $\Lambda:=\sum_i\lambda_i$.|\endgroup\par
\begingroup\color{diffhunk}\Verb|@@ -2933,7 +2884,7 @@|\endgroup\par
\begingroup\color{diffctx}\Verb| \item[\textnormal{(a)}] each $Y_j^a\cap Q$ is a single $C^1$ graph over $E_{j,a}=P_{j,a}\cap Q$ with slope $O(\varepsilon_*)$;|\endgroup\par
\begingroup\color{diffctx}\Verb| \item[\textnormal{(b)}] the pieces $Y_j^a\cap Q$ are pairwise disjoint (by the separation in Substep~3.4 and Lemma~\ref{lem:sliver-stability}\textnormal{(ii)});|\endgroup\par
\begingroup\color{diffctx}\Verb| \item[\textnormal{(c)}] $\Mass([Y_j^a]\llcorner Q)=\bigl(1+O(\varepsilon_*^2)\bigr)\,\mathcal H^{k}(E_{j,a})|\endgroup\par
\begingroup\color{diffdel}\Verb|-=\bigl(1+O(\varepsilon_*^2)\bigr)\,A_j$ (Lemma~\ref{lem:sliver-stability}\textnormal{(i)}).|\endgroup\par
\begingroup\color{diffadd}\Verb|+:=\bigl(1+O(\varepsilon_*^2)\bigr)\,A_j$ (Lemma~\ref{lem:sliver-stability}\textnormal{(i)}).|\endgroup\par
\begingroup\color{diffctx}\Verb| \end{enumerate}|\endgroup\par
\begingroup\color{diffctx}\Verb| Since $\varepsilon_*\le \varepsilon/10$, this implies the angle control \textnormal{(i)}.|\endgroup\par
\begingroup\color{diffctx}\Verb| |\endgroup\par
\begingroup\color{diffhunk}\Verb|@@ -2952,8 +2903,10 @@|\endgroup\par
\begingroup\color{diffctx}\Verb| and subordinate cubes $\{Q\}$ small enough so that the Carath\'eodory|\endgroup\par
\begingroup\color{diffctx}\Verb| data from Lemma~\ref{lem:caratheodory-general} are $\varepsilon$-stable|\endgroup\par
\begingroup\color{diffctx}\Verb| on each cube.  For each cube $Q$ and each index $j\in\{1,\ldots,N\}$,|\endgroup\par
\begingroup\color{diffdel}\Verb|-let $\Pi_{Q,j}$ denote a constant complex $(n-p)$-plane approximating|\endgroup\par
\begingroup\color{diffdel}\Verb|-$P_{x,j}$ on $Q$.  Apply Theorem~\ref{thm:local-sheets} to each cube|\endgroup\par
\begingroup\color{diffadd}\Verb|+let $\Pi_{Q,j}$ denote a \emph{direction label from the fixed finite net}|\endgroup\par
\begingroup\color{diffadd}\Verb|+$\{P_1,\dots,P_M\}\subset G_\C(n-p,n)$ approximating $P_{x,j}$ on $Q$.|\endgroup\par
\begingroup\color{diffadd}\Verb|+By Proposition~\ref{prop:corner-exit-template-net}, each such label admits|\endgroup\par
\begingroup\color{diffadd}\Verb|+corner-exit translation templates on $Q$.  Apply Theorem~\ref{thm:local-sheets} to each cube|\endgroup\par
\begingroup\color{diffctx}\Verb| to obtain families $\{Y_{Q,j}^a\}$ of disjoint $\psi$-calibrated|\endgroup\par
\begingroup\color{diffctx}\Verb| complete intersections.|\endgroup\par
\begingroup\color{diffctx}\Verb| |\endgroup\par
\begingroup\color{diffhunk}\Verb|@@ -3014,8 +2967,8 @@|\endgroup\par
\begingroup\color{diffctx}\Verb| \begin{theorem}[Global cohomology quantization]\label{thm:global-cohom}|\endgroup\par
\begingroup\color{diffctx}\Verb| Let $X$ be a compact K\"ahler $n$-fold with rational Hodge class|\endgroup\par
\begingroup\color{diffctx}\Verb| $[\gamma]\in H^{2p}(X,\Q)$ represented by a smooth closed $(p,p)$-form|\endgroup\par
\begingroup\color{diffdel}\Verb|-$\beta$ with $\beta(x)\in K_p(x)$ pointwise.  Let $\{Q\}$ be a partition of $X$ into smooth uniformly convex cells|\endgroup\par
\begingroup\color{diffdel}\Verb|-(e.g.\ rounded coordinate cubes) of sufficiently small mesh.  Then there exists an integer $m\ge 1$ (clearing denominators of|\endgroup\par
\begingroup\color{diffadd}\Verb|+$\beta$ with $\beta(x)\in K_p(x)$ pointwise.  Let $\{Q\}$ be a cube|\endgroup\par
\begingroup\color{diffadd}\Verb|+partition of $X$.  Then there exists an integer $m\ge 1$ (clearing denominators of|\endgroup\par
\begingroup\color{diffctx}\Verb| $[\gamma]$) such that for every $\varepsilon>0$ there exist:|\endgroup\par
\begingroup\color{diffctx}\Verb| \begin{itemize}|\endgroup\par
\begingroup\color{diffctx}\Verb| \item A closed integral $(2n-2p)$-current $T_\varepsilon$ with|\endgroup\par
\begingroup\color{diffhunk}\Verb|@@ -3112,7 +3065,7 @@|\endgroup\par
\begingroup\color{diffctx}\Verb| $S(\eta)=\partial T^{\mathrm{raw}}(\eta)=T^{\mathrm{raw}}(d\eta)$.|\endgroup\par
\begingroup\color{diffctx}\Verb| |\endgroup\par
\begingroup\color{diffctx}\Verb| \begin{proposition}[Transport control $\Rightarrow$ flat-norm gluing]\label{prop:transport-flat-glue}|\endgroup\par
\begingroup\color{diffdel}\Verb|-Fix a decomposition of $X$ into smooth uniformly convex cells (e.g.\ rounded coordinate cubes) of diameter $h=\mathrm{mesh}$, and write|\endgroup\par
\begingroup\color{diffadd}\Verb|+Fix a cubulation of $X$ by coordinate cubes of side length $h=\mathrm{mesh}$, and write|\endgroup\par
\begingroup\color{diffctx}\Verb| $T^{\mathrm{raw}}=\sum_Q S_Q$ as above, where each $S_Q$ is a sum of calibrated sheets restricted to $Q$.|\endgroup\par
\begingroup\color{diffctx}\Verb| Assume the following \emph{geometric parameterization} holds on each interior face $F=Q\cap Q'$:|\endgroup\par
\begingroup\color{diffctx}\Verb| \begin{enumerate}|\endgroup\par
\begingroup\color{diffhunk}\Verb|@@ -3211,15 +3164,31 @@|\endgroup\par
\begingroup\color{diffctx}\Verb| |\endgroup\par
\begingroup\color{diffctx}\Verb| \smallskip\noindent|\endgroup\par
\begingroup\color{diffctx}\Verb| \textbf{Step 3 (small-angle model error).}|\endgroup\par
\begingroup\color{diffdel}\Verb|-When the sheets are only $\varepsilon$-graphs over their reference planes (hypothesis \textnormal{(a)}), the slice currents in the chart differ from the|\endgroup\par
\begingroup\color{diffdel}\Verb|-exactly-parallel translated model by a $C^1$ graph distortion of size $O(\varepsilon)$.|\endgroup\par
\begingroup\color{diffdel}\Verb!-Since $\|\eta\|_{\mathrm{comass}}\le 1$, the induced error in evaluating $\eta$ on each slice is bounded by $C\,\varepsilon\,h^{2n-2p}$!\endgroup\par
\begingroup\color{diffdel}\Verb|-uniformly (one factor of $h$ comes from converting the angular error into a tangential displacement on a cell of size $h$).|\endgroup\par
\begingroup\color{diffdel}\Verb|-Summing over the (integer-weighted) family on that face gives an additional error bounded by|\endgroup\par
\begingroup\color{diffdel}\Verb|-\(|\endgroup\par
\begingroup\color{diffdel}\Verb|-C\,\varepsilon\,h^{2n-2p}\,\Mass(\mu_{Q\to F}).|\endgroup\par
\begingroup\color{diffdel}\Verb|-\)|\endgroup\par
\begingroup\color{diffdel}\Verb|-Combining with Step 2 yields the stated face estimate|\endgroup\par
\begingroup\color{diffadd}\Verb|+Hypothesis \textnormal{(a)} implies that each actual slice current appearing in \textnormal{(b)} is obtained from the|\endgroup\par
\begingroup\color{diffadd}\Verb|+corresponding ``flat/parallel'' slice (the one used in Steps~1--2) by a $C^1$ graph perturbation over a cell of diameter $\asymp h$|\endgroup\par
\begingroup\color{diffadd}\Verb|+with slope $O(\varepsilon)$.|\endgroup\par
\begingroup\color{diffadd}\Verb|+In particular, after fixing the face chart, for each parameter $y$ there is a Lipschitz map $\Psi_y$ defined on a neighborhood of|\endgroup\par
\begingroup\color{diffadd}\Verb|+the flat slice such that|\endgroup\par
\begingroup\color{diffadd}\Verb|+\[|\endgroup\par
\begingroup\color{diffadd}\Verb|+\Sigma^{\mathrm{act}}_y\ =\ (\Psi_y)_\#\Sigma^{\mathrm{flat}}_y,|\endgroup\par
\begingroup\color{diffadd}\Verb|+\qquad|\endgroup\par
\begingroup\color{diffadd}\Verb!+\sup_{x\in\spt \Sigma^{\mathrm{flat}}_y}\|\Psi_y(x)-x\|\ \le\ C\,\varepsilon\,h,!\endgroup\par
\begingroup\color{diffadd}\Verb|+\qquad|\endgroup\par
\begingroup\color{diffadd}\Verb|+\Lip(\Psi_y)\ \le\ 1+C\,\varepsilon,|\endgroup\par
\begingroup\color{diffadd}\Verb|+\]|\endgroup\par
\begingroup\color{diffadd}\Verb|+where $C$ depends only on the fixed product chart constants.|\endgroup\par
\begingroup\color{diffadd}\Verb|+Applying Lemma~\ref{lem:flat-C0-deform} with $\phi_0=\Id$, $\phi_1=\Psi_y$ and $\delta\asymp\varepsilon h$ yields|\endgroup\par
\begingroup\color{diffadd}\Verb|+\[|\endgroup\par
\begingroup\color{diffadd}\Verb|+\mathcal F\!\bigl(\Sigma^{\mathrm{act}}_y-\Sigma^{\mathrm{flat}}_y\bigr)|\endgroup\par
\begingroup\color{diffadd}\Verb|+\ \le\ C\,\varepsilon\,h\Bigl(\Mass(\Sigma^{\mathrm{flat}}_y)+\Mass(\partial\Sigma^{\mathrm{flat}}_y)\Bigr).|\endgroup\par
\begingroup\color{diffadd}\Verb|+\]|\endgroup\par
\begingroup\color{diffadd}\Verb|+Summing this estimate over the (integer-weighted) family of slices meeting $F$ gives an additional contribution bounded by|\endgroup\par
\begingroup\color{diffadd}\Verb|+\[|\endgroup\par
\begingroup\color{diffadd}\Verb|+C\,\varepsilon\,h\sum_{\text{slices on }F}\Bigl(\Mass(\Sigma^{\mathrm{flat}}_y)+\Mass(\partial\Sigma^{\mathrm{flat}}_y)\Bigr)|\endgroup\par
\begingroup\color{diffadd}\Verb|+\ \le\ C\,\varepsilon\,h^{2n-2p}\,\Mass(\mu_{Q\to F}),|\endgroup\par
\begingroup\color{diffadd}\Verb|+\]|\endgroup\par
\begingroup\color{diffadd}\Verb|+where the last inequality uses that each flat slice has $(2n-2p-1)$--mass $\asymp h^{2n-2p-1}$ in the fixed chart.|\endgroup\par
\begingroup\color{diffadd}\Verb|+Combining with Step~2 yields the stated face estimate|\endgroup\par
\begingroup\color{diffctx}\Verb| \(|\endgroup\par
\begingroup\color{diffctx}\Verb! |B_F(\eta)|\le C h^{2n-2p-1}(\tau_F+\varepsilon\,\Mass(\mu_{Q\to F})\,h).!\endgroup\par
\begingroup\color{diffctx}\Verb| \)|\endgroup\par
\begingroup\color{diffhunk}\Verb|@@ -3555,19 +3524,11 @@|\endgroup\par
\begingroup\color{diffctx}\Verb| \begin{lemma}[Flat-norm stability under translation]\label{lem:flat-translate}|\endgroup\par
\begingroup\color{diffctx}\Verb| Let $S$ be an integral $\ell$--current in $\R^d$ with finite mass and finite boundary mass.|\endgroup\par
\begingroup\color{diffctx}\Verb| For any translation vector $v\in\R^d$, write $\tau_v(x):=x+v$ and $(\tau_v)_\#S$ for the pushforward.|\endgroup\par
\begingroup\color{diffdel}\Verb|-Then there exist integral currents $Q$ (of dimension $\ell+1$) and $R$ (of dimension $\ell$) such that|\endgroup\par
\begingroup\color{diffdel}\Verb|-\[|\endgroup\par
\begingroup\color{diffdel}\Verb|-(\tau_v)_\#S-S\ =\ R+\partial Q,|\endgroup\par
\begingroup\color{diffdel}\Verb|-\qquad|\endgroup\par
\begingroup\color{diffdel}\Verb!-\Mass(Q)\ \le\ \|v\|\,\Mass(S),!\endgroup\par
\begingroup\color{diffdel}\Verb|-\qquad|\endgroup\par
\begingroup\color{diffdel}\Verb!-\Mass(R)\ \le\ \|v\|\,\Mass(\partial S).!\endgroup\par
\begingroup\color{diffdel}\Verb|-\]|\endgroup\par
\begingroup\color{diffdel}\Verb|-Consequently|\endgroup\par
\begingroup\color{diffadd}\Verb|+Then|\endgroup\par
\begingroup\color{diffctx}\Verb| \[|\endgroup\par
\begingroup\color{diffctx}\Verb; \mathcal F\!\bigl((\tau_v)_\#S-S\bigr)\ \le\ \|v\|\Bigl(\Mass(S)+\Mass(\partial S)\Bigr).;\endgroup\par
\begingroup\color{diffctx}\Verb| \]|\endgroup\par
\begingroup\color{diffdel}\Verb|-In particular, if $S$ is a cycle ($\partial S=0$) one may take $R=0$ and this reduces to|\endgroup\par
\begingroup\color{diffadd}\Verb|+In particular, if $S$ is a cycle ($\partial S=0$) this reduces to|\endgroup\par
\begingroup\color{diffctx}\Verb! $\mathcal F((\tau_v)_\#S-S)\le \|v\|\,\Mass(S)$.!\endgroup\par
\begingroup\color{diffctx}\Verb| \end{lemma}|\endgroup\par
\begingroup\color{diffctx}\Verb| |\endgroup\par
\begingroup\color{diffhunk}\Verb|@@ -3601,6 +3562,43 @@|\endgroup\par
\begingroup\color{diffctx}\Verb| as claimed.|\endgroup\par
\begingroup\color{diffctx}\Verb| \end{proof}|\endgroup\par
\begingroup\color{diffctx}\Verb| |\endgroup\par
\begingroup\color{diffadd}\Verb|+\begin{lemma}[Flat-norm stability under small $C^0$ deformations]\label{lem:flat-C0-deform}|\endgroup\par
\begingroup\color{diffadd}\Verb|+Let $S$ be an integral $\ell$--current in $\R^d$ with finite mass and finite boundary mass.|\endgroup\par
\begingroup\color{diffadd}\Verb|+Let $\phi_0,\phi_1:\R^d\to\R^d$ be Lipschitz maps with|\endgroup\par
\begingroup\color{diffadd}\Verb|+\[|\endgroup\par
\begingroup\color{diffadd}\Verb!+\sup_{x\in\spt S}\|\phi_1(x)-\phi_0(x)\|\ \le\ \delta,!\endgroup\par
\begingroup\color{diffadd}\Verb|+\qquad|\endgroup\par
\begingroup\color{diffadd}\Verb|+\Lip(\phi_0)+\Lip(\phi_1)\ \le\ L.|\endgroup\par
\begingroup\color{diffadd}\Verb|+\]|\endgroup\par
\begingroup\color{diffadd}\Verb|+Then there exists a constant $C_\ell$ depending only on $\ell$ such that|\endgroup\par
\begingroup\color{diffadd}\Verb|+\[|\endgroup\par
\begingroup\color{diffadd}\Verb|+\mathcal F\!\bigl(\phi_{1\#}S-\phi_{0\#}S\bigr)\ \le\ C_\ell\,\delta\,L^{\ell}\Bigl(\Mass(S)+\Mass(\partial S)\Bigr).|\endgroup\par
\begingroup\color{diffadd}\Verb|+\]|\endgroup\par
\begingroup\color{diffadd}\Verb|+\end{lemma}|\endgroup\par
\begingroup\color{diffadd}\Verb|+|\endgroup\par
\begingroup\color{diffadd}\Verb|+\begin{proof}|\endgroup\par
\begingroup\color{diffadd}\Verb|+Consider the straight-line homotopy $H:[0,1]\times\R^d\to\R^d$ given by|\endgroup\par
\begingroup\color{diffadd}\Verb|+$H(t,x):=(1-t)\phi_0(x)+t\,\phi_1(x)$.|\endgroup\par
\begingroup\color{diffadd}\Verb|+Set $Q:=H_\#([0,1]\times S)$ and $R:=H_\#([0,1]\times \partial S)$.|\endgroup\par
\begingroup\color{diffadd}\Verb|+Since $\partial([0,1]\times S)=\{1\}\times S-\{0\}\times S-[0,1]\times\partial S$, the homotopy formula gives|\endgroup\par
\begingroup\color{diffadd}\Verb|+\[|\endgroup\par
\begingroup\color{diffadd}\Verb|+\phi_{1\#}S-\phi_{0\#}S\ =\ R+\partial Q.|\endgroup\par
\begingroup\color{diffadd}\Verb|+\]|\endgroup\par
\begingroup\color{diffadd}\Verb|+On $\spt S$, the differential of $H$ has one ``$t$--direction'' column $\,\partial_t H=\phi_1-\phi_0\,$ whose norm is $\le\delta$,|\endgroup\par
\begingroup\color{diffadd}\Verb|+and $\ell$ ``spatial'' columns bounded by $L$.|\endgroup\par
\begingroup\color{diffadd}\Verb|+Therefore the $(\ell+1)$--Jacobian of $H$ is bounded by $C_\ell\,\delta\,L^\ell$ on $\spt([0,1]\times S)$, and the $\ell$--Jacobian|\endgroup\par
\begingroup\color{diffadd}\Verb|+of $H$ restricted to $\spt([0,1]\times\partial S)$ is bounded by $C_\ell\,\delta\,L^{\ell-1}$.|\endgroup\par
\begingroup\color{diffadd}\Verb|+The standard mass estimate for pushforwards yields|\endgroup\par
\begingroup\color{diffadd}\Verb|+\[|\endgroup\par
\begingroup\color{diffadd}\Verb|+\Mass(Q)\ \le\ C_\ell\,\delta\,L^\ell\,\Mass(S),|\endgroup\par
\begingroup\color{diffadd}\Verb|+\qquad|\endgroup\par
\begingroup\color{diffadd}\Verb|+\Mass(R)\ \le\ C_\ell\,\delta\,L^\ell\,\Mass(\partial S),|\endgroup\par
\begingroup\color{diffadd}\Verb|+\]|\endgroup\par
\begingroup\color{diffadd}\Verb|+(after enlarging $C_\ell$ to absorb the $L^{\ell-1}$ factor).|\endgroup\par
\begingroup\color{diffadd}\Verb|+Taking these $R,Q$ in the definition of $\mathcal F$ gives the claim.|\endgroup\par
\begingroup\color{diffadd}\Verb|+\end{proof}|\endgroup\par
\begingroup\color{diffadd}\Verb|+|\endgroup\par
\begingroup\color{diffadd}\Verb|+|\endgroup\par
\begingroup\color{diffctx}\Verb| \end{editblock}|\endgroup\par
\begingroup\color{diffctx}\Verb| |\endgroup\par
\begingroup\color{diffctx}\Verb| \begin{editblock}|\endgroup\par
\begingroup\color{diffhunk}\Verb|@@ -3628,10 +3626,10 @@|\endgroup\par
\begingroup\color{diffctx}\Verb| \]|\endgroup\par
\begingroup\color{diffctx}\Verb| where, for each $F$, the integral is the corresponding integer-weighted sum over pieces meeting the interface.|\endgroup\par
\begingroup\color{diffctx}\Verb| |\endgroup\par
\begingroup\color{diffdel}\Verb|-If moreover $\Delta_F\le C\,h^2$ for all interfaces and each slice $\Sigma_F(u)$ arises as the interface boundary slice of a piece|\endgroup\par
\begingroup\color{diffadd}\Verb|+If moreover $\Delta_F\le C\,h^2$ for all interfaces and each slice $\Sigma_F(u_a)$ arises as the interface boundary slice of a piece|\endgroup\par
\begingroup\color{diffctx}\Verb| $Y^a\cap Q$ with interior mass $m_a:=\Mass([Y^a]\llcorner Q)$, then Lemma~\ref{lem:uniformly-convex-slice-boundary} gives|\endgroup\par
\begingroup\color{diffctx}\Verb| \[|\endgroup\par
\begingroup\color{diffdel}\Verb|-\Mass(\Sigma_F(u))\ \lesssim\ m_a^{\frac{k-1}{k}},|\endgroup\par
\begingroup\color{diffadd}\Verb|+\Mass(\Sigma_F(u_a))\ \lesssim\ m_a^{\frac{k-1}{k}},|\endgroup\par
\begingroup\color{diffctx}\Verb| \qquad k:=2n-2p,|\endgroup\par
\begingroup\color{diffctx}\Verb| \]|\endgroup\par
\begingroup\color{diffctx}\Verb| and hence, in the common situation where the slice currents on interfaces are cycles (so $\partial\Sigma_F(u)=0$), the global estimate|\endgroup\par
\begingroup\color{diffhunk}\Verb|@@ -3669,7 +3667,7 @@|\endgroup\par
\begingroup\color{diffctx}\Verb| and summing over $F$ yields the first bound.|\endgroup\par
\begingroup\color{diffctx}\Verb| |\endgroup\par
\begingroup\color{diffctx}\Verb| Under the additional assumptions $\Delta_F\le C\,h^2$ and|\endgroup\par
\begingroup\color{diffdel}\Verb|-$\Mass(\Sigma_F(u))+\Mass(\partial\Sigma_F(u))\lesssim m_a^{\frac{k-1}{k}}$ (with $k=2n-2p$),|\endgroup\par
\begingroup\color{diffadd}\Verb|+$\Mass(\Sigma_F(u_a))+\Mass(\partial\Sigma_F(u_a))\lesssim m_a^{\frac{k-1}{k}}$ (with $k=2n-2p$),|\endgroup\par
\begingroup\color{diffctx}\Verb| we obtain|\endgroup\par
\begingroup\color{diffctx}\Verb| \[|\endgroup\par
\begingroup\color{diffctx}\Verb| \mathcal F(B_F)\ \lesssim\ h^2\sum_{a\in\mathcal S(F)} m_{F,a}^{\frac{k-1}{k}}.|\endgroup\par
\begingroup\color{diffhunk}\Verb|@@ -3726,7 +3724,7 @@|\endgroup\par
\begingroup\color{diffctx}\Verb| \item[\textnormal{(c)}] \textbf{Packing:} each cell has at most $N_Q\lesssim \varepsilon^{-2p}$ disjoint pieces per direction family (Lemma~\ref{lem:sliver-packing});|\endgroup\par
\begingroup\color{diffctx}\Verb| \item[\textnormal{(d)}] \textbf{Mass scale:} the total mass per cell satisfies $M_Q:=\sum_{a\in\mathcal S(Q)}m_{Q,a}\asymp m\,h^{2n}$ (coming from the smooth form $m\beta$).|\endgroup\par
\begingroup\color{diffctx}\Verb| \end{enumerate}|\endgroup\par
\begingroup\color{diffdel}\Verb|-Then the weighted face estimate (Corollary~\ref{cor:global-flat-weighted}) and the Hölder/packing bound of Remark~\ref{rem:weighted-scaling} give|\endgroup\par
\begingroup\color{diffadd}\Verb|+Then the weighted face estimate (Corollary~\ref{cor:global-flat-weighted}) and the H\"older/packing bound of Remark~\ref{rem:weighted-scaling} give|\endgroup\par
\begingroup\color{diffctx}\Verb| \[|\endgroup\par
\begingroup\color{diffctx}\Verb| \mathcal F(\partial T^{\mathrm{raw}})\ \lesssim\ m^{\frac{k-1}{k}}\,h^{\,2-\frac{2n}{k}}\;\varepsilon^{-\frac{2p}{k}}\ +\ O(\varepsilon\,m).|\endgroup\par
\begingroup\color{diffctx}\Verb| \]|\endgroup\par
\begingroup\color{diffhunk}\Verb|@@ -3855,7 +3853,7 @@|\endgroup\par
\begingroup\color{diffctx}\Verb| Then, after pairing atoms by the identity pairing $y_a\leftrightarrow y_a$, the mismatch current $B_F$ satisfies|\endgroup\par
\begingroup\color{diffctx}\Verb| \[|\endgroup\par
\begingroup\color{diffctx}\Verb| \mathcal F(B_F)\ \le\ C\,h^2\,\Biggl(\sum_{a=1}^{N_F} w_a\Bigl(\Mass(\Sigma_{\Phi_{Q,F}y_a})+\Mass(\partial\Sigma_{\Phi_{Q,F}y_a})\Bigr)|\endgroup\par
\begingroup\color{diffdel}\Verb|-\;+\sum_{a=1}^{N_F} w_a\Bigl(\Mass(\Sigma_{\Phi_{Q',F}y_a})+\Mass(\partial\Sigma_{\Phi_{Q',F}y_a})\Bigr)\Biggr)\ +\ C\,\varepsilon\,M_F,|\endgroup\par
\begingroup\color{diffadd}\Verb|+\;+\sum_{a=1}^{N_F} w_a\Bigl(\Mass(\Sigma_{\Phi_{Q',F}y_a})+\Mass(\partial\Sigma_{\Phi_{Q',F}y_a})\Bigr)\Biggr)\ +\ C_{\angle}\,\varepsilon\,M_F,|\endgroup\par
\begingroup\color{diffctx}\Verb| \]|\endgroup\par
\begingroup\color{diffctx}\Verb| where $M_F$ denotes the total $(2n-2p)$-mass of pieces meeting the interface (so $M_F\lesssim M_Q+M_{Q'}$) and|\endgroup\par
\begingroup\color{diffctx}\Verb| $\varepsilon$ is the small-angle/graph parameter from Proposition~\ref{prop:transport-flat-glue}\textnormal{(a)}.|\endgroup\par
\begingroup\color{diffhunk}\Verb|@@ -3889,8 +3887,16 @@|\endgroup\par
\begingroup\color{diffctx}\Verb| \]|\endgroup\par
\begingroup\color{diffctx}\Verb| The same bound holds with $Q$ and $Q'$ swapped; combining yields the symmetric form stated.|\endgroup\par
\begingroup\color{diffctx}\Verb| |\endgroup\par
\begingroup\color{diffdel}\Verb|-For $\varepsilon>0$, compare each sheet to the corresponding flat slice in the tubular chart; the $C^1$ graph distortion contributes an|\endgroup\par
\begingroup\color{diffdel}\Verb|-additional $C\,\varepsilon\,M_F$ term exactly as in Proposition~\ref{prop:transport-flat-glue} (after enlarging $C$).|\endgroup\par
\begingroup\color{diffadd}\Verb|+For $\varepsilon>0$, write each actual boundary slice on $F$ as a pushforward of its flat/parallel model slice|\endgroup\par
\begingroup\color{diffadd}\Verb|+by a Lipschitz graph map in the tubular chart.|\endgroup\par
\begingroup\color{diffadd}\Verb|+Hypothesis \textnormal{(a)} gives a uniform displacement bound $\delta\lesssim\varepsilon h$ and a uniform Lipschitz bound.|\endgroup\par
\begingroup\color{diffadd}\Verb|+Applying Lemma~\ref{lem:flat-C0-deform} to each slice yields a flat-norm error of size|\endgroup\par
\begingroup\color{diffadd}\Verb|+\[|\endgroup\par
\begingroup\color{diffadd}\Verb|+\mathcal F\!\bigl(\Sigma^{\mathrm{act}}_y-\Sigma^{\mathrm{flat}}_y\bigr)\ \le\ C\,\varepsilon\,h\Bigl(\Mass(\Sigma^{\mathrm{flat}}_y)+\Mass(\partial\Sigma^{\mathrm{flat}}_y)\Bigr).|\endgroup\par
\begingroup\color{diffadd}\Verb|+\]|\endgroup\par
\begingroup\color{diffadd}\Verb|+Summing over the integer-weighted family of slices meeting $F$ and using that the total $(2n-2p)$--mass of pieces meeting $F$|\endgroup\par
\begingroup\color{diffadd}\Verb|+controls the sum of slice masses at scale $h$ gives the additional term|\endgroup\par
\begingroup\color{diffadd}\Verb|+$C_{\angle}\,\varepsilon\,M_F$ in the statement.|\endgroup\par
\begingroup\color{diffctx}\Verb| \end{proof}|\endgroup\par
\begingroup\color{diffctx}\Verb| |\endgroup\par
\begingroup\color{diffctx}\Verb| |\endgroup\par
\begingroup\color{diffhunk}\Verb|@@ -3910,7 +3916,7 @@|\endgroup\par
\begingroup\color{diffctx}\Verb| coming from the unmatched part (so $B_F=B_F^{\wedge}+B_F^{\mathrm{un}}$).|\endgroup\par
\begingroup\color{diffctx}\Verb| Then|\endgroup\par
\begingroup\color{diffctx}\Verb| \[|\endgroup\par
\begingroup\color{diffdel}\Verb|-\mathcal F(B_F^{\wedge})\ \le\ C\,h^2\Bigl(\Mass(\partial S_Q\llcorner F)+\Mass(\partial S_{Q'}\llcorner F)\Bigr)\ +\ O(\varepsilon\,M_F),|\endgroup\par
\begingroup\color{diffadd}\Verb|+\mathcal F(B_F^{\wedge})\ \le\ C\,h^2\Bigl(\Mass(\partial S_Q\llcorner F)+\Mass(\partial S_{Q'}\llcorner F)\Bigr)\ +\ C_{\angle}\,\varepsilon\,M_F,|\endgroup\par
\begingroup\color{diffctx}\Verb| \]|\endgroup\par
\begingroup\color{diffctx}\Verb| and, moreover,|\endgroup\par
\begingroup\color{diffctx}\Verb| \[|\endgroup\par
\begingroup\color{diffhunk}\Verb|@@ -3923,8 +3929,7 @@|\endgroup\par
\begingroup\color{diffctx}\Verb| The matched part $B_F^{\wedge}$ is obtained by applying the two face maps to the \emph{same} common submeasure $\nu^{\wedge}$.|\endgroup\par
\begingroup\color{diffctx}\Verb| Therefore Lemma~\ref{lem:template-displacement} applies directly and yields the stated bound for $B_F^{\wedge}$.|\endgroup\par
\begingroup\color{diffctx}\Verb| |\endgroup\par
\begingroup\color{diffdel}\Verb|-For the unmatched part, $B_F^{\mathrm{un}}$ is an integral $(k-1)$--cycle supported on the (relative) interior of the face patch $F$|\endgroup\par
\begingroup\color{diffdel}\Verb|-(any possible edge contributions are treated separately in the global bookkeeping/corner-exit package).|\endgroup\par
\begingroup\color{diffadd}\Verb|+For the unmatched part, $B_F^{\mathrm{un}}$ is an integral $(k-1)$--cycle supported on the face patch $F$.|\endgroup\par
\begingroup\color{diffctx}\Verb| Since $\mathrm{diam}(F)\lesssim h$, Lemma~\ref{lem:flat-diameter} gives|\endgroup\par
\begingroup\color{diffctx}\Verb| \[|\endgroup\par
\begingroup\color{diffctx}\Verb| \mathcal F(B_F^{\mathrm{un}})\ \le\ C\,h\,\Mass(B_F^{\mathrm{un}}).|\endgroup\par
\begingroup\color{diffhunk}\Verb|@@ -3945,14 +3950,14 @@|\endgroup\par
\begingroup\color{diffctx}\Verb| for some $\theta_F\in[0,1]$.|\endgroup\par
\begingroup\color{diffctx}\Verb| Then|\endgroup\par
\begingroup\color{diffctx}\Verb| \[|\endgroup\par
\begingroup\color{diffdel}\Verb|-\mathcal F(B_F)\ \le\ C\,h^2\Bigl(\Mass(\partial S_Q\llcorner F)+\Mass(\partial S_{Q'}\llcorner F)\Bigr)\ +\ C\,h\,\theta_F\Bigl(\Mass(\partial S_Q\llcorner F)+\Mass(\partial S_{Q'}\llcorner F)\Bigr)\ +\ O(\varepsilon\,M_F).|\endgroup\par
\begingroup\color{diffadd}\Verb|+\mathcal F(B_F)\ \le\ C\,h^2\Bigl(\Mass(\partial S_Q\llcorner F)+\Mass(\partial S_{Q'}\llcorner F)\Bigr)\ +\ C\,h\,\theta_F\Bigl(\Mass(\partial S_Q\llcorner F)+\Mass(\partial S_{Q'}\llcorner F)\Bigr)\ +\ C_{\angle}\,\varepsilon\,M_F.|\endgroup\par
\begingroup\color{diffctx}\Verb| \]|\endgroup\par
\begingroup\color{diffctx}\Verb| In particular, if $\theta_F\lesssim h$ then the unmatched contribution is of the same $h^2\times(\text{boundary mass})$ order as the matched displacement term.|\endgroup\par
\begingroup\color{diffctx}\Verb| \end{lemma}|\endgroup\par
\begingroup\color{diffctx}\Verb| |\endgroup\par
\begingroup\color{diffctx}\Verb| \begin{proof}|\endgroup\par
\begingroup\color{diffctx}\Verb| Decompose $B_F=B_F^{\wedge}+B_F^{\mathrm{un}}$ as in Lemma~\ref{lem:template-displacement-edits}.|\endgroup\par
\begingroup\color{diffdel}\Verb|-Lemma~\ref{lem:template-displacement-edits} gives the $h^2$--scale bound for $\mathcal F(B_F^{\wedge})$ (plus the $O(\varepsilon\,M_F)$ term), and also gives|\endgroup\par
\begingroup\color{diffadd}\Verb|+Lemma~\ref{lem:template-displacement-edits} gives the $h^2$--scale bound for $\mathcal F(B_F^{\wedge})$ (plus the $C_{\angle}\,\varepsilon\,M_F$ term), and also gives|\endgroup\par
\begingroup\color{diffctx}\Verb| \(|\endgroup\par
\begingroup\color{diffctx}\Verb| \mathcal F(B_F^{\mathrm{un}})\le C h\,\Mass(B_F^{\mathrm{un}}).|\endgroup\par
\begingroup\color{diffctx}\Verb| \)|\endgroup\par
\begingroup\color{diffhunk}\Verb|@@ -4003,7 +4008,7 @@|\endgroup\par
\begingroup\color{diffctx}\Verb| \]|\endgroup\par
\begingroup\color{diffctx}\Verb| then|\endgroup\par
\begingroup\color{diffctx}\Verb| \[|\endgroup\par
\begingroup\color{diffdel}\Verb|-\mathcal F(B_F)\ \le\ C\,h^2\Bigl(\Mass(\partial S_Q\llcorner F)+\Mass(\partial S_{Q'}\llcorner F)\Bigr)\ +\ O(\varepsilon\,M_F),|\endgroup\par
\begingroup\color{diffadd}\Verb|+\mathcal F(B_F)\ \le\ C\,h^2\Bigl(\Mass(\partial S_Q\llcorner F)+\Mass(\partial S_{Q'}\llcorner F)\Bigr)\ +\ C_{\angle}\,\varepsilon\,M_F,|\endgroup\par
\begingroup\color{diffctx}\Verb| \]|\endgroup\par
\begingroup\color{diffctx}\Verb| with $C$ depending only on $(n,p,X)$ and the uniform tubular-face charts.|\endgroup\par
\begingroup\color{diffctx}\Verb| \end{proposition}|\endgroup\par
\begingroup\color{diffhunk}\Verb|@@ -4055,7 +4060,7 @@|\endgroup\par
\begingroup\color{diffctx}\Verb| For each interior interface $F=Q\cap Q'$, Proposition~\ref{prop:prefix-template-coherence} provides a bound of the form|\endgroup\par
\begingroup\color{diffctx}\Verb| \[|\endgroup\par
\begingroup\color{diffctx}\Verb| \mathcal F(B_F)|\endgroup\par
\begingroup\color{diffdel}\Verb|-\ \le\ C\,h^2\Bigl(\Mass(\partial S_Q\llcorner F)+\Mass(\partial S_{Q'}\llcorner F)\Bigr)\ +\ O(\varepsilon\,M_F),|\endgroup\par
\begingroup\color{diffadd}\Verb|+\ \le\ C\,h^2\Bigl(\Mass(\partial S_Q\llcorner F)+\Mass(\partial S_{Q'}\llcorner F)\Bigr)\ +\ C_{\angle}\,\varepsilon\,M_F,|\endgroup\par
\begingroup\color{diffctx}\Verb| \]|\endgroup\par
\begingroup\color{diffctx}\Verb| where $M_F$ is the total interior mass of pieces meeting $F$.|\endgroup\par
\begingroup\color{diffctx}\Verb| Summing over all interior faces and using subadditivity of $\mathcal F$ gives|\endgroup\par
\begingroup\color{diffhunk}\Verb|@@ -4990,18 +4995,6 @@|\endgroup\par
\begingroup\color{diffctx}\Verb| $\mathcal F(\partial T^{\mathrm{raw}})=o(m)$ in the scaling regime of Remark~\ref{rem:weighted-scaling}.|\endgroup\par
\begingroup\color{diffctx}\Verb| \end{proof}|\endgroup\par
\begingroup\color{diffctx}\Verb| |\endgroup\par
\begingroup\color{diffdel}\Verb|-\begin{remark}[External inputs for integer rounding]\label{rem:integer-rounding-external}|\endgroup\par
\begingroup\color{diffdel}\Verb|-\textbf{This proposition relies on external inputs from discrete optimization.}  Steps 2 and 4 use integer rounding lemmas whose proofs invoke:|\endgroup\par
\begingroup\color{diffdel}\Verb|-\begin{itemize}|\endgroup\par
\begingroup\color{diffdel}\Verb|-\item the Barvinok--Bar\'any--Grinberg discrepancy bounds for integer approximation in fixed-dimensional polytopes (Lemma~\ref{lem:barany-grinberg});|\endgroup\par
\begingroup\color{diffdel}\Verb|-\item the observation that the constraint dimension $b=\mathrm{rank}\,H^{2n-2p}(X,\Z)$ is fixed (independent of mesh refinement), so that|\endgroup\par
\begingroup\color{diffdel}\Verb|-  discrepancy bounds do not blow up.|\endgroup\par
\begingroup\color{diffdel}\Verb|-\end{itemize}|\endgroup\par
\begingroup\color{diffdel}\Verb|-Reference: Barvinok, \emph{Integer Programming} \cite{Barvinok-IntProg}.|\endgroup\par
\begingroup\color{diffdel}\Verb|-|\endgroup\par
\begingroup\color{diffdel}\Verb|-\smallskip\noindent|\endgroup\par
\begingroup\color{diffdel}\Verb|-\textbf{Adversarial concern:} The claim that global period-fixing does not break the local slow-variation bounds depends on the bounded-correction absorption mechanism (Remark~\ref{rem:bounded-corrections}).  Any audit should verify that the correction vectors have uniformly bounded entries and that this bound is independent of mesh refinement.|\endgroup\par
\begingroup\color{diffdel}\Verb|-\end{remark}|\endgroup\par
\begingroup\color{diffctx}\Verb| |\endgroup\par
\begingroup\color{diffctx}\Verb| \begin{remark}[Making the ``prefix-balanced face population'' explicit]|\endgroup\par
\begingroup\color{diffctx}\Verb| The previous proposition treats each vertex template separately.|\endgroup\par
\begingroup\color{diffhunk}\Verb|@@ -5696,23 +5689,9 @@|\endgroup\par
\begingroup\color{diffctx}\Verb| (Theorem~\ref{thm:sliver-mass-matching-on-template} and Corollary~\ref{cor:global-flat-weighted}), one obtains the quantitative estimate|\endgroup\par
\begingroup\color{diffctx}\Verb| \[|\endgroup\par
\begingroup\color{diffctx}\Verb| \mathcal F\!\left(\partial T^{\mathrm{raw}}\right)\ \le\ \varepsilon_{\mathrm{glue}}(m,\delta,\varepsilon,\mathrm{mesh})\cdot m,|\endgroup\par
\begingroup\color{diffdel}\Verb|-\]|\endgroup\par
\begingroup\color{diffdel}\Verb|-\noindent where $\varepsilon_{\mathrm{glue}}\to 0$ under the global parameter schedule of \S\ref{sec:parameter-schedule}.|\endgroup\par
\begingroup\color{diffdel}\Verb|-A concrete sufficient regime (with explicit scale relations between $\varepsilon$ and $\mathrm{mesh}$ in the range $p<\frac{n}{2}$, and the|\endgroup\par
\begingroup\color{diffdel}\Verb|-borderline replacement at $p=\frac{n}{2}$) is recorded in Lemma~\ref{lem:flatnorm-o-m}.|\endgroup\par
\begingroup\color{diffdel}\Verb|-By definition of $\mathcal F$ there exist integral currents|\endgroup\par
\begingroup\color{diffdel}\Verb|-$R$ and $Q$ with $\partial T^{\mathrm{raw}}=R+\partial Q$ and $\Mass(R)+\Mass(Q)\le 2\mathcal F(\partial T^{\mathrm{raw}})$.|\endgroup\par
\begingroup\color{diffdel}\Verb|-Moreover $R=\partial(T^{\mathrm{raw}}-Q)$ is itself a boundary (hence null-homologous); by the Federer--Fleming|\endgroup\par
\begingroup\color{diffdel}\Verb|-isoperimetric inequality there exists an integral filling $Q_R$ with $\partial Q_R=R$ and|\endgroup\par
\begingroup\color{diffdel}\Verb|-\[|\endgroup\par
\begingroup\color{diffdel}\Verb|-\Mass(Q_R)\le C\,\Mass(R)^{\frac{2n-2p}{2n-2p-1}}.|\endgroup\par
\begingroup\color{diffdel}\Verb|-\]|\endgroup\par
\begingroup\color{diffdel}\Verb|-Setting|\endgroup\par
\begingroup\color{diffdel}\Verb|-\[|\endgroup\par
\begingroup\color{diffdel}\Verb|-R_{\mathrm{glue}}:=-(Q+Q_R)|\endgroup\par
\begingroup\color{diffdel}\Verb|-\]|\endgroup\par
\begingroup\color{diffdel}\Verb|-gives $\partial R_{\mathrm{glue}}=-\partial T^{\mathrm{raw}}$ and $\Mass(R_{\mathrm{glue}})$ as small as desired once|\endgroup\par
\begingroup\color{diffdel}\Verb|-$\mathcal F(\partial T^{\mathrm{raw}})$ is small.|\endgroup\par
\begingroup\color{diffadd}\Verb|+\qquad \varepsilon_{\mathrm{glue}}\xrightarrow[\delta,\varepsilon\to 0,\ \mathrm{mesh}\to 0,\ m\to\infty]{}0.|\endgroup\par
\begingroup\color{diffadd}\Verb|+\]|\endgroup\par
\begingroup\color{diffadd}\Verb|+\noindent A concrete sufficient parameter/scaling regime yielding $\mathcal F(\partial T^{\mathrm{raw}})=o(m)$ is recorded in Lemma~\ref{lem:flatnorm-o-m}.|\endgroup\par
\begingroup\color{diffctx}\Verb| |\endgroup\par
\begingroup\color{diffctx}\Verb| \begin{lemma}[Federer--Fleming filling on $X$ for small cycles]\label{lem:FF-filling-X}|\endgroup\par
\begingroup\color{diffctx}\Verb| Let $X$ be the fixed compact Riemannian manifold in the projective setting of the paper, and fix $k\ge 2$.|\endgroup\par
\begingroup\color{diffhunk}\Verb|@@ -5731,7 +5710,7 @@|\endgroup\par
\begingroup\color{diffctx}\Verb| \begin{proof}|\endgroup\par
\begingroup\color{diffctx}\Verb| Choose a finite atlas of $X$ by coordinate charts with uniformly controlled bi-Lipschitz constants at the scale of injectivity radius.|\endgroup\par
\begingroup\color{diffctx}\Verb| For $\Mass(R)$ sufficiently small, the support of $R$ is contained in a single chart (after decomposing $R$ into finitely many pieces if needed),|\endgroup\par
\begingroup\color{diffdel}\Verb|-so the Euclidean Federer--Fleming isoperimetric inequality in $\R^N$ applies to the chart image.|\endgroup\par
\begingroup\color{diffadd}\Verb|+so the Euclidean Federer--Fleming isoperimetric inequality in $\R^N$ applies to the chart image. |\endgroup\par
\begingroup\color{diffctx}\Verb| Pushing the resulting filling forward to $X$ and absorbing the chart distortion constants yields the stated bound with $C_X$ and $\delta_X$|\endgroup\par
\begingroup\color{diffctx}\Verb| depending only on $(X,g)$ and $k$.|\endgroup\par
\begingroup\color{diffctx}\Verb| A detailed proof in the Riemannian setting can be found in standard GMT references (e.g.\ \cite{FF60,Fed69,Sim83}).|\endgroup\par
\begingroup\color{diffhunk}\Verb|@@ -5751,35 +5730,41 @@|\endgroup\par
\begingroup\color{diffctx}\Verb| In particular, $T^{\mathrm{raw}}+R_{\mathrm{glue}}$ is a closed integral cycle.|\endgroup\par
\begingroup\color{diffctx}\Verb| \end{proposition}|\endgroup\par
\begingroup\color{diffctx}\Verb| \begin{proof}|\endgroup\par
\begingroup\color{diffadd}\Verb|+Fix $\kappa>0$.|\endgroup\par
\begingroup\color{diffctx}\Verb| Let $\delta:=\mathcal F(\partial T^{\mathrm{raw}})$.|\endgroup\par
\begingroup\color{diffdel}\Verb|-Choose $R,Q$ in the definition of $\mathcal F$ with|\endgroup\par
\begingroup\color{diffdel}\Verb|-\[|\endgroup\par
\begingroup\color{diffdel}\Verb|-\partial T^{\mathrm{raw}}=R+\partial Q,|\endgroup\par
\begingroup\color{diffadd}\Verb|+By definition of $\mathcal F$, choose integral currents $R$ and $Q$ in $X$ such that|\endgroup\par
\begingroup\color{diffadd}\Verb|+\[|\endgroup\par
\begingroup\color{diffadd}\Verb|+\partial T^{\mathrm{raw}}\ =\ R+\partial Q,|\endgroup\par
\begingroup\color{diffctx}\Verb| \qquad|\endgroup\par
\begingroup\color{diffdel}\Verb|-\Mass(R)+\Mass(Q)\le 2\delta.|\endgroup\par
\begingroup\color{diffadd}\Verb|+\Mass(R)+\Mass(Q)\ \le\ 2\delta.|\endgroup\par
\begingroup\color{diffctx}\Verb| \]|\endgroup\par
\begingroup\color{diffctx}\Verb| Since $\partial(\partial T^{\mathrm{raw}})=0$, we have $\partial R=0$.|\endgroup\par
\begingroup\color{diffctx}\Verb| Moreover $R$ is itself a boundary in $X$ because|\endgroup\par
\begingroup\color{diffctx}\Verb| \[|\endgroup\par
\begingroup\color{diffdel}\Verb|-R=\partial T^{\mathrm{raw}}-\partial Q=\partial\!\bigl(T^{\mathrm{raw}}-Q\bigr).|\endgroup\par
\begingroup\color{diffadd}\Verb|+R\ =\ \partial T^{\mathrm{raw}}-\partial Q\ =\ \partial\!\bigl(T^{\mathrm{raw}}-Q\bigr).|\endgroup\par
\begingroup\color{diffctx}\Verb| \]|\endgroup\par
\begingroup\color{diffctx}\Verb| Let $k:=2n-2p$ (the dimension of $T^{\mathrm{raw}}$).|\endgroup\par
\begingroup\color{diffdel}\Verb|-For $\delta$ sufficiently small we have $\Mass(R)\le 2\delta\le \delta_X$ from Lemma~\ref{lem:FF-filling-X}, hence there exists an integral|\endgroup\par
\begingroup\color{diffdel}\Verb|-$k$--current $Q_R$ with $\partial Q_R=R$ and|\endgroup\par
\begingroup\color{diffdel}\Verb|-\[|\endgroup\par
\begingroup\color{diffdel}\Verb|-\Mass(Q_R)\ \le\ C_X\,\Mass(R)^{\frac{k}{k-1}}\ \le\ C_X\,(2\delta)^{\frac{k}{k-1}}.|\endgroup\par
\begingroup\color{diffadd}\Verb|+Choose the microstructure parameters so that $\delta\le \delta_X$ from Lemma~\ref{lem:FF-filling-X}.|\endgroup\par
\begingroup\color{diffadd}\Verb|+Applying Lemma~\ref{lem:FF-filling-X} to the $(k-1)$--cycle $R$ produces an integral $k$--current $Q_R$ with|\endgroup\par
\begingroup\color{diffadd}\Verb|+$\partial Q_R=R$ and|\endgroup\par
\begingroup\color{diffadd}\Verb|+\[|\endgroup\par
\begingroup\color{diffadd}\Verb|+\Mass(Q_R)\ \le\ C_X\,\Mass(R)^{\frac{k}{k-1}}|\endgroup\par
\begingroup\color{diffadd}\Verb|+\ \le\ C_X\,(2\delta)^{\frac{k}{k-1}}.|\endgroup\par
\begingroup\color{diffctx}\Verb| \]|\endgroup\par
\begingroup\color{diffctx}\Verb| Define|\endgroup\par
\begingroup\color{diffctx}\Verb| \[|\endgroup\par
\begingroup\color{diffdel}\Verb|-R_{\mathrm{glue}}:=-(Q+Q_R).|\endgroup\par
\begingroup\color{diffdel}\Verb|-\]|\endgroup\par
\begingroup\color{diffdel}\Verb|-Then $\partial R_{\mathrm{glue}}=-\partial T^{\mathrm{raw}}$ and|\endgroup\par
\begingroup\color{diffdel}\Verb|-\[|\endgroup\par
\begingroup\color{diffdel}\Verb|-\Mass(R_{\mathrm{glue}})\ \le\ \Mass(Q)+\Mass(Q_R)\ \le\ 2\delta+C_X\,(2\delta)^{\frac{k}{k-1}}|\endgroup\par
\begingroup\color{diffdel}\Verb|-\ \xrightarrow[\delta\to 0]{}\ 0,|\endgroup\par
\begingroup\color{diffdel}\Verb|-\]|\endgroup\par
\begingroup\color{diffdel}\Verb|-as claimed.|\endgroup\par
\begingroup\color{diffdel}\Verb|-\end{proof}|\endgroup\par
\begingroup\color{diffadd}\Verb|+R_{\mathrm{glue}}\ :=\ -(Q+Q_R).|\endgroup\par
\begingroup\color{diffadd}\Verb|+\]|\endgroup\par
\begingroup\color{diffadd}\Verb|+Then $\partial R_{\mathrm{glue}}=-\partial T^{\mathrm{raw}}$, hence $T^{\mathrm{raw}}+R_{\mathrm{glue}}$ is a cycle in $X$, and|\endgroup\par
\begingroup\color{diffadd}\Verb|+\[|\endgroup\par
\begingroup\color{diffadd}\Verb|+\Mass(R_{\mathrm{glue}})\ \le\ \Mass(Q)+\Mass(Q_R)|\endgroup\par
\begingroup\color{diffadd}\Verb|+\ \le\ 2\delta + C_X\,(2\delta)^{\frac{k}{k-1}}.|\endgroup\par
\begingroup\color{diffadd}\Verb|+\]|\endgroup\par
\begingroup\color{diffadd}\Verb|+Finally, the quantitative estimate preceding the proposition gives $\delta\le \varepsilon_{\mathrm{glue}}(\cdots)\,m$ with|\endgroup\par
\begingroup\color{diffadd}\Verb|+$\varepsilon_{\mathrm{glue}}\to 0$ in the indicated regime.|\endgroup\par
\begingroup\color{diffadd}\Verb|+Thus, for parameters chosen so that $\delta$ is sufficiently small, the right-hand side is $<\kappa$, proving the claim.|\endgroup\par
\begingroup\color{diffadd}\Verb|+\end{proof}|\endgroup\par
\begingroup\color{diffadd}\Verb|+|\endgroup\par
\begingroup\color{diffctx}\Verb| |\endgroup\par
\begingroup\color{diffctx}\Verb| |\endgroup\par
\begingroup\color{diffctx}\Verb| We now return to the global construction.|\endgroup\par
\begingroup\color{diffhunk}\Verb|@@ -5877,7 +5862,7 @@|\endgroup\par
\begingroup\color{diffctx}\Verb! \Bigl|\int_{T^{(1)}}\eta_\ell - m\,I_\ell\Bigr|<1,!\endgroup\par
\begingroup\color{diffctx}\Verb| \qquad T^{(1)}=T^{\mathrm{raw}}+R_{\mathrm{glue}}.|\endgroup\par
\begingroup\color{diffctx}\Verb| \]|\endgroup\par
\begingroup\color{diffdel}\Verb|-Since $\int_{T^{(1)}}\eta_\ell\in\Z$ (Lemma~\ref{lem:integral-periods}),|\endgroup\par
\begingroup\color{diffadd}\Verb|+Since $\int_{T^{(1)}}\eta_\ell\in\Z$ (integral current against an integral class),|\endgroup\par
\begingroup\color{diffctx}\Verb| we conclude $\int_{T^{(1)}}\eta_\ell = m\,I_\ell$ for all $\ell$.|\endgroup\par
\begingroup\color{diffctx}\Verb| Hence|\endgroup\par
\begingroup\color{diffctx}\Verb| \[|\endgroup\par
\begingroup\color{diffhunk}\Verb|@@ -5918,12 +5903,8 @@|\endgroup\par
\begingroup\color{diffctx}\Verb| \end{lemma}|\endgroup\par
\begingroup\color{diffctx}\Verb| |\endgroup\par
\begingroup\color{diffctx}\Verb| \begin{proof}|\endgroup\par
\begingroup\color{diffdel}\Verb|-An integral cycle $T$ determines a class $[T]\in H_k(X,\Z)$ (see Federer, \emph{Geometric Measure Theory}, 1969, \S4.1).|\endgroup\par
\begingroup\color{diffdel}\Verb|-If $[\eta]\in H^k(X,\Z)$ is an integral cohomology class, then the de~Rham pairing gives|\endgroup\par
\begingroup\color{diffdel}\Verb|-\[|\endgroup\par
\begingroup\color{diffdel}\Verb|-\int_T \eta \;=\; \langle [T],[\eta]\rangle \in \Z,|\endgroup\par
\begingroup\color{diffdel}\Verb|-\]|\endgroup\par
\begingroup\color{diffdel}\Verb|-since $H^k(X,\Z)$ pairs integrally with $H_k(X,\Z)$ (universal coefficient theorem / de~Rham theorem).|\endgroup\par
\begingroup\color{diffadd}\Verb|+By definition of integral homology and the de Rham isomorphism, the period of $T$ on any integral cohomology class is an integer.|\endgroup\par
\begingroup\color{diffadd}\Verb|+Explicitly, if $T$ represents an element of $H_k(X,\Z)$ and $[\eta]\in H^k(X,\Z)$, then $\langle [T],[\eta]\rangle\in\Z$ by the universal coefficient theorem.|\endgroup\par
\begingroup\color{diffctx}\Verb| \end{proof}|\endgroup\par
\begingroup\color{diffctx}\Verb| |\endgroup\par
\begingroup\color{diffctx}\Verb| \begin{lemma}[Lattice discreteness]\label{lem:lattice-discreteness}|\endgroup\par
\begingroup\color{diffhunk}\Verb|@@ -6116,9 +6097,7 @@|\endgroup\par
\begingroup\color{diffctx}\Verb| \end{proposition}|\endgroup\par
\begingroup\color{diffctx}\Verb| |\endgroup\par
\begingroup\color{diffctx}\Verb| \begin{proof}|\endgroup\par
\begingroup\color{diffdel}\Verb|-By construction, each local sheet current $S_Q$ is holomorphic and hence $\psi$--calibrated, and the sheet pieces are chosen disjointly on each cell $Q$|\endgroup\par
\begingroup\color{diffdel}\Verb|-(cf.\ the disjointness requirements in the local manufacturing step).|\endgroup\par
\begingroup\color{diffdel}\Verb|-Therefore the sum $S=\sum_Q S_Q$ is $\psi$--calibrated and evaluation/mass add without cancellation.|\endgroup\par
\begingroup\color{diffadd}\Verb|+By construction, each local sheet current $S_Q$ is holomorphic and hence $\psi$--calibrated, so their sum $S$ is $\psi$--calibrated.|\endgroup\par
\begingroup\color{diffctx}\Verb| In particular,|\endgroup\par
\begingroup\color{diffctx}\Verb| \[|\endgroup\par
\begingroup\color{diffctx}\Verb| \Mass(S)=\int_S\psi.|\endgroup\par
\begingroup\color{diffhunk}\Verb|@@ -6201,21 +6180,17 @@|\endgroup\par
\begingroup\color{diffctx}\Verb| This is the compactness/normalization needed for Federer--Fleming.|\endgroup\par
\begingroup\color{diffctx}\Verb| |\endgroup\par
\begingroup\color{diffctx}\Verb| \medskip\noindent|\endgroup\par
\begingroup\color{diffdel}\Verb|-\textbf{Substep 6.2: Compactness (Federer--Fleming + Allard).}|\endgroup\par
\begingroup\color{diffadd}\Verb|+\textbf{Substep 6.2: Varifold compactness \cite{Allard72,Sim83}.}|\endgroup\par
\begingroup\color{diffctx}\Verb| Let $V_k$ be the associated integral varifold of $T_k$.  Uniform mass|\endgroup\par
\begingroup\color{diffdel}\Verb|-bound gives tightness.|\endgroup\par
\begingroup\color{diffdel}\Verb|-Since $\partial T_k=0$ and $\sup_k \Mass(T_k)<\infty$, the Federer--Fleming compactness theorem for integral currents|\endgroup\par
\begingroup\color{diffdel}\Verb|-(Federer--Fleming, \emph{Normal and integral currents}, Ann.~of Math.~72 (1960), 458--520; see also Federer, \emph{GMT}, 1969)|\endgroup\par
\begingroup\color{diffdel}\Verb|-gives, after passing to a subsequence (not relabeled), flat convergence $T_k\to T$ to an integral cycle.|\endgroup\par
\begingroup\color{diffdel}\Verb|-In parallel, Allard's compactness theorem for integral varifolds (Allard, Ann.~of Math.~95 (1972), 417--491)|\endgroup\par
\begingroup\color{diffdel}\Verb|-gives varifold convergence $V_k\to V$.|\endgroup\par
\begingroup\color{diffadd}\Verb|+bound gives tightness; Allard's compactness theorem (Allard, ``On the|\endgroup\par
\begingroup\color{diffadd}\Verb|+first variation of a varifold,'' Ann.~of Math.~95 (1972), 417--491)|\endgroup\par
\begingroup\color{diffadd}\Verb|+gives, after passing to a subsequence (not relabeled):|\endgroup\par
\begingroup\color{diffctx}\Verb| \begin{itemize}|\endgroup\par
\begingroup\color{diffctx}\Verb| \item $V_k\to V$ as varifolds;|\endgroup\par
\begingroup\color{diffctx}\Verb| \item $T_k\to T$ as integral currents in the flat norm;|\endgroup\par
\begingroup\color{diffctx}\Verb| \item $T$ is an integral $(2n-2p)$-cycle with $\partial T=0$;|\endgroup\par
\begingroup\color{diffdel}\Verb|-\item By the period constraints of Proposition~\ref{prop:cohomology-match} (applied to $T_k$) and continuity of current evaluation under flat convergence,|\endgroup\par
\begingroup\color{diffdel}\Verb|-the limit $T$ has the same pairings with a fixed integral basis $\{\Theta_\ell\}$ of $H^{2(n-p)}(X,\Z)$; hence $[T]=\mathrm{PD}(m[\gamma])$|\endgroup\par
\begingroup\color{diffdel}\Verb|-in $H_{2(n-p)}(X,\Z)/\mathrm{tors}$ (equivalently in $H_{2(n-p)}(X,\Q)$).|\endgroup\par
\begingroup\color{diffadd}\Verb|+\item By homological continuity, $[T]=\mathrm{PD}(m[\gamma])$ (since|\endgroup\par
\begingroup\color{diffadd}\Verb|+$T_k$ and $T$ differ by a boundary and cohomology is discrete).|\endgroup\par
\begingroup\color{diffctx}\Verb| \end{itemize}|\endgroup\par
\begingroup\color{diffctx}\Verb| Lower semicontinuity gives|\endgroup\par
\begingroup\color{diffctx}\Verb| \begin{equation}\label{eq:mass-lsc}|\endgroup\par
\end{document}
