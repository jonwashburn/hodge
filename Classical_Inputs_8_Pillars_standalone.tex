
\documentclass[11pt]{article}

\usepackage[T1]{fontenc}
\usepackage[utf8]{inputenc}
\usepackage{lmodern}
\usepackage{amsmath,amssymb,amsthm}
\usepackage[hidelinks]{hyperref}

\title{Classical Inputs (External Published Pillars)}
\author{}
\date{}

\begin{document}
\maketitle

\section*{Scope and standing hypotheses}
Unless stated otherwise, $X$ denotes a compact K\"ahler manifold of complex dimension $n$ with K\"ahler form $\omega$.
When we invoke GAGA, Chow-type algebraicity, or algebraicity of powers of the polarization class,
we additionally assume that $X$ is \emph{projective} and that $[\omega]/2\pi=c_1(L)$ for an ample holomorphic line bundle $L$.

\section*{Eight classical pillars used as external inputs}
Each item below is a \emph{published theorem} used as an external input (i.e.\ not reproved here). For each pillar we record:
(i) a precise statement at the level needed for correct invocation, (ii) a citation, and (iii) a one-line hypothesis check.

\begin{enumerate}

\item \textbf{GAGA comparison (analytic $\leftrightarrow$ algebraic).}
\begin{itemize}
\item \emph{Statement.} If $X$ is a complex projective variety and $X^{\mathrm{an}}$ its analytification, then the functor
``analytification'' induces an equivalence between coherent algebraic sheaves on $X$ and coherent analytic sheaves on $X^{\mathrm{an}}$,
and the induced maps on sheaf cohomology are isomorphisms. In particular, in the projective setting, algebraic and analytic
line bundles/divisors and their cohomological invariants agree under analytification.
\item \emph{Reference.} Serre~\cite{Serre56}.
\item \emph{Hypothesis check.} Invoked only in the projective/polarized setting; the comparison applies to $X^{\mathrm{an}}$.
\end{itemize}

\item \textbf{Existence of flat limits / compactness for integral currents.}
\begin{itemize}
\item \emph{Statement.} Let $M$ be a smooth Riemannian manifold (e.g.\ an open subset of $\mathbb{R}^N$).
If $\{T_j\}$ is a sequence of integral $k$-currents in $M$ with
$\sup_j \mathrm{Mass}(T_j)<\infty$ and $\sup_j \mathrm{Mass}(\partial T_j)<\infty$,
then there exists a subsequence converging in the flat norm to an \emph{integral} $k$-current $T$; moreover $\partial T_j\to \partial T$ in flat norm.
\item \emph{Reference.} Federer--Fleming~\cite{FF60}.
\item \emph{Hypothesis check.} Used only after establishing uniform mass and boundary-mass bounds for the constructed integral currents.
\end{itemize}

\item \textbf{Lower semicontinuity of mass.}
\begin{itemize}
\item \emph{Statement.} If $T_j\to T$ in flat norm (under the normal-current hypotheses provided by the compactness theorem),
then $\mathrm{Mass}(T)\le \liminf_{j\to\infty}\mathrm{Mass}(T_j)$.
\item \emph{Reference.} Federer~\cite{Fed69} (standard GMT lower semicontinuity for normal/integral currents).
\item \emph{Hypothesis check.} Applied only to subsequences produced by the compactness theorem where convergence is in flat norm.
\end{itemize}

\item \textbf{Calibration calculus and defect stability under boundary modifications.}
\begin{itemize}
\item \emph{Statement.} Let $\phi$ be a smooth \emph{closed} $k$-form with comass $\|\phi\|_*\le 1$ (a calibration).
For any integral $k$-current $T$, define the defect $\mathrm{Def}_\phi(T):=\mathrm{Mass}(T)-T(\phi)\ge 0$.
If $T' = T + \partial Q$ for an integral $(k{+}1)$-current $Q$, then $T'(\phi)=T(\phi)$ (by $d\phi=0$ and Stokes),
so $\mathrm{Def}_\phi(T') \le \mathrm{Def}_\phi(T) + \mathrm{Mass}(\partial Q)$ and $\mathrm{Mass}(T')\le \mathrm{Mass}(T)+\mathrm{Mass}(\partial Q)$.
More generally, flatly small modifications change $T(\phi)$ by at most $\|\phi\|_*\,\mathcal{F}(\cdot)$.
\item \emph{Reference.} Standard current calculus as in Federer--Fleming~\cite{FF60} and Federer~\cite{Fed69}.
\item \emph{Hypothesis check.} Used with $\phi$ equal to a closed K\"ahler calibration (e.g.\ $\omega^p/p!$), and with explicitly constructed boundary corrections of controlled mass/flat norm.
\end{itemize}

\item \textbf{Harvey--Lawson: calibrated implies mass-minimizing; Wirtinger equality for complex cycles.}
\begin{itemize}
\item \emph{Statement.} If $\phi$ is a calibration, any $\phi$-calibrated integral current (i.e.\ $T(\phi)=\mathrm{Mass}(T)$)
is homologically mass-minimizing in its integral homology class. On a K\"ahler manifold $(X,\omega)$, the form $\omega^p/p!$
is a calibration and complex analytic $p$-dimensional subvarieties are calibrated by $\omega^p/p!$ (equality case of Wirtinger-type inequalities).
\item \emph{Reference.} Harvey--Lawson~\cite{HL82}.
\item \emph{Hypothesis check.} Applied only once calibration is established (or in comparisons to known complex analytic cycles) on compact K\"ahler $X$.
\end{itemize}

\item \textbf{Hard Lefschetz and Hodge-theoretic Lefschetz isomorphisms.}
\begin{itemize}
\item \emph{Statement.} On a compact K\"ahler manifold $(X,\omega)$ of complex dimension $n$,
cup product with $[\omega]$ satisfies Hard Lefschetz:
for each $r\ge 0$, the map $L^r:H^{n-r}(X,\mathbb{C})\to H^{n+r}(X,\mathbb{C})$ given by $L(\alpha)=[\omega]\smile \alpha$ is an isomorphism,
and it is compatible with the Hodge decomposition; in particular $L^{n-2p}:H^{p,p}(X)\to H^{n-p,n-p}(X)$ is an isomorphism for $p\le n/2$.
\item \emph{Reference.} Voisin~\cite{Voisin02}.
\item \emph{Hypothesis check.} Used under the standing assumption that $X$ is compact K\"ahler.
\end{itemize}

\item \textbf{Uniform interior radius (openness) for the relevant positivity cone.}
\begin{itemize}
\item \emph{Statement.} In the finite-dimensional real vector space $H^{p,p}(X,\mathbb{R})$,
the cone of (strongly) positive $(p,p)$-classes has nonempty interior and is open in its affine span.
Equivalently, if a class $[\beta]$ lies in the interior, then there exists $r_0>0$ (depending on $[\beta]$ and the chosen norm)
such that all classes within distance $<r_0$ of $[\beta]$ remain (strongly) positive.
\item \emph{Reference.} This is a standard convex-topological fact once positivity is identified with a closed convex cone;
for background on positive cones in complex geometry see Demailly~\cite{Demailly12}. For the K\"ahler cone case,
openness is stated and used widely, e.g.\ Demailly--P\u{a}un~\cite{DemaillyPaun04}.
\item \emph{Hypothesis check.} Used only for a fixed interior point of the positivity cone; compactness of the unit sphere in $H^{p,p}(X,\mathbb{R})$ yields a uniform radius in a chosen norm.
\end{itemize}

\item \textbf{Algebraicity of powers of the polarization class.}
\begin{itemize}
\item \emph{Statement.} If $X$ is smooth projective and $H=c_1(\mathcal{O}_X(1))\in H^2(X,\mathbb{Z})$ is the hyperplane (polarization) class,
then $H^r\in H^{2r}(X,\mathbb{Z})$ is Poincar\'e dual to a codimension-$r$ linear section, hence is an algebraic cohomology class.
Equivalently, if $[\omega]/2\pi=c_1(L)$ with $L$ ample, then $[\omega]^r$ is algebraic up to the $(2\pi)^r$ factor.
\item \emph{Reference.} Griffiths--Harris~\cite{GH78}; see also Hartshorne~\cite{Hartshorne77}.
\item \emph{Hypothesis check.} Invoked only in the projective/polarized setting where $\omega$ represents $c_1(L)$ for an ample line bundle.
\end{itemize}

\end{enumerate}

\begin{thebibliography}{99}

\bibitem{Serre56}
J.-P.\ Serre,
\newblock \emph{G\'eom\'etrie alg\'ebrique et g\'eom\'etrie analytique (GAGA)},
\newblock Ann.\ Inst.\ Fourier \textbf{6} (1956), 1--42.

\bibitem{FF60}
H.\ Federer and W.\ H.\ Fleming,
\newblock \emph{Normal and integral currents},
\newblock Ann.\ of Math.\ \textbf{72} (1960), 458--520.

\bibitem{Fed69}
H.\ Federer,
\newblock \emph{Geometric Measure Theory},
\newblock Springer, 1969.

\bibitem{HL82}
R.\ Harvey and H.\ B.\ Lawson,
\newblock \emph{Calibrated geometries},
\newblock Acta Math.\ \textbf{148} (1982), 47--157.

\bibitem{Voisin02}
C.\ Voisin,
\newblock \emph{Hodge Theory and Complex Algebraic Geometry I},
\newblock Cambridge Univ.\ Press, 2002.

\bibitem{Demailly12}
J.-P.\ Demailly,
\newblock \emph{Complex Analytic and Differential Geometry},
\newblock Open book/lecture notes (widely cited; published versions exist in parts), 2012.

\bibitem{DemaillyPaun04}
J.-P.\ Demailly and M.\ P\u{a}un,
\newblock \emph{Numerical characterization of the K\"ahler cone of a compact K\"ahler manifold},
\newblock Ann.\ of Math.\ \textbf{159} (2004), 1247--1274.

\bibitem{GH78}
P.\ Griffiths and J.\ Harris,
\newblock \emph{Principles of Algebraic Geometry},
\newblock Wiley, 1978.

\bibitem{Hartshorne77}
R.\ Hartshorne,
\newblock \emph{Algebraic Geometry},
\newblock Springer, 1977.

\end{thebibliography}

\end{document}
