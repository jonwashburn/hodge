% ==========================================================
% MASTER TEMPLATE FOR COMMUNICATIONS IN ANALYSIS AND GEOMETRY
% Calibration--Coercivity and the Hodge Conjecture
% Final Manuscript Version (Prepared for Submission)
% ==========================================================

\documentclass[11pt]{article}

% ---------- Packages ----------
\usepackage[utf8]{inputenc}
\usepackage[T1]{fontenc}

\usepackage{amsmath, amssymb, amsfonts, amsthm}
\usepackage{mathtools}
\usepackage{mathrsfs}
\usepackage{bm}
\usepackage{geometry}
\usepackage{graphicx}
\usepackage{xcolor}

\geometry{margin=1in}

% Hyperref should generally be loaded last
\usepackage[hypertexnames=false,hidelinks]{hyperref}

% ==========================================================
% Theorem Environments
% ==========================================================
\numberwithin{equation}{section}  % (1.1), (1.2), etc.

\theoremstyle{plain}
\newtheorem{theorem}{Theorem}[section]
\newtheorem{conjecture}[theorem]{Conjecture}
\newtheorem{lemma}[theorem]{Lemma}
\newtheorem{proposition}[theorem]{Proposition}
\newtheorem{corollary}[theorem]{Corollary}
\newtheorem{hypothesis}[theorem]{Hypothesis}

\theoremstyle{definition}
\newtheorem{definition}[theorem]{Definition}
\newtheorem{example}[theorem]{Example}

\theoremstyle{remark}
\newtheorem{remark}[theorem]{Remark}

% ==========================================================
% Macros / Notation
% ==========================================================

% Operators (added by referee patch to avoid undefined controls)
\DeclareMathOperator{\spt}{spt}
\DeclareMathOperator{\Lip}{Lip}

% Basic sets
\newcommand{\R}{\mathbb{R}}
\newcommand{\C}{\mathbb{C}}
\newcommand{\Z}{\mathbb{Z}}
\newcommand{\Q}{\mathbb{Q}}
\newcommand{\N}{\mathbb{N}}

\newcommand{\RR}{\mathbb{R}}
\newcommand{\CC}{\mathbb{C}}
\newcommand{\ZZ}{\mathbb{Z}}
\newcommand{\QQ}{\mathbb{Q}}

\newcommand{\CP}{\mathbb{CP}}
\newcommand{\PP}{\mathbb{P}}

% Small notation
\newcommand{\eps}{\varepsilon}
\newcommand{\ome}{\omega}
\newcommand{\del}{\partial}

\newcommand{\dd}{\mathrm{d}}
\newcommand{\dr}{\mathrm{d}}
\newcommand{\vol}{\mathrm{vol}}
\newcommand{\dvol}{\mathrm{dvol}}    % volume form symbol, e.g. \dvol_\omega

% Script letters
\newcommand{\calH}{\mathcal{H}}
\newcommand{\calO}{\mathcal{O}}
\newcommand{\calC}{\mathcal{C}}
\newcommand{\calK}{\mathcal{K}}
\newcommand{\calU}{\mathcal{U}}
\newcommand{\calV}{\mathcal{V}}
\newcommand{\calB}{\mathcal{B}}
\newcommand{\calG}{\mathcal{G}}

% Blackboard bold misc
\newcommand{\bP}{\mathbb{P}}
\newcommand{\bE}{\mathbb{E}}
\newcommand{\bB}{\mathbb{B}}

% Inner product and norm
\newcommand{\inner}[2]{\left\langle #1, #2 \right\rangle}
\newcommand{\norm}[1]{\left\lVert #1 \right\rVert}

% Linear-algebraic operators
\newcommand{\Id}{\mathrm{Id}}
\newcommand{\tr}{\mathrm{tr}}
\newcommand{\HS}{\mathrm{HS}}         % Hilbert--Schmidt label for norms
\newcommand{\proj}{\mathrm{proj}}     % orthogonal projection

\DeclareMathOperator{\End}{End}
\DeclareMathOperator{\Herm}{Herm}
\DeclareMathOperator{\diag}{diag}
\DeclareMathOperator{\Vol}{Vol}
\DeclareMathOperator{\Mass}{Mass}
\DeclareMathOperator{\M}{M}
\DeclareMathOperator{\Span}{span}

% Geometry / Grassmannians
\newcommand{\Gr}{\mathrm{Gr}}
\newcommand{\Kah}{\mathrm{K\ddot{a}hler}}

\newcommand{\net}{\mathrm{net}}
\newcommand{\dist}{\mathrm{dist}}

% Harmonic / primitive notation
\newcommand{\harm}{\mathrm{harm}}
\newcommand{\gharm}{\gamma_{\harm}}
\newcommand{\prim}{\mathrm{prim}}

% --- Calibration defect & cone distance ---
\newcommand{\Def}{\mathrm{Def}}
\newcommand{\cone}{\mathrm{cone}}

\newcommand{\Defcone}{\Def_{\cone}}          % global calibrated cone defect
\newcommand{\distcone}{\dist_{\cone}}        % pointwise distance to calibrated cone

% --- Kähler calibration form ---
\newcommand{\varphiK}{\varphi}               % symbolic calibration name
\newcommand{\calib}{\omega^{p}/p!}           % actual calibration definition
\newcommand{\calibform}{\frac{\omega^{p}}{p!}} % same, but as a proper fraction

% --- Calibrated Grassmannian (Kähler case) ---
% We will write \Gp(x) for the calibrated Grassmannian at x
\newcommand{\Gp}{G_p}

% --- Parallel calibration notation (Section 11) ---
\newcommand{\distPhi}{\dist_{\Phi}}
\newcommand{\DefPhi}{\Def_{\Phi}}
\newcommand{\Clin}{C_{\mathrm{lin}}}         % C_lin(\Phi) used as \Clin(\Phi)

% ==========================================================
% Edit highlighting
% ==========================================================
% Define colors
\definecolor{editjonColor}{rgb}{0.50,0.00,0.80}  % violet/purple for Jon's elevation plan additions

% Enable colored markup (journal mode: everything black except referee layer)
\newcommand{\editblue}[1]{#1}
\newenvironment{editblock}{\begingroup\color{black}}{\endgroup}
\newcommand{\editref}[1]{#1}
\newenvironment{editrefblock}{\begingroup\color{black}}{\endgroup}
\newcommand{\editp}[1]{#1}
\newenvironment{editpblock}{\begingroup\color{black}}{\endgroup}
\definecolor{editconeColor}{rgb}{0,0,0}
\newcommand{\editcone}[1]{#1}
\newenvironment{editconeblock}{\begingroup\color{black}}{\endgroup}
\definecolor{editjonColor}{rgb}{0,0,0}
\newcommand{\editjon}[1]{#1}
\newenvironment{editjonblock}{\begingroup\color{black}}{\endgroup}

% Referee layer (only this shows in blue)
\newcommand{\editamir}[1]{\textcolor{blue}{#1}}
\newenvironment{editamirblock}{\begingroup\color{blue}}{\endgroup}

% === Referee layer (blue; cumulative across steps) ===
\newcommand{\editamirNEW}[1]{\textcolor{blue}{#1}}
\newenvironment{editamirblockNEW}{\begingroup\color{blue}}{\endgroup}

% AI assistant layer (teal; Jan 2026)
\definecolor{editaiColor}{rgb}{0.00,0.45,0.45}
\newcommand{\editai}[1]{\textcolor{editaiColor}{#1}}
\newenvironment{editaiblock}{\begingroup\color{editaiColor}}{\endgroup}





% ==========================================================
% ==========================================================
% Title & Author Info
% ==========================================================

\title{\bfseries Calibration--Coercivity and the Hodge Conjecture:\\
	A Quantitative Analytic Approach}

\author{
	Jonathan Washburn\thanks{Recognition Science, Recognition Physics Institute,
		Austin, Texas, USA. Email: \texttt{jon@recognitionphysics.org}.}
	\and
	Amir Rahnamai Barghi\thanks{Concord, Ontario, Canada. Corresponding author.
		Email: \texttt{arahnamab@gmail.com}.}
}

\date{\today}
\begin{document}
	\maketitle

\begin{abstract}
\begin{editconeblock}
We reduce the Hodge problem to a \emph{realization/microstructure} statement for smooth closed strongly positive $(p,p)$--forms.
The key algebraic reduction is that any rational Hodge class
\[
\gamma \in H^{2p}(X,\Q)\cap H^{p,p}(X)
\]
admits a signed decomposition $\gamma=\gamma^+-\gamma^-$ with $\gamma^- = N[\omega^p]$ algebraic (complete intersections) and
$\gamma^+=\gamma+N[\omega^p]$ \emph{cone--positive} (i.e.\ admitting a smooth closed cone-valued representative) for $N\gg1$.

For a cone--positive class with representative $\beta$, the main construction produces integral cycles $T_k$
in the fixed class $\mathrm{PD}(m[\gamma^+])$ whose \emph{calibration defects} satisfy
$\Mass(T_k)-\langle T_k,\psi\rangle\to 0$, and hence whose masses converge to the cohomological lower bound
$\Mass(T_k)\to m\int_X \beta\wedge\psi$.
\editamir{By Proposition~\ref{prop:almost-calibration}, the constructed cycles have uniformly bounded mass and $\psi$--defect $\to 0$. Hence, by Federer--Fleming compactness for integral currents on a compact manifold (e.g.\ \cite{Fed69}) and varifold compactness (e.g.\ \cite{Allard72,Sim83}), after passing to a subsequence we obtain a flat/varifold limit $T$ which is $\psi$--calibrated. The identification of such a $T$ as a positive holomorphic chain is Theorem~\ref{thm:syr-realization}; algebraicity on projective $X$ then follows by Remark~\ref{rem:chow-gaga}.}
Combining with the signed decomposition yields algebraicity of $\gamma$ (after reducing to $p\le n/2$ by Hard Lefschetz).

We also record an auxiliary calibration--coercivity observation in the special CPM--bridge regime where the harmonic representative is cone-valued; this is not used in the main realization/SYR chain.
\end{editconeblock}
\end{abstract}

\begin{editamirblock}
\section*{Parameter and notation dictionary (referee layer)}
\noindent\textbf{Purpose.} The manuscript uses several global parameters repeatedly.  This short dictionary
records the intended meaning when a symbol is used without an immediate local definition.  If a later
statement explicitly redefines a symbol, that local definition takes precedence.

\medskip
\noindent\textbf{Geometric data.}
\begin{itemize}
\item $X$: smooth complex \emph{projective} manifold, $\dim_\C X=n$.
\item $L\to X$: ample line bundle (polarization); $\omega\in c_1(L)$ is a fixed K\"ahler form representing the Chern class.
\item $p$: codimension parameter; the Hodge class lives in degree $2p$.
\item $\psi$: the K\"ahler (Wirtinger) calibration $\psi:=*\varphi=\omega^{n-p}/(n-p)!$ of type $(n-p,n-p)$, calibrating complex $(n-p)$--planes (see the paragraph ``Let $\varphi=\omega^p/p!$ and let $\psi:=*\varphi$ in the main text).
\end{itemize}


\noindent\textbf{Holomorphic sections / jets.}
\begin{itemize}
\item $H^0(X,L^N)$: the complex vector space of global holomorphic sections of $L^{\otimes N}$.
\item $J_x^k(L^N)$: the $k$--jet space at $x$, i.e. germs of holomorphic sections of $L^{\otimes N}$ modulo those vanishing to order $k+1$ at $x$.
Equivalently $J_x^k(L^N)\cong \mathcal{O}_X(L^{\otimes N})_x/\mathfrak{m}_x^{k+1}$.
\end{itemize}

\noindent\textbf{Cohomology / cycles.}
\begin{itemize}
\item $\gamma\in H^{2p}(X,\Q)$: a rational $(p,p)$-class (viewed in de~Rham / singular cohomology as needed).
\item $\mathrm{PD}(\gamma)$: Poincar\'e dual homology class.
\item $\mathcal{F}(\cdot)$: Federer--Fleming flat norm on integral currents (Definition~\ref{def:flat-norm}); $\partial$ denotes boundary of a current.
\end{itemize}
\begin{editamirblockNEW}
\begin{definition}[Flat norm on integral currents]\label{def:flat-norm}
Fix an integer $\ell\ge 0$.
For an integral $\ell$--current $T$ on $X$, the \emph{flat norm} is
\[
\mathcal{F}(T)\ :=\ \inf\Bigl\{\Mass(R)+\Mass(Q)\ :\ T=R+\partial Q,\ 
R\ \text{integral $\ell$--current},\ Q\ \text{integral $(\ell+1)$--current}\Bigr\}.
\]
In particular, if $T$ is an integral $\ell$--cycle, then in any decomposition $T=R+\partial Q$ with $R,Q$ integral
one has $\partial R=0$ automatically.
\end{definition}
\end{editamirblockNEW}


\noindent\textbf{Scale parameters (used in the gluing/template constructions).}
\begin{itemize}
\item $m\in\mathbb{N}$: cohomology multiplier in the target class $\mathrm{PD}(m[\gamma])$ (Definition~\ref{def:syr}); it controls the total pairing $c_0=\langle \mathrm{PD}(m[\gamma]),[\psi]\rangle$.
\item $N\gg 1$: (when invoked) the \emph{holomorphic/Bergman quantization parameter} for $L^{\otimes N}$; intrinsic analytic scale is $\sim N^{-1/2}$.
\item \editamir{$N_{\mathrm{Car}}$: Carath\'eodory bound $N_{\mathrm{Car}}(n,p)$ on the number of calibrated directions needed to express a strongly positive form at a point (Lemma~\ref{lem:caratheodory-general}).}
\item $h>0$: mesh size of the cubulation.
\item $\varepsilon$: small tolerance (slope/angle or separation threshold); $\varepsilon_h$ is the direction-net resolution at mesh $h$.
\item \editamir{$\varrho=\varrho(h)\in(0,1]$: \emph{transverse parameter radius factor}; template translation parameters are chosen in a ball of radius $\varrho\,h$ (hence per-face displacement improves to $\Delta_F\lesssim \varrho\,h^{2}$). In the borderline case $p=n/2$ we impose $\varrho=o(\varepsilon)$.}
\item \editamir{$\delta$: transverse grid spacing / separation scale used to keep different sheets/templates disjoint; in the refined schedule we take $\delta\asymp \varrho\,\varepsilon\,h$ so that $B_{\,\varrho h}(0)$ still contains $\asymp \varepsilon^{-2p}$ available lattice sites.}
\item $s$: corner-exit translation scale (size of perturbation used to exit through specified faces).
\end{itemize}

\noindent\textbf{Template-net constants (depend on $(h,\varepsilon_h)$ unless explicitly made uniform).}
\begin{itemize}
\item $\alpha_*(h)$, $\alpha^*(h)$: lower/upper coefficient bounds for the linear template system.
\item $A_*(h)$: bound controlling the size of the coefficient vectors (a conditioning constant).
\item $\Lambda(h)$: Lipschitz/variation constant for the templates as labels vary.
\item $c_0$: fixed universal constant appearing in corner-exit/realization inequalities.
\end{itemize}

\noindent\textbf{Calibration defect.}
\[
\Def_{\mathrm{cal}}(T):=\Mass(T)-\langle T,\psi\rangle,
\]
so ``almost-calibrated'' means $\Def_{\mathrm{cal}}(T)$ is small compared to the relevant scaling.
\end{editamirblock}



	\section{Introduction}
\noindent
\begin{editconeblock}
This section formulates the Hodge problem for a fixed rational $(p,p)$ class on
a smooth complex projective manifold and summarizes the proof strategy used in this manuscript.
The main technical ingredient is a \emph{realization/microstructure} theorem: given a smooth closed cone-valued $(p,p)$--form $\beta$ in a rational class, we construct fixed-class integral cycles whose calibration defects tend to $0$, and hence whose masses converge to the cohomological lower bound.
The calibrated limit is therefore a positive sum of complex analytic subvarieties (Harvey--Lawson), hence algebraic on projective manifolds (Chow/GAGA).
Finally, a signed decomposition reduces an arbitrary rational Hodge class to the cone--positive case, and Hard Lefschetz wires the $p$--range cleanly.

We keep the phrase “calibration--coercivity’’ for historical motivation: in the special CPM--bridge regime where the harmonic representative is pointwise cone-valued, the cone defect is trivially controlled by the $L^2$ distance to $\gamma_{\harm}$ (Section~\ref{sec:cal-coercivity}); however this coercivity observation is not used in the main realization/SYR chain.
\end{editconeblock}
\vspace{0.3cm}

\subsection*{Problem}

Let $X$ be a smooth projective complex variety of complex dimension $n$,
equipped with a K\"ahler form $\omega$ in the Chern class of a fixed ample line bundle $L\to X$ (i.e.\ $[\omega]=c_1(L)$).  Fix an integer $1 \leq p \leq n$ and a
rational Hodge class
\[
\gamma \;\in\; H^{2p}(X,\Q) \cap H^{p,p}(X).
\]
The Hodge problem asks whether there exists an algebraic cycle $Z$ of
codimension $p$ whose cohomology class satisfies
\[
[Z] = \gamma \in H^{2p}(X,\Q).
\]
Equivalently, the problem is to decide whether every rational $(p,p)$ class on a
smooth complex projective manifold admits an algebraic cycle representative.
This is the classical Hodge conjecture for the class $\gamma$.

\subsection*{Route via calibration and energy}

Set the K\"ahler calibration
\[
\varphi := \frac{\omega^{p}}{p!}.
\]
For any smooth closed $2p$--form $\alpha$ representing the class $[\gamma]$, define
its Dirichlet energy
\[
E(\alpha) := \int_{X} \|\alpha\|^{2}\, d\mathrm{vol}_{\omega}.
\]
Let $\gamma_{\harm}$ denote the $\omega$--harmonic representative of $[\gamma]$.

To measure the pointwise misalignment of $\alpha$ from the \emph{strongly positive} calibrated cone
$K_{p}(x)$ associated to $\varphi$, define the pointwise cone distance
\[
\editcone{\dist_{\cone}(\alpha_{x})}
\editcone{:=}
\editcone{\inf_{\beta_x\in K_p(x)}\|\alpha_x-\beta_x\|.}
\]
The global cone defect is then
\[
\editcone{\Def_{\cone}(\alpha)}
\editcone{:=}
\editcone{\int_{X} \dist_{\cone}(\alpha_{x})^{2}\, d\mathrm{vol}_{\omega}.}
\]

This functional quantifies, in an $L^{2}$ sense, how far a closed
representative $\alpha$ lies from the K\"ahler calibrated cone.  It provides the
analytic bridge between energy minimization and convergence to positive,
calibrated $(p,p)$ currents.

\subsection*{Main quantitative theorem (calibration--coercivity, explicit)}

\begin{theorem}[Calibration--coercivity (cone-valued harmonic classes)]\label{thm:cal-coercivity-intro}
\editcone{Assume the $\omega$--harmonic representative satisfies $\gamma_{\harm}(x)\in K_p(x)$ for all $x\in X$.}
\editcone{Then for every smooth closed $2p$--form $\alpha \in [\gamma]$,}
\[
\editcone{E(\alpha) - E(\gamma_{\harm})\ \ge\ \Def_{\cone}(\alpha).}
\]
\end{theorem}
\editcone{(See Theorem~\ref{thm:cal-coercivity} in Section~\ref{sec:cal-coercivity} for the proof; this hypothesis is exactly the CPM--bridge assumption that the energy minimizer already lies in the structured cone.)}


\begin{proof}
This is the simplified introductory statement of the explicit calibration--coercivity theorem proved later in the manuscript for cone-valued harmonic classes. That later argument establishes
\[
E(\alpha)-E(\gamma_{\harm}) \;\ge\; \Def_{\cone}(\alpha)
\]
for every smooth closed representative $\alpha\in[\gamma]$ under the same pointwise cone hypothesis on $\gamma_{\harm}$, so the present formulation follows directly.
\end{proof}

This inequality asserts that the Dirichlet energy gap above the harmonic
representative uniformly controls the global calibration defect of $\alpha$, and
thus links energy minimization quantitatively to geometric alignment with the
K\"ahler calibrated cone.

\subsection*{Consequences for Hodge: cone--positive classes}

For \emph{cone--positive} classes $\gamma$---those admitting a smooth closed cone-valued
representative $\beta$ with $\beta(x) \in K_p(x)$---the microstructure/gluing theorem
recorded in Proposition~\ref{prop:glue-gap} produces fixed-class integral
cycles $T_k$ with $\Mass(T_k)\to c_0$ (equivalently, $\Mass(T_k)-\langle T_k,\psi\rangle\to 0$).
By Theorem~\ref{thm:realization-from-almost}, a subsequential limit is a $\psi$--calibrated integral current; Harvey--Lawson then identifies it as a positive sum of complex analytic subvarieties, hence algebraic on projective $X$ by Chow/GAGA.

\subsection*{Consequences for Hodge: general classes via signed decomposition}

For a general rational Hodge class $\gamma$, the harmonic representative
$\gamma_{\mathrm{harm}}$ need not be cone-valued.  The key observation is that
every such $\gamma$ admits a \emph{signed decomposition}
\[
\gamma = \gamma^{+} - \gamma^{-},
\]
where both $\gamma^{+}$ and $\gamma^{-}$ are cone--positive (in the smooth cone sense).  Specifically:
\begin{itemize}
\item $\gamma^{-} := N[\omega^{p}]$ is already algebraic (represented by
complete intersections of hyperplane sections).
\item $\gamma^{+} := \gamma + N[\omega^{p}]$ becomes cone-valued for $N$
sufficiently large, since the K\"ahler form $\omega^{p}$ is strictly positive
in the calibrated cone.
\end{itemize}

Applying the cone--positive machinery to $\gamma^{+}$ yields an algebraic
cycle $Z^{+}$.  Combined with the algebraic cycle $Z^{-}$ representing
$\gamma^{-}$, we obtain
\[
\gamma = [Z^{+}] - [Z^{-}],
\]
proving that $\gamma$ is algebraic.  \editblue{The signed decomposition is an unconditional reduction:
it reduces the general case to proving algebraicity for cone--positive classes via the
realization/microstructure step.}

\subsection*{What is new}

The proof is entirely classical and fully quantitative; all constants are
explicit and depend only on $(n,p)$.  In particular:

\begin{itemize}
	\item An $\varepsilon$--net on the calibrated Grassmannian with
	$\varepsilon = \tfrac{1}{10}$ satisfies the explicit covering bound
	\[
	N(n,p,\varepsilon) \le 30^{\,2p(n-p)}.
	\]
	
	\item A cone-to-net distortion factor $K$ may be recorded for comparison with the
	ray/net framework, though the cone-based argument does not require it.
	
	\item A uniform pointwise linear-algebra constant controls the distance to the
	calibrated net in terms of the off-type $(p\pm1,p\mp1)$ components and the
	primitive part of the $(p,p)$ component:
	\[
	C_{0}(n,p) = 2.
	\]
\end{itemize}

\begin{editconeblock}
These components are included only as optional quantitative background (nets and Hermitian linear algebra).
The main realization/SYR chain does not use them.
\end{editconeblock}

\subsection*{Idea of the proof}

\begin{editconeblock}
The proof has three conceptual steps.

\paragraph{1. Reduction to $p\le n/2$ and to cone--positive classes.}
By Hard Lefschetz (Remark~\ref{rem:lefschetz-reduction}), it suffices to treat the range $p\le n/2$.
For a general rational Hodge class $\gamma\in H^{2p}(X,\Q)\cap H^{p,p}(X)$, a signed decomposition
$\gamma=\gamma^+-\gamma^-$ with $\gamma^- = N[\omega^p]$ and $\gamma^+=\gamma+N[\omega^p]$
reduces the problem to showing that \emph{cone--positive} classes (those admitting smooth closed cone-valued representatives) are algebraic.

\paragraph{2. Realization (SYR) for a cone-valued representative.}
Fix a cone--positive class $\gamma^+$ with a smooth closed cone-valued representative $\beta$.
Section~\ref{sec:realization} constructs, for a fixed integer $m$, a sequence of integral cycles $T_k$
in the class $\mathrm{PD}(m[\gamma^+])$ such that $\Mass(T_k)-\langle T_k,\psi\rangle\to 0$ (hence $\Mass(T_k)\to m\int_X\beta\wedge\psi$), culminating in the SYR summary theorem (Theorem~\ref{thm:automatic-syr}).
The key technical point is the microstructure/gluing estimate $\mathcal F(\partial T^{\mathrm{raw}})=o(m)$ (Proposition~\ref{prop:glue-gap}),
which is achieved by holomorphic corner-exit slivers and weighted flat-norm summation on a mesh.

\paragraph{3. Calibrated limit and algebraicity.}
Almost-calibration implies that any flat/varifold limit of the $T_k$ is $\psi$--calibrated.
By Harvey--Lawson, the limit is integration along a positive sum of complex analytic subvarieties, hence algebraic on projective $X$ by Chow/GAGA.
Thus $\gamma^+$ is algebraic; together with algebraicity of $\gamma^-$, this yields algebraicity of $\gamma=\gamma^+-\gamma^-$.

\smallskip\noindent
\textbf{Remark on “coercivity”.} Section~\ref{sec:cal-coercivity} records a coercivity inequality in the special CPM--bridge regime where the harmonic representative is cone-valued;
this observation is not used in the main chain above.
\end{editconeblock}

\subsection*{Scope and remarks}

\begin{editpblock}
The analytic estimates are uniform in $(n,p)$.
However, the \emph{microstructure/gluing} scaling regime used to conclude the decisive estimate
$\mathcal F(\partial T^{\mathrm{raw}})=o(m)$ is proved in the range $p\le n/2$
(see Remark~\ref{rem:weighted-scaling}).
This is sufficient for the full Hodge statement because, in the projective setting, Hard Lefschetz reduces the Hodge conjecture to $p\le n/2$
(Remark~\ref{rem:lefschetz-reduction}), and the case $p>n/2$ is recovered by intersecting with hyperplanes.

On K\"ahler manifolds not assumed projective, the construction yields analytic cycles; algebraicity then requires projectivity of $X$.
\end{editpblock}
All constants are explicit and uniform in $(X,\omega)$.
While some constants (e.g.\ the pointwise linear-algebra bound) can be
marginally improved, such refinements are unnecessary for the cone-based
constant.

The bound $N \le 30^{\,2p(n-p)}$ for the covering number of the calibrated
Grassmannian is convenient but not optimal; any standard packing estimate would
suffice.

\subsection*{Notation and conventions}

All norms and inner products are induced by the K\"ahler metric.  Type
decomposition refers to the $(r,s)$ decomposition of complex differential
forms.  The Lefschetz decomposition into primitive and non-primitive components
is orthogonal with respect to $\omega$.  Weak convergence is taken in the sense
of currents.  Energies and $L^{2}$ norms are over $\R$, while cohomology is
taken over $\Q$ when rationality is required.

\subsection*{Organization}

\begin{editconeblock}
Sections~2--6 record geometric/analytic background (K\"ahler preliminaries, calibrated Grassmannian geometry, and auxiliary linear algebra on nets and Hermitian models).
Section~\ref{sec:cal-coercivity} records an optional coercivity observation in the CPM--bridge regime (where the harmonic representative is cone-valued).
Section~\ref{sec:realization} is the heart of the manuscript: it proves the projective tangential approximation and the microstructure/gluing theorem needed to realize smooth cone-valued forms by holomorphic pieces with vanishing flat-norm boundary (after correction by integral fillings), culminating in the SYR summary theorem (Theorem~\ref{thm:automatic-syr}).
Finally, the signed decomposition lemma reduces an arbitrary rational Hodge class to the cone--positive case, and the main theorem follows.
\end{editconeblock}

\subsection*{Proof structure}

The \editblue{overall strategy} has three main components:
\begin{enumerate}
\item \textbf{Signed decomposition:} Any $\gamma$ equals $\gamma^{+} - \gamma^{-}$
with $\gamma^{\pm}$ cone--positive.  Here $\gamma^{-} = N[\omega^{p}]$ is already
algebraic.
\item \textbf{Cone--positive $\Rightarrow$ algebraic:} For cone--positive classes,
\begin{editconeblock}
the realization/SYR construction produces almost-calibrated integral cycles and a calibrated limit current (Theorem~\ref{thm:automatic-syr}), which is
algebraic by Harvey--Lawson and Chow/GAGA.
\end{editconeblock}
\item \textbf{Conclusion:}
$\gamma = [Z^{+}] - [Z^{-}]$ is algebraic.
\end{enumerate}

\begin{editconeblock}
\subsection*{Referee dependency checklist (one page)}
\begin{center}
\fbox{\begin{minipage}{0.94\linewidth}
\small
\textbf{Main closure chain (used for Theorem~\ref{thm:main-hodge}).}
\begin{enumerate}
\item \textbf{Hard Lefschetz reduction} (Remark~\ref{rem:lefschetz-reduction}): reduces the Hodge problem to the range $p\le n/2$.
\item \textbf{Signed decomposition} (Lemma~\ref{lem:signed-decomp}): $\gamma=\gamma^+-\gamma^-$ with $\gamma^- = N[\omega^p]$ and $\gamma^+$ cone--positive.
\item \textbf{Algebraicity of $\gamma^-$} (Lemma~\ref{lem:gamma-minus-alg}): $[\omega^p]$ is represented by complete intersections, hence $\gamma^-$ is algebraic.
\item \textbf{Microstructure/gluing estimate} (Proposition~\ref{prop:glue-gap}): $\mathcal F(\partial T^{\mathrm{raw}})=o(m)$ for the constructed sheet-sum on a mesh (in the range $p\le n/2$; see Remark~\ref{rem:weighted-scaling}).
\item \textbf{Mass convergence / almost-calibration} (Proposition~\ref{prop:almost-calibration}): for the corrected cycles $T_\epsilon=S-U_\epsilon$ one has
$\Mass(T_\epsilon)-\langle T_\epsilon,\psi\rangle\to 0$ and hence $\Mass(T_\epsilon)\to c_0$ with $c_0=\langle \mathrm{PD}(m[\gamma^+]),[\psi]\rangle$.
\item \textbf{Automatic SYR} (Theorem~\ref{thm:automatic-syr}): starting from a smooth closed cone-valued representative $\beta$ of $\gamma^+$, the construction yields fixed-class integral cycles with vanishing calibration defect (hence $\Mass(T_k)\to c_0$).
\item \textbf{Calibrated limit and algebraicity}:
Theorem~\ref{thm:realization-from-almost} gives a $\psi$--calibrated integral limit current; Harvey--Lawson identifies it with a positive sum of complex analytic subvarieties, which are algebraic on projective $X$ by Remark~\ref{rem:chow-gaga}.
\end{enumerate}

\smallskip
\textbf{Explicitly not used in the main chain above:}
the Hermitian/PSD and net linear-algebra discussions (Sections~\ref{sec:energy-gap}--\ref{sec:linear-algebra}) and the optional coercivity statement for cone-valued harmonic representatives (Section~\ref{sec:cal-coercivity}).
\end{minipage}}
\end{center}
\end{editconeblock}

\section{Notation and K\"ahler Preliminaries}

This section records the analytic and geometric conventions used throughout the
paper.  All norms, operators, and identities are taken with respect to the
K\"ahler metric $g(\cdot,\cdot)=\omega(\cdot,J\cdot)$ and the associated volume
form $d\mathrm{vol}_\omega=\omega^{n}/n!$.  These preliminaries fix the
\begin{editconeblock}
functional-analytic framework for calibrations, currents, and the gluing estimates used later.
\end{editconeblock}

% ----------------------------------------------------------
\paragraph{Ambient setting.}
Let $X$ be a smooth projective complex manifold of complex dimension $n$, with
K\"ahler form $\omega$ and integrable complex structure $J$. \editamir{Fix an ample line bundle $L\to X$ with a Hermitian metric whose curvature form equals $\omega$ (so $[\omega]=c_1(L)\in H^2(X,\Z)$).}
We assume $\omega$ lies in the Chern class of a fixed ample line bundle $L\to X$ (so $[\omega]=c_1(L)$).
The associated Riemannian metric is
\[
g(\cdot,\cdot)=\omega(\cdot,J\cdot),
\qquad
d\mathrm{vol}_\omega=\frac{\omega^{n}}{n!}.
\]
Throughout the paper, all pointwise and $L^2$ norms are taken with respect to
$g$ (equivalently,~$\omega$).

% ----------------------------------------------------------
\paragraph{Forms, inner products, and energy.}
For $k\ge0$, let $\Lambda^{k}T^{*}X$ denote the bundle of real $k$–forms and
$\Lambda_{\C}^{k}T^{*}X=\Lambda^{k}T^{*}X\otimes\C$ its complexification.
The Hodge star
\[
*:\Lambda^{k}T^{*}X\longrightarrow\Lambda^{2n-k}T^{*}X
\]
satisfies
\[
\langle \alpha,\beta\rangle_{x}\,d\mathrm{vol}_\omega
=
\alpha\wedge *\beta,
\]
and the pointwise norm is $\|\alpha\|^{2}=\langle \alpha,\alpha\rangle$.
The $L^{2}$ inner product and norm are
\[
\langle \alpha,\beta\rangle_{L^{2}}
:=
\int_{X}\langle \alpha,\beta\rangle\,d\mathrm{vol}_\omega,
\qquad
\|\alpha\|^{2}_{L^{2}}
:=
\int_{X}\|\alpha\|^{2}\,d\mathrm{vol}_\omega.
\]
For any measurable $2p$–form $\alpha$, the Dirichlet energy agrees with its
$L^{2}$ norm:
\[
E(\alpha)
=
\|\alpha\|^{2}_{L^{2}}
=
\int_{X}\|\alpha\|^{2}\,d\mathrm{vol}_\omega.
\]

% ----------------------------------------------------------
\paragraph{Exterior calculus and Hodge theory.}
Let $d$ be the exterior derivative and $d^{*}$ its formal adjoint.
The Hodge Laplacian is
\[
\Delta = dd^{*}+d^{*}d.
\]
A smooth form $\eta$ is \emph{harmonic} if $\Delta\eta=0$.
Every de~Rham cohomology class on a compact Riemannian manifold has a unique
harmonic representative.

If $\alpha$ is a smooth closed $k$–form representing a class $[\gamma]$, then
there exists a $(k-1)$–form $\xi$ with $d^{*}\xi=0$ (Coulomb gauge) such that
\[
\alpha=\gharm+d\xi,
\qquad
E(\alpha)-E(\gharm)=\|d\xi\|^{2}_{L^{2}}.
\tag{2}
\]

% ----------------------------------------------------------
\paragraph{Type decomposition.}
Complexifying the cotangent bundle gives
\[
T^{*}X\otimes\C
=
T^{1,0*}X\oplus T^{0,1*}X.
\]
Taking wedge powers yields the $(r,s)$–splitting
\[
\Lambda_{\C}^{k}T^{*}X
=
\bigoplus_{r+s=k}\Lambda^{r,s}T^{*}X.
\]
For a complex form $\alpha$, we write $\alpha^{(r,s)}$ for its $(r,s)$
component.  In particular, any complex $2p$–form decomposes as
\[
\alpha
=
\alpha^{(p+1,p-1)}
+
\alpha^{(p,p)}
+
\alpha^{(p-1,p+1)}.
\]
On a K\"ahler manifold,
\[
d=\partial+\bar\partial,
\qquad
\partial:\Lambda^{r,s}\to\Lambda^{r+1,s},
\quad
\bar\partial:\Lambda^{r,s}\to\Lambda^{r,s+1}.
\]
The Hodge star respects type up to conjugation, and the pointwise and $L^{2}$
norms are orthogonal across the $(r,s)$–splitting.

% ----------------------------------------------------------
\paragraph{Lefschetz operators and primitive forms.}
The Lefschetz operator
\[
L:\Lambda_{\C}^{\bullet}T^{*}X\to\Lambda_{\C}^{\bullet+2}T^{*}X,
\qquad
L(\eta)=\omega\wedge\eta,
\]
has $L^{2}$–adjoint $\Lambda$ (contraction with $\omega$).
A form $\eta$ is \emph{primitive} if $\Lambda\eta=0$.

The Lefschetz decomposition expresses any $(p,p)$–form as an orthogonal sum
\[
\alpha^{(p,p)}=\sum_{r\ge0}L^{r}\eta_{r},
\qquad
\eta_{r}\ \text{primitive}.
\]
We write $(\cdot)_{\prim}$ for the orthogonal projection onto the primitive
subspace.

% ----------------------------------------------------------
\paragraph{K\"ahler identities (used implicitly).}
On a K\"ahler manifold one has the commutator identities
\[
[\Lambda,\partial]=i\,\bar\partial^{*},
\qquad
[\Lambda,\bar\partial]=-\,i\,\partial^{*},
\]
and their adjoints.
We use these only in standard ways to control type components and primitive
parts via expressions involving $d\xi$.

% ==========================================================
% SECTION 3 — Calibrated Grassmannian and Pointwise Cone Geometry (Revised)
% ==========================================================

\section{Calibrated Grassmannian and Pointwise Cone Geometry}
\label{sec:calibrated-grassmannian}

\paragraph{Calibrated Grassmannian.}
Fix a point $x\in X$.  
Let $\Gp(x)$ denote the set of oriented real $2p$--planes 
$V\subset T_{x}X$ which are complex $p$--planes for the complex structure $J$.
Equivalently, $\Gp(x)$ is naturally identified with the complex
Grassmannian $G_{\C}(p,n)$ of $p$--dimensional complex subspaces of
$T^{1,0}_{x}X$.  

Given such a $V\in \Gp(x)$, let $\phi_{V}$ be the normalized
calibrated simple $(p,p)$--form associated to $V$, defined by
\[
\phi_{V}\bigl( v_{1},Jv_{1},\ldots,v_{p},Jv_{p} \bigr) = 1
\]
for any orthonormal basis $\{v_{1},\ldots,v_{p}\}$ of $V$.
Thus each $\phi_{V}$ has unit pointwise norm and determines the calibrated
direction corresponding to the holomorphic $p$--plane $V$.

\paragraph{Calibrated cone at a point.}
Let
\[
\varphi \;=\; \calibform \;=\; \frac{\omega^{p}}{p!}
\]
be the Kähler calibration.
Define the (closed, convex) calibrated cone in $\Lambda^{2p}T^{*}_{x}X$ by
\[
\mathcal{C}_{x}
:=
\Bigl\{
\sum_{j} a_{j} \phi_{V_{j}}
\;:\;
a_{j}\ge 0,\;
V_{j}\in \Gp(x)
\Bigr\}.
\]
Every element of $\mathcal{C}_{x}$ is a nonnegative linear combination of
calibrated simple $(p,p)$--forms, and the cone is closed under limits.

\begin{editamirblock}
\begin{lemma}[Closure of the calibrated cone]\label{lem:calibrated-cone-closed}
For each $x\in X$, the cone $\mathcal{C}_{x}\subset \Lambda^{2p}T_x^*X$ is closed.
In particular, for every $\alpha_x$ the infimum in $\dist(\alpha_x,\mathcal{C}_x)$ is attained.
\end{lemma}

\begin{proof}
Let $\alpha_k\in\mathcal{C}_x$ be a convergent sequence with $\alpha_k\to \alpha$.
By Carath\'eodory's theorem for convex cones in finite-dimensional vector spaces, each $\alpha_k$ admits a representation
\[
\alpha_k=\sum_{j=1}^{M} a_{k,j}\,\phi_{V_{k,j}},
\qquad a_{k,j}\ge 0,\ \ V_{k,j}\in \Gp(x),
\]
where $M=\dim_{\R}\Lambda^{2p}T_x^*X$ (any fixed finite bound suffices).
Each generator has unit norm $\|\phi_{V_{k,j}}\|=1$ and, by the K\"ahler-angle formula,
$\langle \phi_{V},\phi_{W}\rangle\in[0,1]$ for all $V,W\in\Gp(x)$.
Therefore
\[
\|\alpha_k\|^2
=\sum_{i,j}a_{k,i}a_{k,j}\langle\phi_{V_{k,i}},\phi_{V_{k,j}}\rangle
\ \ge\ \sum_{j=1}^{M} a_{k,j}^2,
\]
so the coefficients $\{a_{k,j}\}$ are uniformly bounded (since $\{\alpha_k\}$ converges).
After passing to a subsequence we may assume $a_{k,j}\to a_j\ge 0$ for each $j$.
Since $\Gp(x)\cong G_{\C}(p,n)$ is compact, after further passing to a subsequence we may assume
$V_{k,j}\to V_j\in\Gp(x)$ for each $j$.
By continuity of $V\mapsto \phi_V$ we obtain
\[
\alpha=\lim_{k\to\infty}\alpha_k
=\sum_{j=1}^{M} a_j\,\phi_{V_j}\in\mathcal{C}_x,
\]
so $\mathcal{C}_x$ is closed.  Since $\mathcal{C}_x$ is a closed convex subset of a finite-dimensional inner-product space,
nearest-point projection exists and the distance infimum is attained.
\end{proof}
\end{editamirblock}

We write
\[
\distcone(\alpha_{x})
:=
\dist\!\bigl(\alpha_{x},\mathcal{C}_{x}\bigr)
\]
for the pointwise distance (with respect to the $g$--norm) from a real
$2p$--form $\alpha_{x}$ to the calibrated cone at $x$.

\paragraph{Finite calibrated frame (net viewpoint).}
Fix $\varepsilon = \tfrac{1}{10}$.
Choose a maximal $\varepsilon$--separated subset 
$\{V_{1},\ldots,V_{N}\}\subset \Gp(x)$, i.e.\ an $\varepsilon$--net
of the calibrated Grassmannian with respect to its standard homogeneous
Riemannian metric.  
Standard packing estimates on the complex Grassmannian yield the explicit
bound
\[
N \;\le\; 30^{\,2p(n-p)}.
\]


\iffalse
% --- [DEPRECATED: incorrect cone-to-span distortion claim; retained for record] ---
Let $\Xi_{x}$ denote the linear span of 
$\{\phi_{V_{1}},\ldots,\phi_{V_{N}}\}$ inside $\Lambda^{2p}T^{*}_{x}X$.
For any form $\alpha_{x}$, let
\[
\dist(\alpha_{x}, \Xi_{x})
\]
be the pointwise norm of the orthogonal projection of $\alpha_{x}$ onto the
orthogonal complement of $\Xi_{x}$.

For convenience we record the cone--to--net comparison constant
\[
K = \Bigl(\tfrac{11}{9}\Bigr)^{2} = \frac{121}{81},
\]
satisfying
\[
\distcone(\alpha_{x})^{2}
\;\le\;
K \,\dist\bigl(\alpha_{x},\Xi_{x}\bigr)^{2}.
\]
The main cone--based proof uses the calibrated cone $\mathcal{C}_{x}$
directly and does not rely on the factor $K$, but the net viewpoint is
included for completeness.\fi

\begin{editamirblockNEW}\editamir{
\noindent\textbf{Finite calibrated frame (net viewpoint, corrected).}
Fix $\varepsilon=\tfrac{1}{10}$ and let $\{V_{1},\ldots,V_{N}\}\subset \Gp(x)$ be an $\varepsilon$--net.
Set
\[
M(\alpha_{x}) := \max_{V\in \Gp(x)} \langle \alpha_{x},\phi_{V}\rangle_{+},
\qquad
M_{\varepsilon}(\alpha_{x}) := \max_{1\le j\le N}\langle \alpha_{x},\phi_{V_{j}}\rangle_{+}.
\]
Using Lemma~\ref{lem:kahler-angle} (equivalently, a Lipschitz bound for $V\mapsto \phi_{V}$ on $\Gp(x)$)
one has $\|\phi_{V}-\phi_{V'}\|\le C_{n,p}\,d_{\Gp}(V,V')$ for some $C_{n,p}$ depending only on $(n,p)$.
Therefore, for any $V$ and a net point $V_{j}$ with $d_{\Gp}(V,V_{j})\le \varepsilon$,
\[
\bigl|\langle \alpha_{x},\phi_{V}\rangle - \langle \alpha_{x},\phi_{V_{j}}\rangle\bigr|
\le \|\alpha_{x}\|\,\|\phi_{V}-\phi_{V_{j}}\|
\le C_{n,p}\,\varepsilon\,\|\alpha_{x}\|.
\]
In particular,
\[
0\le M(\alpha_{x})-M_{\varepsilon}(\alpha_{x})\le C_{n,p}\,\varepsilon\,\|\alpha_{x}\|,
\]
so the $\varepsilon$--net yields a quantitative \emph{discretization} of the support functional
$V\mapsto \langle \alpha_{x},\phi_{V}\rangle_{+}$ used in \eqref{eq:ray-defect-formula}.
No equivalence between $\distcone(\alpha_{x})=\dist(\alpha_{x},\mathcal{C}_{x})$ and the distance to a finite-dimensional
linear span is asserted or needed anywhere in the proof.}
\end{editamirblockNEW}

% ----------------------------------------------------------
% Ray distance vs. convex calibrated cone
% ----------------------------------------------------------

\subsection*{Ray distance vs.\ convex calibrated cone}

For a calibrated simple form $\phi_{V}$ and any real $2p$--form 
$\alpha_{x}\in \Lambda^{2p}T^{*}_{x}X$, consider the ray generated by $\phi_{V}$.
The pointwise distance from $\alpha_{x}$ to this ray is
\[
\dist\bigl(\alpha_{x}, \R_{\ge 0}\,\phi_{V}\bigr)
:=
\inf_{\lambda\ge 0} \|\alpha_{x}-\lambda\phi_{V}\|.
\]
Minimizing over all calibrated rays yields the \emph{ray defect}
\[
\Def_{\mathrm{ray}}(\alpha_{x})
:=
\inf_{V\in \Gp(x)}
\dist\!\left(
\alpha_{x},\,
\R_{\ge 0}\,\phi_{V}
\right).
\]

Since the convex calibrated cone
\[
\mathcal{C}_{x} = \cone\{\phi_{V} : V\in \Gp(x)\}
\]
contains every such ray, one always has
\[
\distcone(\alpha_{x})
\;=\;
\dist\bigl(\alpha_{x},\mathcal{C}_{x}\bigr)
\;\le\;
\Def_{\mathrm{ray}}(\alpha_{x}).
\]
\begin{editamirblockNEW}\editamir{
\noindent\textbf{Remark (no cone--to--span distortion needed).}
The only general relationship used later is the trivial inclusion
$\R_{\ge 0}\phi_{V}\subset \mathcal{C}_{x}$ for each $V\in\Gp(x)$, which gives
\[
\dist\bigl(\alpha_{x},\mathcal{C}_{x}\bigr)\;\le\;\Def_{\mathrm{ray}}(\alpha_{x}).
\]
We do \emph{not} use (and do not claim) any converse estimate comparing
$\dist(\alpha_{x},\mathcal{C}_{x})$ to the distance from $\alpha_{x}$ to the
linear span of a finite net.  When a finite net is needed, it is used only
to discretize the support functional in \eqref{eq:ray-defect-formula}, as explained above.}
\end{editamirblockNEW}

% ----------------------------------------------------------
% Radial minimization along a calibrated ray
% ----------------------------------------------------------


\iffalse
% --- [DEPRECATED: this lemma conflated the ray distance with the convex cone distance] ---
\begin{lemma}[Explicit minimization in the radial parameter]
	\label{lem:radial-min-deprecated}
	Fix a point $x \in X$ and a calibrated unit covector
	$\xi \in \Gp(x)$.
	For any real $2p$--form $\alpha_{x} \in \Lambda^{2p}T^{*}_{x}X$, the map
	\[
	\lambda \;\longmapsto\; \|\alpha_{x} - \lambda \xi\|^{2},
	\qquad \lambda \ge 0,
	\]
	is minimized at
	\[
	\lambda^{*} \;=\; \max\{0, \langle \alpha_{x}, \xi \rangle\}.
	\]
	Moreover,
	\[
	\min_{\lambda \ge 0} \|\alpha_{x} - \lambda \xi\|^{2}
	\;=\;
	\|\alpha_{x}\|^{2}
	\;-\;
	\bigl(\langle \alpha_{x}, \xi \rangle_{+}\bigr)^{2},
	\]
	where
	\[
	\langle u, v \rangle_{+}
	\;:=\;
	\max\{0, \langle u, v \rangle\}.
	\]
	Consequently,
	\begin{equation}\label{eq:ray-defect-formula-deprecated}
		\distcone(\alpha_{x})^{2}
		\;=\;
		\|\alpha_{x}\|^{2}
		\;-\;
		\Bigl(
		\max_{\xi \in \Gp(x)}
		\langle \alpha_{x}, \xi \rangle_{+}
		\Bigr)^{2}.
	\end{equation}
\end{lemma}\fi

\begin{editamirblockNEW}
\begin{lemma}[Explicit minimization along a calibrated ray]
\label{lem:radial-min}
Fix $x\in X$ and write $\mathcal{C}_{x}:=\mathrm{cone}\{\phi_{V}:V\in\Gp(x)\}\subset \Lambda^{2p}T_x^*X$.
Define the \emph{ray defect} of a real $2p$--form $\alpha_{x}$ by
\[
\Def_{\mathrm{ray}}(\alpha_{x})
\;:=\;
\inf_{V\in\Gp(x)} \dist\bigl(\alpha_{x},\R_{\ge 0}\,\phi_{V}\bigr)
\;=\;
\inf_{V\in\Gp(x)}\ \inf_{\lambda\ge 0}\ \|\alpha_{x}-\lambda\,\phi_{V}\|.
\]
Then for each fixed $V\in\Gp(x)$, the inner minimization is attained at
$\lambda^{*}=\langle \alpha_{x},\phi_{V}\rangle_{+}:=\max\{0,\langle \alpha_{x},\phi_{V}\rangle\}$, and
\begin{equation}\label{eq:ray-defect-formula}
\Def_{\mathrm{ray}}(\alpha_{x})^{2}
\;=\;
\|\alpha_{x}\|^{2}
-
\Bigl(
\max_{V\in \Gp(x)} \langle \alpha_{x},\phi_{V}\rangle_{+}
\Bigr)^{2}.
\end{equation}
Moreover, since $\R_{\ge 0}\phi_{V}\subset \mathcal{C}_{x}$ for every $V$,
one always has the elementary comparison
\[
\distcone(\alpha_{x})=\dist(\alpha_{x},\mathcal{C}_{x})
\;\le\;
\Def_{\mathrm{ray}}(\alpha_{x}).
\]
\end{lemma}

\begin{proof}
Fix $V$ and consider $f(\lambda):=\|\alpha_{x}-\lambda\phi_{V}\|^{2}
=\|\alpha_{x}\|^{2}-2\lambda\langle \alpha_{x},\phi_{V}\rangle+\lambda^{2}$ for $\lambda\ge 0$.
This is a convex quadratic with unconstrained minimizer at $\lambda=\langle \alpha_{x},\phi_{V}\rangle$.
Imposing $\lambda\ge 0$ yields $\lambda^{*}=\langle \alpha_{x},\phi_{V}\rangle_{+}$ and
\[
\min_{\lambda\ge 0}\|\alpha_{x}-\lambda\phi_{V}\|^{2}
=\|\alpha_{x}\|^{2}-\langle \alpha_{x},\phi_{V}\rangle_{+}^{2}.
\]
Taking the infimum over $V\in\Gp(x)$ gives \eqref{eq:ray-defect-formula}.
The final inequality follows from $\bigcup_{V\in\Gp(x)}\R_{\ge 0}\phi_{V}\subset\mathcal{C}_{x}$.
\end{proof}
\end{editamirblockNEW}

% ----------------------------------------------------------
% Trace L^2 control (used later with Hermitian model)
% ----------------------------------------------------------

\begin{lemma}[Trace $L^{2}$ control]\label{lem:trace-L2}
	Let $\eta$ be the Coulomb potential with $d^{*}\eta = 0$ and
	\[
	\alpha = \gharm + d\eta.
	\]
	Define
	\[
	\beta := (d\eta)^{(p,p)},
	\]
	and let
	\[
	H_{\beta}(x) := \mathcal{I}(\beta_{x}) \in \Herm\bigl(\Lambda^{p,0}_{x}X\bigr),
	\]
	where $d := \dim_{\C}\Lambda^{p,0}_{x}X = \binom{n}{p}$ and
	$\mathcal{I}$ is any fixed isometric identification between
	$\Lambda^{p,p}_{x}T^{*}X$ and $\Herm(\Lambda^{p,0}_{x}X)$.
	Set
	\[
	\mu(x) := \frac{1}{d}\,\tr H_{\beta}(x).
	\]
	Then
	\begin{equation}\label{eq:trace-L2-bound}
		\|\mu\|_{L^{2}}
		\;\le\;
		C_{\Lambda}(n,p)\,\|d\eta\|_{L^{2}},
		\qquad
		C_{\Lambda}(n,p) = d^{-1/2}.
	\end{equation}
\end{lemma}

\begin{proof}
	Pointwise at each $x\in X$, apply Cauchy--Schwarz for the Hilbert--Schmidt
	inner product on $\Herm(\Lambda^{p,0}_{x}X)$:
	\[
	\bigl|\tr H_{\beta}(x)\bigr|
	\;\le\;
	\sqrt{d}\,\|H_{\beta}(x)\|_{\HS}.
	\]
	Hence
	\[
	|\mu(x)|
	= \frac{1}{d}\,\bigl|\tr H_{\beta}(x)\bigr|
	\;\le\;
	d^{-1/2}\,\|H_{\beta}(x)\|_{\HS}.
	\]
	By construction, the identification
	\[
	\mathcal{I} : \Lambda^{p,p}_{x}T^{*}X \longrightarrow \Herm(\Lambda^{p,0}_{x}X)
	\]
	is an isometry with respect to the pointwise norms, so
	\[
	\|H_{\beta}(x)\|_{\HS}
	= \|\beta(x)\|.
	\]
	Moreover, since $\beta$ is the $(p,p)$--component of $d\eta$ and the
	$(r,s)$--components are orthogonal in the Kähler metric, we have the
	pointwise inequality
	\[
	\|\beta(x)\| \;\le\; \|d\eta(x)\|.
	\]
	Combining these estimates gives
	\[
	|\mu(x)|
	\;\le\;
	d^{-1/2}\,\|d\eta(x)\|
	\quad\text{for all } x\in X.
	\]
	Squaring and integrating over $X$ yields
	\[
	\|\mu\|_{L^{2}}
	\;\le\;
	d^{-1/2}\,\|d\eta\|_{L^{2}},
	\]
	which is exactly \eqref{eq:trace-L2-bound}.
\end{proof}

% ----------------------------------------------------------
% Basic properties of the calibration distance
% ----------------------------------------------------------


\iffalse
% --- [DEPRECATED: this proposition used the ray-defect formula as if it were the convex cone distance] ---
\begin{proposition}[Well-posedness and basic properties]
	\label{prop:dist-cal-properties-deprecated}
	For each point $x \in X$ and each real $2p$--form 
	$\alpha_{x} \in \Lambda^{2p}T^{*}_{x}X$, the calibration distance
	$\distcone(\alpha_{x})$ enjoys the following properties.
	\begin{enumerate}
		\item[\textnormal{(1)}] \textbf{Compactness and attainment.}
		The calibrated Grassmannian $\Gp(x)$ is compact.
		Consequently, the maximum in \eqref{eq:ray-defect-formula} is attained,
		and the infimum in the definition of $\distcone(\alpha_{x})$ is in fact a
		minimum.
		
		\item[\textnormal{(2)}] \textbf{Positive homogeneity and Lipschitz continuity.}
		For every scalar $t \ge 0$,
		\[
		\distcone(t\alpha_{x})
		\;=\;
		t\,\distcone(\alpha_{x}).
		\]
		Moreover, for all real $2p$--forms $\alpha_{x},\beta_{x}$ one has
		\[
		\bigl|
		\distcone(\alpha_{x})
		-
		\distcone(\beta_{x})
		\bigr|
		\;\le\;
		\|\alpha_{x} - \beta_{x}\|.
		\]
		
		\item[\textnormal{(3)}] \textbf{Measurability and regularity in $x$.}
		If $\alpha$ is a measurable $2p$--form on $X$, then the map
		\[
		x \longmapsto \distcone(\alpha_{x})
		\]
		is measurable.  
		If $\alpha$ is continuous (respectively smooth), then
		$x \mapsto \distcone(\alpha_{x})$ is continuous
		(respectively smooth away from the locus where the maximizing
		calibrated direction in \eqref{eq:ray-defect-formula} changes).
		
		\item[\textnormal{(4)}] \textbf{Zero-defect characterization.}
		One has $\distcone(\alpha_{x}) = 0$ if and only if
		$\alpha_{x}$ belongs to a calibrated ray, i.e.
		\[
		\alpha_{x} \in \R_{\ge 0}\cdot \Gp(x).
		\]
	\end{enumerate}
\end{proposition}\fi

\begin{editamirblockNEW}
\begin{proposition}[Well-posedness and basic properties]
\label{prop:dist-cal-properties}
Fix $x\in X$ and consider the calibrated cone
$\mathcal{C}_{x}:=\mathrm{cone}\{\phi_{V}:V\in\Gp(x)\}\subset \Lambda^{2p}T_x^*X$,
which is a closed convex cone in the Euclidean space $(\Lambda^{2p}T_x^*X,\langle\cdot,\cdot\rangle)$.
Define $\distcone(\alpha_{x}):=\dist(\alpha_{x},\mathcal{C}_{x})$.
Then:
\begin{enumerate}
\item[\textnormal{(1)}] \textbf{Existence/uniqueness of projection.}
There exists a unique nearest point $\Pi_{\mathcal{C}_{x}}(\alpha_{x})\in\mathcal{C}_{x}$ such that
\[
\distcone(\alpha_{x})=\|\alpha_{x}-\Pi_{\mathcal{C}_{x}}(\alpha_{x})\|.
\]
\item[\textnormal{(2)}] \textbf{Positive homogeneity and $1$--Lipschitz continuity.}
For every $t\ge 0$,
$\distcone(t\alpha_{x})=t\,\distcone(\alpha_{x})$,
and for all $\alpha_{x},\beta_{x}$,
\[
\bigl|\distcone(\alpha_{x})-\distcone(\beta_{x})\bigr|\le \|\alpha_{x}-\beta_{x}\|.
\]
\item[\textnormal{(3)}] \textbf{Dependence on $x$.}
If $\alpha$ is measurable, then $x\mapsto \distcone(\alpha_{x})$ is measurable.
Moreover, in a local unitary trivialization of $TX$ (identifying $T_xX\simeq\C^{n}$),
the cone $\mathcal{C}_{x}$ identifies with a fixed model cone; hence if $\alpha$ is continuous
(respectively smooth) then $x\mapsto \distcone(\alpha_{x})$ is continuous (respectively smooth).
\item[\textnormal{(4)}] \textbf{Zero characterization.}
One has $\distcone(\alpha_{x})=0$ if and only if $\alpha_{x}\in\mathcal{C}_{x}$.
In contrast, the \emph{ray defect} $\Def_{\mathrm{ray}}$ vanishes if and only if
$\alpha_{x}\in \R_{\ge 0}\phi_{V}$ for some $V\in\Gp(x)$ (equivalently, the maximum in
\eqref{eq:ray-defect-formula} equals $\|\alpha_{x}\|$).
\end{enumerate}
\end{proposition}

\begin{proof}
Items (1)--(2) are standard facts for closed convex sets in finite-dimensional Hilbert spaces
(existence/uniqueness of the metric projection and $1$--Lipschitz property of the distance function).
Item (3) follows because in unitary coordinates the cone depends only on $(n,p)$, and $\dist(\cdot,\mathcal{C})$
is continuous (indeed, $1$--Lipschitz) in its argument.  Item (4) is immediate from the definition of distance.
\end{proof}
\end{editamirblockNEW}

% ----------------------------------------------------------
% Optional: Kähler-angle parametrization (for intuition)
% ----------------------------------------------------------

\subsection*{Optional: K\"ahler-angle parametrization (for intuition)}

Let $x \in X$ and let $V,V' \in \Gp(x)$ be complex $p$--planes.
The relative position of $(V,V')$ is encoded by their $p$ Kähler angles
$\theta_{1},\ldots,\theta_{p} \in [0,\tfrac{\pi}{2})$, the canonical angles
arising from the $U(n)$--invariant geometry of the Grassmannian.
In an adapted unitary frame one has the classical identity
\[
\langle \phi_{V},\phi_{V'} \rangle
= \prod_{j=1}^{p} \cos\theta_{j},
\]
where $\phi_{V}$ and $\phi_{V'}$ denote the associated unit calibrated
simple $(p,p)$--forms.

For small angles, the expansion
\[
\cos\theta
= 1 - \tfrac{1}{2}\theta^{2} + \tfrac{1}{24}\theta^{4}
+ O(\theta^{6})
\]
provides a second--order approximation of the inner product in terms of
$\sum_{j} \sin^{2}\theta_{j}$.  This relation between calibrated directions
and the Kähler angles yields the following quadratic control estimate.


\begin{lemma}[Quadratic control for small Jordan angles (principal angles)]
	\label{lem:kahler-angle}
	Let $V,V' \in \Gp(x)$ have Kähler angles
	$\theta_{1},\ldots,\theta_{p}$ satisfying
	\[
	\sum_{j=1}^{p} \theta_{j}^{2} \;\le\; 10^{-2}.
	\]
	Then the corresponding calibrated unit covectors $\phi_{V}$ and $\phi_{V'}$
	satisfy the estimate
	\begin{equation}\label{eq:kahler-angle-est}
		0.25\sum_{j=1}^{p} \sin^{2}\theta_{j}
		\;\le\;
		1 - \langle \phi_{V}, \phi_{V'} \rangle
		\;\le\;
		0.51\sum_{j=1}^{p} \sin^{2}\theta_{j}.
	\end{equation}
\end{lemma}

\begin{proof}
	Using the standard principal-angle identity
	\(
	\langle \phi_{V},\phi_{V'}\rangle=\prod_{j=1}^{p}\cos\theta_{j},
	\)
	it suffices to control $1-\prod_j\cos\theta_j$.
	For $0\le\theta\le 0.1$ one has
	\[
	1-\cos\theta \;=\; 2\sin^2(\theta/2)
	\;\ge\; \tfrac12\,\sin^2\theta,
	\]
	and also, since $\cos(\theta/2)\ge \cos(0.05)$ on this range,
	\[
	1-\cos\theta \;=\; \frac{\sin^2\theta}{2\cos^2(\theta/2)}
	\;\le\; \frac{1}{2\cos^2(0.05)}\,\sin^2\theta
	\;\le\; 0.51\,\sin^2\theta.
	\]
	Let $a_j:=1-\cos\theta_j\ge 0$.  Since $\sum_j\theta_j^2\le 10^{-2}$, we have $0\le \theta_j\le 0.1$ and hence
	$\sum_j a_j \le 0.51\sum_j\sin^2\theta_j\le 0.51\cdot 10^{-2}<1$.
	Now
	\[
	1-\prod_{j=1}^p\cos\theta_j
	\;=\;1-\prod_{j=1}^p(1-a_j)
	\;\le\;\sum_{j=1}^p a_j
	\;\le\;0.51\sum_{j=1}^p\sin^2\theta_j.
	\]
	For the lower bound, use $\prod_j(1-a_j)\le e^{-\sum_j a_j}$ to get
	\[
	1-\prod_{j=1}^p\cos\theta_j
	\;=\;1-\prod_{j=1}^p(1-a_j)
	\;\ge\;1-e^{-\sum_j a_j}
	\;\ge\;\tfrac12\sum_{j=1}^p a_j
	\;\ge\;0.25\sum_{j=1}^p \sin^2\theta_j,
	\]
	using $1-e^{-t}\ge t/2$ for $t\in[0,1]$ and $a_j\ge \tfrac12\sin^2\theta_j$.
\end{proof}


\begin{remark}[Geometric meaning of Lemma~\ref{lem:kahler-angle}]
	Lemma~\ref{lem:kahler-angle} shows that, when the Kähler angles between two
	complex $p$--planes are small, the deviation of their calibrated directions is
	quadratically controlled by the sum of the squared angles.  Since
	$\langle\phi_{V},\phi_{V'}\rangle = \prod_{j=1}^{p}\cos\theta_{j}$, the
	quantity
	\[
	1 - \langle \phi_{V},\phi_{V'}\rangle
	\]
	measures the pointwise misalignment between the two calibrated simple
	$(p,p)$--forms.  Lemma~\ref{lem:kahler-angle} asserts that this misalignment is
	comparable, up to uniform constants, to the elementary quadratic quantity
	$\sum_{j=1}^{p}\sin^{2}\theta_{j}$ whenever $\sum \theta_{j}^{2}$ is suitably
	small.  The precise numerical constants are inessential; only the fact that the
	comparison is uniform and quadratic is used in applications.
\end{remark}

	% ============================================================
%                    SECTION 4
% ============================================================

\section{Energy Gap and Primitive/Off--Type Controls}
\label{sec:energy-gap}

\begin{editamirblock}
Let $(X,\omega)$ be a compact K\"ahler manifold of complex dimension $n$.
Fix a real Hodge class
\[
[\alpha]\in H^{2p}(X,\RR)\cap H^{p,p}(X),
\]
and let $\alpha$ be a smooth real closed $2p$--form representing $[\alpha]$.
\end{editamirblock}

\begin{editconeblock}
The purpose of this section is to record standard K\"ahler/Hodge estimates controlling
off--type components and the primitive part of a closed form in terms of the energy of its Coulomb potential.
These estimates provide analytic background for optional “coercivity’’ discussions; they are not used in the main realization/SYR chain.
\end{editconeblock}

\subsection*{Coulomb potential}
Fix a representative $\alpha$ of $[\alpha]$.  Since $d\alpha = 0$, the elliptic
equation
\[
d^{*}d\eta = d^{*}\alpha
\]
admits a unique solution $\eta$ orthogonal to $\ker d$, giving the Hodge
decomposition
\[
\alpha
= \gamma_{\harm} + d\eta,
\]
where $\gamma_{\harm}$ is the unique harmonic representative of $[\alpha]$.
We define the energy of $\alpha$ by
\[
E(\alpha) := \|d\eta\|^{2}_{L^{2}}.
\]

\begin{editamirblock}
\subsection*{Energy identity and type decomposition}
We recall a standard Hodge--theoretic fact: on a compact K\"ahler manifold, the space of harmonic forms decomposes into harmonic \((r,s)\) types, and the harmonic representative of a cohomology class in \(H^{r,s}(X)\) is of type \((r,s)\) (see, e.g., Wells \cite[Ch.~5]{Wells}).
In particular, since \( [\alpha]\in H^{p,p}(X)\), the harmonic representative \(\gamma_{\harm}\) of \([\alpha]\) has pure type \((p,p)\).

\smallskip
Fix a representative \(\alpha\) of \([\alpha]\).  Since \(d\alpha=0\), the elliptic equation
\[
d^{*}d\eta = d^{*}\alpha
\]
admits a unique solution \(\eta\) orthogonal to \(\ker d\) (equivalently, orthogonal to harmonic \((2p-1)\)-forms), giving the Hodge decomposition
\[
\alpha = \gamma_{\harm} + d\eta,
\qquad d^*\eta=0.
\]
We define the energy of \(\alpha\) by
\[
E(\alpha):=\|d\eta\|_{L^{2}}^{2}.
\]

\subsection*{Energy split}
Since \(\gamma_{\harm}\perp d\eta\) in \(L^2\), we have
\begin{equation}\label{eq:energy-split}
	E(\alpha)
	= \|d\eta\|_{L^{2}}^{2}
	= \|\alpha\|_{L^{2}}^{2} - \|\gamma_{\harm}\|_{L^{2}}^{2}.
	\tag{11}
\end{equation}

\subsection*{Type split}
Decompose \(\alpha\) into types \(\alpha=\sum_{r+s=2p}\alpha^{(r,s)}\).
Because \(\gamma_{\harm}\) has type \((p,p)\), all off--type components of \(\alpha\) are exact and belong to \(d\eta\):
\[
\alpha^{(r,s)}=(d\eta)^{(r,s)}
\qquad\text{for all }(r,s)\neq(p,p).
\]
Orthogonality of distinct types yields
\begin{equation}\label{eq:type-split}
	\|\alpha-\gamma_{\harm}\|_{L^{2}}^{2}
	=
	\sum_{\substack{r+s=2p\\(r,s)\neq(p,p)}}\|\alpha^{(r,s)}\|_{L^{2}}^{2}
	+\|(\alpha^{(p,p)}-\gamma_{\harm})\|_{L^{2}}^{2}.
	\tag{12}
\end{equation}

\subsection*{Primitive/off--type control}
Let \((\cdot)_{\prim}\) denote the \(L^2\)-orthogonal projection onto \(\omega\)-primitive \((p,p)\)-forms.
Elliptic control on the Coulomb slice gives a uniform bound (depending only on \(X,\omega,p\)):
\begin{equation}\label{eq:primitive-control}
	\sum_{\substack{r+s=2p\\(r,s)\neq(p,p)}}\|\alpha^{(r,s)}\|_{L^{2}}
	+\|(\alpha^{(p,p)}-\gamma_{\harm})_{\prim}\|_{L^{2}}
	\;\le\;
	C(X,\omega,p)\,\|d\eta\|_{L^{2}}.
	\tag{13}
\end{equation}

\smallskip
\noindent\emph{Remark.} If \([\alpha]\) has a nonzero harmonic off--type component, then no estimate of the form \eqref{eq:primitive-control} can hold with right--hand side \(\|d\eta\|_{L^2}\), since harmonic components are invisible to the Coulomb energy.
\end{editamirblock}

\begin{editamirblock}
\begin{lemma}[Elliptic estimate on the Coulomb slice]\label{lem:elliptic-coulomb}
Let $\eta$ be a smooth $(2p-1)$--form on a compact K\"ahler manifold with $d^*\eta=0$ and $\eta\perp \ker d$.
Then there exists a constant $C=C(X,\omega,p)$ such that
\[
\|\eta\|_{H^1}\ \le\ C\,\|d\eta\|_{L^2}.
\]
In particular, the $L^2$ norms of all first-order type components $\partial\eta^{(r,s)}$ and $\bar\partial\eta^{(r,s)}$ are bounded by $C\,\|d\eta\|_{L^2}$.
\end{lemma}

\begin{proof}
This is a standard elliptic estimate for the Hodge operator $d+d^*$ (equivalently for the Laplacian) on the Coulomb slice $d^*\eta=0$, restricted to the orthogonal complement of harmonic forms.
One convenient formulation is
\[
\|\eta\|_{H^1}\ \le\ C\bigl(\|d\eta\|_{L^2}+\|d^*\eta\|_{L^2}\bigr),
\]
valid on any compact Riemannian manifold; imposing $d^*\eta=0$ gives the stated bound.
See, for example, Wells, \emph{Differential Analysis on Complex Manifolds}, Chapter~5, or any standard Hodge theory reference.
\end{proof}
\end{editamirblock}

\begin{editamirblock}
\begin{lemma}[Coulomb decomposition and energy identity]\label{lem:coulomb}
Let \([\alpha]\in H^{2p}(X,\RR)\cap H^{p,p}(X)\) and let \(\alpha\) be a smooth closed real \(2p\)–form representing \([\alpha]\).
Write \(\alpha=\gamma_{\harm}+d\eta\) for its Coulomb decomposition with \(d^*\eta=0\) and \(\eta\perp\ker d\).
Then:

\begin{enumerate}
\item
\(\displaystyle
E(\alpha)=\|d\eta\|_{L^{2}}^{2}
=\|\alpha\|_{L^{2}}^{2}-\|\gamma_{\harm}\|_{L^{2}}^{2},
\)
as in~\eqref{eq:energy-split}.

\item
The difference from the harmonic representative satisfies the orthogonal type split
\[
\|\alpha-\gamma_{\harm}\|_{L^{2}}^{2}
=
\sum_{\substack{r+s=2p\\(r,s)\neq(p,p)}}\|\alpha^{(r,s)}\|_{L^{2}}^{2}
+\|(\alpha^{(p,p)}-\gamma_{\harm})\|_{L^{2}}^{2},
\]
as in~\eqref{eq:type-split}.

\item
All off--type components and the primitive part of the \((p,p)\) component are controlled by the Coulomb energy:
\[
\sum_{\substack{r+s=2p\\(r,s)\neq(p,p)}}\|\alpha^{(r,s)}\|_{L^{2}}
+\|(\alpha^{(p,p)}-\gamma_{\harm})_{\prim}\|_{L^{2}}
\;\le\;
C(X,\omega,p)\,\sqrt{E(\alpha)},
\]
consistent with~\eqref{eq:primitive-control}.
\end{enumerate}
\end{lemma}

\begin{proof}
Item (i) is the Pythagorean identity coming from the orthogonality \(\gamma_{\harm}\perp d\eta\).
Item (ii) is orthogonality of distinct \((r,s)\) types, using that \(\gamma_{\harm}\) has pure type \((p,p)\).

For (iii), since \(\gamma_{\harm}\) has type \((p,p)\), we have \(\alpha^{(r,s)}=(d\eta)^{(r,s)}\) for all \((r,s)\neq(p,p)\).
Each such component \((d\eta)^{(r,s)}\) is a linear combination of first--order operators \(\partial\) and \(\bar\partial\) applied to type components of \(\eta\),
so
\[
\sum_{\substack{r+s=2p\\(r,s)\neq(p,p)}}\|\alpha^{(r,s)}\|_{L^{2}}
\le C\,\|\eta\|_{H^{1}}.
\]
By Lemma~\ref{lem:elliptic-coulomb}, \(\|\eta\|_{H^{1}}\le C\,\|d\eta\|_{L^{2}}=C\sqrt{E(\alpha)}\), yielding the desired bound for off--type components.
Finally, \((\alpha^{(p,p)}-\gamma_{\harm})_{\prim}\) is the \(L^{2}\)-orthogonal projection of \((d\eta)^{(p,p)}\) to primitive \((p,p)\)-forms, hence
\(\|(\alpha^{(p,p)}-\gamma_{\harm})_{\prim}\|_{L^{2}}\le \|(d\eta)^{(p,p)}\|_{L^{2}}\le \|d\eta\|_{L^{2}}\),
and the same elliptic estimate completes the proof.
\end{proof}
\end{editamirblock}



% ------------------------------------------------------------
% SECTION 5 — The Calibrated Grassmannian and an Explicit ε–Net
% ------------------------------------------------------------

\section{The Calibrated Grassmannian and an Explicit \texorpdfstring{$\varepsilon$}{epsilon}--Net}

\subsection*{Fiberwise geometry}

Fix $x\in X$ and set
\[
\varphi := \frac{\omega^{p}}{p!}.
\]
Define the calibrated Grassmannian at $x$ by
\[
G_{p}(x)
:=
\Big\{
\xi \in \Lambda^{2p}T^{*}_{x}X :
\|\xi\| = 1,\;
\xi\ \text{simple of type $(p,p)$},\;
\varphi_{x}(\xi)=1
\Big\}.
\]
This is the set of unit simple $(p,p)$ covectors saturated by the K\"ahler
calibration $\varphi_{x}$.  Equivalently, $G_{p}(x)$ is the image of the
complex Grassmannian $G_{\C}(p,n)$ under the map sending a $p$--plane
$V\subset T^{1,0}_{x}X$ to its associated calibrated covector $\phi_{V}$.
With the metric induced by $\omega$, this map is an isometric embedding
(up to normalization), and therefore
\[
G_{p}(x) \cong G_{\C}(p,n)
\]
with its standard Fubini--Study metric.

\begin{editamirblock}
\noindent\textbf{Referee clarification (fiber model and metric).}
For the arguments below we only use that each fiber $G_{p}(x)$ is a \emph{fixed} compact homogeneous manifold
(isomorphic to $G_{\C}(p,n)$) and that $d_{\mathrm{FS}}$ denotes any fixed $U(n)$--invariant Riemannian distance on that model.
Different normalizations of the invariant metric are bi--Lipschitz equivalent with constants depending only on $(n,p)$,
so all covering/packing constants in Lemma~\ref{lem:covering-number} may be taken uniform in $x$.
\end{editamirblock}

In particular, $G_{p}(x)$ is
compact, smooth, homogeneous, and has real dimension
\[
d := \dim_{\R} G_{p}(x)
= 2p(n-p).
\]

\subsection*{$\varepsilon$–nets and covering estimates}

Fix $\varepsilon = \tfrac{1}{10}$.  

\begin{editamirblock}
\noindent\textbf{Referee note (choice of the net).}
The maximal $\varepsilon$--separated set $\{\xi(x)_{\ell}\}_{\ell}$ is chosen \emph{independently for each} $x\in X$.
Only the uniform cardinality bound \eqref{eq:grass-cover} is used later; no global smooth/measurable dependence of the selection on $x$
is required in the main proof chain.
\end{editamirblock}


On each fiber $G_{p}(x)$ (with the Fubini--Study geodesic distance
$d_{\mathrm{FS}}$), choose a maximal $\varepsilon$–separated set
\[
\{\xi(x)_\ell\}_{\ell=1}^{N(x)}
\subset G_{p}(x),
\qquad
d_{\mathrm{FS}}(\xi(x)_\ell,\xi(x)_m) \ge \varepsilon
\ \text{for all }\ell\ne m,
\]
such that no additional point of $G_{p}(x)$ can be added while preserving
this separation property.

By compactness and the standard packing principle on compact homogeneous
spaces, such maximal $\varepsilon$–separated sets are automatically
$\varepsilon$–nets: for every $\xi \in G_{p}(x)$ there exists an index
$\ell$ with  
\[
d_{\mathrm{FS}}(\xi,\xi(x)_\ell) \le \varepsilon.
\]

\begin{lemma}[Covering number]\label{lem:covering-number}
	Let $d = 2p(n-p)$.  
	There exists a constant $C(n,p)$ depending only on $(n,p)$ such that every
	maximal $\varepsilon$–separated set in $G_{p}(x)$ satisfies
	\begin{equation}\label{eq:grass-cover}
		N(x) \;\le\; C(n,p)\,\varepsilon^{-d}.
		\tag{5.1}
	\end{equation}
\end{lemma}

\begin{proof}
	Cover $G_{p}(x)$ by the geodesic balls
	\[
	B\!\left(\xi(x)_\ell,\,\tfrac{\varepsilon}{2}\right),
	\qquad \ell=1,\dots,N(x),
	\]
	of radius $\varepsilon/2$ in the Fubini--Study metric.  
	Because the points are $\varepsilon$–separated, these balls are pairwise
	disjoint.  By maximality of the separated set, the $\varepsilon$–balls
	\[
	B\!\left(\xi(x)_\ell,\,\varepsilon\right)
	\]
	cover $G_{p}(x)$.
	
	Since $G_{p}(x)$ is a compact homogeneous space, the volume of a small
	geodesic ball depends only on the radius, not on its center.  
	Let $V(r)$ denote the volume of a geodesic ball of radius $r$.  
	Then disjointness gives
	\[
	N(x)\,V(\varepsilon/2)
	\;\le\; \Vol\bigl(G_{p}(x)\bigr),
	\]
	while the covering property yields
	\[
	\Vol\bigl(G_{p}(x)\bigr)
	\;\le\; N(x)\,V(\varepsilon).
	\]
	
	For small $r$ one has the uniform expansion
	\[
	V(r) = c_{d}\,r^{d} + O(r^{d+2}),
	\]
	with $c_{d}>0$ depending only on $d = \dim_{\R} G_{p}(x)$.  
	Since $G_{p}(x)$ is homogeneous, there exist constants $A(n,p)$ and $B(n,p)$
	such that
	\[
	A(n,p)\,r^{d} \le V(r) \le B(n,p)\,r^{d}
	\qquad\text{for } 0<r\le 1.
	\]
	
	Combining the two volume inequalities gives
	\[
	N(x)\,A(n,p)\,(\varepsilon/2)^{d}
	\;\le\; \Vol\bigl(G_{p}(x)\bigr)
	\;\le\; N(x)\,B(n,p)\,\varepsilon^{d},
	\]
	so cancelling $\Vol(G_{p}(x))$ yields
	\[
	N(x) \;\le\;
	\frac{B(n,p)}{A(n,p)}\,(2^{d})\,
	\varepsilon^{-d}.
	\]
	
	Absorbing the constants into
	\[
	C(n,p) := \frac{B(n,p)}{A(n,p)}\,2^{d},
	\]
	we obtain the desired estimate \eqref{eq:grass-cover}.
\end{proof}
% ============================================================
% SECTION 6 — Pointwise Linear Algebra: Controlling the Net Distance
% ============================================================

\section{Pointwise Linear Algebra: Controlling the Net Distance}
\label{sec:linear-algebra}

\begin{editconeblock}
\noindent\textbf{Nonessential background (Hermitian/PSD context).}
This section records optional quantitative linear-algebra estimates (nets, Hermitian models, and the PSD-vs-calibrated-cone distinction) for context and comparison.
It is \emph{not} used in the main realization/SYR $\,+\,$ signed-decomposition chain leading to Theorem~\ref{thm:main-hodge}.
\end{editconeblock}

In this section we develop the pointwise linear--algebraic estimates
that control the distance of a real $2p$--form to the calibrated
span generated by the $\varepsilon$--net constructed in Section~5.
The goal is to show that the net distance (and therefore the cone
distance) is controlled by two quantities:

\begin{editamirblockNEW}
\begin{itemize}
\item the full off--type component $\alpha_x^{\mathrm{off}}:=\sum_{\substack{r+s=2p\\(r,s)\neq(p,p)}}\alpha_x^{(r,s)}$;
\item the $(p,p)$--ray distance to the net, i.e.\ $\min_{1\le \ell\le N(x),\,\lambda\ge 0}\|\alpha_x^{(p,p)}-\lambda\,\xi_\ell(x)\|$.
\end{itemize}
\end{editamirblockNEW}

\begin{editconeblock}
These pointwise inequalities are recorded as optional linear-algebra background (nets/Hermitian models).
They are not used in the main realization/SYR chain.
\end{editconeblock}

% ------------------------------------------------------------
\subsection*{Calibrated span}

Fix $x\in X$ and let 
\[
\{\xi_{\ell}(x)\}_{\ell=1}^{N(x)} \subset G_{p}(x)
\]
be the $\varepsilon$--net of Section~5, with $\varepsilon=\tfrac{1}{10}$.
Define the calibrated span at $x$ by
\[
\Xi_{x}:=
\Span\{\xi_{\ell}(x):1\le \ell \le N(x)\}
\subset \Lambda^{p,p}T_{x}^{*}X.
\]

Each $\xi_{\ell}(x)$ is a unit simple $(p,p)$--covector, hence lies
entirely in the $(p,p)$--subspace of $\Lambda^{2p}T_{x}^{*}X$ and is
orthogonal to all off--type $(p+1,p-1)$ and $(p-1,p+1)$ components
with respect to the K\"ahler metric.

Thus every $\alpha_{x}\in\Lambda^{2p}T_{x}^{*}X$ admits an
orthogonal type decomposition
\begin{editamirblockNEW}
\begin{equation}\label{eq:typesplit-orth}
\alpha_{x}
=
\alpha_{x}^{(p,p)}
\;+\;
\alpha_{x}^{\mathrm{off}},
\qquad
\alpha_{x}^{\mathrm{off}}
:=
\sum_{\substack{r+s=2p\\(r,s)\neq(p,p)}}\alpha_{x}^{(r,s)}.
\tag{21}
\end{equation}
\end{editamirblockNEW}

% ------------------------------------------------------------
\subsection*{Pointwise net distance}

Define the pointwise net distance
\[
D_{\mathrm{net}}(\alpha_{x})
:=
\min_{\ell,\;\lambda\ge 0}
\|\alpha_{x} - \lambda\xi_{\ell}(x)\|.
\]

\begin{lemma}[Off--type separation for $D_{\mathrm{net}}$]\label{lem:typesplit}
\begin{editamirblockNEW}
For every $x$ and every $\alpha_{x}\in\Lambda^{2p}T^{*}_{x}X$,
\begin{equation}\label{eq:Dnet-typesplit}
D_{\mathrm{net}}(\alpha_{x})^{2}
=
\|\alpha_{x}^{\mathrm{off}}\|^{2}
+
\min_{1\le \ell\le N(x),\,\lambda\ge 0}
\bigl\|\alpha_{x}^{(p,p)} - \lambda \,\xi_{\ell}(x)\bigr\|^{2}.
\tag{22}
\end{equation}
\end{editamirblockNEW}
\end{lemma}

\begin{proof}
\begin{editamirblockNEW}
Since each $\xi_\ell(x)\in\Lambda^{p,p}T_x^*X$, the orthogonal splitting
\eqref{eq:typesplit-orth} implies that for every $\lambda\ge 0$,
\[
\|\alpha_x-\lambda\xi_\ell(x)\|^2
=
\|\alpha_x^{\mathrm{off}}\|^2
+
\|\alpha_x^{(p,p)}-\lambda\xi_\ell(x)\|^2.
\]
Taking $\inf_{\lambda\ge 0}$ and then $\min_{1\le \ell\le N(x)}$ gives \eqref{eq:Dnet-typesplit}.
\end{editamirblockNEW}
\end{proof}

% ------------------------------------------------------------
\subsection*{Projection estimate}

We now show that the $(p,p)$--term in \eqref{eq:Dnet-typesplit}
is controlled by a purely $(p,p)$ quantity arising from the Hermitian
model for $(p,p)$--forms and a rank--one approximation inequality.

\begin{lemma}[Hermitian model for $(p,p)$]\label{lem:hermitian-model}
	Fix $x$ and identify $\Lambda^{p,0}T_x^{*}X$ with a Hermitian space 
	$\bigl(\mathcal{H},\langle\cdot,\cdot\rangle\bigr)$ of complex dimension 
	$d=\binom{n}{p}$.  
	There is an isometric isomorphism
	\[
	\mathcal{I} : \Lambda^{p,p}T_x^{*}X \;\longrightarrow\; \Herm(\mathcal{H})
	\]
	(with Hilbert--Schmidt norm on the right) such that:
	\begin{enumerate}
		\item for $\alpha_x^{(p,p)}\in\Lambda^{p,p}$, the matrix 
		$H_\alpha := \mathcal{I}(\alpha_x^{(p,p)})$ is Hermitian;
		
		\item for any unit decomposable $p$--vector $v\in\Lambda^{p,0}$,  
		the calibrated covector $\xi_v$ satisfies
		\[
		\mathcal{I}(\xi_v) = P_v := v\otimes v^{*}
		\]
		(the rank--one projector);
		
		\item the contraction (trace) corresponds to the Lefschetz trace:  
		there exists $\mu(\alpha_x)\in\R$ such that
		\[
		\mathcal{I}\bigl( (\alpha_x^{(p,p)})_{\mathrm{prim}} \bigr)
		=
		H_\alpha - \mu(\alpha_x)\, I_{\mathcal{H}},
		\qquad
		\mu(\alpha_x) = \frac{1}{d}\operatorname{tr}(H_\alpha).
		\]
	\end{enumerate}
\begin{editamirblock}
\begin{proof}
Fix unitary coordinates at $x$ and let $\mathcal H:=\Lambda^{p,0}T_x^*X$ with the induced Hermitian inner product.
Given a real $(p,p)$--form $\beta\in\Lambda^{p,p}T_x^*X$, define $H_\beta\in\Herm(\mathcal H)$ by
\[
\langle H_\beta u, v\rangle\ :=\ {\color{blue}(-i)^{-p}}\,\beta(u\wedge \overline{v}),
\qquad u,v\in\mathcal H.
\]
{\color{blue}\emph{Convention:} \(\beta(u\wedge\overline v)\) denotes the coefficient/tensor contraction in a unitary coframe (equivalently the pointwise inner product \(\langle \beta,\,u\wedge\overline v\rangle\)).}
Linearity is immediate.  The reality and $(p,p)$--type of $\beta$ imply $H_\beta$ is Hermitian.

Choose an orthonormal basis $\{e_I\}_{|I|=p}$ of $\mathcal H$ (wedges of an orthonormal basis of $(1,0)$--forms).  In this basis,
the matrix coefficients are $ (H_\beta)_{IJ}={\color{blue}(-i)^{-p}}\,\beta(e_I\wedge\overline{e_J}) $, so
\[
\|H_\beta\|_{\mathrm{HS}}^2=\sum_{I,J} |(H_\beta)_{IJ}|^2=\sum_{I,J}|\beta(e_I\wedge\overline{e_J})|^2=\|\beta\|^2,
\]
which shows $\mathcal I:\beta\mapsto H_\beta$ is an isometry.
Surjectivity follows by reversing the construction: any Hermitian matrix $(h_{IJ})$ defines a unique real $(p,p)$--form by prescribing
its coefficients in the basis $\{e_I\wedge\overline{e_J}\}$ via $\beta(e_I\wedge\overline{e_J})={\color{blue}(-i)^p}\,h_{IJ}$.

For a unit decomposable $p$--vector $v\in\mathcal H$ define the associated simple $(p,p)$--form $\xi_v$ by
\[
\xi_v(u\wedge \overline{w})\ :=\ {\color{blue}(-i)^p}\,\langle u,v\rangle\,\langle v,w\rangle
\qquad (u,w\in\mathcal H),
\]
which is exactly the rank--one projector kernel.  By definition this gives $\mathcal I(\xi_v)=v\otimes v^*$.

Finally, under $\mathcal I$ the K\"ahler form $\omega^p/p!$ corresponds to the identity $I_{\mathcal H}$, so the Lefschetz trace component of $\beta$
corresponds to the scalar matrix component $(\operatorname{tr}H_\beta/d)\,I_{\mathcal H}$.
Thus subtracting $(\operatorname{tr}H_\beta/d)\,I_{\mathcal H}$ corresponds to the primitive (traceless) projection of $\beta$.
\end{proof}
\end{editamirblock}
\end{lemma}

\begin{editamirblock}
\begin{remark}[Calibrated cone in the Hermitian model; not the full PSD cone for $1<p<n-1$]\label{rem:cone-not-full-psd}
Let $\mathcal H=\Lambda^{p,0}T_x^*X$ and let $\mathcal I:\Lambda^{p,p}T_x^*X\to\Herm(\mathcal H)$ be the isometry of Lemma~\ref{lem:hermitian-model}.
Let $\mathsf{Dec}\subset \mathcal H$ denote the set of \emph{decomposable} $p$--vectors.
Then the calibrated/strongly-positive cone $K_p(x)$ satisfies
\[
\mathcal I\bigl(K_p(x)\bigr)
\;=\;
\mathrm{cone}\{\, v\otimes v^*: v\in \mathsf{Dec}\,\}
\;\subset\;
\Herm(\mathcal H)_{\succeq 0}.
\]
For $p=1$ or $p=n-1$, every $v\in\mathcal H$ is decomposable, so the right-hand side is the full PSD cone.
For $1<p<n-1$, there exist non-decomposable $w\in\mathcal H$, hence $w\otimes w^*$ is rank-one PSD but cannot lie in
$\mathrm{cone}\{v\otimes v^*: v\in\mathsf{Dec}\}$:
indeed, if $w\otimes w^*=\sum_j v_j\otimes v_j^*$ with $v_j\in\mathsf{Dec}$, then each summand has range contained in $\mathrm{span}\{w\}$ (because the left-hand side has rank one),
so every $v_j$ is collinear with $w$, forcing $w$ to be decomposable.  Thus the calibrated cone is a strict subcone of the PSD cone when $1<p<n-1$.
\end{remark}
\end{editamirblock}

\begin{lemma}[Rank--one approximation controls the traceless part]\label{lem:rankone}
	There exists a finite constant $C_{\mathrm{rank}}(d)>0$, depending only on
	$d=\dim_{\C}\mathcal{H}$, such that for every $H \in \Herm(\mathcal{H})$,
	\[
	\min_{\substack{v\in\mathcal{H},\,\|v\|=1 \\ \lambda \ge 0}}
	\|H - \lambda(v\otimes v^{*})\|_{\mathrm{HS}}^{2}
	\;\le\;
	C_{\mathrm{rank}}(d)\,\bigl\|H - \tfrac{\tr(H)}{d} I_{\mathcal{H}}\bigr\|_{\mathrm{HS}}^{2}.
	\]
\end{lemma}

\begin{proof}
	Consider the compact ``unit traceless shell''
	\[
	\mathcal{S}
	:=
	\Bigl\{H\in\Herm(\mathcal{H}) \;:\;
	\bigl\|H - \tfrac{\tr(H)}{d} I_{\mathcal{H}}\bigr\|_{\HS}=1\Bigr\}.
	\]
	The functional
	\[
	\Phi(H)
	:=
	\min_{\substack{v\in\mathcal{H},\,\|v\|=1 \\ \lambda \ge 0}}
	\|H - \lambda(v\otimes v^{*})\|_{\mathrm{HS}}^{2}
	\]
	is continuous on $\mathcal{S}$ (the minimization set is compact), hence attains a
	maximum $C_{\mathrm{rank}}(d):=\sup_{H\in\mathcal{S}}\Phi(H)<\infty$.  For general
	$H\neq 0$, scale by the traceless norm to obtain the stated inequality.
\end{proof}

\begin{proposition}[PSD surrogate for the $(p,p)$ projection]\label{prop:pp-projection}
\begin{editamirblockNEW}
There exists a constant $C_{0}=C_{0}(n,p)>0$ such that for every $x\in X$ and every
$\alpha_{x}\in \Lambda^{2p}T_x^*X$, writing $\gamma_{\harm,x}\in \Lambda^{p,p}T_x^*X$
for the fixed reference $(p,p)$--form at $x$, one has
\begin{equation}\label{eq:pp-projection}
\min_{\substack{v\in \mathcal H_x,\ \|v\|=1\\ \lambda\ge 0}}
\bigl\|\alpha_{x}^{(p,p)}-\lambda\,\xi_{v}(x)\bigr\|^{2}
\le
C_{0}\,\bigl\|(\alpha_{x}^{(p,p)}-\gamma_{\harm,x})_{\prim}\bigr\|^{2},
\tag{23}
\end{equation}
where $\xi_{v}(x):=\mathcal I_x^{-1}(v\otimes v^{*})\in \Lambda^{p,p}T_x^*X$ is the rank--one
\emph{positive} ray produced by the Hermitian model $\mathcal I_x$ from
Lemma~\ref{lem:hermitian-model}.
\end{editamirblockNEW}
\end{proposition}

\begin{proof}
\begin{editamirblockNEW}
Let $\beta:=\alpha_{x}^{(p,p)}-\gamma_{\harm,x}$ and set $H:=\mathcal I_x(\beta)\in \Herm(\mathcal H_x)$.
Decompose $H=\frac{\tr H}{d}\,\Id + H_{0}$ with $\tr(H_{0})=0$.
By Lemma~\ref{lem:hermitian-model} the map $\mathcal I_x$ is an isometry, and the primitive part
$\beta_{\prim}$ corresponds precisely to the traceless component $H_{0}$, so that
$\|H_{0}\|_{\mathrm{HS}}=\|\beta_{\prim}\|$.
Lemma~\ref{lem:rankone} provides a unit vector $v\in\mathcal H_x$ and $\lambda\ge 0$ such that
\[
\|H-\lambda\,v\otimes v^{*}\|_{\mathrm{HS}}
\le
C_{0}\,\|H_{0}\|_{\mathrm{HS}}.
\]
Applying $\mathcal I_x^{-1}$ yields \eqref{eq:pp-projection}.
\end{editamirblockNEW}
\end{proof}

\begin{corollary}[Pointwise rank--one PSD surrogate]\label{cor:Dpsd-pointwise}
\begin{editamirblockNEW}
Define the rank--one PSD surrogate distance
\[
D_{\mathrm{PSD}}(\alpha_{x})
:=
\min_{\substack{v\in\mathcal H_x,\ \|v\|=1\\ \lambda\ge 0}}
\|\alpha_{x}-\lambda\,\xi_{v}(x)\|.
\]
Then
\begin{equation}\label{eq:Dpsd-pointwise}
D_{\mathrm{PSD}}(\alpha_{x})^{2}
\le
\|\alpha_{x}^{\mathrm{off}}\|^{2}
+
C_{0}\,\|(\alpha_{x}^{(p,p)}-\gamma_{\harm,x})_{\prim}\|^{2}.
\tag{24}
\end{equation}
\smallskip

\noindent\textbf{Warning.}
This controls distance to the \emph{full} set of rank--one PSD rays (all $v\in \mathcal H_x$),
and therefore does \emph{not} by itself upper--bound the calibrated net distance $D_{\mathrm{net}}$,
which only ranges over decomposable (calibrated) rays and then discretizes them.
\end{editamirblockNEW}
\end{corollary}

\begin{proof}
\begin{editamirblockNEW}
For any rank--one ray $\lambda\xi_v(x)\in\Lambda^{p,p}T_x^*X$ we have, by orthogonality of Hodge types,
\[
\|\alpha_x-\lambda\xi_v(x)\|^2
=
\|\alpha_x^{\mathrm{off}}\|^2
+
\|\alpha_x^{(p,p)}-\lambda\xi_v(x)\|^2.
\]
Taking the infimum over $(v,\lambda)$ and applying Proposition~\ref{prop:pp-projection} yields
\eqref{eq:Dpsd-pointwise}.
\end{editamirblockNEW}
\end{proof}

\paragraph{Fixing a constant.}
\begin{editamirblockNEW}
The arguments above produce some constant $C_{0}(n,p)>0$ depending only on $(n,p)$.
For the remainder of the paper, we fix \emph{one such} admissible choice of $C_{0}(n,p)$
and do not attempt to optimize it.
\end{editamirblockNEW}

\section{Calibration--Coercivity (Explicit) and Its Proof}
\label{sec:cal-coercivity}

Let $(X,\omega)$ be a smooth complex projective manifold and let
$\gamma\in H^{2p}(X,\R)\cap H^{p,p}(X)$ be a de~Rham class.
Denote by $\gharm$ its unique $\omega$–harmonic representative and by
$E(\cdot)$ the Dirichlet energy.

\editcone{For each $x\in X$, let $K_p(x)\subset \Lambda^{p,p}T_x^*X$ denote the \emph{closed convex cone} of \emph{strongly positive} $(p,p)$--forms
(equivalently, the closed convex cone generated by the extremal rays associated to complex $(n-p)$--planes; cf.\ Section~\ref{sec:calibrated-grassmannian}).}
The global cone defect of a form $\alpha$ is
\[
\Defcone(\alpha)
:= \int_X \distcone(\alpha_x)^2\,d\mathrm{vol}_\omega(x),
\qquad
\distcone(\alpha_x)
:= \inf_{\beta_x\in K_p(x)} \|\alpha_x - \beta_x\|.
\]

\editcone{The main estimate of this section is the following explicit calibration--coercivity inequality.}

\begin{theorem}[Calibration--coercivity (cone-valued harmonic classes, explicit)]
	\label{thm:cal-coercivity}
	\editcone{Assume the $\omega$--harmonic representative satisfies $\gharm(x)\in K_p(x)$ for all $x\in X$.}
	Then for every smooth closed representative $\alpha\in[\gamma]$ one has
	\begin{equation}\label{eq:global-coercivity}
		E(\alpha)-E(\gharm) \;\ge\; \Defcone(\alpha).
	\end{equation}
\end{theorem}

\begin{proof}
\editcone{Since $\alpha$ and $\gharm$ represent the same class and are closed, $\alpha-\gharm=d\eta$ is exact; by Hodge orthogonality
$\langle \gharm, d\eta\rangle_{L^2}=0$, hence}
\[
E(\alpha)-E(\gharm)=\|\alpha-\gharm\|_{L^2}^2.
\]
Pointwise, because $\gharm(x)\in K_p(x)$ and $K_p(x)$ is a cone,
\[
\distcone(\alpha_x)\ =\ \inf_{\beta_x\in K_p(x)}\|\alpha_x-\beta_x\|
\ \le\ \|\alpha_x-\gharm(x)\|.
\]
Squaring and integrating yields $\Defcone(\alpha)\le \|\alpha-\gharm\|_{L^2}^2$, hence \eqref{eq:global-coercivity}.
\end{proof}

\begin{editconeblock}
\begin{remark}[On the coercivity hypothesis]\label{rem:coercivity-hypothesis}
The inequality in Theorem~\ref{thm:cal-coercivity} is \emph{purely geometric}: once the energy minimizer $\gharm$ lies in the closed convex cone $K_p(x)$ pointwise,
the cone distance is trivially controlled by the $L^2$ distance to $\gharm$.
No Hermitian spectral/projection formula is needed.

\smallskip\noindent
Conversely, if $\gharm$ fails to be cone-valued, then any statement of the form
$E(\alpha)-E(\gharm)\ge c\,\Defcone(\alpha)$ with $c>0$ cannot hold in general (apply it to $\alpha=\gharm$).
\end{remark}
\end{editconeblock}

% ------------------------------------------------------------
\subsection*{\editamir{Remark: a penalized route (not used in this paper)}}

Define the penalized functional on closed representatives of $[\gamma]$ by
\[
\mathcal{F}_\lambda(\alpha) := E(\alpha) + \lambda\,\Defcone(\alpha),
\qquad \lambda \ge 0.
\]
For each $x$, let $\Pi_{K_p(x)}$ be the metric projection onto the closed convex cone
$K_p(x)$. \editcone{Since $K_p(x)$ is a closed convex cone containing $0$, the metric projection satisfies the Moreau decomposition
$\alpha_x=\Pi_{K_p(x)}(\alpha_x)+\Pi_{K_p(x)^\circ}(\alpha_x)$ with orthogonality (where $K_p(x)^\circ$ is the polar cone); in particular}
\[
\|\alpha_x\|^2 = \|\Pi_{K_p(x)}(\alpha_x)\|^2 + \dist\!\bigl(\alpha_x,K_p(x)\bigr)^2.
\]
Integrating,
\begin{equation}\label{eq:projection-identity}
	E(\alpha) = E\!\bigl(\Pi_K(\alpha)\bigr) + \Defcone(\alpha),
\end{equation}
where $(\Pi_K\alpha)(x):=\Pi_{K_p(x)}(\alpha_x)$.

\begin{remark}[Limitation of pointwise projection]
While \eqref{eq:projection-identity} is a valid pointwise identity, the
fiberwise projection $\Pi_K(\alpha)$ does \emph{not} preserve closedness:
$d(\Pi_K(\alpha)) \neq 0$ in general, so $\Pi_K(\alpha)$ is not a closed
representative of $[\gamma]$.  Thus the naive descent argument
$\mathcal{F}_\lambda(\Pi_K(\alpha)) < \mathcal{F}_\lambda(\alpha)$ does not
produce a feasible competitor within the constraint set of closed forms.
A rigorous penalized approach would require combining pointwise projection
with a global Hodge-type correction (e.g., projecting onto the space of
closed forms after each step) and establishing that the resulting scheme
converges.  We do not pursue this route here; the main proof uses the
explicit SYR/microstructure construction in Section~\ref{sec:realization}.
\end{remark}

% ============================================================
\section{From Cone–Valued Minimizers to Calibrated Currents}\label{sec:realization}
% ============================================================

Let $\varphi=\omega^{p}/p!$ and let $\psi:=*\varphi=\omega^{n-p}/(n-p)!$ denote the
K\"ahler calibration of $\C$–dimension $(n-p)$ planes.  Set $k:=2n-2p$ and write $A=\mathrm{PD}(m[\gamma])\in H_{k}(X,\Z)$ for some $m\ge 1$.

% ------------------------------------------------------------
\begin{editjonblock}
\subsection*{Spine theorem: a single checkable quantitative output}

\begin{theorem}[Quantitative almost--mass--minimizing cycles (referee-checkable spine)]
\label{thm:spine-quantitative}
Let $(X,\omega)$ be a smooth projective K\"ahler manifold of complex dimension $n$, fix $1\le p\le n$, and set $\psi=\omega^{n-p}/(n-p)!$ and $k:=2n-2p$.
Let $[\gamma]\in H^{2p}(X,\Q)\cap H^{p,p}(X)$ admit a smooth closed cone--valued representative $\beta$.
Choose an integer $m\ge 1$ so that $m[\gamma]\in H^{2p}(X,\Z)$, and set
\[
A:=\mathrm{PD}(m[\gamma])\in H_k(X,\Z),
\qquad
c_0:=\langle A,[\psi]\rangle
=m\int_X \beta\wedge\psi.
\]
\editamir{Fix any sequence of mesh scales $h_j\to 0$. For each $j$, apply the constructions assembled in the H1/H2 packages (Propositions~\ref{prop:h1-package} and \ref{prop:transport-flat-glue-weighted}) together with global coherence (Proposition~\ref{prop:global-coherence-all-labels}) to obtain, at scale $h_j$:}
\begin{enumerate}
\item[\textnormal{(H1)}] a \emph{calibrated sheet--sum} integral current $S_j=\sum_Q S_{Q,j}$ built from holomorphic pieces in cells $Q$ (hence $\Mass(S_j)=\langle S_j,\psi\rangle$) and satisfying the single quantitative budget condition
\[
\Mass(S_j)=\langle S_j,\psi\rangle\ \le\ c_0+o(1);
\]
\item[\textnormal{(H2)}] a \emph{gluing current} $G_j$ and a fixed-class \emph{period/rounding choice} such that the corrected current
\[
T_j:=S_j-G_j
\]
satisfies $\partial T_j=0$ and $[T_j]=A$, and $G_j$ obeys the explicit mass bound
\[
\Mass(G_j)\ \le\ C_X\,h_j^2\sum_Q\ \sum_{a\in\mathcal S(Q,j)} m_{Q,a}^{\frac{k-1}{k}},
\qquad m_{Q,a}:=\Mass([Y^{Q,a}]\llcorner Q),
\]
where $\mathcal S(Q,j)$ indexes the holomorphic pieces in $Q$ at scale $h_j$ and $C_X$ depends only on $(X,\omega,n,p)$.
\end{enumerate}
Then the \emph{mass defect} satisfies the brutally simple bound
\[
0\ \le\ \Mass(T_j)-c_0\ \le\ 2\,\Mass(G_j).
\]
In particular, if the per-cell complexity satisfies $|\mathcal S(Q,j)|\le \Lambda_j$ for all $Q$, then
\[
\Mass(G_j)\ \le\ C_X\,c_0^{\frac{k-1}{k}}\,h_j^{\,2-\frac{2n}{k}}\,\Lambda_j^{1/k},
\]
so $\Mass(T_j)\to c_0$ whenever $h_j^{\,2-\frac{2n}{k}}\Lambda_j^{1/k}\to 0$.
\end{theorem}

\begin{proof}
Since $\psi$ is closed and $[T_j]=A$, the pairing is topological:
\[
\langle T_j,\psi\rangle=\langle [T_j],[\psi]\rangle=\langle A,[\psi]\rangle=c_0.
\]
The calibration inequality gives $\Mass(T_j)\ge \langle T_j,\psi\rangle=c_0$, hence $\Mass(T_j)-c_0\ge 0$.
Write $S_j=T_j+G_j$.  Since $S_j$ is $\psi$--calibrated, $\Mass(S_j)=\langle S_j,\psi\rangle$.
Thus, using the triangle inequality for mass and that $\psi$ has comass $\le 1$,
\[
\Mass(T_j)\ \le\ \Mass(S_j)+\Mass(G_j)
\ =\ \langle T_j+G_j,\psi\rangle+\Mass(G_j)
\ \le\ c_0+\bigl|\langle G_j,\psi\rangle\bigr|+\Mass(G_j)
\ \le\ c_0+2\,\Mass(G_j),
\]
which gives the stated defect estimate.

For the complexity bound, write $M_Q:=\sum_{a\in\mathcal S(Q,j)} m_{Q,a}=\Mass(S_{Q,j})$.
By H\"older/concavity,
\[
\sum_{a\in\mathcal S(Q,j)} m_{Q,a}^{\frac{k-1}{k}}
\ \le\ M_Q^{\frac{k-1}{k}}\,|\mathcal S(Q,j)|^{1/k}
\ \le\ M_Q^{\frac{k-1}{k}}\Lambda_j^{1/k}.
\]
Summing over $Q$, using $\sum_Q M_Q=\Mass(S_j)\le c_0+o(1)$, and that the number of $h_j$--cells is $\lesssim h_j^{-2n}$ gives
\[
\sum_Q M_Q^{\frac{k-1}{k}}
\ \le\ (\#\{Q\})^{1/k}\Bigl(\sum_Q M_Q\Bigr)^{\frac{k-1}{k}}
\ \lesssim\ h_j^{-\frac{2n}{k}}\,c_0^{\frac{k-1}{k}}.
\]
Substituting into (H2) yields $\Mass(G_j)\lesssim c_0^{\frac{k-1}{k}}h_j^{2-\frac{2n}{k}}\Lambda_j^{1/k}$.
\end{proof}

\begin{remark}[Where to look for (H1)--(H2) in this manuscript]
In our implementation, (H1) is supplied by the projective tangential approximation / holomorphic patch manufacturing package
(Bergman/peak-section control and finite-template realization; see Lemma~\ref{lem:bergman-control} and the local sheet construction in Theorem~\ref{thm:local-sheets}).
The gluing estimate in (H2) is obtained by combining transport-to-filling on faces (Proposition~\ref{prop:transport-flat-glue-weighted}) with the slice boundary shrinkage
estimate on smooth uniformly convex cells (Lemma~\ref{lem:uniformly-convex-slice-boundary}), packaged globally as Corollary~\ref{cor:global-flat-weighted},
and then enforcing the required face-level matching and global period constraints via the corner-exit vertex-template coherence mechanism
(packaged in Proposition~\ref{prop:global-coherence-all-labels}, with the flat-norm filling estimate recorded in Proposition~\ref{prop:glue-gap}).
\end{remark}

\subsection*{Global parameter schedule (quantifiers and order of choice)}
\label{sec:parameter-schedule}
We make explicit the order of choices used throughout Section~\ref{sec:realization}.
Fix $[\gamma]\in H^{2p}(X,\Q)\cap H^{p,p}(X)$ and a smooth closed cone--valued representative $\beta$ as in Theorem~\ref{thm:spine-quantitative}.
\begin{itemize}
\item \textbf{Choose $m$ first.} Pick an integer $m\ge 1$ so that $m[\gamma]\in H^{2p}(X,\Z)$ and, for a fixed integral basis of
$H^{2n-2p}(X,\Z)$ represented by smooth closed forms $\{\Theta_\ell\}_{\ell=1}^b$, all periods
$m\int_X\beta\wedge\Theta_\ell\in\Z$ (cf.\ Substep~4.3 and Proposition~\ref{prop:cohomology-match}).
\item \textbf{Then choose a mesh sequence.} Choose a sequence of mesh sizes $h_j\downarrow 0$ and a subordinate rounded cubulation by coordinate cubes $Q$ of size $h_j$.
\item \textbf{Then choose small local accuracy parameters as functions of $h_j$.}
Fix a direction-net scale $\varepsilon_{\mathrm{net},j}\ll h_j$ for the finite calibrated dictionary in Proposition~\ref{prop:global-coherence-all-labels}.
Choose a transverse grid spacing $\delta_j=o(h_j)$ on each face tubular chart $\Omega_F\cong B^{2p}(0,ch_j)$ and an angle tolerance $\varepsilon_j=o(1)$ for the small-angle graph model.
\editamirNEW{Choose also the transverse parameter radius factor $\varrho_j$ (defining the corner-exit template scale $s_j\asymp \varrho_j h_j$) such that $\varrho_j=o(\varepsilon_j)$ (required for the borderline $p=n/2$ case; see Lemma~\ref{lem:borderline-p-half}).}
\editamirNEW{In the corner-exit templates, the relevant footprint diameter satisfies $D_Q\asymp s_j$, so the within-label translation separation needed for disjointness is at the footprint scale $\delta_{\mathrm{sep},j}\asymp \varepsilon_j\,s_j$ (cf.\ Lemma~\ref{lem:sliver-stability} and Proposition~\ref{prop:finite-template}).}
\editamirNEW{Fix, for each interior face $F=Q\cap Q'$ in the mesh-$h_j$ cubulation, a tubular/flat face chart $\Omega_F\cong B^{2p}(0,c h_j)$ and linear face-parameter maps
$\Phi_{Q,F},\Phi_{Q',F}:B_{C_0\varrho_j h_j}(0)\to \Omega_F$ satisfying the uniform bounds
$\|\Phi_{Q,F}\|_{\mathrm{op}}+\|\Phi_{Q',F}\|_{\mathrm{op}}\le C_{\Phi,0}$ and $\|\Phi_{Q,F}-\Phi_{Q',F}\|_{\mathrm{op}}\le C_\Phi\,h_j$.
Then for template atoms $y_a$ with $\|y_a\|\le C_0\varrho_j h_j$ one has the \emph{uniform} displacement estimate
$\|\Phi_{Q,F}(y_a)-\Phi_{Q',F}(y_a)\|\le C\,\varrho_j h_j^2$, i.e.\ $\Delta_F\lesssim \varrho_j h_j^2$ (Lemmas~\ref{lem:w1-auto} and~\ref{lem:face-displacement}).}
\item \textbf{Finally choose holomorphic scales and discrete rounding data.}
Choose the holomorphic manufacturing scale (line bundle power) $N_j\to\infty$ large enough that the local sheet/sliver construction at tolerance $\varepsilon_j$
is available uniformly on all $h_j$--cells (Theorem~\ref{thm:local-sheets} and the corner-exit realization package), then choose the integer activation/rounding variables
(counts, prefixes) to meet the local budgets and global period constraints (Proposition~\ref{prop:global-coherence-all-labels} and Proposition~\ref{prop:cohomology-match}).

\item \textbf{Glue-scale closure (absolute).}
\begin{editamirblockNEW}
After fixing $m$ and choosing the mesh and local parameters as above, we now fix an arbitrary tolerance $\epsilon>0$.
Set
\[
\eta(\epsilon):=\min\Bigl\{\frac{\epsilon}{2},\ \Bigl(\frac{\epsilon}{2C_X}\Bigr)^{\frac{k-1}{k}}\Bigr\},
\qquad k:=2n-2p,
\]
where $C_X$ is the filling constant in Lemma~\ref{lem:FF-filling-X}.
Using Corollary~\ref{cor:global-flat-weighted}, choose $j$ large enough so that
\[
\delta_j:=\mathcal F(\partial T^{\mathrm{raw}}_j)\ <\ \eta(\epsilon).
\]
\editamir{(In the borderline case $p=n/2$, this is ensured by imposing the refined displacement schedule $\varrho_j=o(\varepsilon_j)$ so that Lemma~\ref{lem:borderline-p-half} gives $\delta_j\to 0$.)}
Then Proposition~\ref{prop:glue-gap} produces an \emph{integral} $k$--current $U_\epsilon$ with
$\partial U_\epsilon=\partial T^{\mathrm{raw}}_j$ and
\[
\Mass(U_\epsilon)\ \le\ \delta_j + C_X\,\delta_j^{\frac{k}{k-1}}\ <\ \epsilon.
\]
\end{editamirblockNEW}
\end{itemize}
\begin{editamirblockNEW}
Under this schedule (with $m$ fixed), the central quantitative goal is that the \emph{raw boundary mismatch}
\[
\delta_j:=\mathcal F\!\left(\partial T^{\mathrm{raw}}_j\right)
\]
can be made arbitrarily small by choosing $j$ large enough.  In the regime $p<n/2$ this follows from the
face-control estimate in Corollary~\ref{cor:global-flat-weighted} together with the mesh/complexity scaling in the
construction; in the borderline regime $p=n/2$ it is supplied by Lemma~\ref{lem:borderline-p-half}.
Once $\delta_j$ is small, Proposition~\ref{prop:glue-gap} gives an \emph{integral} boundary correction $U_\epsilon$ with
$\partial U_\epsilon=\partial T^{\mathrm{raw}}_j$ and $\Mass(U_\epsilon)<\epsilon$, as used in
Proposition~\ref{prop:cohomology-match}.  Finally, Proposition~\ref{prop:almost-calibration} converts the vanishing glue-mass
into vanishing calibration defect.
\end{editamirblockNEW}

\begin{editamirblockNEW}
\begin{lemma}[Borderline ($p=n/2$): closure via a refined displacement schedule]\label{lem:borderline-p-half}
Assume $p=n/2$ (equivalently $k=2n-2p=n$).
Under the hypotheses of Corollary~\ref{cor:global-flat-weighted} with the \emph{refined} displacement control
\[
\Delta_F\ \lesssim\ \varrho\,h^{2}
\qquad\text{(as in Lemma~\ref{lem:w1-auto}, Lemma~\ref{lem:template-displacement}, and Lemma~\ref{lem:w1-template-edit}),}
\]
and the footprint-scale packing bound $|\mathcal S(Q)|\lesssim \varepsilon^{-2p}=\varepsilon^{-n}$ (Lemma~\ref{lem:sliver-packing}, applied with transverse radius $r\asymp \varrho h$ and separation $\gtrsim \varepsilon r$),
one has the quantitative estimate
\[
\mathcal F(\partial T^{\mathrm{raw}})\ \le\ C\,\frac{\varrho}{\varepsilon}\,\Mass(T^{\mathrm{raw}})^{\frac{n-1}{n}},
\]
where $C$ depends only on $X,n$ (and the fixed geometric data), not on $h,\varepsilon,\varrho$.
In particular, if $\varrho=o(\varepsilon)$ as $h\downarrow 0$ and $\sup_h \Mass(T^{\mathrm{raw}})<\infty$ (as ensured by the construction),
then
\[
\mathcal F(\partial T^{\mathrm{raw}})\ \longrightarrow\ 0
\qquad\text{as }h\downarrow 0
\]
also in the middle-dimensional regime $p=n/2$.
\end{lemma}

\begin{proof}
Start from Corollary~\ref{cor:global-flat-weighted}.  In the refined schedule one has
\[
\mathcal F(\partial T^{\mathrm{raw}})
\ \le\ C\,\varrho\,h^2 \sum_Q \sum_{a\in \mathcal S(Q)} m_{Q,a}^{\frac{n-1}{n}},
\]
with $m_{Q,a}:=\Mass([Y^{Q,a}]\llcorner Q)$ and $M_Q:=\sum_{a\in\mathcal S(Q)} m_{Q,a}$.
Since the map $t\mapsto t^{(n-1)/n}$ is concave on $\R_+$, Jensen (or the standard $\ell^1$--$\ell^{(n-1)/n}$ inequality) yields
\[
\sum_{a\in \mathcal S(Q)} m_{Q,a}^{\frac{n-1}{n}}
\ \le\ |\mathcal S(Q)|^{\frac1n}\,M_Q^{\frac{n-1}{n}}.
\]
Using the packing bound $|\mathcal S(Q)|\lesssim \varepsilon^{-n}$ gives $|\mathcal S(Q)|^{1/n}\lesssim \varepsilon^{-1}$, hence
\[
\mathcal F(\partial T^{\mathrm{raw}})
\ \le\ C\,\frac{\varrho}{\varepsilon}\,h^2 \sum_Q M_Q^{\frac{n-1}{n}}.
\]
Finally, Hölder with exponents $\frac{n}{n-1}$ and $n$ implies
\[
\sum_Q M_Q^{\frac{n-1}{n}}
\ \le\ \bigl(\#\{Q\}\bigr)^{\frac1n}\Bigl(\sum_Q M_Q\Bigr)^{\frac{n-1}{n}}
\ =\ \bigl(\#\{Q\}\bigr)^{\frac1n}\,\Mass(T^{\mathrm{raw}})^{\frac{n-1}{n}}.
\]
Since $\#\{Q\}\asymp h^{-2n}$ for a cubulation at mesh $h$, we have $(\#\{Q\})^{1/n}\lesssim h^{-2}$.
Substituting yields
\[
\mathcal F(\partial T^{\mathrm{raw}})
\ \le\ C\,\frac{\varrho}{\varepsilon}\,\Mass(T^{\mathrm{raw}})^{\frac{n-1}{n}},
\]
as claimed.
\end{proof}



\begin{editamirblockNEW}
\editamir{\noindent\textbf{Referee cleanup.} The proof above is the consolidated argument for Lemma~\ref{lem:borderline-p-half}. The duplicate proof block kept from an earlier patch is disabled to avoid ambiguity.}
\end{editamirblockNEW}
\iffalse

\begin{proof}
In the middle-dimensional case $k=n$, Corollary~\ref{cor:global-flat-weighted} gives
\[
\mathcal F(\partial T^{\mathrm{raw}})
\ \editamir{\lesssim\ \varrho\,h^2} \sum_Q \sum_{a\in\mathcal S(Q)} m_{Q,a}^{\frac{n-1}{n}}.
\]
For each fixed cell $Q$, H\"older yields
\[
\sum_{a\in\mathcal S(Q)} m_{Q,a}^{\frac{n-1}{n}}
\ \le\ \Bigl(\sum_{a\in\mathcal S(Q)} m_{Q,a}\Bigr)^{\frac{n-1}{n}}\;|\mathcal S(Q)|^{\frac1n}
\ \lesssim\ M_Q^{\frac{n-1}{n}}\;\varepsilon^{-1},
\]
where $M_Q:=\sum_{a\in\mathcal S(Q)} m_{Q,a}$ and we used $|\mathcal S(Q)|\lesssim\varepsilon^{-n}$.
Summing in $Q$ and using the concavity of $t\mapsto t^{\frac{n-1}{n}}$ gives
\[
\sum_Q M_Q^{\frac{n-1}{n}}
\ \le\ \Bigl(\sum_Q M_Q\Bigr)^{\frac{n-1}{n}}\;(\#\{Q\})^{\frac1n}
\ \asymp\ \Mass(T^{\mathrm{raw}})^{\frac{n-1}{n}}\;h^{-2},
\]
since the number of mesh cells in real dimension $2n$ is $\#\{Q\}\asymp h^{-2n}$.
Combining the displayed inequalities yields $\mathcal F(\partial T^{\mathrm{raw}})\lesssim \dfrac{\varrho}{\varepsilon}\,\Mass(T^{\mathrm{raw}})^{\frac{n-1}{n}}$.
This tends to $0$ as $h\downarrow 0$ under the schedule $\varrho=o(\varepsilon)$ and $\sup_h \Mass(T^{\mathrm{raw}})<\infty$.
\end{proof}
\fi

\end{editamirblockNEW}


\subsection*{H1/H2 packaged at the point of use (for Theorem~\ref{thm:spine-quantitative})}

\begin{proposition}[H1 package: local holomorphic multi-sheet manufacturing]\label{prop:h1-package}
In the parameter schedule of \S\ref{sec:parameter-schedule}, for each mesh cell $Q$ and each direction family prescribed by the local Carath\'eodory data of $\beta$ on $Q$,
Theorem~\ref{thm:local-sheets} and the projective holomorphic manufacturing machinery supply the required calibrated sheet--sum $S_Q$ satisfying $\Mass(S_Q)=\langle S_Q,\psi\rangle$
with quantitative \editamir{within-direction} disjointness, slope, and budget control.  Thus the hypothesis \textnormal{(H1)} in Theorem~\ref{thm:spine-quantitative} holds in this manuscript.

\begin{editamirblock}
\noindent\textbf{Referee tightening (make \textnormal{(H1)} verifiable from cited inputs).}
Fix an index $j$ and the mesh scale $h_j$.
For each cell $Q$, let $\{(P_{Q,a},m_{Q,a})\}_{a\in\mathcal S(Q,j)}$ denote the local Carath\'eodory data for the cone field $\beta$ on $Q$
(Lemma~\ref{lem:caratheodory-general}, as organized in \S\ref{sec:parameter-schedule}).
For each $a\in\mathcal S(Q,j)$, Proposition~\ref{prop:finite-template} realizes the corresponding finite translation template by holomorphic
complete intersections $Y^{Q,a}$, and Proposition~\ref{prop:cell-scale-linear-model-graph} (with Lemma~\ref{lem:sliver-stability})
upgrades the local model to a single $C^1$ graph over $P_{Q,a}$ on all of $Q$.
Define the per-cell sheet current and its global sum by
\[
S_{Q,j}:=\sum_{a\in\mathcal S(Q,j)} [Y^{Q,a}]\llcorner Q,
\qquad
S_j:=\sum_Q S_{Q,j}.
\]
Then each $S_{Q,j}$ is $\psi$--calibrated (hence $\Mass(S_{Q,j})=\langle S_{Q,j},\psi\rangle$), and the within-direction disjointness
from Theorem~\ref{thm:local-sheets} ensures additivity of the $\psi$--mass inside each cell.
Summing over $Q$ and using that the Carath\'eodory weights encode the $\psi$--pairing with $\beta$ at scale $h_j$
(as in the schedule), we obtain
\[
\Mass(S_j)=\langle S_j,\psi\rangle\ \le\ c_0+o(1),
\]
so $S_j$ satisfies the hypothesis \textnormal{(H1)} in Theorem~\ref{thm:spine-quantitative}.
\end{editamirblock}
\end{proposition}

\begin{proposition}[H2 package: global face coherence and gluing (corner-exit route)]\label{prop:h2-package}
In the parameter schedule of \S\ref{sec:parameter-schedule} (with fixed $m$ and $h_j\downarrow 0$), the corner-exit vertex-template coherence package yields
\begin{itemize}
\item per-face transverse matching (Proposition~\ref{prop:global-coherence-all-labels}, possibly with prefix-edits),
\item global flat-norm estimate $\mathcal F(\partial T^{\mathrm{raw}})\to 0$ for $p<n/2$ (Corollary~\ref{cor:global-flat-weighted}); in the borderline case $p=n/2$, the same conclusion holds under the refined displacement schedule $\varrho=o(\varepsilon)$ (Lemma~\ref{lem:borderline-p-half}),
\item filling with vanishing mass (Proposition~\ref{prop:glue-gap}).
\end{itemize}
The exact-class conclusion is enforced by Proposition~\ref{prop:cohomology-match} together with the per-face matching/transport mechanism (Proposition~\ref{prop:integer-transport}),
rather than relying on a decay exponent in $h$.
\editamir{In the borderline case $p=n/2$, Lemma~\ref{lem:borderline-p-half} supplies $\mathcal F(\partial T^{\mathrm{raw}}_j)\to 0$ under the refined schedule $\varrho_j=o(\varepsilon_j)$, so Proposition~\ref{prop:glue-gap} still produces fillings $U_\epsilon$ of arbitrarily small mass for use in Proposition~\ref{prop:cohomology-match}.}
Thus the hypothesis \textnormal{(H2)} in Theorem~\ref{thm:spine-quantitative} holds \editamir{for $p<n/2$ under the present estimates, and also for $p=n/2$ provided the refined displacement schedule $\varrho=o(\varepsilon)$ from Lemma~\ref{lem:borderline-p-half} is enforced.}
\end{proposition}
\end{editjonblock}

% ------------------------------------------------------------
\subsection*{Closure from almost-calibrated sequences}

\begin{theorem}[Realization from almost--calibrated sequences]\label{thm:realization-from-almost}
Let $(X^n,\omega)$ be a compact K\"ahler manifold, fix $1\le p\le n-1$, and set
\[
\psi := \frac{\omega^{\,n-p}}{(n-p)!}\in \Omega^{2n-2p}(X).
\]
Let $\gamma\in H^{p,p}(X)\cap H^{2p}(X;\Z)$ be an integral Hodge class and choose $m\in\N$
so that $A:=\mathrm{PD}(m[\gamma])\in H_{2n-2p}(X;\Z)$ is an integral homology class.
Define the cohomological lower bound
\[
c_0 := \int_X m\,\gamma\wedge \psi \;=\; \langle A,[\psi]\rangle .
\]
\editamir{Let} a sequence of integral $(2n-2p)$--cycles $\{T_k\}_{k\ge 1}$ with
$\partial T_k=0$ and \editamir{$[T_k]=A$ in $H_{2n-2p}(X;\Z)/\mathrm{Tor}$} such that the calibration defect tends to zero:
\[
\Def_{\mathrm{cal}}(T_k)\ :=\ \Mass(T_k)-\langle T_k,\psi\rangle \ \longrightarrow\ 0.
\]
(Equivalently, since $\psi$ is closed and the homology class is fixed in $H_{2n-2p}(X;\Z)/\mathrm{Tor}$, one has $\langle T_k,\psi\rangle=c_0$ for all $k$ and thus
$\Def_{\mathrm{cal}}(T_k)\to 0$ iff $\Mass(T_k)\to c_0$.)
Then there exists an integral $(2n-2p)$--cycle $T$ with $\partial T=0$ and \editamir{$[T]=A$ in $H_{2n-2p}(X;\Z)/\mathrm{Tor}$} such that
\[
\Mass(T)=\langle T,\psi\rangle=c_0.
\]
In particular, $T$ is $\psi$--calibrated.  Consequently $T$ is a $d$--closed positive locally integral current of bidimension $(p,p)$,
hence a holomorphic chain:
\[
T=\sum_{j=1}^N m_j [V_j],
\]
with $m_j\in\mathbb{N}$ and $V_j\subset X$ irreducible complex analytic subvarieties of codimension $p$.
If $X$ is projective, each $V_j$ is algebraic (Remark~\ref{rem:chow-gaga}), and therefore $[\gamma]\in H^{2p}(X;\Q)$ is an
algebraic cohomology class.
\end{theorem}

\begin{editamirblock}
\begin{proof}
Since \editamir{$[T_k]=A$ in $H_{2n-2p}(X;\Z)/\mathrm{Tor}$} is fixed and $\psi$ is closed, the pairing is constant:
\(
\langle T_k,\psi\rangle=\langle A,[\psi]\rangle=c_0
\)
for all $k$.  The hypothesis $\Def_{\mathrm{cal}}(T_k)\to 0$ therefore gives $\Mass(T_k)\to c_0$, hence $\sup_k\Mass(T_k)<\infty$.
By Federer--Fleming compactness for integral currents on a compact manifold (e.g. \cite{Fed69}), after passing to a subsequence
we may assume $T_k\to T$ in the flat norm for some integral current $T$.
Because $\partial T_k=0$ for all $k$, the limit is also a cycle, $\partial T=0$ (cf.\ Lemma~\ref{lem:flat_limit_of_cycles_is_cycle}),
and the homology class agrees with $A$ in real homology (equivalently, in $H_{2n-2p}(X;\Z)$ modulo torsion). Indeed, for any smooth closed $(2n-2p)$--form $\eta$ on $X$ we have $\langle T_k,\eta\rangle=\langle A,[\eta]\rangle$ for all $k$; by flat convergence, $\langle T,\eta\rangle=\lim_k\langle T_k,\eta\rangle=\langle A,[\eta]\rangle$.

Flat convergence implies convergence against smooth forms, so
\(
\langle T,\psi\rangle=\lim_k \langle T_k,\psi\rangle=c_0.
\)
Lower semicontinuity of mass under flat convergence gives
\(
\Mass(T)\le \liminf_k \Mass(T_k)=c_0.
\)
On the other hand, since $\psi$ has comass $\le 1$ (Wirtinger inequality for the K\"ahler calibration),
\(
\langle T,\psi\rangle \le \Mass(T).
\)
Combining these inequalities yields
\(
\Mass(T)=\langle T,\psi\rangle=c_0,
\)
so $T$ is $\psi$--calibrated (equivalently, $\Def_{\mathrm{cal}}(T)=0$).
For the K\"ahler/Wirtinger calibration, $\psi$--calibration forces the approximate tangent planes of $T$ to be complex $(n-p)$--planes with the standard complex orientation (Harvey--Lawson~\cite{HL82}). Hence $T$ is a \emph{positive} current of bidimension $(p,p)$.
Since $T$ is an integral \emph{cycle}, $\partial T=0$, it is $d$--closed as a current.
King's theorem~\cite{King71} (in the form ``positive $d$--closed locally integral $(p,p)$--currents are holomorphic chains'') then yields
\(
T=\sum_j m_j[V_j],
\)
with $m_j\in\Z_{>0}$ and $V_j$ irreducible analytic subvarieties of codimension $p$.
If $X$ is projective, each $V_j$ is algebraic by Chow/GAGA (Remark~\ref{rem:chow-gaga}).
\end{proof}
\end{editamirblock}


\begin{editamirblock}
\begin{lemma}[Flat limits of cycles are cycles]\label{lem:flat_limit_of_cycles_is_cycle}
Let $T_k$ be integral currents of dimension $m$ on $X$ with $\partial T_k=0$ and
$\sup_k\Mass(T_k)<\infty$.  Assume $T_k\to T$ in the flat norm.  Then $T$ is an
integral $m$--cycle, i.e.\ $\partial T=0$.
\end{lemma}
\begin{proof}
The boundary operator is continuous with respect to flat convergence, hence
$\partial T_k\to \partial T$ in flat norm.  Since $\partial T_k=0$ for all $k$, we get
$\partial T=0$.
Moreover, the class of integral currents is closed under flat limits under the
uniform mass and boundary-mass bounds (Federer--Fleming compactness/closure; see
\cite[\S\S 4.1--4.2]{Fed69}).  Therefore $T$ is integral and, by the first part, a cycle.
\end{proof}
\end{editamirblock}

\begin{lemma}[Almost--calibrated limits are calibrated]\label{lem:limit_is_calibrated}
Let $\psi$ be a smooth closed form with comass $\le 1$.
Let $T_k$ be integral currents with $\sup_k\Mass(T_k)<\infty$ and $T_k\rightharpoonup T$ weakly.
If the calibration deficits \editamir{$\Def_{\mathrm{cal}}(T_k):=\Mass(T_k)-\langle T_k,\psi\rangle$} satisfy $\Def_{\mathrm{cal}}(T_k)\to 0$,
then $T$ is $\psi$--calibrated, i.e. $\Mass(T)=\langle T,\psi\rangle$.
\end{lemma}

\begin{proof}
Weak convergence gives $\langle T_k,\psi\rangle\to \langle T,\psi\rangle$.
Lower semicontinuity of mass under weak convergence (see e.g. \cite[Ch.~XIV]{LangGmT} or \cite{Sim83}) yields
$\Mass(T)\le \liminf_k \Mass(T_k)=\liminf_k\bigl(\langle T_k,\psi\rangle+\Def_{\mathrm{cal}}(T_k)\bigr)
=\langle T,\psi\rangle$.
\editamir{On the other hand, $\mathrm{comass}(\psi)\le 1$ implies $|\langle T,\psi\rangle|\le \Mass(T)$ and hence $\langle T,\psi\rangle\le \Mass(T)$ (see e.g. \cite[Lemma~3.5]{HL82}).}
Thus $\Mass(T)=\langle T,\psi\rangle$, so $T$ is $\psi$--calibrated.
\end{proof}

\begin{remark}[How to use Theorem~\ref{thm:realization-from-almost}]
	\begin{editconeblock}
	Theorem~\ref{thm:realization-from-almost} is an abstract closure principle: once one has a fixed-class sequence of integral cycles whose masses approach the cohomological lower bound $c_0$,
	the limit is automatically $\psi$--calibrated and hence analytic (Harvey--Lawson).
	The remainder of this section explains how to build such almost–calibrated integral cycles starting from a smooth closed cone–valued form $\beta$:
	first in classical situations (e.g.\ codimension one, complete intersections, and other LICD cases), and then (in general codimension) via the microstructure/gluing theorem proved below using the projective tangential approximation framework.
	\end{editconeblock}
\end{remark}

% ------------------------------------------------------------
\subsection*{Unconditional realizability in codimension one (Lefschetz (1,1))}


\begin{theorem}[Codimension one (Lefschetz $(1,1)$)]\label{thm:codim1}
	If $p=1$ and $[\gamma]\in H^{1,1}(X,\Q)$ on a smooth projective $X$, then
	$[\gamma]$ is algebraic.
\end{theorem}

\begin{proof}
Choose $m\ge 1$ so that $m[\gamma]\in H^{1,1}(X,\Z)$.
By the Lefschetz $(1,1)$ theorem, there exists a holomorphic line bundle $L\to X$ with
\(
c_1(L)=m[\gamma].
\)
Equivalently, $m[\gamma]$ lies in the N\'eron--Severi group and is represented by an algebraic divisor class.
Thus the homology class $\mathrm{PD}(m[\gamma])\in H_{2n-2}(X,\Z)$ is represented by a codimension-one algebraic cycle
(\emph{a divisor with integer multiplicities}), and dividing by $m$ shows $[\gamma]$ is algebraic as a rational class.
\end{proof}

\begin{remark}[Mass equality in the effective codimension-one case]
If in addition $m[\gamma]$ is represented by an \emph{effective} divisor $D$ (so $D$ is a complex hypersurface with positive orientation),
then the current $[D]$ is $\psi$--calibrated by $\psi=\omega^{n-1}/(n-1)!$ and satisfies the exact mass identity
\(
\Mass([D])=\int_D\psi=\langle \mathrm{PD}(m[\gamma]),[\psi]\rangle.
\)
In particular, the constant sequence $T_k:=[D]$ is an almost-calibrated realizing sequence with $\Mass(T_k)$ equal to the cohomological pairing.
\end{remark}


% ------------------------------------------------------------
\subsection*{Complete–intersection realizability (very ample slicing)}

\begin{proposition}[Complete intersections]\label{prop:complete-intersection}
	Suppose $[\gamma]\in H^{p,p}(X,\Q)$ can be written as a rational linear
	combination of cohomology classes of complete intersections of $p$ very ample
	divisors. Then there exists a sequence of integral cycles in the class
	$\mathrm{PD}(m[\gamma])$ with masses tending to $c_0$, and the limit is a calibrated
	sum of complex subvarieties realizing $[\gamma]$.
\end{proposition}

\begin{proof}[Idea]
	Very ample divisors are represented by smooth hypersurfaces calibrated by
	$\omega^{n-1}/(n-1)!$. Intersections of $p$ such hypersurfaces produce smooth
	complex submanifolds of codimension $p$ calibrated by $\psi=\omega^{n-p}/(n-p)!$.
	Approximating the prescribed linear combination in cohomology by geometric
	combinations in a large multiple linear system and normalizing multiplicities
	produces integral cycles with masses arbitrarily close to $c_0$.
\end{proof}

% ------------------------------------------------------------
\subsection*{General realizability: a stationarity hypothesis}

\begin{definition}[Stationary Young--measure realizability (SYR)]\label{def:syr}
We say a cone--valued smooth closed $(p,p)$--form $\beta$ (representing the rational Hodge class $[\gamma]$)
is \emph{SYR--realizable} if there exists a sequence of integral $(2n-2p)$--cycles $T_k$ such that
\begin{enumerate}
\item $\partial T_k=0$ and \editamir{$[T_k]=\mathrm{PD}(m[\gamma])$ in $H_{2n-2p}(X;\Z)/\mathrm{Tor}$ (equivalently in $H_{2n-2p}(X;\Q)$)} for some fixed integer $m\ge 1$ (independent of $k$), and
\item the \emph{calibration defect} satisfies
\[
\Def_{\mathrm{cal}}(T_k)\ :=\ \Mass(T_k)-\langle T_k,\psi\rangle\ \longrightarrow\ 0.
\]
\end{enumerate}
Equivalently, since $\psi$ is closed and \editamir{$[T_k]=\mathrm{PD}(m[\gamma])$ in $H_{2n-2p}(X;\Z)/\mathrm{Tor}$}, one has the exact pairing identity
\[
\langle T_k,\psi\rangle=\bigl\langle [T_k],[\psi]\bigr\rangle
=\bigl\langle \mathrm{PD}(m[\gamma]),[\psi]\bigr\rangle
=m\int_X \beta\wedge\psi \;=:\; c_0
\qquad\text{for all }k,
\]
and therefore SYR is equivalent to $\Mass(T_k)\to c_0$.

\end{definition}
\begin{editamirblockNEW}
\noindent\textbf{Notation warning.} The symbol $m$ is used in two distinct roles:
(i) the cohomology multiplier in $\mathrm{PD}(m[\gamma])$ in Definition~\ref{def:syr}; and
(ii) the large tensor power $L^{\otimes N}$ in the Bergman/holomorphic inputs (e.g.\ the $N^{-1/2}$ scale).
\editamir{Throughout the referee-layer patches we reserve $m$ for the cohomology multiplier and use $N$ for the holomorphic tensor power, so the Bergman scale is always $N^{-1/2}$.}
\end{editamirblockNEW}


\begin{theorem}[Calibrated realization under SYR]\label{thm:syr}
Assume $\beta$ is SYR--realizable in the sense of Definition~\ref{def:syr}, and let $\psi=\omega^{n-p}/(n-p)!$.
Then there exists an integral $(2n-2p)$--cycle $T$ with $\partial T=0$ and \editamir{$[T]=\mathrm{PD}(m[\gamma])$ in $H_{2n-2p}(X;\Z)/\mathrm{Tor}$} such that
\[
\Mass(T)=\langle T,\psi\rangle=\bigl\langle \mathrm{PD}(m[\gamma]),[\psi]\bigr\rangle.
\]
\editamir{In particular, $T$ is $\psi$--calibrated.  For the K\"ahler calibration $\psi=\omega^{n-p}/(n-p)!$, calibrated integral currents are $d$--closed and positive of bidimension $(p,p)$ (see \cite{HL82}).  By King's theorem on positive closed locally integral currents \cite{King71}, $T$ is a holomorphic chain}
\[
T=\sum_j m_j[V_j],
\qquad m_j\in\mathbb{N},
\]
\editamir{where $V_j\subset X$ are irreducible complex analytic subvarieties of codimension $p$.  If, moreover, $X$ is projective, then each $V_j$ is algebraic by Chow/GAGA (Remark~\ref{rem:chow-gaga}), so $[\gamma]\in H^{2p}(X;\Q)$ is an algebraic class.}
\end{theorem}


\begin{editamirblockNEW}
\editamir{\noindent\textbf{Referee clarification (holomorphic-chain conclusion).}
Once Theorem~\ref{thm:realization-from-almost} produces an \emph{integral} cycle $T$ with
$\Mass(T)=\langle T,\psi\rangle$ for $\psi=\omega^{n-p}/(n-p)!$, the current $T$ is $\psi$--calibrated.
By Wirtinger/Harvey--Lawson calibrated-geometry results \cite{HL82}, a $\psi$--calibrated integral current is strongly positive and
$d$--closed of the correct Hodge type. King’s theorem then yields the holomorphic-chain representation
$T=\sum_j m_j[V_j]$ \cite{King71}. (Projective $\Rightarrow$ algebraic is handled separately via Chow/GAGA.)}
\end{editamirblockNEW}
\begin{proof}
Let $\{T_k\}$ be the SYR sequence.
By Definition~\ref{def:syr}, $\Def_{\mathrm{cal}}(T_k)\to 0$ and the homology classes $[T_k]=\mathrm{PD}(m[\gamma])$ are fixed.
Since $\psi$ is closed, the pairing $\langle T_k,\psi\rangle$ depends only on the homology class, so
\[
\langle T_k,\psi\rangle=\bigl\langle \mathrm{PD}(m[\gamma]),[\psi]\bigr\rangle=:c_0
\qquad\text{for all }k.
\]
Hence $\Mass(T_k)=\Def_{\mathrm{cal}}(T_k)+\langle T_k,\psi\rangle\to c_0$.
Applying Theorem~\ref{thm:realization-from-almost} to the fixed--class sequence $\{T_k\}$ yields an integral cycle $T$ with \editamir{$[T]=\mathrm{PD}(m[\gamma])$ in $H_{2n-2p}(X;\Z)/\mathrm{Tor}$}
in the same class with $\Mass(T)=\langle T,\psi\rangle=c_0$ and the holomorphic--chain representation
\(T=\sum_j m_j[V_j]\).
When $X$ is projective, algebraicity of analytic subvarieties follows by Remark~\ref{rem:chow-gaga}.
\end{proof}

\begin{remark}
	The SYR condition encodes the “microstructure” step in a purely geometric–measure
	framework (stationarity/compactness). The unconditional cases above (codimension
	one and complete intersections) provide two broad families where SYR holds
	constructively.
\end{remark}

% ------------------------------------------------------------
\subsection*{A classical sufficient criterion for SYR}

We now give a classical, fully geometric–measure–theoretic criterion under which
SYR holds, stated purely in standard language (coverings, Carath\'eodory
decompositions, isoperimetric fillings, and varifold compactness).

\begin{definition}[Locally integrable calibrated decomposition (LICD)]
	We say a smooth closed cone–valued $(p,p)$–form $\beta$ satisfies LICD if there
	exists a finite cover $\{U_\alpha\}$ of $X$ and for each $\alpha$:
	\begin{enumerate}
		\item smooth nonnegative coefficients $a_{\alpha,j}\in C^\infty(U_\alpha)$ and
		\item smooth fields of simple calibrated covectors $\xi_{\alpha,j}$ on $U_\alpha$,
	\end{enumerate}
	with $\beta=\sum_j a_{\alpha,j}\,\xi_{\alpha,j}$ on $U_\alpha$, where each
	$\xi_{\alpha,j}$ arises from a smooth integrable complex distribution of
	$(n-p)$–planes, i.e.\ through each $x\in U_\alpha$ there is a local
	$(n-p)$–dimensional complex submanifold whose oriented tangent plane is calibrated
	by $\psi$ and corresponds to $\xi_{\alpha,j}(x)$.
\end{definition}

\begin{theorem}[Classical SYR under LICD]\label{thm:classical-syr-licd}
	Let $(X,\omega)$ be smooth complex projective, $1\le p\le n$. If a smooth closed
	cone–valued $(p,p)$–form $\beta$ representing $[\gamma]$ satisfies LICD, then $\beta$
	is SYR–realizable. In particular, there exist integral cycles $T_k$ with $\partial T_k=0$,
	$[T_k]=\mathrm{PD}(m[\gamma])$ and $\Def_{\mathrm{cal}}(T_k)\to 0$ (equivalently, $\Mass(T_k)\to c_0$).
\end{theorem}

\begin{proof}[Proof (classical construction in charts)]
	Work in a single $U_\alpha$; a partition of unity reduces the global construction
	to a finite sum of local ones plus negligible overlaps.
	
	\emph{Step 1: Grid approximation and rationalization.} Fix a small mesh scale
	$\varepsilon>0$ and subordinate cubes $\{Q\}$ in a normal coordinate chart so that
	$\omega$ and $\psi$ vary by $O(\varepsilon)$ in each cell. By Carath\'eodory,
	$\beta=\sum_j a_j\,\xi_j$ with finitely many summands; approximate on each $Q$ by
	piecewise–constant smoothings
	\[
	\beta_Q \approx \sum_{j=1}^{N_Q} \theta_{Q,j}\,\xi_{Q,j},
	\qquad \theta_{Q,j}\in \Q_{\ge 0},\ \ \xi_{Q,j}\ \text{constant calibrated covectors},
	\]
	with $\sum_j \theta_{Q,j}$ bounded and the error $O(\varepsilon)$ in $C^0(Q)$.
	Write $\theta_{Q,j}=N_{Q,j}/M_Q$ with $N_{Q,j}\in\N$.
	
	\emph{Step 2: Local lamination by calibrated leaves.} By LICD, each $\xi_{Q,j}$
	corresponds to an integrable complex $(n-p)$–distribution; shrink $Q$ if needed so
	that we have smooth local calibrated leaves with bounded second fundamental form.
	\editamir{Fix $j$. Work in a local foliation box for the integrable distribution underlying $\xi_{Q,j}$.
	Choose $N_{Q,j}$ local plaques (allowing repetition) contained in $Q$ after trimming a collar of width $O(\varepsilon)$.
	For fixed $j$ these plaques can be taken pairwise disjoint inside the foliation box; for different $j$ they may intersect transversely.
	This is harmless because we form an integral current by summing the plaques with multiplicity. Define $S_Q$ as this sum.}
	resulting current $S_Q$ has tangent planes calibrated by $\psi$ almost everywhere
	in $Q$ and satisfies
	\[
	\Mass(S_Q) = \int S_Q\,\psi = \sum_j N_{Q,j}\int_{\mathrm{leaf}_{Q,j}}\psi
	= M_Q\int_Q \sum_j \theta_{Q,j}\,\langle \xi_{Q,j},\psi\rangle \,d\vol + O(\varepsilon\,|Q|),
	\]
	where the error arises from leaf boundaries near $\partial Q$ and the
	metric–calibration variation $O(\varepsilon)$. Since $\xi_{Q,j}$ are calibrated,
	$\langle\xi_{Q,j},\psi\rangle=1$ pointwise, hence $\Mass(S_Q)=M_Q\int_Q \sum_j
	\theta_{Q,j}\,d\vol + o_\varepsilon(1)$.
	
	\emph{Step 3: Closure by a small-mass filling (flat-norm viewpoint).}
	Set $S^{\mathrm{raw}}_\varepsilon:=\sum_Q S_Q$.
	Its boundary $\partial S^{\mathrm{raw}}_\varepsilon$ is supported on the union of cell interfaces.
	\editamir{While $\Mass(\partial S^{\mathrm{raw}}_\varepsilon)$ may be large, the robust quantity for gluing is the \emph{flat norm} $\mathcal F(\partial S^{\mathrm{raw}}_\varepsilon)$.
	By the dual characterization of $\mathcal F$ and Stokes, for any test form $\eta$ with $\|\eta\|_{\mathrm{comass}}\le 1$ and $\|d\eta\|_{\mathrm{comass}}\le 1$ one has
	$\partial S^{\mathrm{raw}}_\varepsilon(\eta)=S^{\mathrm{raw}}_\varepsilon(d\eta)$.
	The cellwise $C^0$ approximation of $\beta$ together with bounded geometry implies $S^{\mathrm{raw}}_\varepsilon(d\eta)=\int_X (m\beta)\wedge d\eta + o_\varepsilon(1)=o_\varepsilon(1)$ (since $d\beta=0$), hence $\mathcal F(\partial S^{\mathrm{raw}}_\varepsilon)\to 0$.}
	By a Federer--Fleming isoperimetric/filling inequality for integral currents on compact manifolds, there exists an integral current $R_\varepsilon$ with
	$\partial R_\varepsilon = -\partial S^{\mathrm{raw}}_\varepsilon$ and $\Mass(R_\varepsilon)\to 0$.
	Define the closed integral cycle $T_\varepsilon:=S^{\mathrm{raw}}_\varepsilon+R_\varepsilon$.
	Since $S^{\mathrm{raw}}_\varepsilon$ is a sum of $\psi$--calibrated pieces, $\int_{S^{\mathrm{raw}}_\varepsilon}\psi=\Mass(S^{\mathrm{raw}}_\varepsilon)$, and therefore
	\[
	0\le \Def_{\mathrm{cal}}(T_\varepsilon)\le 2\,\Mass(R_\varepsilon)\xrightarrow[\varepsilon\to 0]{}0.
	\]
	
	\emph{Step 4: Homology adjustment and mass control.} Pairing with $\psi$ shows
	\[
	\Mass(T_\varepsilon)=\int T_\varepsilon\,\psi
	= \sum_Q \int_Q \sum_j \theta_{Q,j}\,d\vol + o_\varepsilon(1)
	= \int_{U_\alpha}\beta\wedge\psi + o_\varepsilon(1).
	\]
	Using a finite cover $\{U_\alpha\}$ and partition of unity yields a global sequence of closed integral cycles (still denoted $T_\varepsilon$)
	with $\Mass(T_\varepsilon)=m\int_X\beta\wedge\psi + o_\varepsilon(1)$ and $\Def_{\mathrm{cal}}(T_\varepsilon)\to 0$.
	\editamir{To enforce the exact homology class $\mathrm{PD}(m[\gamma])$ (modulo torsion), one may refine the mesh so that the period contributions of individual plaques are uniformly tiny and then apply the same fixed-dimensional rounding / lattice-discreteness argument used later in Proposition~\ref{prop:cohomology-match}.}
	Thus $\beta$ is SYR--realizable in the sense of Definition~\ref{def:syr}.
\end{proof}

\begin{corollary}[Closure of the program under LICD]\label{cor:closure-licd}
	If a given cone–valued representative $\beta$ satisfies LICD, then the sequence produced by Theorem~\ref{thm:classical-syr-licd}
	and Theorem~\ref{thm:realization-from-almost} yields a calibrated integral current
	realizing $[\gamma]$ as a rational algebraic cycle. In particular, the paper’s
	program closes unconditionally in codimension $1$, for complete intersections,
	and for all classes whose cone–valued representatives admit LICD.
\end{corollary}
\begin{proof}
Assume the cone-valued representative $\beta$ satisfies LICD.
By the theorem ``Classical SYR under LICD'', there exists an integer $m\ge 1$ and a sequence of integral $(2n-2p)$-cycles $T_k$
with $\partial T_k=0$, $[T_k]=\mathrm{PD}(m[\gamma])$, and
\[
\Mass(T_k)\to \bigl\langle \mathrm{PD}(m[\gamma]),[\psi]\bigr\rangle
= m\int_X \beta\wedge\psi.
\]
Applying the theorem ``Realization from almost--calibrated sequences'' yields, after passing to a subsequence, a weak limit $T$
with $[T]=\mathrm{PD}(m[\gamma])$, $\Mass(T)=m\int_X \beta\wedge\psi$, and $T$ $\psi$-calibrated.
By Harvey--Lawson structure theory, a $\psi$-calibrated integral cycle in a K\"ahler manifold is a positive sum of currents of integration
over irreducible complex analytic subvarieties of codimension $p$.
Since $X$ is projective, Chow's theorem identifies these analytic cycles with algebraic cycles.
Dividing by $m$ expresses $[\gamma]$ as a rational algebraic cycle.
\end{proof}



% ============================================================
% RIGOROUS SYR CONSTRUCTION (GENERAL p)
% ============================================================

\subsection*{Step 1: Carath\'eodory decomposition in the Hermitian model}

At each $x\in X$, identify $\Lambda^{p,p}(T_x^*X)$ with a finite-dimensional
real vector space $\mathcal{V}_x$ equipped with the inner product induced by
the K\"ahler metric, and let $K_p(x)\subset \mathcal{V}_x$ be the closed convex
cone of strongly positive $(p,p)$-forms.
Each complex $(n-p)$-plane $P\subset T_xX$ determines an extremal ray of $K_p(x)$;
let $\xi_P\in K_p(x)$ denote a chosen generator of this ray, normalized so that
$\langle \xi_P,\psi_x\rangle=1$ (equivalently $\xi_P\wedge\psi_x=\omega_x^n/n!$).

Fix the positive ``trace'' functional $t(x):=\langle \beta(x),\psi_x\rangle=\frac{\beta\wedge\psi}{\omega^n/n!}(x)$.
Then $\widehat\beta(x):=\beta(x)/t(x)$ (on the set $\{t(x)>0\}$) lies in the convex
hull of the normalized generators $\{\xi_P:\ P\in \Gr_{n-p}(T_xX)\}$.
By Carath\'eodory's theorem in $\R^{D}$, $\widehat\beta(x)$ can be written as a convex
combination of at most $D+1$ such generators, where $D=\dim(\mathcal{V}_x)=\binom{n}{p}^2$
is independent of $x$.


\begin{editamirblockNEW}
\iffalse
% ======= Old (truncated) version kept for traceability; superseded in Step 85 =======

\begin{lemma}[Uniform Carath\'eodory decomposition]\label{lem:caratheodory-general-old}
Let $X$ be a compact K\"ahler $n$--fold and fix $p\in\{0,\dots,n\}$.
For each $x\in X$ let $\mathcal V_x:=\Lambda^{p,p}(T_x^*X)_{\R}$ (real $(p,p)$--forms) and let
$K_p(x)\subset \mathcal V_x$ be the closed convex cone of \emph{strongly positive} $(p,p)$--forms.
For each complex $(n-p)$--plane $P\subset T_xX$ let $\xi_P\in K_p(x)$ denote the (normalized) generator of the corresponding extremal ray,
chosen so that $\langle \xi_P,\psi_x\rangle=1$.

Then there exists a number \editamir{$N_{\mathrm{Car}}=N_{\mathrm{Car}}(n,p)$} (one may take $N_{\mathrm{Car}}=\dim(\mathcal V_x)+1=\binom{n}{p}^2+1$) such that for every $x\in X$ and every
$\beta(x)\in K_p(x)$ there exist complex $(n-p)$--planes $P_{x,1},\dots,P_{x,N_{\mathrm{Car}}}\subset T_xX$ and weights
$\theta_{x,j}\ge 0$ with $\sum_{j=1}^{N_{\mathrm{Car}}}\theta_{x,j}=1$ such that
\[
\beta(x)=t(x)\sum_{j=1}^{N_{\mathrm{Car}}}\theta_{x,j}\,\xi_{P_{x,j}},
\qquad t(x):=\langle \beta(x),\psi_x\rangle.
\]
\end{lemma}

\begin{proof}
If $\beta(x)=0$ there is nothing to prove.  Otherwise $t(x)=\langle \beta(x),\psi_x\rangle>0$ and the normalized form
$\widehat\beta(x):=\beta(x)/t(x)$ lies in the affine hyperplane
$H_x:=\{v\in \mathcal V_x:\ \langle v,\psi_x\rangle=1\}$.

By the standard description of the strongly positive cone (see e.g.\ Demailly~\cite[\S~III.1]{Demailly12}),
$K_p(x)$ is the closed convex cone generated by the extremal rays $\R_{\ge 0}\,\xi_P$ as $P$ ranges over complex $(n-p)$--planes.
Intersecting with $H_x$ shows that $\widehat\beta(x)$ lies in the compact convex set
$\mathrm{conv}\{\xi_P:\ P\in \Gr_{n-p}(T_xX)\}\subset H_x$.
Carath\'eodory's theorem in the finite-dimensional vector space $H_x\cong\R^{D-1}$ yields a convex representation using at most $D$ points,
hence using at most $N_{\mathrm{Car}}=D+1$ points in $\mathcal V_x$.
\end{proof}

\medskip\noindent
\textbf{Remark (what is actually used later).}
In the global construction we only require such decompositions at the finitely many cube base points $x_Q$ (Substep~4.1),
so no global measurable selection or continuity-in-$x$ statement is needed for the subsequent arguments.
\fi
% ======= Referee Patch (Step 85): expanded, checkable statement/proof (no change of meaning) =======

\begin{lemma}[Uniform Carath\'eodory decomposition]\label{lem:caratheodory-general}
Let $X$ be a compact K\"ahler $n$--fold and fix $p\in\{0,\dots,n\}$.
For each $x\in X$ let $\mathcal V_x:=\Lambda^{p,p}(T_x^*X)_{\R}$ (real $(p,p)$--forms) and let
$K_p(x)\subset \mathcal V_x$ be the closed convex cone of \emph{strongly positive} $(p,p)$--forms.
For each complex $(n-p)$--plane $P\subset T_xX$ let $\xi_P\in K_p(x)$ denote the (normalized) generator of the corresponding extremal ray,
chosen so that $\langle \xi_P,\psi_x\rangle=1$.

Then there exists a number \editamir{$N_{\mathrm{Car}}=N_{\mathrm{Car}}(n,p)$} (one may take $N_{\mathrm{Car}}=\dim_{\R}(\mathcal V_x)+1=\binom{n}{p}^2+1$) such that for every $x\in X$ and every
$\beta(x)\in K_p(x)$ there exist complex $(n-p)$--planes $P_{x,1},\dots,P_{x,N}\subset T_xX$ and weights
$\theta_{x,j}\ge 0$ with $\sum_{j=1}^{N_{\mathrm{Car}}}\theta_{x,j}=1$ such that
\[
\beta(x)=t(x)\sum_{j=1}^{N_{\mathrm{Car}}}\theta_{x,j}\,\xi_{P_{x,j}},
\qquad t(x):=\langle \beta(x),\psi_x\rangle.
\]
\end{lemma}

\begin{proof}
If $\beta(x)=0$ there is nothing to prove. Otherwise $t(x)=\langle \beta(x),\psi_x\rangle>0$ and the normalized form
$\widehat\beta(x):=\beta(x)/t(x)$ lies in the affine hyperplane
\[
H_x:=\{v\in \mathcal V_x:\ \langle v,\psi_x\rangle=1\}.
\]
By the standard description of the strongly positive cone (see e.g.\ Demailly~\cite[\S~III.1]{Demailly12}),
$K_p(x)$ is the closed convex cone generated by the extremal rays $\R_{\ge 0}\,\xi_P$ as $P$ ranges over complex $(n-p)$--planes.
Intersecting with $H_x$ shows that $\widehat\beta(x)$ lies in the compact convex set
\[
\mathrm{conv}\{\xi_P:\ P\in \Gr_{n-p}(T_xX)\}\subset H_x.
\]
Set $D:=\dim_{\R}(\mathcal V_x)=\binom{n}{p}^2$, so $\dim(H_x)=D-1$.
Carath\'eodory's theorem in the affine space $H_x\cong \R^{D-1}$ yields a convex representation of $\widehat\beta(x)$
using at most $D$ points from $\{\xi_P\}$; padding with zero weights gives a representation with at most $N_{\mathrm{Car}}=D+1$ points.
Multiplying by $t(x)$ gives the claimed decomposition of $\beta(x)$.
\end{proof}

\medskip\noindent
\textbf{Remark (what is actually used later).}
In the global construction we only require such decompositions at the finitely many cube base points $x_Q$
(Substep~4.1), so no global measurable selection or continuity-in-$x$ statement is needed for the subsequent arguments.
\end{editamirblockNEW}

\begin{editblock}
\begin{lemma}[Lipschitz weights from a strongly convex simplex fit]\label{lem:lipschitz-qp-weights}
Let $V$ be a finite-dimensional real inner-product space and let $\xi_1,\dots,\xi_M\in V$.
Let $\Delta_M:=\{w\in\R^M:\ w_i\ge 0,\ \sum_{i=1}^M w_i=1\}$ be the probability simplex.
Fix $\lambda>0$.
For each $b\in V$ define
\[
w(b)\ :=\ \arg\min_{w\in\Delta_M}\ \frac12\Bigl\|\sum_{i=1}^M w_i\xi_i-b\Bigr\|^2+\frac{\lambda}{2}\|w\|^2.
\]
Then:
\begin{enumerate}
\item[\textnormal{(i)}] The minimizer $w(b)$ exists and is unique.
\item[\textnormal{(ii)}] The map $b\mapsto w(b)$ is Lipschitz.  Writing $A:\R^M\to V$ for the linear map
$A e_i:=\xi_i$, one has
\[
\|w(b)-w(b')\|\ \le\ \frac{\|A\|_{\mathrm{op}}}{\lambda}\,\|b-b'\|\qquad\text{for all }b,b'\in V.
\]
\end{enumerate}
\end{lemma}

\begin{proof}
Existence follows from compactness of $\Delta_M$ and continuity of the objective.
Uniqueness follows because the objective is $\lambda$--strongly convex in $w$.

Let $w=w(b)$ and $w'=w(b')$.
The first-order optimality conditions for the constrained minimization read
\[
0\ \in\ A^\top(Aw-b)+\lambda w\ +\ N_{\Delta_M}(w),
\qquad
0\ \in\ A^\top(Aw'-b')+\lambda w'\ +\ N_{\Delta_M}(w'),
\]
where $N_{\Delta_M}$ is the normal cone mapping and $A^\top$ denotes the adjoint.
Choose $\nu\in N_{\Delta_M}(w)$ and $\nu'\in N_{\Delta_M}(w')$ realizing these inclusions.
Subtract the two relations and take the inner product with $(w-w')$ to obtain
\[
\langle A^\top A(w-w'),\,w-w'\rangle\ +\ \lambda\|w-w'\|^2\ +\ \langle \nu-\nu',\,w-w'\rangle
\ =\ \langle A^\top(b-b'),\,w-w'\rangle.
\]
Since $A^\top A$ is positive semidefinite and $N_{\Delta_M}$ is monotone, one has
$\langle A^\top A(w-w'),w-w'\rangle\ge 0$ and $\langle \nu-\nu',w-w'\rangle\ge 0$.
Hence
\[
\lambda\|w-w'\|^2\ \le\ \|A^\top(b-b')\|\,\|w-w'\|
\ \le\ \|A\|_{\mathrm{op}}\|b-b'\|\,\|w-w'\|.
\]
If $w\neq w'$, cancel $\|w-w'\|$; otherwise the desired bound is trivial.  This gives
$\|w-w'\|\le (\|A\|_{\mathrm{op}}/\lambda)\,\|b-b'\|$.
\end{proof}

\begin{remark}[Stable direction labeling via a growing net]\label{rem:direction-net-qp}
In a holomorphic chart $U\subset\C^n$, the calibrated directions are precisely the complex $(n-p)$--planes.
Fix a scale $h$ and choose an $\varepsilon_h$--net $\{P_1,\dots,P_M\}\subset G_{\C}(n-p,n)$ with $\varepsilon_h\ll h$.
For each $x\in U$, let $\xi_i(x)$ denote the corresponding normalized generator in $K_p(x)$ (so $\langle \xi_i(x),\psi_x\rangle=1$).

\smallskip\noindent
Given a smooth normalized target field $b(x)=\widehat\beta(x)$, one may choose \emph{globally labeled} coefficients by applying
Lemma~\ref{lem:lipschitz-qp-weights} (with $V=\Lambda^{p,p}(T_x^*X)$ in a fixed trivialization on $U$) to obtain
weights $w_i(x)$ depending \emph{Lipschitzly} on $b(x)$.  Since $b$ varies by $O(h)$ between adjacent mesh-$h$ cells,
the weights $w_i$ vary by $O(h)$ as well.  This gives a canonical pairing of directions across neighbors (index $i=i'$)
and reduces “stable direction labeling’’ to the quantitative choice of $\varepsilon_h$ and the regularization parameter $\lambda$.
\end{remark}
\end{editblock}

% ------------------------------------------------------------
\subsection*{Step 2: Projective tangential approximation with $C^1$ control}

Fix an ample line bundle $L\to X$ with a Hermitian metric whose curvature
form equals $\omega$.  For $N\in\N$ large, consider the complete linear
system $|L^{\otimes N}|$.
\editamir{(Parameter convention: throughout the holomorphic/Bergman manufacturing block, $N$ denotes the tensor power of $L$ controlling the Bergman scale $N^{-1/2}$; the cohomology multiplier remains $m$ as in Definition~\ref{def:syr}.)}

\begin{editamirblock}
\begin{lemma}[Jet surjectivity for ample powers (pointwise and for finite sets)]\label{lem:jet-surjectivity}
Let $X$ be a smooth complex projective manifold and $L\to X$ an ample line bundle.
Fix an integer $k\ge 1$.

\smallskip
\noindent\textup{(i)} For each point $x\in X$ there exists an integer $N_0(k,x)$ such that for all $N\ge N_0(k,x)$
the natural evaluation map on $k$-jets
\[
H^0(X,L^N)\longrightarrow J^k_x(L^N)
\]
\editamir{Here $J^k_x(L^N):=\mathcal O_X(L^N)_x/\mathfrak m_x^{k+1}\mathcal O_X(L^N)_x$ denotes the $k$-jet space at $x$.}

is surjective.

\smallskip
\noindent\textup{(ii)} More generally, for any \emph{finite} set $S\subset X$ there exists $N_0(k,S)$ such that for all $N\ge N_0(k,S)$
the joint evaluation map
\[
H^0(X,L^N)\longrightarrow \bigoplus_{x\in S} J^k_x(L^N)
\]
is surjective.  In particular (taking $k=1$), prescribed values and first derivatives can be realized simultaneously at finitely many points.
\end{lemma}

\begin{proof}
For \textup{(i)}, let $\mathfrak{m}_x\subset \mathcal{O}_X$ be the maximal ideal at $x$ and consider the exact sequence
\[
0\to L^N\otimes \mathfrak{m}_x^{k+1}\to L^N \to
L^N\otimes \mathcal{O}_X/\mathfrak{m}_x^{k+1}\to 0.
\]
Since $\mathfrak{m}_x^{k+1}$ is coherent and $L$ is ample, Serre vanishing gives
$H^1(X,L^N\otimes \mathfrak{m}_x^{k+1})=0$ for all $N\ge N_0(k,x)$ (see \cite[III, Thm.~5.2]{Hartshorne77}).
Taking global sections yields the desired surjection
\[
H^0(X,L^N)\twoheadrightarrow H^0\!\left(X,L^N\otimes \mathcal{O}_X/\mathfrak{m}_x^{k+1}\right)\cong J^k_x(L^N).
\]

For \textup{(ii)}, apply the same argument to the finite ``fat point'' subscheme
$Z:=\sum_{x\in S}(k{+}1)\,x$ with ideal sheaf $\mathcal{I}_Z:=\bigcap_{x\in S}\mathfrak{m}_x^{k+1}$.
Serre vanishing gives $H^1(X,L^N\otimes \mathcal{I}_Z)=0$ for all $N\ge N_0(k,S)$, hence
\[
H^0(X,L^N)\twoheadrightarrow H^0\!\left(X,L^N\otimes \mathcal{O}_X/\mathcal{I}_Z\right)\cong
\bigoplus_{x\in S} J^k_x(L^N),
\]
as claimed.
\end{proof}
\end{editamirblock}


\begin{lemma}[Uniform $C^1$ control on $N^{-1/2}$-balls via Bergman kernels]
\label{lem:bergman-control}
\begin{editamirblock}
Fix $\varepsilon>0$.  There exists \editamir{$N_1(\varepsilon)$} such that for all
$N\ge \editamir{N_1(\varepsilon)}$, each $x\in X$, and each collection of $p$
complex covectors $\lambda_1,\ldots,\lambda_p\in T_x^{*(1,0)}X$, there exist
sections $s_1,\ldots,s_p\in H^0(X,L^{\editamir{N}})$ with the following properties
in normal holomorphic coordinates centered at $x$:
\begin{enumerate}
\item[\textnormal{(i)}] $s_i(x)=0$ and $ds_i(x)=\lambda_i$ for each $i$;
\item[\textnormal{(ii)}] on the geodesic ball $B_{c\,\editamir{N^{-1/2}}}(x)$
(for a universal constant $c>0$ depending only on $(X,\omega)$),
the gradients satisfy
\[
\|ds_i(y)-\lambda_i\|\le \varepsilon\,\max_{1\le j\le p}\|\lambda_j\|
\quad\text{for all } y\in B_{c\,\editamir{N^{-1/2}}}(x).
\]
\end{enumerate}

\end{editamirblock}
\end{lemma}

\begin{proof}
\begin{editamirblock}
\editamir{
\noindent\textbf{Analytic input (Bergman kernel at the $N^{-1/2}$ scale).}
We invoke a \emph{near off-diagonal} Bergman kernel expansion with \emph{derivative control up to order $2$}
on rescaled balls of radius $\asymp N^{-1/2}$ for the positive Hermitian line bundle $(L,h)$ with curvature $\omega$.
Concretely, we need that in normal coordinates and local unitary frames,
the rescaled kernel admits a $C^2$ asymptotic expansion on $\{|Z|,|Z'|\le \sigma\}$ (fixed $\sigma>0$),
so that first derivatives of the kernel (hence gradients of the induced peak sections) vary by $O(N^{-1/2})$.
This is a standard consequence of the off-diagonal expansion in \cite{MaMarinescu13OffDiag} (see also the classical
diagonal expansions \cite{Tian90,Catlin99,Zelditch98} and the systematic treatment \cite{MaMarinescu07}).
In the present paper, $X$ is compact K\"ahler and $(L,h)$ is positive, so the hypotheses of the cited off-diagonal expansion apply.
We therefore take the $C^2$ near off-diagonal Bergman kernel expansion on the $N^{-1/2}$ scale as a published analytic input
(e.g.\ \cite{MaMarinescu13OffDiag,MaMarinescu07}).}

Set $M:=\max_{1\le j\le p}\|\lambda_j\|$.  If $M=0$, take $s_i\equiv 0$ for all $i$, so assume $M>0$ and replace $\lambda_i$ by $\lambda_i/M$.
Thus we may assume $M=1$.

Fix $x\in X$.  Choose $K$--coordinates and a local holomorphic frame for $(L,h)$ centered at $x$
(as in \cite[\S4.1]{MaMarinescu07}), so that all estimates below are uniform in $x$.
Let $P_N(\cdot,\cdot)$ denote the Bergman kernel of $H^0(X,L^N)$ for the $L^2$ inner product induced by $(h^N,\omega)$, written in these coordinates and this frame.
The \emph{near off-diagonal Bergman kernel expansion} on the Bergman scale
(e.g. \cite[Thm.~1]{MaMarinescu13OffDiag}; see also \cite{MaMarinescu13OffDiag,Tian90,Catlin99,Zelditch98,MaMarinescu07})
implies the following: there exists a universal $\sigma>0$ such that, in rescaled variables
$Z=\sqrt N\,z$ and $Z'=\sqrt N\,z'$, we have

\medskip\noindent\textbf{Normalization check (matching the literature).}
We work with the \emph{prequantum convention}
\[
\omega=\frac{\sqrt{-1}}{2\pi}R^{L},\qquad dv_X=\frac{\omega^n}{n!},
\]
which is the standing assumption in \cite{MaMarinescu13OffDiag}. Under this convention, the model kernel
$P(Z,Z')$ in \cite[(2)]{MaMarinescu13OffDiag} coincides with our Bargmann--Fock kernel
$P_{\mathrm{BF}}(Z,Z')=\exp\!\big(\pi Z\cdot\overline{Z'}-\tfrac{\pi}{2}(|Z|^2+|Z'|^2)\big)$.
(If one uses a different curvature normalization elsewhere, the same statements hold after the corresponding constant
rescaling of the normal coordinates $Z\mapsto cZ$.)
\medskip
\[
N^{-n}\,P_N\!\left(\tfrac Z{\sqrt N},\tfrac{Z'}{\sqrt N}\right)
=
P_{\mathrm{BF}}(Z,Z') + O(N^{-1/2})
\quad\text{in }C^{2}(\{|Z|,|Z'|\le \sigma\}),
\]
where $P_{\mathrm{BF}}(Z,Z')=\exp\!\big(\pi Z\cdot\overline{Z'}-\tfrac{\pi}{2}(|Z|^2+|Z'|^2)\big)$ is the Bargmann--Fock kernel on $\C^n$.

Using the reproducing-kernel representation, differentiate the kernel in the \emph{second} variable at $Z'=0$ to obtain holomorphic sections
\editamir{(for each fixed $Z'$, the map $Z\mapsto P_N(\tfrac Z{\sqrt N},\tfrac{Z'}{\sqrt N})$ represents a holomorphic section in the first variable, so taking a derivative in $Z'$ and evaluating at $Z'=0$ preserves holomorphicity in $Z$).}
More precisely, for $a=1,\dots,n$ define
\[
s_{a,N}(z)\ :=\ N^{-(n+1/2)}\,\partial_{\overline{Z'_a}}
\Big(P_N(\tfrac Z{\sqrt N},\tfrac{Z'}{\sqrt N})\Big)\Big|_{Z'=0},
\]
viewed as a local representative via the chosen frame.
Since $P_{\mathrm{BF}}(Z,Z')$ is linear in $\overline{Z'}$ to first order, the $C^2$--control above implies:
\begin{enumerate}
\item $s_{a,N}(0)=0$ for all $a$;
\item the $1$--jets satisfy $ds_{a,N}(0)=\editamir{\pi\,}dz^a+o(1)$ as $N\to\infty$;
\item moreover, $\sup_{|Z|\le \sigma}\|ds_{a,N}(Z)-ds_{a,N}(0)\|=o(1)$ as $N\to\infty$.
\end{enumerate}
In particular, for $N\ge \editamir{N_1(\varepsilon)}$ the matrix $A_N:=(ds_{a,N}(0))_{a=1}^n$ is invertible and
\[
\sup_{|Z|\le \sigma}\|ds_{a,N}(Z)-ds_{a,N}(0)\|\le \varepsilon
\qquad\text{for all }a=1,\dots,n.
\]

Now write each $\lambda_i=\sum_{a=1}^n \lambda_i^a\,dz^a$ in these coordinates and set
\[
s_i \ :=\ \sum_{a=1}^n b_{i,a}\,s_{a,N},
\qquad\text{where } b_i:=A_N^{-1}(\lambda_i).
\]
Then $s_i(0)=0$ and $ds_i(0)=\lambda_i$ \emph{exactly} by construction.
Using the derivative control above and $\|A_N^{-1}\|=O(1)$ for $N$ large, we obtain
\[
\|ds_i(Z)-\lambda_i\|\le \varepsilon\max_{1\le j\le p}\|\lambda_j\|
\qquad\text{for all }|Z|\le \sigma.
\]
Returning to the original variables gives the desired estimate on $B_{c\,\editamir{N^{-1/2}}}(x)$ with $c:=\sigma$.

Finally, undo the normalization by multiplying the constructed sections by $M$, which yields the general case.

\end{editamirblock}
\end{proof}


\begin{editblock}
\begin{lemma}[Graph control from uniform gradient control]\label{lem:graph-from-grad}
Let $U\subset\C^n$ be a ball and let $\lambda_1,\dots,\lambda_p\in(\C^n)^*$ be complex covectors with linearly independent real and imaginary parts,
so that $\Pi:=\bigcap_{i=1}^p \ker(\lambda_i)$ is a complex $(n-p)$-plane.
Let $s_1,\dots,s_p:U\to\C$ be holomorphic functions such that $s_i(0)=0$ and
\[
\sup_{y\in U}\|ds_i(y)-\lambda_i\|\le \varepsilon
\qquad\text{for all }i=1,\dots,p,
\]
with $\varepsilon$ small compared to $\min\{\|\lambda_i\|\}$.
Then the common zero set $Y:=\{s_1=\cdots=s_p=0\}\cap U$ is a smooth complex submanifold of $U$ and, after shrinking $U$ if needed,
$Y$ is a $C^1$ graph over $\Pi$ with slope $O(\varepsilon)$.
In particular,
\[
\sup_{y\in Y}\angle(T_yY,\Pi)\le C\,\varepsilon
\]
for a constant $C$ depending only on $(n,p)$ and the conditioning of $\{\lambda_i\}$.
\end{lemma}


\begin{proof}
Let $S=(s_1,\dots,s_p):U\to\C^p$.  The differential $dS(y)$ is uniformly close to the constant complex-linear map
$\Lambda=(\lambda_1,\dots,\lambda_p)$ in operator norm.  Since $\Lambda$ is surjective (its kernel is the complex $(n-p)$--plane $\Pi$),
for $\varepsilon$ sufficiently small the perturbation bound implies $dS(y)$ is surjective for all $y\in U$.
Hence $Y=S^{-1}(0)$ is a smooth complex submanifold of $U$ by the holomorphic implicit function theorem.

Write $\C^n=\Pi\oplus \Pi^\perp$ and let $(u,w)$ denote the corresponding coordinates.
Since $\partial_w S$ is uniformly close to $\partial_w\Lambda$ and $\partial_w\Lambda:\Pi^\perp\to\C^p$ is invertible,
the implicit function theorem yields (after shrinking $U$ if needed) a $C^1$ map $g$ with $Y=\{(u,g(u))\}$.
Differentiating $S(u,g(u))=0$ gives $Dg=-(\partial_w S)^{-1}\partial_u S$, so the same uniform closeness estimates imply
$\|Dg\|\le C\,\varepsilon$ for a constant $C$ depending only on $(n,p)$ and the conditioning of $\{\lambda_i\}$.
\end{proof}

\begin{editamirblockNEW}
\noindent\textbf{Referee addendum (explicit slope constant and usable domain).}
In applications, we take $U=B_r(0)\subset\C^n$ and apply the above argument on a \emph{concentric subball}
$U_{1/2}:=B_{r/2}(0)$, so that the implicit-function graph is defined on the full base $\Pi\cap U_{1/2}$.
More precisely, write $\C^n=\Pi\oplus\Pi^\perp$ and let $\Lambda=(\lambda_1,\dots,\lambda_p)$.
Set $\kappa:=\|(\partial_w\Lambda)^{-1}\|$ (finite by transversality). For $\varepsilon\le (4\kappa)^{-1}$ one has
$\|(\partial_wS)^{-1}\|\le 2\kappa$ on $U$, hence the graph map satisfies the uniform estimate
\[
\|Dg\|_{C^0(\Pi\cap U_{1/2})}\ \le\ C_{\mathrm{graph}}\,\varepsilon,
\qquad C_{\mathrm{graph}}:=2\kappa,
\]
and $Y\cap U_{1/2}$ is a single $C^1$ graph over $\Pi\cap U_{1/2}$.
\end{editamirblockNEW}

\end{editblock}

\begin{proposition}[Projective tangential approximation with $C^1$ control]

\label{prop:tangent-approx-full}
Let $x\in X$ and let $\Pi\subset T_xX$ be a complex $(n-p)$-plane.
For every $\varepsilon>0$ there exist $N\gg 0$ and a smooth complete
intersection
\[
Y = \{s_1=0\}\cap \cdots \cap \{s_p=0\}\subset X,
\qquad s_i\in H^0(X,L^N),
\]
such that $x\in Y$, $Y$ is smooth in a neighborhood of $x$, and
\[
\angle\bigl(T_yY,\Pi\bigr)<\varepsilon
\quad\text{for all } y\in B_{c\,N^{-1/2}}(x).
\]
Moreover, $Y$ is $\psi$-calibrated (being a complex submanifold).
\end{proposition}



\begin{editamirblockNEW}
\editamir{\noindent\textbf{Referee correction (removing an unnecessary Bertini/global-smoothness dependency).}
Downstream we only use that the defining sections are \emph{transverse on the Bergman ball}
$B_{c\,N^{-1/2}}(x)$, which yields a \emph{single-sheet $C^1$ graph} there.
No later step requires $Y$ to be globally smooth on all of $X$.
Accordingly, the conclusion may be read as producing an \emph{algebraic complete-intersection cycle}
(possibly with singularities away from $B_{c\,N^{-1/2}}(x)$), whose associated integration current is
$\psi$--calibrated in the sense of Wirtinger/Harvey--Lawson \cite{HL82}.}
\end{editamirblockNEW}

\begin{proof}
Choose covectors $\lambda_1,\ldots,\lambda_p\in T_x^*X$ whose common
kernel equals $\Pi$.  By Lemma~\ref{lem:bergman-control}, pick
$s_1,\ldots,s_p$ with $s_i(x)=0$, $ds_i(x)=\lambda_i$, and
$\|ds_i(y)-\lambda_i\|<\varepsilon/p$ on $B_{c\,N^{-1/2}}(x)$.

\iffalse
For $N\gg 0$ and after a small generic perturbation inside the
finite-dimensional linear system (which does not change jets at $x$
nor the $C^1$ estimates on the small ball), Bertini's theorem ensures
that $Y$ is smooth and $\{ds_1(y),\ldots,ds_p(y)\}$ are linearly
independent on the ball.
\fi

\begin{editamirblockNEW}
\editamir{\noindent\textbf{Referee note.} The above generic-perturbation/Bertini sentence is not needed here.
We keep the sections $s_i$ produced by Lemma~\ref{lem:bergman-control}: by construction they satisfy
$\{ds_1(y),\ldots,ds_p(y)\}$ linearly independent on $B_{c\,N^{-1/2}}(x)$, hence $Y$ is smooth on that ball.
Possible singularities of the global complete intersection away from the ball do not affect the local graph estimate
and do not enter later arguments, which work at the level of integral currents/algebraic cycles.}
\end{editamirblockNEW}


The complex normal space to $Y$ at $y$ is spanned by
$\{ds_1(y),\ldots,ds_p(y)\}$, which is $\varepsilon$-close to
$\{\lambda_1,\ldots,\lambda_p\}$ in the Grassmannian metric.
Hence $T_yY$ is $\varepsilon$-close to $\Pi$ for all $y$ in the ball.

Since $Y$ is a complex submanifold of a K\"ahler manifold, it is
automatically calibrated by $\psi=\omega^{n-p}/(n-p)!$.
\begin{editamirblockNEW}
\editamir{More generally, even if $Y$ is only an analytic/algebraic complete intersection with singularities away from the ball,
the associated integration current $[Y]$ is $\psi$--calibrated (Wirtinger), hence strongly positive and closed; see \cite{HL82,King71}.}
\end{editamirblockNEW}

\end{proof}

\begin{proposition}[Holomorphic density of calibrated directions]
\label{prop:dense-holo}
For every compact $K\subset X$ and $\varepsilon>0$ there exist finitely
many points $x_1,\dots,x_A\in K$ and, for each $\alpha$, finitely many
$\psi$-calibrated $(n-p)$-submanifolds $Y_{\alpha,1},\ldots,Y_{\alpha,N_\alpha}$
(each a smooth complete intersection in $|L^N|$ for some large $N$) such that:
\begin{enumerate}
\item[\textnormal{(i)}] $K\subset \bigcup_{\alpha=1}^A B(x_\alpha,\varepsilon)$;
\item[\textnormal{(ii)}] for each fixed $\alpha$ and each calibrated plane $\Pi\subset T_{x_\alpha}X$
there exists $j\in\{1,\dots,N_\alpha\}$ with
$\mathrm{dist}\!\bigl(T_{x_\alpha}Y_{\alpha,j},\Pi\bigr)<\varepsilon$.
\end{enumerate}
Moreover, each $Y_{\alpha,j}$ can be chosen so that on the Bergman ball $B_{c\,N^{-1/2}}(x_\alpha)$
its tangent planes remain $\varepsilon$-close to $T_{x_\alpha}Y_{\alpha,j}$ (as in Proposition~\ref{prop:tangent-approx-full}).
\end{proposition}

\begin{proof}
Cover $K$ by finitely many coordinate balls $\{B_\alpha\}$ centered at
points $\{x_\alpha\}$.  Refining if needed, assume $B(x_\alpha,\varepsilon)\subset B_\alpha$ so that $K\subset\bigcup_\alpha B(x_\alpha,\varepsilon)$.
On each center $x_\alpha$, take an
$\varepsilon/2$-net of calibrated planes
$\{\Pi_{\alpha,1},\ldots,\Pi_{\alpha,N_\alpha}\}$ in the compact fiber
$G_{n-p}(T_{x_\alpha}X)$.  Apply Proposition~\ref{prop:tangent-approx-full}
to realize each net direction by a calibrated complete intersection
$Y_{\alpha,j}$ through $x_\alpha$ with tangent plane $\varepsilon/2$-close
to $\Pi_{\alpha,j}$ on a ball of radius $c\,N^{-1/2}$. \editamir{(Since there are only finitely many directions in the net, we may take a single $N$ large enough so that all of the resulting local graph estimates hold on the same Bergman radius $c\,N^{-1/2}$.)}

By the $C^1$ control in Proposition~\ref{prop:tangent-approx-full}, for $N$ large enough the tangent planes of each $Y_{\alpha,j}$
remain $\varepsilon$-close to $T_{x_\alpha}Y_{\alpha,j}$ throughout $B_{c\,N^{-1/2}}(x_\alpha)$.
Thus, at each fixed center $x_\alpha$, the finite family $\{Y_{\alpha,j}\}_j$ realizes an $\varepsilon$-net of calibrated directions in $T_{x_\alpha}X$,
and the centers $\{x_\alpha\}$ cover $K$ at scale $\varepsilon$.
\end{proof}

% ------------------------------------------------------------
\subsection*{Step 3: Local calibrated laminates on small cubes (Theorem B)}

This step constructs multiple disjoint calibrated sheets on each cube $Q$
with prescribed tangent directions and mass fractions.

\begin{theorem}[Local multi-sheet construction]\label{thm:local-sheets}
Let $Q\subset X$ be a small coordinate cube.  Let
$\Pi_1,\ldots,\Pi_J\in \Gr_{n-p}(TQ)$ be constant $(n-p)$-planes \editamir{(assumed complex/$\psi$-calibrated in the intended application)}, and let
$\theta_1,\ldots,\theta_J\in\Q_{>0}$ with $\sum_j\theta_j=1$.
For every $\varepsilon,\delta>0$, there exist smooth $\psi$-calibrated
complete intersections $\{Y_j^a\}_{j,a}$ in $X$ such that:
\begin{enumerate}
\item[\textnormal{(i)}] \textbf{Angle control:}
$\sup_{y\in Q}\angle(T_yY_j^a,\Pi_j)<\varepsilon$;
\item[\textnormal{(ii)}] \textbf{Mass fractions:}
$\bigl|\Mass(Y_j^a\llcorner Q)/\sum_{i,b}\Mass(Y_i^b\llcorner Q)-\theta_j\bigr|<\delta$;
\item[\textnormal{(iii)}] \textbf{Disjointness (within each family):} \editamir{For each fixed $j$, the sheets $\{Y_j^a\}_a$ are pairwise disjoint on $Q$ (no disjointness is asserted between different $j$).}
\item[\textnormal{(iv)}] \textbf{Boundary control:}
$\partial([Y_j^a]\llcorner Q)$ is supported on $\partial Q$.
\end{enumerate}
\end{theorem}

\begin{proof}
The proof proceeds in four substeps.

\medskip\noindent
\textbf{Substep 3.1: Local setup and flattening.}
\editamir{Write $h:=\mathrm{diam}(Q)$. For $Q$ small enough (equivalently, for $h$ small enough), there is a holomorphic chart}
$\Phi:U\to B(0,2)\subset\C^n$ with $Q\subset U$,
$\Phi(Q)\subset [-1,1]^{2n}\subset\C^n$, and the K\"ahler form $\omega$
and calibration $\psi=\omega^{n-p}/(n-p)!$ are $C^1$-close to the flat
model on $\C^n$.  The calibration cone $K_{n-p}(x)\subset\Gr_{n-p}(T_xX)$
varies smoothly and stays uniformly close to the flat cone of complex
$(n-p)$-planes.  We prove Theorem~\ref{thm:local-sheets} in this flattened
model; everything is diffeomorphism-invariant, and volume/mass distortions
are controlled by the uniform $C^1$-closeness of the metric.

\medskip\noindent
\textbf{Substep 3.2: Fix calibrated target planes (no minimization needed).}
\begin{editamirblock}
In the closure-chain application, each target direction $\Pi_j$ comes from the calibrated cone decomposition
(hence is already a complex $(n-p)$-plane, i.e. $\psi$-calibrated).  We therefore set
\[
\widetilde\Pi_j := \Pi_j
\]
and no ``projection to the nearest calibrated plane'' is required.
\par\smallskip
\editamir{Referee note: claiming pairwise disjointness for sheets built from \emph{different} directions is generally false (e.g. for $p=1$, non-parallel complex hypersurfaces intersect locally).  Disjointness is only enforced within each fixed-direction family $\{Y_j^a\}_a$.}
\end{editamirblock}

\medskip\noindent
\textbf{Substep 3.3: Choose sheet counts via Diophantine rounding.}
\editamir{Write $k:=2n-2p$.} For fixed $j$, the $\psi$--mass of a flat model translate in $Q$ is
\[
A_j(t)\ :=\ \Mass\bigl([\widetilde\Pi_j+t]\llcorner Q\bigr),\qquad t\in N_j^\perp.
\]
This is continuous in $t$ (it is $\mathcal H^{k}$ of a translated intersection in the flat chart), hence uniformly continuous on bounded sets.
Fix a small translation radius $\rho\ll h$ and choose the translations in Substep~3.4 so that
\[
A_j(t_{j,a})\ =\ A_j\ +\ O(\delta\,A_j)\qquad\text{for all }a,
\]
for some reference value $A_j>0$ (e.g.\ $A_j:=A_j(t_{j,1})$).  With $N_j$ sheets, the total mass in family $j$ is then
$N_jA_j+O(\delta\,N_jA_j)$.
\smallskip
Define
\[
\lambda_j:=\frac{\theta_j}{A_j},\qquad \Lambda:=\sum_i\lambda_i.
\]
For large integer $M$, set
\[
N_j(M):=\Bigl\lfloor M\frac{\lambda_j}{\Lambda}\Bigr\rfloor.
\]
Standard rounding estimates give
\[
\Bigl|N_j(M)-M\frac{\lambda_j}{\Lambda}\Bigr|\le 1,
\]
and hence
\[
\Bigl|\frac{N_j(M)A_j}{\sum_i N_i(M)A_i}-\theta_j\Bigr|=O\Bigl(\frac{1}{M}\Bigr).
\]
Choosing $M$ large and $\rho$ small (so the $O(\delta)$ mass-variation error is subordinate) yields the desired mass-fraction accuracy.
Fix such an $M$ and set $N_j:=N_j(M)$.

\medskip\noindent
\textbf{Substep 3.4: Build flat model sheets with disjoint translations.}
In $\Phi(Q)\subset\C^n$, for each $j$, let $N_j^\perp$ be the complex
$p$-dimensional normal space (the complex orthogonal complement of
$\widetilde\Pi_j$), so that $\C^n=\widetilde\Pi_j\oplus N_j^\perp$.
\begin{editamirblock}
Pick distinct translation vectors
$t_{j,1},\ldots,t_{j,N_j}\in N_j^\perp$ in a small ball $B(0,\rho)$
with $\rho\ll\mathrm{diam}(Q)$, such that for each fixed $j$ the affine spaces
$\widetilde\Pi_j+t_{j,a}$ are pairwise disjoint on $\Phi(Q)$ as $a$ varies.
This is possible since $N_j^\perp$ has real dimension $2p\ge 2$ and we choose only finitely many points.
\editamir{No disjointness is asserted between different directions $j\neq j'$.}
\end{editamirblock}

\begin{editamirblockNEW}
\noindent\textbf{Referee tightening (quantitative separation for persistence).}
To make the later holomorphic perturbation \emph{provably} preserve disjointness on $Q$, we choose the
translations with a margin tied to the eventual $C^1$--graph scale.
Fix $N\gg 1$ (to be chosen in Substep~3.5) and let $\eta_N\to 0$ be the graph parameter produced by
Proposition~\ref{prop:cell-scale-linear-model-graph} for $L^{\otimes N}$ on Bergman balls of radius $\asymp N^{-1/2}$.
Choose the translation vectors so that, for each fixed $j$,
\[
|t_{j,a}|\le c\,h
\qquad\text{and}\qquad
\|t_{j,a}-t_{j,a'}\|\ge 20\,\eta_N\,h\quad(a\neq a'),
\]
where $c>0$ is the constant in Proposition~\ref{prop:cell-scale-linear-model-graph}.
Then the resulting realized sheets $Y_j^a$ in Substep~3.5 satisfy
$\dist\bigl(Y_j^a\cap Q,\ \widetilde\Pi_j+t_{j,a}\bigr)\le 2\,\eta_N\,h$ (by Lemma~\ref{lem:global-graph-contraction}
as used in Proposition~\ref{prop:cell-scale-linear-model-graph}),
so the tubular neighborhoods around distinct planes are disjoint and hence the sheets remain disjoint on $Q$.
\end{editamirblockNEW}



Define
\[
\widetilde Y_j^a:=(\widetilde\Pi_j+t_{j,a})\cap\Phi(Q)\subset\C^n.
\]
These satisfy: (i) $\psi_0$-calibration (complex $(n-p)$-planes);
(ii) $\sup_{y\in Q}\angle(T_y\widetilde Y_j^a,\Pi_j)
=\angle(\widetilde\Pi_j,\Pi_j)<\varepsilon$;
(iii) mass fractions within $\delta$ of $\theta_j$ by construction;
\editamir{(iv) for each fixed $j$, the family $\{\widetilde Y_j^a\}_a$ is pairwise disjoint on $\Phi(Q)$;}
(v) boundary supported on $\partial\Phi(Q)$.

\medskip\noindent
\textbf{Substep 3.5: Upgrade to algebraic complete intersections.}
\begin{editamirblock}
This step uses the polarized/projective hypothesis: fix an ample line bundle $L$ with
$\omega=c_1(L)$ and work with large tensor powers $L^{\otimes N}$.
After shrinking $Q$ (or refining the cube partition), assume $Q$ is contained in a
Bergman ball of radius $O(N^{-1/2})$ for $L^{\otimes N}$.

Using holomorphic peak sections and Bergman kernel asymptotics (Tian--Catlin--Zelditch;
see \cite{Tian90,Catlin99,Zelditch98,MaMarinescu07}), one can construct global holomorphic
sections $s^{(1)}_{j,a},\ldots,s^{(p)}_{j,a}\in H^0(X,L^{\otimes N})$ whose restrictions to $Q$,
after trivializing $L^{\otimes N}$ on $Q$, are $C^2$--close to the affine linear functions cutting out
the model planes $\widetilde Y_j^a$.  For $N\gg 1$, the holomorphic implicit function theorem
then gives that
\[
Y_j^a:=\{s^{(1)}_{j,a}=0\}\cap\cdots\cap\{s^{(p)}_{j,a}=0\}
\]
is a smooth complex $(n-p)$--dimensional complete intersection and, on $Q$, is a single $C^1$ graph
over $\widetilde\Pi_j+t_{j,a}$.  Since the perturbation is $C^1$--small on $Q$, the calibration,
pairwise disjointness, and the mass fraction estimates from Substeps 3.1--3.4 persist.
(For quantitative transversality/persistence estimates in the large $N$ regime, compare
\cite{Donaldson01}.)
\end{editamirblock}

\begin{editamirblockNEW}
\noindent\textbf{Referee tightening (replace informal Bergman appeal by an internal lemma chain).}
The existence of algebraic complete intersections that are $C^1$--close to the flat model planes on $Q$
is provided quantitatively by Proposition~\ref{prop:cell-scale-linear-model-graph},
which itself is proved from Lemma~\ref{lem:bergman-affine-approx-hormander}
(via Lemma~\ref{lem:global-graph-contraction}).
Concretely, after a unitary linear change of coordinates sending $\widetilde\Pi_j$ to $\{w=0\}$,
for each translation $t_{j,a}$ with $|t_{j,a}|\le c\,h$ the proposition yields sections
$\sigma^{(1)}_{j,a},\dots,\sigma^{(p)}_{j,a}\in H^0(X,L^{\otimes N})$ such that
\[
Y_j^a:=\{\sigma^{(1)}_{j,a}=0\}\cap\cdots\cap\{\sigma^{(p)}_{j,a}=0\}
\]
is a smooth complex $(n-p)$--fold near $Q$ and $Y_j^a\cap Q$ is a single $C^1$ graph over
$\widetilde\Pi_j+t_{j,a}$ with slope $\le 2\eta_N$.
Mass persistence then follows from Lemma~\ref{lem:sliver-stability}\textnormal{(i)}:
\[
\Mass([Y_j^a]\llcorner Q)=\bigl(1+O(\eta_N^2)\bigr)\Mass([\widetilde\Pi_j+t_{j,a}]\llcorner Q),
\]
so by choosing $N$ large enough that $\eta_N^2\ll \delta$ and taking $M$ large enough in Substep~3.3,
property \textnormal{(ii)} holds as stated.
\end{editamirblockNEW}


\end{proof}

Fix a finite normal coordinate atlas by geodesic balls of radii $\ll 1$
and subordinate cubes $\{Q\}$ small enough so that the Carath\'eodory
data from Lemma~\ref{lem:caratheodory-general} are $\varepsilon$-stable
on each cube.  For each cube $Q$ and each index $j\in\{1,\ldots,N\}$,
let $\Pi_{Q,j}$ denote a constant complex $(n-p)$-plane approximating
$P_{x,j}$ on $Q$.  Apply Theorem~\ref{thm:local-sheets} to each cube
to obtain families $\{Y_{Q,j}^a\}$ of disjoint $\psi$-calibrated
complete intersections.

Define the local current
\[
S_Q := \sum_{j=1}^{N_{\mathrm{Car}}}\sum_{a=1}^{N_{Q,j}}[Y_{Q,j}^a]\llcorner Q.
\]
By construction, each $Y_{Q,j}^a$ is $\psi$-calibrated; hence $S_Q$ is a
positive $\psi$-calibrated integral current on $Q$.  Its tangent-plane
distribution on $Q$ is a convex combination of directions within
$\varepsilon$ of $\{\Pi_{Q,j}\}$ with weights proportional to the $\psi$--masses
in each family (equivalently proportional to $N_{Q,j}A_{Q,j}$, where $A_{Q,j}$ is
the $\psi$--mass of a single $(Q,j)$-sheet in $Q$).

\begin{lemma}[Local barycenter and mass matching]\label{lem:local-bary}
Fix a cube $Q$ and set
\[
M_Q \ :=\ m\int_Q \beta\wedge\psi.
\]
For any $\delta>0$ there exist integers $N_{Q,1},\ldots,N_{Q,N}$ such that the tangent-plane Young measure of $S_Q$ has barycenter within $\delta$
(in Hilbert--Schmidt norm) of the normalized field $\widehat\beta$ on $Q$, and
\[
\Bigl|\Mass(S_Q)-M_Q\Bigr|\ \le\ \delta\,M_Q.
\]
\end{lemma}

\begin{proof}
Let $k:=2n-2p$.
\editamir{Choose a corner-exit scale $s=s(Q,\delta)\ll \mathrm{side}(Q)$ and, for each direction label $(Q,j)$, use the corner-exit translation template mechanism
(Lemmas~\ref{lem:complex-corner-exit-template} and~\ref{lem:corner-exit-mass-scale}, and the finite-net packaging in Proposition~\ref{prop:corner-exit-template-net})
to arrange that all sheets in a fixed family $(Q,j)$ have identical corner-exit footprints inside $Q$.  In particular, their $\psi$--masses in $Q$ agree up to the common
small-slope distortion factor, so we may denote by $A_{Q,j}>0$ the common $\psi$--mass of a single $(Q,j)$-sheet in $Q$, with scaling $A_{Q,j}\asymp s^{k}$.}
Choose integers $N_{Q,j}$ so that the \emph{mass fractions}
\[
\frac{N_{Q,j}A_{Q,j}}{\sum_i N_{Q,i}A_{Q,i}}
\]
approximate $\theta_{x,j}$ (nearly constant on $Q$) to within $O(\delta)$.
Then the resulting mass-weighted barycenter
\[
\sum_j \frac{N_{Q,j}A_{Q,j}}{\sum_i N_{Q,i}A_{Q,i}}\;\xi_{\Pi_{Q,j}}
\]
is within $\delta$ of $\widehat\beta$ on $Q$.
Because the tangent angles are $<\varepsilon$ and $\varepsilon\ll\delta$, the
Hilbert--Schmidt distance of barycenters is $\le C(\varepsilon+\delta)$.

Finally, calibratedness gives
$\Mass([Y_{Q,j}^a]\llcorner Q)=\int_Q\psi\llcorner[Y_{Q,j}^a]$, hence
\[
\Mass(S_Q)=\sum_j N_{Q,j}A_{Q,j}.
\]
\editamir{By shrinking the corner-exit scale $s$ (hence shrinking all $A_{Q,j}\asymp s^k$ uniformly) one increases the available total sheet count
without changing the cube budget $M_Q=m\int_Q\beta\wedge\psi$.  This provides the discretization resolution needed to
arrange the simultaneous constraints on (i) mass fractions (barycenter) and (ii) total mass, yielding $|\sum_j N_{Q,j}A_{Q,j}-M_Q|\le \delta M_Q$.}
\end{proof}

% ------------------------------------------------------------
\subsection*{Step 4: Global cohomology quantization (Theorem C)}

This step forces the global integral current to represent exactly the
correct homology class $\mathrm{PD}(m[\gamma])$ by using lattice
discreteness.

\begin{theorem}[Global cohomology quantization]\label{thm:global-cohom}
Let $X$ be a smooth complex projective manifold of complex dimension $n$ with a fixed K\"ahler form $\omega=c_1(L)$ coming from an ample line bundle $L$.
Let $[\gamma]\in H^{2p}(X,\Q)$ be a rational Hodge class represented by a smooth closed $(p,p)$-form
$\beta$ with $\beta(x)\in K_p(x)$ pointwise.  Let $\{Q\}$ be a cube
partition of $X$.  Then there exists an integer $m\ge 1$ (clearing denominators of
$[\gamma]$) such that for every $\varepsilon>0$ there exist:
\begin{itemize}
\item A closed integral $(2n-2p)$-current $T_\varepsilon$ with
$[T_\varepsilon]=\mathrm{PD}(m[\gamma])$;
\item A correction current $R_\varepsilon$ with $\Mass(R_\varepsilon)<\varepsilon$;
\end{itemize}
such that the local tangent-plane mass proportions on each $Q$ match
those of $\beta$ up to error $o_{\varepsilon\to 0}(1)$.
\end{theorem}

\begin{proof}
The proof proceeds in three substeps.

\medskip\noindent
\textbf{Substep 4.1: Local quantization.}
Choose the partition $\{Q\}$ fine enough that on each $Q$, $\beta(x)$
is within $\delta$ (in operator norm) of $\beta(x_Q)$ for a base point
$x_Q\in Q$, and the K\"ahler metric is nearly constant (Jacobian and
volume distortion $\le 1+\delta$).

By Lemma~\ref{lem:caratheodory-general}, write
\[
\beta(x_Q)=t_Q\sum_{j=1}^{J(Q)}\theta_{Q,j}\,\xi_{Q,j},
\qquad
t_Q:=\langle \beta(x_Q),\psi_{x_Q}\rangle,
\]
where $\xi_{Q,j}\in K_p(x_Q)$ are normalized extremal generators (coming from
complex $(n-p)$-planes) satisfying $\langle \xi_{Q,j},\psi_{x_Q}\rangle=1$,
the weights satisfy $\theta_{Q,j}\ge 0$, $\sum_j\theta_{Q,j}=1$, and
$J(Q)\le \editamir{N_{\mathrm{Car}}=N_{\mathrm{Car}}(n,p)}$ uniformly bounded.

Since $[\gamma]$ is rational, all its periods lie in $(1/M)\Z$ for some
fixed $M$.  Choose $m\gg 1$ divisible by $M$.

Let $P_{Q,j}\subset T_{x_Q}X$ be the complex $(n-p)$-plane corresponding to $\xi_{Q,j}$.
\editamir{Fix also a corner-exit footprint scale $s=s(h,\delta)\ll h:=\mathrm{side}(Q)$ and write $k:=2n-2p=2(n-p)$.  Using the corner-exit translation templates
(Lemmas~\ref{lem:corner-exit-template-open} and~\ref{lem:corner-exit-mass-scale}, packaged uniformly over a finite direction net in Proposition~\ref{prop:corner-exit-template-net}),
we may choose, for each direction $(Q,j)$, a family of local $\psi$--calibrated sheet pieces in $Q$ whose corner-exit footprints are uniformly fat $k$--simplices of scale $s$ and are \emph{identical} across the family.  In particular, their $\psi$--masses in $Q$ are equal up to the common small-slope distortion factor.  Denote this common value by $A_{Q,j}>0$; by Lemma~\ref{lem:corner-exit-mass-scale} one has the mass scale $A_{Q,j}\asymp s^{k}$ (with constants depending only on $(n,p)$ and the net conditioning constants).}
The target $\psi$--mass in $Q$ is
\[
M_Q := m\int_Q \beta\wedge\psi \;\approx\; m\,t_Q\,\mathrm{Vol}(Q),
\]
up to $O(\delta)$ error from the $C^0$--variation of $\beta$ on $Q$ and the
metric distortion.

Choose integers $N_{Q,j}\ge 0$ so that simultaneously
\[
\Bigl|\frac{N_{Q,j}A_{Q,j}}{\sum_i N_{Q,i}A_{Q,i}}-\theta_{Q,j}\Bigr|\le \delta
\qquad\text{and}\qquad
\Bigl|\sum_j N_{Q,j}A_{Q,j}-M_Q\Bigr|\le \delta\,M_Q.
\]
\editamir{Such choices exist by rounding once the corner-exit scale $s$ is chosen so that $M_Q/A_{Q,j}\gg 1$ uniformly (equivalently, there are many equal-mass pieces available per cube).  Since $A_{Q,j}\asymp s^{k}$ can be made arbitrarily small by shrinking $s$ (with $m$ fixed), this discretization resolution can be achieved without taking $m\to\infty$.}

Apply Theorem~\ref{thm:local-sheets} to realize each direction $(Q,j)$ by a family
of $\psi$--calibrated sheets $Y_{Q,j}^a\subset Q$ ($a=1,\ldots,N_{Q,j}$) with
angle control, \editamir{pairwise disjointness on $Q$ for each fixed $j$ (as $a$ varies),} and boundary supported on $\partial Q$.

Define the raw local current
\[
S_Q:=\sum_{j=1}^{J(Q)}\sum_{a=1}^{N_{Q,j}}[Y_{Q,j}^a]\llcorner Q.
\]

\medskip\noindent
\textbf{Substep 4.2: Gluing across cubes.}
Consider the global raw current
\[
T^{\mathrm{raw}}:=\sum_Q S_Q.
\]
This is integral but not closed: $\partial T^{\mathrm{raw}}$ lives on
the union of cube faces.  View the cube adjacency as a finite graph:
vertices $=$ cubes $Q$, edges $=$ codimension-1 faces $F=Q\cap Q'$.
On each oriented face $F$, the restriction of $\partial S_Q$ induces
a $(2n-2p-1)$-current $B_{Q\to F}$ living on $F$.  Summed over all cubes:
\[
\partial T^{\mathrm{raw}}=\sum_F B_F,
\]
where $B_F$ is the mismatch between the two neighboring cubes.

\textbf{Key point (flat norm, not mass):} In general the individual face currents $B_F$
need not have small mass (cancellation-heavy boundaries can have large mass), so the robust
quantity to control is the \emph{flat norm} of the total mismatch $\partial T^{\mathrm{raw}}$.
Recall the flat norm on $(2n-2p-1)$-currents:
\[
\mathcal F(S):=\inf\{\Mass(R)+\Mass(Q):\ S=R+\partial Q\},
\]
where $R$ is an integral $(2n-2p-1)$-current and $Q$ is an integral $(2n-2p)$-current.
On a compact manifold one has the dual characterization (Federer--Fleming):
\[
\mathcal F(S)=\sup\{S(\eta):\ \eta\in C^\infty\Lambda^{2n-2p-1},\ \|\eta\|_{\mathrm{comass}}\le 1,\
\|d\eta\|_{\mathrm{comass}}\le 1\}.
\]
For $S=\partial T^{\mathrm{raw}}$ and such $\eta$, Stokes gives
$S(\eta)=\partial T^{\mathrm{raw}}(\eta)=T^{\mathrm{raw}}(d\eta)$.

\begin{proposition}[Transport control $\Rightarrow$ flat-norm gluing]\label{prop:transport-flat-glue}
Fix a cubulation of $X$ by coordinate cubes of side length $h=\mathrm{mesh}$, and write
$T^{\mathrm{raw}}=\sum_Q S_Q$ as above, where each $S_Q$ is a sum of calibrated sheets restricted to $Q$.
Assume the following \emph{geometric parameterization} holds on each interior face $F=Q\cap Q'$:
\begin{enumerate}
\item[\textnormal{(a)}] (\textbf{Small-angle graph model}) For each cube $Q$ and each sheet family $(Q,j)$, the sheets crossing $F$
are $C^1$-graphs over a fixed calibrated reference plane $\Pi_{Q,j}$ with
$\sup_{y\in Q}\angle(T_yY_{Q,j}^a,\Pi_{Q,j})\le \varepsilon$.
\item[\textnormal{(b)}] (\textbf{Transverse measures on faces}) After identifying a tubular neighborhood of $F$ with a product
$F\times B^{2p}(0,ch)$ in normal coordinates, the restriction of $\partial S_Q$ to $F$ can be written as a finite sum of translated
slice currents $\Sigma_y$ parameterized by a discrete transverse measure $\mu_{Q\to F}$ on $B^{2p}(0,ch)$ (integer weights), and similarly for $Q'$.
We assume (after the standard edge-trimming/localization away from the $(2n\!-\!2)$--skeleton of the mesh) that each such face slice is a cycle on the interior face,
i.e.\ $\partial\Sigma_y=0$ as a current on $F$ for all parameters $y$ that occur.
\editamir{(This is enforced in the vertex-template/corner-exit regime by the face-edit localization
Proposition~\ref{prop:vertex-template-face-edits}; see Lemma~\ref{lem:face-slice-cycle-mass}.)}
\item[\textnormal{(c)}] (\textbf{$W_1$ face matching}) The two induced transverse measures have the same total mass and satisfy
\[
W_1(\mu_{Q\to F},\mu_{Q'\to F})\ \le\ \tau_F,
\]
where $W_1$ is the $1$-Wasserstein distance on $B^{2p}(0,ch)$.
\end{enumerate}
Then there exists a constant $C=C(n,p,X)$ such that for every smooth $(2n-2p-1)$-form $\eta$ with
$\|\eta\|_{\mathrm{comass}}\le 1$ and $\|d\eta\|_{\mathrm{comass}}\le 1$ one has the face estimate
\[
|B_F(\eta)|\ \le\ C\,h^{2n-2p-1}\,\bigl(\tau_F + \varepsilon\,\Mass(\mu_{Q\to F})\,h\bigr),
\]
and hence
\[
\mathcal F(B_F)\ \le\ C\,h^{2n-2p-1}\,\bigl(\tau_F + \varepsilon\,\Mass(\mu_{Q\to F})\,h\bigr).
\]
Consequently,
\[
\mathcal F\!\left(\partial T^{\mathrm{raw}}\right)
\ \le\ \sum_{F}\mathcal F(B_F)
\ \le\ C\,h^{2n-2p-1}\sum_F \tau_F\ +\ C\,\varepsilon\,h^{2n-2p}\sum_F \Mass(\mu_{Q\to F}).
\]
\end{proposition}


\begin{proof}
Fix an interior face $F=Q\cap Q'$ and a test form $\eta$ with $\|\eta\|_{\mathrm{comass}}\le 1$ and $\|d\eta\|_{\mathrm{comass}}\le 1$.
Work in the tubular product chart from hypothesis \textnormal{(b)}, identifying a neighborhood of $F$ with $F\times B^{2p}(0,ch)$.

\smallskip\noindent
\textbf{Step 1 (a Lipschitz evaluation function).}
For a translated slice current $\Sigma_y$ in hypothesis \textnormal{(b)}, define the scalar function
\[
f_\eta(y)\ :=\ \Sigma_y(\eta).
\]
Let $y,y'\in B^{2p}(0,ch)$ and set $v:=y'-y$.
In the flat/parallel model (i.e.\ when $\Sigma_{y'}=(\tau_v)_\#\Sigma_y$ inside the product chart), consider the straight-line homotopy
$H:[0,1]\times F\to F\times B^{2p}(0,ch)$, $H(t,x)=(x,y+t v)$.
Let $Q_{y\to y'}:=H_\#([0,1]\times \Sigma_y)$.
\editamir{(Homotopy formula for currents under a Lipschitz homotopy; see \cite{Fed69,FF60}.)}
Since $\partial\Sigma_y=0$ on the interior face (hypothesis \textnormal{(b)}), the homotopy formula gives
\[
\Sigma_{y'}-\Sigma_y\ =\ \partial Q_{y\to y'},
\qquad
\Mass(Q_{y\to y'})\ \le\ \|v\|\,\Mass(\Sigma_y),
\]
By Stokes and the comass bound on $d\eta$,
\[
|f_\eta(y')-f_\eta(y)|
=|Q_{y\to y'}(d\eta)|
\le \|v\|\,\Mass(\Sigma_y).
\]
Under the small-angle graph hypothesis \textnormal{(a)} and bounded geometry of the chart, each slice has mass
$\Mass(\Sigma_y)\le C\,h^{2n-2p-1}$ with $C=C(n,p,X)$.
Hence
\[
\mathrm{Lip}(f_\eta)\ \le\ C\,h^{2n-2p-1}.
\]

\smallskip\noindent
\textbf{Step 2 (Kantorovich--Rubinstein).}
By hypothesis \textnormal{(b)}, the face restrictions can be written as
\(
(\partial S_Q)\llcorner F=\int \Sigma_y\,d\mu_{Q\to F}(y)
\)
and similarly for $Q'$, so
\[
B_F(\eta)
=\int f_\eta\,d\mu_{Q\to F}-\int f_\eta\,d\mu_{Q'\to F}.
\]
Since $\mu_{Q\to F}$ and $\mu_{Q'\to F}$ have the same total mass (hypothesis \textnormal{(c)}), adding a constant to $f_\eta$ does not change $B_F(\eta)$.
Therefore, by Kantorovich--Rubinstein duality for $W_1$, \editamir{(e.g.\ \cite{Villani03})}
\[
|B_F(\eta)|
\le \mathrm{Lip}(f_\eta)\,W_1(\mu_{Q\to F},\mu_{Q'\to F})
\le C\,h^{2n-2p-1}\,\tau_F.
\]

\smallskip\noindent
\textbf{Step 3 (small-angle model error).}
Hypothesis \textnormal{(a)} implies that each actual slice current appearing in \textnormal{(b)} is obtained from the
corresponding ``flat/parallel'' slice (the one used in Steps~1--2) by a $C^1$ graph perturbation over a cell of diameter $\asymp h$
with slope $O(\varepsilon)$.
In particular, after fixing the face chart, for each parameter $y$ there is a Lipschitz map $\Psi_y$ defined on a neighborhood of
the flat slice such that
\[
\Sigma^{\mathrm{act}}_y\ =\ (\Psi_y)_\#\Sigma^{\mathrm{flat}}_y,
\qquad
\sup_{x\in\spt \Sigma^{\mathrm{flat}}_y}\|\Psi_y(x)-x\|\ \le\ C\,\varepsilon\,h,
\qquad
\Lip(\Psi_y)\ \le\ 1+C\,\varepsilon,
\]
where $C$ depends only on the fixed product chart constants.
Applying Lemma~\ref{lem:flat-C0-deform} with $\phi_0=\Id$, $\phi_1=\Psi_y$ and $\delta\asymp\varepsilon h$ yields
\[
\mathcal F\!\bigl(\Sigma^{\mathrm{act}}_y-\Sigma^{\mathrm{flat}}_y\bigr)
\ \le\ C\,\varepsilon\,h\Bigl(\Mass(\Sigma^{\mathrm{flat}}_y)+\Mass(\partial\Sigma^{\mathrm{flat}}_y)\Bigr).
\]
Summing this estimate over the (integer-weighted) family of slices meeting $F$ gives an additional contribution bounded by
\[
C\,\varepsilon\,h\sum_{\text{slices on }F}\Bigl(\Mass(\Sigma^{\mathrm{flat}}_y)+\Mass(\partial\Sigma^{\mathrm{flat}}_y)\Bigr)
\ \le\ C\,\varepsilon\,h^{2n-2p}\,\Mass(\mu_{Q\to F}),
\]
where the last inequality uses that each flat slice has $(2n-2p-1)$--mass $\asymp h^{2n-2p-1}$ in the fixed chart.
Combining with Step~2 yields the stated face estimate
\(
|B_F(\eta)|\le C h^{2n-2p-1}(\tau_F+\varepsilon\,\Mass(\mu_{Q\to F})\,h).
\)

\smallskip\noindent
\textbf{Step 4 (flat norm and summation).}
Taking the supremum over $\eta$ in the dual characterization of $\mathcal F$ gives
\(
\mathcal F(B_F)\le C h^{2n-2p-1}(\tau_F+\varepsilon\,\Mass(\mu_{Q\to F})\,h).
\)
Finally, $\partial T^{\mathrm{raw}}=\sum_F B_F$ as currents, so the triangle inequality for $\mathcal F$ implies
\(
\mathcal F(\partial T^{\mathrm{raw}})\le \sum_F \mathcal F(B_F),
\)
which yields the global bound claimed.
\end{proof}


\begin{remark}[Why hypotheses (a)--(b) hold for the local sheet model]\label{rem:transport-hypotheses}
In the flat model of Substep~3.4, each sheet in family $(Q,j)$ is literally an affine calibrated plane
$(\widetilde\Pi_{Q,j}+t_{j,a})\cap Q$, with translation parameter $t_{j,a}\in N_{Q,j}^\perp\cong\R^{2p}$.
For a fixed face $F\subset\partial Q$, the boundary slice current
\[
\Sigma_{F,j}(t):=\partial\big([\widetilde\Pi_{Q,j}+t]\llcorner Q\big)\llcorner F
\]
depends only on $t$ through its component normal to the $(2n-2p-1)$-plane $\widetilde\Pi_{Q,j}\cap TF$.
Thus, in the flat model, $\partial S_Q\llcorner F$ can be written as a finite sum
$\sum_a \Sigma_{F,j}(t_{j,a})$, i.e.\ it is parameterized by the discrete transverse measure
$\mu_{Q\to F}:=\sum_a \delta_{t_{j,a}}$ (with integer weights).

After upgrading to algebraic complete intersections in Substep~3.5, the sheets remain $C^1$-graphs over the flat model on $Q$
(for $k$ large), so the same parameterization persists in a tubular neighborhood of $F$ up to an $O(\varepsilon)$ error
controlled by the graph distortion.  This justifies the use of transverse measures on faces and the small-angle graph model
in Proposition~\ref{prop:transport-flat-glue}.

What is \emph{not} automatic is hypothesis (c): arranging $W_1$ matching across faces simultaneously for all cubes, subject to
the constraint that each sheet’s translation parameter determines its intersection with \emph{all} faces of $Q$ at once.
\smallskip
Equivalently, for a fixed cube $Q$ and family $(Q,j)$, the face measures $\mu_{Q\to F}$ for different faces $F\subset\partial Q$
are not independent choices: they arise as pushforwards of the \emph{same} discrete translation multiset $\{t_{j,a}\}$ under
the corresponding face-slice maps.  Thus the remaining task is a \emph{simultaneous} matching problem.

\begin{editamirblockNEW}
This simultaneous matching is supplied later by the global-coherence/vertex-template program:
Proposition~\ref{prop:global-coherence-all-labels} produces globally consistent integer data whose induced face measures
$\mu_{Q\to F}$ satisfy the hypotheses required in the transport--gluing step (cf.\ Proposition~\ref{prop:transport-flat-glue-weighted}),
thereby verifying hypothesis~\textnormal{(c)} in Theorem~\ref{thm:sliver-mass-matching-on-template} without any extra assumption.
\end{editamirblockNEW}
\end{remark}

\begin{lemma}[Automatic $W_1$-matching from smooth dependence of face maps]\label{lem:w1-auto}
\editamir{Let $\mu$ be a finite Borel measure on $\R^{2p}$ supported in a ball of radius $O(\varrho h)$ and with total mass $\mu(\R^{2p})=N$.}
Let $\Phi,\Phi':\R^{2p}\to\R^{2p}$ be linear maps with $\|\Phi-\Phi'\|_{\mathrm{op}}\le C\,h$.
Then
\[
\editamir{W_1(\Phi_\#\mu,\Phi'_\#\mu)\ \le\ C\,h\int_{\R^{2p}}\|y\|\,d\mu(y)\ \le\ C'\,\varrho\,h^2\,N.}
\]
\end{lemma}


\begin{proof}
Define a coupling $\pi$ of $\Phi_\#\mu$ and $\Phi'_\#\mu$ by pushing $\mu$ forward under the map
$y\mapsto (\Phi y,\Phi' y)$.
Then $\pi$ has first marginal $\Phi_\#\mu$ and second marginal $\Phi'_\#\mu$, and therefore
\[
W_1(\Phi_\#\mu,\Phi'_\#\mu)
\le
\int_{\R^{2p}\times\R^{2p}} \|u-u'\|\,d\pi(u,u')
\;=\;
\int_{\R^{2p}} \|\Phi y-\Phi' y\|\,d\mu(y).
\]
Estimating $\|\Phi y-\Phi' y\|\le \|\Phi-\Phi'\|_{\mathrm{op}}\|y\|$ gives
\[
W_1(\Phi_\#\mu,\Phi'_\#\mu)
\le \|\Phi-\Phi'\|_{\mathrm{op}}\int_{\R^{2p}}\|y\|\,d\mu(y).
\]
\editamir{If $\operatorname{supp}\mu\subset B(0,C_0 \varrho h)$, then $\int\|y\|\,d\mu\le C_0 \varrho h\,\mu(\R^{2p})=C_0 \varrho h\,N$.}
Absorbing constants yields the stated bound.
\end{proof}


\begin{editblock}
\begin{lemma}[Pointwise displacement bound under nearby face maps]\label{lem:face-displacement}
Let $y_1,\dots,y_N\in\R^{2p}$ satisfy $\|y_a\|\le \editamir{C_0\,\varrho\,h}$ and let $\Phi,\Phi':\R^{2p}\to\R^{2p}$ be linear maps with
$\|\Phi-\Phi'\|_{\mathrm{op}}\le C_1\,h$.
Define two multisets $u_a:=\Phi y_a$ and $u'_a:=\Phi' y_a$.
Then the index-wise matching satisfies
\[
\|u_a-u'_a\|\ \le\ \editamir{C_0C_1\,\varrho\,h^2}\qquad\text{for all }a.
\]
In particular, when adjacent cells use the \emph{same} translation template $\{y_a\}$ and their face parameterizations differ by $O(h)$ in operator norm,
the hypothesis of Corollary~\ref{cor:global-flat-weighted} holds with $\editamir{\Delta_F=O(\varrho\,h^2)}$.
\end{lemma}

\begin{proof}
\[
\|u_a-u'_a\|
=\|(\Phi-\Phi')y_a\|
\le \|\Phi-\Phi'\|_{\mathrm{op}}\|y_a\|
\le (C_1h)(C_0\varrho h)=C_0C_1\varrho h^2.
\qedhere
\]
\end{proof}
\end{editblock}

\begin{lemma}[Template stability under small multiset edits]\label{lem:w1-template-edit}
\editamir{Let $\Omega\subset\R^{2p}$ be a bounded domain of diameter $\mathrm{diam}(\Omega)\le C\,\varrho h$.}
Let $\mu=\sum_{a=1}^{N}\delta_{y_a}$ and $\mu'=\sum_{b=1}^{N}\delta_{y'_b}$ be two integer-weighted discrete measures on $\Omega$
with the \emph{same total mass} $N$.
Assume there is a matching of atoms such that $\|y_a-y'_a\|\le \Delta$ for all $a$ (after relabeling).
Then
\[
W_1(\mu,\mu')\ \le\ \Delta\,N.
\]
More generally, if $\mu'$ is obtained from $\mu$ by deleting $r$ atoms and inserting $r$ atoms (so total mass stays $N$), then
\[
\editamir{W_1(\mu,\mu')\ \le\ r\cdot \mathrm{diam}(\Omega)\ \le\ C\,r\,\varrho\,h.}
\]
\end{lemma}

\begin{proof}
For the first claim, couple $\mu$ and $\mu'$ by pairing each $y_a$ to $y'_a$; the transport cost is $\sum_a\|y_a-y'_a\|\le \Delta N$.
For the second claim, transport each deleted atom to an inserted atom at cost at most $\mathrm{diam}(\Omega)$ and keep the unchanged atoms fixed.
\end{proof}

\begin{remark}[How Lemma~\ref{lem:w1-auto} reduces the remaining matching task]\label{rem:w1-auto}
If, for each cube $Q$ and sheet family $(Q,j)$, we choose the translation multiset $\{t_{j,a}\}$ by a \emph{fixed} template in
$N_{Q,j}^\perp$ (e.g.\ a scaled lattice/low-discrepancy set of diameter $O(h)$), then across a shared face $F=Q\cap Q'$ the two
induced transverse measures are related by applying two nearby face-slice maps (coming from nearby plane directions and nearby normal-coordinate identifications).
Since $\beta$ is smooth, these maps differ by $O(h)$ in operator norm, so Lemma~\ref{lem:w1-auto} yields
\[
\editamir{W_1(\mu_{Q\to F},\mu_{Q'\to F})\ \lesssim\ \varrho\,h^2\,N_F,}
\]
where $N_F$ is the number of sheets contributing to that face.
Inserting this into Proposition~\ref{prop:transport-flat-glue} yields a global bound of the form
\[
\editamir{\mathcal F(\partial T^{\mathrm{raw}})\ \lesssim\ m\,\varrho\,h \;+\; O(\varepsilon\,m),}
\]
\begin{editamirblockNEW}
so choosing a refinement schedule $h=h_j\downarrow 0$ and $\varepsilon=\varepsilon_j\downarrow 0$ forces the right-hand side to $0$ for fixed $m$, hence the gluing correction $U_{h_j}$ becomes negligible in the mass equality.
\end{editamirblockNEW}
The remaining task is then to implement this “fixed template” choice while still meeting the cohomological constraints (Substep 4.3).
\smallskip
\editblue{In the \emph{sliver} regime, the count $N_F$ is not controlled by total mass; see Remark~\ref{rem:sliver-vs-template} and
Corollary~\ref{cor:global-flat-weighted} for the weighted replacement.}
\end{remark}

\begin{editblock}
\begin{remark}[Sliver regime: what changes in the global counting estimate]\label{rem:sliver-vs-template}
The global $m\,h$ bound in Remark~\ref{rem:w1-auto} uses an implicit \emph{counting step}: it treats the total face mismatch as scaling like
``(per-sheet mismatch) $\times$ (number of sheet pieces meeting faces)''.  In the constant-mass-per-sheet model this count is controlled by total mass,
because each sheet piece carries $\psi$--mass $\asymp h^{2(n-p)}$ in a cube.

\smallskip\noindent
In the \emph{sliver} regime (Remark~\ref{rem:sliver}), one deliberately allows many pieces of very small mass per cube.
Then the raw counts $N_F$ (or the total number of sheet pieces meeting faces) can be arbitrarily large at fixed total mass, so the crude reduction
to $\Mass(T^{\mathrm{raw}})$ is no longer available.
To make the sliver escape compatible with flat-norm gluing, we therefore use a \emph{weighted} replacement that tracks the actual size of each face slice,
for example a bound in terms of the boundary-size functional
\[
\sum_{F}\sum_{a\in\mathcal S(F)} \Mass\!\big(\partial([Y^a]\llcorner Q)\llcorner F\big),
\]
or an equivalent transverse-parameter integral.  Concretely, Proposition~\ref{prop:transport-flat-glue-weighted} bounds each face flat mismatch by
displacement $\times$ (slice boundary mass), and Lemma~\ref{lem:uniformly-convex-slice-boundary} converts slice boundary mass into a power of the
interior piece mass on smooth curvature-pinched cells.  This is packaged globally as Corollary~\ref{cor:global-flat-weighted}.
\end{remark}
\end{editblock}

\begin{editblock}
\begin{proposition}[Weighted transport $\Rightarrow$ flat-norm face control (sliver-compatible)]\label{prop:transport-flat-glue-weighted}
Work in the tubular/flat model on an interior face $F=Q\cap Q'$.
Assume each sheet piece meeting $F$ contributes an integral slice current $\Sigma(u)$ on $F$ depending on a transverse parameter
$u\in\Omega_F\subset\R^{2p}$, and that $\Sigma(u)$ is obtained from $\Sigma(0)$ by translation in the face chart.
Let the two adjacent cubes induce two multisets of parameters $\{u_a\}_{a=1}^N$ and $\{u'_a\}_{a=1}^N$ (same cardinality), hence two face currents
\[
S_{Q\to F}:=\sum_{a=1}^N \Sigma(u_a),\qquad
S_{Q'\to F}:=\sum_{a=1}^N \Sigma(u'_a),
\qquad
B_F:=S_{Q\to F}-S_{Q'\to F}.
\]
Then
\[
\mathcal F(B_F)\ \le\ \inf_{\sigma\in S_N}\ \sum_{a=1}^N \|u_a-u'_{\sigma(a)}\|\Bigl(\Mass(\Sigma(u_a))+\Mass(\partial\Sigma(u_a))\Bigr).
\]
In particular, if $\Mass(\Sigma(u_a))+\Mass(\partial\Sigma(u_a))\le b_F$ for all $a$ and if
\[
\tau_F:=\inf_{\sigma\in S_N}\ \sum_{a=1}^N \|u_a-u'_{\sigma(a)}\|
\]
(the equal-weight matching cost, i.e.\ $W_1$ of the counting measures), then
\[
\mathcal F(B_F)\ \le\ b_F\,\tau_F.
\]
\end{proposition}


\begin{proof}
Fix a permutation $\sigma\in S_N$.
For each index $a$, apply Lemma~\ref{lem:flat-translate} in the face chart to the translated pair
$\Sigma(u_a)$ and $\Sigma(u'_{\sigma(a)})$.
This yields integral currents $R_a$ and $Q_a$ such that
\[
\Sigma(u_a)-\Sigma(u'_{\sigma(a)})\ =\ R_a+\partial Q_a
\qquad\text{and}\qquad
\Mass(R_a)+\Mass(Q_a)\ \le\ \|u_a-u'_{\sigma(a)}\|\Bigl(\Mass(\Sigma(u_a))+\Mass(\partial\Sigma(u_a))\Bigr).
\]
Summing $R:=\sum_{a=1}^N R_a$ and $Q:=\sum_{a=1}^N Q_a$ gives $B_F=R+\partial Q$ and
\[
\Mass(R)+\Mass(Q)\ \le\ \sum_{a=1}^N \|u_a-u'_{\sigma(a)}\|\Bigl(\Mass(\Sigma(u_a))+\Mass(\partial\Sigma(u_a))\Bigr).
\]
Taking the infimum over $\sigma$ in the definition of $\mathcal F$ proves the claim.
\end{proof}

\end{editblock}

\begin{editjonblock}
\begin{editamirblockNEW}
\begin{proposition}[Integer transverse matching from the master prefix template (constructed here)]
\label{prop:integer-transport}
Let $F=Q\cap Q'$ be an interior $(2n-1)$-face of the cubulation at mesh $h$.
Fix a transverse grid scale $\delta_{\perp}\in(0,h)$ and choose an integer $N_*\ge 0$ together with an ordered list of grid atoms
$\mathbf y=(y_a)_{a=1}^{N_*}$ with $y_a\in \editamir{B_{C_0\varrho h}(0)}\cap \delta_{\perp}\mathbb Z^{2p}$.
For $1\le N\le N_*$ write $\nu^{(N)}:=\sum_{a=1}^N\delta_{y_a}$ (cf.\ Proposition~\ref{prop:prefix-template-coherence}).

Let $\Phi_{Q,F},\Phi_{Q',F}:\editamir{B_{C_0\varrho h}(0)}\to \Omega_F$ be the face maps from
Lemma~\ref{lem:template-displacement-edits} (in particular
$\|\Phi_{Q,F}-\Phi_{Q',F}\|_{\mathrm{op}}\le C_\Phi h$ and
$\|\Phi_{Q,F}\|_{\mathrm{op}}+\|\Phi_{Q',F}\|_{\mathrm{op}}\le C_{\Phi,0}$).

Define the (balanced) transverse measures on $F$ by choosing an integer $N_F$ with $0\le N_F\le N_*$
(the common prefix length after the face-balancing/prefix-edit step of
Proposition~\ref{prop:global-coherence-all-labels}) and setting
\[
\mu_{Q\to F}\ :=\ (\Phi_{Q,F})_\#\nu^{(N_F)},
\qquad
\mu_{Q'\to F}\ :=\ (\Phi_{Q',F})_\#\nu^{(N_F)}.
\]
Then $\mu_{Q\to F}$ and $\mu_{Q'\to F}$ are integer-weighted, supported on the
$\delta_{\perp}$-grid images in $\Omega_F$, and satisfy
\[
\int_{\Omega_F}\mu_{Q\to F}\ =\ \int_{\Omega_F}\mu_{Q'\to F}\ =\ N_F.
\]
Moreover their $W_1$-distance is controlled by the template displacement:
\[
W_1(\mu_{Q\to F},\mu_{Q'\to F})
\ \le\ \editamir{C_\Phi\,C_0\,\varrho\,h^2}\,N_F.
\]
In particular, hypothesis \textnormal{(c)} in Proposition~\ref{prop:transport-flat-glue-weighted}
holds with
\[
\tau_F\ :=\ \editamir{C_\Phi\,C_0\,\varrho\,h^2}\,N_F.
\]
\editamir{Moreover, the same identity pairing yields the \emph{uniform pointwise} displacement bound
$\|\Phi_{Q,F}(y_a)-\Phi_{Q',F}(y_a)\|\le C_\Phi C_0\varrho h^2$ for all $a\le N_F$, so Corollary~\ref{cor:global-flat-weighted} may be applied with $\Delta_F:=C_\Phi C_0\varrho h^2$.}
\end{proposition}

\begin{proof}
Let $\pi$ be the coupling obtained by matching the same template atom on both sides:
\[
\pi\ :=\ \sum_{a=1}^{N_F}\delta_{(\Phi_{Q,F}(y_a),\,\Phi_{Q',F}(y_a))}.
\]
Then $\pi$ has marginals $\mu_{Q\to F}$ and $\mu_{Q'\to F}$ by definition, hence
\[
W_1(\mu_{Q\to F},\mu_{Q'\to F})
\ \le\ \int_{\Omega_F\times\Omega_F}|u-v|\,d\pi(u,v)
\ =\ \sum_{a=1}^{N_F}\big|\Phi_{Q,F}(y_a)-\Phi_{Q',F}(y_a)\big|.
\]
Using Lemma~\ref{lem:template-displacement-edits} and $\editamir{|y_a|\le C_0\varrho h}$ we get
\[
\big|\Phi_{Q,F}(y_a)-\Phi_{Q',F}(y_a)\big|
\ \le\ \|\Phi_{Q,F}-\Phi_{Q',F}\|_{\mathrm{op}}\,|y_a|
\ \le\ \editamir{C_\Phi h\cdot C_0\varrho h\ =\ C_\Phi C_0 \varrho h^2}.
\]
Summing over $a=1,\dots,N_F$ yields the claimed bound.
\end{proof}
\end{editamirblockNEW}
\end{editjonblock}

\begin{editblock}
\begin{remark}[Exact geometric inequality needed for slivers]
Proposition~\ref{prop:transport-flat-glue-weighted} shows that, in the sliver regime, the face mismatch is controlled by a \emph{weighted} matching cost:
displacement $\times$ (slice boundary mass), rather than displacement $\times$ (number of sheets).
Thus the missing geometric input is precisely an estimate of the form
\[
\Mass(\Sigma(u))\ \lesssim\ \Mass([Y]\llcorner Q)^{\frac{k-1}{k}}
\qquad (k:=2n-2p),
\]
uniformly for the relevant family of slices in the chosen cell geometry (balls / rounded cubes).  In the ball model this holds with an explicit sharp constant;
for general smooth uniformly convex cells it is the content of the “boundary shrinkage for plane slices’’ estimate.
\end{remark}
\end{editblock}

\begin{editblock}
\begin{lemma}[Boundary shrinkage for plane slices in smooth uniformly convex cells]\label{lem:uniformly-convex-slice-boundary}
Let $Q\subset\R^d$ be a bounded $C^2$ \emph{uniformly convex} domain of diameter $\asymp h$.
Assume the principal curvatures of $\partial Q$ satisfy
\[
\frac{c}{h}\ \le\ \kappa_i\ \le\ \frac{C}{h}
\qquad\text{everywhere on }\partial Q,
\]
for fixed constants $0<c\le C$.
Fix $1\le k<d$ and a $k$-plane $P$.
For each translate $P+t$ with nonempty intersection, set
\[
v(t):=\mathcal H^{k}\bigl((P+t)\cap Q\bigr),
\qquad
a(t):=\mathcal H^{k-1}\bigl((P+t)\cap \partial Q\bigr).
\]
Then there exists $C_*=C_*(d,k,c,C)$ such that
\[
a(t)\ \le\ C_*\,\bigl(v(t)\bigr)^{\frac{k-1}{k}}
\qquad\text{for all such }t.
\]
\end{lemma}

\begin{proof}
The estimate is scale-invariant, so rescale so that $h\asymp 1$.
Write $K_t:=(P+t)\cap Q\subset P+t\cong\R^k$, so $v(t)=\mathcal H^k(K_t)$ and $a(t)=\mathcal H^{k-1}(\partial K_t)$.

If $v(t)\ge v_0>0$, then $K_t$ is a convex body contained in a fixed $k$--ball of radius $O(1)$, hence $a(t)\le A_0(d,k)$, and the desired bound follows
after increasing $C_*$.
\editamir{(For example, if $K_t\subset B_R\subset \R^k$ with $R=O(1)$, then by convexity $K_t+\rho B_1\subset B_{R+\rho}$ for all $\rho>0$.
Differentiating the corresponding volume bound at $\rho=0$ (Steiner formula) gives a uniform surface-area bound $a(t)=\mathcal H^{k-1}(\partial K_t)\le C(k)\,R^{k-1}$.)}

Assume $v(t)\le v_0$ with $v_0$ small.  The curvature pinching implies an interior/exterior rolling-ball condition with radii
$r_{\mathrm{in}},r_{\mathrm{out}}\asymp 1$ (depending only on $c,C$) at every boundary point of $Q$.
Let $\pi:\R^d\to P^\perp$ be orthogonal projection and set $D:=\pi(Q)\subset P^\perp$.
\editamir{Choose a nearest boundary point $t_0\in\partial D$ to $t$ and set $s:=\|t_0-t\|$ and $u:=(t_0-t)/\|t_0-t\|\in P^\perp$, so $t=t_0-s u$.
By convexity of $D$, the vector $u$ is an outward normal to a supporting hyperplane of $D$ at $t_0$.}
\editamir{Let $x_0\in\partial Q$ be the unique supporting point of $Q$ in direction $u$ (uniqueness by uniform convexity).  Since $u\perp P$,
the support function of $D=\pi(Q)$ in direction $u$ agrees with that of $Q$, hence $\pi(x_0)=t_0$.}

Intersect the tangent balls at $x_0$ with the affine plane $P+t$.  Since $u\perp P$, these intersections are $k$-balls of radii
$\rho_{\mathrm{in}}(s)=\sqrt{2r_{\mathrm{in}}s-s^2}$ and $\rho_{\mathrm{out}}(s)=\sqrt{2r_{\mathrm{out}}s-s^2}$, hence
\[
\omega_k\,\rho_{\mathrm{in}}(s)^k\ \le\ v(t)\ \le\ \omega_k\,\rho_{\mathrm{out}}(s)^k,
\qquad
a(t)\ \le\ \omega_{k-1}\,\rho_{\mathrm{out}}(s)^{k-1}.
\]
For $s$ small one has $\rho_{\mathrm{in}}(s)\gtrsim \sqrt{s}$ and $\rho_{\mathrm{out}}(s)\lesssim \sqrt{s}$, so $v(t)\gtrsim s^{k/2}$ and
$a(t)\lesssim s^{(k-1)/2}$, hence $s\lesssim v(t)^{2/k}$ and $a(t)\lesssim v(t)^{(k-1)/k}$.
\end{proof}

\begin{remark}[References for the geometric inputs]
The implication “principal curvatures pinched at scale $h$ $\Rightarrow$ interior/exterior tangent balls of radius $\asymp h$’’ is the classical
\emph{rolling ball} principle in convex geometry (often attributed to Blaschke).
The supporting-hyperplane/unique-support-point facts used above are standard consequences of strict convexity and $C^2$ regularity of $\partial Q$
(see any standard text on convex bodies, e.g.\ Schneider’s \emph{Convex Bodies: The Brunn--Minkowski Theory}).
\end{remark}
\end{editblock}

\begin{editblock}
\begin{lemma}[Flat-norm stability under translation]\label{lem:flat-translate}
Let $S$ be an integral $\ell$--current in $\R^d$ with finite mass and finite boundary mass.
For any translation vector $v\in\R^d$, write $\tau_v(x):=x+v$ and $(\tau_v)_\#S$ for the pushforward.
Then
\[
\mathcal F\!\bigl((\tau_v)_\#S-S\bigr)\ \le\ \|v\|\Bigl(\Mass(S)+\Mass(\partial S)\Bigr).
\]
In particular, if $S$ is a cycle ($\partial S=0$) this reduces to
$\mathcal F((\tau_v)_\#S-S)\le \|v\|\,\Mass(S)$.
\end{lemma}


\begin{proof}
Let $H:[0,1]\times\R^d\to\R^d$ be the straight-line homotopy $H(t,x)=x+t v$.
Consider the product current $[0,1]\times S$ in $[0,1]\times\R^d$ and set
\(
Q:=H_\#([0,1]\times S).
\)
Set also
\(
R:=H_\#([0,1]\times \partial S).
\)
Since $\partial([0,1]\times S)=\{1\}\times S-\{0\}\times S-[0,1]\times \partial S$, we have
\[
\partial Q
=H_\#(\{1\}\times S)-H_\#(\{0\}\times S)-H_\#([0,1]\times \partial S)
=(\tau_v)_\#S-S-R.
\]
Thus $(\tau_v)_\#S-S=R+\partial Q$.
Moreover, $H$ has Jacobian bounded by $\|v\|$ in the $t$-direction, so the mass estimate for pushforwards gives
\(
\Mass(Q)\le \|v\|\,\Mass(S).
\)
Likewise $\Mass(R)\le \|v\|\,\Mass(\partial S)$.
Taking these $R,Q$ in the definition of $\mathcal F$ yields
\[
\mathcal F((\tau_v)_\#S-S)\le \Mass(R)+\Mass(Q)\le \|v\|\Bigl(\Mass(S)+\Mass(\partial S)\Bigr),
\]
as claimed.
\end{proof}

\begin{lemma}[Flat-norm stability under small $C^0$ deformations]\label{lem:flat-C0-deform}
Let $S$ be an integral $\ell$--current in $\R^d$ with finite mass and finite boundary mass.
Let $\phi_0,\phi_1:\R^d\to\R^d$ be Lipschitz maps with
\[
\sup_{x\in\spt S}\|\phi_1(x)-\phi_0(x)\|\ \le\ \delta,
\qquad
\Lip(\phi_0)+\Lip(\phi_1)\ \le\ L.
\]
Then there exists a constant $C_\ell$ depending only on $\ell$ such that
\[
\mathcal F\!\bigl(\phi_{1\#}S-\phi_{0\#}S\bigr)\ \le\ C_\ell\,\delta\,L^{\ell}\Bigl(\Mass(S)+\Mass(\partial S)\Bigr).
\]
\end{lemma}

\begin{proof}
Consider the straight-line homotopy $H:[0,1]\times\R^d\to\R^d$ given by
$H(t,x):=(1-t)\phi_0(x)+t\,\phi_1(x)$.
Set $Q:=H_\#([0,1]\times S)$ and $R:=H_\#([0,1]\times \partial S)$.
Since $\partial([0,1]\times S)=\{1\}\times S-\{0\}\times S-[0,1]\times\partial S$, the homotopy formula gives
\[
\phi_{1\#}S-\phi_{0\#}S\ =\ R+\partial Q.
\]
On $\spt S$, the differential of $H$ has one ``$t$--direction'' column $\,\partial_t H=\phi_1-\phi_0\,$ whose norm is $\le\delta$,
and $\ell$ ``spatial'' columns bounded by $L$.
Therefore the $(\ell+1)$--Jacobian of $H$ is bounded by $C_\ell\,\delta\,L^\ell$ on $\spt([0,1]\times S)$, and the $\ell$--Jacobian
of $H$ restricted to $\spt([0,1]\times\partial S)$ is bounded by $C_\ell\,\delta\,L^{\ell-1}$.
The standard mass estimate for pushforwards yields
\[
\Mass(Q)\ \le\ C_\ell\,\delta\,L^\ell\,\Mass(S),
\qquad
\Mass(R)\ \le\ C_\ell\,\delta\,L^\ell\,\Mass(\partial S),
\]
(after enlarging $C_\ell$ to absorb the $L^{\ell-1}$ factor).
Taking these $R,Q$ in the definition of $\mathcal F$ gives the claim.
\end{proof}


\end{editblock}

\begin{editblock}

\begin{lemma}[{\color{blue}Interface face-slices are cycles with controlled mass}]\label{lem:face-slice-cycle-mass}
\begin{editamirblockNEW}
Work on an interior interface face $F=Q\cap Q'$ in the flat/tubular chart, and assume the holomorphic sliver pieces
$Y^{Q,a}\cap Q$ satisfy the single-sheet small-slope graph control of Proposition~\ref{prop:cell-scale-linear-model-graph}
(and hence Lemma~\ref{lem:sliver-stability}) on a neighborhood of $Q$.
Let $\Sigma_F(u_a)$ denote the (integral) $(k-1)$--current on $F$ contributed by the boundary trace of the $a$--th sheet on $F$
(as in Proposition~\ref{prop:transport-flat-glue-weighted}), where $k:=2n-2p$.

\smallskip\noindent
Then, after the standard edge-trimming/prefix-edit localization of Proposition~\ref{prop:vertex-template-face-edits}
(which ensures the face trace is supported away from the $(2n-2)$--skeleton of the mesh):
\begin{enumerate}
\item[\textnormal{(i)}] $\partial\Sigma_F(u_a)=0$ as a current on $F$ (i.e.\ the face slice is a cycle on the interior face).
\item[\textnormal{(ii)}] There exists a constant $C=C(X,n,p)>0$ such that
\[
\Mass(\Sigma_F(u_a))\ \le\ \editamir{C\,m_{Q,a}^{\frac{k-1}{k}}},
\qquad
m_{Q,a}:=\Mass([Y^{Q,a}]\llcorner Q).
\]
\end{enumerate}

\smallskip\noindent
\begin{proof}
For \textnormal{(i)}, Proposition~\ref{prop:vertex-template-face-edits} performs the edge-trimming/prefix edits so that
the face trace associated to the $a$--th sheet is supported in the \emph{open} face
\[
F^\circ:=F\setminus \mathcal N_{\rho h}(\mathrm{skel}),
\]
where $\mathrm{skel}$ is the $(2n-2)$--skeleton of the mesh and $\rho>0$ is fixed.
On $F^\circ$ the trace is an integral $(k-1)$--current with no contribution from the boundary of $F^\circ$,
hence its boundary vanishes as a current on $F^\circ$. Since $\Sigma_F(u_a)$ is supported in $F^\circ$,
this is equivalent to $\partial\Sigma_F(u_a)=0$ as a current on the interior interface face.

For \textnormal{(ii)}, by Proposition~\ref{prop:cell-scale-linear-model-graph} the sheet piece $Y^{Q,a}\cap Q$ is a single
$C^1$ graph of slope $\le \varepsilon$ over its calibrated template $k$--simplex $E\subset \Pi$ in the flat chart,
and the face slice $\Sigma_F(u_a)$ is (up to translation by $u_a$ in the face chart) the graph image of the corresponding
template facet $\sigma\subset E$ lying in $F$.
Translation does not change mass, so it suffices to estimate the slice at $u=0$.
By the area formula and Lemma~\ref{lem:small-graph-distortion},
\[
\Mass(\Sigma_F(u_a))\ \le\ (1+C\varepsilon^2)\,\mathcal H^{k-1}(\sigma).
\]
Since $E$ is uniformly fat, Lemma~\ref{lem:corner-simplex-face-mass} gives
$\mathcal H^{k-1}(\sigma)\le C_\star\,v_E^{(k-1)/k}$ where $v_E:=\mathcal H^{k}(E)$.
Finally, Lemma~\ref{lem:sliver-stability} compares the graph mass in $Q$ with the template volume:
$v_E\asymp m_{Q,a}=\Mass([Y^{Q,a}]\llcorner Q)$ (with constants depending only on $X,n,p$ and the fixed fatness bounds).
Combining these estimates yields
\[
\Mass(\Sigma_F(u_a))\ \le\ C\,m_{Q,a}^{\frac{k-1}{k}},
\]
as claimed.
\end{proof}

\end{editamirblockNEW}
\end{lemma}


\begin{corollary}[Global flat-norm bound from weighted face control (sliver-compatible)]\label{cor:global-flat-weighted}
\begin{editamirblockNEW}
Assume the hypotheses of Proposition~\ref{prop:transport-flat-glue-weighted} on each interior interface face $F$ between adjacent mesh cells,
and let $T^{\mathrm{raw}}$ be the global raw current obtained by summing the cell-wise template currents.
For each such face $F$, denote by $B_F$ the boundary mismatch current supported near $F$.
If the parameter multisets on the two sides admit a matching $\sigma$ with
$\|u_a-u'_{\sigma(a)}\|_\infty\le \Delta_F$, then
\[
\mathcal F(B_F)\ \le\ \Delta_F \sum_{a\in \mathcal S(F)}\Bigl(\Mass(\Sigma_F(u_a))+\Mass(\partial\Sigma_F(u_a))\Bigr),
\]
where $\mathcal S(F)$ indexes the pieces meeting $F$ and $\Sigma_F(u_a)$ is the associated sliver slice along $F$.
Consequently,
\[
\mathcal F(\partial T^{\mathrm{raw}})\ \le\ \sum_F \mathcal F(B_F)
\ \le\ \sum_F \Delta_F \sum_{a\in \mathcal S(F)}\Bigl(\Mass(\Sigma_F(u_a))+\Mass(\partial\Sigma_F(u_a))\Bigr).
\]

In the vertex-template holomorphic-sliver regime, Lemma~\ref{lem:face-slice-cycle-mass} supplies
$\partial\Sigma_F(u_a)=0$ and $\Mass(\Sigma_F(u_a))\le \editamir{C\, m_{Q,a}^{\frac{k-1}{k}}}$ (with $k:=2n-2p$)
for every slice meeting a cell $Q$.
Assuming in addition the schedule/face parameterization control of Lemma~\ref{lem:face-displacement} (so that $\editamir{\Delta_F=O(\varrho\,h^2)}$ on all interior faces),
we obtain
\[
\mathcal F(\partial T^{\mathrm{raw}})\ \le\ \editamir{C\,\varrho\, h^2} \sum_Q \sum_{a\in \mathcal S(Q)} m_{Q,a}^{\frac{k-1}{k}},
\]
for a constant $C$ depending only on $X$ (and the fixed geometric data), not on $h$ or the multiplicities.
\end{editamirblockNEW}
\end{corollary}

\begin{proof}
\editamir{Apply Proposition~\ref{prop:transport-flat-glue-weighted} facewise, then sum over interfaces and use the triangle inequality for $\mathcal F$.}

Since $T^{\mathrm{raw}}=\sum_Q S_Q$, we have
\[
\partial T^{\mathrm{raw}}=\sum_Q \partial S_Q.
\]
On each interface $F=Q\cap Q'$, the restriction of $\partial T^{\mathrm{raw}}$ to $F$ is exactly the mismatch current
\(
B_F=(\partial S_Q)\llcorner F-(\partial S_{Q'})\llcorner F
\)
(with the induced orientations), and hence
\(
\partial T^{\mathrm{raw}}=\sum_F B_F
\)
as a sum over all interfaces $F$.
By the triangle inequality for the flat norm,
\[
\mathcal F(\partial T^{\mathrm{raw}})
\ \le\
\sum_F \mathcal F(B_F).
\]

For a fixed interface $F$, the translation model hypothesis and a matching $\sigma$ with
$\|u_a-u'_{\sigma(a)}\|\le \Delta_F$ give the per-face estimate
\[
\mathcal F(B_F)\ \le\ \Delta_F\sum_{a=1}^N \Bigl(\Mass(\Sigma_F(u_a))+\Mass(\partial\Sigma_F(u_a))\Bigr),
\]
so summing over $F$ yields the first bound.

Under the additional assumptions $\editamir{\Delta_F\le C\,\varrho\,h^2}$ and
$\Mass(\Sigma_F(u_a))+\Mass(\partial\Sigma_F(u_a))\lesssim m_a^{\frac{k-1}{k}}$ (with $k=2n-2p$),
we obtain
\[
\mathcal F(B_F)\ \editamir{\lesssim\ \varrho\,h^2}\sum_{a\in\mathcal S(F)} m_{F,a}^{\frac{k-1}{k}}.
\]
Finally, each piece $Y^{Q,a}\llcorner Q$ meets only $O(1)$ interfaces of its cell, so reorganizing the sum over faces into a sum over
cells and their pieces gives
\[
\mathcal F\!\left(\partial T^{\mathrm{raw}}\right)
\ \editamir{\lesssim\ \varrho\,h^2}\sum_Q\ \sum_{a\in\mathcal S(Q)} m_{Q,a}^{\frac{k-1}{k}},
\]
as claimed.
\end{proof}



\begin{remark}[Consistency with the constant-mass-per-sheet template regime]
If every piece in a cell has comparable mass $m_{Q,a}\asymp h^{k}$ (the naive “one sheet type’’ model), then
$m_{Q,a}^{(k-1)/k}\asymp h^{k-1}$ and $\sum_a m_{Q,a}^{(k-1)/k}\asymp N_Q h^{k-1}\asymp M_Q/h$, where $M_Q=\sum_a m_{Q,a}$ is the total mass in $Q$.
The corollary then yields $\mathcal F(\partial T^{\mathrm{raw}})\lesssim \editamir{\varrho\,h^2}\sum_Q(M_Q/h)=\editamir{\varrho\,h}\,\sum_Q M_Q\editamir{\asymp m\,\varrho\,h}$,
recovering the unweighted “template’’ scaling from Remark~\ref{rem:w1-auto}.
\end{remark}

\begin{remark}[Scaling consequence: weighted gluing + packing]\label{rem:weighted-scaling}
\begin{editamirblockNEW}
\noindent\textbf{Parameter synchronization for holomorphic corner--exit.}
In the holomorphic realization steps (Propositions~\ref{prop:finite-template} and \ref{prop:holomorphic-corner-exit-L1}), the symbols
$N,h,\varepsilon$ must be synchronized as follows (here $N$ denotes the tensor power of $L$ in the Bergman/holomorphic inputs, while $m$ denotes the fixed cohomology multiplier in $\mathrm{PD}(m[\gamma])$):
\begin{itemize}
\item choose $N$ large and set the cell scale $h$ so that $h\sim c\,N^{-1/2}$ (Bergman scale in Lemma~\ref{lem:bergman-control});
\item choose the graph-slope/separation parameter $\varepsilon=\varepsilon(N)\downarrow0$ as required in Proposition~\ref{prop:finite-template};
\item when using a direction net (Proposition~\ref{prop:corner-exit-template-net}), take $\varepsilon_h\lesssim\varepsilon$ so the net resolves the
relevant directions at the same scale as the slope parameter;
\item \emph{track the net constants} $\alpha_*(h),A_*(h),\Lambda(h)$ from Proposition~\ref{prop:corner-exit-template-net} and enforce the
corner-exit scale restriction
\[
s\ \le\ \frac{c_0}{C(n,p)}\cdot \frac{h}{(1+A_*(h))\,\Lambda(h)}
\]
whenever Proposition~\ref{prop:holomorphic-corner-exit-L1} is invoked uniformly over labels;
\item take the translation separation in Proposition~\ref{prop:finite-template} at the footprint scale:
\editamir{if $D_Q:=\max_a\mathrm{diam}((P+t_a)\cap Q)$, then require $\delta:=10\,\varepsilon\,D_Q$ (in the corner-exit regime $D_Q\asymp s\asymp \varrho h$).}
\end{itemize}
\editamir{Note: read this scaling remark only after the global coherence construction is established (Proposition~\ref{prop:global-coherence-all-labels}), to avoid any circular use of the conclusion.}
\end{editamirblockNEW}

Assume we are in the regime where adjacent cells use the same translation template and their face parameterizations differ by $O(h)$,
so Lemma~\ref{lem:face-displacement} gives \editamir{$\Delta_F\lesssim \varrho\,h^2$}.
Assume further that in each cell, each family of disjoint $C^1$ sliver graphs over a fixed direction has slope $\le \varepsilon$ and satisfies the separation
needed for disjointness; then Lemma~\ref{lem:sliver-packing} yields $N_Q\lesssim \varepsilon^{-2p}$ pieces per family.
Writing $M_Q:=\sum_{a\in\mathcal S(Q)} m_{Q,a}$, the concavity/H\"older bound gives
\[
\sum_{a\in\mathcal S(Q)} m_{Q,a}^{\frac{k-1}{k}}
\ \le\ M_Q^{\frac{k-1}{k}}\;|\mathcal S(Q)|^{\frac1k}
\ \lesssim\ M_Q^{\frac{k-1}{k}}\;\varepsilon^{-\frac{2p}{k}},
\qquad k:=2n-2p.
\]
Combining with Corollary~\ref{cor:global-flat-weighted} and $M_Q\asymp m h^{2n}$ yields the global scaling
\[
\mathcal F(\partial T^{\mathrm{raw}})
\ \lesssim\ \varrho\;m^{\frac{k-1}{k}}\,h^{\,2-\frac{2n}{k}}\;\varepsilon^{-\frac{2p}{k}}.
\]
At the intrinsic Bergman cell size $h\sim N^{-1/2}$ this becomes
\[
\frac{\mathcal F(\partial T^{\mathrm{raw}})}{m}\ \lesssim\ \varrho\;m^{-\frac1k}\,N^{-\frac{k-n}{k}}\;\varepsilon^{-\frac{2p}{k}},
\]
and in particular $\mathcal F(\partial T^{\mathrm{raw}})=o(m)$ can be achieved for fixed $m$ by choosing a refinement schedule
$h\downarrow 0$ with $\varepsilon=\varepsilon(h)\downarrow 0$ slowly enough (and, in the borderline case $p=n/2$, by ensuring the refined displacement
regime of Lemma~\ref{lem:borderline-p-half}).
By Remark~\ref{rem:lefschetz-reduction}, it suffices for the unconditional Hodge program to treat $p\le n/2$, which lies in this range.
\end{remark}

\begin{remark}[On vanishing per-piece masses (no hidden lower bound)]\label{rem:no-vanishing-piece-mass}
The weighted flat-norm estimate of Corollary~\ref{cor:global-flat-weighted}
\[
\mathcal F(\partial T^{\mathrm{raw}})\ \editamir{\lesssim\ \varrho\,h^2}\sum_Q\sum_{a\in\mathcal S(Q)} m_{Q,a}^{\frac{k-1}{k}}
\]
holds \emph{without} any hypothesis that the individual piece masses $m_{Q,a}$ are bounded below by a fixed multiple of $h^{k}$.
This is crucial in the sliver regime, where one may intentionally split a cell budget $M_Q$ into many tiny pieces in order to obtain large
template degrees of freedom and good interface matching.

\smallskip\noindent
What the gluing bookkeeping needs is instead a \emph{no-heavy-tail} condition: along each face, tail pieces created by a prefix edit must not carry
disproportionately large face-slice boundary mass compared to the matched prefix.  In the corner-exit route this is enforced by deterministic
face incidence (G1-iff) and uniform per-face comparability (G2) for holomorphic corner-exit slivers
(Proposition~\ref{prop:holomorphic-corner-exit-g1g2} and Corollary~\ref{cor:holomorphic-corner-exit-inherits}), together with the prefix-tail reduction
in Lemma~\ref{lem:oh-face-edit-regime}.
\end{remark}

\end{editblock}

\begin{editblock}
\begin{remark}[Model scaling at the Bergman cell size]\label{rem:sliver-bergman-scaling}
This remark records a simplified scaling calculation explaining why a “sliver’’ mechanism could, in principle, coexist with the intrinsic
holomorphic control scale $h\sim N^{-1/2}$ at tensor power $L^{\otimes N}$.

\smallskip\noindent
Assume cells have diameter $h\asymp N^{-1/2}$ (as suggested by Lemma~\ref{lem:bergman-control}) so that uniform $C^1$ graph control holds on each cell.
Then the number of cells is $\asymp h^{-2n}\asymp N^{n}$, and the target mass per cell is
\[
M_Q\ \sim\ m\int_Q \beta\wedge\psi\ \asymp\ m\,h^{2n}\ \asymp\ m\,N^{-n}.
\]
In a smooth convex flat model (e.g.\ a ball cell), if $M_Q$ is split into $N_Q$ \emph{equal} sliver pieces of mass $M_Q/N_Q$, then the
$(2n-2p-1)$--dimensional boundary size of a single piece scales like $(M_Q/N_Q)^{\frac{k-1}{k}}$ (with $k:=2n-2p$), hence the total boundary size
on the cell boundary scales like
\[
\mathrm{Bdry}(Q)\ \asymp\ N_Q\Bigl(\frac{M_Q}{N_Q}\Bigr)^{\frac{k-1}{k}}
\ =\ M_Q^{\frac{k-1}{k}}\,N_Q^{\frac1k}.
\]
If, across a shared interface, the corresponding face slices are displaced by $\|v\|=O(h^2)$ (as in the template/face-map variation \editamir{estimate (cf.\ Lemma~\ref{lem:template-displacement})}),
then Lemma~\ref{lem:flat-translate} gives a per-piece flat mismatch $\lesssim \|v\|\times$(boundary mass).  A crude summation therefore yields a
\editamir{per-face mismatch of order} $h^n\,\varepsilon_{\mathcal T}$, and summing over the $\sim h^{-n}$ faces yields a
\[
\mathcal F(B_F)\ \editamir{\lesssim\ \varrho\,h^2}\,\mathrm{Bdry}(Q)\ \editamir{\asymp\ \varrho\,h^2}\,M_Q^{\frac{k-1}{k}}\,N_Q^{\frac1k}.
\]
Summing over $\asymp h^{-2n}$ faces gives the global \editamir{scaling} bound
\[
\mathcal F(\partial T^{\mathrm{raw}})\ \lesssim\ h^{-2n}\cdot h^2\cdot M_Q^{\frac{k-1}{k}}\,N_Q^{\frac1k}
\ \asymp\ \varrho\,m^{\frac{k-1}{k}}\,N^{-\frac{k-n}{k}}\,N_Q^{\frac1k}.
\]
In particular, for $p<n/2$ (so $k>n$) the factor $N^{-(k-n)/k}\to 0$ as $N\to\infty$, and one expects
$\mathcal F(\partial T^{\mathrm{raw}})=o(m)$ for fixed $m$ provided $N_Q=o(N^{k-n})$ (e.g.\ polynomial growth in $N$ with exponent $<k-n$).
Making any version of this calculation rigorous inside the cubical/face framework requires precisely the weighted bookkeeping estimate flagged in
Remark~\ref{rem:sliver-vs-template}.
\end{remark}
\end{editblock}

\begin{remark}[Handling slowly varying multiplicities]\label{rem:w1-multiplicity}
In practice the number of sheets in a given family $(Q,j)$ will vary with $Q$ because the target weights depend on $\beta(x_Q)$.
If adjacent cubes $Q,Q'$ have sheet counts differing by $r=|N_{Q,j}-N_{Q',j}|$, one can view their face measures as arising from the
same template after $r$ insertions/deletions.  Lemma~\ref{lem:w1-template-edit} then gives an additional contribution
$W_1\lesssim r\,h$ (since the transverse domain has diameter $O(h)$).
Thus, once one has a quantitative bound $r\le C\,h\,N_{Q,j}$ (slow variation), this term is of order
$W_1\lesssim h^2 N_{Q,j}$ and is absorbed into the $h^2 N$ scaling of Lemma~\ref{lem:w1-auto}.
Making this “slow variation of integer counts” rigorous is a rounding/Diophantine bookkeeping problem, separate from the geometric transport estimates.
\end{remark}

\begin{editblock}
\begin{lemma}[Flat norm of a cycle supported in diameter $\lesssim h$]\label{lem:flat-diameter}
Let $S$ be an integral $\ell$-cycle in $\R^d$ with finite mass.
Assume $\mathrm{diam}(\mathrm{spt}\,S)\le D$.
Then
\[
\mathcal F(S)\ \le\ C(\ell)\,D\,\Mass(S).
\]
In particular, if $\mathrm{diam}(\mathrm{spt}\,S)\lesssim h$ then $\mathcal F(S)\lesssim h\,\Mass(S)$.
\end{lemma}

\begin{proof}
Fix $x_0$ in the convex hull of $\mathrm{spt}\,S$, so that $\|x-x_0\|\le D$ for all $x\in \mathrm{spt}\,S$.
Consider the straight-line homotopy $H:[0,1]\times\R^d\to\R^d$ given by
\(
H(t,x)=(1-t)x+t x_0.
\)
Let $Q:=H_\#([0,1]\times S)$.
Since $S$ is a cycle, $\partial([0,1]\times S)=\{1\}\times S-\{0\}\times S$, and therefore
\[
\partial Q
=H_\#(\{1\}\times S)-H_\#(\{0\}\times S)
=0-S
=-S,
\]
because $H(1,\cdot)\equiv x_0$ is constant and pushes any positive-dimensional current to $0$.
Thus $\partial(-Q)=S$, so taking $R=0$ in the definition of $\mathcal F$ gives $\mathcal F(S)\le \Mass(Q)$.

Finally, the cone/Jacobian estimate for $H$ yields $\Mass(Q)\le C(\ell)\,D\,\Mass(S)$ for a constant $C(\ell)$ depending only on $\ell$.
Combining gives the claim.
\end{proof}


\begin{lemma}[Template displacement $\Rightarrow$ per-face flat-norm mismatch]\label{lem:template-displacement}
Work in the setting of Proposition~\ref{prop:transport-flat-glue}\textnormal{(a)}--\textnormal{(b)} on an interior interface $F=Q\cap Q'$ at mesh $h$.
\editamir{In the global-coherence regime (Proposition~\ref{prop:global-coherence-all-labels}), the boundary slices on $F$ are parameterized by the \emph{same} integer-weighted discrete measure}
\editamir{$\nu=\sum_{a=1}^{N_F} w_a\,\delta_{y_a}$ supported in a ball of radius $C_0\,\varrho h\subset\R^{2p}$ via linear face maps}
$\mu_{Q\to F}=(\Phi_{Q,F})_\#\nu$ and $\mu_{Q'\to F}=(\Phi_{Q',F})_\#\nu$.
Assume $\|\Phi_{Q,F}\|_{\mathrm{op}}+\|\Phi_{Q',F}\|_{\mathrm{op}}\le C_{\Phi,0}$ and $\|\Phi_{Q,F}-\Phi_{Q',F}\|_{\mathrm{op}}\le C_\Phi h$.
Then, after pairing atoms by the identity pairing $y_a\leftrightarrow y_a$, the mismatch current $B_F$ satisfies
\[
\mathcal F(B_F)\ \le\ \editamir{C\,\varrho\,h^2}\,\Biggl(\sum_{a=1}^{N_F} w_a\Bigl(\Mass(\Sigma_{\Phi_{Q,F}y_a})+\Mass(\partial\Sigma_{\Phi_{Q,F}y_a})\Bigr)
\;+\sum_{a=1}^{N_F} w_a\Bigl(\Mass(\Sigma_{\Phi_{Q',F}y_a})+\Mass(\partial\Sigma_{\Phi_{Q',F}y_a})\Bigr)\Biggr)\ +\ C_{\angle}\,\varepsilon\,M_F,
\]
where $M_F$ denotes the total $(2n-2p)$-mass of pieces meeting the interface (so $M_F\lesssim M_Q+M_{Q'}$) and
$\varepsilon$ is the small-angle/graph parameter from Proposition~\ref{prop:transport-flat-glue}\textnormal{(a)}.
\end{lemma}

\begin{proof}
Write $\nu=\sum_{a=1}^{N_F} w_a\,\delta_{y_a}$.
In the flat/parallel model ($\varepsilon=0$), the slice current on $F$ associated to a parameter $z\in\R^{2p}$ is a translate of a fixed model slice:
$\Sigma_z=(\tau_z)_\#\Sigma_0$ in the face chart.
Thus
\[
(\partial S_Q)\llcorner F=\sum_{a=1}^{N_F} w_a\,\Sigma_{\Phi_{Q,F}y_a},
\qquad
(\partial S_{Q'})\llcorner F=\sum_{a=1}^{N_F} w_a\,\Sigma_{\Phi_{Q',F}y_a},
\]
and hence
\[
B_F=\sum_{a=1}^{N_F} w_a\bigl(\Sigma_{\Phi_{Q,F}y_a}-\Sigma_{\Phi_{Q',F}y_a}\bigr).
\]
For each atom $y_a$ define the translation vector $v_a:=(\Phi_{Q,F}-\Phi_{Q',F})y_a$.
\editamir{Since $\|y_a\|\le C_0\varrho h$ and $\|\Phi_{Q,F}-\Phi_{Q',F}\|_{\mathrm{op}}\le C_\Phi h$, we have $\|v_a\|\le C\,\varrho\,h^2$.}
Lemma~\ref{lem:flat-translate} then gives
\[
\mathcal F\!\bigl(\Sigma_{\Phi_{Q,F}y_a}-\Sigma_{\Phi_{Q',F}y_a}\bigr)
\le \|v_a\|\Bigl(\Mass(\Sigma_{\Phi_{Q,F}y_a})+\Mass(\partial\Sigma_{\Phi_{Q,F}y_a})\Bigr)
\le \editamir{C\,\varrho\,h^2}\Bigl(\Mass(\Sigma_{\Phi_{Q,F}y_a})+\Mass(\partial\Sigma_{\Phi_{Q,F}y_a})\Bigr).
\]
By subadditivity of $\mathcal F$ and summing over $a$ (with weights $w_a$),
\[
\mathcal F(B_F)\le \editamir{C\,\varrho\,h^2}\sum_{a=1}^{N_F} w_a\,\Bigl(\Mass(\Sigma_{\Phi_{Q,F}y_a})+\Mass(\partial\Sigma_{\Phi_{Q,F}y_a})\Bigr).
\]
The same bound holds with $Q$ and $Q'$ swapped; combining yields the symmetric form stated.

For $\varepsilon>0$, write each actual boundary slice on $F$ as a pushforward of its flat/parallel model slice
by a Lipschitz graph map in the tubular chart.
Hypothesis \textnormal{(a)} gives a uniform displacement bound $\delta\lesssim\varepsilon h$ and a uniform Lipschitz bound.
Applying Lemma~\ref{lem:flat-C0-deform} to each slice yields a flat-norm error of size
\[
\mathcal F\!\bigl(\Sigma^{\mathrm{act}}_y-\Sigma^{\mathrm{flat}}_y\bigr)\ \le\ C\,\varepsilon\,h\Bigl(\Mass(\Sigma^{\mathrm{flat}}_y)+\Mass(\partial\Sigma^{\mathrm{flat}}_y)\Bigr).
\]
\editamir{Summing over the integer-weighted family of slices meeting $F$ it therefore suffices to bound
$\sum_{a}w_a\bigl(\Mass(\Sigma^{\mathrm{flat}}_{\Phi_{Q,F}y_a})+\Mass(\partial\Sigma^{\mathrm{flat}}_{\Phi_{Q,F}y_a})\bigr)$
in terms of the total interior mass $M_F$ of pieces meeting the interface.
In the small-angle graph regime the model slice $\Sigma^{\mathrm{flat}}$ is the boundary trace of a single-sheet piece inside a cell of diameter $\asymp h$,
so by the uniform geometric slice estimate (e.g.\ Lemma~\ref{lem:uniformly-convex-slice-boundary} in the rounded-cell model, or the corner-simplex face-mass bound in the vertex-template model)
one has $\Mass(\Sigma^{\mathrm{flat}}_y)\lesssim h^{-1}\,\Mass(\text{corresponding piece})$ (and $\partial\Sigma^{\mathrm{flat}}_y=0$ after edge-trimming, cf.\ Lemma~\ref{lem:face-slice-cycle-mass}).
Summing over the finitely many pieces meeting $F$ yields
$\sum_a w_a(\Mass(\Sigma^{\mathrm{flat}}_{\Phi_{Q,F}y_a})+\Mass(\partial\Sigma^{\mathrm{flat}}_{\Phi_{Q,F}y_a}))\lesssim h^{-1}M_F$,
and hence the total model-error contribution is $\lesssim \varepsilon h\cdot (h^{-1}M_F)=C_{\angle}\,\varepsilon\,M_F$ as claimed.}
\end{proof}


\begin{lemma}[Template displacement with insertions/deletions]\label{lem:template-displacement-edits}
Work in the setting of Lemma~\ref{lem:template-displacement} on an interior interface $F=Q\cap Q'$ at mesh $h$.
Assume the two sides admit template representations
\[
\mu_{Q\to F}=(\Phi_{Q,F})_\#\nu,
\qquad
\mu_{Q'\to F}=(\Phi_{Q',F})_\#\nu',
\]
where $\nu$ and $\nu'$ are integer-weighted discrete measures supported in $\editamir{B_{C_0\varrho h}(0)}\subset\R^{2p}$ and the face maps satisfy
$\|\Phi_{Q,F}\|_{\mathrm{op}}+\|\Phi_{Q',F}\|_{\mathrm{op}}\le C_{\Phi,0}$ and $\|\Phi_{Q,F}-\Phi_{Q',F}\|_{\mathrm{op}}\le C_\Phi h$.
Write $\nu=\nu^{\wedge}+\nu^{+}$ and $\nu'=\nu^{\wedge}+\nu^{-}$, where $\nu^{\wedge}$ is any common submeasure (matched part) and
$\nu^{\pm}$ are the unmatched remainders (insertions/deletions).
Let $B_F^{\wedge}$ be the mismatch current coming from the matched part $\nu^{\wedge}$ and let $B_F^{\mathrm{un}}$ be the mismatch current
coming from the unmatched part (so $B_F=B_F^{\wedge}+B_F^{\mathrm{un}}$).
Then
\[
\mathcal F(B_F^{\wedge})\ \le\ \editamir{C\,\varrho\,h^2}\Bigl(\Mass(\partial S_Q\llcorner F)+\Mass(\partial S_{Q'}\llcorner F)\Bigr)\ +\ C_{\angle}\,\varepsilon\,M_F,
\]
and, moreover,
\[
\mathcal F(B_F^{\mathrm{un}})\ \le\ C\,h\,\Mass(B_F^{\mathrm{un}})\ \le\ C\,h\Bigl(\Mass(\partial S_Q\llcorner F)+\Mass(\partial S_{Q'}\llcorner F)\Bigr),
\]
where $C$ depends only on $(n,p,X)$ and the uniform tubular-face charts.
\end{lemma}

\begin{proof}
The matched part $B_F^{\wedge}$ is obtained by applying the two face maps to the \emph{same} common submeasure $\nu^{\wedge}$.
Therefore Lemma~\ref{lem:template-displacement} applies directly and yields the stated bound for $B_F^{\wedge}$.

For the unmatched part, $B_F^{\mathrm{un}}$ is an integral $(k-1)$--cycle supported on the face patch $F$.
Since $\mathrm{diam}(F)\lesssim h$, Lemma~\ref{lem:flat-diameter} gives
\[
\mathcal F(B_F^{\mathrm{un}})\ \le\ C\,h\,\Mass(B_F^{\mathrm{un}}).
\]
Finally, $\Mass(B_F^{\mathrm{un}})$ is bounded by the total face boundary mass coming from the unpaired sheets, hence by
\(
\Mass(\partial S_Q\llcorner F)+\Mass(\partial S_{Q'}\llcorner F).
\)
Combining these yields the claimed inequalities.
\end{proof}


\begin{lemma}[If edits are an $O(h)$ fraction, they are $h^2$ in flat norm]\label{lem:template-edits-oh}
In the setting of Lemma~\ref{lem:template-displacement-edits}, assume moreover that the unmatched part satisfies
\[
\Mass(B_F^{\mathrm{un}})\ \le\ \theta_F\Bigl(\Mass(\partial S_Q\llcorner F)+\Mass(\partial S_{Q'}\llcorner F)\Bigr)
\]
for some $\theta_F\in[0,1]$.
Then
\[
\mathcal F(B_F)\ \le\ C\,h^2\Bigl(\Mass(\partial S_Q\llcorner F)+\Mass(\partial S_{Q'}\llcorner F)\Bigr)\ +\ C\,h\,\theta_F\Bigl(\Mass(\partial S_Q\llcorner F)+\Mass(\partial S_{Q'}\llcorner F)\Bigr)\ +\ C_{\angle}\,\varepsilon\,M_F.
\]
In particular, if $\theta_F\lesssim h$ then the unmatched contribution is of the same $h^2\times(\text{boundary mass})$ order as the matched displacement term.
\end{lemma}

\begin{proof}
Decompose $B_F=B_F^{\wedge}+B_F^{\mathrm{un}}$ as in Lemma~\ref{lem:template-displacement-edits}.
Lemma~\ref{lem:template-displacement-edits} gives the $h^2$--scale bound for $\mathcal F(B_F^{\wedge})$ (plus the $C_{\angle}\,\varepsilon\,M_F$ term), and also gives
\(
\mathcal F(B_F^{\mathrm{un}})\le C h\,\Mass(B_F^{\mathrm{un}}).
\)
Using the hypothesis $\Mass(B_F^{\mathrm{un}})\le \theta_F(\Mass(\partial S_Q\llcorner F)+\Mass(\partial S_{Q'}\llcorner F))$ and subadditivity of $\mathcal F$
yields the stated inequality for $\mathcal F(B_F)$.
\end{proof}


\begin{remark}[Bounded global corrections do not spoil the $O(h)$ edit regime]\label{rem:bounded-corrections}
In applications, one often needs to adjust rounded counts by a bounded amount (e.g.\ to enforce finitely many global period constraints).
If $N_Q\gtrsim h^{-1}$ uniformly and $\widetilde N_Q:=N_Q+\Delta_Q$ with $|\Delta_Q|\le C_0$, then
\[
\frac{|\widetilde N_Q-N_Q|}{\widetilde N_Q}\ \le\ \frac{C_0}{\widetilde N_Q}\ \lesssim\ C_0\,h.
\]
Thus such bounded corrections create only an $O(h)$ \emph{fraction} of insertions/deletions in a nested prefix-template scheme
(Remark~\ref{rem:nested-template-scheme}) and are absorbed by Lemma~\ref{lem:template-edits-oh} for $h\ll 1$.
\end{remark}

\begin{remark}[Nested prefix-template scheme]\label{rem:nested-template-scheme}
Fix, for each direction label, an \emph{ordered} master template of transverse atoms $(y_a)_{a\ge 1}\subset \editamir{B_{C_0\varrho h}(0)}\subset\R^{2p}$.
For example, Lemma~\ref{lem:sphere-quantize-nested} produces a nested ordered sequence on a sphere (uniform density), and scaling embeds it into $\editamir{B_{C_0\varrho h}(0)}$.
For each cell $Q$ choose an integer count $N_Q$ and take the cell template to be the prefix
\(
\nu^{(N_Q)}:=\sum_{a=1}^{N_Q}\delta_{y_a}.
\)
Then across an interface $F=Q\cap Q'$ the two sides differ by a \emph{prefix edit} of size $|N_Q-N_{Q'}|$.
If the target counts come from rounding a smooth density, Lemma~\ref{lem:slow-variation-discrepancy} implies $|N_Q-N_{Q'}|/N_Q=O(h)$ in the “many pieces’’ regime.
Thus it suffices to ensure the \emph{unpaired boundary slice mass} on $F$ is an $O(h)$ fraction of the total face boundary mass; Lemma~\ref{lem:template-edits-oh}
then upgrades this to an $O(h^2)$ flat-norm contribution, matching the displacement bookkeeping.
\end{remark}

\begin{proposition}[Prefix templates $\Rightarrow$ interface coherence up to $O(h)$ edits]\label{prop:prefix-template-coherence}
Work in the setting of Lemma~\ref{lem:template-displacement-edits} on an interior interface $F=Q\cap Q'$ at mesh $h$.
Fix an ordered template of transverse atoms $(y_a)_{a\ge 1}\subset \editamir{B_{C_0\varrho h}(0)}\subset\R^{2p}$ and define prefixes
\[
\nu^{(N)}\ :=\ \sum_{a=1}^{N}\delta_{y_a}.
\]
Assume the two sides arise from prefixes:
\[
\mu_{Q\to F}=(\Phi_{Q,F})_\#\nu^{(N_Q)},\qquad
\mu_{Q'\to F}=(\Phi_{Q',F})_\#\nu^{(N_{Q'})},
\]
and write $B_F$ for the resulting mismatch current on $F$.
If the unmatched part satisfies the $O(h)$-fraction hypothesis
\[
\Mass(B_F^{\mathrm{un}})\ \le\ \theta_F\Bigl(\Mass(\partial S_Q\llcorner F)+\Mass(\partial S_{Q'}\llcorner F)\Bigr)
\qquad\text{with}\qquad \theta_F\lesssim h,
\]
then
\[
\mathcal F(B_F)\ \le\ C\,h^2\Bigl(\Mass(\partial S_Q\llcorner F)+\Mass(\partial S_{Q'}\llcorner F)\Bigr)\ +\ C_{\angle}\,\varepsilon\,M_F,
\]
with $C$ depending only on $(n,p,X)$ and the uniform tubular-face charts.
\end{proposition}

\begin{proof}
Let $N_{\min}:=\min\{N_Q,N_{Q'}\}$ and decompose the two prefixes into a common matched prefix plus tails:
\[
\nu^{(N_Q)}=\nu^{(N_{\min})}+\nu^{+},
\qquad
\nu^{(N_{Q'})}=\nu^{(N_{\min})}+\nu^{-}.
\]
This is exactly the decomposition in Lemma~\ref{lem:template-displacement-edits} with $\nu^{\wedge}=\nu^{(N_{\min})}$.
Applying Lemma~\ref{lem:template-displacement-edits} controls the matched displacement contribution and bounds the unmatched part by the diameter estimate.
Then Lemma~\ref{lem:template-edits-oh} (using $\theta_F\lesssim h$) upgrades the unmatched contribution to the same $h^2$ scale.
\end{proof}


\begin{theorem}[Global prefix-template activation / mass matching (template bookkeeping)]\label{thm:sliver-mass-matching-on-template}
Fix a mesh-$h$ decomposition by smooth uniformly convex cells (rounded cubes) and fix a direction label $j$ with paired calibrated reference planes across neighbors.
Fix an \emph{ordered} master template of transverse atoms $(y_a)_{a\ge 1}\subset \editamir{B_{C_0\varrho h}(0)}\subset\R^{2p}$.
For each cell $Q$, let $N_Q\in\Z_{\ge 0}$ be the desired integer count for family $j$ (derived from the Lipschitz target weights) and let
$M_Q\ge 0$ be the corresponding target mass budget for that family (obtained from the smooth form $m\beta$).
Assume:
\begin{enumerate}
\item[\textnormal{(i)}] (\textbf{Many pieces}) $N_Q\gtrsim h^{-1}$ on the region where $M_Q$ is not negligible;
\item[\textnormal{(ii)}] (\textbf{Slow variation}) $|N_Q-N_{Q'}|\le C\,h\,\min\{N_Q,N_{Q'}\}$ for adjacent cells $Q\sim Q'$;
\item[\textnormal{(iii)}] (\textbf{Local realizability on a fixed template}) for each $Q$ there exist disjoint $\psi$--calibrated holomorphic pieces
$Y^1,\dots,Y^{N_Q}$ in $Q$ whose transverse parameters are the prefix $\{y_a\}_{a\le N_Q}$, and whose total mass satisfies
\[
\sum_{a=1}^{N_Q}\Mass([Y^a]\llcorner Q)\ =\ M_Q\ +\ o(M_Q)
\]
as $h\to 0$ (uniformly over $Q$).
\item[\textnormal{(iv)}] (\textbf{$O(h)$ edit regime on faces}) For every interior interface $F=Q\cap Q'$, the unmatched part satisfies the
$O(h)$--fraction hypothesis of Proposition~\ref{prop:prefix-template-coherence}.
\end{enumerate}
Then the resulting raw current built from these pieces satisfies the per-face flat-norm mismatch bound of Proposition~\ref{prop:prefix-template-coherence}.
Consequently one obtains the global estimate
\[
\mathcal F(\partial T^{\mathrm{raw}})\ \editamir{\lesssim\ \varrho\,h^2}\sum_Q\sum_{a\in\mathcal S(Q)} m_{Q,a}^{\frac{k-1}{k}}\ +\ O(\varepsilon\,m),
\qquad k:=2n-2p,
\]
where $m_{Q,a}:=\Mass([Y^{Q,a}]\llcorner Q)$ and $\varepsilon$ is the small-angle parameter.
In particular, under the parameter regime of Remark~\ref{rem:weighted-scaling} (e.g.\ Bergman scale $h\sim N^{-1/2}$ at holomorphic power $N\to\infty$, polynomial piece count per cell, and $p\le n/2$),
one has $\mathcal F(\partial T^{\mathrm{raw}})=o(m)$.
\end{theorem}


\begin{proof}
For each interior interface $F=Q\cap Q'$, Proposition~\ref{prop:prefix-template-coherence} provides a bound of the form
\[
\mathcal F(B_F)
\ \le\ C\,h^2\Bigl(\Mass(\partial S_Q\llcorner F)+\Mass(\partial S_{Q'}\llcorner F)\Bigr)\ +\ C_{\angle}\,\varepsilon\,M_F,
\]
where $M_F$ is the total interior mass of pieces meeting $F$.
Summing over all interior faces and using subadditivity of $\mathcal F$ gives
\[
\mathcal F(\partial T^{\mathrm{raw}})
\le \sum_F \mathcal F(B_F)
\le C\,h^2\sum_F\Bigl(\Mass(\partial S_Q\llcorner F)+\Mass(\partial S_{Q'}\llcorner F)\Bigr)\ +\ O(\varepsilon\,m),
\]
since $\sum_F M_F\lesssim m$ (each piece meets only $O(1)$ faces).

\smallskip\noindent
Each face boundary mass is a sum of slice masses $\Mass(\Sigma_F(u_a))$ coming from pieces $Y^{Q,a}\cap Q$ meeting $F$.
By Lemma~\ref{lem:uniformly-convex-slice-boundary},
\[
\Mass(\Sigma_F(u_a))\ \lesssim\ m_{Q,a}^{\frac{k-1}{k}},
\qquad m_{Q,a}:=\Mass([Y^{Q,a}]\llcorner Q),\qquad k:=2n-2p.
\]
Therefore,
\[
\sum_F\Bigl(\Mass(\partial S_Q\llcorner F)+\Mass(\partial S_{Q'}\llcorner F)\Bigr)
\ \lesssim\ \sum_Q\sum_{a\in\mathcal S(Q)} m_{Q,a}^{\frac{k-1}{k}},
\]
because each piece contributes to only finitely many faces. \editamir{(Finite-overlap depends only on the fixed cell-complex combinatorics.)}
Substituting yields the stated global estimate for $\mathcal F(\partial T^{\mathrm{raw}})$.
Finally, the $o(m)$ conclusion follows from the scaling/packing computation in Remark~\ref{rem:weighted-scaling}.
\end{proof}



\begin{remark}[Status of the activation hypotheses in the corner-exit route]\label{rem:activation-hypotheses-status}
Theorem~\ref{thm:sliver-mass-matching-on-template} is stated as a bookkeeping reduction: it converts per-cell realization and an $O(h)$ face-edit regime
into the global flat-norm bound needed for gluing.
In the corner-exit vertex-template construction, the hypotheses are verified as follows.
\begin{itemize}
\item \textbf{(i)--(ii)} Many pieces and slow variation follow from rounding Lipschitz targets: see Lemma~\ref{lem:slow-variation-rounding} and the
$0$--$1$ stability Lemma~\ref{lem:slow-variation-discrepancy} (the lower bound $N_Q\gtrsim h^{-1}$ holds on regions where the target density is bounded below).
\item \textbf{(iii)--(iv)} Local realizability on a fixed ordered template and the $O(h)$ face-edit regime are certified for corner-exit vertex templates by
Corollary~\ref{cor:corner-exit-iii-iv} (using Propositions~\ref{prop:holomorphic-corner-exit-L1}, \ref{prop:vertex-template-mass-matching},
and \ref{prop:vertex-template-face-edits} / \ref{prop:checkerboard-face-oh-edit}).
\item \textbf{All labels simultaneously (B1)} The all-direction packaged execution is recorded in Proposition~\ref{prop:global-coherence-all-labels}.
\end{itemize}
Thus the “global activation gate’’ is unconditional in the corner-exit route; the remaining work is purely expository (keeping these references prominent at the point of use).
\end{remark}


\begin{proposition}[Flat-ball model: prefix activation is feasible]\label{prop:prefix-activation-flat-ball}
In the Euclidean ball-cell model of Proposition~\ref{prop:flat-sliver-local}, fix a radius $r\in(0,h)$ so that each affine piece
$[P+t]\llcorner B_h(0)$ with $t\in S^{2p-1}(r)$ has the same mass $\mu(r)$.
Fix $\delta>0$ and an ordered $\delta$--separated list $(t_a)_{a=1}^{N}\subset S^{2p-1}(r)$ (so $N$ is finite), and for each $1\le N'\le N$ define the prefix
\(
\nu^{(N')}:=\sum_{a=1}^{N'} \delta_{t_a}.
\)
Then for any target mass $M\ge 0$, choosing $N':=\lfloor M/\mu(r)\rceil$ (and taking $N$ large enough that $N'\le N$) gives
\[
\Bigl|\sum_{a=1}^{N'} \Mass([P+t_a]\llcorner B_h(0))\ -\ M\Bigr|\ \le\ \mu(r),
\qquad
\frac{\mu(r)}{M}\ =\ O\!\left(\frac1{N'}\right)\ \text{ when }M\gg \mu(r).
\]
Moreover, if two neighboring cells choose counts $N_1$ and $N_2$ with $|N_1-N_2|\le \theta\,\min\{N_1,N_2\}$, then the induced prefix edit is a $\theta$--fraction
of the pieces (hence of the face-boundary mass, since all pieces have comparable slice boundary by the ball scaling law).
\end{proposition}

\begin{proof}
Since $Q=B_h(0)$ is rotationally symmetric, the cross-sectional volume
\(
\Mass([P+t]\llcorner B_h(0))=\mathcal H^{2(n-p)}\bigl((P+t)\cap B_h(0)\bigr)
\)
depends only on $\|t\|$ (equivalently, only on the distance from the center to the affine plane $P+t$).
Hence it is constant on the sphere $S^{2p-1}(r)$; denote this constant by $\mu(r)$.

For the mass-budget estimate, take $N=\lfloor M/\mu(r)\rceil$.  Then by nearest-integer rounding,
\(
|N\mu(r)-M|\le \mu(r),
\)
which is exactly the displayed inequality.

For the edit claim, suppose two cells choose counts $N$ and $N'$, and assume (as in the ball model) that the relevant face-slice boundary masses are equal
or uniformly comparable across indices.
Then the unmatched tail has size $|N-N'|$, so the unmatched face boundary mass is a fraction $\asymp |N-N'|/\min\{N,N'\}\le \theta$ of the total.
\end{proof}


\begin{corollary}[Holomorphic prefix activation on a Bergman-scale ball cell]\label{cor:prefix-activation-holo}
In the setting of Corollary~\ref{cor:holomorphic-flat-sliver-local}, take $\rho\equiv 1$ on the sphere $S^{2p-1}(r)$ and choose a separated ordered template
$(t_a)_{a=1}^{N}$ as in Proposition~\ref{prop:prefix-activation-flat-ball}.
Then the resulting holomorphic pieces $Y^1,\dots,Y^N$ on the cell $Q$ satisfy
\[
\Mass([Y^a]\llcorner Q)=(1+O(\varepsilon^2))\,\mu(r)
\qquad\text{for all }a,
\]
so selecting a prefix of length $N_Q$ matches a target mass budget $M_Q$ up to a relative error $O(1/N_Q)+O(\varepsilon^2)$, and prefix edits of size
$|N_Q-N_{Q'}|$ contribute only an $O(|N_Q-N_{Q'}|/\min\{N_Q,N_{Q'}\})$ fraction of face-boundary mass.
\end{corollary}
\begin{proof}
When $\rho$ is constant on the sphere $S^{2p-1}(r)$, the flat slices in the template have equal mass:
each affine piece over $P+t_a$ contributes the same interior mass $\mu(r)$ on the cell $Q$.
The ordered template from the flat prefix-activation construction therefore has the property that taking a prefix of length $N_Q$
produces total interior mass $N_Q\,\mu(r)$, so choosing $N_Q$ by rounding a target mass budget produces a relative error $O(1/N_Q)$.
Moreover, on a fixed face $F$ the per-piece face-slice boundary masses are equal (or uniformly comparable) across the template,
so changing from $N_Q$ to $N_{Q'}$ across a neighbor interface affects only the unmatched tail of size $|N_Q-N_{Q'}|$ and hence changes
the face boundary mass by an $O(|N_Q-N_{Q'}|/\min\{N_Q,N_{Q'}\})$ fraction.

The holomorphic upgrade replaces each affine slice by a holomorphic complete intersection piece $Y^a$ that is a $C^1$ graph of slope $O(\varepsilon)$,
hence its Jacobian differs from the affine Jacobian by $1+O(\varepsilon^2)$.
In particular,
\(
\Mass([Y^a]\llcorner Q)=(1+O(\varepsilon^2))\,\mu(r)
\)
for every $a$, and the same $1+O(\varepsilon^2)$ comparability holds for the face-slice boundary masses on any interface.
Therefore the flat prefix activation conclusions transfer verbatim, with the additional $O(\varepsilon^2)$ relative error claimed in the statement.
\end{proof}



\begin{lemma}[A sufficient condition for the $O(h)$ face-edit regime]\label{lem:oh-face-edit-regime}
\begin{editamirblockNEW}
Fix an interior interface $F=Q\cap Q'$ and a paired direction label $j$, and assume $N_Q\ge N_{Q'}$.
Write $N_{\min}:=N_{Q'}$ and $r:=N_Q-N_{Q'}$.
Let the face-slice boundary masses on $F$ of the pieces indexed by the master template be
\[
b_a(F)\ :=\ \Mass\!\big(\partial([Y^a]\llcorner Q)\llcorner F\big)\ \ge\ 0,
\qquad a=1,\dots,N_Q,
\]
so that $\Mass(\partial S_Q\llcorner F)=\sum_{a=1}^{N_Q} b_a(F)$.
Assume:
\begin{enumerate}
\item[\textnormal{(a)}] (\textbf{Prefix activation on the face}) after aligning the order of indices across $Q$ and $Q'$,
the paired part on $F$ is exactly the common prefix $\{1,\dots,N_{\min}\}$, and the unpaired part on $F$ is the tail
$\{N_{\min}+1,\dots,N_{\min}+r\}$ coming from the larger side;
\item[\textnormal{(b)}] (\textbf{No heavy tail}) there exists $\kappa\ge 1$ such that every tail term is bounded by the prefix average:
\[
b_{a}(F)\ \le\ \kappa\cdot \frac{1}{N_{\min}}\sum_{i=1}^{N_{\min}} b_i(F)
\qquad\text{for all }a>N_{\min};
\]
\item[\textnormal{(c)}] (\textbf{Slow count variation}) $r\le C\,h\,N_{\min}$.
\end{enumerate}
Then the unpaired face boundary mass satisfies the $O(h)$-fraction hypothesis
\[
\sum_{a>N_{\min}} b_a(F)\ \le\ \theta_F\sum_{a\le N_Q} b_a(F)
\qquad\text{with}\qquad \theta_F\ \le\ (\kappa C)\,h.
\]
In particular, hypothesis (iv) in Theorem~\ref{thm:sliver-mass-matching-on-template} holds (after absorbing constants).
\end{editamirblockNEW}
\end{lemma}

\begin{proof}
By (b),
\[
\sum_{a>N_{\min}} b_a(F)\ \le\ r\cdot \kappa\,\frac{1}{N_{\min}}\sum_{i=1}^{N_{\min}} b_i(F).
\]
By (c), $r\le C h N_{\min}$, hence the right-hand side is $\le (\kappa C)h \sum_{i=1}^{N_{\min}} b_i(F)\le (\kappa C)h \sum_{a\le N_Q} b_a(F)$.
\end{proof}

\begin{remark}[Item \textnormal{(iv)}: tail-heaviness and how it is enforced]\label{rem:iv-what-remains}
\begin{editamirblockNEW}
Lemma~\ref{lem:oh-face-edit-regime} isolates the only nontrivial ingredient needed for the $O(h)$ face-edit estimate in item~\textnormal{(iv)} of Theorem~\ref{thm:sliver-mass-matching-on-template}: when passing from a prefix template to a longer template, the added ``tail'' pieces must not contribute a disproportionate amount of face-slice boundary mass on any interior face $F$.

In the present paper this requirement is built into the \emph{finite corner-exit direction net} and the associated activation rule.
By Proposition~\ref{prop:corner-exit-template-net}, all slivers in a fixed template label have identical footprint geometry; in particular, for each fixed interior face $F$ the per-piece face-slice boundary masses are \emph{uniform} within the label.
Consequently the tail-heaviness hypothesis \textnormal{(b)} in Lemma~\ref{lem:oh-face-edit-regime} holds with $\kappa=1$ (uniformly in the refinement parameter), and item~\textnormal{(iv)} follows by combining Lemma~\ref{lem:oh-face-edit-regime} with the checkerboard face-edit estimate of Proposition~\ref{prop:checkerboard-face-oh-edit}.
\editamir{(In the vertex-template activation route, Proposition~\ref{prop:vertex-template-face-edits} provides the same $O(h)$ face-edit conclusion without invoking checkerboarding.)}
\end{editamirblockNEW}

\smallskip
For intuition, in the dense-sheet translation-invariant model the same uniformity is automatic because each sheet is a translate of a fixed slice current; the net construction above is the robust replacement used in the sliver regime.
\end{remark}
\end{editblock}

\begin{remark}[Parameter tension and the chosen regime]\label{rem:param-tension}
\begin{editamirblockNEW}
There is a genuine tension between (a) taking many pieces per cube (to average fluctuations in counting) and (b) keeping the face mismatch small enough that the flat gluing error tends to zero as the mesh is refined.
The present paper resolves this tension by working in the \emph{sliver / corner-exit regime} and by using \emph{weighted} matching rather than raw counting:

\begin{itemize}
\item The finite direction net and the corner-exit realization produce, within each activated template label, pieces with identical footprint geometry (Proposition~\ref{prop:corner-exit-template-net}). In particular, per-piece slice and boundary masses are uniform at the label level.
\item The global matching is then performed at the level of integer-weighted face measures and transported with flat control (Propositions~\ref{prop:global-coherence-all-labels} and~\ref{prop:transport-flat-glue-weighted}), yielding the required $W_1$-type matching while keeping the induced boundary mismatch $o(m)$ in the weighted scaling (Remark~\ref{rem:weighted-scaling}).
\end{itemize}

In this way, the proof does not rely on a dense-sheet cancellation principle; the smallness of the gluing error is obtained directly from the weighted transport bounds together with the finite-net uniformity.
\end{editamirblockNEW}
\end{remark}

\begin{editblock}
\begin{remark}[Hard Lefschetz reduction to $p\le n/2$]\label{rem:lefschetz-reduction} \cite[Ch.~6]{Voisin02}
Because $X$ is projective, the K\"ahler class $[\omega]=c_1(L)$ is algebraic (hyperplane class).
By hard Lefschetz, for $p>\frac{n}{2}$ the map
\[
L^{2p-n}:\ H^{2(n-p)}(X,\Q)\longrightarrow H^{2p}(X,\Q),\qquad \eta\mapsto [\omega]^{2p-n}\wedge \eta,
\]
is an isomorphism.  Since $[\omega]\in H^{1,1}(X)\cap H^{2}(X,\Q)$, this is a $\Q$--linear morphism of Hodge structures, so its inverse
preserves both rationality and Hodge type.  Hence any rational Hodge class $\gamma\in H^{2p}(X,\Q)\cap H^{p,p}(X)$ can be written uniquely as
$\gamma=[\omega]^{2p-n}\wedge\eta$ with $\eta\in H^{2(n-p)}(X,\Q)\cap H^{n-p,n-p}(X)$.
If $\eta$ is represented by an algebraic cycle $Z$ of codimension $(n-p)$, then intersecting $Z$ with $(2p-n)$ generic hyperplanes produces
an algebraic cycle representing $\gamma$.
Therefore, for the unconditional closure of the Hodge conjecture, it is enough to prove the realization step for $p\le \frac{n}{2}$.
\end{remark}
\end{editblock}

\begin{lemma}[Mass tunability of plane slices in the flat model]\label{lem:mass-tunable}
In the flat chart model, fix a calibrated affine $(2n-2p)$-plane $P\subset\R^{2n}$ and a \editblue{\emph{smooth convex} cell $Q$ of diameter $h$
(e.g.\ a Euclidean ball, or a cube with rounded corners).}
The function
\[
t\ \longmapsto\ \Mass\big([P+t]\llcorner Q\big)
\]
is continuous in the translation parameter $t\in P^\perp\cong\R^{2p}$ and takes values in an interval $[0,A_{\max}]$ with $A_{\max}\asymp h^{2(n-p)}$.
In particular, for any $a\in(0,A_{\max})$ there exist translations $t$ such that $\Mass([P+t]\llcorner Q)=a$.
\end{lemma}


\begin{proof}
Write $k:=2(n-p)$.  In the flat model one has
\[
\Mass([P+t]\llcorner Q)=\mathcal H^{k}\bigl((P+t)\cap Q\bigr).
\]
Continuity in $t$ follows because this is the integral of the indicator function $\mathbf 1_Q$ over the translated plane:
for any sequence $t_\nu\to t$, the sets $(P+t_\nu)\cap Q$ converge to $(P+t)\cap Q$ in the sense of characteristic functions on $P$
after identifying $P+t_\nu$ with $P$ by translation, and dominated convergence applies since $\mathbf 1_Q$ is bounded.

The maximum $A_{\max}$ is achieved by some translate intersecting the bulk of $Q$ and satisfies $A_{\max}\asymp h^{k}$
because $Q$ contains and is contained in Euclidean balls of radii comparable to $h$ (uniform convexity/diameter control).
The value $0$ occurs for translates $P+t$ far enough that $(P+t)\cap Q=\emptyset$.
Therefore the image contains an interval $[0,A_{\max}]$, and the intermediate value theorem yields translations realizing any $a\in(0,A_{\max})$.
\end{proof}


\begin{remark}[Sliver pieces and fixed-$m$ microstructure]\label{rem:sliver}
Lemma~\ref{lem:mass-tunable} indicates a potential escape from the dense-vs-gluing tension at fixed $m$:
one may take \emph{many} parallel calibrated sheets in a cube but choose their translations so that each sheet contributes only a tiny mass
(``sliver pieces''), with the total mass still matching $m\int_Q\beta\wedge\psi$.
If such tunability persists under the holomorphic complete-intersection upgrade (Substep~3.5) with uniform control, then one can have
large sheet counts per face (good for $W_1$ matching) while keeping the total mass $O(m)$.
Making this quantitative in the projective setting is part of the remaining realization problem.
\end{remark}

\begin{editblock}
\begin{lemma}[Quantizing a Lipschitz density on a sphere]\label{lem:sphere-quantize}
Let $d\ge 2$ and let $S^{d-1}(r)\subset\R^d$ be the Euclidean sphere of radius $r>0$.
Let $\rho$ be a nonnegative Lipschitz function on $S^{d-1}(r)$ with total mass
\[
M:=\int_{S^{d-1}(r)} \rho\,d\sigma.
\]
Then for every $N\in\N$ there exist points $t_1,\dots,t_N\in S^{d-1}(r)$ such that the equal-weight atomic measure
\[
\mu_N:=\sum_{a=1}^N \frac{M}{N}\,\delta_{t_a}
\]
satisfies the transport bound
\[
W_1(\mu_N,\rho\,d\sigma)\ \le\ C(d)\,r\,\Bigl(M+\mathrm{Lip}(\rho)\,r^{d-1}\Bigr)\,N^{-\frac{1}{d-1}}.
\]
Moreover, the points may be chosen $\delta$--separated with
\[
\|t_a-t_b\|\ \ge\ c(d)\,r\,N^{-\frac{1}{d-1}}
\qquad (a\neq b).
\]
\end{lemma}


\begin{proof}
This is a standard $W_1$ quantization bound on the $(d\!-\!1)$--sphere.
One concrete route is to start from a maximal $\delta$--separated set $\{t_a\}\subset S^{d-1}(r)$ with
\(
\delta\asymp r\,N^{-1/(d-1)},
\)
which has cardinality $\asymp N$ by packing.  By choosing the implicit constant in $\delta\asymp r\,N^{-1/(d-1)}$ small enough, we may assume this maximal set has
cardinality at least $N$, and then select any $N$ of its points (preserving $\delta$--separation).
Let $\{C_a\}$ be the associated Voronoi cells; then $\mathrm{diam}(C_a)\lesssim \delta$.

Define the cell-averaged atomic measure $\widetilde\mu:=\sum_a \bigl(\int_{C_a}\rho\,d\sigma\bigr)\delta_{t_a}$.
Transporting the mass of each cell $C_a$ to its representative $t_a$ gives
\[
W_1(\widetilde\mu,\rho\,d\sigma)\ \le\ \sum_a \mathrm{diam}(C_a)\int_{C_a}\rho\,d\sigma\ \lesssim\ \delta\,M.
\]
To convert $\widetilde\mu$ to the equal-weight measure $\mu_N=\sum_{a=1}^N \frac{M}{N}\delta_{t_a}$, rebalance the atomic weights.
Since $\rho$ is Lipschitz and each cell has diameter $\lesssim\delta$, the discrepancy between the cell masses and the equal weight $M/N$
is controlled at scale $\lesssim \mathrm{Lip}(\rho)\,\delta\,r^{d-1}$.
Rebalancing these weights can be done by transporting mass between nearby cells at cost $\lesssim \delta$ per unit mass, yielding the stated bound
\(
W_1(\mu_N,\rho\,d\sigma)\lesssim \delta\,(M+\mathrm{Lip}(\rho)\,r^{d-1}).
\)
We record the rate and dependencies here; a detailed implementation of this standard quantization argument can be found, for example, in texts on optimal quantization
or empirical $W_1$ convergence on compact manifolds.
\end{proof}


\begin{lemma}[Nested equal-weight quantization of the uniform sphere]\label{lem:sphere-quantize-nested}
Let $d\ge 2$ and let $S^{d-1}(r)\subset\R^d$ be the Euclidean sphere of radius $r>0$, with normalized surface measure $\sigma_r$.
There exists an (infinite) sequence of points $(t_a)_{a\ge 1}\subset S^{d-1}(r)$ such that for every $N\ge 1$ the equal-weight empirical measure
\[
\mu_N\ :=\ \frac{1}{N}\sum_{a=1}^N \delta_{t_a}
\]
satisfies
\[
W_1(\mu_N,\sigma_r)\ \le\ C(d)\,r\,N^{-\frac{1}{d-1}}.
\]
\end{lemma}

\begin{proof}
Build a nested sequence of partitions of $S^{d-1}(r)$ into $\asymp 2^{(d-1)k}$ measurable cells at level $k$, each of diameter $\lesssim r\,2^{-k}$
and with $\sigma_r$-mass exactly $2^{-(d-1)k}$ (for example, by inductively bisecting cells by smooth hypersurfaces; existence of equal-area partitions with
controlled diameter is standard on the sphere).
Choose one representative point in each cell and enumerate these points in increasing level order to obtain a single infinite sequence $(t_a)_{a\ge 1}$.

For $N\asymp 2^{(d-1)k}$, the first $N$ points consist of one representative from each cell at level $k$.
Transporting the mass of each cell to its representative costs at most $\mathrm{diam}(\text{cell})\cdot\sigma_r(\text{cell})\lesssim r\,2^{-k}\cdot 2^{-(d-1)k}$,
and summing over the $2^{(d-1)k}$ cells yields $W_1(\mu_N,\sigma_r)\lesssim r\,2^{-k}\asymp r\,N^{-1/(d-1)}$.
For intermediate $N$, compare to the nearest dyadic level and absorb constants.
\end{proof}

\end{editblock}

\begin{editblock}
\begin{proposition}[Flat ball model slivers achieve $W_1$ transverse approximation]\label{prop:flat-sliver-local}
Work in the flat decomposition $\R^{2n}=\R^{2(n-p)}\oplus\R^{2p}$ and let $P:=\R^{2(n-p)}\times\{0\}$.
Let $Q:=B_h(0)\subset\R^{2n}$ be the Euclidean ball of radius $h$.
Fix a radius $r\in(0,h)$ and let $\sigma_r$ denote surface measure on $S^{2p-1}(r)\subset P^\perp\cong\R^{2p}$.
Let $\rho$ be a nonnegative Lipschitz density on $S^{2p-1}(r)$ with total mass
$M=\int_{S^{2p-1}(r)}\rho\,d\sigma_r$.
Then for every $N\in\N$ there exist translations $t_1,\dots,t_N\in S^{2p-1}(r)$ such that the affine calibrated pieces
\[
T_N\ :=\ \sum_{a=1}^N \bigl([P+t_a]\llcorner Q\bigr)
\]
are pairwise disjoint and:
\begin{enumerate}
\item[\textnormal{(i)}] (\textbf{Equal sliver masses}) $\Mass([P+t_a]\llcorner Q)=\Mass([P+t_1]\llcorner Q)$ for all $a$ (depends only on $r$);
\item[\textnormal{(ii)}] (\textbf{Transverse $W_1$ approximation}) with $\mu_N:=\sum_{a=1}^N \frac{M}{N}\delta_{t_a}$ one has
\[
W_1(\mu_N,\rho\,d\sigma_r)\ \le\ C(p)\,r\,\Bigl(M+\mathrm{Lip}(\rho)\,r^{2p-1}\Bigr)\,N^{-\frac{1}{2p-1}}.
\]
\end{enumerate}
\end{proposition}


\begin{proof}
For \textnormal{(i)}, note that $\Mass([P+t]\llcorner Q)=\mathcal H^{2(n-p)}((P+t)\cap B_h(0))$ depends only on the distance from the center to the affine plane
$P+t$, i.e.\ only on $\|t\|$, by rotational symmetry of the Euclidean ball.  Hence it is constant on $S^{2p-1}(r)$.

For \textnormal{(ii)}, apply Lemma~\ref{lem:sphere-quantize} with $d=2p$ to the Lipschitz density $\rho$ on $S^{2p-1}(r)$ to obtain points $t_a\in S^{2p-1}(r)$
such that the equal-weight atomic measure $\mu_N=\sum_{a=1}^N \frac{M}{N}\delta_{t_a}$ satisfies the stated $W_1$ bound.

Disjointness of the pieces $[P+t_a]\llcorner Q$ is immediate because the affine planes $P+t_a$ are parallel and distinct whenever $t_a\neq t_b$.
\end{proof}

\end{editblock}

\begin{editblock}
\begin{corollary}[Holomorphic upgrade on a ball cell]\label{cor:holomorphic-flat-sliver-local}
In the setting of Proposition~\ref{prop:flat-sliver-local}, assume $Q$ lies in a holomorphic chart and that $P$ is a calibrated complex
$(n-p)$-plane in those coordinates with normal covectors $\lambda_1,\dots,\lambda_p$.
Fix $\varepsilon>0$ and choose a holomorphic tensor power \editamir{$N_{\mathrm{hol}}\ge N_1(\varepsilon)$} (Lemma~\ref{lem:bergman-control}) with
$\mathrm{diam}(Q)\le c\,\editamir{N_{\mathrm{hol}}^{-1/2}}$.
Then, after possibly reducing $N$ by a dimensional constant (absorbed into $C(p)$), the translations $t_a$ may be chosen so that
\[
\|t_a-t_b\|\ \ge\ 10\,\varepsilon\,\mathrm{diam}(Q)\qquad (a\neq b),
\]
and Proposition~\ref{prop:finite-template} produces $\psi$-calibrated holomorphic complete intersections $Y^1,\dots,Y^N$ whose restricted
pieces on $Q$ are disjoint $C^1$ graphs over $P+t_a$ with
\[
\Mass([Y^a]\llcorner Q)=(1+O(\varepsilon^2))\,\Mass([P+t_a]\llcorner Q).
\]
Consequently, the induced transverse measure $\sum_a \Mass([Y^a]\llcorner Q)\,\delta_{t_a}$ approximates $\rho\,d\sigma_r$ in $W_1$ with error
bounded by the right-hand side of Proposition~\ref{prop:flat-sliver-local} plus an additional $O(\varepsilon^2)\,M$ term.
\end{corollary}
\begin{proof}
Apply the flat model construction to obtain translations $t_1,\dots,t_N$ and the corresponding affine calibrated pieces over $P+t_a$
with the stated $W_1$ approximation to $\rho\,d\sigma_r$.
By a standard packing/subselection argument on the sphere (discarding at most a dimensional constant fraction of the points),
we may replace the family by a subfamily (renaming and keeping the same notation) so that
\(
\|t_a-t_b\|\ge 10\,\varepsilon\,\mathrm{diam}(Q)
\)
for all $a\neq b$.

With \editamir{$N_{\mathrm{hol}}\ge N_1(\varepsilon)$} and $\mathrm{diam}(Q)\le c\,\editamir{N_{\mathrm{hol}}^{-1/2}}$, the Bergman-scale $C^1$ control and the holomorphic finite-template
construction apply at each translation parameter $t_a$, producing $\psi$-calibrated holomorphic complete intersections
$Y^1,\dots,Y^N$ whose restrictions to $Q$ are disjoint $C^1$ graphs over $P+t_a$ with slope $O(\varepsilon)$.
In particular their masses satisfy
\[
\Mass([Y^a]\llcorner Q)=(1+O(\varepsilon^2))\,\Mass([P+t_a]\llcorner Q)
\qquad\text{for each }a.
\]

Let
\(
\mu_{\mathrm{flat}}:=\sum_a \Mass([P+t_a]\llcorner Q)\,\delta_{t_a}
\)
and
\(
\mu_{\mathrm{holo}}:=\sum_a \Mass([Y^a]\llcorner Q)\,\delta_{t_a}.
\)
The mass comparison gives $\mu_{\mathrm{holo}}=(1+O(\varepsilon^2))\,\mu_{\mathrm{flat}}$, hence
\(
W_1(\mu_{\mathrm{holo}},\mu_{\mathrm{flat}})\lesssim \varepsilon^2\,M
\)
(with the domain diameter absorbed into the implicit constant), where $M=\int_\Omega\rho$ is the total target mass.
Combining this with the $W_1(\mu_{\mathrm{flat}},\rho\,d\sigma_r)$ estimate from the flat model yields the stated conclusion.
\end{proof}


\end{editblock}

\begin{editblock}
\begin{remark}[Interpretation]
Proposition~\ref{prop:flat-sliver-local} shows that the \emph{transverse-measure approximation} requirement in the sliver program is achievable
in a clean flat ball model using exact affine calibrated pieces.
The remaining nontrivial step in this \emph{sliver program} is the \emph{holomorphic complete-intersection upgrade with uniform $C^1$ control}
(captured by Lemma~\ref{lem:bergman-control} and Proposition~\ref{prop:finite-template}) together with cube/face compatibility for gluing.
This conjectural sliver route is included only for context; the unconditional proof in this manuscript proceeds instead via the corner-exit vertex-template mechanism
(Propositions~\ref{prop:holomorphic-corner-exit-L1}, \ref{prop:vertex-template-face-edits}, \ref{prop:glue-gap}, and the all-label package \ref{prop:global-coherence-all-labels})
and does \emph{not} rely on Conjecture~\ref{conj:sliver-local}.
\end{remark}
\end{editblock}

\begin{editblock}
\begin{conjecture}[Local sliver-sheet realizability (quantitative target)]\label{conj:sliver-local}
\textbf{Note.} This conjecture is \emph{not used} in the proof of the main theorems; it is stated only as a quantitative target for an alternative ``sliver'' route.
\smallskip
Fix a sufficiently small \emph{smooth convex} coordinate cell $Q$ of diameter $h$ inside a holomorphic chart
(e.g.\ a geodesic ball, or a cubical cell with rounded corners), and fix a calibrated direction
$P\in K_{n-p}(x_Q)$ with normal space $P^\perp\cong\R^{2p}$.
Let $\rho$ be a nonnegative Lipschitz density on a bounded transverse domain $\Omega\subset P^\perp$ with total mass
$\int_\Omega \rho = M$.
Then for every $N\in\N$ there exist \emph{calibrated} holomorphic complete intersections
$Y^1,\dots,Y^N\subset X$ such that:
\begin{enumerate}
\item[\textnormal{(i)}] (\textbf{Small-angle / graph control}) each $Y^a$ is $C^1$-close to an affine translate $P+t_a$ on $Q$
with $\sup_{y\in Q}\angle(T_yY^a,P)\le \varepsilon(h)$ and $\varepsilon(h)\to 0$ as $h\to 0$;
\item[\textnormal{(ii)}] (\textbf{Sliver masses}) the restricted pieces satisfy
\[
\Mass([Y^a]\llcorner Q)\ \le\ C\,\frac{M}{N}
\qquad\text{for all }a,
\]
and $\sum_a \Mass([Y^a]\llcorner Q)=M+o(1)$;
\item[\textnormal{(iii)}] (\textbf{Transverse measure approximation}) the induced transverse measure
$\mu_N:=\sum_a \Mass([Y^a]\llcorner Q)\,\delta_{t_a}$ satisfies
\[
W_1(\mu_N,\rho\,dt)\ \le\ \tau(N,h),\qquad \tau(N,h)\xrightarrow[N\to\infty,\ h\to 0]{}0.
\]
\end{enumerate}
\end{conjecture}

\begin{remark}[Why we ask for a smooth convex cell]\label{rem:sliver-cell-shape}
The “sliver’’ mechanism relies on being able to make \emph{both} the interior mass and the induced boundary slices small when a sheet translate
approaches the edge of the cell.  This behavior is clean in smooth convex models (e.g.\ balls), where plane sections shrink in a controlled way.
For sharp cubical cells, a plane section can have arbitrarily small $k$-volume while still having $O(h^{k-1})$ boundary on a face (thin long slices),
so additional geometry would be needed to keep boundary slices small.  Thus smooth convexity is a natural technical condition for any rigorous
sliver bookkeeping estimate.
\editblue{One explicit alternative is a \emph{corner-exit / simplex} mechanism, combined with \emph{global vertex templates}: force each sliver footprint inside a cube
to meet only a fixed set of $k\!+\!1$ faces adjacent to a vertex and to have uniformly nondegenerate simplex shape, and choose the slivers from a fixed ordered template
anchored at each grid vertex.  This yields $a\lesssim v^{(k-1)/k}$ even in sharp cubes and also resolves the face-population/prefix obstruction for gluing;
see Proposition~\ref{prop:vertex-template-face-edits}.}
\end{remark}

\subsection*{Sharp-cube variant: corner-exit slivers and global vertex templates (model)}
\begin{remark}[Why templates should live at vertices (pan-vertex distribution)]
If one concentrates all slivers in a cube $Q$ near a single vertex, then an interior face $F=Q\cap Q'$ can be populated on one side and essentially empty on the other,
creating a one-sided mismatch that is not a tail effect.
Moreover, even if both sides use the same \emph{cellwise} master template, it is not automatic that the pieces that actually meet a given face $F$ are the \emph{early}
pieces in the chosen prefix.

\smallskip\noindent
A clean way to remove both issues is to define templates at the \emph{grid vertices} and to distribute each cube’s mass among its vertices.
Then any two cubes sharing a vertex $v$ use the same ordered geometric sequence of slivers anchored at $v$, so across every shared face the mismatch reduces to a
pure prefix-count difference at the shared vertices.
\end{remark}

\begin{definition}[Global vertex template (flat cubical model)]\label{def:vertex-template}
\begin{editamirblockNEW}
Fix a cubical grid in $\R^{2n}$ with mesh $h$ and vertex set $\Lambda:=(h\Z)^{2n}$, and fix a calibrated $(2n-2p)$--plane $P$.
For each vertex $v\in\Lambda$, fix an infinite ordered family of affine planes
\[
P_{v,a}\ :=\ P+v+t_{v,a},\qquad a\ge 1,
\]
with translation vectors $t_{v,a}\in P^\perp$ satisfying the following \emph{cellwise corner-exit} properties.
Let $Q$ be any cube of the grid containing $v$, set $k:=2n-2p$, and write
\[
E_{v,a}(Q)\ :=\ P_{v,a}\cap Q\ \subset\ Q.
\]
\begin{enumerate}
\item[\textnormal{(i)}] (\textbf{Corner localization}) there exists $c_0\in(0,1)$, independent of $h,v,Q,a$, such that
\[
E_{v,a}(Q)\ \subset\ B(v,c_0h)
\qquad\text{for every }a\ge 1.
\]
\item[\textnormal{(ii)}] (\textbf{Uniform corner-exit simplex type}) for each $Q\ni v$, every footprint $E_{v,a}(Q)$ is a $k$--simplex
which meets \emph{exactly} the same set of $k\!+\!1$ coordinate $(2n\!-\!1)$--faces of $Q$ through $v$ (the ``designated exit faces''),
and meets no other codimension--$1$ faces of $Q$.
Equivalently, for any face $F\subset\partial Q$ incident to $v$, either $E_{v,a}(Q)\cap F\neq\emptyset$ for all $a$,
or $E_{v,a}(Q)\cap F=\emptyset$ for all $a$.
\item[\textnormal{(iii)}] (\textbf{Equal / uniformly comparable face-slice masses}) for each $Q\ni v$ and each designated exit face $F\subset\partial Q$
met by $E_{v,a}(Q)$, the slice masses
\[
b_{v,a}(F;Q)\ :=\ \mathcal H^{k-1}\bigl(E_{v,a}(Q)\cap F\bigr)
\]
are independent of $a$ (or, more generally, satisfy
$c\,b_{v,1}(F;Q)\le b_{v,a}(F;Q)\le C\,b_{v,1}(F;Q)$ for all $a$ with constants $c,C$ independent of $h,v,Q$).
\end{enumerate}
We refer to $(P_{v,a})_{a\ge 1}$ as a \emph{global vertex template} for direction $P$.
\end{editamirblockNEW}
\end{definition}


\begin{lemma}[A concrete \emph{complex} corner-exit translation template in a cube]\label{lem:complex-corner-exit-template}
Work in $\C^n=\C^{n-p}\times\C^p$ with coordinates $z=(u,w)$, where $u=(u_1,\dots,u_{n-p})$ and $w=(w_1,\dots,w_p)$.
Let $Q:=[0,h]^{2n}\subset\R^{2n}\cong\C^n$ be the coordinate cube with vertex $0$.
Fix a constant $0<c_0<1$ and choose a scale $s>0$ with $s\le c_0 h/100$.

\smallskip\noindent
Define a complex $(n-p)$--plane $P\subset\C^n$ as the graph of the linear map $A:\C^{n-p}\to\C^p$ given by
\[
w_1\ =\ -(1-i)\sum_{j=1}^{n-p}u_j,\qquad w_2=\cdots=w_p=0.
\]
For translation parameters $t=(t_1,\dots,t_p)\in\C^p$, write $P_t:=\{(u,Au+t):u\in\C^{n-p}\}$ (parallel translate of $P$).
Assume $t$ satisfies the \emph{interior-margin} bounds
\[
\Re t_1=s,\qquad 2s\le \Im t_1\le 3s,
\qquad
2s\le \Re t_j,\Im t_j\le 3s\ \ (2\le j\le p).
\]
Then:
\begin{enumerate}
\item[\textnormal{(i)}] (\textbf{Corner-exit simplex footprint}) The footprint $E(t):=P_t\cap Q$ is a $k$--simplex with $k=2n-2p$,
contained in $B(0,c_0h)$.
\item[\textnormal{(ii)}] (\textbf{Fixed designated exit faces}) The $k\!+\!1$ facets of $E(t)$ lie on the $k\!+\!1$ coordinate faces
\[
F_{\Re u_j=0},\ F_{\Im u_j=0}\ (1\le j\le n-p),\qquad\text{and}\qquad F_{\Re w_1=0},
\]
and $E(t)$ meets no other codimension-$1$ faces of $Q$.
\item[\textnormal{(iii)}] (\textbf{Uniform fatness and equal slice mass}) The family $E(t)$ is uniformly fat (with constants depending only on $(n,p)$),
and $\mathcal H^k(E(t))$ is independent of $t$ in the above parameter box (hence equal across indices).
\end{enumerate}
In particular, this admissible parameter box has real dimension $2p-1$, so for any separation scale $\delta>0$ one can choose an ordered $\delta$--separated
list $(t_a)_{a=1}^{N}$ inside it with identical footprints $P_{t_a}\cap Q$, where $N$ may be taken as large as allowed by packing (equivalently, $N\to\infty$
as $\delta\downarrow 0$ with $s$ fixed).
\end{lemma}
\begin{proof}
Write $u_j=x_j+i y_j$ with $x_j=\Re u_j$ and $y_j=\Im u_j$.
On $P_t$ one computes
\[
\Re w_1\ =\ \Re t_1\ +\ \Re\!\Bigl(-(1-i)\sum_{j=1}^{n-p}u_j\Bigr)
\ =\ s\ -\ \sum_{j=1}^{n-p}(x_j+y_j),
\]
and
\[
\Im w_1\ =\ \Im t_1\ +\ \Im\!\Bigl(-(1-i)\sum_{j=1}^{n-p}u_j\Bigr)
\ =\ \Im t_1\ +\ \sum_{j=1}^{n-p}(x_j-y_j).
\]
The cube constraints on $w_2,\dots,w_p$ are automatic since $w_j\equiv t_j$ and $t_j\in(0,h)^2$ with margin $\gtrsim s$.
Moreover, on the region cut out by $x_j,y_j\ge 0$ and $\sum_j(x_j+y_j)\le s$, one has
$\bigl|\sum_j(x_j-y_j)\bigr|\le \sum_j(x_j+y_j)\le s$, hence
\[
\Im w_1\ \in\ [\Im t_1-s,\ \Im t_1+s]\ \subset\ [s,4s]\ \subset\ (0,h),
\]
so both faces $\{\Im w_1=0\}$ and $\{\Im w_1=h\}$ are avoided.
Likewise $\Re w_1\in[0,s]\subset(0,h)$ avoids $\{\Re w_1=h\}$, and $x_j,y_j\le s\ll h$ avoids the far faces
$\{\Re u_j=h\}$ and $\{\Im u_j=h\}$.

\smallskip\noindent
Consequently, $E(t)=P_t\cap Q$ is cut out on $P_t$ exactly by the inequalities
\[
x_j\ge 0,\quad y_j\ge 0\quad (1\le j\le n-p),\qquad\text{and}\qquad \Re w_1\ge 0,
\]
i.e.\ by $\sum_j(x_j+y_j)\le s$ together with nonnegativity of the $k=2(n-p)$ coordinates $(x_1,y_1,\dots,x_{n-p},y_{n-p})$.
This is the standard $k$--simplex in $\R^{k}$ (embedded linearly as a graph in $\R^{2n}$), proving (i) and (ii).
Uniform fatness follows because this simplex is affine-equivalent to the standard simplex with distortion depending only on the fixed linear map $A$,
and the slice mass $\mathcal H^k(E(t))\asymp s^k$ is independent of $t$ since the defining inequalities do not depend on $t$ inside the admissible box.
Finally, packing a $\delta$--separated family inside a $(2p-1)$--dimensional box gives a quantitative bound $N\gtrsim (s/\delta)^{2p-1}$; in particular, by
taking $\delta$ small one can make the ordered list as long as needed.
\end{proof}


\begin{lemma}[Corner-exit simplex mass scale and no-heavy-tail uniformity]\label{lem:corner-exit-mass-scale}
In the setting of Lemma~\ref{lem:complex-corner-exit-template}, fix a scale $s>0$ and let $E(t)=P_t\cap Q$ be the resulting corner-exit simplex of
dimension $k=2n-2p$.
Then there exist constants $0<c\le C<\infty$ depending only on $(n,p)$ such that for every admissible $t$ (with the fixed scale $s$):
\[
c\,s^{k}\ \le\ \mathcal H^{k}(E(t))\ \le\ C\,s^{k},
\qquad
c\,s^{k-1}\ \le\ \mathcal H^{k-1}(E(t)\cap F_i)\ \le\ C\,s^{k-1}\ \ (i=0,\dots,k),
\]
where $F_0,\dots,F_k$ are the designated exit faces from Lemma~\ref{lem:complex-corner-exit-template}.
In particular, if one chooses $s=\theta\,h$ for a fixed $\theta\in(0,1)$ (so $s$ is a fixed fraction of the cell size), then each footprint has
$\mathcal H^k(E(t))\asymp h^k$ and each designated face slice has $\mathcal H^{k-1}(E(t)\cap F_i)\asymp h^{k-1}$.
Moreover, throughout the admissible parameter box in Lemma~\ref{lem:complex-corner-exit-template} (with fixed $\Re t_1=s$), the footprints are identical,
so $\mathcal H^{k}(E(t))$ and the facet measures $\mathcal H^{k-1}(E(t)\cap F_i)$ are in fact independent of $t$.

\smallskip\noindent
Consequently, an ordered $\delta$--separated list $(t_a)$ in that box yields a template whose pieces have \emph{exactly equal} footprint masses and
per-face slice masses (no heavy tails along the order).  If $Y^a\cap Q$ is an $\varepsilon$--slope graph over $E(t_a)$, then
Lemma~\ref{lem:small-graph-distortion} gives the corresponding holomorphic equal-mass/equal-slice-mass conclusions up to a common $(1+O(\varepsilon^2))$ factor.
\end{lemma}
\begin{proof}
In the proof of Lemma~\ref{lem:complex-corner-exit-template}, $E(t)$ is cut out on the $k$ real coordinates
$(x_1,y_1,\dots,x_{n-p},y_{n-p})\in\R^{k}$ by the inequalities
$x_j\ge 0$, $y_j\ge 0$, and $\sum_j(x_j+y_j)\le s$, which define a standard simplex of size $s$.
Thus $\mathcal H^{k}(E(t))\asymp s^{k}$ and each facet has $\mathcal H^{k-1}\asymp s^{k-1}$, with constants depending only on $k$ (hence only on $(n,p)$).
Independence of $t$ inside the parameter box is immediate because the defining inequalities on $P_t$ do not depend on $t$ once $\Re t_1=s$ is fixed.
Finally, Lemma~\ref{lem:small-graph-distortion} gives the $1+O(\varepsilon^2)$ distortion bounds for small-slope graphs, uniformly in $a$.
\end{proof}


\begin{lemma}[Corner-exit translation templates for a quantitative family of complex planes]\label{lem:corner-exit-template-open}
Work in $\C^n=\C^{n-p}\times\C^p$ with coordinates $z=(u,w)$ and identify $\C^n\cong\R^{2n}$.
Let \editamir{$h>0$ denote the mesh (cube side-length) parameter,} and let $Q:=[0,h]^{2n}$ be the coordinate cube.
Fix $0<c_0<1$ and parameters $\alpha_*,\alpha^*,A_*>0$.

\smallskip\noindent
Let $P\subset\C^n$ be a complex $(n-p)$--plane written as a graph
\[
P\ =\ \{(u,Au):u\in\C^{n-p}\},
\]
for some complex linear map $A:\C^{n-p}\to\C^p$ with operator norm $\|A\|\le A_*$.
\editamir{Suppose that} for some choice of a \emph{slanted} coordinate $w_r$ (one of the $p$ components of $w$), the corresponding row of $A$ has coefficients
$c_j=a_j+i b_j$ ($1\le j\le n-p$) satisfying the quantitative nondegeneracy bounds
\[
\alpha_*\ \le\ |a_j|\ \le\ \alpha^*,\qquad \alpha_*\ \le\ |b_j|\ \le\ \alpha^*\qquad (1\le j\le n-p).
\]
Define the conditioning ratio $\Lambda:=\alpha^*/\alpha_*$.

\begin{editamirblockNEW}
\noindent\textbf{Referee note (tracking the conditioning constants).}
The parameters $\alpha_*,\alpha^*,A_*$ (hence $\Lambda=\alpha^*/\alpha_*$) are \emph{inputs} describing a quantitative transversality of the chosen
``slanted'' coordinate row of $A$ for the specific plane $P$.
When this lemma is applied to a \emph{finite direction net} (Proposition~\ref{prop:corner-exit-template-net}), one typically takes
\[
\alpha_*(h):=\min_{P_i\in\mathcal N_h}\alpha_*(P_i),\qquad 
\alpha^*(h):=\max_{P_i\in\mathcal N_h}\alpha^*(P_i),\qquad 
A_*(h):=\max_{P_i\in\mathcal N_h}A_*(P_i),\qquad 
\Lambda(h):=\alpha^*(h)/\alpha_*(h),
\]
so these constants may depend on $(h,\varepsilon_h)$ as the net is refined.
Unless a uniform-in-$h$ lower bound on $\alpha_*(h)$ is proved, the later scaling schedule must keep the dependence on $(1+A_*(h))\Lambda(h)$ explicit.
\end{editamirblockNEW}

\smallskip\noindent
Then there exists a choice of a \emph{vertex} $v$ of $Q$ (equivalently, a choice of which incident coordinate faces of $Q$ provide the “orthant’’ constraints)
and a choice of a translation parameter $t\in\C^p$ with a scale $s:=|\,\Re t_r\,|$ satisfying
\[
s\ \le\ \frac{c_0}{C(n,p)}\cdot \frac{h}{(1+A_*)\,\Lambda},
\]
such that, writing $P_t:=P+t$ and $E:=P_t\cap Q$, the footprint $E$ is a $k$--simplex ($k=2n-2p$) contained in $B(v,c_0h)$ whose $k\!+\!1$ facets lie on
exactly $k\!+\!1$ coordinate faces of $Q$ incident to $v$ (a designated exit-face set), and the simplex is uniformly fat with constant depending only on
$(n,p,\Lambda)$.

\smallskip\noindent
Moreover, one may choose $t$ from a $(2p\!-\!1)$--dimensional parameter box (fixing $\Re t_r=\pm s$ and varying the remaining real components with margin $\asymp s$),
so that the resulting footprints are \emph{identical} (hence have equal slice mass) throughout that box.  In particular, for any separation scale $\delta>0$ one can
extract an ordered $\delta$--separated list $(t_a)_{a=1}^{N}$ of translations producing identical corner-exit simplex footprints, with $N\to\infty$ as $\delta\downarrow 0$
(for fixed $s$).
\end{lemma}

\begin{proof}
Write $u_j=x_j+i y_j$.
By reflecting real coordinates $x_j\mapsto h-x_j$ and/or $y_j\mapsto h-y_j$ (which corresponds to choosing a vertex $v$ of $Q$),
we may replace $(x_j,y_j)$ by nonnegative coordinates $(x'_j,y'_j)\in[0,h]$ so that the affine inequality
$\Re w_r\ge 0$ restricted to $P_t$ becomes
\[
\sum_{j=1}^{n-p} (|a_j|\,x'_j+|b_j|\,y'_j)\ \le\ s,
\]
after absorbing the resulting additive constants into the choice of $\Re t_r$.
Together with the orthant constraints $x'_j\ge 0$, $y'_j\ge 0$, this cuts out a $k$--simplex in the $k=2(n-p)$ real variables.
The bound $s\ll h$ prevents meeting the far faces in the $u$-coordinates.

\smallskip\noindent
By $\|A\|\le A_*$ and the simplex bound $|u|\lesssim s/\alpha_*$, all other cube coordinates (the remaining $w$ components and the $\Im w_r$ coordinate)
vary by at most $O(A_* s/\alpha_*)$ on $E$.  Choosing the remaining components of $t$ with margin $\asymp s$ and taking
$s\le c_0\,h/(C(1+A_*)\Lambda)$ forces these coordinates to stay in $(0,h)$, so no additional faces are met.
Uniform fatness and volume scaling follow by an affine change of variables on $\R^k$ controlled by $\Lambda$.
\smallskip

\noindent
Finally, to obtain a template family with identical footprints, fix $\Re t_r=\pm s$ and vary the remaining real components of $t$
in a box of sidelength $\asymp s$ chosen so that all the non-$u$ cube coordinates remain strictly inside $(0,h)$ as above.
On this parameter box, the defining inequalities in the $(x'_j,y'_j)$ variables are unchanged, so the footprint in $Q$ is identical for all such $t$.
Extracting a $\delta$--separated ordered list from the box is a standard packing argument in dimension $2p-1$.
\end{proof}


\begin{proposition}[Robust corner-exit templates for a finite direction net]\label{prop:corner-exit-template-net}
Fix $h>0$ and a tolerance $\varepsilon_h>0$.
In any fixed holomorphic coordinate chart, there exists a finite set of calibrated directions
\[
\mathcal N_h=\{P_1,\dots,P_M\}\subset G_\C(n-p,n)
\]
which is an $\varepsilon_h$--net in $G_\C(n-p,n)$ and has the following property:
for each $P_i\in\mathcal N_h$ there is a corner-exit translation template family in the cube $Q=[0,h]^{2n}$ (allowing choice of vertex and exit-face set)
whose footprints are uniformly fat corner-exit simplices, and which supplies, for each $\delta>0$, a $\delta$--separated ordered list of translations of length $N(\delta)$
(with $N(\delta)\to\infty$ as $\delta\downarrow 0$)
with identical footprint geometry (hence uniform per-piece slice mass within each label).
Moreover, because $\mathcal N_h$ is finite, the fatness/locality constants may be chosen \emph{uniformly} over all directions in $\mathcal N_h$.
\end{proposition}

\begin{proof}
Let $\mathcal U\subset G_\C(n-p,n)$ be the set of planes for which there exists some coordinate splitting and some choice of slanted coordinate $w_r$
so that the corresponding row coefficients satisfy $a_j\neq 0$ and $b_j\neq 0$ for all $j$; this is a finite union of complements of algebraic
degeneracy loci (vanishing of Pl\"ucker minors and coordinate coefficients), hence dense.
Start with any $\varepsilon_h/2$--net and perturb each point by $<\varepsilon_h/2$ into $\mathcal U$; compactness gives a finite net $\mathcal N_h\subset\mathcal U$.

\smallskip\noindent
For each $P_i\in\mathcal N_h$, choose a witnessing splitting and slanted coordinate, and let $\alpha_*(i),\alpha^*(i),A_*(i)$ be the resulting quantitative constants.
Since $\mathcal N_h$ is finite and all required coefficients are nonzero, one has
\[
\alpha_*:=\min_i\alpha_*(i)>0,\qquad \alpha^*:=\max_i\alpha^*(i)<\infty,\qquad A_*:=\max_iA_*(i)<\infty,
\]
hence $\Lambda:=\alpha^*/\alpha_*$ is finite.
Apply Lemma~\ref{lem:corner-exit-template-open} with these uniform constants to obtain uniform corner-exit templates for every $P_i$.
\begin{editamirblockNEW}
\noindent\textbf{Net constants $\alpha_*(h),A_*(h),\Lambda(h)$.}
For each fixed $h$ we choose the net $\mathcal N_h\subset\mathcal U$ inside the open ``nondegenerate'' set $\mathcal U$ (defined below) so that the
row-coefficient nonvanishing required by Lemma~\ref{lem:corner-exit-template-open} holds for every $P_i\in\mathcal N_h$.
Because $\mathcal N_h$ is finite, we may define the \emph{net constants}
\[
\alpha_*(h):=\min_i\alpha_*(i),\qquad 
\alpha^*(h):=\max_i\alpha^*(i),\qquad 
A_*(h):=\max_iA_*(i),\qquad 
\Lambda(h):=\alpha^*(h)/\alpha_*(h),
\]
which are uniform \emph{across labels} for this fixed $h$.

\smallskip\noindent
A uniform-in-$h$ positive lower bound for $\alpha_*(h)$ is \emph{not} established here (as $\varepsilon_h\downarrow0$ the net may approach
degeneracy loci where some row coefficients become small).
Therefore we adopt option \textbf{(2)} in the closure chain:
the later parameter schedule (Remark~\ref{rem:weighted-scaling}) keeps the dependence on $(1+A_*(h))\Lambda(h)$ explicit and enforces the
corner-exit scale condition from Lemma~\ref{lem:corner-exit-template-open},
\[
s\ \le\ \frac{c_0}{C(n,p)}\cdot \frac{h}{(1+A_*(h))\,\Lambda(h)},
\]
whenever Proposition~\ref{prop:holomorphic-corner-exit-L1} is applied uniformly over all labels.
\end{editamirblockNEW}

\end{proof}


\begin{remark}[Supplying corner-exit template families for the direction net]\label{rem:corner-exit-direction-net}
The global activation/gluing bookkeeping (Theorem~\ref{thm:sliver-mass-matching-on-template} and Proposition~\ref{prop:checkerboard-face-oh-edit})
is \emph{direction-by-direction}: one fixes a calibrated direction label $j$ and activates an ordered template by choosing only prefix lengths.
Thus, to run the corner-exit route in the holomorphic setting, it suffices to ensure the following for each direction label $j$
in the finite direction net used to approximate $m\beta$ on the mesh:
\begin{itemize}
\item \textbf{(Template existence)} in the local holomorphic chart for a cell $Q$, there is a complex reference plane $P_j$ and a supply of translation
parameters $t_{v,a}^{(j)}$ near each vertex $v$ so that the footprints $(P_j+v+t_{v,a}^{(j)})\cap Q$ are uniformly fat corner-exit simplices with a fixed
designated exit-face set (hence satisfy the geometric hypotheses of Proposition~\ref{prop:holomorphic-corner-exit-g1g2}),
and
\item \textbf{(Holomorphic realization)} these translated templates can be realized by disjoint holomorphic complete intersections on $Q$ with cell-scale single-sheet
graph control.
\end{itemize}

\smallskip\noindent
Lemma~\ref{lem:complex-corner-exit-template} provides a completely explicit complex corner-exit translation template in a coordinate cube, and
Lemma~\ref{lem:corner-exit-template-open} + Proposition~\ref{prop:corner-exit-template-net} provide the robust finite-net supply needed for the global scheme:
one can choose the direction dictionary/net used to approximate $\widehat\beta$ so that \emph{every} direction label admits a corner-exit translation template, with
constants uniform over the finite net.
In practice, one chooses the direction net \emph{inside} the open set of calibrated planes for which an analogous “one-coordinate slanted inequality’’
produces a corner simplex in $Q$ (after choosing the appropriate anchored vertex $v$ and designated faces among those incident to $v$).
Since the net is finite at each mesh scale, all geometric constants (fatness, locality radius $c_0$, and per-face comparability constants) may be taken uniform by
min/max over the finitely many labels.
\end{remark}

\begin{lemma}[Corner-exit simplex slices have optimal boundary scaling]\label{lem:cube-vertex-slice-boundary}
\begin{editamirblockNEW}
Let $Q=[0,h]^{2n}\subset\R^{2n}$ and fix $1\le k<2n$.
Let $E\subset Q$ be a $k$--dimensional simplex such that:
(i) $E\subset B(0,c_0h)$ for some fixed $c_0\in(0,1)$, and
(ii) exactly $k\!+\!1$ of the $(k\!-\!1)$--faces of $E$ lie in the $k\!+\!1$ coordinate $(2n\!-\!1)$--faces of $Q$ through $0$,
with all dihedral angles of $E$ bounded below by a fixed constant (uniform nondegeneracy).
Then there exists $C=C(k,c_0,\textnormal{nondeg})$ such that
\[
\mathcal H^{k-1}(\partial E)\ \le\ C\,\bigl(\mathcal H^{k}(E)\bigr)^{\frac{k-1}{k}}.
\]
\end{editamirblockNEW}
\end{lemma}


\begin{proof}
Let $\Pi$ be the affine $k$--plane containing $E$.  By the uniform nondegeneracy assumption (dihedral angles bounded below),
there exists an affine isomorphism $A:\Pi\to\R^k$ whose distortion (operator norm and inverse norm) is bounded in terms of $k$ alone, such that
$A(E)=\Delta_s$ is a standard $k$--simplex of scale $s$ (i.e.\ affine-equivalent to $\{x\in\R^k: x_i\ge 0,\ \sum_{i=1}^k x_i\le s\}$).

For the standard simplex one computes explicitly
\(
\mathcal H^{k}(\Delta_s)=c_k\,s^k
\)
and
\(
\mathcal H^{k-1}(\partial\Delta_s)=c'_k\,s^{k-1}
\)
for dimensional constants $c_k,c'_k>0$.
Eliminating $s$ yields
\(
\mathcal H^{k-1}(\partial\Delta_s)\le C(k)\,\bigl(\mathcal H^k(\Delta_s)\bigr)^{(k-1)/k}.
\)
Applying the change-of-variables bounds under $A$ (which distort $k$-- and $(k-1)$--dimensional Hausdorff measures by at most a multiplicative factor depending only on $k$)
gives the stated inequality for $E$.
\end{proof}


\begin{proposition}[Vertex-template prefix lengths match local mass budgets (L2, cube model)]\label{prop:vertex-template-mass-matching}
\begin{editamirblockNEW}
Work in the setting of Definition~\ref{def:vertex-template} for a fixed cube $Q$, and \editamir{suppose that} the vertex templates have equal (or uniformly comparable) slice masses as in
Definition~\ref{def:vertex-template}\textnormal{(iii)}.  Assume further that the geometric templates are realized in $Q$ by holomorphic pieces with small-slope graph control,
so that Lemma~\ref{lem:sliver-stability}\textnormal{(i)} applies uniformly on $Q$.

\smallskip\noindent
Let $M_Q\ge 0$ be the target mass budget for this direction family in $Q$.  Choose any vertex splitting $M_{Q,v}\ge 0$ with
\(
\sum_{v\in\mathrm{Vert}(Q)} M_{Q,v}=M_Q
\)
(for instance the equal split $M_{Q,v}=2^{-d}M_Q$).
For each vertex $v\in\mathrm{Vert}(Q)$, let $\mu_{Q,v}>0$ denote the common mass scale in $Q$ for the $v$-anchored template pieces, i.e.
\[
\Mass([Y_{Q,v}^a]\llcorner Q)\ =\ \bigl(1+O(\varepsilon^2)\bigr)\,\mu_{Q,v}\qquad\text{uniformly in }a,
\]
with the implied constant independent of $Q,v,a$.
Define the prefix length by nearest-integer rounding
\[
N_{Q,v}\ :=\ \Bigl\lfloor \frac{M_{Q,v}}{\mu_{Q,v}}\Bigr\rceil .
\]
Then the realized total mass satisfies
\[
\sum_{v\in\mathrm{Vert}(Q)}\sum_{a=1}^{N_{Q,v}} \Mass([Y_{Q,v}^a]\llcorner Q)
\ =\ M_Q\ +\ O\!\left(\sum_{v}\mu_{Q,v}\right)\ +\ O(\varepsilon^2)\,M_Q.
\]
In particular, whenever $M_{Q,v}\gg \mu_{Q,v}$ (equivalently $N_{Q,v}\to\infty$), the relative error per vertex is
\(O(1/N_{Q,v})+O(\varepsilon^2)\).
\end{editamirblockNEW}
\end{proposition}


\begin{proof}
\begin{editamirblockNEW}
Fix a vertex $v$.  By nearest-integer rounding,
\[
\Bigl|N_{Q,v}\,\mu_{Q,v}-M_{Q,v}\Bigr|\ \le\ \frac12\,\mu_{Q,v}.
\]
By the holomorphic small-slope graph control and Lemma~\ref{lem:sliver-stability}\textnormal{(i)}, each realized piece satisfies
\[
\Mass([Y_{Q,v}^a]\llcorner Q)\ =\ \bigl(1+\theta_{Q,v,a}\bigr)\,\mu_{Q,v},
\qquad |\theta_{Q,v,a}|\ \le\ C\,\varepsilon^2,
\]
with $C$ uniform in $Q,v,a$.
Therefore
\[
\sum_{a=1}^{N_{Q,v}}\Mass([Y_{Q,v}^a]\llcorner Q)
\ =\ N_{Q,v}\,\mu_{Q,v}\ +\ O(\varepsilon^2)\,N_{Q,v}\,\mu_{Q,v}.
\]
Using $N_{Q,v}\,\mu_{Q,v}=M_{Q,v}+O(\mu_{Q,v})$ from rounding, we obtain
\[
\Bigl|\sum_{a=1}^{N_{Q,v}}\Mass([Y_{Q,v}^a]\llcorner Q)-M_{Q,v}\Bigr|
\ \le\ \frac12\,\mu_{Q,v}\ +\ O(\varepsilon^2)\,M_{Q,v}\ +\ O(\varepsilon^2)\,\mu_{Q,v}.
\]
Absorbing the last term into the $O(\mu_{Q,v})$ contribution (since $\varepsilon\le 1$ in the regime of interest) yields
\[
\Bigl|\sum_{a=1}^{N_{Q,v}}\Mass([Y_{Q,v}^a]\llcorner Q)-M_{Q,v}\Bigr|
\ \le\ O(\mu_{Q,v})\ +\ O(\varepsilon^2)\,M_{Q,v}.
\]
Summing over the finitely many vertices $v\in\mathrm{Vert}(Q)$ and using $\sum_v M_{Q,v}=M_Q$ gives
\[
\sum_{v}\sum_{a=1}^{N_{Q,v}} \Mass([Y_{Q,v}^a]\llcorner Q)
\ =\ M_Q\ +\ O\!\left(\sum_{v}\mu_{Q,v}\right)\ +\ O(\varepsilon^2)\,M_Q.
\]
Finally, if $M_{Q,v}\gg \mu_{Q,v}$ then $M_{Q,v}\simeq N_{Q,v}\mu_{Q,v}$ and dividing the per-vertex estimate by $M_{Q,v}$ gives the relative error
\(O(1/N_{Q,v})+O(\varepsilon^2)\).
\end{editamirblockNEW}
\end{proof}



\begin{proposition}[Vertex templates $\Rightarrow$ face-level $O(h)$ edit regime (hypothesis \textnormal{(iv)})]\label{prop:vertex-template-face-edits}
\begin{editamirblockNEW}
Work in the setting of Definition~\ref{def:vertex-template}, and fix one paired direction family.
For each cube $Q$ and each vertex $v\in\mathrm{Vert}(Q)$, let $N_{Q,v}\in\Z_{\ge 0}$ and \editamir{suppose that} inside $Q$ we realize the
vertex-prefix $\{P_{v,a}\}_{1\le a\le N_{Q,v}}$ by corresponding (disjoint) pieces, so that the face-slice boundary masses along a face
are indexed by the same order $a=1,2,\dots$.

Assume the \emph{slow-variation} bound holds at every shared vertex: for any two adjacent cubes $Q\sim Q'$ and any shared vertex
$v\in Q\cap Q'$,
\[
|N_{Q,v}-N_{Q',v}|\ \le\ C\,h\,\min\{N_{Q,v},N_{Q',v}\}.
\]
Assume moreover that the face-slice boundary masses of $v$-anchored pieces meeting a fixed interior face $F$ are uniformly comparable
in the index $a$ (with constant $\kappa$ independent of $h,Q,Q',v$), i.e.\ for each such $F$ and $v\in\mathrm{Vert}(F)$,
\[
b_{v,a}(F)\ \le\ \kappa\cdot \frac{1}{N_{\min}}\sum_{i=1}^{N_{\min}} b_{v,i}(F)
\qquad\text{for all }a>N_{\min},
\]
where $N_{\min}:=\min\{N_{Q,v},N_{Q',v}\}$ and $b_{v,a}(F)$ denotes the face-slice boundary mass on $F$ of the $a$-th $v$-anchored piece
(from the side where it is present).

Then for every interior interface face $F=Q\cap Q'$, the unmatched part of the boundary on $F$ satisfies the $O(h)$--fraction hypothesis
of Proposition~\ref{prop:prefix-template-coherence}:
there exists $\theta_F\lesssim h$ (depending only on $C,\kappa$ and dimension) such that
\[
\Mass(B_F^{\mathrm{un}})\ \le\ \theta_F\Bigl(\Mass(\partial S_Q\llcorner F)+\Mass(\partial S_{Q'}\llcorner F)\Bigr),
\]
where $B_F^{\mathrm{un}}$ is the unpaired (tail) part of the mismatch on $F$.
\end{editamirblockNEW}
\end{proposition}


\begin{proof}
\begin{editamirblockNEW}
Fix an interior interface face $F=Q\cap Q'$.
By the corner localization property in Definition~\ref{def:vertex-template}\textnormal{(i)}, any piece anchored at a vertex
$v\notin\mathrm{Vert}(F)$ is supported in $B(v,c_0h)$, which does not intersect $F$.
Hence only pieces anchored at vertices $v\in\mathrm{Vert}(F)$ can contribute to the face-restricted boundaries
$\partial S_Q\llcorner F$ and $\partial S_{Q'}\llcorner F$, and therefore
\[
\Mass(B_F^{\mathrm{un}})\ \le\ \sum_{v\in\mathrm{Vert}(F)} \Mass\bigl(B_{F,v}^{\mathrm{un}}\bigr),
\]
where $B_{F,v}^{\mathrm{un}}$ denotes the unpaired (tail) mismatch on $F$ coming from the $v$-anchored prefixes.

Fix such a vertex $v\in\mathrm{Vert}(F)$ and, without loss of generality, assume $N_{Q,v}\ge N_{Q',v}$.
Set $N_{\min}:=N_{Q',v}$ and $r:=N_{Q,v}-N_{Q',v}$, and write $b_a(F):=b_{v,a}(F)$ for the corresponding ordered face-slice masses
from the $Q$-side (so that the $Q'$-side contributes only the prefix $a\le N_{\min}$).
Because both cubes activate prefixes of the \emph{same} vertex order at $v$, the paired part is $\{1,\dots,N_{\min}\}$ and the unpaired
part is the tail $\{N_{\min}+1,\dots,N_{\min}+r\}$, i.e.\ hypothesis \textnormal{(a)} of Lemma~\ref{lem:oh-face-edit-regime} holds.
The assumed uniform comparability of the face-slice masses gives hypothesis \textnormal{(b)} with constant $\kappa$, and the vertex-wise
slow-variation bound gives hypothesis \textnormal{(c)} with the same constant $C$.
Therefore Lemma~\ref{lem:oh-face-edit-regime} yields
\[
\sum_{a>N_{\min}} b_{v,a}(F)\ \le\ (\kappa C)\,h\sum_{a\le N_{Q,v}} b_{v,a}(F).
\]
Summing this bound over the finitely many vertices $v\in\mathrm{Vert}(F)$ and absorbing the fixed vertex-count into the constant
gives $\Mass(B_F^{\mathrm{un}})\le \theta_F\bigl(\Mass(\partial S_Q\llcorner F)+\Mass(\partial S_{Q'}\llcorner F)\bigr)$ with
$\theta_F\lesssim h$, as claimed.
\end{editamirblockNEW}
\end{proof}



\begin{corollary}[Corner-exit vertex templates verify the activation hypotheses (iii)–(iv)]\label{cor:corner-exit-iii-iv}
\begin{editamirblockNEW}
Fix one direction label $j$ and work on a mesh-$h$ cubulation in the cube model of Definition~\ref{def:vertex-template}.
\editamir{Suppose that} for this label the following are implemented:
\begin{enumerate}
\item[\textnormal{(1)}] (\textbf{Holomorphic corner-exit manufacturing (L1)}) for each cube $Q$ and each vertex $v\in\mathrm{Vert}(Q)$,
the affine planes $\{P_{v,a}\}_{a\ge 1}$ are realized in $Q$ by disjoint $\psi$--calibrated holomorphic pieces
$\{Y_{Q,v}^a\}_{a\ge 1}$ with uniform small-slope graph control, and each pair $(E_{Q,v}^a,Y_{Q,v}^a)$
satisfies the corner-exit face-control conclusions of Proposition~\ref{prop:holomorphic-corner-exit-L1}
(with vertex-star coherence as in Remark~\ref{rem:vertex-star-coherence});
\item[\textnormal{(2)}] (\textbf{Local mass-budget matching (L2)}) for each cube $Q$, the prefix lengths $N_{Q,v}$ are chosen to match the local
vertex budgets as in Proposition~\ref{prop:vertex-template-mass-matching}, so that
\[
\sum_{v\in\mathrm{Vert}(Q)}\sum_{a=1}^{N_{Q,v}} \Mass([Y_{Q,v}^a]\llcorner Q)
\ =\ M_Q\ +\ O\!\left(\sum_{v}\mu_{Q,v}\right)\ +\ O(\varepsilon^2)\,M_Q;
\]
\item[\textnormal{(3)}] (\textbf{Slow variation of counts}) at shared vertices $v\in Q\cap Q'$ one has
$|N_{Q,v}-N_{Q',v}|\lesssim h\,\min\{N_{Q,v},N_{Q',v}\}$
(e.g.\ by Lemma~\ref{lem:slow-variation-rounding} applied to Lipschitz target budgets).
\end{enumerate}
Then, for this direction label $j$, the two nontrivial activation hypotheses \textnormal{(iii)}–\textnormal{(iv)} in
Theorem~\ref{thm:sliver-mass-matching-on-template} hold (after absorbing constants).
Moreover, if one prefers to express the activation via a single ordered per-cube prefix (rather than per-vertex prefixes),
one may implement the block-uniform coded interleaving of Definitions~\ref{def:block-uniform-codes}--\ref{def:checkerboard-anchoring}
and then use Proposition~\ref{prop:checkerboard-face-oh-edit} for the face-edit estimate.
\end{editamirblockNEW}
\end{corollary}
\begin{proof}
\begin{editamirblockNEW}
\emph{Hypothesis \textnormal{(iii)}.}
For each cube $Q$, consider the disjoint family of holomorphic pieces
\[
\mathcal Y_Q\ :=\ \{\,Y_{Q,v}^a:\ v\in\mathrm{Vert}(Q),\ 1\le a\le N_{Q,v}\,\},
\qquad N_Q:=\#\mathcal Y_Q=\sum_{v\in\mathrm{Vert}(Q)}N_{Q,v}.
\]
Assumption \textnormal{(1)} provides the local realizability (existence, calibration, and disjointness) on $Q$,
with transverse parameters inherited from the fixed vertex-template ordering at each $v$.
Assumption \textnormal{(2)} gives the mass match on $Q$ in the quantitative form stated in
Proposition~\ref{prop:vertex-template-mass-matching}.
In particular, on the region where the \emph{many pieces} hypothesis \textnormal{(i)} of
Theorem~\ref{thm:sliver-mass-matching-on-template} holds (so $N_{Q,v}\to\infty$ and $\sum_v\mu_{Q,v}=o(M_Q)$),
and under the parameter regime $\varepsilon\to 0$, the right-hand side equals $M_Q+o(M_Q)$ as required.

\emph{Hypothesis \textnormal{(iv)}.}
Let $F=Q\cap Q'$ be an interior interface face.
By assumption \textnormal{(3)}, the slow-variation bound holds at each shared vertex $v\in\mathrm{Vert}(F)$.
Therefore Proposition~\ref{prop:vertex-template-face-edits} applies and yields that the unmatched boundary mass on $F$
is an $O(h)$ fraction of the total face boundary mass; this is exactly the face-edit regime \textnormal{(iv)} of
Theorem~\ref{thm:sliver-mass-matching-on-template}.
(Alternatively, under the block-uniform coded interleaving, one may invoke Proposition~\ref{prop:checkerboard-face-oh-edit}.)

With \textnormal{(iii)}--\textnormal{(iv)} verified for label $j$, Theorem~\ref{thm:sliver-mass-matching-on-template}
applies to this direction family.
\end{editamirblockNEW}
\end{proof}


\begin{remark}[{\color{blue}Referee map: downstream invocations of Proposition~\ref{prop:holomorphic-corner-exit-L1}}]\label{rem:L1-downstream-map}
\begin{editamirblockNEW}
For later proof-spine checks, Proposition~\ref{prop:holomorphic-corner-exit-L1} is used downstream only through the following interfaces:
\begin{enumerate}
\item Corollary~\ref{cor:corner-exit-iii-iv}: it supplies the holomorphic corner-exit slivers (with (G1-iff)/(G2) and the $L^1$ boundary-face mass control) needed to certify activation hypothesis \textnormal{(iii)} for the vertex-template program.
\item Proposition~\ref{prop:global-coherence-all-labels}: it invokes Corollary~\ref{cor:corner-exit-iii-iv} label-by-label and then applies only the rounding/slow-variation and face-edit machinery; no additional geometric input beyond Proposition~\ref{prop:holomorphic-corner-exit-L1} enters there.
\end{enumerate}
Earlier forward references to Proposition~\ref{prop:holomorphic-corner-exit-L1} (e.g.\ Remark~\ref{rem:weighted-scaling} and Proposition~\ref{prop:corner-exit-template-net})
are parameter-synchronization notes rather than additional proof dependencies.
Constants in the $L^1$ bounds may depend on $(k,\Lambda,\varepsilon)$ as recorded in Proposition~\ref{prop:holomorphic-corner-exit-L1};
uniformity across labels is enforced by the finite direction net and the schedule in Remark~\ref{rem:weighted-scaling}.
\end{editamirblockNEW}
\end{remark}





\begin{proposition}[Global coherence across all direction labels (B1, packaged)]\label{prop:global-coherence-all-labels}
Fix a mesh-$h$ cubulation by coordinate cubes $Q$ (subordinate to a holomorphic atlas) and let $\beta$ be a smooth closed strongly positive $(p,p)$-form.
Fix a small scale $\varepsilon_h\ll h$ and choose, in each chart, an $\varepsilon_h$--net of calibrated directions
$\{P_1,\dots,P_M\}\subset G_\C(n-p,n)$ together with uniform corner-exit translation templates as in Proposition~\ref{prop:corner-exit-template-net}.

\smallskip\noindent
\editamir{Choose \emph{globally labeled}} Lipschitz weights $w_i(x)$ against this dictionary (e.g.\ by the strongly convex simplex fit of
Lemma~\ref{lem:lipschitz-qp-weights} applied to $\widehat\beta(x)$ in local trivializations), and define per-cell target mass budgets
$M_{Q,i}\ge 0$ accordingly, with $\sum_i M_{Q,i}=M_Q$ and Lipschitz variation across neighbors.
For each label $i$, realize the corresponding corner-exit template holomorphically on each vertex star by applying
Proposition~\ref{prop:holomorphic-corner-exit-L1} (with vertex-star coherence as in Remark~\ref{rem:vertex-star-coherence}) to the template planes provided by
Proposition~\ref{prop:corner-exit-template-net}; this yields corner-exit holomorphic slivers with (G1-iff)/(G2) and equal/comparable per-piece masses
(hence Proposition~\ref{prop:vertex-template-mass-matching} applies).

\smallskip\noindent
Then one can choose integer counts $N_{Q,v,i}$ simultaneously for all $(Q,v,i)$ so that:
\begin{enumerate}
\item[\textnormal{(a)}] (\textbf{Local mass/barycenter accuracy}) for each cube $Q$ and label $i$ the realized mass in direction $i$ matches $M_{Q,i}$
up to the rounding error $O(1/N)+O(\varepsilon^2)$ from Proposition~\ref{prop:vertex-template-mass-matching};
\item[\textnormal{(b)}] (\textbf{Slow variation}) for each interior adjacency $Q\sim Q'$ and each shared vertex $v\in Q\cap Q'$, one has
$|N_{Q,v,i}-N_{Q',v,i}|\lesssim h\min\{N_{Q,v,i},N_{Q',v,i}\}$ on the region where $M_{Q,i}$ is not negligible (e.g.\ via Lemma~\ref{lem:slow-variation-rounding}
and the $0$--$1$ stability Lemma~\ref{lem:slow-variation-discrepancy});
\item[\textnormal{(c)}] (\textbf{Cohomology periods}) after clearing denominators by the fixed cohomology multiplier $m$ (as in \S\ref{sec:parameter-schedule})
and applying fixed-dimension discrepancy rounding
(Lemma~\ref{lem:barany-grinberg}), one can choose the integer activations so that the \emph{raw} current has the desired periods up to an error $<\tfrac14$
on a fixed integral cohomology basis; after applying the gluing correction with sufficiently small mass, the resulting \emph{closed} glued cycle has the
exact integral periods and hence the exact class $\mathrm{PD}(m[\gamma])$ in rational homology (Proposition~\ref{prop:cohomology-match}).
\end{enumerate}
Consequently, for each label $i$ the activation hypotheses (iii)–(iv) in Theorem~\ref{thm:sliver-mass-matching-on-template} hold (by Corollary~\ref{cor:corner-exit-iii-iv}),
and summing the resulting per-label flat-norm mismatch bounds yields $\mathcal F(\partial T^{\mathrm{raw}})=o(m)$ under the parameter regime of
Remark~\ref{rem:weighted-scaling}.
\end{proposition}

\begin{editamirblockNEW}
\editamir{\noindent\textbf{Referee closure of the “simultaneous matching” hinge.}
For each interior face $F=Q\cap Q'$, Proposition~\ref{prop:global-coherence-all-labels} furnishes a \emph{common prefix length} $N_F$
(and hence equal integer masses on both sides after the prefix-edit/balancing step).
Fix, at this mesh scale, a transverse grid $\delta_\perp$ and choose a master grid-atom list $(y_a)_{a=1}^{N_*}$ with $N_*\ge \max_F N_F$.
With each resulting $N_F$, Proposition~\ref{prop:integer-transport} \emph{constructs} the integer-weighted measures
$\mu_{Q\to F}$ and $\mu_{Q'\to F}$ and gives the quantitative $W_1$ bound needed in the transport/glue estimate.
Because the construction is facewise and uses the globally coherent master template, the hypothesis (c) in
Proposition~\ref{prop:transport-flat-glue-weighted} is discharged \emph{simultaneously for all interior faces}.}
\end{editamirblockNEW}


\begin{proof}
\begin{editamirblock}
\noindent\textbf{Dependency packaging (no new axioms).}
This statement is a \emph{packaging} of previously proved components; the role of this proof is to
make the dependency chain explicit and eliminate any hidden “assume-as-needed” steps.

\smallskip\noindent
\textbf{(1) Existence of globally labeled weights.}
The Lipschitz weights $w_i(x)$ are produced by Lemma~\ref{lem:lipschitz-qp-weights} applied to the local coefficient vector
$\widehat\beta(x)$ in the chosen trivializations; no additional hypothesis is introduced here.

\smallskip\noindent
\textbf{(2) Local holomorphic realizability for each label.}
For each calibrated direction label $i$, the template planes and their coherence data come from
Proposition~\ref{prop:corner-exit-template-net} and Remark~\ref{rem:vertex-star-coherence}.
Applying Proposition~\ref{prop:holomorphic-corner-exit-L1} in each vertex star yields the corresponding holomorphic corner-exit slivers
with the uniform geometry properties (G1-iff)/(G2) and the per-piece comparability needed to invoke
Proposition~\ref{prop:vertex-template-mass-matching}.
No extra “activation” assumption is made: the hypotheses required by Proposition~\ref{prop:holomorphic-corner-exit-L1}
are exactly those ensured by the template-net construction together with the parameter regime fixed in Remark~\ref{rem:weighted-scaling}.

\smallskip\noindent
\textbf{(3) Integer rounding with slow variation.}
\editamir{Given the per-cell mass budgets $M_{Q,i}$, choose any vertex split $M_{Q,v,i}\ge 0$ with $\sum_{v\in\mathrm{Vert}(Q)}M_{Q,v,i}=M_{Q,i}$ (e.g.\ equal split).}
\editamir{For each $(Q,v,i)$ let $\mu_{Q,v,i}>0$ denote the common per-piece mass scale in $Q$ for the $v$-anchored pieces of label $i$ (as in Proposition~\ref{prop:vertex-template-mass-matching}), and define}
\[
\editamir{N_{Q,v,i}\ :=\ \Bigl\lfloor \frac{M_{Q,v,i}}{\mu_{Q,v,i}}\Bigr\rceil.}
\]
\editamir{Then item \textnormal{(a)} is exactly Proposition~\ref{prop:vertex-template-mass-matching} applied label-by-label.}
\editamir{Item \textnormal{(b)} follows by applying Lemma~\ref{lem:slow-variation-rounding} (and the $0$--$1$ stability Lemma~\ref{lem:slow-variation-discrepancy})
to the Lipschitz targets $(Q,v)\mapsto M_{Q,v,i}/\mu_{Q,v,i}$ at shared vertices, on the region where $M_{Q,i}$ is not negligible.}

\smallskip\noindent
\textbf{(4) Period control and gluing.}
The cohomology/period statement \textnormal{(c)} is obtained by the fixed-dimension discrepancy rounding
Lemma~\ref{lem:barany-grinberg} and the class-identification Proposition~\ref{prop:cohomology-match}.
The passage from the raw current to a closed glued cycle with the exact integral periods uses the gluing correction mechanism
(Proposition~\ref{prop:glue-gap}) with the small-boundary input supplied by the global flat-weighted estimates
(Corollary~\ref{cor:global-flat-weighted} under Remark~\ref{rem:weighted-scaling}).

\smallskip\noindent
\textbf{(5) Activation hypotheses (iii)--(iv) and boundary mismatch.}
Finally, the verification that the sliver/template activation hypotheses \textnormal{(iii)}--\textnormal{(iv)} hold for each label
is exactly Corollary~\ref{cor:corner-exit-iii-iv}, once the local corner-exit realizability from (2) is in place.
Summing the per-label flat-norm mismatch bounds is then a direct application of Corollary~\ref{cor:global-flat-weighted},
yielding $\mathcal F(\partial T^{\mathrm{raw}})=o(m)$ in the parameter regime of Remark~\ref{rem:weighted-scaling}.
\end{editamirblock}
\end{proof}


\begin{editamirblock}
\begin{corollary}[Flat boundary of the raw current in the weighted scaling regime]\label{cor:raw-boundary-flat-small}
\begin{editamirblockNEW}
Assume the hypotheses of Proposition~\ref{prop:global-coherence-all-labels} and work in the parameter regime of
Remark~\ref{rem:weighted-scaling} (in particular, $h\sim c\,N^{-1/2}$ for the holomorphic scale $N\to\infty$ and the corner-exit/graph parameters are synchronized as there).
Then the associated raw current $T^{\mathrm{raw}}$ satisfies
\[
\mathcal F(\partial T^{\mathrm{raw}})=o(m)\qquad\text{as }h\downarrow 0.
\]
In the borderline case $p=n/2$, this conclusion is understood under the additional refined displacement schedule of
Lemma~\ref{lem:borderline-p-half} (e.g.\ $\varrho=o(\varepsilon)$).
Equivalently, for every $\eta>0$ there exists $h_0>0$ such that for all sufficiently small mesh sizes $h<h_0$,
\[
\mathcal F(\partial T^{\mathrm{raw}})\le \eta\,m.
\]
\end{editamirblockNEW}
\begin{editamirblockNEW}
In particular, for fixed cohomology multiplier $m$, any bound of the form $\mathcal{F}(\partial T^{\mathrm{raw}})=o(m)$ yields $\mathcal{F}(\partial T^{\mathrm{raw}})\to 0$ as $h\downarrow 0$.
\end{editamirblockNEW}

\end{corollary}

\begin{proof}
\begin{editamirblockNEW}
By Proposition~\ref{prop:global-coherence-all-labels}, each interior interface mismatch $B_F$ fits the weighted translation model
of Corollary~\ref{cor:global-flat-weighted} with a uniform displacement control $\Delta_F\lesssim h^2$.
Moreover, the face-level $O(h)$ edit regime required there is ensured by the local face-edit estimates
(e.g.\ Proposition~\ref{prop:vertex-template-face-edits} for vertex-prefix activation, or its checkerboard analogue),
so Corollary~\ref{cor:global-flat-weighted} gives
\[
\mathcal F\!\left(\partial T^{\mathrm{raw}}\right)
\ \editamir{\lesssim\ \varrho\,h^2}\sum_Q\ \sum_{a\in\mathcal S(Q)} m_{Q,a}^{\frac{k-1}{k}},
\qquad k:=2n-2p.
\]
Remark~\ref{rem:weighted-scaling} identifies the local packing bound in each cell (coming from the holomorphic corner-exit realization)
and converts the right-hand side into the global scaling estimate
\[
\mathcal F(\partial T^{\mathrm{raw}})
\ \lesssim\ \varrho\;m^{\frac{k-1}{k}}\,h^{\,2-\frac{2n}{k}}\;\varepsilon^{-\frac{2p}{k}}.
\]
Choosing a refinement schedule $h\downarrow 0$ with $\varepsilon=\varepsilon(h)\downarrow 0$ slowly enough (as in Remark~\ref{rem:weighted-scaling})
gives $\mathcal F(\partial T^{\mathrm{raw}})/m\to 0$, i.e.\ $\mathcal F(\partial T^{\mathrm{raw}})=o(m)$.
\end{editamirblockNEW}
\end{proof}

\end{editamirblock}


\begin{editamirblockNEW}
\noindent\textbf{Referee cleanup.} The proof block that followed here was a duplicate of the packaged dependency check above.
It has been removed to prevent two competing “proofs” from coexisting in the manuscript.
\end{editamirblockNEW}

\begin{remark}[Making the ``prefix-balanced face population'' explicit]
The previous proposition treats each vertex template separately.
If one prefers a \emph{single} global ordered template whose prefixes automatically populate every interior face in a balanced way, one can interleave the vertex templates
by a deterministic block scheme (a ``vertex-code'' ordering) and align the vertex anchoring across the grid by a checkerboard parity rule.
This removes the possibility that the $F$-hitting pieces concentrate in a tail of the master order.
See Proposition~\ref{prop:checkerboard-face-oh-edit} below.
\end{remark}

\begin{definition}[Cubical grid parity and checkerboard vertex anchoring]\label{def:checkerboard-anchoring}
Fix $d\ge 2$ and mesh $h>0$ and index cubes by $g\in\Z^d$ via
\(
Q_g:=\prod_{\ell=1}^d[g_\ell h,(g_\ell+1)h].
\)
Define the parity vector $\pi(g)\in\{0,1\}^d$ by $\pi(g)_\ell:=g_\ell\bmod 2$, and let $\oplus$ denote bitwise XOR.
For a vertex-code $u\in\{0,1\}^d$, define the anchored vertex of $Q_g$ by
\[
v_g(u)\ :=\ \bigl(g+(u\oplus\pi(g))\bigr)h\ \in\ \R^d,
\]
so $u$ selects a cube-vertex in a checkerboard-consistent way across neighbors.
\end{definition}

\begin{definition}[Block-uniform vertex-code sequence]\label{def:block-uniform-codes}
Let $\mathcal V:=\{0,1\}^d$ and fix any bijection $\sigma:\{1,\dots,2^d\}\to\mathcal V$.
Define an infinite sequence $(u_a)_{a\ge 1}\subset\mathcal V$ by repeating $\sigma$ in blocks:
\[
u_{b\cdot 2^d+r}\ :=\ \sigma(r)\qquad (b\ge 0,\ 1\le r\le 2^d).
\]
\end{definition}

\begin{lemma}[Prefix discrepancy for block-uniform codes]\label{lem:prefix-discrepancy}
Let $S\subset\mathcal V$ and define
\(
A_S(N):=\#\{1\le a\le N:\ u_a\in S\}.
\)
Then for all $N\ge 1$,
\[
\Bigl|A_S(N) - \frac{|S|}{2^d}\,N\Bigr|\ \le\ 2^{d+1},
\]
and for all $N,N'\ge 1$,
\[
|A_S(N)-A_S(N')|\ \le\ \frac{|S|}{2^d}\,|N-N'| + 2^{d+1}.
\]
\end{lemma}
\begin{editamirblockNEW}
\begin{proof}
Write $N=q\,2^d+r$ with $q\in\Z_{\ge 0}$ and $0\le r<2^d$.  By Definition~\ref{def:block-uniform-codes}, each complete block of
length $2^d$ contains each code in $\mathcal V$ exactly once, so each complete block contributes exactly $|S|$ hits.  Hence
\[
A_S(N)\ =\ q\,|S| + A_S(r),
\]
where $A_S(r):=\#\{1\le a\le r:\ u_a\in S\}$ counts hits in the initial segment of one block.  In particular
$0\le A_S(r)\le \min\{r,|S|\}\le 2^d$.
Since $\frac{|S|}{2^d}N=q|S|+\frac{|S|}{2^d}r$, we obtain
\[
\Bigl|A_S(N)-\frac{|S|}{2^d}N\Bigr|
\ =\ \Bigl|A_S(r)-\frac{|S|}{2^d}r\Bigr|
\ \le\ A_S(r)+\frac{|S|}{2^d}r
\ \le\ 2^d+|S|
\ \le\ 2^{d+1}.
\]
For the Lipschitz bound, write
\[
A_S(N)-A_S(N')\ =\ \Bigl(A_S(N)-\frac{|S|}{2^d}N\Bigr) - \Bigl(A_S(N')-\frac{|S|}{2^d}N'\Bigr)
\ +\ \frac{|S|}{2^d}(N-N'),
\]
and use the previous estimate for each bracketed term to get
\[
|A_S(N)-A_S(N')|
\ \le\ \frac{|S|}{2^d}|N-N'| + 2^{d+1}.
\]
\end{proof}
\end{editamirblockNEW}

\begin{lemma}[Two-sided face population is automatic under checkerboarding]\label{lem:two-sided-face-pop}
Fix a coordinate direction $\ell\in\{1,\dots,d\}$ and an interior interface face $F:=Q_g\cap Q_{g+e_\ell}$.
Let $S_{g,\ell}^+\subset\mathcal V$ be the set of codes whose anchored vertex in $Q_g$ lies on the positive $\ell$-face of $Q_g$, and let
$S_{g+e_\ell,\ell}^-\subset\mathcal V$ be the set of codes whose anchored vertex in $Q_{g+e_\ell}$ lies on the negative $\ell$-face of $Q_{g+e_\ell}$ (the same hyperplane).
Then $S_{g,\ell}^+=S_{g+e_\ell,\ell}^-$ and hence, for every $N$,
\[
\{a\le N:\ v_g(u_a)\in F\}\ =\ \{a\le N:\ v_{g+e_\ell}(u_a)\in F\}.
\]
\end{lemma}
\begin{editamirblockNEW}
\begin{proof}
By Definition~\ref{def:checkerboard-anchoring}, the anchored vertex $v_g(u)$ lies on the \emph{positive} $\ell$-face of $Q_g$
if and only if the $\ell$-th coordinate of $g+(u\oplus\pi(g))$ equals $g_\ell+1$, i.e.
\[
(u\oplus\pi(g))_\ell=1.
\]
Since $\pi(g+e_\ell)=\pi(g)\oplus e_\ell$, we have
\[
(u\oplus\pi(g+e_\ell))_\ell \ =\ (u\oplus\pi(g)\oplus e_\ell)_\ell \ =\ (u\oplus\pi(g))_\ell\oplus 1,
\]
so $(u\oplus\pi(g))_\ell=1$ if and only if $(u\oplus\pi(g+e_\ell))_\ell=0$.
But $(u\oplus\pi(g+e_\ell))_\ell=0$ is exactly the condition that $v_{g+e_\ell}(u)$ lies on the \emph{negative} $\ell$-face
of $Q_{g+e_\ell}$, which is the same hyperplane $F=Q_g\cap Q_{g+e_\ell}$.  Therefore $S_{g,\ell}^+=S_{g+e_\ell,\ell}^-$.
The equality of the index-sets for every $N$ follows immediately from the definition of $A_S(N)$.
\end{proof}
\end{editamirblockNEW}

\begin{editamirblockNEW}
\begin{proposition}[Checkerboard corner assignment implies a face-level $O(h)$ edit regime]\label{prop:checkerboard-face-oh-edit}
Fix $d\ge 2$ and a cubical grid $(Q_g)$ of mesh $h>0$.
Assume the ordered sliver activation in each cube $Q_g$ uses the block-uniform code sequence $(u_a)_{a\ge 1}$
(Definition~\ref{def:block-uniform-codes}), anchored by the checkerboard vertex rule $a\mapsto v_g(u_a)$
(Definition~\ref{def:checkerboard-anchoring}).
Assume the following geometric features hold uniformly for the activated slivers in each cube:
\begin{enumerate}
\item[\textnormal{(G1)}] (\textbf{Locality}) For an interior face $F=Q_g\cap Q_{g+e_\ell}$ and an index $a$,
the boundary slice $\partial([Y_g^a]\llcorner Q_g)\llcorner F$ is nonzero if and only if $v_g(u_a)\in F$,
and in that case it is supported in a patch of diameter $\lesssim h$ near $v_g(u_a)$.
\item[\textnormal{(G2)}] (\textbf{Comparable face mass}) There exist constants $0<c_0\le C_0$ and face-scale parameters $b_g\ge 0$
such that for every interior face $F$ and every $a$ with $v_g(u_a)\in F$,
\[
c_0\,b_g\ \le\ \Mass\!\bigl(\partial([Y_g^a]\llcorner Q_g)\llcorner F\bigr)\ \le\ C_0\,b_g.
\]
\end{enumerate}
Let $F=Q_g\cap Q_{g+e_\ell}$ be an interior face and let $N:=N_g$, $N':=N_{g+e_\ell}$ be the chosen prefix lengths on the two sides,
with $N_{\min}:=\min\{N,N'\}$.
Assume $N_{\min}\ge 2^{d+3}$ (in particular this holds in the regime $N_{\min}\gtrsim h^{-1}$ for $h\ll 1$).
Then the unmatched boundary mass on $F$ coming from tail indices $\{N_{\min}+1,\dots,\max\{N,N'\}\}$ satisfies
\[
\Mass(B_F^{\mathrm{un}})\ \le\ C\left(\frac{|N-N'|}{N_{\min}}+\frac{2^d}{N_{\min}}\right)
\Bigl(\Mass(\partial S_{Q_g}\llcorner F)+\Mass(\partial S_{Q_{g+e_\ell}}\llcorner F)\Bigr),
\]
with $C$ depending only on $(d,c_0,C_0)$.
In particular, if $|N-N'|\le \theta\,N_{\min}$ with $\theta\lesssim h$ and $N_{\min}\gtrsim h^{-1}$, then
\[
\Mass(B_F^{\mathrm{un}})\ \le\ C'\,h\,
\Bigl(\Mass(\partial S_{Q_g}\llcorner F)+\Mass(\partial S_{Q_{g+e_\ell}}\llcorner F)\Bigr),
\]
so the $O(h)$ face-edit regime (item \textnormal{(iv)} in Theorem~\ref{thm:sliver-mass-matching-on-template}) holds.
\end{proposition}

\begin{proof}
Let $S\subset\mathcal V=\{0,1\}^d$ be the set of codes whose anchored vertex in $Q_g$ lies on the interface face
$F=Q_g\cap Q_{g+e_\ell}$.  Then $|S|=2^{d-1}$.
By Lemma~\ref{lem:two-sided-face-pop}, the set of indices $\{a\le N:\ v_g(u_a)\in F\}$ agrees with
$\{a\le N:\ v_{g+e_\ell}(u_a)\in F\}$ for every $N$, so the only unmatched boundary contributions on $F$ come from those
tail indices $a\in(N_{\min},N_{\max}]$ with $u_a\in S$, where $N_{\max}:=\max\{N,N'\}$.

\smallskip\noindent\emph{Counting unmatched tail indices.}
By Lemma~\ref{lem:prefix-discrepancy} applied to the set $S$,
\[
\#\{N_{\min}<a\le N_{\max}:\ u_a\in S\}
\ =\ |A_S(N_{\max})-A_S(N_{\min})|
\ \le\ \frac{|S|}{2^d}|N-N'|+2^{d+1}
\ =\ \frac12|N-N'|+2^{d+1}.
\]
Each such unmatched index contributes at most $C_0\,b_g$ (or $C_0\,b_{g+e_\ell}$) to the boundary mass on the side where it appears,
by (G2).  Hence
\[
\Mass(B_F^{\mathrm{un}})
\ \le\ C_0\Bigl(\tfrac12|N-N'|+2^{d+1}\Bigr)\,(b_g+b_{g+e_\ell}).
\]

\smallskip\noindent\emph{Lower bound for the total activated boundary mass on $F$.}
Again by Lemma~\ref{lem:prefix-discrepancy},
\[
A_S(N_{\min})\ \ge\ \frac{|S|}{2^d}N_{\min}-2^{d+1}\ =\ \frac12 N_{\min}-2^{d+1}.
\]
Since $N_{\min}\ge 2^{d+3}$, the right-hand side is $\ge \frac14 N_{\min}$.
Each of these $A_S(N_{\min})$ indices appears on \emph{both} sides of $F$, and by (G2) contributes at least $c_0 b_g$ and
at least $c_0 b_{g+e_\ell}$ to $\Mass(\partial S_{Q_g}\llcorner F)$ and $\Mass(\partial S_{Q_{g+e_\ell}}\llcorner F)$ respectively.
Therefore,
\[
\Mass(\partial S_{Q_g}\llcorner F)+\Mass(\partial S_{Q_{g+e_\ell}}\llcorner F)
\ \ge\ c_0\,A_S(N_{\min})\,(b_g+b_{g+e_\ell})
\ \ge\ \frac{c_0}{4}\,N_{\min}\,(b_g+b_{g+e_\ell}).
\]

\smallskip\noindent\emph{Conclusion.}
Combining the previous two displays yields
\[
\Mass(B_F^{\mathrm{un}})
\ \le\ C\left(\frac{|N-N'|}{N_{\min}}+\frac{2^d}{N_{\min}}\right)
\Bigl(\Mass(\partial S_{Q_g}\llcorner F)+\Mass(\partial S_{Q_{g+e_\ell}}\llcorner F)\Bigr),
\]
with $C$ depending only on $(d,c_0,C_0)$ (absorbing fixed powers of $2$ into the constant), as claimed.
The final $O(h)$ specialization follows immediately under $|N-N'|\le \theta N_{\min}$ with $\theta\lesssim h$ and $N_{\min}\gtrsim h^{-1}$.
\end{proof}
\end{editamirblockNEW}


\begin{remark}[Rounded cubes]\label{rem:smooth-cells}
For the combinatorics of Substep~4.2 (adjacency graph, faces, cochain constraints), it is convenient to work with a cubulation.
For the sliver bookkeeping, it is convenient to replace each sharp cube by a \emph{rounded cube} of comparable diameter $h$ whose boundary is $C^2$
and uniformly convex with principal curvatures pinched at scale $h$ (so Lemma~\ref{lem:uniformly-convex-slice-boundary} applies).
This rounding changes only constants and does not change the adjacency graph.
\end{remark}

\begin{remark}[Where the remaining analytic difficulty really lives]\label{rem:bergman-not-enough}
It is tempting to argue that Bergman kernel localization or Tian--Yau--Zelditch universality alone forces the desired
face-incidence and per-face boundary-mass properties of slivers.  However, \emph{pointwise decay of a holomorphic section does not
localize its exact zero set} in the strong sense needed for gluing.

\smallskip\noindent
The correct ``critical checkpoint'' is instead the following: on a \emph{whole cell} $Q$ (not just infinitesimally near one point),
the defining holomorphic map must be \emph{uniformly $C^1$-close} to a fixed linear model so that the zero set in $Q$ is a \emph{single sheet}
graph over the intended template plane.  Once this global-graph property holds, the corner-exit geometry immediately forces
(G1-iff) and (G2) (exit-face stability and per-face mass comparability), and the remaining face bookkeeping is purely combinatorial.
\end{remark}

\begin{lemma}[Global quantitative graph lemma (contraction criterion)]\label{lem:global-graph-contraction}
Let $U=U_u\times U_w\subset \R^{k}\times \R^{d-k}$ be a product of convex sets and fix $r>0$ with $B_w(0,r)\subset U_w$.
Let $F:U\to \R^{d-k}$ be $C^{1}$ and fix an invertible matrix $A\in GL(d-k,\R)$.
Assume:
\begin{enumerate}
\item[\textnormal{(i)}] (\textbf{Uniform linearization in the $w$-directions})
\[
\sup_{(u,w)\in U}\,\|\partial_w F(u,w)-A\|\ \le\ \eta,
\qquad
\|A^{-1}\|\,\eta\ \le\ \frac12;
\]
\item[\textnormal{(ii)}] (\textbf{Small offset on the $w=0$ slice})
\[
\sup_{u\in U_u}\,\|A^{-1}F(u,0)\|\ \le\ \frac{r}{2}.
\]
\end{enumerate}
Then for every $u\in U_u$ there exists a \emph{unique} $w=g(u)\in B_w(0,r)$ such that $F(u,g(u))=0$.
Hence $\{F=0\}\cap (U_u\times B_w(0,r))$ is the graph of $g$.

\smallskip\noindent
If in addition $\sup_{(u,w)\in U}\|\partial_u F(u,w)\|\le \eta$, then $g$ is Lipschitz and, wherever differentiable,
\[
\|Dg\|\ \le\ \frac{\|A^{-1}\|\,\eta}{1-\|A^{-1}\|\,\eta}\ \le\ 2\,\|A^{-1}\|\,\eta.
\]
\begin{editamirblockNEW}
\editamir{In particular, since $F$ is $C^1$ and $\partial_wF(u,g(u))$ is invertible for all $u\in U_u$, the implicit function theorem implies $g\in C^1(U_u)$; hence the displayed bound holds for every $u\in U_u$.}
\end{editamirblockNEW}

\end{lemma}
\begin{proof}
Fix $u\in U_u$ and define $T_u:B_w(0,r)\to \R^{d-k}$ by
\[
T_u(w)\ :=\ w - A^{-1}F(u,w).
\]
Write
\[
T_u(w)= -A^{-1}F(u,0)\ +\ \Bigl[w - A^{-1}(F(u,w)-F(u,0))\Bigr].
\]
By the mean value theorem in the $w$-variable,
\[
F(u,w)-F(u,0)=\Bigl(\int_0^1 \partial_w F(u,tw)\,dt\Bigr)\,w,
\]
hence
\[
w - A^{-1}(F(u,w)-F(u,0))
=\Bigl(I - A^{-1}\int_0^1 \partial_wF(u,tw)\,dt\Bigr)\,w.
\]
Using $\|\partial_wF-A\|\le \eta$ and $\|A^{-1}\|\eta\le \tfrac12$ gives
\[
\Bigl\|I - A^{-1}\int_0^1 \partial_wF(u,tw)\,dt\Bigr\|\ \le\ \|A^{-1}\|\,\eta\ \le\ \frac12,
\]
so for $w\in B_w(0,r)$,
\[
\|T_u(w)\|\ \le\ \|A^{-1}F(u,0)\| + \frac12\|w\|\ \le\ \frac r2 + \frac12 r = r.
\]
Thus $T_u$ maps $B_w(0,r)$ into itself.

Similarly, for $w,w'\in B_w(0,r)$, the mean value theorem yields
\[
T_u(w)-T_u(w')
=\Bigl(I - A^{-1}\int_0^1 \partial_wF(u,w'+t(w-w'))\,dt\Bigr)\,(w-w'),
\]
so $\|T_u(w)-T_u(w')\|\le \tfrac12\|w-w'\|$.  Hence $T_u$ is a contraction, and Banach's fixed point theorem
gives a unique fixed point $g(u)\in B_w(0,r)$ with $T_u(g(u))=g(u)$, i.e.\ $F(u,g(u))=0$.

For the slope bound, differentiate $F(u,g(u))=0$ where $g$ is differentiable:
\[
\partial_uF(u,g(u)) + (\partial_wF(u,g(u)))\,Dg(u)=0,
\qquad\text{so}\qquad
Dg(u)=-(\partial_wF)^{-1}\partial_uF.
\]
Since $\|\partial_wF-A\|\le \eta$ and $\|A^{-1}\|\eta\le \tfrac12$, Neumann series gives
$\|(\partial_wF)^{-1}\|\le \|A^{-1}\|/(1-\|A^{-1}\|\eta)$, yielding the stated estimate.

\begin{editamirblockNEW}
\editamir{Finally, since $F$ is $C^1$ and $\partial_wF(u,g(u))$ is invertible, the implicit function theorem upgrades $g$ to a $C^1$ map on $U_u$, so the derivative identity and bound hold for all $u\in U_u$.}
\end{editamirblockNEW}

\end{proof}

\begin{remark}[Memorializing the new checkpoint: ``graph on the whole cell'']\label{rem:graph-whole-cell}
With the corner-exit Euclidean templates and the small-slope stability package in hand, the remaining microstructure/gluing
difficulty becomes sharply focused.

\smallskip\noindent
\begin{editamirblockNEW}
\textbf{Blocker A (cell-scale single-sheet control --- \emph{resolved}).}
This checkpoint is now achieved by Proposition~\ref{prop:cell-scale-linear-model-graph}, which builds holomorphic complete intersections
whose local defining map $F(u,w)$ is a small perturbation of the invertible linear model in the $w$-variables and hence yields a \emph{unique}
$C^1$ graph $w=g(u)$ on all of $Q$ by Lemma~\ref{lem:global-graph-contraction}.
In particular, each holomorphic sliver in a cell is a \emph{single sheet} over its template plane on a region containing $Q$ (with slope as small as desired).

\smallskip\noindent
\textbf{Blocker B (per-sliver mass control / no heavy tails --- \emph{resolved}).}
Once the single-sheet small-slope graph property holds on $Q$, mass and face-slice masses are quantitatively controlled by area distortion:
Lemma~\ref{lem:sliver-stability} gives $\Mass([Y]\llcorner Q)=(1+O(\varepsilon^2))\Mass([P]\llcorner Q)$ for the underlying template plane $P$,
and Proposition~\ref{prop:holomorphic-corner-exit-g1g2} (hence Proposition~\ref{prop:holomorphic-corner-exit-L1}) controls the boundary-face contributions.
Therefore there are no ``heavy tails'' at cell scale, and the remaining mass-budget matching (L2) reduces to the discrete prefix-length bookkeeping
(with $O(1/N)+O(\varepsilon^2)$ rounding error) in the template-matching stage.
\end{editamirblockNEW}
\smallskip\noindent
\textbf{How to apply Lemma~\ref{lem:global-graph-contraction} to holomorphic complete intersections.}
In a holomorphic chart, write the local coefficients of the defining sections as a map
$F=(f_1,\dots,f_p):U\to \C^p\cong\R^{2p}$.  Choose real coordinates $(u,w)\in\R^{k}\times\R^{2p}$ so that the template
plane is $\{w=0\}$ and the linear model is $w\mapsto Aw$ with $A$ invertible.
If one can construct the sections so that, on a ball containing $Q$,
\[
\|\partial_wF-A\|_{L^\infty}\le \eta,\qquad \|\partial_uF\|_{L^\infty}\le \eta,\qquad
\|F(\cdot,0)\|_{L^\infty(U_u)}\le \eta\,h,
\]
with $\|A^{-1}\|\eta\ll 1$, then Lemma~\ref{lem:global-graph-contraction} gives a global graph $w=g(u)$ on all of $Q$.
This is exactly the ``graph on the whole cell'' checkpoint highlighted in the microstructure roadmap.

\smallskip\noindent
\textbf{Two standard routes to produce the needed uniform $C^1$ control} are:
\begin{itemize}
\item peak sections plus $\bar\partial$-solving (H\"ormander $L^2$ estimates) to approximate prescribed affine-linear holomorphic models on
Bergman-scale balls, and
\item Bergman kernel asymptotics / jet right-inverses (Tian--Catlin--Zelditch--Donaldson) to achieve the same $C^1$ control directly.
\end{itemize}
\end{remark}

\begin{lemma}[Bergman-scale affine model approximation via $\bar\partial$-solving]\label{lem:bergman-affine-approx-hormander}
Fix a holomorphic chart $\varphi:U\to B_{\rho}(0)\subset\C^n$ and a local holomorphic frame $e$ of $L$ over $U$ with
$|e|_h^2=e^{-\phi}$ and $i\partial\bar\partial\phi=\omega$ on $U$.
Fix $R>0$ and let $\ell(z)=a\cdot z+b$ be an affine-linear holomorphic function on $\C^n$ with $|a|+|b|\le 1$.
Then for all sufficiently large $N$ there exists a global section $s_{\ell,N}\in H^0(X,L^N)$ such that, writing
$s_{\ell,N}=f_{\ell,N}\,e^{\otimes N}$ on $B_{\rho/8}(0)$, one has on the Bergman-scale ball
$B_{R/\sqrt N}(0)\subset B_{\rho/8}(0)$:
\[
\sup_{|z|\le R/\sqrt N}\Bigl(|f_{\ell,N}(z)-\ell(z)|+\sqrt N\,|\nabla(f_{\ell,N}-\ell)(z)|\Bigr)\ \le\ \varepsilon_N,
\qquad \varepsilon_N\xrightarrow[N\to\infty]{}0,
\]
with constants uniform over the finitely many charts in a fixed atlas on $X$.
\editamirNEW{Moreover, the construction in the proof yields the quantitative bound $\varepsilon_N\le C_R\,e^{-c N}$ for some $c>0$ (uniform over $\ell$ with $|a|+|b|\le 1$ and over charts in a fixed finite atlas). In particular, $\varepsilon_N=o(N^{-1/2})$.}
\end{lemma}

\begin{proof}
Choose a cutoff $\chi$ supported in $B_{\rho/2}(0)$ with $\chi\equiv 1$ on $B_{\rho/4}(0)$ and set
$\tilde s:=\chi\,\ell\,e^{\otimes N}$ (extended by $0$ outside $U$).
Then $\bar\partial\tilde s=(\bar\partial\chi)\,\ell\,e^{\otimes N}$ is supported in the annulus
$\{\rho/4\le |z|\le \rho/2\}$ \editamir{where, after scaling the local frame by a constant (equivalently adding a constant to $\phi$), we may assume $\inf_{\rho/4\le |z|\le \rho/2}\phi\ge c_0>0$.}
Solve $\bar\partial u=\bar\partial\tilde s$ using H\"ormander $L^2$ estimates \editamirNEW{(see \cite{Demailly12,MaMarinescu07})} for the positive bundle $(L^N,h^N)$; the weight $e^{-N\phi}$
forces $\|u\|_{L^2(h^N)}\le C\,e^{-c N}$.
On the inner ball $B_{\rho/4}(0)$ one has $\bar\partial u=0$, so $u$ is holomorphic there.
Standard local $L^2\to C^1$ estimates for holomorphic sections on Bergman balls (mean-value inequality plus Cauchy estimates at scale $N^{-1/2}$)
give $\|u\|_{C^1(B_{R/\sqrt N})}\le C_R\,e^{-c N}$.
Setting $s_{\ell,N}:=\tilde s-u$ yields $s_{\ell,N}$ holomorphic and
$f_{\ell,N}=\ell-\text{(holomorphic error)}$ on $B_{R/\sqrt N}$ with the stated bound.
\end{proof}


\begin{proposition}[\editamir{Cell-scale linear-model complete intersections are single-sheet graphs}]\label{prop:cell-scale-linear-model-graph}
\iffalse
Fix a holomorphic chart identifying a neighborhood of a cell $Q$ with a domain in $\C^{n}=\C^{n-p}\times\C^{p}$ with coordinates $z=(u,w)$,
and assume $Q\subset B_{R/\sqrt N}(0)$ for some fixed $R$.
\fi
\editamir{Fix a holomorphic chart identifying a neighborhood of a cell $Q$ with a domain in $\C^{n}=\C^{n-p}\times\C^{p}$ with coordinates $z=(u,w)$, and assume $Q\subset B_{R/\sqrt N}(0)$ for some fixed $R$. Assume moreover that the cell diameter satisfies $h\asymp N^{-1/2}$.}
Let $t\in\C^p$ satisfy $|t|\le c\,h$ (with \editamir{$h\asymp N^{-1/2}$}).
Then for all sufficiently large $N$ there exist sections $\sigma_1,\dots,\sigma_p\in H^0(X,L^N)$ such that, writing
$\sigma_j=F_j\,e^{\otimes N}$ in a local frame on $B_{R/\sqrt N}(0)$ and setting $F=(F_1,\dots,F_p)$, one has
\[
\|\partial_w F-I\|_{L^\infty(B_{R/\sqrt N})}\ +\ \|\partial_u F\|_{L^\infty(B_{R/\sqrt N})}\ \le\ \eta_N,
\qquad
\sup_{u:\ (u,t)\in B_{R/\sqrt N}} |F(u,t)|\ \le\ \eta_N\,h,
\]
with $\eta_N\to 0$.
Consequently, for $N$ large enough, the common zero set
$Y_t:=\{\sigma_1=\cdots=\sigma_p=0\}$ satisfies that $Y_t\cap Q$ is a \emph{single} $C^1$ graph over the affine complex plane
$\{w=t\}$ on all of $Q$, with slope $O(\eta_N)$ (hence as small as desired).
\end{proposition}

\begin{proof}
\begin{editamirblockNEW}
Apply Lemma~\ref{lem:bergman-affine-approx-hormander} to the affine-linear holomorphic functions
$\ell_0\equiv 1$ and $\ell_j(z)=w_j$ ($1\le j\le p$).  Thus for each $j=0,1,\dots,p$ we obtain a holomorphic section
$s_j\in H^0(X,L^N)$ with local coefficient $f_j$ on $B_{R/\sqrt N}$ such that
\[
\sup_{B_{R/\sqrt N}}\Bigl(|f_0-1|+\sqrt N\,|\nabla(f_0-1)|\Bigr)\le \varepsilon_N,
\qquad
\sup_{B_{R/\sqrt N}}\Bigl(|f_j-w_j|+\sqrt N\,|\nabla(f_j-w_j)|\Bigr)\le \varepsilon_N\ (1\le j\le p),
\]
for some $\varepsilon_N\to 0$ (as in Lemma~\ref{lem:bergman-affine-approx-hormander}).

For $t=(t_1,\dots,t_p)$ define $\sigma_j:=s_j-t_j s_0$ and write
$\sigma_j=F_j\cdot e^{\otimes N}$ in the chosen local frame, so
\[
F_j(u,w)=f_j(u,w)-t_j f_0(u,w).
\]
Since $\nabla(w_j)=e_j$ and $\nabla(1)=0$ in the Euclidean chart, the above estimates imply
\[
\|\partial_w F-I\|_{L^\infty(B_{R/\sqrt N})}+\|\partial_u F\|_{L^\infty(B_{R/\sqrt N})}
\ \le\ C\,\frac{\varepsilon_N}{\sqrt N},
\]
and at $w=t$ we have
\[
F_j(u,t)=(f_j-w_j)(u,t)-t_j(f_0-1)(u,t),
\qquad\text{hence}\qquad
\sup_{u:(u,t)\in B_{R/\sqrt N}}|F(u,t)|\le C\,\varepsilon_N .
\]
Set $\eta_N:=C\,\varepsilon_N/h$, where $h=\mathrm{diam}(Q)\asymp N^{-1/2}$ and (by Lemma~\ref{lem:bergman-affine-approx-hormander}) $\varepsilon_N=o(N^{-1/2})$; hence $\eta_N\to 0$. Moreover, for $N$ large one has $C\,\varepsilon_N/\sqrt N\le \eta_N$, so
\[
\|\partial_w F-I\|_{L^\infty}+\|\partial_u F\|_{L^\infty}\le \eta_N,
\qquad
\sup_{u:(u,t)\in B_{R/\sqrt N}}|F(u,t)|\le \eta_N\,h .
\]

Introduce the translated variable $\widetilde w:=w-t$ and the translated map
\[
\widetilde F(u,\widetilde w):=F(u,\widetilde w+t).
\]
Then $\partial_{\widetilde w}\widetilde F=\partial_w F$ and $\partial_u\widetilde F=\partial_u F$, and
$\sup_{u}|\widetilde F(u,0)|=\sup_{u}|F(u,t)|\le \eta_N h$.

Choose $r\simeq h$ and a product set $U_u\times U_{\widetilde w}\subset B_{R/\sqrt N}$
with $Q\subset U_u\times (t+U_{\widetilde w})$ and $U_{\widetilde w}\subset B_{\widetilde w}(0,r)$.
For $N$ large we have $\eta_N\ll 1$ and $\eta_N h\le r/2$, so Lemma~\ref{lem:global-graph-contraction}
applies to $\widetilde F$ on $U_u\times U_{\widetilde w}$ with $A=I$.
It produces a unique $C^1$ graph $\widetilde w=g(u)$ solving $\widetilde F(u,\widetilde w)=0$ on $U_u$,
hence $w=t+g(u)$ solves $F(u,w)=0$.
Therefore,
\[
Y_t\cap Q=\{(u,w)\in Q:\sigma_1=\cdots=\sigma_p=0\}
\]
is a single $C^1$ graph over the affine plane $\{w=t\}$ on all of $Q$,
and the slope estimate follows from Lemma~\ref{lem:global-graph-contraction}:
$\|Dg\|_{L^\infty}\le 2\eta_N$.
\end{editamirblockNEW}
\end{proof}




\begin{lemma}[Vertex-ball locality excludes nonincident faces]\label{lem:ball-excludes-faces}
Let $Q=[0,h]^d\subset\R^d$ and let $v$ be a vertex of $Q$.
Let $F\subset\partial Q$ be any codimension-$1$ face. If $v\notin F$, then $\dist(v,F)=h$.
Consequently, if $E\subset Q$ satisfies
\[
E\subset B(v,c_0h)\qquad\text{for some }0<c_0<1,
\]
then $E\cap F=\emptyset$ for every face $F$ not containing $v$.
\end{lemma}
\begin{proof}
After translation we may assume $v=0$.
Every codimension-$1$ face of $Q$ is of the form $\{x_j=0\}$ or $\{x_j=h\}$.
If $0\notin F$, then $F=\{x_j=h\}$ for some $j$, hence $\dist(0,F)=h$.
If $E\subset B(0,c_0h)$ with $c_0<1$, then $E$ cannot intersect any set at distance $h$ from $0$.
\end{proof}

\begin{lemma}[{\color{blue}Fat corner simplices force ``if'' on the designated exit faces}]\label{lem:corner-simplex-hits-designated-faces}
\begin{editamirblockNEW}
Fix $d\ge 2$ and $1\le k<d$.  Let $Q=[0,h]^d$ and let $v$ be a vertex.
Assume $E\subset Q$ is a $k$--simplex satisfying:
\begin{enumerate}
\item[\textup{(C1)}]\label{C1cornerfaces}
There exist \emph{distinct} codimension--$1$ faces $F_0,\dots,F_k$ of $Q$ incident to $v$ such that, for each $i$, the intersection $E\cap F_i$ is a $(k-1)$--dimensional \emph{facet} of $E$ (equivalently, $E\cap F_i$ has nonempty relative interior inside the affine hyperplane $F_i$).
\item[\textup{(C2)}]\label{C2cornerboundary}
The boundary footprint meets no other codimension--$1$ faces:
\[
E\cap \partial Q \subset \bigcup_{i=0}^k F_i.
\]
\item[\textup{(C3)}]\label{C3cornerlocal}
$E$ is localized near $v$, i.e.\ $E\subset B(v,c_0h)$ for some $0<c_0<1$.
\end{enumerate}
Then for any codimension--$1$ face $F$ of $Q$,
\[
\mathcal H^{k-1}(E\cap F)>0
\quad\Longleftrightarrow\quad
F\in\{F_0,\dots,F_k\}.
\]
Moreover, if $F$ is not incident to $v$, then $E\cap F=\emptyset$.
\end{editamirblockNEW}
\end{lemma}

\begin{proof}
\begin{editamirblockNEW}
For each $i$, since $E\cap F_i$ is a facet of the $k$--simplex $E$, it contains a relatively open subset of the $(k-1)$--dimensional affine hyperplane $F_i$. Hence $\mathcal H^{k-1}(E\cap F_i)>0$.

Let $F$ be a codimension--$1$ face of $Q$ \emph{not} incident to $v$.  Using \eqref{C3cornerlocal} with $c_0<1$, Lemma~\ref{lem:ball-excludes-faces} implies $E\cap F=\emptyset$.

Now let $F$ be incident to $v$ but $F\notin\{F_0,\dots,F_k\}$.  By \eqref{C2cornerboundary},
\[
E\cap F \subset E\cap\partial Q \subset \bigcup_{i=0}^k F_i,
\]
so $E\cap F\subset\bigcup_{i=0}^k (E\cap F\cap F_i)$.  For each $i$, since $F\neq F_i$ the intersection $F\cap F_i$ is contained in a codimension--$2$ face of $Q$, and $E\cap F_i$ has nonempty relative interior in $F_i$; therefore $E\cap F\cap F_i$ is contained in a proper boundary piece of the facet $E\cap F_i$ and has Hausdorff dimension at most $k-2$.  In particular $\mathcal H^{k-1}(E\cap F\cap F_i)=0$ for all $i$, hence $\mathcal H^{k-1}(E\cap F)=0$.

This proves the claimed ``if and only if'' statement.
\end{editamirblockNEW}
\end{proof}


\begin{lemma}[{\color{blue}Uniform per--face boundary mass for fat corner simplices}]\label{lem:corner-simplex-face-mass}
\begin{editamirblockNEW}
Fix $d\ge 2$, $1\le k<d$, and a fatness parameter $\Lambda\ge 1$.
Let $E\subset\R^d$ be a $k$--simplex and write $v_E:=\mathcal H^k(E)$.
\editamir{Suppose that} $E$ is \emph{$\Lambda$--fat} in the following quantitative sense: if $\Pi:=\mathrm{aff}(E)$ is the affine span of $E$, then there exists an affine isomorphism $A:\Pi\to\R^k$ such that
\[
\|DA\|\le \Lambda,\qquad \|(DA)^{-1}\|\le \Lambda,
\]
and $A(E)=\Delta_s$, the standard $k$--simplex of scale $s>0$.
Let $\sigma_0,\dots,\sigma_k$ denote the $(k-1)$--dimensional facets of $E$, and set $a_i:=\mathcal H^{k-1}(\sigma_i)$.
Then there exist constants $0<c_\star(k,\Lambda)\le C_\star(k,\Lambda)$ such that for every $i=0,\dots,k$,
\[
c_\star\, v_E^{(k-1)/k}\ \le\ a_i\ \le\ C_\star\, v_E^{(k-1)/k}.
\]
\end{editamirblockNEW}
\end{lemma}

\begin{proof}
\begin{editamirblockNEW}
Because $A$ is affine on $\Pi$, both the $k$--Jacobian and the $(k-1)$--Jacobian of $A$ are constant on $\Pi$ and are controlled by the operator--norm bounds:
\[
\Lambda^{-k}\ \lesssim_k\ J_k(A)\ \lesssim_k\ \Lambda^{k},
\qquad
\Lambda^{-(k-1)}\ \lesssim_k\ J_{k-1}(A)\ \lesssim_k\ \Lambda^{k-1},
\]
and the same holds for $A^{-1}$ (here $\lesssim_k$ hides constants depending only on $k$).
Consequently,
\[
v_E=\mathcal H^k(E)\ \simeq_{k,\Lambda}\ \mathcal H^k(\Delta_s),
\qquad
a_i=\mathcal H^{k-1}(\sigma_i)\ \simeq_{k,\Lambda}\ \mathcal H^{k-1}(\partial\Delta_s),
\]
with comparability constants depending only on $(k,\Lambda)$.

In the standard simplex $\Delta_s$ the $k$--volume scales like $s^k$ and each facet area scales like $s^{k-1}$, i.e.
\[
\mathcal H^k(\Delta_s)=c_k\, s^k,
\qquad
\mathcal H^{k-1}(\text{any facet of }\Delta_s)=c_{k-1}\, s^{k-1},
\]
for explicit dimensional constants $c_k,c_{k-1}>0$.  Eliminating $s$ gives
$\mathcal H^{k-1}(\text{facet})\simeq_k \mathcal H^k(\Delta_s)^{(k-1)/k}$.
Combining with the previous comparability under $A$ yields
\[
a_i\ \simeq_{k,\Lambda}\ v_E^{(k-1)/k},
\]
uniformly in $i$, proving the claim.
\end{editamirblockNEW}
\end{proof}



\begin{lemma}[Small-slope graph distortion on $k$-- and $(k\!-\!1)$--areas]\label{lem:small-graph-distortion}
Let $E\subset\R^k$ be measurable and let $G:E\to\R^{d-k}$ be $C^1$ with $\|DG\|\le\varepsilon$.
Let $\Gamma:=\{(y,G(y)):y\in E\}\subset\R^d$ be the graph.
Then
\[
\mathcal H^k(\Gamma)\ =\ (1+O(\varepsilon^2))\,\mathcal H^k(E).
\]
If $E_0\subset E$ is contained in a $(k\!-\!1)$--dimensional affine hyperplane and
$\Gamma_0:=\{(y,G(y)):y\in E_0\}$, then likewise
\[
\mathcal H^{k-1}(\Gamma_0)\ =\ (1+O(\varepsilon^2))\,\mathcal H^{k-1}(E_0),
\]
where the implied constants depend only on $k$.
\end{lemma}
\begin{proof}
This is the area formula for graphs.  The $m$--dimensional Jacobian of a graph is
$\sqrt{\det(I+(DG)^T DG)}$ on $m$--planes.  If $\|DG\|\le\varepsilon$, then the eigenvalues of $(DG)^T DG$ are $\le \varepsilon^2$,
so $\sqrt{\det(I+(DG)^T DG)}=1+O(\varepsilon^2)$ uniformly.  Apply with $m=k$ and $m=k-1$.
\end{proof}


\iffalse
\begin{proposition}[{\color{blue}Corner--exit footprint geometry is preserved under small--slope holomorphic graphs}]\label{prop:holomorphic-corner-exit-g1g2-old1}
\begin{editamirblockNEW}
Fix $d\ge 2$ and $1\le k<d$.  Let $Q=[0,h]^d$ and let $v$ be a vertex.
Let $P$ be an affine $k$--plane and set $E:=P\cap Q$.
Assume:
\begin{enumerate}
\item[\textup{(H1)}]\label{H1corner-old1}
(\textbf{Corner--exit simplex footprint}) $E$ is a $k$--simplex contained in $B(v,c_0h)$ for some $0<c_0<1$, and there exist \emph{distinct} codimension--$1$ faces $F_0,\dots,F_k$ of $Q$ incident to $v$ such that each $E\cap F_i$ is a $(k-1)$--dimensional facet of $E$ and $E$ meets no other codimension--$1$ faces of $Q$.
\item[\textup{(H2)}]\label{H2cornerfat-old1}
(\textbf{Uniform fatness}) $E$ is $\Lambda$--fat (in the quantitative sense of Lemma~\ref{lem:corner-simplex-face-mass}), hence
$\mathcal H^{k-1}(E\cap F_i)\simeq_{k,\Lambda} v_E^{(k-1)/k}$ for $v_E:=\mathcal H^k(E)$.
\end{enumerate}
Let $Y\subset\R^d$ be a smooth oriented $k$--dimensional submanifold such that $Y\cap Q$ is a single $C^1$ graph over $E$
with slope at most $\varepsilon$, i.e.\ there is a $C^1$ embedding $\Phi:E\to\R^d$ with $\Phi(E)=Y\cap Q$ and
$\|D\Phi-\mathrm{Id}\|\le C\,\varepsilon$ in the coordinates of $P$.
Let $\delta:=\min\{\dist(E,F): F\ \text{a codimension--$1$ face of $Q$ with }F\notin\{F_0,\dots,F_k\}\}>0$ (positivity follows from \eqref{H1corner-old1}). Assume $\varepsilon$ is small enough so that $\Phi(E)\subset B(v,(c_0+\tfrac12)h)$ and $\sup_{x\in E}|\Phi(x)-x|<\delta/2$.

Then:
\begin{enumerate}
\item[\textup{(G1)}]\label{G1iff-old1}
(\textbf{Face incidence}) For any codimension--$1$ face $F$ of $Q$,
\[
\mathcal H^{k-1}(Y\cap F)>0
\quad\Longleftrightarrow\quad
F\in\{F_0,\dots,F_k\}.
\]
\item[\textup{(G2)}]\label{G2mass-old1}
(\textbf{Per--face boundary mass comparability}) For each $i=0,\dots,k$, the intersection $Y\pitchfork F_i$ is a smooth oriented $(k-1)$--submanifold and
\[
\Mass\bigl(\partial([Y]\llcorner Q)\llcorner F_i\bigr)
=\mathcal H^{k-1}(Y\cap F_i)
\simeq_{k,\Lambda,\varepsilon}\ \mathcal H^{k-1}(E\cap F_i)
\simeq_{k,\Lambda}\ v_E^{(k-1)/k},
\]
with distortion factor $\simeq_{k,\Lambda,\varepsilon}$ tending to $1$ as $\varepsilon\to 0$ (in fact $1+O_{k}(\varepsilon^2)$).
\end{enumerate}
\end{editamirblockNEW}
\end{proposition}
\fi

\iffalse
\begin{proposition}[{\color{blue}Corner--exit footprint geometry for small--slope graphs}]\label{prop:holomorphic-corner-exit-g1g2-old2}
\begin{editamirblockNEW}
Fix $d\ge 2$ and $1\le k<d$.  Let $Q=[0,h]^d\subset\R^d$ and let $v$ be a vertex of $Q$.
Let $P\subset\R^d$ be an affine $k$--plane and set $E:=P\cap Q$.

Assume:
\begin{enumerate}
\item[\textup{(H1)}]\label{H1corner-old2}
(\textbf{Corner--exit simplex footprint})
$E$ is a $k$--simplex with one vertex at $v$, contained in $B(v,c_0 h)$ for some $0<c_0<1$.
Moreover, there exist \emph{distinct} codimension--$1$ faces $F_0,\dots,F_k$ of $Q$, each incident to $v$,
such that the $k\!+\!1$ facets of $E$ are exactly the sets $E\cap F_i$ $(i=0,\dots,k)$; in particular,
$E$ meets no other codimension--$1$ faces of $Q$.

\item[\textup{(H2)}]\label{H2cornerfat-old2}
(\textbf{Uniform fatness})
$E$ is $\Lambda$--fat in the sense of Lemma~\ref{lem:corner-simplex-face-mass}.  Hence, writing $v_E:=\mathcal H^k(E)$,
\[
\mathcal H^{k-1}(E\cap F_i)\ \simeq_{k,\Lambda}\ v_E^{(k-1)/k}
\qquad (i=0,\dots,k).
\]
\end{enumerate}

Let $Y\subset\R^d$ be a smooth oriented $k$--dimensional submanifold such that $Y\cap Q$ is a single $C^1$ graph over $E$
with slope at most $\varepsilon$, i.e.\ there is a $C^1$ embedding $\Phi:E\to\R^d$ with $\Phi(E)=Y\cap Q$ and
\[
\|D\Phi-\mathrm{Id}\|\le C\,\varepsilon
\quad\text{(in coordinates on $P$),}
\]
for a universal constant $C$.
Let
\[
\delta:=\min\{\dist(E,F): F\ \text{a codimension--$1$ face of $Q$ with}\ E\cap F=\emptyset\}.
\]
Assume $\varepsilon$ is small enough that $\Phi(E)\subset B(v,(c_0+\tfrac12)h)$ and also
\[
\sup_{x\in E}\,|\Phi(x)-x|\ <\ \delta/2.
\]

Then:
\begin{enumerate}
\item[\textup{(G1)}]\label{G1faces}
(\textbf{Exit faces are preserved}) For every codimension--$1$ face $F$ of $Q$,
\[
Y\cap F\neq\emptyset
\quad\Longleftrightarrow\quad
F\in\{F_0,\dots,F_k\}.
\]

\item[\textup{(G2)}]\label{G2mass-old2}
(\textbf{Per--face boundary mass comparability})
For each $i=0,\dots,k$, the intersection $Y\pitchfork F_i$ is a smooth oriented $(k-1)$--submanifold and
\[
\Mass\bigl(\partial([Y]\llcorner Q)\llcorner F_i\bigr)
=\mathcal H^{k-1}(Y\cap F_i)
=\bigl(1+O_k(\varepsilon^2)\bigr)\,\mathcal H^{k-1}(E\cap F_i)
\simeq_{k,\Lambda}\ v_E^{(k-1)/k}.
\]
\end{enumerate}
\end{editamirblockNEW}
\end{proposition}
\fi

\begin{proposition}[{\color{blue}Corner--exit footprint geometry for small--slope graphs}]\label{prop:holomorphic-corner-exit-g1g2}
\begin{editamirblockNEW}
Fix $d\ge 2$ and $1\le k<d$.  Let $Q=[0,h]^d\subset\R^d$ and let $v$ be a vertex of $Q$.
Let $P\subset\R^d$ be an affine $k$--plane and set $E:=P\cap Q$.
Write $v_E:=\mathcal H^k(E)$.

Assume:
\begin{enumerate}
\item[\textup{(H1)}]\label{H1corner}
(\textbf{Corner--exit simplex footprint})
$E$ is a $k$--simplex with one vertex at $v$.  Moreover, there exist \emph{distinct} codimension--$1$ faces
$F_0,\dots,F_k$ of $Q$, each incident to $v$, such that the $k\!+\!1$ facets of $E$ are exactly the sets $E\cap F_i$
$(i=0,\dots,k)$; in particular, $E$ meets no other codimension--$1$ faces of $Q$.

\item[\textup{(H2)}]\label{H2cornerfat}
(\textbf{Uniform fatness})
$E$ is $\Lambda$--fat (in the quantitative sense of Lemma~\ref{lem:corner-simplex-face-mass}); hence
\[
\mathcal H^{k-1}(E\cap F_i)\ \simeq_{k,\Lambda}\ v_E^{(k-1)/k}\qquad (i=0,\dots,k).
\]
\end{enumerate}

Let $Y\subset\R^d$ be a smooth oriented $k$--dimensional submanifold such that $Y\cap Q$ is a single $C^1$ graph over $E$
with slope at most $\varepsilon$, i.e.\ there is a $C^1$ embedding $\Phi:E\to\R^d$ with $\Phi(E)=Y\cap Q$ and
$\|D\Phi-\mathrm{Id}\|_{C^0(E)}\le C\,\varepsilon$ in the coordinates of $P$.
Let
\[
\delta:=\min\{\dist(E,F):\ F\ \text{a codimension--$1$ face of $Q$ with }F\notin\{F_0,\dots,F_k\}\}\ >0.
\]
Assume in addition that
\[
\sup_{x\in E}|\Phi(x)-x|\ <\ \delta/2.
\]

Then:
\begin{enumerate}
\item[\textup{(G1)}]\label{G1iff}
(\textbf{Face incidence})
For any codimension--$1$ face $F$ of $Q$,
\[
Y\cap F\neq\emptyset
\quad\Longleftrightarrow\quad
F\in\{F_0,\dots,F_k\}.
\]

\item[\textup{(G2)}]\label{G2mass}
(\textbf{Per--face boundary mass comparability})
For each $i=0,\dots,k$, the intersection $Y\pitchfork F_i$ is a smooth oriented $(k-1)$--submanifold and
\[
\Mass\bigl(\partial([Y]\llcorner Q)\llcorner F_i\bigr)
=\mathcal H^{k-1}(Y\cap F_i)
=\bigl(1+O_k(\varepsilon^2)\bigr)\,\mathcal H^{k-1}(E\cap F_i)
\simeq_{k,\Lambda}\ v_E^{(k-1)/k}.
\]
\end{enumerate}
\end{editamirblockNEW}
\end{proposition}

\begin{proof}
\begin{editamirblockNEW}
For \eqref{G1iff}, let $F$ be a codimension--$1$ face of $Q$ not in $\{F_0,\dots,F_k\}$.
By definition of $\delta$, we have $\dist(E,F)\ge \delta$.
If $y\in Y\cap F$, then $y=\Phi(x)$ for some $x\in E$, and hence
\[
\delta\ \le\ \dist(x,F)\ \le\ |x-\Phi(x)|\ <\ \delta/2,
\]
a contradiction.  Thus $Y\cap F=\emptyset$ for every non--designated face $F$.
Conversely, if $F=F_i$ is one of the designated faces, then $E\cap F_i$ is a facet of $E$ and is nonempty.
For $\varepsilon$ small the graph is transverse to $F_i$ along that facet, hence $Y\cap F_i\neq\emptyset$.

For \eqref{G2mass}, since $Y$ is smooth, $\partial[Y]=0$. Therefore
\[
\partial([Y]\llcorner Q)= [Y]\llcorner \partial Q,
\]
and restricting to a face $F_i$ gives
$\partial([Y]\llcorner Q)\llcorner F_i = [Y]\llcorner F_i$ with the induced orientation.
Thus $\Mass(\partial([Y]\llcorner Q)\llcorner F_i)=\mathcal H^{k-1}(Y\cap F_i)$.
Because $\Phi$ is a $C^1$ graph map with slope $\le \varepsilon$, the area formula gives
\[
\mathcal H^{k-1}(Y\cap F_i)=\bigl(1+O_k(\varepsilon^2)\bigr)\,\mathcal H^{k-1}(E\cap F_i),
\]
and the final comparison follows from the fatness estimate in \eqref{H2cornerfat}.
\end{editamirblockNEW}
\end{proof}

\begin{editamirblockNEW}
\editamirNEW{\noindent\textbf{Referee cleanup.} The additional proof blocks that followed here were earlier draft variants of the same argument.
They have been removed to prevent multiple competing proofs from coexisting in the compiled manuscript.}
\end{editamirblockNEW}


\begin{corollary}[{\color{blue}Corner--exit faces persist uniformly across a finite template family}]\label{cor:holomorphic-corner-exit-inherits}
\begin{editamirblockNEW}
Fix $d\ge 2$ and $1\le k<d$.  Let $Q=[0,h]^d$ and let $v$ be a vertex.
Let $\{P_a\}_{a=1}^N$ be a finite family of affine $k$--planes and set $E_a:=P_a\cap Q$.
\editamir{Suppose that} for each $a$:
\begin{enumerate}
\item[\textup{(T1)}]\label{T1}
$E_a$ is a $k$--simplex satisfying the footprint hypotheses \eqref{H1corner} of Proposition~\ref{prop:holomorphic-corner-exit-g1g2}
(with designated exit faces $F_0^{(a)},\dots,F_k^{(a)}$ incident to $v$).
\item[\textup{(T2)}]\label{T2}
$E_a$ is $\Lambda$--fat with the same fatness parameter $\Lambda$.
\end{enumerate}
\editamir{Assume moreover that $\varepsilon>0$ is chosen small enough (depending only on $k$ and $\Lambda$) so that Proposition~\ref{prop:holomorphic-corner-exit-g1g2} applies to every pair $(E_a,Y^{(a)})$.}

Define the uniform gap
\[
\delta_\star:=\min_{1\le a\le N}\ \min\{\dist(E_a,F): F\ \text{a codimension--$1$ face of $Q$ with }F\notin\{F_0^{(a)},\dots,F_k^{(a)}\}\}\ >0.
\]
Let $Y^{(a)}$ be smooth oriented $k$--submanifolds such that each $Y^{(a)}\cap Q$ is a single $C^1$ graph over $E_a$
with slope at most $\varepsilon$, realized by an embedding $\Phi_a:E_a\to\R^d$ with $\Phi_a(E_a)=Y^{(a)}\cap Q$ and
$\sup_{x\in E_a}|\Phi_a(x)-x|<\delta_\star/2$.

Then for every $a$ the conclusions \eqref{G1iff}--\eqref{G2mass} of Proposition~\ref{prop:holomorphic-corner-exit-g1g2} hold for $(E_a,Y^{(a)})$,
with constants depending only on $(k,\Lambda)$ and on the graph--slope bound (equivalently on $\varepsilon$).
\end{editamirblockNEW}
\end{corollary}

\begin{proof}
\begin{editamirblockNEW}
Apply Proposition~\ref{prop:holomorphic-corner-exit-g1g2} to each pair $(E_a,Y^{(a)})$.
The only additional point is that the smallness requirement on the graph (encoded there by the gap parameter $\delta$) can be chosen uniformly in $a$,
because the family is finite and $\delta_\star>0$ by definition.
\editamir{This is legitimate because the smallness threshold in Proposition~\ref{prop:holomorphic-corner-exit-g1g2} depends only on $(k,\Lambda)$ (and the slope bound), not on $a$.}
\end{editamirblockNEW}
\end{proof}


\begin{remark}[Recognition Science interpretation (updated)]\label{rem:rs-interpretation}
\begin{editamirblockNEW}
\editamir{\noindent\textbf{Non-logical commentary.} This remark is optional exposition only and is \emph{not used} anywhere in the proof chain.
All logical dependencies for Theorem~\ref{thm:main-hodge} are contained in the stated results and the published “Classical Inputs” ledger.}
\end{editamirblockNEW}

\iffalse From the Recognition Science perspective (see \texttt{Source-Super.txt} and \texttt{recognition-geometry-dec-6.tex}), \fi
the microstructure/gluing step is a ``ledger closure'' requirement: local recognition events (slivers) must be manufactured so that
their interface mismatch is negligible.  In this language a mesh cell $Q$ plays the role of a \emph{resolution cell} (a region on which the ``event alphabet''
is stable), and the natural analytic resolution scale in K\"ahler quantization is the Bergman scale \editamir{$N^{-1/2}$}.
Thus the correct classical checkpoint is a \emph{finite-resolution stability statement} on a ball containing $Q$:
construct holomorphic equations that are uniformly $C^1$-close to a fixed linear model on a Bergman ball (via Bergman kernel/peak-section control,
e.g.\ Lemma~\ref{lem:bergman-control} or the cutoff+$\bar\partial$ route in Lemma~\ref{lem:bergman-affine-approx-hormander}), and then conclude that the zero set
is a \emph{single sheet} on all of $Q$ by a quantitative contraction/implicit-function argument (Lemma~\ref{lem:global-graph-contraction}).
Once this cell-scale single-sheet property holds, the corner-exit geometry forces deterministic face incidence and uniform per-face mass (Proposition~\ref{prop:holomorphic-corner-exit-g1g2}).
\end{remark}
\end{editblock}

\begin{editblock}
\begin{lemma}[Sliver stability under $C^1$-graph perturbations]\label{lem:sliver-stability}
\begin{editamirblockNEW}
Let $Q\subset\R^{2n}$ be a cube of diameter $h$, and let $P$ be an affine calibrated $(2n-2p)$--plane.
Let $Y$ be a smooth $(2n-2p)$--submanifold such that $Y\cap Q$ is a $C^1$ graph over $P\cap Q$ with slope $\le \varepsilon$, i.e.\
in suitable coordinates
\[
Y\cap Q=\{x+u(x):x\in P\cap Q\},
\qquad
u:P\cap Q\to P^\perp,
\qquad
\|Du\|_{C^0}\le \varepsilon.
\]
Then:
\begin{enumerate}
\item[\textnormal{(i)}] (\textbf{Mass comparability})
\[
\Mass([Y]\llcorner Q) = \bigl(1+O(\varepsilon^2)\bigr)\,\Mass([P]\llcorner Q),
\]
where the implied constant depends only on $(n,p)$ (and in particular the ratio is $\ge 1$).
\item[\textnormal{(ii)}] (\textbf{Disjointness persistence, with an anchor}) 
Let $t_1,t_2\in P^\perp$ and suppose $Y_1,Y_2$ are $C^1$ graphs of slope $\le \varepsilon$ over the parallel planes
$P+t_1$ and $P+t_2$ on $(P+t_i)\cap Q$, realized as $Y_i\cap Q=\{x+u_i(x):x\in (P+t_i)\cap Q\}$ with $\|Du_i\|_{C^0}\le\varepsilon$.
Assume further that for each $i\in\{1,2\}$ there exists an \emph{anchor point} $x_i\in (P+t_i)\cap Q$ with $x_i\in Y_i$
(equivalently $u_i(x_i)=0$).
\editamir{Let $D_i:=\mathrm{diam}\bigl((P+t_i)\cap Q\bigr)\le h$. If $\|t_1-t_2\|\ge 10\,\varepsilon\,\max\{D_1,D_2\}$, then $Y_1\cap Q$ and $Y_2\cap Q$ are disjoint.}
\end{enumerate}
\end{editamirblockNEW}
\end{lemma}


\begin{proof}
\begin{editamirblockNEW}
\textnormal{(i)} Write $k:=2n-2p$ and parametrize $Y\cap Q$ as the graph of $u:P\cap Q\to P^\perp$ with $\|Du\|_{C^0}\le\varepsilon$.
By the area formula for graphs,
\[
\Mass([Y]\llcorner Q)
=\int_{P\cap Q}\sqrt{\det(I+Du^\top Du)}\,d\mathcal H^{k}.
\]
Since $Du^\top Du$ is positive semidefinite and $\|Du^\top Du\|\le \|Du\|^2\le \varepsilon^2$, one has
\[
1\ \le\ \sqrt{\det(I+Du^\top Du)}\ \le\ 1+C(n,p)\,\varepsilon^2,
\]
hence $\Mass([Y]\llcorner Q)=\bigl(1+O(\varepsilon^2)\bigr)\Mass([P]\llcorner Q)$.

\textnormal{(ii)} Fix $i\in\{1,2\}$ and let $x_i\in (P+t_i)\cap Q$ be an anchor with $u_i(x_i)=0$.
For any $x\in (P+t_i)\cap Q$,
\[
|u_i(x)|\ =\ |u_i(x)-u_i(x_i)|\ \le\ \|Du_i\|_{C^0}\,|x-x_i|\ \le\ \varepsilon\,D_i\ \le\ \varepsilon\,h,
\]
\editamir{since $|x-x_i|\le \mathrm{diam}((P+t_i)\cap Q)=D_i\le h$.}
Therefore every point $y=x+u_i(x)\in Y_i\cap Q$ satisfies
$\editamir{\dist(y,P+t_i)\le \varepsilon D_i}$, i.e.
\[
Y_i\cap Q\ \subset\ \editamir{\mathcal N_{\varepsilon D_i}}(P+t_i)\cap Q.
\]
\editamir{If $\|t_1-t_2\|\ge 10\,\varepsilon\,\max\{D_1,D_2\}$, then the tubular neighborhoods $\mathcal N_{\varepsilon D_1}(P+t_1)$ and $\mathcal N_{\varepsilon D_2}(P+t_2)$ are disjoint, hence so are $Y_1\cap Q$ and $Y_2\cap Q$.}
\end{editamirblockNEW}
\end{proof}


\begin{lemma}[Packing bound for disjoint sliver graphs]\label{lem:sliver-packing}
Let $Q\subset\R^{2n}$ be a bounded domain of diameter $h$ and fix an affine $(2n-2p)$-plane $P$ with transverse space $P^\perp\cong\R^{2p}$.
Assume we have affine translates $P+t_1,\dots,P+t_N$ such that each $(P+t_a)\cap Q\neq\emptyset$ and
\[
\|t_a-t_b\|\ \ge\ 10\,\varepsilon\,h
\qquad (a\neq b).
\]
Then $N\le C(n,p)\,\varepsilon^{-2p}$.
\editamir{More generally, if the translation parameters lie in a transverse ball $B_r(0)\subset P^\perp$ and satisfy
$\|t_a-t_b\|\ge 10\,\varepsilon\,r$ for $a\neq b$, then the same conclusion holds.}
\end{lemma}

\begin{proof}
\editamir{For the first claim, since $(P+t_a)\cap Q\neq\emptyset$ and $\mathrm{diam}(Q)=h$, pick $a_0$ and points $x_a\in (P+t_a)\cap Q$.
Projecting $x_a-x_{a_0}$ orthogonally to $P^\perp$ gives $t_a-t_{a_0}$, hence $|t_a-t_{a_0}|\le |x_a-x_{a_0}|\le h$.
After translating the transverse coordinates by $-t_{a_0}$, we may assume $\{t_a\}\subset B_h(0)\subset P^\perp$.}
\editamir{The balls $B(t_a,5\varepsilon h)\subset P^\perp$ are pairwise disjoint and contained in $B_{(1+5\varepsilon)h}(0)$.}
\editamir{Comparing Euclidean volumes in $\R^{2p}$ gives}
\[
N\,(5\varepsilon h)^{2p}\ \lesssim\ h^{2p},
\]
hence $N\lesssim \varepsilon^{-2p}$ as claimed.
\editamir{The “ball of radius $r$” variant is identical with $h$ replaced by $r$.}
\end{proof}
\end{editblock}

\begin{editblock}
\begin{proposition}[Realizing a finite translation template locally]\label{prop:finite-template}
\begin{editamirblockNEW}
\noindent\textbf{Role in the closure chain.}
Proposition~\ref{prop:finite-template} is the \emph{analytic local holomorphic-realization} input.
It depends only on:
(i) Bergman/H\"ormander control (Lemma~\ref{lem:bergman-control}),
(ii) an implicit-function/graph lemma from gradient control (Lemma~\ref{lem:graph-from-grad}),
and (iii) stability/disjointness of small-slope slivers under plane separation (Lemma~\ref{lem:sliver-stability}).
It does \emph{not} assume any of the corner-exit net constants $\alpha_*(h),A_*(h),\Lambda(h)$; those enter only through
how one chooses the translations $t_a$ and the separation scale $\delta=10\,\varepsilon\,\mathrm{diam}(Q)$ in later applications
(e.g. Proposition~\ref{prop:holomorphic-corner-exit-L1} using Proposition~\ref{prop:corner-exit-template-net}).
\end{editamirblockNEW}
Fix a holomorphic chart identifying a neighborhood of a cell $Q$ with a domain in $\C^n$, and fix a calibrated complex $(n-p)$-plane
$P\subset\C^n$ with normal covectors $\lambda_1,\dots,\lambda_p$ (so $\bigcap_i\ker\lambda_i=P$).
Let $t_1,\dots,t_N\in P^\perp\cong\R^{2p}$ be translation vectors such that the affine planes $(P+t_a)$ are pairwise disjoint on $Q$ and
\editamir{let $D_Q:=\max_{1\le a\le N}\mathrm{diam}\bigl((P+t_a)\cap Q\bigr)\le \mathrm{diam}(Q)$, and assume the translations are separated by
$\|t_a-t_b\|\ge 10\,C_{\mathrm{graph}}\,\varepsilon\,D_Q$.}
Fix $\varepsilon>0$ and choose a holomorphic tensor power \editamir{$N_{\mathrm{hol}}\ge N_1(\varepsilon)$} as in Lemma~\ref{lem:bergman-control},
with $N_{\mathrm{hol}}$ large enough that
\[
\mathrm{diam}(Q)\ \le\ c\,\editamir{N_{\mathrm{hol}}^{-1/2}},
\]
where $c>0$ is the universal constant in Lemma~\ref{lem:bergman-control}.\editamir{ By increasing $N_{\mathrm{hol}}$ if necessary, we may assume in fact $\mathrm{diam}(Q)\le (c/2)\,N_{\mathrm{hol}}^{-1/2}$; then $Q\subset B_{(c/2)\,N_{\mathrm{hol}}^{-1/2}}(x)$ for every $x\in Q$.}
For each $a$, pick any point $x_a\in (P+t_a)\cap Q$.
Then there exist $\psi$-calibrated holomorphic complete intersections $Y^1,\dots,Y^N\subset X$ such that, on $Q$:
\begin{enumerate}
\item[\textnormal{(i)}] $Y^a$ is a $C^1$ graph over $P+t_a$ with slope \editamir{$\le C_{\mathrm{graph}}\varepsilon$} (hence $\angle(T_yY^a,P)\le C\varepsilon$);
\item[\textnormal{(ii)}] the pieces $Y^a\cap Q$ are pairwise disjoint;
\item[\textnormal{(iii)}] $\Mass([Y^a]\llcorner Q)=(1+O(\varepsilon^2))\,\Mass([P+t_a]\llcorner Q)$.
\end{enumerate}
\end{proposition}

\begin{proof}
For each $a$, apply Lemma~\ref{lem:bergman-control} at $x_a$ with covectors $\lambda_i$ to obtain sections
$s_{a,1},\dots,s_{a,p}\in H^0(X,L^{N_{\mathrm{hol}}})$ whose first-order jets at $x_a$ realize the covectors $\lambda_i$. \editamir{This applies on $Q$ since $Q\subset B_{(c/2)\,N_{\mathrm{hol}}^{-1/2}}(x_a)\subset B_{c\,N_{\mathrm{hol}}^{-1/2}}(x_a)$ (because $x_a\in Q$ and we ensured $\mathrm{diam}(Q)\le (c/2)\,N_{\mathrm{hol}}^{-1/2}$).}
Let $Y^a:=\{s_{a,1}=\cdots=s_{a,p}=0\}$.
\editamir{Let $C_{\mathrm{graph}}$ be the slope constant from Lemma~\ref{lem:graph-from-grad}.} Then Lemma~\ref{lem:graph-from-grad} gives (i) with slope $\le C_{\mathrm{graph}}\varepsilon$, and Lemma~\ref{lem:sliver-stability}(i) gives (iii) with the same slope parameter. \editamir{Applying Lemma~\ref{lem:sliver-stability}(ii) with $\varepsilon$ replaced by $C_{\mathrm{graph}}\varepsilon$ and using the separation assumption $\|t_a-t_b\|\ge 10\,C_{\mathrm{graph}}\varepsilon\,D_Q$ yields (ii).}
\end{proof}
\end{editblock}

\begin{editblock}
\iffalse
\begin{proposition}[{\color{blue}Corner--exit: $L^1$ interface mass control on boundary faces}]\label{prop:holomorphic-corner-exit-L1-old}
\begin{editamirblockNEW}
Work in the setting of Proposition~\ref{prop:finite-template}, and fix a cube $Q=[0,h]^d$ (with a chosen vertex $v$) inside a holomorphic coordinate chart.
Let $\{P_a\}_{a=1}^N$ be the associated finite family of affine $k$--planes from the corner--exit template construction, and write $E_a:=P_a\cap Q$.
\editamir{Suppose that} each $E_a$ is a $\Lambda$--fat corner--exit simplex in the sense of Proposition~\ref{prop:holomorphic-corner-exit-g1g2} (so it comes with designated exit faces
$F_0^{(a)},\dots,F_k^{(a)}$ incident to $v$ and meets no other codimension--$1$ faces).  Set $v_{E_a}:=\mathcal H^k(E_a)$.

Let $Y^{(a)}$ be the holomorphic complete intersections produced by Proposition~\ref{prop:finite-template} on $Q$ using anchor points $x_a\in E_a$, so that:
\begin{enumerate}
\item[\textup{(a)}]\label{L1a}
$Y^{(a)}\cap Q$ is a single $C^1$ graph over $E_a$ with slope $O(\varepsilon)$ (hence admits a graph embedding $\Phi_a:E_a\to\R^d$ with $\Phi_a(E_a)=Y^{(a)}\cap Q$).
Moreover, since $x_a\in Y^{(a)}$ one may choose $\Phi_a$ with $\Phi_a(x_a)=x_a$, and then
$\sup_{x\in E_a}|\Phi_a(x)-x|\ \lesssim\ \varepsilon\,h$.
\item[\textup{(b)}]\label{L1b}
the submanifolds $Y^{(a)}$ are pairwise disjoint in $Q$,
\item[\textup{(c)}]\label{L1c}
each pair $(E_a,Y^{(a)})$ satisfies the corner--exit conclusions \eqref{G1iff}--\eqref{G2mass} of Proposition~\ref{prop:holomorphic-corner-exit-g1g2}; in particular,
$Y^{(a)}$ meets only the designated exit faces and, on each such face,
\[
\Mass\bigl(\partial([Y^{(a)}]\llcorner Q)\llcorner F_i^{(a)}\bigr)
=\mathcal H^{k-1}(Y^{(a)}\cap F_i^{(a)})\ \simeq_{k,\Lambda,\varepsilon}\ v_{E_a}^{(k-1)/k}.
\]
\end{enumerate}
Consequently, since $\partial[Y^{(a)}]=0$ and by \eqref{G1iff} the boundary $\partial([Y^{(a)}]\llcorner Q)$ is supported on the union of the designated faces,
\[
\Mass\bigl(\partial([Y^{(a)}]\llcorner Q)\bigr)
\ \le\ \sum_{i=0}^k \Mass\bigl(\partial([Y^{(a)}]\llcorner Q)\llcorner F_i^{(a)}\bigr)
\ \lesssim_{k,\Lambda,\varepsilon}\ v_{E_a}^{(k-1)/k}.
\]
In particular, summing over $a$,
\[
\sum_{a=1}^N \Mass\bigl(\partial([Y^{(a)}]\llcorner Q)\bigr)
\ \lesssim_{k,\Lambda,\varepsilon}\ \sum_{a=1}^N v_{E_a}^{(k-1)/k}.
\]
\end{editamirblockNEW}
\end{proposition}

\begin{proof}
\begin{editamirblockNEW}
Items \eqref{L1a}--\eqref{L1b} are exactly the output of Proposition~\ref{prop:finite-template} on the cube $Q$.
(The $C^0$ bound in \eqref{L1a} follows because $x_a\in Y^{(a)}$ by Lemma~\ref{lem:bergman-control}(i), so in graph coordinates
$\Phi_a(x)=x+u_a(x)$ with $u_a(x_a)=0$ and $\|Du_a\|_{C^0}\lesssim\varepsilon$, hence $\sup_{E_a}|u_a|\lesssim \varepsilon\,\mathrm{diam}(E_a)\lesssim \varepsilon h$.)

For \eqref{L1c}, apply Corollary~\ref{cor:holomorphic-corner-exit-inherits} to the finite family $\{E_a\}_{a=1}^N$ and the graphs $Y^{(a)}\cap Q$.
Since the family is finite, the uniform gap parameter $\delta_\star$ in Corollary~\ref{cor:holomorphic-corner-exit-inherits} is positive, and for $\varepsilon$ small
the bound $\sup_{E_a}|\Phi_a-\mathrm{Id}|\lesssim \varepsilon h$ ensures $\sup_{E_a}|\Phi_a-\mathrm{Id}|<\delta_\star/2$ for all $a$.

Finally, because each $Y^{(a)}$ is a complete intersection, $\partial[Y^{(a)}]=0$, hence
$\partial([Y^{(a)}]\llcorner Q) = [Y^{(a)}]\llcorner \partial Q$ (up to sign), and $\partial Q$ is the disjoint union of its codimension--$1$ faces.
By \eqref{G1iff}, only the designated faces $F_i^{(a)}$ contribute; therefore
\[
\Mass\bigl(\partial([Y^{(a)}]\llcorner Q)\bigr)
\le \sum_{i=0}^k \Mass\bigl(\partial([Y^{(a)}]\llcorner Q)\llcorner F_i^{(a)}\bigr)
\lesssim_{k,\Lambda,\varepsilon} v_{E_a}^{(k-1)/k},
\]
and summing over $a$ gives the stated $L^1$ bound.\end{editamirblockNEW}
\end{proof}
\fi

\begin{proposition}[{\color{blue}Corner--exit: $L^1$ interface mass control on boundary faces}]\label{prop:holomorphic-corner-exit-L1}
\begin{editamirblockNEW}
Work in a holomorphic coordinate chart identifying a neighborhood of a cell $Q=[0,h]^d\subset\R^d$, with a chosen vertex $v$.
Fix $1\le k<d$ and let $\{P_a\}_{a=1}^N$ be a finite family of affine $k$--planes.
Set $E_a:=P_a\cap Q$ and $v_{E_a}:=\mathcal H^k(E_a)$.

\editamir{Suppose that} each $E_a$ is a $\Lambda$--fat corner--exit simplex footprint in the sense of
Proposition~\ref{prop:holomorphic-corner-exit-g1g2}, with designated exit faces
$F_0^{(a)},\dots,F_k^{(a)}$ incident to $v$.

Let $Y^{(a)}$ be the holomorphic complete intersections produced by Proposition~\ref{prop:finite-template} on $Q$ using anchor points
$x_a\in E_a$, so that $Y^{(a)}\cap Q$ is a single $C^1$ graph over $E_a$ with slope at most $C_{\mathrm{graph}}\varepsilon$,
realized by an embedding $\Phi_a:E_a\to\R^d$ with $\Phi_a(E_a)=Y^{(a)}\cap Q$ and $\Phi_a(x_a)=x_a$.
Assume moreover that $\varepsilon>0$ is chosen small enough (depending only on $(k,\Lambda)$ and on $C_{\mathrm{graph}}$) so that the
conclusions \eqref{G1iff}--\eqref{G2mass} of Proposition~\ref{prop:holomorphic-corner-exit-g1g2} apply to every pair $(E_a,Y^{(a)})$
whenever $\sup_{E_a}|\Phi_a-\mathrm{Id}|<\delta_\star/2$, where
\[
\delta_\star:=\min_{1\le a\le N}\ \min\{\dist(E_a,F):\ F\ \text{a codimension--$1$ face of $Q$ with }F\notin\{F_0^{(a)},\dots,F_k^{(a)}\}\}\ >0.
\]
Assume $\sup_{E_a}|\Phi_a-\mathrm{Id}|<\delta_\star/2$ for all $a$.

Then for each $a$,
\[
\operatorname{spt}\bigl(\partial([Y^{(a)}]\llcorner Q)\bigr)\ \subset\ \bigcup_{i=0}^k F_i^{(a)},
\]
and
\[
\Mass\bigl(\partial([Y^{(a)}]\llcorner Q)\bigr)
\le \sum_{i=0}^k \Mass\bigl(\partial([Y^{(a)}]\llcorner Q)\llcorner F_i^{(a)}\bigr)
\ \lesssim_{k,\Lambda,\varepsilon}\ v_{E_a}^{(k-1)/k}.
\]
In particular,
\[
\sum_{a=1}^N \Mass\bigl(\partial([Y^{(a)}]\llcorner Q)\bigr)
\ \lesssim_{k,\Lambda,\varepsilon}\ \sum_{a=1}^N v_{E_a}^{(k-1)/k}.
\]
\end{editamirblockNEW}
\end{proposition}

\begin{proof}
\begin{editamirblockNEW}
The geometric graph and disjointness conclusions (existence of $\Phi_a$ and the bound $\sup_{E_a}|\Phi_a-\mathrm{Id}|\lesssim \varepsilon h$)
are the output of Proposition~\ref{prop:finite-template}.
Since the family $\{E_a\}_{a=1}^N$ is finite, the uniform gap $\delta_\star$ is positive, and for $\varepsilon$ small the displacement bound
ensures $\sup_{E_a}|\Phi_a-\mathrm{Id}|<\delta_\star/2$ for all $a$.

Apply Corollary~\ref{cor:holomorphic-corner-exit-inherits} (equivalently, Proposition~\ref{prop:holomorphic-corner-exit-g1g2})
to each pair $(E_a,Y^{(a)})$ (with slope parameter $\varepsilon':=C_{\mathrm{graph}}\varepsilon$).
By \eqref{G1iff}, $Y^{(a)}$ meets only the designated faces $F_i^{(a)}$ of $Q$.

Finally, because each $Y^{(a)}$ is a holomorphic complete intersection, it is a closed oriented $k$--cycle in $Q$, i.e.\ $\partial[Y^{(a)}]=0$.
Thus
\[
\partial([Y^{(a)}]\llcorner Q)= [Y^{(a)}]\llcorner \partial Q,
\]
and $\partial Q$ is the disjoint union of its codimension--$1$ faces.  Using \eqref{G2mass} on each designated face gives
\[
\Mass\bigl(\partial([Y^{(a)}]\llcorner Q)\llcorner F_i^{(a)}\bigr)
=\mathcal H^{k-1}(Y^{(a)}\cap F_i^{(a)})
\ \lesssim_{k,\Lambda,\varepsilon}\ v_{E_a}^{(k-1)/k},
\]
hence the stated bound for $\Mass(\partial([Y^{(a)}]\llcorner Q))$, and summing over $a$ yields the final estimate.
\end{editamirblockNEW}
\end{proof}


\begin{remark}[Vertex-star coherence (how to make the same template live across adjacent cubes)]\label{rem:vertex-star-coherence}
For the global gluing/plumbing, one wants the \emph{same index-$a$ sliver} anchored at a vertex $v$ to be used by every cube incident to $v$,
so that across any shared face the mismatch reduces to a pure prefix-count difference (rather than a geometric displacement mismatch).

\smallskip\noindent
This is achieved by choosing the anchor points $x_a$ in Proposition~\ref{prop:finite-template} (hence the Bergman balls on which the $C^1$ control holds)
to be \emph{vertex-centered}: take $x_a\in (P+t_a)\cap B(v,c_0h)$ (for instance $x_a=v+t_a$ in a coordinate model).
If the mesh satisfies $h\lesssim \editamir{N_{\mathrm{hol}}^{-1/2}}$ with a small enough constant, then the Bergman ball $B_{c\,\editamir{N_{\mathrm{hol}}^{-1/2}}}(x_a)$ contains the entire
vertex star $\mathrm{Star}(v)$ (the union of the finitely many cubes meeting at $v$), so the resulting holomorphic complete intersection $Y^a$
is a single-sheet graph over the same affine translate $P+t_a$ \emph{on every cube in $\mathrm{Star}(v)$ simultaneously}.
Thus the vertex template is realized by a single global holomorphic object $Y^a$, and restricting to each cube produces coherent
face slices at that vertex.
\end{remark}
\end{editblock}

\begin{lemma}[Slow variation under rounding of Lipschitz targets]\label{lem:slow-variation-rounding}
Let $\{Q\}$ be a cubulation of mesh $h$, and let $f: X\to\R_{\ge 0}$ be a Lipschitz function with constant
$\mathrm{Lip}(f)\le L$ on each chart used for the cubulation.
Fix $m\ge 1$ and set the target real counts
\[
n_Q := m\,h^{2p}\, f(x_Q),
\]
for chosen basepoints $x_Q\in Q$.
Define integer counts by nearest-integer rounding $N_Q:=\lfloor n_Q\rceil$.
Then for adjacent cubes $Q\sim Q'$ one has
\[
|N_Q-N_{Q'}|\ \le\ L\,m\,h^{2p+1}\ +\ 1.
\]
If moreover $f\ge f_0>0$ and $m\,h^{2p+1}\ge 2/f_0$, then there is a constant $C=C(L,f_0)$ such that
\[
|N_Q-N_{Q'}|\ \le\ C\,h\,N_Q.
\]
\end{lemma}

\begin{proof}
Nearest-integer rounding satisfies $|N_Q-N_{Q'}|\le |n_Q-n_{Q'}|+1$.
By the Lipschitz bound, $|f(x_Q)-f(x_{Q'})|\le L\,\mathrm{dist}(x_Q,x_{Q'})\le Lh$, hence
$|n_Q-n_{Q'}|\le m\,h^{2p}\cdot Lh = L\,m\,h^{2p+1}$, proving the first inequality.

If $f\ge f_0$, then $n_Q\ge m\,h^{2p} f_0$, so $N_Q\ge n_Q-1 \ge m\,h^{2p}f_0-1$.
Under $m\,h^{2p+1}\ge 2/f_0$ one has $m\,h^{2p}f_0\ge 2/h$, hence $N_Q\ge (1/h)$.
Therefore $1\le hN_Q$ and
\[
|N_Q-N_{Q'}|\le L\,m\,h^{2p+1}+1 \le \Bigl(\frac{L}{f_0}+1\Bigr)\,hN_Q,
\]
which yields the stated form.
\end{proof}

\begin{remark}[Interpretation of the “many pieces” hypothesis in fixed-$m$ regimes]
\editamir{The relative slow-variation form $|N_Q-N_{Q'}|\le C\,h\,N_Q$ is automatic once $N_Q\gtrsim h^{-1}$, i.e.\ once the additive rounding error is negligible compared to the local count.}
\editamir{In the sliver/corner-exit regime at fixed cohomology multiplier $m$, this lower bound is achieved by shrinking the per-piece mass scale $A\asymp s^{k}$ (with $s\ll h$), so that $N_Q\sim M_Q/A$ is large even though the total budget $M_Q\asymp m\,h^{2n}$ is $O(m)$.}
\end{remark}
\begin{editblock}
\begin{lemma}[Slow variation persists under $0$--$1$ discrepancy rounding]\label{lem:slow-variation-discrepancy}
In the setting of Lemma~\ref{lem:slow-variation-rounding}, suppose instead of nearest-integer rounding we choose integers of the form
\[
N_Q\ :=\ \lfloor n_Q\rfloor\ +\ \varepsilon_Q,
\qquad \varepsilon_Q\in\{0,1\}.
\]
Then for adjacent cubes $Q\sim Q'$ one has
\[
|N_Q-N_{Q'}|\ \le\ L\,m\,h^{2p+1}\ +\ 2.
\]
If moreover $f\ge f_0>0$ and $m\,h^{2p+1}\ge 4/f_0$, then there is a constant $C=C(L,f_0)$ such that
\[
|N_Q-N_{Q'}|\ \le\ C\,h\,N_Q.
\]
\end{lemma}
\begin{proof}
For adjacent $Q\sim Q'$, one has
\[
|N_Q-N_{Q'}|
\ \le\ |\lfloor n_Q\rfloor-\lfloor n_{Q'}\rfloor|\ +\ |\varepsilon_Q-\varepsilon_{Q'}|
\ \le\ |n_Q-n_{Q'}|\ +\ 1\ +\ 1.
\]
The Lipschitz estimate from Lemma~\ref{lem:slow-variation-rounding} gives $|n_Q-n_{Q'}|\le L\,m\,h^{2p+1}$, proving the first claim.

For the relative bound, if $f\ge f_0$ then $n_Q\ge m\,h^{2p}f_0$ and hence
$N_Q\ge \lfloor n_Q\rfloor \ge n_Q-1 \ge m\,h^{2p}f_0-1$.
Under $m\,h^{2p+1}\ge 4/f_0$ we have $m\,h^{2p}f_0\ge 4/h$, so $N_Q\ge 3/h$ and thus $2\le hN_Q$.
Therefore
\[
|N_Q-N_{Q'}|
\ \le\ L\,m\,h^{2p+1}+2
\ \le\ \Bigl(\frac{L}{f_0}+1\Bigr)\,h\,(m\,h^{2p}f_0)\ +\ hN_Q
\ \le\ \Bigl(\frac{L}{f_0}+2\Bigr)\,h\,N_Q,
\]
after absorbing $m\,h^{2p}f_0\le n_Q\le N_Q+1$ into the constant and using $1\le hN_Q$.
\end{proof}
\end{editblock}
The local sheet construction is designed so that, uniformly for these test forms $d\eta$,
\[
T^{\mathrm{raw}}(d\eta)\approx \int_X (m\beta)\wedge d\eta,
\]
with an error controlled by $(\delta+\varepsilon+\mathrm{mesh})\cdot m$ (for fixed cohomology multiplier $m$).
Since $\beta$ is closed and $X$ has no boundary, $\int_X (m\beta)\wedge d\eta=\pm\int_X d(m\beta\wedge \eta)=0$.
\begin{editamirblock}
\begin{lemma}[Flat-norm control of the gluing mismatch]\label{lem:flatnorm-gluing-mismatch}
In Substep~4.2, for the raw current $T^{\mathrm{raw}}$ built from the cube-local sheets, one has
\[
\mathcal F\!\left(\partial T^{\mathrm{raw}}\right)\ \le\ \varepsilon_{\mathrm{glue}}(m,\delta,\varepsilon,\mathrm{mesh})\cdot m,
\qquad
\varepsilon_{\mathrm{glue}}\xrightarrow[\delta,\varepsilon\to 0,\ \mathrm{mesh}\to 0]{}0
\qquad\text{for fixed $m$ (chosen once to clear denominators of $[\gamma]$).}
\]
\end{lemma}

\begin{proof}
Let $k:=2n-2p$ so that $\partial T^{\mathrm{raw}}$ is a $(k-1)$--current.
By the Federer--Fleming dual characterization of the flat norm, it suffices to test $\partial T^{\mathrm{raw}}$
against smooth compactly-supported $(k-1)$--forms $\eta$ with $\|\eta\|_{\infty}\le 1$ and $\|d\eta\|_{\infty}\le 1$.
Decompose $\partial T^{\mathrm{raw}}$ as the alternating sum of face-mismatch currents across adjacent cubes in the partition.

For each codimension-one face $F$, Proposition~\ref{prop:transport-flat-glue} bounds the contribution of the face mismatch
to $\langle \partial T^{\mathrm{raw}},\eta\rangle$ by the Wasserstein transport cost $\tau_F$ plus the explicit cubewise
template/rounding error terms. Summing over all interior faces and invoking the global bookkeeping estimates from
Theorem~\ref{thm:sliver-mass-matching-on-template}, Corollary~\ref{cor:global-flat-weighted}, and
Proposition~\ref{prop:global-coherence-all-labels} yields the stated bound with
$\varepsilon_{\mathrm{glue}}(m,\delta,\varepsilon,\mathrm{mesh})\to 0$ in the cited parameter regime.
\end{proof}

\begin{remark}[Referee note: this is the quantitative bottleneck]\label{rem:lean-bottleneck-flatnorm}
For the Lean formalization, the nontrivial input encapsulated here is precisely the quantitative estimate delivered by
Proposition~\ref{prop:transport-flat-glue} and the cited bookkeeping results.
All subsequent uses of this estimate (Proposition~\ref{prop:glue-gap} and Proposition~\ref{prop:almost-calibration})
require only flat-norm calculus and standard filling/isoperimetric inequalities.
\editai{(AI note: when auditing Proposition~\ref{prop:transport-flat-glue}, the key places to re-derive are the Lipschitz bound for $f_\eta$ via the homotopy formula (including the boundary-slice term), the precise Kantorovich--Rubinstein duality hypotheses on the face parameter domain, and the summability of the small-angle/model-error term under the global parameter schedule.)}
\end{remark}
\end{editamirblock}

\begin{editamirblock}
\begin{editamirblockNEW}
\begin{lemma}[Federer--Fleming filling on $X$ for bounding cycles]\label{lem:FF-filling-X}
Let $X$ be a fixed compact smooth Riemannian manifold and fix an integer $k\ge 2$.
There exists a constant $C_X>0$ (depending only on $(X,k)$) with the following property:

\smallskip\noindent
If $R$ is an integral $(k-1)$--current in $X$ with $\partial R=0$ which bounds in $X$
(i.e.\ $R=\partial W$ for some integral $k$--current $W$ in $X$), then there exists an integral $k$--current $Q_R$ in $X$
such that $\partial Q_R=R$ and
\[
\Mass(Q_R)\ \le\ C_X\,\Mass(R)^{\frac{k}{k-1}}.
\]
\end{lemma}

\begin{proof}
Choose an (isometric) Nash embedding $\iota:X\hookrightarrow \R^N$ for some $N$.
Let $U$ be a tubular neighborhood of $\iota(X)$ and let $\pi:U\to X$ be the nearest--point projection.
Then $\pi$ is Lipschitz with $\Lip(\pi)$ depending only on $X$.

Since $R$ bounds in $X$, the pushforward $\iota_\# R$ bounds in $\R^N$ and is supported in the embedded submanifold $\iota(X)\subset U$.
By the \emph{relative} Euclidean isoperimetric (filling) inequality for integral currents in a fixed open set
(applied in the tubular neighborhood $U$; see Federer--Fleming \cite{FF60} and Federer \cite[\S4.2]{Fed69}),
there exists an integral $k$--current $Q$ supported in $U$ with $\partial Q=\iota_\# R$ and
\[
\Mass(Q)\ \le\ C_{N,k}\,\Mass(\iota_\# R)^{\frac{k}{k-1}}.
\]
Define $Q_R:=\pi_\# Q$, which is an integral $k$--current in $X$ (pushforward under a Lipschitz map preserves integrality; \cite{Fed69}).
Then
\[
\partial Q_R\ =\ \pi_\#(\partial Q)\ =\ \pi_\#(\iota_\# R)\ =\ (\pi\circ \iota)_\# R\ =\ R.
\]
Moreover,
\[
\Mass(Q_R)\ \le\ \Lip(\pi)^k\,\Mass(Q)
\ \le\ \bigl(\Lip(\pi)^k C_{N,k}\bigr)\,\Mass(R)^{\frac{k}{k-1}},
\]
since $\iota$ is an isometric embedding and hence $\Mass(\iota_\# R)=\Mass(R)$.
Absorb the constants into $C_X$.
\end{proof}
\end{editamirblockNEW}



\begin{editamirblockNEW}

\begin{proposition}[Microstructure/gluing estimate]\label{prop:glue-gap}
Let $T^{\mathrm{raw}}\in \mathcal I_k(X)$ be the (generally non-closed) integral $k$--current built from the microstructure pieces on a mesh of size $h$.
Set $R:=\partial T^{\mathrm{raw}}\in \mathcal I_{k-1}(X)$ and let $\delta:=\mathcal F(R)$ be the \emph{integral flat norm} from Definition~\ref{def:flat-norm}.
Then there exists an integral $k$--current $R_{\mathrm{glue}}\in\mathcal I_k(X)$ with
\[
\partial R_{\mathrm{glue}}=-R,\qquad
\Mass(R_{\mathrm{glue}})\le \delta + C_X\,\delta^{\frac{k}{k-1}},
\]
where $C_X>0$ depends only on $X$ (and $k$).
Equivalently, $U_{\mathrm{glue}}:=-R_{\mathrm{glue}}$ satisfies $\partial U_{\mathrm{glue}}=R$ and the same mass bound.
\end{proposition}

\begin{proof}
Fix $\eta>0$. By Definition~\ref{def:flat-norm} choose integral currents $R_0\in\mathcal I_{k-1}(X)$ and $Q\in\mathcal I_k(X)$ such that
\[
R=R_0+\partial Q,\qquad \Mass(R_0)+\Mass(Q)\le \delta+\eta .
\]
Then $\partial R_0=\partial R-\partial^2Q=0$, so $R_0$ is an integral $(k-1)$--cycle. Moreover $R_0$ bounds in $X$ since
\[
R_0=\partial(T^{\mathrm{raw}}+Q).
\]
Apply Lemma~\ref{lem:FF-filling-X} to $R_0$ to obtain an integral $k$--current $Q_0\in\mathcal I_k(X)$ with
\[
\partial Q_0 = R_0,\qquad \Mass(Q_0)\le C_X\,\Mass(R_0)^{\frac{k}{k-1}}.
\]
Define
\[
R_{\mathrm{glue}}:=-(Q+Q_0).
\]
Then $\partial R_{\mathrm{glue}}=-(\partial Q+\partial Q_0)=-(R-R_0+R_0)=-R$, and
\[
\Mass(R_{\mathrm{glue}})\le \Mass(Q)+\Mass(Q_0)
\le \Mass(Q)+C_X\,\Mass(R_0)^{\frac{k}{k-1}}
\le (\delta+\eta)+C_X\,(\delta+\eta)^{\frac{k}{k-1}}.
\]
Letting $\eta\downarrow 0$ yields the claimed bound.
\end{proof}
\end{editamirblockNEW}

\begin{editamirblockNEW}
\begin{remark}[Choosing the glue scale to make the correction negligible]\label{rem:glue-scaling}
Let $k=2n-2p$ and set $\delta:=\mathcal F(\partial T^{\mathrm{raw}})$.
Proposition~\ref{prop:glue-gap} yields
\[
\Mass(R_{\mathrm{glue}})\ \le\ \delta\ +\ C_X\,\delta^{\frac{k}{k-1}}.
\]
Hence $\Mass(R_{\mathrm{glue}})=o(m)$ whenever $\delta=o\!\left(m^{\frac{k-1}{k}}\right)$.
In the regime where Lemma~\ref{lem:flatnorm-gluing-mismatch} gives $\delta\le \varepsilon_{\mathrm{glue}}(m,\delta,\varepsilon,\mathrm{mesh})\cdot m$,
it is enough to choose parameters so that $\varepsilon_{\mathrm{glue}}(m,\delta,\varepsilon,\mathrm{mesh})=o\!\left(m^{-1/k}\right)$.
\end{remark}
\end{editamirblockNEW}

\end{editamirblock}





We now return to the global construction.
Fix $\varepsilon>0$, and choose the mesh/activation parameters so that the gluing correction $R_{\mathrm{glue}}$ from
Proposition~\ref{prop:glue-gap} satisfies $\Mass(R_{\mathrm{glue}})\le\varepsilon/2$.
Define the closed glued cycle
\[
T^{(1)}:=T^{\mathrm{raw}}+R_{\mathrm{glue}}.
\]
Then $T^{(1)}$ is closed and integral.

\medskip\noindent
\textbf{Substep 4.3: Forcing the cohomology class via lattice discreteness.}
Fix harmonic $(2n-2p)$-forms $\{\eta_\ell\}_{\ell=1}^b$ whose cohomology classes form an integral basis of the free part
$H^{2n-2p}(X,\Z)/\mathrm{tors}$.
These harmonic representatives detect only the free part of integral cohomology, hence the period computation determines the class in
$H_{2n-2p}(X,\Z)/\mathrm{tors}$.
If one wants an equality in full integral homology, let $m_{\mathrm{tors}}$ be the exponent of the torsion subgroup of $H_{2n-2p}(X,\Z)$ and replace
$(m,T^{(1)})$ by $(m_{\mathrm{tors}}m,\ m_{\mathrm{tors}}T^{(1)})$ (and correspondingly shrink the target $\varepsilon$), which kills any possible torsion discrepancy.
The homology class of any closed integral current $T$ is determined (up to torsion) by the pairings
\[
\langle[T],[\eta_\ell]\rangle=\int_T\eta_\ell.
\]
Since $[\gamma]$ is rational, for each integral cohomology generator $\eta_\ell$
the period
\[
I_\ell:=\int_X \beta\wedge \eta_\ell\in\Q
\]
has bounded denominator.  Choose $m\ge 1$ so that $m\,I_\ell\in\Z$ for all $\ell$.

\begin{lemma}[Fixed-dimension discrepancy rounding (B\'ar\'any--Grinberg)]\label{lem:barany-grinberg}
Let $d\ge 1$ and let $v_1,\dots,v_M\in\R^d$ satisfy $\|v_i\|_{\ell^\infty}\le 1$.
For any coefficients $a_1,\dots,a_M\in[0,1]$, there exist $\varepsilon_1,\dots,\varepsilon_M\in\{0,1\}$ such that
\[
\Bigl\|\sum_{i=1}^M (\varepsilon_i-a_i)\,v_i\Bigr\|_{\ell^\infty}\ \le\ d.
\]
\end{lemma}


\begin{proof}
Set $x:=\sum_{i=1}^M a_i v_i\in\R^d$ and let $V$ be the $d\times M$ matrix whose $i$th column is $v_i$.
Consider the (nonempty) polytope
\[
P\ :=\ \bigl\{\,t\in[0,1]^M:\ Vt=x\,\bigr\},
\]
which contains $a:=(a_1,\dots,a_M)$.  Choose an extreme point $t^*\in P$.
Let $F:=\{\,i:\ 0<t_i^*<1\,\}$ be the set of fractional coordinates.

Write $r:=\mathrm{rank}(V)\le d$.  The affine constraints $Vt=x$ impose $r$ independent linear equalities.
At an extreme point of $P$, at least $M$ linearly independent constraints are active; at most $r$ of them come from $Vt=x$,
so at least $M-r$ of the box constraints $t_i=0$ or $t_i=1$ must be active.
Hence $|F|\le r\le d$.

Now define $\varepsilon_i:=t_i^*$ for $i\notin F$ (so $\varepsilon_i\in\{0,1\}$) and choose any $\varepsilon_i\in\{0,1\}$ for $i\in F$.
Since $Vt^*=Va$, we have
\[
\sum_{i=1}^M (\varepsilon_i-a_i)v_i
\;=\;
\sum_{i=1}^M (\varepsilon_i-t_i^*)v_i,
\]
and only indices in $F$ contribute on the right-hand side.
For each coordinate $1\le j\le d$,
\[
\Bigl|\sum_{i=1}^M (\varepsilon_i-t_i^*)\,v_{i,j}\Bigr|
\;\le\;
\sum_{i\in F} |\varepsilon_i-t_i^*|\,|v_{i,j}|
\;\le\;
\sum_{i\in F} 1
\;=\;
|F|
\;\le\;
d,
\]
because $|\varepsilon_i-t_i^*|\le 1$ and $\|v_i\|_{\ell^\infty}\le 1$.
Taking the maximum over $j$ gives the claimed $\ell^\infty$ bound.
\end{proof}

\begin{remark}
Lemma~\ref{lem:barany-grinberg} is a standard “rounding in fixed dimension” discrepancy estimate
(see B\'ar\'any--Grinberg, \emph{On some combinatorial questions in finite-dimensional vector spaces}, 1981).
The key feature is that the bound depends only on the dimension $d$, not on $M$.
\end{remark}

By refining the cube decomposition (so each individual sheet piece has very small contribution
to each pairing) and choosing the integers $N_{Q,j}$ using Lemma~\ref{lem:barany-grinberg}
(applied to the fractional parts of the target real counts), one can ensure that for all $\ell$,
\[
\Bigl|\int_{T^{\mathrm{raw}}}\eta_\ell - m\,I_\ell\Bigr|<\tfrac12.
\]
Moreover, the gluing correction $R_{\mathrm{glue}}$ has arbitrarily small mass (Proposition~\ref{prop:glue-gap}), hence
its pairing with each fixed smooth $\eta_\ell$ is arbitrarily small:
$\bigl|\int_{R_{\mathrm{glue}}}\eta_\ell\bigr|\le \|\eta_\ell\|_{C^0}\Mass(R_{\mathrm{glue}})$.
Choosing parameters so that this error is $<\tfrac12$ as well yields
\[
\Bigl|\int_{T^{(1)}}\eta_\ell - m\,I_\ell\Bigr|<1,
\qquad T^{(1)}=T^{\mathrm{raw}}+R_{\mathrm{glue}}.
\]
Since $\int_{T^{(1)}}\eta_\ell\in\Z$ (integral current against an integral class),
we conclude $\int_{T^{(1)}}\eta_\ell = m\,I_\ell$ for all $\ell$.
Hence
\[
[T^{(1)}]=\mathrm{PD}(m[\gamma]).
\]

Set $R_\varepsilon:=R_{\mathrm{glue}}$ and $T_\varepsilon:=T^{(1)}$.  This satisfies all requirements.
\end{proof}

Let $\{\Theta_\ell\}_{\ell=1}^{b}$ be a fixed integral basis of
$H^{2(n-p)}(X,\Z)$ represented by smooth closed forms.  Since $\beta$
represents $[\gamma]$, we have for every $\ell$,
\[
I_\ell := \int_X \beta\wedge \Theta_\ell
= \langle [\gamma], [\Theta_\ell]\rangle \in \Q.
\]
Choose a common positive integer multiplier $m=m(\gamma)$ so that
$m\,I_\ell\in\Z$ for all $\ell$.

On each cube $Q$, the current $S_Q$ constructed above satisfies, for
each $\ell$,
\[
S_Q(\Theta_\ell)
= \sum_{j,a} \int_{Y_{Q,j}^a\cap Q} \Theta_\ell
= \int_Q \Bigl(\sum_{j}\tfrac{N_{Q,j}}{m_Q}\,\xi_{\Pi_{Q,j}}\Bigr)
  \wedge \Theta_\ell + O(\eta_Q),
\]
with $\eta_Q\to 0$ as $\varepsilon,\delta\to 0$.  Summing over all cubes yields
\[
\sum_Q S_Q(\Theta_\ell)
= \int_X \beta\wedge \Theta_\ell + O\Bigl(\sum_Q \eta_Q\Bigr).
\]

\begin{editjonblock}
\begin{editamirblockNEW}
\begin{lemma}[Integral periods of integral cycles]\label{lem:integral-periods}
Let $X$ be a compact manifold and let $T$ be a closed integral $k$--cycle (equivalently, an integral $k$--current with $\partial T=0$).
Let $\eta$ be a smooth closed $k$--form whose de~Rham cohomology class lies in the image of
$H^{k}(X,\Z)\to H^{k}(X,\R)$ (i.e.\ $[\eta]\in H^{k}(X,\Z)$ is an integral class).
Then
\[
\int_{T}\eta \;=\; T(\eta)\ \in\ \Z.
\]
\end{lemma}

\begin{proof}
A closed integral current $T$ determines an integral homology class $[T]\in H_{k}(X,\Z)$ (Federer--Fleming).
An integral cohomology class $[\eta]\in H^{k}(X,\Z)$ defines an integer-valued homomorphism on $H_{k}(X,\Z)$ via the Kronecker pairing,
so $\langle [\eta],[T]\rangle\in\Z$.
Under the de~Rham isomorphism, this pairing is represented by integration of a smooth closed form representative, hence
$\langle [\eta],[T]\rangle=\int_{T}\eta$.
\end{proof}
\end{editamirblockNEW}

\begin{editamirblockNEW}
\begin{lemma}[Lattice discreteness]\label{lem:lattice-discreteness}
Let $z\in\Z$ and $r\in\R$ satisfy $|z-r|<\tfrac12$.  Then $z$ is the unique integer within distance $\tfrac12$ of $r$.
In particular, if $c\in\Z$ and $|z-c|<\tfrac12$, then $z=c$.
Consequently, if $\int_T\eta\in\Z$ and $c\in\Z$ satisfy $|\int_T\eta-c|<\tfrac12$, then $\int_T\eta=c$.
\end{lemma}

\begin{proof}
If $z\neq c$ are integers then $|z-c|\ge 1$.  Thus no real number can lie within distance $\tfrac12$ of two distinct integers.
\end{proof}
\end{editamirblockNEW}
\end{editjonblock}

\begin{proposition}[Integral cohomology constraints]\label{prop:cohomology-match}
Given $\epsilon>0$, by refining the cube decomposition and choosing the
integers $N_{Q,j}$ appropriately, one can achieve simultaneously for all
$\ell=1,\ldots,b$ that
\[
\biggl|\sum_Q S_Q(\Theta_\ell) - m\,I_\ell\biggr| < \tfrac14.
\]
Let $S:=\sum_Q S_Q$ and let $U_\epsilon$ be any integral $(2n-2p)$--current with $\partial U_\epsilon=\partial S$ and
\[
\Mass(U_\epsilon)\ <\ \min\Bigl\{\epsilon,\ \frac{1}{4\,\max_\ell\|\Theta_\ell\|_{C^0}}\Bigr\}.
\]
Then $T_\epsilon:=S-U_\epsilon$ is a closed integral cycle and
\[
\int_{T_\epsilon}\Theta_\ell\ =\ m\,I_\ell\qquad\text{for all }\ell=1,\dots,b.
\]
\editamir{(Here $S_Q:=\sum_{j=1}^{N_{\mathrm{Car}}}\sum_{a=1}^{N_{Q,j}}[Y_{Q,j}^a]\llcorner Q$ is the local integral current built from the sheet pieces, and $S_Q(\Theta_\ell):=\int_{S_Q}\Theta_\ell=\sum_{j,a}\int_{Y_{Q,j}^a\cap Q}\Theta_\ell$.)}

In particular, $[T_\epsilon]=\mathrm{PD}(m[\gamma])$ in $H_{2(n-p)}(X,\Z)/\mathrm{tors}$ (equivalently in $H_{2(n-p)}(X,\Q)$).
\end{proposition}

\editai{(AI note: this is the “period locking” hinge. A clean referee check is to re-derive the \(\tfrac14+\tfrac14<\tfrac12\) budget: (i) mesh refinement makes each marginal sheet contribution \(v_{Q,j}\) small enough for B\'ar\'any--Grinberg in fixed dimension \(b\); (ii) the chosen filling \(U_\epsilon\) satisfies \(|\int_{U_\epsilon}\Theta_\ell|<\tfrac14\) uniformly in \(\ell\) from \(\Mass(U_\epsilon)\cdot\|\Theta_\ell\|_{C^0}\); (iii) torsion is handled consistently with the intended identification of \([T_\epsilon]\) with \(\mathrm{PD}(m[\gamma])\).)}

\begin{proof}
We make the fixed-dimension rounding in Substep~4.3 explicit.

\smallskip\noindent
\textbf{Step 1: Real targets and base--marginal decomposition.}
\begin{editamirblockNEW}
Fix a fine cube decomposition $\{Q\}$ (mesh $h$) and the associated families of sheet pieces
$\{Y_{Q,j}^a\}_{a\ge 1}$ produced in the preceding prefix--template construction.
For each pair $(Q,j)$ let $n_{Q,j}\in\R_{\ge 0}$ denote the \emph{real} target sheet count coming from the local bookkeeping.
Write
\[
n_{Q,j}=B_{Q,j}+a_{Q,j},\qquad  B_{Q,j}:=\lfloor n_{Q,j}\rfloor\in\Z_{\ge 0},\quad a_{Q,j}\in[0,1).
\]
Define the \emph{base} (integral) current and the \emph{marginal} sheet-current by
\[
S^{0}:=\sum_{Q,j}\sum_{a=1}^{B_{Q,j}}[Y_{Q,j}^a]\llcorner Q,
\qquad
Z_{Q,j}:=[Y_{Q,j}^{B_{Q,j}+1}]\llcorner Q.
\]
(If $a_{Q,j}=0$ we may set $Z_{Q,j}:=0$; then it plays no role in the fractional combination below.)
For any choice $\varepsilon_{Q,j}\in\{0,1\}$ set
\[
S(\varepsilon):=S^{0}+\sum_{Q,j}\varepsilon_{Q,j}\,Z_{Q,j},
\qquad
N_{Q,j}:=B_{Q,j}+\varepsilon_{Q,j}.
\]
Then $S(\varepsilon)$ is exactly the current obtained by taking the prefix of length $N_{Q,j}$ in each family.
The corresponding \emph{fractional} (real) combination is
\[
S^{\mathrm{frac}}:=S^{0}+\sum_{Q,j}a_{Q,j}\,Z_{Q,j}.
\]
Thus the rounding problem is to choose $\varepsilon_{Q,j}\in\{0,1\}$ so that the period error
$\int_{S(\varepsilon)}\Theta_\ell-\int_{S^{\mathrm{frac}}}\Theta_\ell$ is uniformly small for all $\ell$.
\end{editamirblockNEW}

\smallskip\noindent
\textbf{Step 2: Set up the rounding vectors.}
\begin{editamirblockNEW}
For each $(Q,j)$ define the \emph{marginal contribution vector}
\[
v_{Q,j}:=\Bigl(\int_{Z_{Q,j}}\Theta_1,\ \dots,\ \int_{Z_{Q,j}}\Theta_b\Bigr)\in\R^b.
\]
By Theorem~\ref{thm:local-sheets}, each marginal sheet $Y_{Q,j}^{B_{Q,j}+1}\cap Q$ is a $C^1$ graph over its template plane on a region containing $Q$,
with slope $\lesssim \varepsilon$ and uniform $C^1$ control. In particular, there is a constant $C_0$ such that
\[
\Mass(Z_{Q,j})\le C_0\,h^{2(n-p)}
\qquad\text{and hence}\qquad
\|v_{Q,j}\|_{\ell^\infty}\le C_0\,h^{2(n-p)}\cdot \max_{\ell}\|\Theta_\ell\|_{C^0}.
\]
Choosing the mesh $h$ small (depending on $\max_\ell\|\Theta_\ell\|_{C^0}$ and $b$) we may assume
\[
\|v_{Q,j}\|_{\ell^\infty}\le \frac{1}{8b}\qquad\text{for all }(Q,j).
\]
\end{editamirblockNEW}

\smallskip\noindent
\textbf{Step 3: Apply B\'ar\'any--Grinberg.}
Apply Lemma~\ref{lem:barany-grinberg} in dimension $d=b$ to the normalized vectors
$\widetilde v_{Q,j}:=(8b)\,v_{Q,j}$ (so $\|\widetilde v_{Q,j}\|_{\ell^\infty}\le 1$) with coefficients $a_{Q,j}$.
This yields choices $\varepsilon_{Q,j}\in\{0,1\}$ such that
\[
\Bigl\|\sum_{Q,j}(\varepsilon_{Q,j}-a_{Q,j})\,\widetilde v_{Q,j}\Bigr\|_{\ell^\infty}\le b.
\]
Undoing the normalization gives
\[
\Bigl\|\sum_{Q,j}(\varepsilon_{Q,j}-a_{Q,j})\,v_{Q,j}\Bigr\|_{\ell^\infty}\le \frac18.
\]
\begin{editamirblockNEW}
Equivalently, for each $\ell=1,\dots,b$,
\[
\Bigl|\int_{S(\varepsilon)}\Theta_\ell-\int_{S^{\mathrm{frac}}}\Theta_\ell\Bigr|
=\Bigl|\sum_{Q,j}(\varepsilon_{Q,j}-a_{Q,j})\int_{Z_{Q,j}}\Theta_\ell\Bigr|
\le \frac18.
\]
It therefore suffices to choose the continuous targets $\{n_{Q,j}\}$ (equivalently the fractional current $S^{\mathrm{frac}}$) so that
\[
\Bigl|\int_{S^{\mathrm{frac}}}\Theta_\ell - m I_\ell\Bigr|<\frac18\qquad\text{for all }\ell,
\]
which is the quantitative period-matching output of the local Carath{\'e}odory decomposition of $m\beta$
(Lemma~\ref{lem:caratheodory-general} together with the error bounds in the preceding construction, obtained by taking $\delta$ and $h$ sufficiently small).
Combining the two inequalities yields
\[
\Bigl|\sum_Q S_Q(\Theta_\ell)-mI_\ell\Bigr|
=\Bigl|\int_{S(\varepsilon)}\Theta_\ell- m I_\ell\Bigr|
<\frac14
\qquad (\ell=1,\dots,b).
\]
Set $S:=S(\varepsilon)=\sum_Q S_Q$.
\end{editamirblockNEW}

\smallskip\noindent
\begin{editamirblockNEW}
\textbf{Step 4: Lock the periods via a small boundary correction.}
Choose an integral $(2n-2p)$--current $U_\epsilon$ with $\partial U_\epsilon=\partial S$ and
\[
\Mass(U_\epsilon)<\min\Bigl\{\epsilon,\ \frac{1}{4\max_\ell\|\Theta_\ell\|_{C^0}}\Bigr\},
\]
so that $|\int_{U_\epsilon}\Theta_\ell|<\frac14$ for all $\ell$.
(Existence of such $U_\epsilon$ is established in \emph{Step~5} below, using Proposition~\ref{prop:glue-gap}
and Corollary~\ref{cor:global-flat-weighted}.)

\smallskip\noindent
\textbf{Step 5: (See the next subsection.)}
Then $T_\epsilon:=S-U_\epsilon$ is a closed integral cycle, hence by Lemma~\ref{lem:integral-periods} each $\int_{T_\epsilon}\Theta_\ell\in\Z$.
For each $\ell$ we have
\[
\int_{T_\epsilon}\Theta_\ell=\int_{S}\Theta_\ell-\int_{U_\epsilon}\Theta_\ell,
\]
so the previous estimate implies $\bigl|\int_{T_\epsilon}\Theta_\ell-mI_\ell\bigr|<\frac12$.
By Lemma~\ref{lem:lattice-discreteness} it follows that $\int_{T_\epsilon}\Theta_\ell=mI_\ell$ for all $\ell$.
This identifies the Poincar\'e dual class of $T_\epsilon$ with $m\gamma$ (modulo torsion), as claimed.
\end{editamirblockNEW}
\end{proof}


% ------------------------------------------------------------
\subsection*{Step 5: Boundary correction with vanishing mass}

The sum $S:=\sum_Q S_Q$ is supported in the union of cubes and typically
has a boundary supported on the inter-cube faces.
By the microstructure/gluing estimate established in Proposition~\ref{prop:glue-gap}
(i.e.\ a quantitative bound forcing $\mathcal F(\partial S)\to 0$ as the local errors $\delta,\varepsilon\to 0$ and the mesh size $\to 0$).

\begin{editamirblockNEW}
Write $k:=2n-2p$ and $\delta:=\mathcal F(\partial S)$.
By Definition~\ref{def:flat-norm}, for any $\eta>0$ there exist \emph{integral} currents
$R$ (a $(k-1)$--current) and $Q$ (a $k$--current) in $X$ such that
\[
\partial S\ =\ R+\partial Q,
\qquad
\Mass(R)+\Mass(Q)\ \le\ \delta+\eta.
\]
\editamir{Taking boundaries gives $\partial R=\partial^2S-\partial^2Q=0$, so $R$ is an integral cycle.  Moreover $R=\partial(S-Q)$, hence $R$ bounds in $X$ by an \emph{integral} $k$--current.}
Applying Lemma~\ref{lem:FF-filling-X} yields an integral $k$--current $Q_R$ with $\partial Q_R=R$ and
\[
\Mass(Q_R)\ \le\ C_X\,\Mass(R)^{\frac{k}{k-1}}.
\]
Define
\[
U_\epsilon:=-(Q+Q_R).
\]
Then $\partial U_\epsilon=\partial S$ and
\[
\Mass(U_\epsilon)\ \le\ \Mass(Q)+\Mass(Q_R)
\ \le\ (\delta+\eta)\ +\ C_X\,(\delta+\eta)^{\frac{k}{k-1}}.
\]
Letting $\eta\downarrow 0$ gives $\Mass(U_\epsilon)\le \delta + C_X\,\delta^{\frac{k}{k-1}}$.
Therefore, once $\delta$ is arranged small enough (using Corollary~\ref{cor:global-flat-weighted} and, if $p=n/2$,
Lemma~\ref{lem:borderline-p-half}), the bound
\[
\Mass(U_\epsilon)\ <\ \min\Bigl\{\epsilon,\ \frac{1}{4\,\max_\ell\|\Theta_\ell\|_{C^0}}\Bigr\}
\]
required in Proposition~\ref{prop:cohomology-match} holds.
\end{editamirblockNEW}

\begin{editamirblockNEW}
\[
\Mass(T_\epsilon)
\le \Mass(S) + \Mass(U_\epsilon)
\to m\int_X \beta\wedge \psi,
\]
\end{editamirblockNEW}
since $\Mass(U_\epsilon)\to 0$.


\begin{editamirblock}
\noindent\textbf{Referee bridge (Step 21 $\rightarrow$ Step 22 $\rightarrow$ Proposition~\ref{prop:almost-calibration}).}
Let $T^{\mathrm{raw}}=T^{\mathrm{raw}}_h$ be the raw mesh current.
By Proposition~\ref{prop:glue-gap} there is a correction $R_{\mathrm{glue}}=R_{\mathrm{glue}}(h)$ with
$\partial R_{\mathrm{glue}}=-\partial T^{\mathrm{raw}}$ and $\Mass(R_{\mathrm{glue}})\to 0$.
Set $U_h:=-R_{\mathrm{glue}}$ and $T_h:=T^{\mathrm{raw}}-U_h=T^{\mathrm{raw}}+R_{\mathrm{glue}}$; then $\partial T_h=0$ and $\Mass(U_h)\to 0$.
Moreover Proposition~\ref{prop:cohomology-match} gives the required period equalities, hence $[T_h]=\mathrm{PD}(m[\gamma])$
(mod torsion), so the hypotheses of Proposition~\ref{prop:almost-calibration} are met with $\epsilon=h$.
\end{editamirblock}

\begin{editaiblock}
\begin{proposition}[Almost--calibration and global mass convergence for the glued cycles]\label{prop:almost-calibration}
Let $\psi$ be a smooth closed $(2n-2p)$--form with comass $\le 1$.
Let $(S_\epsilon)_{\epsilon>0}$ be a family of integral $(2n-2p)$--currents built as finite sums of local $\psi$--calibrated sheet pieces
(so $\Mass(S_\epsilon)=\int_{S_\epsilon}\psi$ for each $\epsilon$).
Let $U_\epsilon$ be integral currents such that
\[
\partial U_\epsilon=\partial S_\epsilon,
\qquad
\Mass(U_\epsilon)\xrightarrow[\epsilon\to 0]{}0,
\]
for instance the gluing corrections $U_h$ constructed in Proposition~\ref{prop:glue-gap} (with $\epsilon\sim h$).
Define the closed integral cycles
\[
T_\epsilon := S_\epsilon-U_\epsilon,
\qquad
\partial T_\epsilon=0.
\]
Assume moreover that
\[
[T_\epsilon]=\mathrm{PD}(m[\gamma])\quad\text{in }H_{2(n-p)}(X,\Z)/\mathrm{tors}\ \text{for all }\epsilon
\]
(for instance by Proposition~\ref{prop:cohomology-match}).

Then:
\begin{enumerate}
\item[\textnormal{(i)}] \textbf{Exact cohomological pairing.}
Since $d\psi=0$ and $[T_\epsilon]$ is fixed, the number
\[
\int_{T_\epsilon}\psi
=\bigl\langle [T_\epsilon],[\psi]\bigr\rangle
=\bigl\langle \mathrm{PD}(m[\gamma]),[\psi]\bigr\rangle
=:c_0
\]
is independent of $\epsilon$.
\item[\textnormal{(ii)}] \textbf{Almost--calibration.}
Writing the calibration defect
\[
\Def_{\mathrm{cal}}(T_\epsilon)\ :=\ \Mass(T_\epsilon)-\int_{T_\epsilon}\psi\ \ge\ 0,
\]
one has the explicit estimate
\[
0\ \le\ \Def_{\mathrm{cal}}(T_\epsilon)\ \le\ 2\,\Mass(U_\epsilon)\ \xrightarrow[\epsilon\to 0]{}\ 0.
\]
\item[\textnormal{(iii)}] \textbf{Mass convergence.}
In particular,
\[
c_0\ \le\ \Mass(T_\epsilon)\ \le\ c_0+2\,\Mass(U_\epsilon),
\qquad\text{so}\qquad
\Mass(T_\epsilon)\to c_0.
\]
\end{enumerate}
\end{proposition}

\begin{proof}
By the comass bound, $|\int_{U_\epsilon}\psi|\le \Mass(U_\epsilon)$ and $\int_{T_\epsilon}\psi\le \Mass(T_\epsilon)$.
Since each sheet piece of $S_\epsilon$ is $\psi$--calibrated, $\Mass(S_\epsilon)=\int_{S_\epsilon}\psi$.

For \textnormal{(i)}, $d\psi=0$ and the homology hypothesis give
$\int_{T_\epsilon}\psi=\langle[T_\epsilon],[\psi]\rangle=\langle\mathrm{PD}(m[\gamma]),[\psi]\rangle=:c_0$.

For \textnormal{(ii)}, write
\[
\Def_{\mathrm{cal}}(T_\epsilon)
=\Mass(T_\epsilon)-\int_{S_\epsilon}\psi+\int_{U_\epsilon}\psi
\le \bigl(\Mass(S_\epsilon)+\Mass(U_\epsilon)\bigr)-\Mass(S_\epsilon)+\bigl|\int_{U_\epsilon}\psi\bigr|
\le 2\,\Mass(U_\epsilon),
\]
using $\Mass(T_\epsilon)\le \Mass(S_\epsilon)+\Mass(U_\epsilon)$ and $|\int_{U_\epsilon}\psi|\le \Mass(U_\epsilon)$.
The lower bound $\Def_{\mathrm{cal}}(T_\epsilon)\ge 0$ is $\int_{T_\epsilon}\psi\le \Mass(T_\epsilon)$.

Finally \textnormal{(iii)} follows from $\Mass(T_\epsilon)=\int_{T_\epsilon}\psi+\Def_{\mathrm{cal}}(T_\epsilon)=c_0+\Def_{\mathrm{cal}}(T_\epsilon)$.
\end{proof}
\end{editaiblock}


\begin{remark}[The correction current need not be positive]\label{rem:correction-not-positive}
The filling currents $U_\epsilon$ (or $R_{\mathrm{glue}}$ in Substep~4.2) are produced by the flat-norm decomposition and the
Federer--Fleming isoperimetric inequality.  They are \editamir{not required} to be $\psi$--calibrated, nor to have any positivity/type property.
This causes no difficulty: the only input used later is the vanishing-mass estimate $\Mass(U_\epsilon)\to 0$.
\editai{By Proposition~\ref{prop:almost-calibration}\textnormal{(ii)}, this forces the calibration defect of $T_\epsilon=S_\epsilon-U_\epsilon$ to vanish, so any subsequential}
limit is $\psi$--calibrated (hence positive of type $(p,p)$ in the Harvey--Lawson sense).
\end{remark}


% ------------------------------------------------------------
\subsection*{Step 6: SYR realization via compactness (Theorem D)}

This step establishes that the limit of the approximating cycles is
$\psi$-calibrated and realizes the SYR property.

\begin{theorem}[SYR Realization]\label{thm:syr-realization}
\begin{editaiblock}
Assume that for each $\varepsilon>0$ we have an integral $(2n-2p)$--current $S_\varepsilon$ built as a finite sum of local $\psi$--calibrated holomorphic sheet pieces (Theorem~\ref{thm:local-sheets}),
and an integral current $U_\varepsilon$ with $\partial U_\varepsilon=\partial S_\varepsilon$ such that $\Mass(U_\varepsilon)\to 0$ as $\varepsilon\to 0$ (Proposition~\ref{prop:glue-gap}).
Set $T_\varepsilon:=S_\varepsilon-U_\varepsilon$, so $\partial T_\varepsilon=0$ and $T_\varepsilon$ is integral.
Assume Proposition~\ref{prop:cohomology-match} so that $[T_\varepsilon]=\mathrm{PD}(m[\gamma])$ in $H_{2n-2p}(X;\R)$ (equivalently in $H_{2n-2p}(X;\Z)/\mathrm{Tor}$),
and Proposition~\ref{prop:almost-calibration} so that $0\le \Mass(T_\varepsilon)-\int_{T_\varepsilon}\psi\le 2\,\Mass(U_\varepsilon)\to 0$.
\end{editaiblock}
the sequence $T_\varepsilon$ has:
\begin{enumerate}
\item[\textnormal{(i)}] $\Mass(T_\varepsilon)\to m\int_X\beta\wedge\psi$;
\item[\textnormal{(ii)}] A subsequential limit $T$ that is $\psi$-calibrated
and represents $\mathrm{PD}(m[\gamma])$\editamir{ in $H_{2n-2p}(X;\R)$ (equivalently in $H_{2n-2p}(X;\Z)/\mathrm{Tor}$).}
\end{enumerate}
In particular, $\beta$ is SYR-realizable in the sense of Definition~\ref{def:syr}.
\end{theorem}

\begin{proof}
The proof proceeds in four substeps.

\medskip\noindent
\textbf{Substep 6.1: Uniform mass bound and homology class.}
\editamir{Set $T_k:=T_{1/k}$ and write $U_{1/k}$ for the gluing correction from Proposition~\ref{prop:glue-gap}.  By Proposition~\ref{prop:almost-calibration} we have}
\begin{editamirblock}
\[
\Mass(T_k)\le \int_{T_k}\psi + 2\,\Mass(U_{1/k})
= \langle[T_k],[\psi]\rangle + 2\,\Mass(U_{1/k})
= m\int_X\beta\wedge\psi + 2\,\Mass(U_{1/k}),
\]
\end{editamirblock}
\editamir{and $\Mass(U_{1/k})\to 0$ as $k\to\infty$.}
\editamir{By construction, $T_k$ is obtained from the raw current $T^{\mathrm{raw}}_{1/k}$ by the gluing correction from Proposition~\ref{prop:glue-gap}:
write $R_{\mathrm{glue}}=R_{\mathrm{glue}}(1/k)$ and set $U_{1/k}:=-R_{\mathrm{glue}}$, so $T_k=T^{\mathrm{raw}}_{1/k}-U_{1/k}=T^{\mathrm{raw}}_{1/k}+R_{\mathrm{glue}}$ and $\partial T_k=0$.
Proposition~\ref{prop:cohomology-match} gives the period equalities against the chosen integral basis $\{\Theta_\ell\}$, hence $[T_k]=\mathrm{PD}(m[\gamma])$ (mod torsion).}
(Equivalently, Proposition~\ref{prop:almost-calibration} isolates this global mass control in the sharper ``almost--calibration'' form
$0\le \Mass(T_k)-\int_{T_k}\psi\le 2\,\Mass(U_{1/k})=o(1)$, together with the exact pairing
$\int_{T_k}\psi=m\int_X\beta\wedge\psi$.)
By the calibration inequality applied to any
cycle $S$ in class $\mathrm{PD}(m[\gamma])$:
\[
\Mass(S)\ge\langle[S],[\psi]\rangle=\langle\mathrm{PD}(m[\gamma]),[\psi]\rangle
=m\int_X\gamma\wedge\psi=m\int_X\beta\wedge\psi.
\]
Thus $\Mass(T_k)\ge m\int_X\beta\wedge\psi-o(1)$ as well.  We conclude:
\begin{itemize}
\item $\sup_k\Mass(T_k)<\infty$;
\item All $T_k$ are cycles: $\partial T_k=0$;
\item Their homology class is constant: $[T_k]=\mathrm{PD}(m[\gamma])$.
\end{itemize}
This is the compactness/normalization needed for Federer--Fleming.

\medskip\noindent
\editai{\textbf{Substep 6.2: Compactness (integral currents).}}
\begin{editaiblock}
By Federer--Fleming compactness for integral currents on the compact manifold $X$, using $\sup_k\Mass(T_k)<\infty$ and $\partial T_k=0$, after passing to a subsequence we obtain $T_k\to T$ in the flat norm.
\end{editaiblock}

\begin{itemize}
\item $T_k\to T$ as integral currents in the flat norm;
\item \editamir{By Lemma~\ref{lem:flat_limit_of_cycles_is_cycle}, since $T_k\to T$ in the flat norm (hence weakly) and $\partial T_k=0$, we have $\partial T=0$; thus $T$ is an integral $(2n-2p)$-cycle;}
\item \editamir{(Homology identification.) For every smooth closed $(2n-2p)$--form $\eta$ on $X$,
\[
\langle T,\eta\rangle=\lim_{k\to\infty}\langle T_k,\eta\rangle
=\bigl\langle \mathrm{PD}(m[\gamma]),[\eta]\bigr\rangle.
\]
Hence $[T]=\mathrm{PD}(m[\gamma])$ in $H_{2n-2p}(X;\R)$ (and therefore in $H_{2n-2p}(X;\Q)$ after quotienting by torsion).}
\end{itemize}
Lower semicontinuity gives
\begin{equation}\label{eq:mass-lsc}
\Mass(T)\le\liminf_{k\to\infty}\Mass(T_k)\le m\int_X\beta\wedge\psi.
\end{equation}

\medskip\noindent
\textbf{Substep 6.3 (optional): tangent-plane viewpoint (not used below).}
\editamir{One can interpret $\Def_{\mathrm{cal}}(T_k)$ as measuring, in an averaged sense, how far the approximate tangent planes of $T_k$ are from the calibrated Grassmannian determined by $\psi$.  This can be formalized using standard tangent-measure/varifold language, but it is not used in the argument below.}
\iffalse
For each $k$, the tangent planes of $T_k$ around $x$ induce a probability
measure $\nu_x^{(k)}$ on $\Gr_{n-p}(T_xX)$, where $\mu_k$ denotes the mass measure of $T_k$.

\medskip\noindent
\emph{Calibration deficit forces concentration on calibrated planes.}
Since $[T_k]=\mathrm{PD}(m[\gamma])$ and $\psi$ is closed, the cohomological pairing gives
\[
\int_{T_k}\psi=\langle[T_k],[\psi]\rangle=\langle\mathrm{PD}(m[\gamma]),[\psi]\rangle
=m\int_X\beta\wedge\psi.
\]
By Proposition~\ref{prop:almost-calibration}\textnormal{(ii)}, the calibration deficit
\[
\Def_{\mathrm{cal}}(T_k):=\Mass(T_k)-\int_{T_k}\psi
\]
satisfies $\Def_{\mathrm{cal}}(T_k)\to 0$.
Equivalently (writing $V_k$ for the associated integral varifold),
\[
\Def_{\mathrm{cal}}(T_k)=\int_{X\times \Gr_{n-p}(TX)}\bigl(1-\psi(P)\bigr)\,dV_k(x,P)
=\int_X\int_{\Gr_{n-p}(T_xX)}\bigl(1-\psi(P)\bigr)\,d\nu_x^{(k)}(P)\,d\mu_k(x)\ \to\ 0.
\]
By the Wirtinger/K\"ahler-angle comparison (cf.\ the pointwise estimate
$1-\psi(P)\asymp \mathrm{dist}\bigl(P,K_{n-p}(x)\bigr)^2$ on the Grassmannian),
it follows that
\[
\int_X\int \mathrm{dist}\!\bigl(P,K_{n-p}(x)\bigr)^2\,d\nu_x^{(k)}(P)\,d\mu_k(x)\ \to\ 0.
\]

\medskip\noindent
\emph{Remark (optional barycenter bookkeeping).}
The microstructure construction can be organized to track local plane-mixtures at the mesh scale via Young-measure language, but this bookkeeping is not needed for the calibrated-limit closure and is not used in the Hodge argument.  In particular, we do \emph{not} require any pointwise barycenter identity for the calibrated limit current.
\fi

\medskip\noindent
\textbf{Substep 6.4: Calibration of the limit.}
Since $\psi$ is closed and $[T_k]=\mathrm{PD}(m[\gamma])$, the pairing $\langle T_k,\psi\rangle$ is constant in $k$.
\[
\langle T_k,\psi\rangle=\langle[T_k],[\psi]\rangle=m\int_X\beta\wedge\psi\qquad\text{for all }k.
\]
By weak convergence $T_k\rightharpoonup T$ and closedness of $\psi$, we have
\[
\langle T,\psi\rangle=\lim_{k\to\infty}\langle T_k,\psi\rangle=m\int_X\beta\wedge\psi.
\]
\editamir{By Proposition~\ref{prop:almost-calibration}, the calibration defect satisfies $\Def_{\mathrm{cal}}(T_k)=\Mass(T_k)-\int_{T_k}\psi\le 2\,\Mass(U_{1/k})\to 0$, hence $\Mass(T_k)\to m\int_X\beta\wedge\psi$.}
Combining with \eqref{eq:mass-lsc} and the calibration inequality $\langle T,\psi\rangle\le \Mass(T)$ yields $\Mass(T)=\langle T,\psi\rangle$, so $T$ is $\psi$-calibrated.\editamir{ (This is exactly Lemma~\ref{lem:limit_is_calibrated} in this setting.)}
In particular, $\Mass(T)=m\int_X\beta\wedge\psi$ and $[T]=\mathrm{PD}(m[\gamma])$.

\textbf{Conclusion:} We have established:
\begin{enumerate}
\item Mass convergence / vanishing calibration defect:
$\Mass(T_k)\to m\int_X\beta\wedge\psi$ and $\Def_{\mathrm{cal}}(T_k)\to 0$;
\item Limit cycle: $T$ is an integral $\psi$-calibrated $(2n-2p)$-cycle
with $[T]=\mathrm{PD}(m[\gamma])$.
\end{enumerate}
Thus $\beta$ is SYR-realizable in the sense of Definition~\ref{def:syr}.
\end{proof}

\begin{editamirblock}
\begin{corollary}[SYR limit is a holomorphic (hence algebraic) cycle]\label{cor:syr-limit-holomorphic-chain}
Under the hypotheses of Theorem~\ref{thm:syr-realization}, any subsequential flat limit $T$ is a $\psi$--calibrated integral $(2n-2p)$--cycle representing $\mathrm{PD}(m[\gamma])$ in real homology. In particular, $T$ is a positive holomorphic $(n-p)$--cycle (a holomorphic chain).
If $X$ is projective, then by Chow/GAGA this holomorphic cycle is algebraic, so $\gamma$ is represented by an algebraic cycle with rational coefficients.
\end{corollary}

\begin{proof}
Everything up to ``$\psi$--calibrated integral cycle'' is proved in Theorem~\ref{thm:syr-realization}.  To pass from $\psi$--calibrated to holomorphic,
apply Theorem~\ref{thm:realization-from-almost} (which invokes King~\cite[Thm.~5.2.1]{King71}, with supporting calibrated-geometry background in Harvey--Lawson~\cite{HL82}).  The projective algebraicity statement is Chow/GAGA (e.g.\ Serre~\cite{Serre56}).
\end{proof}
\end{editamirblock}

% ------------------------------------------------------------
\subsection*{Addressing potential objections to the SYR construction}
% ------------------------------------------------------------

We address three potential objections to the construction above.

\begin{remark}[The ``density vs.\ mass'' objection]\label{rem:density-mass}
\textbf{Objection:} ``Integral cycles are supported on measure-zero sets,
while $\beta$ is non-zero everywhere.  To approximate $\beta$ everywhere,
the cycles would need infinite mass.''

\textbf{Response:} This objection rests on a fundamental misunderstanding
of what SYR accomplishes.  The construction does \emph{not} claim that
$T_k$ approximates $\beta$ as a measure on all of $X$.  Rather:
\begin{itemize}
\item Each $T_k$ is an integral $(2n-2p)$-cycle (a rectifiable current), supported on a $(2n-2p)$-dimensional set; in our construction it is a sum of holomorphic pieces plus a small integral filling used to close the boundary.
\item We do not approximate $\beta$ as a measure on all of $X$; what we control is the \emph{fixed homology class} $[T_k]=\mathrm{PD}(m[\gamma])$ and the \emph{calibration defect} $\Def_{\mathrm{cal}}(T_k)=\Mass(T_k)-\langle T_k,\psi\rangle\to 0$.
\item The calibrated limit current $T$ is supported on a $(2n-2p)$-dimensional complex analytic set (Harvey--Lawson), which is exactly the geometric object required by the Hodge conjecture.
\end{itemize}
In particular, the limiting calibrated current is \emph{not} the smooth form $\beta$; $\beta$ is only a design target used to choose local holomorphic sheets and to identify the cohomological lower bound $\langle \mathrm{PD}(m[\gamma]),[\psi]\rangle=m\int_X\beta\wedge\psi$.
\end{remark}

\begin{remark}[Harvey--Lawson applicability]\label{rem:hl-applicable}
\textbf{Objection:} ``The limit $T$ might be a smooth current (integration
against $\beta$), which is not rectifiable, so Harvey--Lawson doesn't apply.''

\textbf{Response:} This objection is factually incorrect.  The sequence
$\{T_k\}$ consists of \emph{integral cycles}.  In the construction, each $T_k$
is obtained from a finite sum of holomorphic complete-intersection pieces (from Theorem~\ref{thm:local-sheets})
by adding/subtracting an integral filling current of vanishing mass to close the boundary (Substep~4.2 and Proposition~\ref{prop:almost-calibration}).
In particular, each $T_k$ is an integral current with integer multiplicities, so Federer--Fleming applies.  By the
\emph{Federer--Fleming compactness theorem} (Federer--Fleming,
``Normal and integral currents,'' Ann.~of Math.~72 (1960), 458--520):
\begin{quote}
\emph{If $\{T_k\}$ is a sequence of integral currents with uniformly
bounded mass and boundary mass, then a subsequence converges in the
flat norm to an integral current $T$.}
\end{quote}
In our case:
\begin{itemize}
\item $\Mass(T_k)\le C$ uniformly (Substep 6.1);
\item $\partial T_k=0$ for all $k$ (they are cycles);
\item Hence the limit $T$ is an \emph{integral} current.
\end{itemize}
Integral currents are rectifiable by definition.  The limit $T$ is
\emph{not} a smooth current; it is a rectifiable current supported on
an $(n-p)$-rectifiable set with integer multiplicities.  Harvey--Lawson
applies to such currents when they are $\psi$-calibrated, which $T$ is.
\end{remark}

\begin{remark}[The gluing/non-integrability objection]\label{rem:gluing}
\textbf{Objection:} ``The plane field $x\mapsto\beta(x)$ is generically
non-integrable.  Local sheets cannot be glued without accumulating mass.''

\textbf{Response:} This objection conflates two different things:
\begin{enumerate}
\item[(a)] \emph{Integrating a plane field} into a single foliation
(which requires the Frobenius condition);
\item[(b)] \emph{Building many separate calibrated sheets} whose tangent
planes locally approximate a given decomposition.
\end{enumerate}
The construction does (b), not (a).  We are \emph{not} trying to find a
submanifold whose tangent planes equal $\beta(x)$ everywhere---that would
indeed require integrability.  Instead:
\begin{itemize}
\item On each cube $Q$, we decompose $\beta(x_Q)$ as a convex combination
of calibrated planes via Carath\'eodory.
\item We build finitely many \emph{separate, disjoint} calibrated
complete intersections through $Q$, each with a \emph{constant} tangent
plane (up to $\varepsilon$-error on the small cube).
\item The complete intersections are algebraic subvarieties---they exist
by Bertini's theorem, regardless of whether $\beta$ is integrable.
\end{itemize}
The non-integrability of $\beta$ as a plane field is irrelevant because
we never integrate it.  The ``gluing'' step (Theorem~\ref{thm:global-cohom},
Substep 4.2) uses Federer--Fleming to fill boundary mismatches.  The
key estimate is formulated in \emph{flat norm}:
\[
\mathcal F\!\left(\partial T^{\mathrm{raw}}\right)\ \le\ \varepsilon_{\mathrm{glue}}(m,\delta,\varepsilon,\mathrm{mesh})\cdot m,
\]
This is the robust target because the individual face mismatches can have large mass even when there is strong cancellation.
\medskip\noindent
Concretely, by the dual characterization of $\mathcal F$ and Stokes, for every smooth
$(2n-2p-1)$-form $\eta$ with $\|\eta\|_{\mathrm{comass}}\le 1$ and $\|d\eta\|_{\mathrm{comass}}\le 1$ one has
\[
\partial T^{\mathrm{raw}}(\eta)=T^{\mathrm{raw}}(d\eta)\approx \int_X (m\beta)\wedge d\eta.
\]
Since $\beta$ is closed and $X$ has no boundary, $\int_X (m\beta)\wedge d\eta=\pm\int_X d(m\beta\wedge\eta)=0$.
Thus the remaining task is to make the approximation error quantitative in terms of
$(\delta,\varepsilon,\mathrm{mesh},m)$; see Proposition~\ref{prop:glue-gap}.
Once $\mathcal F(\partial T^{\mathrm{raw}})$ is small, the correction current $R_{\mathrm{glue}}$ is produced by
the flat-norm decomposition and the Federer--Fleming isoperimetric inequality as in Substep~4.2.
The smoothness of $\beta$ is essential here---it ensures the local
decompositions are compatible across cube boundaries.
\end{remark}

\begin{remark}[Why the construction succeeds]\label{rem:why-success}
The SYR construction succeeds because it exploits three key facts:
\begin{enumerate}
\item \textbf{Algebraic density:} By Bergman kernel asymptotics, any
calibrated plane at any point can be approximated by the tangent plane
of an algebraic complete intersection (Proposition~\ref{prop:tangent-approx-full}).
\item \textbf{Carath\'eodory decomposition:} Any cone-valued form $\beta(x)$
is a finite convex combination of calibrated planes, with uniformly
bounded number of terms (Lemma~\ref{lem:caratheodory-general}).
\item \textbf{Federer--Fleming compactness:} Integral cycles with bounded
mass converge to integral cycles, preserving rectifiability.
\end{enumerate}
The construction builds integral cycles $T_k$ that are finite unions of
raw holomorphic pieces (finite unions of algebraic complete intersections), and then corrects the residual boundary mismatch by integral fillings of vanishing relative mass in flat norm.
Thus the final cycles $T_k$ are integral, but need not themselves be holomorphic/algebraic term-by-term.  The limit $T$ is again an integral current (by
Federer--Fleming), and it is $\psi$-calibrated (by the mass equality
argument in Substep~6.4).  Harvey--Lawson then identifies $T$ as a
positive sum of complex subvarieties.

Critically, the form $\beta$ is \emph{never} the limit current.  The
limit $T$ is an algebraic cycle whose \emph{existence} is guaranteed by
compactness, whose \emph{homology class} is $\mathrm{PD}(m[\gamma])$ by
construction, and whose \emph{calibrated structure} follows from the
mass equality.
\end{remark}

% ------------------------------------------------------------
\subsection*{Automatic SYR: summary theorem}

\begin{theorem}[Automatic SYR for cone-valued forms]\label{thm:automatic-syr}
Let $(X,\omega)$ be a smooth complex projective manifold of complex
dimension $n$, and let $1\le p\le \frac{n}{2}$.
(For $p>\frac{n}{2}$ one reduces to the complementary degree $n-p$ by Hard Lefschetz; see Remark~\ref{rem:lefschetz-reduction}.)
Let $\beta$ be a smooth closed cone--valued $(p,p)$--form representing a rational Hodge class $[\gamma]\in H^{p,p}(X;\Q)$.
Then $\beta$ is SYR--realizable in the sense of Definition~\ref{def:syr}; equivalently,
there exist integral $(2n-2p)$--cycles $T_k$ with $\partial T_k=0$ and
\editamir{$[T_k]=\mathrm{PD}(m[\gamma])$ in $H_{2n-2p}(X;\Z)/\mathrm{Tor}$ (equivalently in $H_{2n-2p}(X;\Q)$)} for some fixed $m\in\N$ independent of $k$, such that
\[
\Def_{\mathrm{cal}}(T_k)=\Mass(T_k)-\langle T_k,\psi\rangle\ \longrightarrow\ 0.
\]
Consequently, $[\gamma]$ is algebraic.
\end{theorem}
\begin{editamirblock}
\begin{proof}
Fix a mesh parameter $\epsilon>0$.
The construction in Proposition~\ref{prop:global-coherence-all-labels} (built from Theorem~\ref{thm:local-sheets} and Proposition~\ref{prop:finite-template}) produces a ``raw'' integral current $S_\epsilon$
built as a finite sum of local $\psi$--calibrated sheets (so $\Mass(S_\epsilon)=\langle S_\epsilon,\psi\rangle$ on each cell),
whose boundary is supported in the gluing region.
Proposition~\ref{prop:glue-gap} provides an integral current $U_\epsilon$ with
\(
\partial U_\epsilon=\partial S_\epsilon
\)
and
\(
\Mass(U_\epsilon)\to 0
\)
as $\epsilon\to 0$.
Define the closed cycle
\(
T_\epsilon:=S_\epsilon-U_\epsilon.
\)

The cohomological bookkeeping (Theorem~\ref{thm:global-cohom}, together with Proposition~\ref{prop:cohomology-match})
implies that for some fixed $m\in\N$ independent of $\epsilon$,
\(
\editamir{[T_\epsilon]=\mathrm{PD}(m[\gamma])\in H_{2n-2p}(X;\Z)/\mathrm{Tor}\quad\text{(equivalently in }H_{2n-2p}(X;\Q)\text{)}.}
\)
Finally, Proposition~\ref{prop:almost-calibration} yields the quantitative almost--calibration estimate
\[
\Def_{\mathrm{cal}}(T_\epsilon)\le 2\,\Mass(U_\epsilon)\ \longrightarrow\ 0.
\]
Choosing any sequence $\epsilon_k\downarrow 0$ and setting $T_k:=T_{\epsilon_k}$ gives the SYR sequence required by
Definition~\ref{def:syr}.
Applying Theorem~\ref{thm:syr} concludes that $[\gamma]$ is represented by a holomorphic chain and, since $X$ is projective (Remark~\ref{rem:chow-gaga}),
is algebraic.
\end{proof}
\end{editamirblock}





% ============================================================
\subsection*{Signed decomposition: the unconditional step}
% ============================================================

The preceding machinery applies to \emph{cone--positive} classes---those admitting
smooth closed cone-valued representatives.  The following lemma shows that \emph{every} rational
Hodge class reduces to this case.

\begin{definition}[Cone--positive class (smooth $K_p$--positive)]
A cohomology class $\gamma \in H^{2p}(X,\R) \cap H^{p,p}(X)$ is called
\emph{cone--positive} if there exists a smooth closed $(p,p)$--form $\beta$
representing $\gamma$ such that $\beta(x) \in K_p(x)$ for all $x \in X$.
(We avoid the word ``effective'' here, which in algebraic geometry refers to \emph{algebraic} cycles with nonnegative coefficients.)
\end{definition}

\begin{lemma}[Strict interior positivity of the K\"ahler power]\label{lem:kahler-positive}
The $(p,p)$--form $\omega^p$ is strictly positive in the calibrated cone: for all $x\in X$,
\[
\omega^p(x)\in \mathrm{int}\,K_p(x).
\]
Moreover, there exists a uniform radius $r_0=r_0(X,\omega,p)>0$ such that for every $x\in X$,
\[
B\bigl(\omega^p(x),\,r_0\bigr)\ \subset\ K_p(x),
\]
where $B(\cdot,r_0)$ denotes the ball in $\Lambda^{p,p}T_x^*X$ for the pointwise metric induced by $\omega$.
\end{lemma}

\begin{editamirblock}
\begin{proof}
Fix $x\in X$ and choose a unitary frame for $(T^{1,0}_xX,\omega_x)$.
In these coordinates, $\omega_x=\frac{i}{2}\sum_{j=1}^n dz^j\wedge d\bar z^j$, hence
$\omega_x^p$ is a strictly (strongly) positive $(p,p)$-form: it is a positive linear
combination of decomposable forms $i^p\,\eta\wedge\bar\eta$ with $\eta\in\Lambda^{p,0}$.
Equivalently, $\omega_x^p$ lies in the interior of the cone of strongly positive $(p,p)$-forms.
To obtain a \emph{uniform} interior radius, define
\[
r(x)\ :=\ \dist\!\Bigl(\omega^p(x),\,\Lambda^{p,p}T_x^*X\setminus K_p(x)\Bigr)
\]
with respect to the pointwise norm induced by $\omega$.  Since $\omega^p(x)\in\mathrm{int}\,K_p(x)$, one has $r(x)>0$ for every $x$.
In a local unitary trivialization of $\Lambda^{p,p}T^*X$, the cone $K_p(x)$ identifies with a fixed model cone (depending only on $(n,p)$),
so $x\mapsto r(x)$ is continuous.  By compactness of $X$,
\(
r_0:=\min_{x\in X} r(x)>0,
\)
and then $B(\omega^p(x),r_0)\subset K_p(x)$ for all $x$.  For background on positivity cones see Harvey--Lawson~\cite{HL82}
or Demailly~\cite{Demailly12}.
\end{proof}
\end{editamirblock}

\begin{lemma}[Signed Decomposition]\label{lem:signed-decomp}
Let $\gamma \in H^{2p}(X,\Q) \cap H^{p,p}(X)$ be any rational Hodge class.
Then there exist cone--positive classes $\gamma^+$ and $\gamma^-$ such that
\[
\gamma \;=\; \gamma^+ - \gamma^-.
\]
Moreover, both $\gamma^+$ and $\gamma^-$ are rational Hodge classes,
and $\gamma^-$ can be taken to be a positive rational multiple of $[\omega^p]$.
\end{lemma}

\begin{proof}
Let $\alpha$ be any smooth closed $(p,p)$--form representing $\gamma$.
\begin{editamirblock}
Let $r_0>0$ be the uniform interior radius from Lemma~\ref{lem:kahler-positive}.
Set
\[
M\ :=\ \sup_{x\in X}\|\alpha(x)\|\ <\ \infty,
\]
finite by compactness of $X$ and smoothness of $\alpha$.
Choose $N\in\Q_{>0}$ with $N> M/r_0$ (possible since $\Q$ is dense in $\R$).
Then for every $x\in X$ we have $\|\alpha(x)/N\|<r_0$, hence
\[
\omega^p(x) + \frac{1}{N}\alpha(x)\ \in\ B\bigl(\omega^p(x),r_0\bigr)\ \subset\ K_p(x).
\]
Since $K_p(x)$ is a cone, multiplying by $N$ yields $\alpha(x)+N\,\omega^p(x)\in K_p(x)$ for all $x$.
\end{editamirblock}

Define $\gamma^+ := \gamma + N \cdot [\omega^p]$ and
$\gamma^- := N \cdot [\omega^p]$.
Then $\gamma = \gamma^+ - \gamma^-$ by construction,
$\gamma^+$ is cone--positive (represented by the cone-valued form
$\alpha + N \cdot \omega^p$),
$\gamma^-$ is cone--positive (represented by $N \cdot \omega^p$),
and both are rational Hodge classes since $[\omega^p] = c_1(L)^p$ is
rational for the ample bundle $L$.
\end{proof}

\begin{editamirblock}
\begin{lemma}[$\omega^p$ is algebraic]\label{lem:gamma-minus-alg}
Let $X$ be a smooth complex projective manifold and fix an ample line bundle $L\to X$
as in the global assumptions, so that $[\omega]=c_1(L)$.
Then the class $[\omega^p]=c_1(L)^p\in H^{2p}(X,\Q)\cap H^{p,p}(X)$ lies in the $\Q$--span
of algebraic cycle classes.  More concretely, for $m\gg 0$ there exists a smooth complete
intersection $Z\subset X$ of codimension $p$, cut out by $p$ generic divisors in the
linear system $|L^{\otimes q}|$, such that
\[
\mathrm{PD}([Z]) \;=\; c_1(L^{\otimes q})^p \;=\; q^p\,[\omega^p].
\]
In particular, any rational multiple of $[\omega^p]$ is an algebraic class.
\end{lemma}

\begin{proof}
Choose $q\gg 0$ so that $L^{\otimes q}$ is very ample and basepoint free.
Let $D_1,\dots,D_p\in |L^{\otimes q}|$ be generic divisors and set
\[
Z \;:=\; D_1\cap\cdots\cap D_p.
\]
By Bertini's theorem, $Z$ is smooth of codimension $p$ for generic choices
(see e.g.\ Hartshorne~\cite{Hartshorne77}).
In cohomology,
\[
\mathrm{PD}([Z]) \;=\; c_1(\mathcal O(D_1))\smile\cdots\smile c_1(\mathcal O(D_p))
\;=\; c_1(L^{\otimes q})^p \;=\; q^p\,c_1(L)^p \;=\; q^p\,[\omega^p],
\]
as claimed.
\end{proof}
\end{editamirblock}

\begin{theorem}[Cone--positive classes are algebraic]\label{thm:effective-algebraic}
Let $\gamma^+ \in H^{2p}(X,\Q) \cap H^{p,p}(X)$ be a cone--positive rational
Hodge class on a smooth complex projective manifold, and assume $p\le n/2$.
Then $\gamma^+$
is algebraic.
\end{theorem}

\begin{proof}
\begin{editamirblock}
Let $\beta$ be a smooth closed cone--valued $(p,p)$--form representing the class $\gamma^+$ (this is the meaning of cone--positivity in the manuscript).
By Theorem~\ref{thm:automatic-syr}, $\beta$ is SYR--realizable in the sense of Definition~\ref{def:syr}; in particular there exists $m\in\N$ and integral cycles $T_k$ with
\editamir{$[T_k]=\mathrm{PD}(m[\gamma^+])$ in $H_{2n-2p}(X;\Z)/\mathrm{Tor}$ (equivalently in $H_{2n-2p}(X;\Q)$)} and $\Def_{\mathrm{cal}}(T_k)\to 0$.
Applying Theorem~\ref{thm:syr} to this SYR data produces a $\psi$--calibrated integral cycle
\(T=\sum_j m_j[V_j]\)
with \editamir{$[T]=\mathrm{PD}(m[\gamma^+])$ in $H_{2n-2p}(X;\Z)/\mathrm{Tor}$ (equivalently in $H_{2n-2p}(X;\Q)$)}, hence an analytic cycle representative of $m[\gamma^+]$.
Since $X$ is projective, each $V_j$ is algebraic by Remark~\ref{rem:chow-gaga}, so $m[\gamma^+]$ is algebraic and therefore $\gamma^+$ is algebraic.
\end{editamirblock}

\end{proof}



\begin{remark}[Chow/GAGA for analytic subvarieties]\label{rem:chow-gaga}
If $X$ is projective, any complex analytic subvariety of $X$ is algebraic.
\editamir{This is a standard consequence of Chow's theorem (for projective space) together with Serre's GAGA.
See, for example, \cite{Hartshorne77,GH78,Serre56}.}
\end{remark}


% ============================================================
\subsection*{Main theorem: Hodge conjecture for rational $(p,p)$ classes}
% ============================================================

\begin{remark}[Convention: algebraic classes]\label{rem:algebraic-class-convention}
By an \emph{algebraic class} in $H^{2p}(X,\Q)$ we mean a class in the $\Q$--span of
cohomology classes of codimension-$p$ algebraic cycles (equivalently, in the image of the
cycle class map with $\Q$--coefficients).  In particular, algebraic classes form a
$\Q$--vector subspace of $H^{2p}(X,\Q)$.
\end{remark}

\begin{theorem}[Hodge Conjecture for rational $(p,p)$ classes]
\label{thm:main-hodge}
Let $X$ be a smooth complex projective manifold.  Then every rational Hodge
class $\gamma \in H^{2p}(X,\Q) \cap H^{p,p}(X)$ is algebraic.
\end{theorem}

\begin{proof}
\begin{editamirblock}
By Remark~\ref{rem:lefschetz-reduction} it suffices to treat the range $p\le n/2$.
Let $\gamma\in H^{2p}(X,\Q)\cap H^{p,p}(X)$.
Apply Lemma~\ref{lem:signed-decomp} to write
$\gamma=\gamma^{+}-\gamma^{-}$ where $\gamma^{+}$ and $\gamma^{-}$ are cone--positive
rational Hodge classes and $\gamma^{-}=N\,[\omega^{p}]$ for some $N\in\Q_{>0}$.
By Lemma~\ref{lem:gamma-minus-alg}, the class $[\omega^{p}]$ (hence $\gamma^{-}$) is algebraic.
By Theorem~\ref{thm:effective-algebraic}, the cone--positive class $\gamma^{+}$ is algebraic.
Since algebraic classes form a $\Q$--vector subspace of $H^{2p}(X,\Q)$, the difference
$\gamma=\gamma^{+}-\gamma^{-}$ is algebraic.
\end{editamirblock}
\end{proof}


\begin{editamirblock}
\editamir{[Editorial note: an older duplicate proof of Theorem~\ref{thm:main-hodge} is disabled below to avoid two proofs in the compiled manuscript.]}
\end{editamirblock}
\iffalse

\begin{proof}
\editp{By Hard Lefschetz (Remark~\ref{rem:lefschetz-reduction}), it suffices to treat the range $p\le n/2$.  We henceforth assume $p\le n/2$.}
By Lemma~\ref{lem:signed-decomp}, write $\gamma = \gamma^+ - \gamma^-$
where $\gamma^+$ and $\gamma^- = N[\omega^p]$ are both cone--positive rational
Hodge classes.

\editamir{By Lemma~\ref{lem:gamma-minus-alg}, $\gamma^-$ is algebraic: it is represented (in $H^{2p}(X,\Q)$) by a rational multiple of a complete intersection $Z^-\subset X$.}

By Theorem~\ref{thm:effective-algebraic}, $\gamma^+$ is algebraic:
it is represented by an algebraic cycle $Z^+$ obtained from the
\editcone{SYR/microstructure construction (Theorem~\ref{thm:automatic-syr}).}

Therefore:
\[
\gamma \;=\; \gamma^+ - \gamma^-
\;=\; [Z^+] - [Z^-],
\]
where $Z^+ - Z^-$ denotes the formal difference in the group of algebraic
cycles tensored with $\Q$.  Hence $\gamma$ is algebraic.
\end{proof}
\fi


\begin{corollary}[Full Hodge conjecture]\label{cor:full-hodge}
	Every rational $(p,p)$ class on a smooth complex projective manifold is represented
	by an algebraic cycle.
\end{corollary}

\begin{proof}
\begin{editamirblock}
Let $\alpha\in H^{2p}(X;\Q)\cap H^{p,p}(X)$.  This is exactly the hypothesis of Theorem~\ref{thm:main-hodge}, which shows that
$\alpha$ is an \emph{algebraic class}.  Equivalently, there exists an algebraic cycle (with rational coefficients) whose cohomology
class equals $\alpha$.  Hence $\alpha$ is represented by an algebraic cycle.
\end{editamirblock}
\end{proof}

\begin{remark}[Why signed decomposition is the key]
The signed decomposition sidesteps the fundamental obstruction that the
harmonic representative $\gamma_{\mathrm{harm}}$ of a general Hodge class
need not be cone-valued.  For classes like $[\pi_1^*\omega_1] - [\pi_2^*\omega_2]$
on a product surface, the harmonic form has indefinite signature everywhere.
We do \emph{not} claim that every Hodge class has a cone-valued representative;
we only use that every Hodge class is a \emph{difference} of two that do.
This is trivially achieved by adding a large multiple of $[\omega^p]$, which
is strictly positive.
\end{remark}

\appendix
\begin{editjonblock}
\section{Referee packet (verification scaffold)}
\label{app:referee-packet}

\subsection*{A. Dependency graph (main chain only)}
\begin{center}
\fbox{\parbox{0.93\textwidth}{
\small
\textbf{Theorem~\ref{thm:main-hodge}}
$\Leftarrow$
Hard Lefschetz reduction (Remark~\ref{rem:lefschetz-reduction})
$\Leftarrow$
Signed decomposition (Lemma~\ref{lem:signed-decomp}) + algebraicity of $\gamma^-$ (Lemma~\ref{lem:gamma-minus-alg})
$\Leftarrow$
Cone--positive $\Rightarrow$ algebraic (Theorem~\ref{thm:effective-algebraic})
$\Leftarrow$
Automatic SYR (Theorem~\ref{thm:automatic-syr})
$\Leftarrow$
Spine theorem (Theorem~\ref{thm:spine-quantitative}) under the global schedule (\S\ref{sec:parameter-schedule}), with
\textbf{(H1)} supplied by Theorem~\ref{thm:local-sheets} (packaged in Proposition~\ref{prop:h1-package}),
\textbf{(H2)} supplied by the corner-exit coherence package (Proposition~\ref{prop:h2-package}, ultimately from Proposition~\ref{prop:global-coherence-all-labels} $\Rightarrow$ Corollary~\ref{cor:global-flat-weighted} $\Rightarrow$ Proposition~\ref{prop:glue-gap}; and in the borderline case $p=n/2$ by Lemma~\ref{lem:borderline-p-half} via Proposition~\ref{prop:integer-transport}),
and exact class enforced by Proposition~\ref{prop:cohomology-match} (using Lemmas~\ref{lem:integral-periods} and \ref{lem:lattice-discreteness});
vanishing defect is Proposition~\ref{prop:almost-calibration}.
$\Leftarrow$
Calibrated-limit closure (Theorem~\ref{thm:realization-from-almost}) + Harvey--Lawson + Chow/GAGA (Remark~\ref{rem:chow-gaga}).
}}
\end{center}

\subsection*{B. Quantifier table (global choices vs.\ scale choices)}
\begin{center}
\fbox{\parbox{0.93\textwidth}{
\small
\textbf{Choose once:} $m\ge 1$ so that $m[\gamma]\in H^{2p}(X,\Z)$ and all integral periods $m\int_X\beta\wedge\Theta_\ell\in\Z$ (Substep~4.3 / Proposition~\ref{prop:cohomology-match}).\par
\textbf{Choose a mesh sequence:} $h_j\downarrow 0$ with rounded cubulation by coordinate cubes $Q$ (size $h_j$).\par
\textbf{Choose local accuracy scales:} $\varepsilon_{\mathrm{net},j}\ll h_j$ (direction dictionary), $\delta_j=o(h_j)$ (transverse grid), $\varepsilon_j=o(1)$ (small-angle tolerance), and a transverse radius factor $\varrho_j\in(0,1]$ (template parameters live in $B_{\varrho_j h_j}$; in the borderline case $p=n/2$ impose $\varrho_j=o(\varepsilon_j)$).\par
\textbf{Choose holomorphic scale:} $N_j\to\infty$ large enough for the local holomorphic manufacturing at tolerance $\varepsilon_j$.\par
\textbf{Choose discrete data at each $j$:} integer activations/prefix lengths satisfying (i) local budgets, (ii) slow variation / face-edit control, and (iii) global period constraints.\par
\textbf{Target inequalities:} $\mathcal F(\partial T^{\mathrm{raw}}_j)\to 0$ $\Rightarrow$ $\Mass(R_{\mathrm{glue},j})\to 0$ (Proposition~\ref{prop:glue-gap}) $\Rightarrow$ $\Def_{\mathrm{cal}}(T_j)\to 0$ (Proposition~\ref{prop:almost-calibration}).}}
\end{center}

\subsection*{C. External theorem ledger (full citations + hypothesis-check bullets)}
\begin{itemize}
\item \textbf{Hard Lefschetz (Hodge index / Lefschetz decomposition).}
	\begin{itemize}
	\item \textbf{Citation:} C.~Voisin, \emph{Hodge Theory and Complex Algebraic Geometry I}, Cambridge (2002), Ch.~6; or D.~Huybrechts, \emph{Complex Geometry}, Springer (2005), \S3.3.
	\item \textbf{Used in this manuscript:} Remark~\ref{rem:lefschetz-reduction} to reduce general $p$ to $p\le n/2$.
	\item \textbf{Hypotheses checked here:} $X$ is compact K\"ahler (assumed throughout) and the K\"ahler class $[\omega]$ is fixed.
	\end{itemize}

\item \textbf{Federer--Fleming compactness and isoperimetric filling.}
	\begin{itemize}
	\item \textbf{Citation (primary):} H.~Federer and W.~H.~Fleming, \emph{Normal and integral currents}, Annals of Mathematics \textbf{72} (1960), 458--520.
	\item \textbf{Citation (textbook):} H.~Federer, \emph{Geometric Measure Theory}, Springer (1969); L.~Simon, \emph{Lectures on Geometric Measure Theory}, ANU (1983).
	\item \textbf{Used in this manuscript:} Theorem~\ref{thm:realization-from-almost} (compactness of integral currents under mass bounds); Proposition~\ref{prop:glue-gap} and Substep~4.2 (constructing a small-mass correction $R_{\mathrm{glue}}$ with $\partial R_{\mathrm{glue}}=-\partial T^{\mathrm{raw}}$ via the flat-norm/isoperimetric filling).
	\item \textbf{Hypotheses checked here:} $X$ is compact (mass bounds yield tightness); all currents are integral; dimension is finite.
	\end{itemize}

\item \textbf{Harvey--Lawson calibrations and structure theorem for positive currents.}
	\begin{itemize}
	\item \textbf{Citation (primary):} R.~Harvey and H.~B.~Lawson, Jr., \emph{Calibrated geometries}, Acta Mathematica \textbf{148} (1982), 47--157.
	\item \textbf{Used in this manuscript:} Theorem~\ref{thm:realization-from-almost} (calibrated integral currents are holomorphic chains) and the algebraicity conclusion.
	\item \textbf{Hypotheses checked here:} $\psi=\omega^{n-p}/(n-p)!$ is a calibration (Wirtinger); the limit current $T$ is $\psi$-calibrated (proved from mass equality).
	\end{itemize}

\item \textbf{Chow's theorem and Serre's GAGA (analytic $\Rightarrow$ algebraic on projective $X$).}
	\begin{itemize}
	\item \textbf{Citation (GAGA):} J.-P.~Serre, \emph{G\'eom\'etrie alg\'ebrique et g\'eom\'etrie analytique}, Annales de l'Institut Fourier \textbf{6} (1956), 1--42.
	\item \textbf{Citation (Chow / standard texts):} R.~Hartshorne, \emph{Algebraic Geometry}, Springer GTM 52 (1977), Appendix~B; or P.~Griffiths and J.~Harris, \emph{Principles of Algebraic Geometry}, Wiley (1978), Ch.~1.
	\item \textbf{Used in this manuscript:} Remark~\ref{rem:chow-gaga} and in the final algebraicity step in Theorems~\ref{thm:realization-from-almost} and \ref{thm:effective-algebraic}.
	\item \textbf{Hypotheses checked here:} $X$ is assumed smooth projective (hence compact complex and algebraic), so complex analytic subvarieties of $X$ are algebraic.
	\end{itemize}

\item \textbf{Holomorphic manufacturing input (H\"ormander--Serre / Bergman kernel asymptotics / peak sections).}
	\begin{itemize}
	\item \textbf{Citation (H\"ormander $L^2$ $\bar\partial$ estimates):} L.~H\"ormander, \emph{An Introduction to Complex Analysis in Several Variables}, North-Holland (3rd ed., 1990).
	\item \textbf{Citation (Bergman/Szeg\H{o} asymptotics, standard sources):} G.~Tian, \emph{On a set of polarized K\"ahler metrics on algebraic manifolds}, J.\ Differential Geom.\ \textbf{32} (1990), 99--130; D.~Catlin, \emph{The Bergman kernel and a theorem of Tian}, in \emph{Analysis and Geometry in Several Complex Variables} (Katata, 1997), Birkh\"auser (1999); S.~Zelditch, \emph{Szeg\H{o} kernels and a theorem of Tian}, International Mathematics Research Notices (1998), no.~6, 317--331.
	\item \textbf{Citation (Serre vanishing / ampleness machinery):} Hartshorne, \emph{Algebraic Geometry}, \S~III.5.
	\item \textbf{Used in this manuscript:} the projective/Bergman subsection feeding Theorem~\ref{thm:local-sheets}, which supplies the local holomorphic sheets/slivers with controlled $C^1$ geometry.
	\item \textbf{Hypotheses checked here:} $X$ is smooth projective, $L\to X$ is ample with a Hermitian metric of curvature $\omega$ (fixed in the projective/Bergman step), and the construction is performed for sufficiently large tensor powers $L^M$ on Bergman-scale balls.
	\end{itemize}

\item \textbf{B\'ar\'any--Grinberg discrepancy rounding (fixed-dimensional rounding).}
	\begin{itemize}
	\item \textbf{Citation (primary):} I.~B\'ar\'any and V.~S.~Grinberg, \emph{On some combinatorial questions in finite-dimensional vector spaces}, Israel Journal of Mathematics \textbf{40} (1981), 147--156.
	\item \textbf{Used in this manuscript:} Lemma~\ref{lem:barany-grinberg} inside Proposition~\ref{prop:cohomology-match} to enforce finitely many integral period constraints simultaneously.
	\item \textbf{Hypotheses checked here:} the rounding dimension is $b=\mathrm{rank}\,H^{2n-2p}(X,\Z)$ (fixed); contributions $v_{Q,j}$ are made uniformly small by refining the mesh so $\|v_{Q,j}\|_{\ell^\infty}\le 1$ after normalization, exactly as in the proof of Proposition~\ref{prop:cohomology-match}.
	\end{itemize}
\end{itemize}

\subsection*{D. Sanity checks (explicitly recorded in the manuscript)}
\begin{itemize}
\item \textbf{$p=1$ case}: Lefschetz $(1,1)$ (Theorem~\ref{thm:codim1}).
\item \textbf{Complete intersections}: Proposition~\ref{prop:complete-intersection}.
\item \textbf{No "coercivity without cone-valued harmonic representative"}: built into the statement of calibration--coercivity and the remarks around Section~\ref{sec:cal-coercivity}.
\item \textbf{Borderline $p=n/2$}: handled by Lemma~\ref{lem:borderline-p-half}.
\end{itemize}
\end{editjonblock}


\begin{editamirblock}
%=========================================================
\section*{Classical Inputs (Published Theorems Used as External Pillars)}
\label{sec:classical-inputs}
%=========================================================

Throughout, $X$ is a compact K\"ahler manifold of complex dimension $n$ with K\"ahler form $\omega$.
Whenever we invoke \emph{algebraicity} (Chow/GAGA) or interpret $\omega$ as the curvature form of an ample line bundle,
we explicitly assume $X$ is \emph{smooth complex projective} and that
\[
[\omega]/2\pi = c_1(L)\in H^2(X;\mathbb{Z})
\quad\text{for some ample holomorphic line bundle }L\to X.
\]

We do not re-prove the following classical results; instead, at each invocation point we verify that hypotheses match.
The main external pillars used in the present manuscript are:

\begin{itemize}
\item \textbf{Geometric Measure Theory: integral currents, compactness, semicontinuity.}
We use the Federer--Fleming theory of integral currents (compactness in the flat topology, deformation/slicing tools, and
mass lower semicontinuity) as foundational background for the limiting and gluing arguments.%
\cite{FF60,Fed69,Sim83,LangGmT}
When varifold compactness/regularity is invoked (e.g.\ weak-* compactness under uniform mass and first variation bounds),
we cite Allard's framework.%
\cite{Allard72}

\item \textbf{Calibrations and Wirtinger-type inequalities.}
We use the calibration formalism for complex submanifolds and the Wirtinger inequality to compare the mass of a
$2p$-current against the K\"ahler calibration $\omega^p/p!$, and to interpret (almost) calibration as a quantitative
positivity statement.%
\cite{HL82,GH78,Voisin02}

\item \textbf{Positive closed currents and analytic cycles.}
At the endgame we appeal to standard identifications between certain \emph{positive, $d$-closed, rectifiable currents}
(with integer multiplicities) and analytic cycles, in the form used in King-type results and their modern treatments.%
\cite{King71,Demailly12}

\item \textbf{H\"ormander $L^2$ $\bar\partial$-methods and Bergman kernel asymptotics.}
The local holomorphic realization steps use $L^2$-existence/estimate inputs for $\bar\partial$ (in the positive line bundle setting)
and near/off-diagonal Bergman kernel expansions with controlled derivatives at the \editamir{$N^{-1/2}$} scale.%
\cite{Demailly12,Tian90,Zelditch98,Catlin99,MaMarinescu07,MaMarinescu13OffDiag,Donaldson01}

\item \textbf{Hodge theory, Hard Lefschetz, and projective algebraicity tools.}
We use standard Hodge decomposition and Hard Lefschetz for smooth complex projective manifolds, and we reduce to the range
$p\le n/2$ via Lefschetz-type arguments where stated.%
\cite{Voisin02,GH78}
When concluding that an analytic cycle is algebraic, we invoke Chow's theorem and Serre's GAGA principle
(\emph{projectivity is required at this point}).%
\cite{Hartshorne77,Serre56}

\item \textbf{Optimal transport duality (for $W_1$ bounds).}
Whenever Wasserstein-$1$ matching/transport bounds are used to control boundary mismatch and to construct couplings,
we rely on the Kantorovich--Rubinstein dual characterization and standard stability estimates.%
\cite{Villani03}
\end{itemize}

\medskip
\noindent\textbf{Referee note.} The phrase ``unconditional'' in this manuscript is to be read as:
\emph{unconditional modulo the published pillars listed above}, with hypotheses checked locally at each invocation site.
\end{editamirblock}


\begin{thebibliography}{99}

\bibitem{Allard72}
W.~K. Allard.
\newblock On the first variation of a varifold.
\newblock {\em Annals of Mathematics}, 95(3):417--491, 1972.

\bibitem{Catlin99}
D.~Catlin.
\newblock The Bergman kernel and a theorem of Tian.
\newblock In {\em Analysis and Geometry in Several Complex Variables}, Trends in Mathematics,
pages 1--23. Birkh\"auser, 1999.

\bibitem{Demailly12}
J.-P. Demailly.
\newblock {\em Complex Analytic and Differential Geometry}.
\newblock Open book/lecture notes, version 2012. Available at \url{https://www-fourier.ujf-grenoble.fr/~demailly/manuscripts/agbook.pdf}.

\bibitem{Donaldson01}
S.~K. Donaldson.
\newblock Scalar curvature and projective embeddings. I.
\newblock {\em Journal of Differential Geometry}, 59(3):479--522, 2001.

\bibitem{FF60}
H.~Federer and W.~H. Fleming.
\newblock Normal and integral currents.
\newblock {\em Annals of Mathematics}, 72(3):458--520, 1960.

\bibitem{Fed69}
H.~Federer.
\newblock {\em Geometric Measure Theory}.
\newblock Springer, 1969.

\bibitem{GH78}
P.~Griffiths and J.~Harris.
\newblock {\em Principles of Algebraic Geometry}.
\newblock Wiley-Interscience, 1978.

\bibitem{Hartshorne77}
R.~Hartshorne.
\newblock {\em Algebraic Geometry}.
\newblock Graduate Texts in Mathematics 52. Springer, 1977.

\bibitem{HL82}
R.~Harvey and H.~B. Lawson, Jr.
\newblock Calibrated geometries.
\newblock {\em Acta Mathematica}, 148:47--157, 1982.

\bibitem{King71}
J.~R. King.
\newblock The currents defined by analytic varieties.
\newblock {\em Acta Mathematica}, 127:185--220, 1971.

\bibitem{LangGmT}
S.~Lang.
\newblock {\em Fundamentals of Differential Geometry}.
\newblock Graduate Texts in Mathematics 191. Springer, 1999.
\newblock (See Ch.~XIV for currents and the compactness theorem.)

\bibitem{MaMarinescu07}
X.~Ma and G.~Marinescu.
\newblock {\em Holomorphic Morse Inequalities and Bergman Kernels}.
\newblock Progress in Mathematics 254. Birkh\"auser, 2007.

\bibitem{Serre56}
J.-P. Serre.
\newblock G\'eom\'etrie alg\'ebrique et g\'eom\'etrie analytique ({GAGA}).
\newblock {\em Annales de l'Institut Fourier}, 6:1--42, 1956.

\bibitem{Sim83}
L.~Simon.
\newblock {\em Lectures on Geometric Measure Theory}.
\newblock Proceedings of the Centre for Mathematical Analysis, Australian National University,
Vol.~3, 1983.

\bibitem{MaMarinescu13OffDiag}
X.~Ma and G.~Marinescu.
\newblock Remark on the off-diagonal expansion of the Bergman kernel on compact K\"ahler manifolds.
\newblock {\em Communications in Mathematics and Statistics}, 1:37--41, 2013.
\newblock arXiv:1302.2346.

\bibitem{Tian90}
G.~Tian.
\newblock On a set of polarized {K}\"ahler metrics on algebraic manifolds.
\newblock {\em Journal of Differential Geometry}, 32(1):99--130, 1990.

\bibitem{Voisin02}
C.~Voisin.
\newblock {\em Hodge Theory and Complex Algebraic Geometry I}.
\newblock Cambridge Studies in Advanced Mathematics 76. Cambridge University Press, 2002.

\bibitem{Zelditch98}
S.~Zelditch.
\newblock Szeg\H{o} kernels and a theorem of Tian.
\newblock {\em International Mathematics Research Notices}, 1998(6):317--331, 1998.



\bibitem{Villani03}
C.~Villani.
\newblock \emph{Topics in Optimal Transportation}.
\newblock Graduate Studies in Mathematics, Vol.~58, American Mathematical Society, 2003.

\bibitem{Wells}\editamir{R.~O.~Wells, Jr.\ \emph{Differential Analysis on Complex Manifolds}. Graduate Texts in Mathematics, vol.~65. Springer, New York, 3rd ed., 2008.}
\end{thebibliography}


\end{document}
