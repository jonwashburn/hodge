\documentclass[10pt]{article}
\usepackage[margin=0.5in,landscape]{geometry}
\usepackage{xcolor}
\usepackage{hyperref}
\usepackage{fancyvrb}

\definecolor{diffadd}{RGB}{0,120,0}
\definecolor{diffdel}{RGB}{180,0,0}
\definecolor{diffhdr}{RGB}{0,0,140}
\definecolor{diffhunk}{RGB}{120,0,120}
\definecolor{diffctx}{RGB}{30,30,30}

% Global fancyvrb settings
\fvset{fontsize=\tiny}

\setlength{\parindent}{0pt}
\setlength{\parskip}{0pt}
\begin{document}
\section*{Changes: Hodge-v6-w-Jon-Update-MERGED-v7.tex → Hodge-v6-w-Jon-Update-MERGED.tex}
\textbf{Date generated:} 2025-12-30\\
\textbf{Old:} \texttt{/Users/jonathanwashburn/Projects/hodge/Hodge-v6-w-Jon-Update-MERGED-v7.tex}\\
\textbf{New:} \texttt{/Users/jonathanwashburn/Projects/hodge/Hodge-v6-w-Jon-Update-MERGED.tex}\\
\textbf{Unified-diff stats:} 464 additions, 283 deletions, 40 hunks\par\medskip
\hrule\medskip
\begingroup\color{diffhdr}\Verb|--- Hodge-v6-w-Jon-Update-MERGED-v7.tex|\endgroup\par
\begingroup\color{diffhdr}\Verb|+++ Hodge-v6-w-Jon-Update-MERGED.tex|\endgroup\par
\begingroup\color{diffhunk}\Verb|@@ -459,6 +459,27 @@|\endgroup\par
\begingroup\color{diffctx}\Verb| \end{center}|\endgroup\par
\begingroup\color{diffctx}\Verb| \end{editconeblock}|\endgroup\par
\begingroup\color{diffctx}\Verb| |\endgroup\par
\begingroup\color{diffadd}\Verb|+\subsection*{External inputs (adversarial disclosure)}|\endgroup\par
\begingroup\color{diffadd}\Verb|+|\endgroup\par
\begingroup\color{diffadd}\Verb|+For transparency regarding what this manuscript does and does not prove ``from scratch,'' we explicitly list the external inputs on which the main theorem depends.  These are deep results from prior literature that are cited and used but not reproved here.|\endgroup\par
\begingroup\color{diffadd}\Verb|+|\endgroup\par
\begingroup\color{diffadd}\Verb|+\begin{enumerate}|\endgroup\par
\begingroup\color{diffadd}\Verb|+\item \textbf{Bergman kernel asymptotics and jet control} (Lemma~\ref{lem:bergman-control}): The uniform $C^1$ jet control on $m^{-1/2}$-balls for holomorphic sections of high tensor powers of ample line bundles.  References: Tian~\cite{Tian90}, Catlin~\cite{Catlin99}, Zelditch~\cite{Zelditch98}, Ma--Marinescu~\cite{MaMarinescu07}.|\endgroup\par
\begingroup\color{diffadd}\Verb|+|\endgroup\par
\begingroup\color{diffadd}\Verb|+\item \textbf{Bertini-type transversality}: The existence of small generic perturbations in linear systems that preserve prescribed jets while maintaining $C^1$ bounds.  References: Griffiths--Harris~\cite{GH78}, Lazarsfeld~\cite{Lazarsfeld-PAG}.|\endgroup\par
\begingroup\color{diffadd}\Verb|+|\endgroup\par
\begingroup\color{diffadd}\Verb|+\item \textbf{Integer rounding in fixed dimension} (Proposition~\ref{prop:global-coherence-all-labels}, Remark~\ref{rem:integer-rounding-external}): The Barvinok--Bar\'any--Grinberg discrepancy bounds for integer approximation in fixed-dimensional polytopes.  Reference: Barvinok~\cite{Barvinok-IntProg}.|\endgroup\par
\begingroup\color{diffadd}\Verb|+|\endgroup\par
\begingroup\color{diffadd}\Verb|+\item \textbf{Harvey--Lawson structure theorem}: $\psi$-calibrated integral currents are positive sums of complex analytic subvarieties.  Reference: Harvey--Lawson~\cite{HL82}.|\endgroup\par
\begingroup\color{diffadd}\Verb|+|\endgroup\par
\begingroup\color{diffadd}\Verb|+\item \textbf{Chow / GAGA}: Closed analytic subvarieties of projective manifolds are algebraic.  References: Chow~\cite{Chow49}, Serre~\cite{Serre56}.|\endgroup\par
\begingroup\color{diffadd}\Verb|+|\endgroup\par
\begingroup\color{diffadd}\Verb|+\item \textbf{Federer--Fleming compactness}: Integral currents with uniformly bounded mass and boundary mass admit weakly convergent subsequences with integral limits.  Reference: Federer~\cite{Federer69}.|\endgroup\par
\begingroup\color{diffadd}\Verb|+\end{enumerate}|\endgroup\par
\begingroup\color{diffadd}\Verb|+|\endgroup\par
\begingroup\color{diffadd}\Verb|+\noindent|\endgroup\par
\begingroup\color{diffadd}\Verb|+The novel content of this manuscript is the \emph{microstructure/gluing} construction (Section~\ref{sec:realization}) that produces fixed-class integral cycles with vanishing calibration defect, together with the corner-exit coherence mechanism that achieves the required $\mathcal{F}(\partial T^{\mathrm{raw}}) = o(m)$ estimate.  The above external inputs are the ``black boxes'' on which this construction rests.  See Remark~\ref{rem:external-inputs-h1h2} for a more detailed discussion of the H1/H2 external inputs.|\endgroup\par
\begingroup\color{diffadd}\Verb|+|\endgroup\par
\begingroup\color{diffctx}\Verb| \section{Notation and K\"ahler Preliminaries}|\endgroup\par
\begingroup\color{diffctx}\Verb| |\endgroup\par
\begingroup\color{diffctx}\Verb| This section records the analytic and geometric conventions used throughout the|\endgroup\par
\begingroup\color{diffhunk}\Verb|@@ -473,6 +494,13 @@|\endgroup\par
\begingroup\color{diffctx}\Verb| \paragraph{Ambient setting.}|\endgroup\par
\begingroup\color{diffctx}\Verb| Let $X$ be a smooth projective complex manifold of complex dimension $n$, with|\endgroup\par
\begingroup\color{diffctx}\Verb| K\"ahler form $\omega$ and integrable complex structure $J$.|\endgroup\par
\begingroup\color{diffadd}\Verb|+Since $X$ is projective, we may (and do) fix $\omega$ so that its cohomology class is the hyperplane/ample class:|\endgroup\par
\begingroup\color{diffadd}\Verb|+\[|\endgroup\par
\begingroup\color{diffadd}\Verb|+[\omega]=c_1(L)\in H^2(X,\Z)|\endgroup\par
\begingroup\color{diffadd}\Verb|+\]|\endgroup\par
\begingroup\color{diffadd}\Verb|+for some ample holomorphic line bundle $L\to X$ (equivalently, after choosing an embedding $X\hookrightarrow\mathbb P^M$, take $\omega$ to be a|\endgroup\par
\begingroup\color{diffadd}\Verb|+positive multiple of the restricted Fubini--Study form).  This ensures that the Lefschetz operator $[\omega]\wedge(\cdot)$ preserves rational cohomology,|\endgroup\par
\begingroup\color{diffadd}\Verb|+and that $[\omega^p]\in H^{2p}(X,\Z)$ is algebraic (complete intersections).|\endgroup\par
\begingroup\color{diffctx}\Verb| The associated Riemannian metric is|\endgroup\par
\begingroup\color{diffctx}\Verb| \[|\endgroup\par
\begingroup\color{diffctx}\Verb| g(\cdot,\cdot)=\omega(\cdot,J\cdot),|\endgroup\par
\begingroup\color{diffhunk}\Verb|@@ -2178,14 +2206,17 @@|\endgroup\par
\begingroup\color{diffctx}\Verb| \begin{proof}|\endgroup\par
\begingroup\color{diffctx}\Verb| In this regime, the exponent $2-\frac{2n}{k}=0$ when $k=n$, so the naive mass estimate does not guarantee decay.|\endgroup\par
\begingroup\color{diffctx}\Verb| However, Proposition~\ref{prop:integer-transport} yields $\mathcal F(\partial T^{\mathrm{raw}})\to 0$ directly from the slow-variation and face-edit control at lattice scale $\delta_j=o(h_j)$.|\endgroup\par
\begingroup\color{diffdel}\Verb|-Proposition~\ref{prop:glue-gap} then gives $\Mass(R_{\mathrm{glue}})\to 0$, and Proposition~\ref{prop:almost-calibration} concludes.|\endgroup\par
\begingroup\color{diffadd}\Verb|+Proposition~\ref{prop:glue-gap} then gives a filling $U_j$ with $\partial U_j=\partial T^{\mathrm{raw}}$ and $\Mass(U_j)\to 0$;|\endgroup\par
\begingroup\color{diffadd}\Verb|+taking $R_{\mathrm{glue},j}:=-U_j$ yields a vanishing-mass correction, and Proposition~\ref{prop:almost-calibration} concludes.|\endgroup\par
\begingroup\color{diffctx}\Verb| \end{proof}|\endgroup\par
\begingroup\color{diffctx}\Verb| |\endgroup\par
\begingroup\color{diffctx}\Verb| \subsection*{H1/H2 packaged at the point of use (for Theorem~\ref{thm:spine-quantitative})}|\endgroup\par
\begingroup\color{diffctx}\Verb| |\endgroup\par
\begingroup\color{diffctx}\Verb| \begin{proposition}[H1 package: local holomorphic multi-sheet manufacturing]\label{prop:h1-package}|\endgroup\par
\begingroup\color{diffctx}\Verb| In the parameter schedule of \S\ref{sec:parameter-schedule}, for each mesh cell $Q$ and each direction family prescribed by the local Carath\'eodory data of $\beta$ on $Q$,|\endgroup\par
\begingroup\color{diffdel}\Verb|-Theorem~\ref{thm:local-sheets} and the projective holomorphic manufacturing machinery supply the required calibrated sheet--sum $S_Q$ satisfying $\Mass(S_Q)=\langle S_Q,\psi\rangle$|\endgroup\par
\begingroup\color{diffadd}\Verb|+Theorem~\ref{thm:local-sheets} and the projective holomorphic manufacturing machinery (implemented concretely via Bergman-scale $C^1$ jet control, Lemma~\ref{lem:bergman-control},|\endgroup\par
\begingroup\color{diffadd}\Verb|+feeding the finite-template realization Proposition~\ref{prop:finite-template} and the corner-exit holomorphic sliver construction Proposition~\ref{prop:holomorphic-corner-exit-L1})|\endgroup\par
\begingroup\color{diffadd}\Verb|+supply the required calibrated sheet--sum $S_Q$ satisfying $\Mass(S_Q)=\langle S_Q,\psi\rangle$|\endgroup\par
\begingroup\color{diffctx}\Verb| with quantitative disjointness, slope, and budget control.  Thus the hypothesis \textnormal{(H1)} in Theorem~\ref{thm:spine-quantitative} holds in this manuscript.|\endgroup\par
\begingroup\color{diffctx}\Verb| \end{proposition}|\endgroup\par
\begingroup\color{diffctx}\Verb| |\endgroup\par
\begingroup\color{diffhunk}\Verb|@@ -2200,6 +2231,27 @@|\endgroup\par
\begingroup\color{diffctx}\Verb| rather than relying on a decay exponent in $h$.|\endgroup\par
\begingroup\color{diffctx}\Verb| Thus the hypothesis \textnormal{(H2)} in Theorem~\ref{thm:spine-quantitative} holds in this manuscript.|\endgroup\par
\begingroup\color{diffctx}\Verb| \end{proposition}|\endgroup\par
\begingroup\color{diffadd}\Verb|+|\endgroup\par
\begingroup\color{diffadd}\Verb|+\begin{remark}[External inputs for H1/H2 (adversarial disclosure)]\label{rem:external-inputs-h1h2}|\endgroup\par
\begingroup\color{diffadd}\Verb|+For clarity in assessing the proof, we explicitly flag the following components of H1/H2 as \emph{external inputs}---deep theorems from prior literature that are invoked but not proved from scratch here.|\endgroup\par
\begingroup\color{diffadd}\Verb|+|\endgroup\par
\begingroup\color{diffadd}\Verb|+\smallskip\noindent|\endgroup\par
\begingroup\color{diffadd}\Verb|+\textbf{External inputs for H1:}|\endgroup\par
\begingroup\color{diffadd}\Verb|+\begin{enumerate}|\endgroup\par
\begingroup\color{diffadd}\Verb|+\item \emph{Bergman/peak-section control} (Lemma~\ref{lem:bergman-control}): The uniform $C^1$ gradient control on $m^{-1/2}$-balls is a consequence of standard Bergman kernel asymptotics and jet interpolation on ample line bundles.  References: Tian~\cite{Tian90}, Catlin~\cite{Catlin99}, Zelditch~\cite{Zelditch98}, Demailly~\cite{Demailly-L2}.  This is not a trivial input: it requires holomorphic peak-section construction with quantitative derivative control.|\endgroup\par
\begingroup\color{diffadd}\Verb|+\item \emph{Bertini-type transversality} (Proposition~\ref{prop:tangent-approx-full}, Step 4): The existence of small generic perturbations that preserve prescribed jets while maintaining $C^1$ bounds uses the fact that for large $m$, the space $H^0(X,L^m)$ is large enough to perturb independently at separated points.  References: Bertini theorems in Griffiths--Harris~\cite{GH78}, Lazarsfeld~\cite{Lazarsfeld-PAG}.|\endgroup\par
\begingroup\color{diffadd}\Verb|+\end{enumerate}|\endgroup\par
\begingroup\color{diffadd}\Verb|+|\endgroup\par
\begingroup\color{diffadd}\Verb|+\smallskip\noindent|\endgroup\par
\begingroup\color{diffadd}\Verb|+\textbf{External inputs for H2:}|\endgroup\par
\begingroup\color{diffadd}\Verb|+\begin{enumerate}|\endgroup\par
\begingroup\color{diffadd}\Verb|+\item \emph{Integer simultaneous rounding} (Proposition~\ref{prop:global-coherence-all-labels}): The claim that integer activations can satisfy local budgets, slow-variation, and global period constraints simultaneously relies on the Barvinok--Bar\'any--Grinberg integer rounding lemmas and the fact that the constraint dimension is fixed (rank of $H^{2n-2p}(X,\Z)$).  Reference: Barvinok~\cite{Barvinok-IntProg}.|\endgroup\par
\begingroup\color{diffadd}\Verb|+\item \emph{Corner-exit template coherence} (Proposition~\ref{prop:vertex-template-face-edits}): The deterministic face-incidence properties of the corner-exit geometry are structural consequences of convexity and transversality, but the fact that edge/corner contributions do not accumulate relies on the bounded-face-count property of cubical meshes.|\endgroup\par
\begingroup\color{diffadd}\Verb|+\end{enumerate}|\endgroup\par
\begingroup\color{diffadd}\Verb|+|\endgroup\par
\begingroup\color{diffadd}\Verb|+\smallskip\noindent|\endgroup\par
\begingroup\color{diffadd}\Verb|+\textbf{Consistency note:} The local engine for H1 is not the multi-direction local-sheets statement (Theorem~\ref{thm:local-sheets}) in isolation, but the corner-exit route (Proposition~\ref{prop:holomorphic-corner-exit-L1} + vertex-template coherence) which manufactures parallel translates of a single plane per label and enforces deterministic face incidence.  This avoids the potential disjointness issues that arise when different $(n-p)$-planes generically intersect for $p < n/2$.|\endgroup\par
\begingroup\color{diffadd}\Verb|+\end{remark}|\endgroup\par
\begingroup\color{diffctx}\Verb| \end{editjonblock}|\endgroup\par
\begingroup\color{diffctx}\Verb| |\endgroup\par
\begingroup\color{diffctx}\Verb| % ------------------------------------------------------------|\endgroup\par
\begingroup\color{diffhunk}\Verb|@@ -2654,18 +2706,21 @@|\endgroup\par
\begingroup\color{diffctx}\Verb| See Lazarsfeld, \emph{Positivity in Algebraic Geometry~I}, Theorem~1.8.5.|\endgroup\par
\begingroup\color{diffctx}\Verb| \end{proof}|\endgroup\par
\begingroup\color{diffctx}\Verb| |\endgroup\par
\begingroup\color{diffdel}\Verb|-\begin{lemma}[Uniform $C^1$ control on $m^{-1/2}$-balls via Bergman kernels]|\endgroup\par
\begingroup\color{diffdel}\Verb|-\label{lem:bergman-control}|\endgroup\par
\begingroup\color{diffdel}\Verb|-Fix $\varepsilon>0$.  There exists $m_1(\varepsilon)$ such that for all|\endgroup\par
\begingroup\color{diffdel}\Verb|-$m\ge m_1(\varepsilon)$, each $x\in X$, and each collection of $p$|\endgroup\par
\begingroup\color{diffdel}\Verb|-complex covectors $\lambda_1,\ldots,\lambda_p\in T_x^*X$, there exist|\endgroup\par
\begingroup\color{diffdel}\Verb|-sections $s_1,\ldots,s_p\in H^0(X,L^m)$ with the following properties|\endgroup\par
\begingroup\color{diffdel}\Verb|-in normal holomorphic coordinates centered at $x$:|\endgroup\par
\begingroup\color{diffadd}\Verb|+\begin{lemma}[Bergman-scale peak sections and quantitative $C^1$ jet control]\label{lem:bergman-control}|\endgroup\par
\begingroup\color{diffadd}\Verb|+Fix $\varepsilon>0$.|\endgroup\par
\begingroup\color{diffadd}\Verb|+There exist $m_1(\varepsilon)\in\N$ and a constant $c>0$ (depending only on $(X,\omega)$) such that for all|\endgroup\par
\begingroup\color{diffadd}\Verb|+$m\ge m_1(\varepsilon)$ and each $x\in X$ the following holds in normal holomorphic coordinates centered at $x$|\endgroup\par
\begingroup\color{diffadd}\Verb|+and a local holomorphic frame of $L^m$ on $B_{c\,m^{-1/2}}(x)$:|\endgroup\par
\begingroup\color{diffctx}\Verb| \begin{enumerate}|\endgroup\par
\begingroup\color{diffdel}\Verb|-\item[\textnormal{(i)}] $s_i(x)=0$ and $ds_i(x)=\lambda_i$ for each $i$;|\endgroup\par
\begingroup\color{diffdel}\Verb|-\item[\textnormal{(ii)}] on the geodesic ball $B_{c\,m^{-1/2}}(x)$|\endgroup\par
\begingroup\color{diffdel}\Verb|-(for a universal constant $c>0$ depending only on $(X,\omega)$),|\endgroup\par
\begingroup\color{diffdel}\Verb|-the gradients satisfy|\endgroup\par
\begingroup\color{diffadd}\Verb|+\item[\textnormal{(0)}] \textbf{A nonvanishing reference section.}|\endgroup\par
\begingroup\color{diffadd}\Verb|+There exists $s_0\in H^0(X,L^m)$ with $s_0(x)=1$ and|\endgroup\par
\begingroup\color{diffadd}\Verb|+\[|\endgroup\par
\begingroup\color{diffadd}\Verb!+\sup_{y\in B_{c\,m^{-1/2}}(x)}|s_0(y)-1|\ +\ \sup_{y\in B_{c\,m^{-1/2}}(x)}\|ds_0(y)\|\ \le\ \varepsilon.!\endgroup\par
\begingroup\color{diffadd}\Verb|+\]|\endgroup\par
\begingroup\color{diffadd}\Verb!+In particular, if $\varepsilon\le \tfrac12$ then $|s_0(y)|\ge \tfrac12$ throughout $B_{c\,m^{-1/2}}(x)$.!\endgroup\par
\begingroup\color{diffadd}\Verb|+\item[\textnormal{(i)}] \textbf{1-jet interpolation with uniform $C^1$ control.}|\endgroup\par
\begingroup\color{diffadd}\Verb|+For each collection of $p$ complex covectors $\lambda_1,\ldots,\lambda_p\in T_x^*X$, there exist sections|\endgroup\par
\begingroup\color{diffadd}\Verb|+$s_1,\ldots,s_p\in H^0(X,L^m)$ such that $s_i(x)=0$, $ds_i(x)=\lambda_i$, and|\endgroup\par
\begingroup\color{diffctx}\Verb| \[|\endgroup\par
\begingroup\color{diffctx}\Verb! \|ds_i(y)-\lambda_i\|\le \varepsilon!\endgroup\par
\begingroup\color{diffctx}\Verb| \quad\text{for all } y\in B_{c\,m^{-1/2}}(x).|\endgroup\par
\begingroup\color{diffhunk}\Verb|@@ -2674,12 +2729,13 @@|\endgroup\par
\begingroup\color{diffctx}\Verb| \end{lemma}|\endgroup\par
\begingroup\color{diffctx}\Verb| |\endgroup\par
\begingroup\color{diffctx}\Verb| \begin{proof}|\endgroup\par
\begingroup\color{diffdel}\Verb|-This is a standard consequence of the peak-section construction together with the|\endgroup\par
\begingroup\color{diffdel}\Verb|-Bergman kernel asymptotic expansion and its $C^\ell$-control on Bergman-scale balls.|\endgroup\par
\begingroup\color{diffdel}\Verb|-For the basic expansion see Tian~\cite{Tian90} and the refinements of Catlin~\cite{Catlin99}|\endgroup\par
\begingroup\color{diffdel}\Verb|-and Zelditch~\cite{Zelditch98}; quantitative jet-interpolation and $C^\ell$ estimates|\endgroup\par
\begingroup\color{diffdel}\Verb|-suitable for projective embeddings can be found for example in Donaldson~\cite{Donaldson01}|\endgroup\par
\begingroup\color{diffdel}\Verb|-or in the exposition of Ma--Marinescu~\cite{MaMarinescu07}.|\endgroup\par
\begingroup\color{diffadd}\Verb|+This is a standard “peak section / Bergman kernel” theorem: after rescaling by $m^{-1/2}$, the Bergman kernel of $(L^m,\omega)$ admits a full|\endgroup\par
\begingroup\color{diffadd}\Verb|+asymptotic expansion with uniform $C^\ell$ control in $x$, and the evaluation map on $1$-jets admits a uniformly bounded right inverse in Bergman-scale norms.|\endgroup\par
\begingroup\color{diffadd}\Verb|+One can deduce the existence of $s_0$ and of the $s_i$ by applying this right inverse to the jet data $(s_0(x),ds_0(x))=(1,0)$ and|\endgroup\par
\begingroup\color{diffadd}\Verb|+$(s_i(x),ds_i(x))=(0,\lambda_i)$.|\endgroup\par
\begingroup\color{diffadd}\Verb|+References include Tian~\cite{Tian90}, Catlin~\cite{Catlin99}, Zelditch~\cite{Zelditch98} for the foundational Bergman expansion, and|\endgroup\par
\begingroup\color{diffadd}\Verb|+Ma--Marinescu~\cite[\S4.1]{MaMarinescu07} for a systematic treatment with derivatives and peak sections; see also Donaldson~\cite{Donaldson01}|\endgroup\par
\begingroup\color{diffadd}\Verb|+and Demailly~\cite{Demailly-L2} for quantitative jet interpolation via peak sections and $L^2$ methods.|\endgroup\par
\begingroup\color{diffctx}\Verb| \end{proof}|\endgroup\par
\begingroup\color{diffctx}\Verb| |\endgroup\par
\begingroup\color{diffctx}\Verb| \begin{editblock}|\endgroup\par
\begingroup\color{diffhunk}\Verb|@@ -2789,16 +2845,22 @@|\endgroup\par
\begingroup\color{diffctx}\Verb| with prescribed tangent directions and mass fractions.|\endgroup\par
\begingroup\color{diffctx}\Verb| |\endgroup\par
\begingroup\color{diffctx}\Verb| \begin{theorem}[Local multi-sheet construction]\label{thm:local-sheets}|\endgroup\par
\begingroup\color{diffdel}\Verb|-Let $Q\subset X$ be a small coordinate cube.  Let|\endgroup\par
\begingroup\color{diffdel}\Verb|-$\Pi_1,\ldots,\Pi_J\in \Gr_{n-p}(TQ)$ be constant $(n-p)$-planes, and let|\endgroup\par
\begingroup\color{diffdel}\Verb|-$\theta_1,\ldots,\theta_J\in\Q_{>0}$ with $\sum_j\theta_j=1$.|\endgroup\par
\begingroup\color{diffdel}\Verb|-For every $\varepsilon,\delta>0$, there exist smooth $\psi$-calibrated|\endgroup\par
\begingroup\color{diffadd}\Verb|+Let $Q\subset X$ be a small coordinate cube of diameter $h$ contained in a holomorphic chart.|\endgroup\par
\begingroup\color{diffadd}\Verb|+Let $\Pi_1,\ldots,\Pi_J$ be constant \emph{complex} $(n-p)$-planes in this chart (hence $\psi$--calibrated directions).|\endgroup\par
\begingroup\color{diffadd}\Verb|+Assume the direction labels $\{\Pi_j\}$ lie in the finite calibrated direction dictionary/net used in the corner-exit route, so that each $\Pi_j$|\endgroup\par
\begingroup\color{diffadd}\Verb|+admits a corner-exit translation template in $Q$ as supplied by Proposition~\ref{prop:corner-exit-template-net}.|\endgroup\par
\begingroup\color{diffadd}\Verb|+Let $\theta_1,\ldots,\theta_J\in\Q_{>0}$ with $\sum_j\theta_j=1$.|\endgroup\par
\begingroup\color{diffadd}\Verb|+|\endgroup\par
\begingroup\color{diffadd}\Verb|+For every $\varepsilon,\delta>0$, there exist a line-bundle power $M$ large enough that $h\le c\,M^{-1/2}$ and smooth $\psi$-calibrated|\endgroup\par
\begingroup\color{diffctx}\Verb| complete intersections $\{Y_j^a\}_{j,a}$ in $X$ such that:|\endgroup\par
\begingroup\color{diffctx}\Verb| \begin{enumerate}|\endgroup\par
\begingroup\color{diffctx}\Verb| \item[\textnormal{(i)}] \textbf{Angle control:}|\endgroup\par
\begingroup\color{diffctx}\Verb| $\sup_{y\in Q}\angle(T_yY_j^a,\Pi_j)<\varepsilon$;|\endgroup\par
\begingroup\color{diffctx}\Verb| \item[\textnormal{(ii)}] \textbf{Mass fractions:}|\endgroup\par
\begingroup\color{diffdel}\Verb!-$\bigl|\Mass(Y_j^a\llcorner Q)/\sum_{i,b}\Mass(Y_i^b\llcorner Q)-\theta_j\bigr|<\delta$;!\endgroup\par
\begingroup\color{diffadd}\Verb|+\[|\endgroup\par
\begingroup\color{diffadd}\Verb!+\Bigl|\frac{\sum_{a}\Mass(Y_j^a\llcorner Q)}{\sum_{i,b}\Mass(Y_i^b\llcorner Q)}-\theta_j\Bigr|\ <\ \delta!\endgroup\par
\begingroup\color{diffadd}\Verb|+\qquad\text{for each }j;|\endgroup\par
\begingroup\color{diffadd}\Verb|+\]|\endgroup\par
\begingroup\color{diffctx}\Verb| \item[\textnormal{(iii)}] \textbf{Disjointness:} The $Y_j^a$ are pairwise disjoint on $Q$;|\endgroup\par
\begingroup\color{diffctx}\Verb| \item[\textnormal{(iv)}] \textbf{Boundary control:}|\endgroup\par
\begingroup\color{diffctx}\Verb| $\partial([Y_j^a]\llcorner Q)$ is supported on $\partial Q$.|\endgroup\par
\begingroup\color{diffhunk}\Verb|@@ -2806,98 +2868,84 @@|\endgroup\par
\begingroup\color{diffctx}\Verb| \end{theorem}|\endgroup\par
\begingroup\color{diffctx}\Verb| |\endgroup\par
\begingroup\color{diffctx}\Verb| \begin{proof}|\endgroup\par
\begingroup\color{diffdel}\Verb|-The proof proceeds in four substeps.|\endgroup\par
\begingroup\color{diffadd}\Verb|+Fix $\varepsilon,\delta>0$ and set the internal small-slope tolerance|\endgroup\par
\begingroup\color{diffadd}\Verb|+\[|\endgroup\par
\begingroup\color{diffadd}\Verb|+\varepsilon_*:=\min\Bigl\{\tfrac{1}{10}\varepsilon,\ \tfrac{1}{10}\sqrt{\delta}\Bigr\}.|\endgroup\par
\begingroup\color{diffadd}\Verb|+\]|\endgroup\par
\begingroup\color{diffadd}\Verb|+Since the desired angle bound is $<\varepsilon$, it suffices to manufacture the pieces at the smaller scale $\varepsilon_*$.|\endgroup\par
\begingroup\color{diffadd}\Verb|+The proof proceeds in five substeps.|\endgroup\par
\begingroup\color{diffctx}\Verb| |\endgroup\par
\begingroup\color{diffctx}\Verb| \medskip\noindent|\endgroup\par
\begingroup\color{diffctx}\Verb| \textbf{Substep 3.1: Local setup and flattening.}|\endgroup\par
\begingroup\color{diffdel}\Verb|-Shrink $Q$ so that there is a holomorphic chart|\endgroup\par
\begingroup\color{diffdel}\Verb|-$\Phi:U\to B(0,2)\subset\C^n$ with $Q\subset U$,|\endgroup\par
\begingroup\color{diffdel}\Verb|-$\Phi(Q)\subset [-1,1]^{2n}\subset\C^n$, and the K\"ahler form $\omega$|\endgroup\par
\begingroup\color{diffdel}\Verb|-and calibration $\psi=\omega^{n-p}/(n-p)!$ are $C^1$-close to the flat|\endgroup\par
\begingroup\color{diffdel}\Verb|-model on $\C^n$.  The calibration cone $K_{n-p}(x)\subset\Gr_{n-p}(T_xX)$|\endgroup\par
\begingroup\color{diffdel}\Verb|-varies smoothly and stays uniformly close to the flat cone of complex|\endgroup\par
\begingroup\color{diffdel}\Verb|-$(n-p)$-planes.  We prove Theorem~\ref{thm:local-sheets} in this flattened|\endgroup\par
\begingroup\color{diffdel}\Verb|-model; everything is diffeomorphism-invariant, and volume/mass distortions|\endgroup\par
\begingroup\color{diffdel}\Verb|-are controlled by the uniform $C^1$-closeness of the metric.|\endgroup\par
\begingroup\color{diffadd}\Verb|+Work in the given holomorphic chart containing $Q$ and identify $Q$ with a coordinate cube of diameter $h$ in $\C^n$.|\endgroup\par
\begingroup\color{diffadd}\Verb|+Since $h$ is small, $\omega$ and $\psi=\omega^{n-p}/(n-p)!$ are $C^1$-close to the flat model on $Q$; in particular, angles and masses in $Q$|\endgroup\par
\begingroup\color{diffadd}\Verb|+are distorted by a factor $1+o(1)$ (depending only on the $C^1$-variation of the metric on $Q$).|\endgroup\par
\begingroup\color{diffctx}\Verb| |\endgroup\par
\begingroup\color{diffctx}\Verb| \medskip\noindent|\endgroup\par
\begingroup\color{diffctx}\Verb| \textbf{Substep 3.2: Approximate target planes by calibrated planes.}|\endgroup\par
\begingroup\color{diffdel}\Verb|-At each $x\in Q$, the set $K_{n-p}(x)$ of $\psi$-calibrated complex|\endgroup\par
\begingroup\color{diffdel}\Verb|-$(n-p)$-planes is a compact subset of $\Gr_{n-p}(T_xX)$ (isomorphic to|\endgroup\par
\begingroup\color{diffdel}\Verb|-the complex Grassmannian $G_{\C}(n-p,n)$).  For any real $(n-p)$-plane|\endgroup\par
\begingroup\color{diffdel}\Verb|-$\Pi_j$, compactness guarantees the existence of a calibrated plane|\endgroup\par
\begingroup\color{diffdel}\Verb|-$\widetilde\Pi_j \in K_{n-p}(x)$ minimizing the Grassmannian distance:|\endgroup\par
\begingroup\color{diffdel}\Verb|-\[|\endgroup\par
\begingroup\color{diffdel}\Verb|-\widetilde\Pi_j := \arg\min_{P \in K_{n-p}(x)} \angle(\Pi_j, P).|\endgroup\par
\begingroup\color{diffdel}\Verb|-\]|\endgroup\par
\begingroup\color{diffdel}\Verb|-Since $K_{n-p}(x)$ spans the full complex Grassmannian (every complex|\endgroup\par
\begingroup\color{diffdel}\Verb|-$(n-p)$-plane is calibrated), and $\Pi_j$ arises from a Carath\'eodory|\endgroup\par
\begingroup\color{diffdel}\Verb|-decomposition of $\beta(x) \in K_p(x)$, we have|\endgroup\par
\begingroup\color{diffdel}\Verb|-$\angle(\Pi_j, \widetilde\Pi_j) \le \eta$ for some $\eta > 0$ controlled|\endgroup\par
\begingroup\color{diffdel}\Verb|-by the $C^0$-norm of $\beta$.|\endgroup\par
\begingroup\color{diffdel}\Verb|-Choose $\eta \le \varepsilon/2$ so that sheets with tangent plane|\endgroup\par
\begingroup\color{diffdel}\Verb|-$\widetilde\Pi_j$ automatically satisfy|\endgroup\par
\begingroup\color{diffdel}\Verb|-$\angle(T_y Y_j^a, \Pi_j) < \varepsilon$.|\endgroup\par
\begingroup\color{diffadd}\Verb|+In the application to Carath\'eodory decompositions of $\beta(x)\in K_p(x)$, the directions are already complex $(n-p)$--planes, hence calibrated.|\endgroup\par
\begingroup\color{diffadd}\Verb|+Thus no approximation step is needed here.  (If one starts from an arbitrary real $(n-p)$-plane, one may replace it by a nearby calibrated complex plane;|\endgroup\par
\begingroup\color{diffadd}\Verb|+this only relaxes the required angle budget.)|\endgroup\par
\begingroup\color{diffctx}\Verb| |\endgroup\par
\begingroup\color{diffctx}\Verb| \medskip\noindent|\endgroup\par
\begingroup\color{diffdel}\Verb|-\textbf{Substep 3.3: Choose sheet counts via Diophantine rounding.}|\endgroup\par
\begingroup\color{diffdel}\Verb|-For fixed $j$, all parallel copies of $\widetilde\Pi_j$ have identical|\endgroup\par
\begingroup\color{diffdel}\Verb|-$\psi$-mass $A_j>0$ in $Q$.  With $N_j$ sheets, the total mass in family|\endgroup\par
\begingroup\color{diffdel}\Verb|-$j$ is $N_jA_j$.  Define|\endgroup\par
\begingroup\color{diffdel}\Verb|-\[|\endgroup\par
\begingroup\color{diffdel}\Verb|-\lambda_j:=\frac{\theta_j}{A_j},\qquad \Lambda:=\sum_i\lambda_i.|\endgroup\par
\begingroup\color{diffdel}\Verb|-\]|\endgroup\par
\begingroup\color{diffdel}\Verb|-For large integer $m$, set|\endgroup\par
\begingroup\color{diffdel}\Verb|-\[|\endgroup\par
\begingroup\color{diffdel}\Verb|-N_j(m):=\Bigl\lfloor m\frac{\lambda_j}{\Lambda}\Bigr\rfloor.|\endgroup\par
\begingroup\color{diffdel}\Verb|-\]|\endgroup\par
\begingroup\color{diffdel}\Verb|-Standard rounding estimates give|\endgroup\par
\begingroup\color{diffdel}\Verb|-\[|\endgroup\par
\begingroup\color{diffdel}\Verb!-\Bigl|N_j(m)-m\frac{\lambda_j}{\Lambda}\Bigr|\le 1,!\endgroup\par
\begingroup\color{diffdel}\Verb|-\]|\endgroup\par
\begingroup\color{diffdel}\Verb|-and hence|\endgroup\par
\begingroup\color{diffdel}\Verb|-\[|\endgroup\par
\begingroup\color{diffdel}\Verb!-\Bigl|\frac{N_j(m)A_j}{\sum_i N_i(m)A_i}-\theta_j\Bigr|=O\Bigl(\frac{1}{m}\Bigr).!\endgroup\par
\begingroup\color{diffdel}\Verb|-\]|\endgroup\par
\begingroup\color{diffdel}\Verb|-Choose $m$ so large that this error is $<\delta$.|\endgroup\par
\begingroup\color{diffadd}\Verb|+\textbf{Substep 3.3: Choose sheet counts (mass-fraction rounding).}|\endgroup\par
\begingroup\color{diffadd}\Verb|+Write $k:=2n-2p$.|\endgroup\par
\begingroup\color{diffadd}\Verb|+For each $j$, fix a corner-exit translation template for direction $\Pi_j$ in $Q$ as supplied by Proposition~\ref{prop:corner-exit-template-net}.|\endgroup\par
\begingroup\color{diffadd}\Verb|+By the template property (Definition~\ref{def:global-vertex-template}), the corresponding flat footprints in $Q$ have equal $k$-dimensional mass; denote this common|\endgroup\par
\begingroup\color{diffadd}\Verb|+value by $A_j>0$ (so $A_j\asymp h^k$ by Lemma~\ref{lem:corner-exit-mass-scale}).|\endgroup\par
\begingroup\color{diffadd}\Verb|+|\endgroup\par
\begingroup\color{diffadd}\Verb|+Define $\lambda_j:=\theta_j/A_j$ and $\Lambda:=\sum_i\lambda_i$.|\endgroup\par
\begingroup\color{diffadd}\Verb|+Choose a large integer $N_{\mathrm{round}}$ and set|\endgroup\par
\begingroup\color{diffadd}\Verb|+\[|\endgroup\par
\begingroup\color{diffadd}\Verb|+N_j:=\Bigl\lfloor N_{\mathrm{round}}\frac{\lambda_j}{\Lambda}\Bigr\rfloor.|\endgroup\par
\begingroup\color{diffadd}\Verb|+\]|\endgroup\par
\begingroup\color{diffadd}\Verb!+Then $\bigl|\frac{N_jA_j}{\sum_i N_iA_i}-\theta_j\bigr|\le C/N_{\mathrm{round}}$.!\endgroup\par
\begingroup\color{diffadd}\Verb|+Choose $N_{\mathrm{round}}$ so large that this rounding error is $<\delta/10$.|\endgroup\par
\begingroup\color{diffctx}\Verb| |\endgroup\par
\begingroup\color{diffctx}\Verb| \medskip\noindent|\endgroup\par
\begingroup\color{diffctx}\Verb| \textbf{Substep 3.4: Build flat model sheets with disjoint translations.}|\endgroup\par
\begingroup\color{diffdel}\Verb|-In $\Phi(Q)\subset\C^n$, for each $j$, let $N_j^\perp$ be the complex|\endgroup\par
\begingroup\color{diffdel}\Verb|-$p$-dimensional normal space (the complex orthogonal complement of|\endgroup\par
\begingroup\color{diffdel}\Verb|-$\widetilde\Pi_j$), so that $\C^n=\widetilde\Pi_j\oplus N_j^\perp$.|\endgroup\par
\begingroup\color{diffdel}\Verb|-Pick distinct translation vectors|\endgroup\par
\begingroup\color{diffdel}\Verb|-$t_{j,1},\ldots,t_{j,N_j}\in N_j^\perp$ in a small ball $B(0,\rho)$|\endgroup\par
\begingroup\color{diffdel}\Verb|-with $\rho\ll\mathrm{diam}(Q)$, such that all affine spaces|\endgroup\par
\begingroup\color{diffdel}\Verb|-$\widetilde\Pi_j+t_{j,a}$ are pairwise disjoint on $\Phi(Q)$ as|\endgroup\par
\begingroup\color{diffdel}\Verb|-$(j,a)$ ranges over all indices.  This is possible since $N_j^\perp$|\endgroup\par
\begingroup\color{diffdel}\Verb|-has real dimension $2p\ge 2$ and we choose only finitely many points.|\endgroup\par
\begingroup\color{diffdel}\Verb|-|\endgroup\par
\begingroup\color{diffdel}\Verb|-Define|\endgroup\par
\begingroup\color{diffdel}\Verb|-\[|\endgroup\par
\begingroup\color{diffdel}\Verb|-\widetilde Y_j^a:=(\widetilde\Pi_j+t_{j,a})\cap\Phi(Q)\subset\C^n.|\endgroup\par
\begingroup\color{diffdel}\Verb|-\]|\endgroup\par
\begingroup\color{diffdel}\Verb|-These satisfy: (i) $\psi_0$-calibration (complex $(n-p)$-planes);|\endgroup\par
\begingroup\color{diffdel}\Verb|-(ii) $\sup_{y\in Q}\angle(T_y\widetilde Y_j^a,\Pi_j)|\endgroup\par
\begingroup\color{diffdel}\Verb|-=\angle(\widetilde\Pi_j,\Pi_j)<\varepsilon$;|\endgroup\par
\begingroup\color{diffdel}\Verb|-(iii) mass fractions within $\delta$ of $\theta_j$ by construction;|\endgroup\par
\begingroup\color{diffdel}\Verb|-(iv) pairwise disjoint on $\Phi(Q)$;|\endgroup\par
\begingroup\color{diffdel}\Verb|-(v) boundary supported on $\partial\Phi(Q)$.|\endgroup\par
\begingroup\color{diffadd}\Verb|+Choose pairwise distinct vertices $v_1,\dots,v_J$ of the cube $Q$.|\endgroup\par
\begingroup\color{diffadd}\Verb|+This is possible since $J$ is uniformly bounded (in applications, $J\le N(n,p)$ from Carath\'eodory) and a $2n$--dimensional cube has $2^{2n}$ vertices.|\endgroup\par
\begingroup\color{diffadd}\Verb|+For each $j$, use Proposition~\ref{prop:corner-exit-template-net} (applied to the chosen direction label $\Pi_j$ and vertex $v_j$) to obtain an ordered|\endgroup\par
\begingroup\color{diffadd}\Verb|+list of translation vectors $(t_{j,a})_{a\ge 1}\subset \Pi_j^\perp$ such that the affine planes|\endgroup\par
\begingroup\color{diffadd}\Verb|+\[|\endgroup\par
\begingroup\color{diffadd}\Verb|+P_{j,a}:=\Pi_j+v_j+t_{j,a}|\endgroup\par
\begingroup\color{diffadd}\Verb|+\]|\endgroup\par
\begingroup\color{diffadd}\Verb|+have identical corner-exit footprints|\endgroup\par
\begingroup\color{diffadd}\Verb|+\[|\endgroup\par
\begingroup\color{diffadd}\Verb|+E_{j,a}:=P_{j,a}\cap Q\subset B(v_j,c_0h),|\endgroup\par
\begingroup\color{diffadd}\Verb|+\]|\endgroup\par
\begingroup\color{diffadd}\Verb|+with $c_0<\tfrac12$, and are separated by $\dist(P_{j,a},P_{j,b})\ge 10\,\varepsilon_*\,h$ for $a\neq b$.|\endgroup\par
\begingroup\color{diffadd}\Verb|+By construction the footprint masses satisfy $\mathcal H^k(E_{j,a})\equiv A_j$ for each fixed $j$ (independent of $a$), and the supports are disjoint across|\endgroup\par
\begingroup\color{diffadd}\Verb|+all indices $(j,a)$ because different $j$ live in disjoint vertex balls while fixed $j$ are separated at scale $\varepsilon_*h$.|\endgroup\par
\begingroup\color{diffctx}\Verb| |\endgroup\par
\begingroup\color{diffctx}\Verb| \medskip\noindent|\endgroup\par
\begingroup\color{diffdel}\Verb|-\textbf{Substep 3.5: Upgrade to algebraic complete intersections.}|\endgroup\par
\begingroup\color{diffdel}\Verb|-Use Kodaira embedding and H\"ormander $L^2$-techniques: for large $k$,|\endgroup\par
\begingroup\color{diffdel}\Verb|-pick global sections $s_{j,a}^{(1)},\ldots,s_{j,a}^{(p)}\in H^0(X,L^k)$|\endgroup\par
\begingroup\color{diffdel}\Verb|-whose restrictions to $Q$ are $C^2$-close to the linear defining|\endgroup\par
\begingroup\color{diffdel}\Verb|-functions of $\widetilde Y_j^a$.  For $k$ large:|\endgroup\par
\begingroup\color{diffadd}\Verb|+\textbf{Substep 3.5: Holomorphic realization at the Bergman scale (L1).}|\endgroup\par
\begingroup\color{diffadd}\Verb|+Choose the line-bundle power $M$ large enough that:|\endgroup\par
\begingroup\color{diffctx}\Verb| \begin{itemize}|\endgroup\par
\begingroup\color{diffdel}\Verb|-\item $Y_j^a:=\{s_{j,a}^{(1)}=0\}\cap\cdots\cap\{s_{j,a}^{(p)}=0\}$|\endgroup\par
\begingroup\color{diffdel}\Verb|-is a smooth complex $(n-p)$-dimensional submanifold;|\endgroup\par
\begingroup\color{diffdel}\Verb|-\item On $Q$, $Y_j^a$ is $C^1$-close to $\widetilde Y_j^a$;|\endgroup\par
\begingroup\color{diffdel}\Verb|-\item Calibration, disjointness, and mass estimates persist under small|\endgroup\par
\begingroup\color{diffdel}\Verb|-$C^1$ perturbations.|\endgroup\par
\begingroup\color{diffadd}\Verb|+\item Lemma~\ref{lem:bergman-control} holds at tolerance $\varepsilon_*$;|\endgroup\par
\begingroup\color{diffadd}\Verb|+\item the cell diameter satisfies $h\le c\,M^{-1/2}$ (so Bergman-scale $C^1$ control holds uniformly on $Q$).|\endgroup\par
\begingroup\color{diffctx}\Verb| \end{itemize}|\endgroup\par
\begingroup\color{diffdel}\Verb|-Pulling back by $\Phi^{-1}$ gives the desired family on $Q$.|\endgroup\par
\begingroup\color{diffadd}\Verb|+For each direction label $j$, apply Proposition~\ref{prop:holomorphic-corner-exit-L1} to the list of affine template planes|\endgroup\par
\begingroup\color{diffadd}\Verb|+$P_{j,a}=\Pi_j+v_j+t_{j,a}$ ($a=1,\dots,N_j$) from Substep~3.4, at slope scale $\varepsilon_*$.|\endgroup\par
\begingroup\color{diffadd}\Verb|+This yields $\psi$--calibrated holomorphic complete intersections $Y_j^1,\dots,Y_j^{N_j}$ such that, on $Q$:|\endgroup\par
\begingroup\color{diffadd}\Verb|+\begin{enumerate}|\endgroup\par
\begingroup\color{diffadd}\Verb|+\item[\textnormal{(a)}] each $Y_j^a\cap Q$ is a single $C^1$ graph over $E_{j,a}=P_{j,a}\cap Q$ with slope $O(\varepsilon_*)$;|\endgroup\par
\begingroup\color{diffadd}\Verb|+\item[\textnormal{(b)}] the pieces $Y_j^a\cap Q$ are pairwise disjoint (by the separation in Substep~3.4 and Lemma~\ref{lem:sliver-stability}\textnormal{(ii)});|\endgroup\par
\begingroup\color{diffadd}\Verb|+\item[\textnormal{(c)}] $\Mass([Y_j^a]\llcorner Q)=\bigl(1+O(\varepsilon_*^2)\bigr)\,\mathcal H^{k}(E_{j,a})|\endgroup\par
\begingroup\color{diffadd}\Verb|+=\bigl(1+O(\varepsilon_*^2)\bigr)\,A_j$ (Lemma~\ref{lem:sliver-stability}\textnormal{(i)}).|\endgroup\par
\begingroup\color{diffadd}\Verb|+\end{enumerate}|\endgroup\par
\begingroup\color{diffadd}\Verb|+Since $\varepsilon_*\le \varepsilon/10$, this implies the angle control \textnormal{(i)}.|\endgroup\par
\begingroup\color{diffadd}\Verb|+|\endgroup\par
\begingroup\color{diffadd}\Verb|+For the mass fractions \textnormal{(ii)}, summing \textnormal{(c)} over $a$ gives|\endgroup\par
\begingroup\color{diffadd}\Verb|+\[|\endgroup\par
\begingroup\color{diffadd}\Verb|+\sum_{a=1}^{N_j}\Mass([Y_j^a]\llcorner Q)=\bigl(1+O(\varepsilon_*^2)\bigr)\,N_jA_j.|\endgroup\par
\begingroup\color{diffadd}\Verb|+\]|\endgroup\par
\begingroup\color{diffadd}\Verb|+Thus the normalized family masses differ from $\frac{N_jA_j}{\sum_i N_iA_i}$ by $O(\varepsilon_*^2)\le \delta/100$ (by the definition of $\varepsilon_*$),|\endgroup\par
\begingroup\color{diffadd}\Verb|+and they differ from $\theta_j$ by at most $\delta/10+\delta/100<\delta$ using the rounding choice in Substep~3.3.|\endgroup\par
\begingroup\color{diffadd}\Verb|+|\endgroup\par
\begingroup\color{diffadd}\Verb|+Finally, since each $Y_j^a$ is a closed complex submanifold of $X$, restricting to $Q$ gives $\partial([Y_j^a]\llcorner Q)$ supported on $\partial Q$,|\endgroup\par
\begingroup\color{diffadd}\Verb|+proving \textnormal{(iv)}.|\endgroup\par
\begingroup\color{diffctx}\Verb| \end{proof}|\endgroup\par
\begingroup\color{diffctx}\Verb| |\endgroup\par
\begingroup\color{diffctx}\Verb| Fix a finite normal coordinate atlas by geodesic balls of radii $\ll 1$|\endgroup\par
\begingroup\color{diffhunk}\Verb|@@ -2966,8 +3014,8 @@|\endgroup\par
\begingroup\color{diffctx}\Verb| \begin{theorem}[Global cohomology quantization]\label{thm:global-cohom}|\endgroup\par
\begingroup\color{diffctx}\Verb| Let $X$ be a compact K\"ahler $n$-fold with rational Hodge class|\endgroup\par
\begingroup\color{diffctx}\Verb| $[\gamma]\in H^{2p}(X,\Q)$ represented by a smooth closed $(p,p)$-form|\endgroup\par
\begingroup\color{diffdel}\Verb|-$\beta$ with $\beta(x)\in K_p(x)$ pointwise.  Let $\{Q\}$ be a cube|\endgroup\par
\begingroup\color{diffdel}\Verb|-partition of $X$.  Then there exists an integer $m\ge 1$ (clearing denominators of|\endgroup\par
\begingroup\color{diffadd}\Verb|+$\beta$ with $\beta(x)\in K_p(x)$ pointwise.  Let $\{Q\}$ be a partition of $X$ into smooth uniformly convex cells|\endgroup\par
\begingroup\color{diffadd}\Verb|+(e.g.\ rounded coordinate cubes) of sufficiently small mesh.  Then there exists an integer $m\ge 1$ (clearing denominators of|\endgroup\par
\begingroup\color{diffctx}\Verb| $[\gamma]$) such that for every $\varepsilon>0$ there exist:|\endgroup\par
\begingroup\color{diffctx}\Verb| \begin{itemize}|\endgroup\par
\begingroup\color{diffctx}\Verb| \item A closed integral $(2n-2p)$-current $T_\varepsilon$ with|\endgroup\par
\begingroup\color{diffhunk}\Verb|@@ -3064,7 +3112,7 @@|\endgroup\par
\begingroup\color{diffctx}\Verb| $S(\eta)=\partial T^{\mathrm{raw}}(\eta)=T^{\mathrm{raw}}(d\eta)$.|\endgroup\par
\begingroup\color{diffctx}\Verb| |\endgroup\par
\begingroup\color{diffctx}\Verb| \begin{proposition}[Transport control $\Rightarrow$ flat-norm gluing]\label{prop:transport-flat-glue}|\endgroup\par
\begingroup\color{diffdel}\Verb|-Fix a cubulation of $X$ by coordinate cubes of side length $h=\mathrm{mesh}$, and write|\endgroup\par
\begingroup\color{diffadd}\Verb|+Fix a decomposition of $X$ into smooth uniformly convex cells (e.g.\ rounded coordinate cubes) of diameter $h=\mathrm{mesh}$, and write|\endgroup\par
\begingroup\color{diffctx}\Verb| $T^{\mathrm{raw}}=\sum_Q S_Q$ as above, where each $S_Q$ is a sum of calibrated sheets restricted to $Q$.|\endgroup\par
\begingroup\color{diffctx}\Verb| Assume the following \emph{geometric parameterization} holds on each interior face $F=Q\cap Q'$:|\endgroup\par
\begingroup\color{diffctx}\Verb| \begin{enumerate}|\endgroup\par
\begingroup\color{diffhunk}\Verb|@@ -3074,6 +3122,10 @@|\endgroup\par
\begingroup\color{diffctx}\Verb| \item[\textnormal{(b)}] (\textbf{Transverse measures on faces}) After identifying a tubular neighborhood of $F$ with a product|\endgroup\par
\begingroup\color{diffctx}\Verb| $F\times B^{2p}(0,ch)$ in normal coordinates, the restriction of $\partial S_Q$ to $F$ can be written as a finite sum of translated|\endgroup\par
\begingroup\color{diffctx}\Verb| slice currents parameterized by a discrete transverse measure $\mu_{Q\to F}$ on $B^{2p}(0,ch)$ (integer weights), and similarly for $Q'$.|\endgroup\par
\begingroup\color{diffadd}\Verb|+In addition, we assume each slice current $\Sigma_y$ in this parameterization is a \emph{cycle in the face chart}:|\endgroup\par
\begingroup\color{diffadd}\Verb|+\[|\endgroup\par
\begingroup\color{diffadd}\Verb|+\partial\Sigma_y=0.|\endgroup\par
\begingroup\color{diffadd}\Verb|+\]|\endgroup\par
\begingroup\color{diffctx}\Verb| \item[\textnormal{(c)}] (\textbf{$W_1$ face matching}) The two induced transverse measures have the same total mass and satisfy|\endgroup\par
\begingroup\color{diffctx}\Verb| \[|\endgroup\par
\begingroup\color{diffctx}\Verb| W_1(\mu_{Q\to F},\mu_{Q'\to F})\ \le\ \tau_F,|\endgroup\par
\begingroup\color{diffhunk}\Verb|@@ -3112,17 +3164,24 @@|\endgroup\par
\begingroup\color{diffctx}\Verb| In the flat/parallel model (i.e.\ when $\Sigma_{y'}=(\tau_v)_\#\Sigma_y$ inside the product chart), consider the straight-line homotopy|\endgroup\par
\begingroup\color{diffctx}\Verb| $H:[0,1]\times F\to F\times B^{2p}(0,ch)$, $H(t,x)=(x,y+t v)$.|\endgroup\par
\begingroup\color{diffctx}\Verb| Let $Q_{y\to y'}:=H_\#([0,1]\times \Sigma_y)$.|\endgroup\par
\begingroup\color{diffdel}\Verb|-Then $\partial Q_{y\to y'}=\Sigma_{y'}-\Sigma_y$ and|\endgroup\par
\begingroup\color{diffdel}\Verb|-\[|\endgroup\par
\begingroup\color{diffdel}\Verb!-\Mass(Q_{y\to y'})\ \le\ \|v\|\,\Mass(\Sigma_y).!\endgroup\par
\begingroup\color{diffdel}\Verb|-\]|\endgroup\par
\begingroup\color{diffdel}\Verb|-By Stokes and the comass bound on $d\eta$,|\endgroup\par
\begingroup\color{diffadd}\Verb|+Set also $R_{y\to y'}:=H_\#([0,1]\times \partial\Sigma_y)$.|\endgroup\par
\begingroup\color{diffadd}\Verb|+Then|\endgroup\par
\begingroup\color{diffadd}\Verb|+\[|\endgroup\par
\begingroup\color{diffadd}\Verb|+\Sigma_{y'}-\Sigma_y\ =\ R_{y\to y'}+\partial Q_{y\to y'},|\endgroup\par
\begingroup\color{diffadd}\Verb|+\qquad|\endgroup\par
\begingroup\color{diffadd}\Verb!+\Mass(Q_{y\to y'})\ \le\ \|v\|\,\Mass(\Sigma_y),!\endgroup\par
\begingroup\color{diffadd}\Verb|+\qquad|\endgroup\par
\begingroup\color{diffadd}\Verb!+\Mass(R_{y\to y'})\ \le\ \|v\|\,\Mass(\partial\Sigma_y).!\endgroup\par
\begingroup\color{diffadd}\Verb|+\]|\endgroup\par
\begingroup\color{diffadd}\Verb|+By Stokes and the comass bounds on $\eta$ and $d\eta$,|\endgroup\par
\begingroup\color{diffctx}\Verb| \[|\endgroup\par
\begingroup\color{diffctx}\Verb! |f_\eta(y')-f_\eta(y)|!\endgroup\par
\begingroup\color{diffdel}\Verb!-=|(\Sigma_{y'}-\Sigma_y)(\eta)|!\endgroup\par
\begingroup\color{diffdel}\Verb!-=|Q_{y\to y'}(d\eta)|!\endgroup\par
\begingroup\color{diffdel}\Verb!-\le \Mass(Q_{y\to y'})\|d\eta\|_{\mathrm{comass}}!\endgroup\par
\begingroup\color{diffdel}\Verb!-\le \|v\|\,\Mass(\Sigma_y).!\endgroup\par
\begingroup\color{diffadd}\Verb!+=|Q_{y\to y'}(d\eta)+R_{y\to y'}(\eta)|!\endgroup\par
\begingroup\color{diffadd}\Verb!+\le \|v\|\Bigl(\Mass(\Sigma_y)+\Mass(\partial\Sigma_y)\Bigr).!\endgroup\par
\begingroup\color{diffadd}\Verb|+\]|\endgroup\par
\begingroup\color{diffadd}\Verb|+In our setting $\partial\Sigma_y=0$ (hypothesis \textnormal{(b)}), hence $R_{y\to y'}=0$ and the preceding estimate simplifies to|\endgroup\par
\begingroup\color{diffadd}\Verb|+\[|\endgroup\par
\begingroup\color{diffadd}\Verb!+|f_\eta(y')-f_\eta(y)|\ \le\ \|v\|\,\Mass(\Sigma_y).!\endgroup\par
\begingroup\color{diffctx}\Verb| \]|\endgroup\par
\begingroup\color{diffctx}\Verb| Under the small-angle graph hypothesis \textnormal{(a)} and bounded geometry of the chart, each slice has mass|\endgroup\par
\begingroup\color{diffctx}\Verb| $\Mass(\Sigma_y)\le C\,h^{2n-2p-1}$ with $C=C(n,p,X)$.|\endgroup\par
\begingroup\color{diffhunk}\Verb|@@ -3292,7 +3351,7 @@|\endgroup\par
\begingroup\color{diffctx}\Verb| \[|\endgroup\par
\begingroup\color{diffctx}\Verb| \mathcal F(\partial T^{\mathrm{raw}})\ \lesssim\ m\,h \;+\; O(\varepsilon\,m),|\endgroup\par
\begingroup\color{diffctx}\Verb| \]|\endgroup\par
\begingroup\color{diffdel}\Verb|-so choosing $h=h(m)\to 0$ slowly (e.g.\ $h=m^{-\alpha}$ with $\alpha>0$ small) makes the gluing correction $R_{\mathrm{glue}}$|\endgroup\par
\begingroup\color{diffadd}\Verb|+so choosing $h=h(m)\to 0$ slowly (e.g.\ $h=m^{-\alpha}$ with $\alpha>0$ small) makes the gluing correction $U_h$|\endgroup\par
\begingroup\color{diffctx}\Verb| sublinear in $m$ and hence negligible in the mass equality as $m\to\infty$.|\endgroup\par
\begingroup\color{diffctx}\Verb| The remaining task is then to implement this “fixed template” choice while still meeting the cohomological constraints (Substep 4.3).|\endgroup\par
\begingroup\color{diffctx}\Verb| \smallskip|\endgroup\par
\begingroup\color{diffhunk}\Verb|@@ -3324,45 +3383,62 @@|\endgroup\par
\begingroup\color{diffctx}\Verb| \begin{editblock}|\endgroup\par
\begingroup\color{diffctx}\Verb| \begin{proposition}[Weighted transport $\Rightarrow$ flat-norm face control (sliver-compatible)]\label{prop:transport-flat-glue-weighted}|\endgroup\par
\begingroup\color{diffctx}\Verb| Work in the tubular/flat model on an interior face $F=Q\cap Q'$.|\endgroup\par
\begingroup\color{diffdel}\Verb|-Assume each sheet piece meeting $F$ contributes a \emph{cycle slice} current $\Sigma(u)$ on $F$ depending on a transverse parameter|\endgroup\par
\begingroup\color{diffadd}\Verb|+Assume each sheet piece meeting $F$ contributes an integral slice current $\Sigma(u)$ on $F$ depending on a transverse parameter|\endgroup\par
\begingroup\color{diffctx}\Verb| $u\in\Omega_F\subset\R^{2p}$, and that $\Sigma(u)$ is obtained from $\Sigma(0)$ by translation in the face chart.|\endgroup\par
\begingroup\color{diffdel}\Verb|-Let the two adjacent cubes induce two multisets of parameters $\{u_a\}_{a=1}^N$ and $\{u'_a\}_{a=1}^N$ (same cardinality), hence two face currents|\endgroup\par
\begingroup\color{diffdel}\Verb|-\[|\endgroup\par
\begingroup\color{diffdel}\Verb|-S_{Q\to F}:=\sum_{a=1}^N \Sigma(u_a),\qquad|\endgroup\par
\begingroup\color{diffdel}\Verb|-S_{Q'\to F}:=\sum_{a=1}^N \Sigma(u'_a),|\endgroup\par
\begingroup\color{diffadd}\Verb|+Let the two adjacent cubes induce two \emph{integer-weighted} discrete measures on $\Omega_F$ of the same total mass,|\endgroup\par
\begingroup\color{diffadd}\Verb|+\[|\endgroup\par
\begingroup\color{diffadd}\Verb|+\mu:=\sum_{a=1}^{A} w_a\,\delta_{u_a},|\endgroup\par
\begingroup\color{diffadd}\Verb|+\qquad|\endgroup\par
\begingroup\color{diffadd}\Verb|+\mu':=\sum_{b=1}^{B} w'_b\,\delta_{u'_b},|\endgroup\par
\begingroup\color{diffadd}\Verb|+\qquad|\endgroup\par
\begingroup\color{diffadd}\Verb|+w_a,w'_b\in\Z_{\ge 0},|\endgroup\par
\begingroup\color{diffadd}\Verb|+\qquad|\endgroup\par
\begingroup\color{diffadd}\Verb|+\mu(\Omega_F)=\mu'(\Omega_F)=:N,|\endgroup\par
\begingroup\color{diffadd}\Verb|+\]|\endgroup\par
\begingroup\color{diffadd}\Verb|+hence two face currents|\endgroup\par
\begingroup\color{diffadd}\Verb|+\[|\endgroup\par
\begingroup\color{diffadd}\Verb|+S_{Q\to F}:=\int_{\Omega_F}\Sigma(u)\,d\mu(u),\qquad|\endgroup\par
\begingroup\color{diffadd}\Verb|+S_{Q'\to F}:=\int_{\Omega_F}\Sigma(u)\,d\mu'(u),|\endgroup\par
\begingroup\color{diffctx}\Verb| \qquad|\endgroup\par
\begingroup\color{diffctx}\Verb| B_F:=S_{Q\to F}-S_{Q'\to F}.|\endgroup\par
\begingroup\color{diffctx}\Verb| \]|\endgroup\par
\begingroup\color{diffctx}\Verb| Then|\endgroup\par
\begingroup\color{diffctx}\Verb| \[|\endgroup\par
\begingroup\color{diffdel}\Verb!-\mathcal F(B_F)\ \le\ \inf_{\sigma\in S_N}\ \sum_{a=1}^N \|u_a-u'_{\sigma(a)}\|\,\Mass(\Sigma(u_a)).!\endgroup\par
\begingroup\color{diffdel}\Verb|-\]|\endgroup\par
\begingroup\color{diffdel}\Verb|-In particular, if $\Mass(\Sigma(u_a))\le b_F$ for all $a$ and if|\endgroup\par
\begingroup\color{diffdel}\Verb|-\[|\endgroup\par
\begingroup\color{diffdel}\Verb!-\tau_F:=\inf_{\sigma\in S_N}\ \sum_{a=1}^N \|u_a-u'_{\sigma(a)}\|!\endgroup\par
\begingroup\color{diffdel}\Verb|-\]|\endgroup\par
\begingroup\color{diffdel}\Verb|-(the equal-weight matching cost, i.e.\ $W_1$ of the counting measures), then|\endgroup\par
\begingroup\color{diffdel}\Verb|-\[|\endgroup\par
\begingroup\color{diffdel}\Verb|-\mathcal F(B_F)\ \le\ b_F\,\tau_F.|\endgroup\par
\begingroup\color{diffadd}\Verb!+\mathcal F(B_F)\ \le\ \inf_{\pi\in\Gamma(\mu,\mu')}\ \int_{\Omega_F\times\Omega_F}\|u-u'\|\Bigl(\Mass(\Sigma(u))+\Mass(\partial\Sigma(u))\Bigr)\,d\pi(u,u'),!\endgroup\par
\begingroup\color{diffadd}\Verb|+\]|\endgroup\par
\begingroup\color{diffadd}\Verb|+where $\Gamma(\mu,\mu')$ is the set of couplings between $\mu$ and $\mu'$.|\endgroup\par
\begingroup\color{diffadd}\Verb|+In particular, if $\Mass(\Sigma(u))+\Mass(\partial\Sigma(u))\le b_F$ for all $u\in\Omega_F$, then|\endgroup\par
\begingroup\color{diffadd}\Verb|+\[|\endgroup\par
\begingroup\color{diffadd}\Verb|+\mathcal F(B_F)\ \le\ b_F\,W_1(\mu,\mu').|\endgroup\par
\begingroup\color{diffadd}\Verb|+\]|\endgroup\par
\begingroup\color{diffadd}\Verb|+In the special case $\mu=\sum_{a=1}^N\delta_{u_a}$ and $\mu'=\sum_{a=1}^N\delta_{u'_a}$ (equal weights, same cardinality),|\endgroup\par
\begingroup\color{diffadd}\Verb|+the first bound reduces to the permutation formula|\endgroup\par
\begingroup\color{diffadd}\Verb|+\[|\endgroup\par
\begingroup\color{diffadd}\Verb!+\mathcal F(B_F)\ \le\ \inf_{\sigma\in S_N}\ \sum_{a=1}^N \|u_a-u'_{\sigma(a)}\|\Bigl(\Mass(\Sigma(u_a))+\Mass(\partial\Sigma(u_a))\Bigr).!\endgroup\par
\begingroup\color{diffctx}\Verb| \]|\endgroup\par
\begingroup\color{diffctx}\Verb| \end{proposition}|\endgroup\par
\begingroup\color{diffctx}\Verb| |\endgroup\par
\begingroup\color{diffctx}\Verb| |\endgroup\par
\begingroup\color{diffctx}\Verb| \begin{proof}|\endgroup\par
\begingroup\color{diffdel}\Verb|-Fix a permutation $\sigma\in S_N$.|\endgroup\par
\begingroup\color{diffdel}\Verb|-For each index $a$, the difference $\Sigma(u_a)-\Sigma(u'_{\sigma(a)})$ is the difference of two translated copies of the same|\endgroup\par
\begingroup\color{diffdel}\Verb|-integral cycle in the face chart, hence is itself a boundary.|\endgroup\par
\begingroup\color{diffdel}\Verb|-By Lemma~\ref{lem:flat-translate} there exists an integral filling current $Q_a$ with|\endgroup\par
\begingroup\color{diffdel}\Verb|-\[|\endgroup\par
\begingroup\color{diffdel}\Verb|-\partial Q_a=\Sigma(u_a)-\Sigma(u'_{\sigma(a)})|\endgroup\par
\begingroup\color{diffadd}\Verb|+Fix a coupling $\pi\in\Gamma(\mu,\mu')$.|\endgroup\par
\begingroup\color{diffadd}\Verb|+Since $\mu$ and $\mu'$ have integer weights and equal total mass $N$, we may realize $\pi$ as an integer-valued transport plan:|\endgroup\par
\begingroup\color{diffadd}\Verb|+equivalently, after expanding each atom $u_a$ into $w_a$ copies and each atom $u'_b$ into $w'_b$ copies, $\pi$ corresponds to a matching of two|\endgroup\par
\begingroup\color{diffadd}\Verb|+lists of length $N$.|\endgroup\par
\begingroup\color{diffadd}\Verb|+Write the matched pairs as $(u^{(i)},u'^{(i)})$ for $i=1,\dots,N$.|\endgroup\par
\begingroup\color{diffadd}\Verb|+|\endgroup\par
\begingroup\color{diffadd}\Verb|+For each $i$, apply Lemma~\ref{lem:flat-translate} in the face chart to the translated pair $\Sigma(u^{(i)})$ and $\Sigma(u'^{(i)})$.|\endgroup\par
\begingroup\color{diffadd}\Verb|+This yields integral currents $R_i$ and $Q_i$ such that|\endgroup\par
\begingroup\color{diffadd}\Verb|+\[|\endgroup\par
\begingroup\color{diffadd}\Verb|+\Sigma(u^{(i)})-\Sigma(u'^{(i)})\ =\ R_i+\partial Q_i|\endgroup\par
\begingroup\color{diffctx}\Verb| \qquad\text{and}\qquad|\endgroup\par
\begingroup\color{diffdel}\Verb!-\Mass(Q_a)\ \le\ \|u_a-u'_{\sigma(a)}\|\,\Mass(\Sigma(u_a)).!\endgroup\par
\begingroup\color{diffdel}\Verb|-\]|\endgroup\par
\begingroup\color{diffdel}\Verb|-Summing $Q:=\sum_{a=1}^N Q_a$ yields $\partial Q=B_F$ and|\endgroup\par
\begingroup\color{diffdel}\Verb|-\[|\endgroup\par
\begingroup\color{diffdel}\Verb!-\Mass(Q)\ \le\ \sum_{a=1}^N \|u_a-u'_{\sigma(a)}\|\,\Mass(\Sigma(u_a)).!\endgroup\par
\begingroup\color{diffdel}\Verb|-\]|\endgroup\par
\begingroup\color{diffdel}\Verb|-Taking $R:=0$ in the definition of the flat norm gives $\mathcal F(B_F)\le \Mass(Q)$, and then taking the infimum over $\sigma$ proves the claim.|\endgroup\par
\begingroup\color{diffadd}\Verb!+\Mass(R_i)+\Mass(Q_i)\ \le\ \|u^{(i)}-u'^{(i)}\|\Bigl(\Mass(\Sigma(u^{(i)}))+\Mass(\partial\Sigma(u^{(i)}))\Bigr).!\endgroup\par
\begingroup\color{diffadd}\Verb|+\]|\endgroup\par
\begingroup\color{diffadd}\Verb|+Summing $R:=\sum_{i=1}^N R_i$ and $Q:=\sum_{i=1}^N Q_i$ gives $B_F=R+\partial Q$ and|\endgroup\par
\begingroup\color{diffadd}\Verb|+\[|\endgroup\par
\begingroup\color{diffadd}\Verb|+\mathcal F(B_F)\ \le\ \Mass(R)+\Mass(Q)|\endgroup\par
\begingroup\color{diffadd}\Verb!+\ \le\ \sum_{i=1}^N \|u^{(i)}-u'^{(i)}\|\Bigl(\Mass(\Sigma(u^{(i)}))+\Mass(\partial\Sigma(u^{(i)}))\Bigr).!\endgroup\par
\begingroup\color{diffadd}\Verb|+\]|\endgroup\par
\begingroup\color{diffadd}\Verb|+Interpreting the sum as the $\pi$-integral yields the stated bound.  Taking the infimum over $\pi\in\Gamma(\mu,\mu')$ proves the proposition.|\endgroup\par
\begingroup\color{diffctx}\Verb| \end{proof}|\endgroup\par
\begingroup\color{diffctx}\Verb| |\endgroup\par
\begingroup\color{diffctx}\Verb| \end{editblock}|\endgroup\par
\begingroup\color{diffhunk}\Verb|@@ -3477,12 +3553,22 @@|\endgroup\par
\begingroup\color{diffctx}\Verb| |\endgroup\par
\begingroup\color{diffctx}\Verb| \begin{editblock}|\endgroup\par
\begingroup\color{diffctx}\Verb| \begin{lemma}[Flat-norm stability under translation]\label{lem:flat-translate}|\endgroup\par
\begingroup\color{diffdel}\Verb|-Let $S$ be an integral $\ell$-cycle in $\R^d$ (so $\partial S=0$) with finite mass.|\endgroup\par
\begingroup\color{diffadd}\Verb|+Let $S$ be an integral $\ell$--current in $\R^d$ with finite mass and finite boundary mass.|\endgroup\par
\begingroup\color{diffctx}\Verb| For any translation vector $v\in\R^d$, write $\tau_v(x):=x+v$ and $(\tau_v)_\#S$ for the pushforward.|\endgroup\par
\begingroup\color{diffdel}\Verb|-Then|\endgroup\par
\begingroup\color{diffdel}\Verb|-\[|\endgroup\par
\begingroup\color{diffdel}\Verb;-\mathcal F\!\bigl((\tau_v)_\#S-S\bigr)\ \le\ \|v\|\,\Mass(S).;\endgroup\par
\begingroup\color{diffdel}\Verb|-\]|\endgroup\par
\begingroup\color{diffadd}\Verb|+Then there exist integral currents $Q$ (of dimension $\ell+1$) and $R$ (of dimension $\ell$) such that|\endgroup\par
\begingroup\color{diffadd}\Verb|+\[|\endgroup\par
\begingroup\color{diffadd}\Verb|+(\tau_v)_\#S-S\ =\ R+\partial Q,|\endgroup\par
\begingroup\color{diffadd}\Verb|+\qquad|\endgroup\par
\begingroup\color{diffadd}\Verb!+\Mass(Q)\ \le\ \|v\|\,\Mass(S),!\endgroup\par
\begingroup\color{diffadd}\Verb|+\qquad|\endgroup\par
\begingroup\color{diffadd}\Verb!+\Mass(R)\ \le\ \|v\|\,\Mass(\partial S).!\endgroup\par
\begingroup\color{diffadd}\Verb|+\]|\endgroup\par
\begingroup\color{diffadd}\Verb|+Consequently|\endgroup\par
\begingroup\color{diffadd}\Verb|+\[|\endgroup\par
\begingroup\color{diffadd}\Verb;+\mathcal F\!\bigl((\tau_v)_\#S-S\bigr)\ \le\ \|v\|\Bigl(\Mass(S)+\Mass(\partial S)\Bigr).;\endgroup\par
\begingroup\color{diffadd}\Verb|+\]|\endgroup\par
\begingroup\color{diffadd}\Verb|+In particular, if $S$ is a cycle ($\partial S=0$) one may take $R=0$ and this reduces to|\endgroup\par
\begingroup\color{diffadd}\Verb!+$\mathcal F((\tau_v)_\#S-S)\le \|v\|\,\Mass(S)$.!\endgroup\par
\begingroup\color{diffctx}\Verb| \end{lemma}|\endgroup\par
\begingroup\color{diffctx}\Verb| |\endgroup\par
\begingroup\color{diffctx}\Verb| |\endgroup\par
\begingroup\color{diffhunk}\Verb|@@ -3492,20 +3578,26 @@|\endgroup\par
\begingroup\color{diffctx}\Verb| \(|\endgroup\par
\begingroup\color{diffctx}\Verb| Q:=H_\#([0,1]\times S).|\endgroup\par
\begingroup\color{diffctx}\Verb| \)|\endgroup\par
\begingroup\color{diffdel}\Verb|-Since $\partial([0,1]\times S)=\{1\}\times S-\{0\}\times S-[0,1]\times \partial S$ and $\partial S=0$, we have|\endgroup\par
\begingroup\color{diffadd}\Verb|+Set also|\endgroup\par
\begingroup\color{diffadd}\Verb|+\(|\endgroup\par
\begingroup\color{diffadd}\Verb|+R:=H_\#([0,1]\times \partial S).|\endgroup\par
\begingroup\color{diffadd}\Verb|+\)|\endgroup\par
\begingroup\color{diffadd}\Verb|+Since $\partial([0,1]\times S)=\{1\}\times S-\{0\}\times S-[0,1]\times \partial S$, we have|\endgroup\par
\begingroup\color{diffctx}\Verb| \[|\endgroup\par
\begingroup\color{diffctx}\Verb| \partial Q|\endgroup\par
\begingroup\color{diffdel}\Verb|-=H_\#(\{1\}\times S)-H_\#(\{0\}\times S)|\endgroup\par
\begingroup\color{diffdel}\Verb|-=(\tau_v)_\#S-S.|\endgroup\par
\begingroup\color{diffdel}\Verb|-\]|\endgroup\par
\begingroup\color{diffadd}\Verb|+=H_\#(\{1\}\times S)-H_\#(\{0\}\times S)-H_\#([0,1]\times \partial S)|\endgroup\par
\begingroup\color{diffadd}\Verb|+=(\tau_v)_\#S-S-R.|\endgroup\par
\begingroup\color{diffadd}\Verb|+\]|\endgroup\par
\begingroup\color{diffadd}\Verb|+Thus $(\tau_v)_\#S-S=R+\partial Q$.|\endgroup\par
\begingroup\color{diffctx}\Verb! Moreover, $H$ has Jacobian bounded by $\|v\|$ in the $t$-direction, so the mass estimate for pushforwards gives!\endgroup\par
\begingroup\color{diffctx}\Verb| \(|\endgroup\par
\begingroup\color{diffctx}\Verb! \Mass(Q)\le \|v\|\,\Mass(S).!\endgroup\par
\begingroup\color{diffctx}\Verb| \)|\endgroup\par
\begingroup\color{diffdel}\Verb|-Taking $R:=0$ in the definition of $\mathcal F$ yields|\endgroup\par
\begingroup\color{diffdel}\Verb|-\(|\endgroup\par
\begingroup\color{diffdel}\Verb!-\mathcal F((\tau_v)_\#S-S)\le \Mass(Q)\le \|v\|\,\Mass(S),!\endgroup\par
\begingroup\color{diffdel}\Verb|-\)|\endgroup\par
\begingroup\color{diffadd}\Verb!+Likewise $\Mass(R)\le \|v\|\,\Mass(\partial S)$.!\endgroup\par
\begingroup\color{diffadd}\Verb|+Taking these $R,Q$ in the definition of $\mathcal F$ yields|\endgroup\par
\begingroup\color{diffadd}\Verb|+\[|\endgroup\par
\begingroup\color{diffadd}\Verb!+\mathcal F((\tau_v)_\#S-S)\le \Mass(R)+\Mass(Q)\le \|v\|\Bigl(\Mass(S)+\Mass(\partial S)\Bigr),!\endgroup\par
\begingroup\color{diffadd}\Verb|+\]|\endgroup\par
\begingroup\color{diffctx}\Verb| as claimed.|\endgroup\par
\begingroup\color{diffctx}\Verb| \end{proof}|\endgroup\par
\begingroup\color{diffctx}\Verb| |\endgroup\par
\begingroup\color{diffhunk}\Verb|@@ -3518,31 +3610,31 @@|\endgroup\par
\begingroup\color{diffctx}\Verb| \[|\endgroup\par
\begingroup\color{diffctx}\Verb| B_F:=\bigl(\partial S_Q\bigr)\llcorner F\ -\ \bigl(\partial S_{Q'}\bigr)\llcorner F|\endgroup\par
\begingroup\color{diffctx}\Verb| \]|\endgroup\par
\begingroup\color{diffdel}\Verb|-fits the translation model of Proposition~\ref{prop:transport-flat-glue-weighted} with parameter multisets|\endgroup\par
\begingroup\color{diffdel}\Verb|-$\{u_a\}_{a=1}^N$ and $\{u'_a\}_{a=1}^N$.|\endgroup\par
\begingroup\color{diffdel}\Verb|-If there exists a matching $\sigma\in S_N$ with a uniform displacement bound|\endgroup\par
\begingroup\color{diffdel}\Verb|-\[|\endgroup\par
\begingroup\color{diffdel}\Verb!-\|u_a-u'_{\sigma(a)}\|\ \le\ \Delta_F\qquad\text{for all }a,!\endgroup\par
\begingroup\color{diffadd}\Verb|+fits the translation model of Proposition~\ref{prop:transport-flat-glue-weighted} with integer-weighted parameter measures|\endgroup\par
\begingroup\color{diffadd}\Verb|+$\mu_{Q\to F}$ and $\mu_{Q'\to F}$ on $\Omega_F\subset\R^{2p}$.|\endgroup\par
\begingroup\color{diffadd}\Verb|+If there exists a coupling $\pi_F\in\Gamma(\mu_{Q\to F},\mu_{Q'\to F})$ supported on pairs with a uniform displacement bound|\endgroup\par
\begingroup\color{diffadd}\Verb|+\[|\endgroup\par
\begingroup\color{diffadd}\Verb!+\|u-u'\|\ \le\ \Delta_F\qquad\text{for $\pi_F$--a.e.\ }(u,u'),!\endgroup\par
\begingroup\color{diffctx}\Verb| \]|\endgroup\par
\begingroup\color{diffctx}\Verb| then|\endgroup\par
\begingroup\color{diffctx}\Verb| \[|\endgroup\par
\begingroup\color{diffdel}\Verb|-\mathcal F(B_F)\ \le\ \Delta_F\sum_{a=1}^N \Mass(\Sigma(u_a)).|\endgroup\par
\begingroup\color{diffadd}\Verb|+\mathcal F(B_F)\ \le\ \Delta_F\int_{\Omega_F}\Bigl(\Mass(\Sigma(u))+\Mass(\partial\Sigma(u))\Bigr)\,d\mu_{Q\to F}(u).|\endgroup\par
\begingroup\color{diffctx}\Verb| \]|\endgroup\par
\begingroup\color{diffctx}\Verb| Consequently,|\endgroup\par
\begingroup\color{diffctx}\Verb| \[|\endgroup\par
\begingroup\color{diffctx}\Verb| \mathcal F\!\left(\partial T^{\mathrm{raw}}\right)|\endgroup\par
\begingroup\color{diffctx}\Verb| \ \le\ \sum_F \mathcal F(B_F)|\endgroup\par
\begingroup\color{diffdel}\Verb|-\ \le\ \sum_F \Delta_F\sum_{a\in\mathcal S(F)} \Mass(\Sigma_F(u_a)),|\endgroup\par
\begingroup\color{diffdel}\Verb|-\]|\endgroup\par
\begingroup\color{diffdel}\Verb|-where $\mathcal S(F)$ indexes the pieces meeting the interface $F$.|\endgroup\par
\begingroup\color{diffdel}\Verb|-|\endgroup\par
\begingroup\color{diffdel}\Verb|-If moreover $\Delta_F\le C\,h^2$ for all interfaces and each slice $\Sigma_F(u_a)$ arises as the interface boundary slice of a piece|\endgroup\par
\begingroup\color{diffadd}\Verb|+\ \le\ \sum_F \Delta_F\int_{\Omega_F}\Bigl(\Mass(\Sigma_F(u))+\Mass(\partial\Sigma_F(u))\Bigr)\,d\mu_{Q\to F}(u),|\endgroup\par
\begingroup\color{diffadd}\Verb|+\]|\endgroup\par
\begingroup\color{diffadd}\Verb|+where, for each $F$, the integral is the corresponding integer-weighted sum over pieces meeting the interface.|\endgroup\par
\begingroup\color{diffadd}\Verb|+|\endgroup\par
\begingroup\color{diffadd}\Verb|+If moreover $\Delta_F\le C\,h^2$ for all interfaces and each slice $\Sigma_F(u)$ arises as the interface boundary slice of a piece|\endgroup\par
\begingroup\color{diffctx}\Verb| $Y^a\cap Q$ with interior mass $m_a:=\Mass([Y^a]\llcorner Q)$, then Lemma~\ref{lem:uniformly-convex-slice-boundary} gives|\endgroup\par
\begingroup\color{diffctx}\Verb| \[|\endgroup\par
\begingroup\color{diffdel}\Verb|-\Mass(\Sigma_F(u_a))\ \lesssim\ m_a^{\frac{k-1}{k}},|\endgroup\par
\begingroup\color{diffadd}\Verb|+\Mass(\Sigma_F(u))\ \lesssim\ m_a^{\frac{k-1}{k}},|\endgroup\par
\begingroup\color{diffctx}\Verb| \qquad k:=2n-2p,|\endgroup\par
\begingroup\color{diffctx}\Verb| \]|\endgroup\par
\begingroup\color{diffdel}\Verb|-and hence the global estimate|\endgroup\par
\begingroup\color{diffadd}\Verb|+and hence, in the common situation where the slice currents on interfaces are cycles (so $\partial\Sigma_F(u)=0$), the global estimate|\endgroup\par
\begingroup\color{diffctx}\Verb| \[|\endgroup\par
\begingroup\color{diffctx}\Verb| \mathcal F\!\left(\partial T^{\mathrm{raw}}\right)|\endgroup\par
\begingroup\color{diffctx}\Verb| \ \lesssim\ h^2\sum_Q\ \sum_{a\in\mathcal S(Q)} m_{Q,a}^{\frac{k-1}{k}}.|\endgroup\par
\begingroup\color{diffhunk}\Verb|@@ -3569,15 +3661,15 @@|\endgroup\par
\begingroup\color{diffctx}\Verb| \sum_F \mathcal F(B_F).|\endgroup\par
\begingroup\color{diffctx}\Verb| \]|\endgroup\par
\begingroup\color{diffctx}\Verb| |\endgroup\par
\begingroup\color{diffdel}\Verb|-For a fixed interface $F$, the translation model hypothesis and a matching $\sigma$ with|\endgroup\par
\begingroup\color{diffdel}\Verb!-$\|u_a-u'_{\sigma(a)}\|\le \Delta_F$ give the per-face estimate!\endgroup\par
\begingroup\color{diffdel}\Verb|-\[|\endgroup\par
\begingroup\color{diffdel}\Verb|-\mathcal F(B_F)\ \le\ \Delta_F\sum_{a=1}^N \Mass(\Sigma_F(u_a)),|\endgroup\par
\begingroup\color{diffdel}\Verb|-\]|\endgroup\par
\begingroup\color{diffdel}\Verb|-so summing over $F$ yields the first bound.|\endgroup\par
\begingroup\color{diffadd}\Verb!+For a fixed interface $F$, choose a coupling $\pi_F\in\Gamma(\mu_{Q\to F},\mu_{Q'\to F})$ supported on pairs with $\|u-u'\|\le \Delta_F$.!\endgroup\par
\begingroup\color{diffadd}\Verb|+Proposition~\ref{prop:transport-flat-glue-weighted} then gives the per-face estimate|\endgroup\par
\begingroup\color{diffadd}\Verb|+\[|\endgroup\par
\begingroup\color{diffadd}\Verb|+\mathcal F(B_F)\ \le\ \Delta_F\int_{\Omega_F}\Bigl(\Mass(\Sigma_F(u))+\Mass(\partial\Sigma_F(u))\Bigr)\,d\mu_{Q\to F}(u),|\endgroup\par
\begingroup\color{diffadd}\Verb|+\]|\endgroup\par
\begingroup\color{diffadd}\Verb|+and summing over $F$ yields the first bound.|\endgroup\par
\begingroup\color{diffctx}\Verb| |\endgroup\par
\begingroup\color{diffctx}\Verb| Under the additional assumptions $\Delta_F\le C\,h^2$ and|\endgroup\par
\begingroup\color{diffdel}\Verb|-$\Mass(\Sigma_F(u_a))\lesssim m_a^{\frac{k-1}{k}}$ (with $k=2n-2p$),|\endgroup\par
\begingroup\color{diffadd}\Verb|+$\Mass(\Sigma_F(u))+\Mass(\partial\Sigma_F(u))\lesssim m_a^{\frac{k-1}{k}}$ (with $k=2n-2p$),|\endgroup\par
\begingroup\color{diffctx}\Verb| we obtain|\endgroup\par
\begingroup\color{diffctx}\Verb| \[|\endgroup\par
\begingroup\color{diffctx}\Verb| \mathcal F(B_F)\ \lesssim\ h^2\sum_{a\in\mathcal S(F)} m_{F,a}^{\frac{k-1}{k}}.|\endgroup\par
\begingroup\color{diffhunk}\Verb|@@ -3624,6 +3716,38 @@|\endgroup\par
\begingroup\color{diffctx}\Verb| which tends to $0$ for fixed $\varepsilon>0$ whenever $k>n-1$ (equivalently $p<\frac{n+1}{2}$).|\endgroup\par
\begingroup\color{diffctx}\Verb| By Remark~\ref{rem:lefschetz-reduction}, it suffices for the unconditional Hodge program to treat $p\le n/2$, which lies in this range.|\endgroup\par
\begingroup\color{diffctx}\Verb| \end{remark}|\endgroup\par
\begingroup\color{diffadd}\Verb|+|\endgroup\par
\begingroup\color{diffadd}\Verb|+\begin{lemma}[Parameter/scaling regime implying $\mathcal F(\partial T^{\mathrm{raw}})=o(m)$]\label{lem:flatnorm-o-m}|\endgroup\par
\begingroup\color{diffadd}\Verb|+Fix a homology multiple $m$ and a mesh scale $h\to 0$, and set $k:=2n-2p$.|\endgroup\par
\begingroup\color{diffadd}\Verb|+Assume the raw current $T^{\mathrm{raw}}$ is built from holomorphic sliver pieces on the mesh as in Substep~4.2, with:|\endgroup\par
\begingroup\color{diffadd}\Verb|+\begin{enumerate}|\endgroup\par
\begingroup\color{diffadd}\Verb|+\item[\textnormal{(a)}] \textbf{Small-slope graphs:} on each cell, each piece is a $C^1$ graph of slope $\le \varepsilon$ over its flat model;|\endgroup\par
\begingroup\color{diffadd}\Verb|+\item[\textnormal{(b)}] \textbf{Template displacement:} across each interface face one has $\Delta_F\lesssim h^2$ (e.g.\ Lemma~\ref{lem:face-displacement});|\endgroup\par
\begingroup\color{diffadd}\Verb|+\item[\textnormal{(c)}] \textbf{Packing:} each cell has at most $N_Q\lesssim \varepsilon^{-2p}$ disjoint pieces per direction family (Lemma~\ref{lem:sliver-packing});|\endgroup\par
\begingroup\color{diffadd}\Verb|+\item[\textnormal{(d)}] \textbf{Mass scale:} the total mass per cell satisfies $M_Q:=\sum_{a\in\mathcal S(Q)}m_{Q,a}\asymp m\,h^{2n}$ (coming from the smooth form $m\beta$).|\endgroup\par
\begingroup\color{diffadd}\Verb|+\end{enumerate}|\endgroup\par
\begingroup\color{diffadd}\Verb|+Then the weighted face estimate (Corollary~\ref{cor:global-flat-weighted}) and the Hölder/packing bound of Remark~\ref{rem:weighted-scaling} give|\endgroup\par
\begingroup\color{diffadd}\Verb|+\[|\endgroup\par
\begingroup\color{diffadd}\Verb|+\mathcal F(\partial T^{\mathrm{raw}})\ \lesssim\ m^{\frac{k-1}{k}}\,h^{\,2-\frac{2n}{k}}\;\varepsilon^{-\frac{2p}{k}}\ +\ O(\varepsilon\,m).|\endgroup\par
\begingroup\color{diffadd}\Verb|+\]|\endgroup\par
\begingroup\color{diffadd}\Verb|+In particular, in the range $p<\frac{n}{2}$ (equivalently $k>n$ so that $2-\frac{2n}{k}>0$), choosing $\varepsilon=\varepsilon(h)\downarrow 0$ such that|\endgroup\par
\begingroup\color{diffadd}\Verb|+\[|\endgroup\par
\begingroup\color{diffadd}\Verb|+h^{\,2-\frac{2n}{k}}\;\varepsilon(h)^{-\frac{2p}{k}}\ \longrightarrow\ 0|\endgroup\par
\begingroup\color{diffadd}\Verb|+\qquad\text{as }h\downarrow 0|\endgroup\par
\begingroup\color{diffadd}\Verb|+\]|\endgroup\par
\begingroup\color{diffadd}\Verb|+(e.g.\ $\varepsilon(h)=h^{\alpha}$ for any $0<\alpha<\frac{k-n}{p}=\frac{n-2p}{p}$) yields $\mathcal F(\partial T^{\mathrm{raw}})\to 0$.|\endgroup\par
\begingroup\color{diffadd}\Verb|+Consequently, $\mathcal F(\partial T^{\mathrm{raw}})=o(m)$ along this mesh refinement.|\endgroup\par
\begingroup\color{diffadd}\Verb|+|\endgroup\par
\begingroup\color{diffadd}\Verb|+\smallskip\noindent|\endgroup\par
\begingroup\color{diffadd}\Verb|+In the borderline case $p=\frac{n}{2}$, the decay exponent $2-\frac{2n}{k}$ vanishes, so one uses the discrete face-transport route:|\endgroup\par
\begingroup\color{diffadd}\Verb|+Proposition~\ref{prop:integer-transport} with $\delta=o(h)$ and $\varepsilon=o(1)$ implies $\mathcal F(\partial T^{\mathrm{raw}})\to 0$|\endgroup\par
\begingroup\color{diffadd}\Verb|+(recorded explicitly in Lemma~\ref{lem:borderline-p-half}).|\endgroup\par
\begingroup\color{diffadd}\Verb|+|\endgroup\par
\begingroup\color{diffadd}\Verb|+\smallskip\noindent|\endgroup\par
\begingroup\color{diffadd}\Verb|+\textbf{Consistency with the global schedule:} the scale relations used above are compatible with \S\ref{sec:parameter-schedule}:|\endgroup\par
\begingroup\color{diffadd}\Verb|+choose $m$ first, then $h_j\downarrow 0$, then choose the holomorphic power $M_j\to\infty$ so that $h_j\le c\,M_j^{-1/2}$ (Bergman control),|\endgroup\par
\begingroup\color{diffadd}\Verb|+and choose $\delta_j=o(h_j)$ and $\varepsilon_j\to 0$ as required by the matching/graph hypotheses.|\endgroup\par
\begingroup\color{diffadd}\Verb|+\end{lemma}|\endgroup\par
\begingroup\color{diffctx}\Verb| |\endgroup\par
\begingroup\color{diffctx}\Verb| \begin{remark}[On vanishing per-piece masses (no hidden lower bound)]\label{rem:no-vanishing-piece-mass}|\endgroup\par
\begingroup\color{diffctx}\Verb| The weighted flat-norm estimate of Corollary~\ref{cor:global-flat-weighted}|\endgroup\par
\begingroup\color{diffhunk}\Verb|@@ -3730,7 +3854,8 @@|\endgroup\par
\begingroup\color{diffctx}\Verb! Assume $\|\Phi_{Q,F}\|_{\mathrm{op}}+\|\Phi_{Q',F}\|_{\mathrm{op}}\le C_{\Phi,0}$ and $\|\Phi_{Q,F}-\Phi_{Q',F}\|_{\mathrm{op}}\le C_\Phi h$.!\endgroup\par
\begingroup\color{diffctx}\Verb| Then, after pairing atoms by the identity pairing $y_a\leftrightarrow y_a$, the mismatch current $B_F$ satisfies|\endgroup\par
\begingroup\color{diffctx}\Verb| \[|\endgroup\par
\begingroup\color{diffdel}\Verb|-\mathcal F(B_F)\ \le\ C\,h^2\,\Bigl(\Mass(\partial S_Q\llcorner F)+\Mass(\partial S_{Q'}\llcorner F)\Bigr)\ +\ O(\varepsilon\,M_F),|\endgroup\par
\begingroup\color{diffadd}\Verb|+\mathcal F(B_F)\ \le\ C\,h^2\,\Biggl(\sum_{a=1}^{N_F} w_a\Bigl(\Mass(\Sigma_{\Phi_{Q,F}y_a})+\Mass(\partial\Sigma_{\Phi_{Q,F}y_a})\Bigr)|\endgroup\par
\begingroup\color{diffadd}\Verb|+\;+\sum_{a=1}^{N_F} w_a\Bigl(\Mass(\Sigma_{\Phi_{Q',F}y_a})+\Mass(\partial\Sigma_{\Phi_{Q',F}y_a})\Bigr)\Biggr)\ +\ C\,\varepsilon\,M_F,|\endgroup\par
\begingroup\color{diffctx}\Verb| \]|\endgroup\par
\begingroup\color{diffctx}\Verb| where $M_F$ denotes the total $(2n-2p)$-mass of pieces meeting the interface (so $M_F\lesssim M_Q+M_{Q'}$) and|\endgroup\par
\begingroup\color{diffctx}\Verb| $\varepsilon$ is the small-angle/graph parameter from Proposition~\ref{prop:transport-flat-glue}\textnormal{(a)}.|\endgroup\par
\begingroup\color{diffhunk}\Verb|@@ -3755,18 +3880,17 @@|\endgroup\par
\begingroup\color{diffctx}\Verb| Lemma~\ref{lem:flat-translate} then gives|\endgroup\par
\begingroup\color{diffctx}\Verb| \[|\endgroup\par
\begingroup\color{diffctx}\Verb| \mathcal F\!\bigl(\Sigma_{\Phi_{Q,F}y_a}-\Sigma_{\Phi_{Q',F}y_a}\bigr)|\endgroup\par
\begingroup\color{diffdel}\Verb!-\le \|v_a\|\,\Mass(\Sigma_{\Phi_{Q,F}y_a})!\endgroup\par
\begingroup\color{diffdel}\Verb|-\le C h^2\,\Mass(\Sigma_{\Phi_{Q,F}y_a}).|\endgroup\par
\begingroup\color{diffadd}\Verb!+\le \|v_a\|\Bigl(\Mass(\Sigma_{\Phi_{Q,F}y_a})+\Mass(\partial\Sigma_{\Phi_{Q,F}y_a})\Bigr)!\endgroup\par
\begingroup\color{diffadd}\Verb|+\le C h^2\Bigl(\Mass(\Sigma_{\Phi_{Q,F}y_a})+\Mass(\partial\Sigma_{\Phi_{Q,F}y_a})\Bigr).|\endgroup\par
\begingroup\color{diffctx}\Verb| \]|\endgroup\par
\begingroup\color{diffctx}\Verb| By subadditivity of $\mathcal F$ and summing over $a$ (with weights $w_a$),|\endgroup\par
\begingroup\color{diffctx}\Verb| \[|\endgroup\par
\begingroup\color{diffdel}\Verb|-\mathcal F(B_F)\le C h^2\sum_{a=1}^{N_F} w_a\,\Mass(\Sigma_{\Phi_{Q,F}y_a})|\endgroup\par
\begingroup\color{diffdel}\Verb|-\le C h^2\,\Mass(\partial S_Q\llcorner F).|\endgroup\par
\begingroup\color{diffadd}\Verb|+\mathcal F(B_F)\le C h^2\sum_{a=1}^{N_F} w_a\,\Bigl(\Mass(\Sigma_{\Phi_{Q,F}y_a})+\Mass(\partial\Sigma_{\Phi_{Q,F}y_a})\Bigr).|\endgroup\par
\begingroup\color{diffctx}\Verb| \]|\endgroup\par
\begingroup\color{diffctx}\Verb| The same bound holds with $Q$ and $Q'$ swapped; combining yields the symmetric form stated.|\endgroup\par
\begingroup\color{diffctx}\Verb| |\endgroup\par
\begingroup\color{diffctx}\Verb| For $\varepsilon>0$, compare each sheet to the corresponding flat slice in the tubular chart; the $C^1$ graph distortion contributes an|\endgroup\par
\begingroup\color{diffdel}\Verb|-additional $O(\varepsilon\,M_F)$ term exactly as in Proposition~\ref{prop:transport-flat-glue}.|\endgroup\par
\begingroup\color{diffadd}\Verb|+additional $C\,\varepsilon\,M_F$ term exactly as in Proposition~\ref{prop:transport-flat-glue} (after enlarging $C$).|\endgroup\par
\begingroup\color{diffctx}\Verb| \end{proof}|\endgroup\par
\begingroup\color{diffctx}\Verb| |\endgroup\par
\begingroup\color{diffctx}\Verb| |\endgroup\par
\begingroup\color{diffhunk}\Verb|@@ -3799,7 +3923,8 @@|\endgroup\par
\begingroup\color{diffctx}\Verb| The matched part $B_F^{\wedge}$ is obtained by applying the two face maps to the \emph{same} common submeasure $\nu^{\wedge}$.|\endgroup\par
\begingroup\color{diffctx}\Verb| Therefore Lemma~\ref{lem:template-displacement} applies directly and yields the stated bound for $B_F^{\wedge}$.|\endgroup\par
\begingroup\color{diffctx}\Verb| |\endgroup\par
\begingroup\color{diffdel}\Verb|-For the unmatched part, $B_F^{\mathrm{un}}$ is an integral $(k-1)$--cycle supported on the face patch $F$.|\endgroup\par
\begingroup\color{diffadd}\Verb|+For the unmatched part, $B_F^{\mathrm{un}}$ is an integral $(k-1)$--cycle supported on the (relative) interior of the face patch $F$|\endgroup\par
\begingroup\color{diffadd}\Verb|+(any possible edge contributions are treated separately in the global bookkeeping/corner-exit package).|\endgroup\par
\begingroup\color{diffctx}\Verb| Since $\mathrm{diam}(F)\lesssim h$, Lemma~\ref{lem:flat-diameter} gives|\endgroup\par
\begingroup\color{diffctx}\Verb| \[|\endgroup\par
\begingroup\color{diffctx}\Verb| \mathcal F(B_F^{\mathrm{un}})\ \le\ C\,h\,\Mass(B_F^{\mathrm{un}}).|\endgroup\par
\begingroup\color{diffhunk}\Verb|@@ -4824,7 +4949,9 @@|\endgroup\par
\begingroup\color{diffctx}\Verb! $|N_{Q,v,i}-N_{Q',v,i}|\lesssim h\min\{N_{Q,v,i},N_{Q',v,i}\}$ on the region where $M_{Q,i}$ is not negligible (e.g.\ via Lemma~\ref{lem:slow-variation-rounding}!\endgroup\par
\begingroup\color{diffctx}\Verb| and the $0$--$1$ stability Lemma~\ref{lem:slow-variation-discrepancy});|\endgroup\par
\begingroup\color{diffctx}\Verb| \item[\textnormal{(c)}] (\textbf{Cohomology periods}) after clearing denominators by choosing $m$ and applying fixed-dimension discrepancy rounding|\endgroup\par
\begingroup\color{diffdel}\Verb|-(Lemma~\ref{lem:barany-grinberg} in the form of Proposition~\ref{prop:cohomology-match}), the resulting raw current satisfies the integral period constraints.|\endgroup\par
\begingroup\color{diffadd}\Verb|+(Lemma~\ref{lem:barany-grinberg}), one can choose the integer activations so that the \emph{raw} current has the desired periods up to an error $<\tfrac14$|\endgroup\par
\begingroup\color{diffadd}\Verb|+on a fixed integral cohomology basis; after applying the gluing correction with sufficiently small mass, the resulting \emph{closed} glued cycle has the|\endgroup\par
\begingroup\color{diffadd}\Verb|+exact integral periods and hence the exact class $\mathrm{PD}(m[\gamma])$ in rational homology (Proposition~\ref{prop:cohomology-match}).|\endgroup\par
\begingroup\color{diffctx}\Verb| \end{enumerate}|\endgroup\par
\begingroup\color{diffctx}\Verb| Consequently, for each label $i$ the activation hypotheses (iii)–(iv) in Theorem~\ref{thm:sliver-mass-matching-on-template} hold (by Corollary~\ref{cor:corner-exit-iii-iv}),|\endgroup\par
\begingroup\color{diffctx}\Verb| and summing the resulting per-label flat-norm mismatch bounds yields $\mathcal F(\partial T^{\mathrm{raw}})=o(m)$ under the parameter regime of|\endgroup\par
\begingroup\color{diffhunk}\Verb|@@ -4863,6 +4990,18 @@|\endgroup\par
\begingroup\color{diffctx}\Verb| $\mathcal F(\partial T^{\mathrm{raw}})=o(m)$ in the scaling regime of Remark~\ref{rem:weighted-scaling}.|\endgroup\par
\begingroup\color{diffctx}\Verb| \end{proof}|\endgroup\par
\begingroup\color{diffctx}\Verb| |\endgroup\par
\begingroup\color{diffadd}\Verb|+\begin{remark}[External inputs for integer rounding]\label{rem:integer-rounding-external}|\endgroup\par
\begingroup\color{diffadd}\Verb|+\textbf{This proposition relies on external inputs from discrete optimization.}  Steps 2 and 4 use integer rounding lemmas whose proofs invoke:|\endgroup\par
\begingroup\color{diffadd}\Verb|+\begin{itemize}|\endgroup\par
\begingroup\color{diffadd}\Verb|+\item the Barvinok--Bar\'any--Grinberg discrepancy bounds for integer approximation in fixed-dimensional polytopes (Lemma~\ref{lem:barany-grinberg});|\endgroup\par
\begingroup\color{diffadd}\Verb|+\item the observation that the constraint dimension $b=\mathrm{rank}\,H^{2n-2p}(X,\Z)$ is fixed (independent of mesh refinement), so that|\endgroup\par
\begingroup\color{diffadd}\Verb|+  discrepancy bounds do not blow up.|\endgroup\par
\begingroup\color{diffadd}\Verb|+\end{itemize}|\endgroup\par
\begingroup\color{diffadd}\Verb|+Reference: Barvinok, \emph{Integer Programming} \cite{Barvinok-IntProg}.|\endgroup\par
\begingroup\color{diffadd}\Verb|+|\endgroup\par
\begingroup\color{diffadd}\Verb|+\smallskip\noindent|\endgroup\par
\begingroup\color{diffadd}\Verb|+\textbf{Adversarial concern:} The claim that global period-fixing does not break the local slow-variation bounds depends on the bounded-correction absorption mechanism (Remark~\ref{rem:bounded-corrections}).  Any audit should verify that the correction vectors have uniformly bounded entries and that this bound is independent of mesh refinement.|\endgroup\par
\begingroup\color{diffadd}\Verb|+\end{remark}|\endgroup\par
\begingroup\color{diffctx}\Verb| |\endgroup\par
\begingroup\color{diffctx}\Verb| \begin{remark}[Making the ``prefix-balanced face population'' explicit]|\endgroup\par
\begingroup\color{diffctx}\Verb| The previous proposition treats each vertex template separately.|\endgroup\par
\begingroup\color{diffhunk}\Verb|@@ -5557,11 +5696,13 @@|\endgroup\par
\begingroup\color{diffctx}\Verb| (Theorem~\ref{thm:sliver-mass-matching-on-template} and Corollary~\ref{cor:global-flat-weighted}), one obtains the quantitative estimate|\endgroup\par
\begingroup\color{diffctx}\Verb| \[|\endgroup\par
\begingroup\color{diffctx}\Verb| \mathcal F\!\left(\partial T^{\mathrm{raw}}\right)\ \le\ \varepsilon_{\mathrm{glue}}(m,\delta,\varepsilon,\mathrm{mesh})\cdot m,|\endgroup\par
\begingroup\color{diffdel}\Verb|-\qquad \varepsilon_{\mathrm{glue}}\xrightarrow[\delta,\varepsilon\to 0,\ \mathrm{mesh}\to 0,\ m\to\infty]{}0.|\endgroup\par
\begingroup\color{diffdel}\Verb|-\]|\endgroup\par
\begingroup\color{diffadd}\Verb|+\]|\endgroup\par
\begingroup\color{diffadd}\Verb|+\noindent where $\varepsilon_{\mathrm{glue}}\to 0$ under the global parameter schedule of \S\ref{sec:parameter-schedule}.|\endgroup\par
\begingroup\color{diffadd}\Verb|+A concrete sufficient regime (with explicit scale relations between $\varepsilon$ and $\mathrm{mesh}$ in the range $p<\frac{n}{2}$, and the|\endgroup\par
\begingroup\color{diffadd}\Verb|+borderline replacement at $p=\frac{n}{2}$) is recorded in Lemma~\ref{lem:flatnorm-o-m}.|\endgroup\par
\begingroup\color{diffctx}\Verb| By definition of $\mathcal F$ there exist integral currents|\endgroup\par
\begingroup\color{diffctx}\Verb| $R$ and $Q$ with $\partial T^{\mathrm{raw}}=R+\partial Q$ and $\Mass(R)+\Mass(Q)\le 2\mathcal F(\partial T^{\mathrm{raw}})$.|\endgroup\par
\begingroup\color{diffdel}\Verb|-Moreover $R$ is a boundary (since $\partial T^{\mathrm{raw}}$ is), hence null-homologous; by the Federer--Fleming|\endgroup\par
\begingroup\color{diffadd}\Verb|+Moreover $R=\partial(T^{\mathrm{raw}}-Q)$ is itself a boundary (hence null-homologous); by the Federer--Fleming|\endgroup\par
\begingroup\color{diffctx}\Verb| isoperimetric inequality there exists an integral filling $Q_R$ with $\partial Q_R=R$ and|\endgroup\par
\begingroup\color{diffctx}\Verb| \[|\endgroup\par
\begingroup\color{diffctx}\Verb| \Mass(Q_R)\le C\,\Mass(R)^{\frac{2n-2p}{2n-2p-1}}.|\endgroup\par
\begingroup\color{diffhunk}\Verb|@@ -5572,73 +5713,79 @@|\endgroup\par
\begingroup\color{diffctx}\Verb| \]|\endgroup\par
\begingroup\color{diffctx}\Verb| gives $\partial R_{\mathrm{glue}}=-\partial T^{\mathrm{raw}}$ and $\Mass(R_{\mathrm{glue}})$ as small as desired once|\endgroup\par
\begingroup\color{diffctx}\Verb| $\mathcal F(\partial T^{\mathrm{raw}})$ is small.|\endgroup\par
\begingroup\color{diffadd}\Verb|+|\endgroup\par
\begingroup\color{diffadd}\Verb|+\begin{lemma}[Federer--Fleming filling on $X$ for small cycles]\label{lem:FF-filling-X}|\endgroup\par
\begingroup\color{diffadd}\Verb|+Let $X$ be the fixed compact Riemannian manifold in the projective setting of the paper, and fix $k\ge 2$.|\endgroup\par
\begingroup\color{diffadd}\Verb|+There exist constants $\delta_X>0$ and $C_X>0$ (depending on $(X,g)$ and $k$) such that the following holds.|\endgroup\par
\begingroup\color{diffadd}\Verb|+|\endgroup\par
\begingroup\color{diffadd}\Verb|+If $R$ is an integral $(k-1)$--current in $X$ with $\partial R=0$ and $R=\partial S$ for some integral $k$--current $S$ in $X$|\endgroup\par
\begingroup\color{diffadd}\Verb|+(i.e.\ $R$ bounds), and if $\Mass(R)\le \delta_X$, then there exists an integral $k$--current $Q_R$ in $X$ with|\endgroup\par
\begingroup\color{diffadd}\Verb|+\[|\endgroup\par
\begingroup\color{diffadd}\Verb|+\partial Q_R\ =\ R,|\endgroup\par
\begingroup\color{diffadd}\Verb|+\qquad|\endgroup\par
\begingroup\color{diffadd}\Verb|+\Mass(Q_R)\ \le\ C_X\,\Mass(R)^{\frac{k}{k-1}}.|\endgroup\par
\begingroup\color{diffadd}\Verb|+\]|\endgroup\par
\begingroup\color{diffadd}\Verb|+In particular, $\Mass(Q_R)\to 0$ as $\Mass(R)\to 0$.|\endgroup\par
\begingroup\color{diffadd}\Verb|+\end{lemma}|\endgroup\par
\begingroup\color{diffadd}\Verb|+|\endgroup\par
\begingroup\color{diffadd}\Verb|+\begin{proof}|\endgroup\par
\begingroup\color{diffadd}\Verb|+Choose a finite atlas of $X$ by coordinate charts with uniformly controlled bi-Lipschitz constants at the scale of injectivity radius.|\endgroup\par
\begingroup\color{diffadd}\Verb|+For $\Mass(R)$ sufficiently small, the support of $R$ is contained in a single chart (after decomposing $R$ into finitely many pieces if needed),|\endgroup\par
\begingroup\color{diffadd}\Verb|+so the Euclidean Federer--Fleming isoperimetric inequality in $\R^N$ applies to the chart image.|\endgroup\par
\begingroup\color{diffadd}\Verb|+Pushing the resulting filling forward to $X$ and absorbing the chart distortion constants yields the stated bound with $C_X$ and $\delta_X$|\endgroup\par
\begingroup\color{diffadd}\Verb|+depending only on $(X,g)$ and $k$.|\endgroup\par
\begingroup\color{diffadd}\Verb|+A detailed proof in the Riemannian setting can be found in standard GMT references (e.g.\ \cite{FF60,Fed69,Sim83}).|\endgroup\par
\begingroup\color{diffadd}\Verb|+\end{proof}|\endgroup\par
\begingroup\color{diffadd}\Verb|+|\endgroup\par
\begingroup\color{diffctx}\Verb| \begin{proposition}[Microstructure/gluing estimate]\label{prop:glue-gap}|\endgroup\par
\begingroup\color{diffctx}\Verb| Let $T^{\mathrm{raw}}=\sum_Q S_Q$ be the raw integral current built from the microstructure pieces on a mesh of size $h$|\endgroup\par
\begingroup\color{diffctx}\Verb| as in Substep~4.2.|\endgroup\par
\begingroup\color{diffdel}\Verb|-Assume that for every interior interface $F=Q\cap Q'$ (i.e.\ a codimension-$1$ face shared by two distinct cells of the mesh)|\endgroup\par
\begingroup\color{diffdel}\Verb|-the face mismatch current|\endgroup\par
\begingroup\color{diffdel}\Verb|-\[|\endgroup\par
\begingroup\color{diffdel}\Verb|-B_F\ :=\ \bigl(\partial S_Q\bigr)\llcorner F\ -\ \bigl(\partial S_{Q'}\bigr)\llcorner F|\endgroup\par
\begingroup\color{diffdel}\Verb|-\]|\endgroup\par
\begingroup\color{diffdel}\Verb|-admits the translation model of Proposition~\ref{prop:transport-flat-glue-weighted} with parameter multisets|\endgroup\par
\begingroup\color{diffdel}\Verb|-$\{u_{F,a}\}_{a=1}^{N_F}$ and $\{u'_{F,a}\}_{a=1}^{N_F}$, and that there exists a matching|\endgroup\par
\begingroup\color{diffdel}\Verb|-$\sigma_F\in S_{N_F}$ satisfying the uniform displacement bound|\endgroup\par
\begingroup\color{diffdel}\Verb|-\[|\endgroup\par
\begingroup\color{diffdel}\Verb!-\|u_{F,a}-u'_{F,\sigma_F(a)}\|\ \le\ \Delta_F\qquad\text{for all }a.!\endgroup\par
\begingroup\color{diffdel}\Verb|-\]|\endgroup\par
\begingroup\color{diffdel}\Verb|-Let $Q_F$ be the integral filling current produced in the proof of Proposition~\ref{prop:transport-flat-glue-weighted}|\endgroup\par
\begingroup\color{diffdel}\Verb|-for the matching $\sigma_F$, so that $\partial Q_F=B_F$ and|\endgroup\par
\begingroup\color{diffdel}\Verb|-\[|\endgroup\par
\begingroup\color{diffdel}\Verb!-\Mass(Q_F)\ \le\ \sum_{a=1}^{N_F}\|u_{F,a}-u'_{F,\sigma_F(a)}\|\,\Mass(\Sigma_F(u_{F,a}))!\endgroup\par
\begingroup\color{diffdel}\Verb|-\ \le\ \Delta_F\sum_{a=1}^{N_F}\Mass(\Sigma_F(u_{F,a})).|\endgroup\par
\begingroup\color{diffdel}\Verb|-\]|\endgroup\par
\begingroup\color{diffdel}\Verb|-Define the global correction current and glued cycle|\endgroup\par
\begingroup\color{diffdel}\Verb|-\[|\endgroup\par
\begingroup\color{diffdel}\Verb|-U\ :=\ \sum_F Q_F,|\endgroup\par
\begingroup\color{diffadd}\Verb|+Assume we are in a parameter regime where $\mathcal F(\partial T^{\mathrm{raw}})\to 0$ along the mesh refinement|\endgroup\par
\begingroup\color{diffadd}\Verb|+(for example, the regime of Lemma~\ref{lem:flatnorm-o-m} for $p<\frac{n}{2}$, and in the borderline case $p=\frac{n}{2}$ via Lemma~\ref{lem:borderline-p-half}).|\endgroup\par
\begingroup\color{diffadd}\Verb|+Then there exists an integral current $R_{\mathrm{glue}}$ with|\endgroup\par
\begingroup\color{diffadd}\Verb|+\[|\endgroup\par
\begingroup\color{diffadd}\Verb|+\partial R_{\mathrm{glue}}=-\partial T^{\mathrm{raw}}|\endgroup\par
\begingroup\color{diffadd}\Verb|+\qquad\text{and}\qquad|\endgroup\par
\begingroup\color{diffadd}\Verb|+\Mass(R_{\mathrm{glue}})\ \xrightarrow[\mathcal F(\partial T^{\mathrm{raw}})\to 0]{}\ 0.|\endgroup\par
\begingroup\color{diffadd}\Verb|+\]|\endgroup\par
\begingroup\color{diffadd}\Verb|+In particular, $T^{\mathrm{raw}}+R_{\mathrm{glue}}$ is a closed integral cycle.|\endgroup\par
\begingroup\color{diffadd}\Verb|+\end{proposition}|\endgroup\par
\begingroup\color{diffadd}\Verb|+\begin{proof}|\endgroup\par
\begingroup\color{diffadd}\Verb|+Let $\delta:=\mathcal F(\partial T^{\mathrm{raw}})$.|\endgroup\par
\begingroup\color{diffadd}\Verb|+Choose $R,Q$ in the definition of $\mathcal F$ with|\endgroup\par
\begingroup\color{diffadd}\Verb|+\[|\endgroup\par
\begingroup\color{diffadd}\Verb|+\partial T^{\mathrm{raw}}=R+\partial Q,|\endgroup\par
\begingroup\color{diffctx}\Verb| \qquad|\endgroup\par
\begingroup\color{diffdel}\Verb|-T\ :=\ T^{\mathrm{raw}}-U.|\endgroup\par
\begingroup\color{diffdel}\Verb|-\]|\endgroup\par
\begingroup\color{diffdel}\Verb|-Here and below, $\sum_F$ ranges over all interior interfaces $F=Q\cap Q'$ (each counted once).|\endgroup\par
\begingroup\color{diffdel}\Verb|-Then $U$ is integral, $\partial U=\partial T^{\mathrm{raw}}$, and hence $T$ is an integral cycle with|\endgroup\par
\begingroup\color{diffdel}\Verb|-$[T]=[T^{\mathrm{raw}}]=\mathrm{PD}(m[\gamma])$.|\endgroup\par
\begingroup\color{diffdel}\Verb|-Moreover,|\endgroup\par
\begingroup\color{diffdel}\Verb|-\[|\endgroup\par
\begingroup\color{diffdel}\Verb|-\Mass(U)\ \le\ \sum_F \Mass(Q_F)\ \le\ \sum_F \Delta_F\sum_{a=1}^{N_F}\Mass(\Sigma_F(u_{F,a})).|\endgroup\par
\begingroup\color{diffdel}\Verb|-\]|\endgroup\par
\begingroup\color{diffdel}\Verb|-In particular, in the parameter regime of Corollary~\ref{cor:global-flat-weighted} and Remark~\ref{rem:weighted-scaling}|\endgroup\par
\begingroup\color{diffdel}\Verb|-(where $\Delta_F\lesssim h^2$ and the right-hand side is $o(m)$ as $h\downarrow 0$), we obtain a family of integral fillings|\endgroup\par
\begingroup\color{diffdel}\Verb|-$U_h$ (i.e.\ the above $U$ at mesh size $h$) with $\partial U_h=\partial T^{\mathrm{raw}}$ and $\Mass(U_h)=o(m)$; consequently|\endgroup\par
\begingroup\color{diffdel}\Verb|-\[|\endgroup\par
\begingroup\color{diffdel}\Verb|-\mathcal F\!\left(\partial T^{\mathrm{raw}}\right)\ \le\ \Mass(U_h)\ =\ o(m).|\endgroup\par
\begingroup\color{diffdel}\Verb|-\]|\endgroup\par
\begingroup\color{diffdel}\Verb|-\end{proposition}|\endgroup\par
\begingroup\color{diffdel}\Verb|-\begin{proof}|\endgroup\par
\begingroup\color{diffdel}\Verb|-Fix an interior interface $F=Q\cap Q'$ and a matching $\sigma_F$ as in the hypothesis.|\endgroup\par
\begingroup\color{diffdel}\Verb|-In the translation model, each slice $\Sigma_F(u_{F,a})$ is a translate of $\Sigma_F(0)$ in face coordinates, so|\endgroup\par
\begingroup\color{diffdel}\Verb|-Proposition~\ref{prop:transport-flat-glue-weighted} (see its proof via Lemma~\ref{lem:flat-translate})|\endgroup\par
\begingroup\color{diffdel}\Verb|-produces an integral filling current $Q_F$ with $\partial Q_F=B_F$ and|\endgroup\par
\begingroup\color{diffdel}\Verb|-\[|\endgroup\par
\begingroup\color{diffdel}\Verb!-\Mass(Q_F)\ \le\ \sum_{a=1}^{N_F}\|u_{F,a}-u'_{F,\sigma_F(a)}\|\,\Mass(\Sigma_F(u_{F,a}))!\endgroup\par
\begingroup\color{diffdel}\Verb|-\ \le\ \Delta_F\sum_{a=1}^{N_F}\Mass(\Sigma_F(u_{F,a})).|\endgroup\par
\begingroup\color{diffdel}\Verb|-\]|\endgroup\par
\begingroup\color{diffdel}\Verb|-Summing over all interior interfaces and setting $U:=\sum_F Q_F$ gives an integral current with|\endgroup\par
\begingroup\color{diffdel}\Verb|-\[|\endgroup\par
\begingroup\color{diffdel}\Verb|-\partial U=\sum_F \partial Q_F=\sum_F B_F.|\endgroup\par
\begingroup\color{diffdel}\Verb|-\]|\endgroup\par
\begingroup\color{diffdel}\Verb|-By the oriented face decomposition of $\partial T^{\mathrm{raw}}$ one has $\partial T^{\mathrm{raw}}=\sum_F B_F$, hence|\endgroup\par
\begingroup\color{diffdel}\Verb|-$\partial U=\partial T^{\mathrm{raw}}$ and $T:=T^{\mathrm{raw}}-U$ is an integral cycle with $[T]=[T^{\mathrm{raw}}]$.|\endgroup\par
\begingroup\color{diffdel}\Verb|-The mass bound follows from the triangle inequality:|\endgroup\par
\begingroup\color{diffdel}\Verb|-\[|\endgroup\par
\begingroup\color{diffdel}\Verb|-\Mass(U)\ \le\ \sum_F \Mass(Q_F)\ \le\ \sum_F \Delta_F\sum_{a=1}^{N_F}\Mass(\Sigma_F(u_{F,a})).|\endgroup\par
\begingroup\color{diffdel}\Verb|-\]|\endgroup\par
\begingroup\color{diffdel}\Verb|-Finally, taking $R:=0$ and $Q:=U$ in the definition of the flat norm gives|\endgroup\par
\begingroup\color{diffdel}\Verb|-$\mathcal F(\partial T^{\mathrm{raw}})\le \Mass(U)$, and the stated $o(m)$ regime follows from|\endgroup\par
\begingroup\color{diffdel}\Verb|-Corollary~\ref{cor:global-flat-weighted} and Remark~\ref{rem:weighted-scaling}.|\endgroup\par
\begingroup\color{diffdel}\Verb|-\end{proof}|\endgroup\par
\begingroup\color{diffdel}\Verb|-|\endgroup\par
\begingroup\color{diffdel}\Verb|-|\endgroup\par
\begingroup\color{diffdel}\Verb|-We now return to the global construction.  Fix $\varepsilon>0$, and choose the partition and $m$ so that|\endgroup\par
\begingroup\color{diffdel}\Verb|-$\Mass(R_{\mathrm{glue}})\le\varepsilon/2$.  Define|\endgroup\par
\begingroup\color{diffadd}\Verb|+\Mass(R)+\Mass(Q)\le 2\delta.|\endgroup\par
\begingroup\color{diffadd}\Verb|+\]|\endgroup\par
\begingroup\color{diffadd}\Verb|+Since $\partial(\partial T^{\mathrm{raw}})=0$, we have $\partial R=0$.|\endgroup\par
\begingroup\color{diffadd}\Verb|+Moreover $R$ is itself a boundary in $X$ because|\endgroup\par
\begingroup\color{diffadd}\Verb|+\[|\endgroup\par
\begingroup\color{diffadd}\Verb|+R=\partial T^{\mathrm{raw}}-\partial Q=\partial\!\bigl(T^{\mathrm{raw}}-Q\bigr).|\endgroup\par
\begingroup\color{diffadd}\Verb|+\]|\endgroup\par
\begingroup\color{diffadd}\Verb|+Let $k:=2n-2p$ (the dimension of $T^{\mathrm{raw}}$).|\endgroup\par
\begingroup\color{diffadd}\Verb|+For $\delta$ sufficiently small we have $\Mass(R)\le 2\delta\le \delta_X$ from Lemma~\ref{lem:FF-filling-X}, hence there exists an integral|\endgroup\par
\begingroup\color{diffadd}\Verb|+$k$--current $Q_R$ with $\partial Q_R=R$ and|\endgroup\par
\begingroup\color{diffadd}\Verb|+\[|\endgroup\par
\begingroup\color{diffadd}\Verb|+\Mass(Q_R)\ \le\ C_X\,\Mass(R)^{\frac{k}{k-1}}\ \le\ C_X\,(2\delta)^{\frac{k}{k-1}}.|\endgroup\par
\begingroup\color{diffadd}\Verb|+\]|\endgroup\par
\begingroup\color{diffadd}\Verb|+Define|\endgroup\par
\begingroup\color{diffadd}\Verb|+\[|\endgroup\par
\begingroup\color{diffadd}\Verb|+R_{\mathrm{glue}}:=-(Q+Q_R).|\endgroup\par
\begingroup\color{diffadd}\Verb|+\]|\endgroup\par
\begingroup\color{diffadd}\Verb|+Then $\partial R_{\mathrm{glue}}=-\partial T^{\mathrm{raw}}$ and|\endgroup\par
\begingroup\color{diffadd}\Verb|+\[|\endgroup\par
\begingroup\color{diffadd}\Verb|+\Mass(R_{\mathrm{glue}})\ \le\ \Mass(Q)+\Mass(Q_R)\ \le\ 2\delta+C_X\,(2\delta)^{\frac{k}{k-1}}|\endgroup\par
\begingroup\color{diffadd}\Verb|+\ \xrightarrow[\delta\to 0]{}\ 0,|\endgroup\par
\begingroup\color{diffadd}\Verb|+\]|\endgroup\par
\begingroup\color{diffadd}\Verb|+as claimed.|\endgroup\par
\begingroup\color{diffadd}\Verb|+\end{proof}|\endgroup\par
\begingroup\color{diffadd}\Verb|+|\endgroup\par
\begingroup\color{diffadd}\Verb|+|\endgroup\par
\begingroup\color{diffadd}\Verb|+We now return to the global construction.|\endgroup\par
\begingroup\color{diffadd}\Verb|+Fix $\varepsilon>0$, and choose the mesh/activation parameters so that the gluing correction $R_{\mathrm{glue}}$ from|\endgroup\par
\begingroup\color{diffadd}\Verb|+Proposition~\ref{prop:glue-gap} satisfies $\Mass(R_{\mathrm{glue}})\le\varepsilon/2$.|\endgroup\par
\begingroup\color{diffadd}\Verb|+Define the closed glued cycle|\endgroup\par
\begingroup\color{diffctx}\Verb| \[|\endgroup\par
\begingroup\color{diffctx}\Verb| T^{(1)}:=T^{\mathrm{raw}}+R_{\mathrm{glue}}.|\endgroup\par
\begingroup\color{diffctx}\Verb| \]|\endgroup\par
\begingroup\color{diffhunk}\Verb|@@ -5646,9 +5793,13 @@|\endgroup\par
\begingroup\color{diffctx}\Verb| |\endgroup\par
\begingroup\color{diffctx}\Verb| \medskip\noindent|\endgroup\par
\begingroup\color{diffctx}\Verb| \textbf{Substep 4.3: Forcing the cohomology class via lattice discreteness.}|\endgroup\par
\begingroup\color{diffdel}\Verb|-Fix a basis of harmonic $(2n-2p)$-forms $\{\eta_\ell\}_{\ell=1}^b$|\endgroup\par
\begingroup\color{diffdel}\Verb|-that generate $H^{2n-2p}(X,\Z)$.  The homology class of any closed|\endgroup\par
\begingroup\color{diffdel}\Verb|-integral current $T$ is determined by the pairings|\endgroup\par
\begingroup\color{diffadd}\Verb|+Fix harmonic $(2n-2p)$-forms $\{\eta_\ell\}_{\ell=1}^b$ whose cohomology classes form an integral basis of the free part|\endgroup\par
\begingroup\color{diffadd}\Verb|+$H^{2n-2p}(X,\Z)/\mathrm{tors}$.|\endgroup\par
\begingroup\color{diffadd}\Verb|+These harmonic representatives detect only the free part of integral cohomology, hence the period computation determines the class in|\endgroup\par
\begingroup\color{diffadd}\Verb|+$H_{2p}(X,\Z)/\mathrm{tors}$.|\endgroup\par
\begingroup\color{diffadd}\Verb|+If one wants an equality in full integral homology, let $m_{\mathrm{tors}}$ be the exponent of the torsion subgroup of $H_{2p}(X,\Z)$ and replace|\endgroup\par
\begingroup\color{diffadd}\Verb|+$(m,T^{(1)})$ by $(m_{\mathrm{tors}}m,\ m_{\mathrm{tors}}T^{(1)})$ (and correspondingly shrink the target $\varepsilon$), which kills any possible torsion discrepancy.|\endgroup\par
\begingroup\color{diffadd}\Verb|+The homology class of any closed integral current $T$ is determined (up to torsion) by the pairings|\endgroup\par
\begingroup\color{diffctx}\Verb| \[|\endgroup\par
\begingroup\color{diffctx}\Verb| \langle[T],[\eta_\ell]\rangle=\int_T\eta_\ell.|\endgroup\par
\begingroup\color{diffctx}\Verb| \]|\endgroup\par
\begingroup\color{diffhunk}\Verb|@@ -5718,7 +5869,7 @@|\endgroup\par
\begingroup\color{diffctx}\Verb| \[|\endgroup\par
\begingroup\color{diffctx}\Verb! \Bigl|\int_{T^{\mathrm{raw}}}\eta_\ell - m\,I_\ell\Bigr|<\tfrac12.!\endgroup\par
\begingroup\color{diffctx}\Verb| \]|\endgroup\par
\begingroup\color{diffdel}\Verb|-Moreover, the gluing correction $R_{\mathrm{glue}}$ has arbitrarily small mass, hence|\endgroup\par
\begingroup\color{diffadd}\Verb|+Moreover, the gluing correction $R_{\mathrm{glue}}$ has arbitrarily small mass (Proposition~\ref{prop:glue-gap}), hence|\endgroup\par
\begingroup\color{diffctx}\Verb| its pairing with each fixed smooth $\eta_\ell$ is arbitrarily small:|\endgroup\par
\begingroup\color{diffctx}\Verb! $\bigl|\int_{R_{\mathrm{glue}}}\eta_\ell\bigr|\le \|\eta_\ell\|_{C^0}\Mass(R_{\mathrm{glue}})$.!\endgroup\par
\begingroup\color{diffctx}\Verb| Choosing parameters so that this error is $<\tfrac12$ as well yields|\endgroup\par
\begingroup\color{diffhunk}\Verb|@@ -5726,15 +5877,14 @@|\endgroup\par
\begingroup\color{diffctx}\Verb! \Bigl|\int_{T^{(1)}}\eta_\ell - m\,I_\ell\Bigr|<1,!\endgroup\par
\begingroup\color{diffctx}\Verb| \qquad T^{(1)}=T^{\mathrm{raw}}+R_{\mathrm{glue}}.|\endgroup\par
\begingroup\color{diffctx}\Verb| \]|\endgroup\par
\begingroup\color{diffdel}\Verb|-Since $\int_{T^{(1)}}\eta_\ell\in\Z$ (integral current against an integral class),|\endgroup\par
\begingroup\color{diffadd}\Verb|+Since $\int_{T^{(1)}}\eta_\ell\in\Z$ (Lemma~\ref{lem:integral-periods}),|\endgroup\par
\begingroup\color{diffctx}\Verb| we conclude $\int_{T^{(1)}}\eta_\ell = m\,I_\ell$ for all $\ell$.|\endgroup\par
\begingroup\color{diffctx}\Verb| Hence|\endgroup\par
\begingroup\color{diffctx}\Verb| \[|\endgroup\par
\begingroup\color{diffctx}\Verb| [T^{(1)}]=\mathrm{PD}(m[\gamma]).|\endgroup\par
\begingroup\color{diffctx}\Verb| \]|\endgroup\par
\begingroup\color{diffctx}\Verb| |\endgroup\par
\begingroup\color{diffdel}\Verb|-Set $R_\varepsilon:=R_{\mathrm{glue}}$ (plus any additional small|\endgroup\par
\begingroup\color{diffdel}\Verb|-fillings), and $T_\varepsilon:=T^{(1)}$.  This satisfies all requirements.|\endgroup\par
\begingroup\color{diffadd}\Verb|+Set $R_\varepsilon:=R_{\mathrm{glue}}$ and $T_\varepsilon:=T^{(1)}$.  This satisfies all requirements.|\endgroup\par
\begingroup\color{diffctx}\Verb| \end{proof}|\endgroup\par
\begingroup\color{diffctx}\Verb| |\endgroup\par
\begingroup\color{diffctx}\Verb| Let $\{\Theta_\ell\}_{\ell=1}^{b}$ be a fixed integral basis of|\endgroup\par
\begingroup\color{diffhunk}\Verb|@@ -5768,8 +5918,12 @@|\endgroup\par
\begingroup\color{diffctx}\Verb| \end{lemma}|\endgroup\par
\begingroup\color{diffctx}\Verb| |\endgroup\par
\begingroup\color{diffctx}\Verb| \begin{proof}|\endgroup\par
\begingroup\color{diffdel}\Verb|-By definition of integral homology and the de Rham isomorphism, the period of $T$ on any integral cohomology class is an integer.|\endgroup\par
\begingroup\color{diffdel}\Verb|-Explicitly, if $T$ represents an element of $H_k(X,\Z)$ and $[\eta]\in H^k(X,\Z)$, then $\langle [T],[\eta]\rangle\in\Z$ by the universal coefficient theorem.|\endgroup\par
\begingroup\color{diffadd}\Verb|+An integral cycle $T$ determines a class $[T]\in H_k(X,\Z)$ (see Federer, \emph{Geometric Measure Theory}, 1969, \S4.1).|\endgroup\par
\begingroup\color{diffadd}\Verb|+If $[\eta]\in H^k(X,\Z)$ is an integral cohomology class, then the de~Rham pairing gives|\endgroup\par
\begingroup\color{diffadd}\Verb|+\[|\endgroup\par
\begingroup\color{diffadd}\Verb|+\int_T \eta \;=\; \langle [T],[\eta]\rangle \in \Z,|\endgroup\par
\begingroup\color{diffadd}\Verb|+\]|\endgroup\par
\begingroup\color{diffadd}\Verb|+since $H^k(X,\Z)$ pairs integrally with $H_k(X,\Z)$ (universal coefficient theorem / de~Rham theorem).|\endgroup\par
\begingroup\color{diffctx}\Verb| \end{proof}|\endgroup\par
\begingroup\color{diffctx}\Verb| |\endgroup\par
\begingroup\color{diffctx}\Verb| \begin{lemma}[Lattice discreteness]\label{lem:lattice-discreteness}|\endgroup\par
\begingroup\color{diffhunk}\Verb|@@ -5787,11 +5941,17 @@|\endgroup\par
\begingroup\color{diffctx}\Verb| integers $N_{Q,j}$ appropriately, one can achieve simultaneously for all|\endgroup\par
\begingroup\color{diffctx}\Verb| $\ell=1,\ldots,b$ that|\endgroup\par
\begingroup\color{diffctx}\Verb| \[|\endgroup\par
\begingroup\color{diffdel}\Verb!-\biggl|\sum_Q S_Q(\Theta_\ell) - m\,I_\ell\biggr| < \tfrac12.!\endgroup\par
\begingroup\color{diffdel}\Verb|-\]|\endgroup\par
\begingroup\color{diffdel}\Verb|-Consequently, by integrality, $\sum_Q S_Q(\Theta_\ell) = m\,I_\ell$ for|\endgroup\par
\begingroup\color{diffdel}\Verb|-all $\ell$, i.e., the class of $\sum_Q S_Q$ in $H_{2(n-p)}(X,\Z)$ equals|\endgroup\par
\begingroup\color{diffdel}\Verb|-$\mathrm{PD}(m[\gamma])$.|\endgroup\par
\begingroup\color{diffadd}\Verb!+\biggl|\sum_Q S_Q(\Theta_\ell) - m\,I_\ell\biggr| < \tfrac14.!\endgroup\par
\begingroup\color{diffadd}\Verb|+\]|\endgroup\par
\begingroup\color{diffadd}\Verb|+Let $S:=\sum_Q S_Q$ and let $U_\epsilon$ be any integral $(2n-2p)$--current with $\partial U_\epsilon=\partial S$ and|\endgroup\par
\begingroup\color{diffadd}\Verb|+\[|\endgroup\par
\begingroup\color{diffadd}\Verb!+\Mass(U_\epsilon)\ <\ \min\Bigl\{\epsilon,\ \frac{1}{4\,\max_\ell\|\Theta_\ell\|_{C^0}}\Bigr\}.!\endgroup\par
\begingroup\color{diffadd}\Verb|+\]|\endgroup\par
\begingroup\color{diffadd}\Verb|+Then $T_\epsilon:=S-U_\epsilon$ is a closed integral cycle and|\endgroup\par
\begingroup\color{diffadd}\Verb|+\[|\endgroup\par
\begingroup\color{diffadd}\Verb|+\int_{T_\epsilon}\Theta_\ell\ =\ m\,I_\ell\qquad\text{for all }\ell=1,\dots,b.|\endgroup\par
\begingroup\color{diffadd}\Verb|+\]|\endgroup\par
\begingroup\color{diffadd}\Verb|+In particular, $[T_\epsilon]=\mathrm{PD}(m[\gamma])$ in $H_{2(n-p)}(X,\Z)/\mathrm{tors}$ (equivalently in $H_{2(n-p)}(X,\Q)$).|\endgroup\par
\begingroup\color{diffctx}\Verb| \end{proposition}|\endgroup\par
\begingroup\color{diffctx}\Verb| |\endgroup\par
\begingroup\color{diffctx}\Verb| \begin{proof}|\endgroup\par
\begingroup\color{diffhunk}\Verb|@@ -5823,32 +5983,43 @@|\endgroup\par
\begingroup\color{diffctx}\Verb| \[|\endgroup\par
\begingroup\color{diffctx}\Verb! \|v_{Q,j}\|_{\ell^\infty}\ \le\ C_0\,h^{2(n-p)}.!\endgroup\par
\begingroup\color{diffctx}\Verb| \]|\endgroup\par
\begingroup\color{diffdel}\Verb|-Choose the mesh $h$ so small that $C_0\,h^{2(n-p)}\le \frac{1}{4b}$.|\endgroup\par
\begingroup\color{diffadd}\Verb|+Choose the mesh $h$ so small that $C_0\,h^{2(n-p)}\le \frac{1}{8b}$.|\endgroup\par
\begingroup\color{diffctx}\Verb| |\endgroup\par
\begingroup\color{diffctx}\Verb| \smallskip\noindent|\endgroup\par
\begingroup\color{diffctx}\Verb| \textbf{Step 3: Apply B\'ar\'any--Grinberg.}|\endgroup\par
\begingroup\color{diffctx}\Verb| Apply Lemma~\ref{lem:barany-grinberg} in dimension $d=b$ to the normalized vectors|\endgroup\par
\begingroup\color{diffdel}\Verb!-$\widetilde v_{Q,j}:=(4b)\,v_{Q,j}$ (so $\|\widetilde v_{Q,j}\|_{\ell^\infty}\le 1$) with coefficients $a_{Q,j}$.!\endgroup\par
\begingroup\color{diffadd}\Verb!+$\widetilde v_{Q,j}:=(8b)\,v_{Q,j}$ (so $\|\widetilde v_{Q,j}\|_{\ell^\infty}\le 1$) with coefficients $a_{Q,j}$.!\endgroup\par
\begingroup\color{diffctx}\Verb| This yields choices $\varepsilon_{Q,j}\in\{0,1\}$ such that|\endgroup\par
\begingroup\color{diffctx}\Verb| \[|\endgroup\par
\begingroup\color{diffctx}\Verb! \Bigl\|\sum_{Q,j} (\varepsilon_{Q,j}-a_{Q,j})\,\widetilde v_{Q,j}\Bigr\|_{\ell^\infty}\ \le\ b.!\endgroup\par
\begingroup\color{diffctx}\Verb| \]|\endgroup\par
\begingroup\color{diffctx}\Verb| Undoing the normalization gives|\endgroup\par
\begingroup\color{diffctx}\Verb| \[|\endgroup\par
\begingroup\color{diffdel}\Verb!-\Bigl\|\sum_{Q,j} (\varepsilon_{Q,j}-a_{Q,j})\, v_{Q,j}\Bigr\|_{\ell^\infty}\ \le\ \frac{1}{4}.!\endgroup\par
\begingroup\color{diffadd}\Verb!+\Bigl\|\sum_{Q,j} (\varepsilon_{Q,j}-a_{Q,j})\, v_{Q,j}\Bigr\|_{\ell^\infty}\ \le\ \frac{1}{8}.!\endgroup\par
\begingroup\color{diffctx}\Verb| \]|\endgroup\par
\begingroup\color{diffctx}\Verb| Equivalently, for every $\ell$,|\endgroup\par
\begingroup\color{diffctx}\Verb| \[|\endgroup\par
\begingroup\color{diffdel}\Verb!-\Bigl|\sum_{Q,j} (N_{Q,j}-n_{Q,j})\,\int_{Y_{Q,j}\cap Q}\Theta_\ell\Bigr|\ \le\ \frac{1}{4}.!\endgroup\par
\begingroup\color{diffadd}\Verb!+\Bigl|\sum_{Q,j} (N_{Q,j}-n_{Q,j})\,\int_{Y_{Q,j}\cap Q}\Theta_\ell\Bigr|\ \le\ \frac{1}{8}.!\endgroup\par
\begingroup\color{diffctx}\Verb| \]|\endgroup\par
\begingroup\color{diffctx}\Verb| Thus, provided the continuous targets $n_{Q,j}$ were chosen so that|\endgroup\par
\begingroup\color{diffdel}\Verb|-$\sum_{Q,j} n_{Q,j}\int_{Y_{Q,j}\cap Q}\Theta_\ell$ equals $mI_\ell$ up to $<\frac14$ error (achieved by taking $\delta$ small in the local|\endgroup\par
\begingroup\color{diffadd}\Verb|+$\sum_{Q,j} n_{Q,j}\int_{Y_{Q,j}\cap Q}\Theta_\ell$ equals $mI_\ell$ up to $<\frac18$ error (achieved by taking $\delta$ small in the local|\endgroup\par
\begingroup\color{diffctx}\Verb| Carath\'eodory approximation), we obtain|\endgroup\par
\begingroup\color{diffctx}\Verb| \[|\endgroup\par
\begingroup\color{diffdel}\Verb!-\Bigl|\sum_Q S_Q(\Theta_\ell)-mI_\ell\Bigr|<\frac12!\endgroup\par
\begingroup\color{diffadd}\Verb!+\Bigl|\sum_Q S_Q(\Theta_\ell)-mI_\ell\Bigr|<\frac14!\endgroup\par
\begingroup\color{diffctx}\Verb| \qquad\text{for all }\ell=1,\dots,b.|\endgroup\par
\begingroup\color{diffctx}\Verb| \]|\endgroup\par
\begingroup\color{diffdel}\Verb|-The integrality conclusion is then as stated.|\endgroup\par
\begingroup\color{diffadd}\Verb|+Now choose the gluing correction $U_\epsilon$ so that $\partial U_\epsilon=\partial S$ and|\endgroup\par
\begingroup\color{diffadd}\Verb!+$\Mass(U_\epsilon)<\frac{1}{4\max_\ell\|\Theta_\ell\|_{C^0}}$, so that $|\int_{U_\epsilon}\Theta_\ell|<\frac14$ for all $\ell$.!\endgroup\par
\begingroup\color{diffadd}\Verb|+Then $T_\epsilon:=S-U_\epsilon$ is a closed integral cycle, so by Lemma~\ref{lem:integral-periods} each $\int_{T_\epsilon}\Theta_\ell\in\Z$.|\endgroup\par
\begingroup\color{diffadd}\Verb|+Moreover,|\endgroup\par
\begingroup\color{diffadd}\Verb|+\[|\endgroup\par
\begingroup\color{diffadd}\Verb!+\Bigl|\int_{T_\epsilon}\Theta_\ell-mI_\ell\Bigr|!\endgroup\par
\begingroup\color{diffadd}\Verb|+\le|\endgroup\par
\begingroup\color{diffadd}\Verb!+\Bigl|\int_S\Theta_\ell-mI_\ell\Bigr|+\Bigl|\int_{U_\epsilon}\Theta_\ell\Bigr|!\endgroup\par
\begingroup\color{diffadd}\Verb|+<\tfrac12,|\endgroup\par
\begingroup\color{diffadd}\Verb|+\]|\endgroup\par
\begingroup\color{diffadd}\Verb|+so Lemma~\ref{lem:lattice-discreteness} forces $\int_{T_\epsilon}\Theta_\ell=mI_\ell$ for all $\ell$.|\endgroup\par
\begingroup\color{diffadd}\Verb|+This determines $[T_\epsilon]$ in $H_{2(n-p)}(X,\Z)/\mathrm{tors}$ (hence in rational homology), as claimed.|\endgroup\par
\begingroup\color{diffctx}\Verb| \end{proof}|\endgroup\par
\begingroup\color{diffctx}\Verb| |\endgroup\par
\begingroup\color{diffctx}\Verb| % ------------------------------------------------------------|\endgroup\par
\begingroup\color{diffhunk}\Verb|@@ -5885,9 +6056,9 @@|\endgroup\par
\begingroup\color{diffctx}\Verb| T_\epsilon := S - U_\epsilon,|\endgroup\par
\begingroup\color{diffctx}\Verb| \qquad \partial T_\epsilon=0.|\endgroup\par
\begingroup\color{diffctx}\Verb| \]|\endgroup\par
\begingroup\color{diffdel}\Verb|-By construction, the homology class|\endgroup\par
\begingroup\color{diffdel}\Verb|-$[T_\epsilon]=[S]=\mathrm{PD}(m[\gamma])$|\endgroup\par
\begingroup\color{diffdel}\Verb|-(Proposition~\ref{prop:cohomology-match}).  Moreover, calibratedness|\endgroup\par
\begingroup\color{diffadd}\Verb|+By Proposition~\ref{prop:cohomology-match}, the closed integral cycle $T_\epsilon$ satisfies|\endgroup\par
\begingroup\color{diffadd}\Verb|+$[T_\epsilon]=\mathrm{PD}(m[\gamma])$ in $H_{2(n-p)}(X,\Z)/\mathrm{tors}$.|\endgroup\par
\begingroup\color{diffadd}\Verb|+Moreover, calibratedness|\endgroup\par
\begingroup\color{diffctx}\Verb| of the $S_Q$ pieces gives|\endgroup\par
\begingroup\color{diffctx}\Verb| \[|\endgroup\par
\begingroup\color{diffctx}\Verb| \Mass(T_\epsilon)|\endgroup\par
\begingroup\color{diffhunk}\Verb|@@ -5897,7 +6068,7 @@|\endgroup\par
\begingroup\color{diffctx}\Verb| since $\Mass(U_\epsilon)\to 0$.|\endgroup\par
\begingroup\color{diffctx}\Verb| |\endgroup\par
\begingroup\color{diffctx}\Verb| \begin{proposition}[Almost--calibration and global mass convergence for the glued cycles]\label{prop:almost-calibration}|\endgroup\par
\begingroup\color{diffdel}\Verb|-Let $S$ be an integral $(2n-2p)$--current (typically not closed) in the class $\mathrm{PD}(m[\gamma])$.|\endgroup\par
\begingroup\color{diffadd}\Verb|+Let $S$ be an integral $(2n-2p)$--current (typically not closed) built as a sum of local $\psi$--calibrated sheet pieces.|\endgroup\par
\begingroup\color{diffctx}\Verb| Let $U_\epsilon$ be integral currents such that|\endgroup\par
\begingroup\color{diffctx}\Verb| \[|\endgroup\par
\begingroup\color{diffctx}\Verb| \partial U_\epsilon=\partial S,|\endgroup\par
\begingroup\color{diffhunk}\Verb|@@ -5912,6 +6083,7 @@|\endgroup\par
\begingroup\color{diffctx}\Verb| \qquad|\endgroup\par
\begingroup\color{diffctx}\Verb| \partial T_\epsilon=0.|\endgroup\par
\begingroup\color{diffctx}\Verb| \]|\endgroup\par
\begingroup\color{diffadd}\Verb|+Assume (as ensured by Proposition~\ref{prop:cohomology-match}) that $[T_\epsilon]=\mathrm{PD}(m[\gamma])$ in $H_{2(n-p)}(X,\Q)$.|\endgroup\par
\begingroup\color{diffctx}\Verb| Then:|\endgroup\par
\begingroup\color{diffctx}\Verb| \begin{enumerate}|\endgroup\par
\begingroup\color{diffctx}\Verb| \item[\textnormal{(i)}] \textbf{Exact calibration pairing.}|\endgroup\par
\begingroup\color{diffhunk}\Verb|@@ -5944,7 +6116,9 @@|\endgroup\par
\begingroup\color{diffctx}\Verb| \end{proposition}|\endgroup\par
\begingroup\color{diffctx}\Verb| |\endgroup\par
\begingroup\color{diffctx}\Verb| \begin{proof}|\endgroup\par
\begingroup\color{diffdel}\Verb|-By construction, each local sheet current $S_Q$ is holomorphic and hence $\psi$--calibrated, so their sum $S$ is $\psi$--calibrated.|\endgroup\par
\begingroup\color{diffadd}\Verb|+By construction, each local sheet current $S_Q$ is holomorphic and hence $\psi$--calibrated, and the sheet pieces are chosen disjointly on each cell $Q$|\endgroup\par
\begingroup\color{diffadd}\Verb|+(cf.\ the disjointness requirements in the local manufacturing step).|\endgroup\par
\begingroup\color{diffadd}\Verb|+Therefore the sum $S=\sum_Q S_Q$ is $\psi$--calibrated and evaluation/mass add without cancellation.|\endgroup\par
\begingroup\color{diffctx}\Verb| In particular,|\endgroup\par
\begingroup\color{diffctx}\Verb| \[|\endgroup\par
\begingroup\color{diffctx}\Verb| \Mass(S)=\int_S\psi.|\endgroup\par
\begingroup\color{diffhunk}\Verb|@@ -6027,17 +6201,21 @@|\endgroup\par
\begingroup\color{diffctx}\Verb| This is the compactness/normalization needed for Federer--Fleming.|\endgroup\par
\begingroup\color{diffctx}\Verb| |\endgroup\par
\begingroup\color{diffctx}\Verb| \medskip\noindent|\endgroup\par
\begingroup\color{diffdel}\Verb|-\textbf{Substep 6.2: Varifold compactness \cite{Allard72,Sim83}.}|\endgroup\par
\begingroup\color{diffadd}\Verb|+\textbf{Substep 6.2: Compactness (Federer--Fleming + Allard).}|\endgroup\par
\begingroup\color{diffctx}\Verb| Let $V_k$ be the associated integral varifold of $T_k$.  Uniform mass|\endgroup\par
\begingroup\color{diffdel}\Verb|-bound gives tightness; Allard's compactness theorem (Allard, ``On the|\endgroup\par
\begingroup\color{diffdel}\Verb|-first variation of a varifold,'' Ann.~of Math.~95 (1972), 417--491)|\endgroup\par
\begingroup\color{diffdel}\Verb|-gives, after passing to a subsequence (not relabeled):|\endgroup\par
\begingroup\color{diffadd}\Verb|+bound gives tightness.|\endgroup\par
\begingroup\color{diffadd}\Verb|+Since $\partial T_k=0$ and $\sup_k \Mass(T_k)<\infty$, the Federer--Fleming compactness theorem for integral currents|\endgroup\par
\begingroup\color{diffadd}\Verb|+(Federer--Fleming, \emph{Normal and integral currents}, Ann.~of Math.~72 (1960), 458--520; see also Federer, \emph{GMT}, 1969)|\endgroup\par
\begingroup\color{diffadd}\Verb|+gives, after passing to a subsequence (not relabeled), flat convergence $T_k\to T$ to an integral cycle.|\endgroup\par
\begingroup\color{diffadd}\Verb|+In parallel, Allard's compactness theorem for integral varifolds (Allard, Ann.~of Math.~95 (1972), 417--491)|\endgroup\par
\begingroup\color{diffadd}\Verb|+gives varifold convergence $V_k\to V$.|\endgroup\par
\begingroup\color{diffctx}\Verb| \begin{itemize}|\endgroup\par
\begingroup\color{diffctx}\Verb| \item $V_k\to V$ as varifolds;|\endgroup\par
\begingroup\color{diffctx}\Verb| \item $T_k\to T$ as integral currents in the flat norm;|\endgroup\par
\begingroup\color{diffctx}\Verb| \item $T$ is an integral $(2n-2p)$-cycle with $\partial T=0$;|\endgroup\par
\begingroup\color{diffdel}\Verb|-\item By homological continuity, $[T]=\mathrm{PD}(m[\gamma])$ (since|\endgroup\par
\begingroup\color{diffdel}\Verb|-$T_k$ and $T$ differ by a boundary and cohomology is discrete).|\endgroup\par
\begingroup\color{diffadd}\Verb|+\item By the period constraints of Proposition~\ref{prop:cohomology-match} (applied to $T_k$) and continuity of current evaluation under flat convergence,|\endgroup\par
\begingroup\color{diffadd}\Verb|+the limit $T$ has the same pairings with a fixed integral basis $\{\Theta_\ell\}$ of $H^{2(n-p)}(X,\Z)$; hence $[T]=\mathrm{PD}(m[\gamma])$|\endgroup\par
\begingroup\color{diffadd}\Verb|+in $H_{2(n-p)}(X,\Z)/\mathrm{tors}$ (equivalently in $H_{2(n-p)}(X,\Q)$).|\endgroup\par
\begingroup\color{diffctx}\Verb| \end{itemize}|\endgroup\par
\begingroup\color{diffctx}\Verb| Lower semicontinuity gives|\endgroup\par
\begingroup\color{diffctx}\Verb| \begin{equation}\label{eq:mass-lsc}|\endgroup\par
\begingroup\color{diffhunk}\Verb|@@ -6187,6 +6365,7 @@|\endgroup\par
\begingroup\color{diffctx}\Verb| \[|\endgroup\par
\begingroup\color{diffctx}\Verb| \mathcal F\!\left(\partial T^{\mathrm{raw}}\right)\ \le\ \varepsilon_{\mathrm{glue}}(m,\delta,\varepsilon,\mathrm{mesh})\cdot m,|\endgroup\par
\begingroup\color{diffctx}\Verb| \]|\endgroup\par
\begingroup\color{diffadd}\Verb|+with $\varepsilon_{\mathrm{glue}}\to 0$ in the global parameter schedule (see Lemma~\ref{lem:flatnorm-o-m}).|\endgroup\par
\begingroup\color{diffctx}\Verb| This is the robust target because the individual face mismatches can have large mass even when there is strong cancellation.|\endgroup\par
\begingroup\color{diffctx}\Verb| \medskip\noindent|\endgroup\par
\begingroup\color{diffctx}\Verb| Concretely, by the dual characterization of $\mathcal F$ and Stokes, for every smooth|\endgroup\par
\begingroup\color{diffhunk}\Verb|@@ -6196,9 +6375,11 @@|\endgroup\par
\begingroup\color{diffctx}\Verb| \]|\endgroup\par
\begingroup\color{diffctx}\Verb| Since $\beta$ is closed and $X$ has no boundary, $\int_X (m\beta)\wedge d\eta=\pm\int_X d(m\beta\wedge\eta)=0$.|\endgroup\par
\begingroup\color{diffctx}\Verb| Thus the remaining task is to make the approximation error quantitative in terms of|\endgroup\par
\begingroup\color{diffdel}\Verb|-$(\delta,\varepsilon,\mathrm{mesh},m)$; see Proposition~\ref{prop:glue-gap}.|\endgroup\par
\begingroup\color{diffadd}\Verb|+$(\delta,\varepsilon,\mathrm{mesh},m)$; this is achieved by the corner-exit bookkeeping and the scaling/packing lemma (Lemma~\ref{lem:flatnorm-o-m},|\endgroup\par
\begingroup\color{diffadd}\Verb|+with the borderline case treated by Lemma~\ref{lem:borderline-p-half}).|\endgroup\par
\begingroup\color{diffctx}\Verb| Once $\mathcal F(\partial T^{\mathrm{raw}})$ is small, the correction current $R_{\mathrm{glue}}$ is produced by|\endgroup\par
\begingroup\color{diffdel}\Verb|-the flat-norm decomposition and the Federer--Fleming isoperimetric inequality as in Substep~4.2.|\endgroup\par
\begingroup\color{diffadd}\Verb|+the flat-norm decomposition together with the Federer--Fleming filling estimate on $X$ (Lemma~\ref{lem:FF-filling-X}), as packaged in|\endgroup\par
\begingroup\color{diffadd}\Verb|+Proposition~\ref{prop:glue-gap}.|\endgroup\par
\begingroup\color{diffctx}\Verb| The smoothness of $\beta$ is essential here---it ensures the local|\endgroup\par
\begingroup\color{diffctx}\Verb| decompositions are compatible across cube boundaries.|\endgroup\par
\begingroup\color{diffctx}\Verb| \end{remark}|\endgroup\par
\begingroup\color{diffhunk}\Verb|@@ -6504,7 +6685,7 @@|\endgroup\par
\begingroup\color{diffctx}\Verb| 	\begin{itemize}|\endgroup\par
\begingroup\color{diffctx}\Verb| 	\item \textbf{Citation (primary):} H.~Federer and W.~H.~Fleming, \emph{Normal and integral currents}, Annals of Mathematics \textbf{72} (1960), 458--520.|\endgroup\par
\begingroup\color{diffctx}\Verb| 	\item \textbf{Citation (textbook):} H.~Federer, \emph{Geometric Measure Theory}, Springer (1969); L.~Simon, \emph{Lectures on Geometric Measure Theory}, ANU (1983).|\endgroup\par
\begingroup\color{diffdel}\Verb|-	\item \textbf{Used in this manuscript:} Theorem~\ref{thm:realization-from-almost} (compactness of integral currents under mass bounds); Proposition~\ref{prop:glue-gap} and Substep~4.2 (isoperimetric filling to control $\Mass(R_{\mathrm{glue}})$ from $\mathcal F(\partial T^{\mathrm{raw}})$).|\endgroup\par
\begingroup\color{diffadd}\Verb|+	\item \textbf{Used in this manuscript:} Theorem~\ref{thm:realization-from-almost} (compactness of integral currents under mass bounds); Proposition~\ref{prop:glue-gap} and Substep~4.2 (constructing a small-mass correction $R_{\mathrm{glue}}$ with $\partial R_{\mathrm{glue}}=-\partial T^{\mathrm{raw}}$ via the flat-norm/isoperimetric filling).|\endgroup\par
\begingroup\color{diffctx}\Verb| 	\item \textbf{Hypotheses checked here:} $X$ is compact (mass bounds yield tightness); all currents are integral; dimension is finite.|\endgroup\par
\begingroup\color{diffctx}\Verb| 	\end{itemize}|\endgroup\par
\begingroup\color{diffctx}\Verb| |\endgroup\par
\end{document}
