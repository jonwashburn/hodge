\documentclass[11pt]{article}

\usepackage[T1]{fontenc}
\usepackage[utf8]{inputenc}
\usepackage{lmodern}
\usepackage{geometry}
\usepackage{amsmath,amssymb}
\usepackage{hyperref}

% Local operator definitions used in the manuscript
\DeclareMathOperator{\Mass}{Mass}

\geometry{margin=1in}
\hypersetup{hidelinks}

\title{\bfseries Referee Patch Report (Amir v1)\\
\large Critical blockers raised by a hostile referee, and the concrete fixes applied}
\author{Auto-generated in \texttt{docs/referee/} from the Cursor session}
\date{\today}

\begin{document}
\maketitle

\section*{Scope}
This note summarizes two \emph{critical} referee blockers previously identified in \texttt{Hodge\_REFEREE\_Amir-v1.tex}, and documents the corresponding edits that were applied:
\begin{itemize}
\item the impossible coverage statement in Proposition \texttt{prop:dense-holo};
\item the per-cube mass matching step in Theorem \texttt{thm:global-cohom} (and its local precursor \texttt{lem:local-bary}), including the quantifier/scaling issue around fixed \(m\).
\item \textbf{hygiene/readability fixes} prompted by the AI referee notes: removal of stray duplicate proof blocks and addition of a brief convention note for “algebraic class”.
\item \textbf{analytic/auxiliary fixes}: (i) the LICD ``classical SYR'' gluing step was corrected to use flat-norm smallness (not a false global boundary-mass bound); (ii) a scaling typo was corrected in the Bergman-kernel jet-control lemma used in the holomorphic manufacturing package.
\end{itemize}
This report is \emph{not} a proof audit of the full manuscript; it is a surgical “what changed and why” memo.

\section{Blocker 1: Proposition \texttt{prop:dense-holo} (finite family covering every point)}
\subsection*{Referee objection}
The manuscript stated (paraphrasing) that for a compact set \(K\subset X\) and \(\varepsilon>0\), there exists a \emph{finite} family of proper calibrated submanifolds \(Y_1,\dots,Y_M\) such that for \emph{every} \(x\in K\) and \emph{every} calibrated plane \(\Pi\subset T_xX\), one can find \(j\) with \(x\in Y_j\) and \(T_xY_j\) \(\varepsilon\)-close to \(\Pi\).

For \(p\ge 1\) and \(K\) with nonempty interior, a finite union of proper submanifolds cannot contain all points of \(K\). Hence the original formulation is false as stated (even if morally intended as a “finite net at centers” statement).

\subsection*{Correction applied}
The statement was replaced by the correct finite-net formulation:
\begin{itemize}
\item choose finitely many centers \(x_\alpha\) covering \(K\) at scale \(\varepsilon\);
\item at each center \(x_\alpha\), realize a finite \(\varepsilon/2\)-net of calibrated planes by calibrated complete intersections \(Y_{\alpha,j}\) through \(x_\alpha\);
\item compare an arbitrary \(x\in K\) to a nearby center via a fixed identification of tangent spaces in a coordinate ball (e.g.\ chart differential or parallel transport).
\end{itemize}
An explicit “Referee note” was added directly under the proposition explaining why the stronger global coverage formulation cannot hold for a finite family.

\subsection*{Why this resolves the blocker}
The revised statement matches what the proof actually constructs and removes an impossible topological/dimension claim. It also aligns with later discretization logic: directions are approximated at finitely many sample points and propagated by continuity estimates.

\section{Blocker 2: Theorem \texttt{thm:global-cohom} (per-cube mass matching / quantifier swap)}
\subsection*{Referee objection}
In the per-cube local quantization step, the manuscript previously asserted:
\begin{enumerate}
\item “any affine calibrated sheet with tangent plane \(P_{Q,j}\) has the same \(\psi\)-mass in \(Q\), call it \(A_{Q,j}\)” (translation-independence in a cube);
\item choose integers \(N_{Q,j}\) so that \(\sum_j N_{Q,j}A_{Q,j}\approx M_Q\), where \(M_Q:=m\int_Q\beta\wedge\psi\);
\item justify existence by “\(m\) may be taken arbitrarily large.”
\end{enumerate}
The referee objection had two parts:
\begin{itemize}
\item The “common sheet mass” assertion is false for generic orientations (even in \(\mathbb R^2\), line--square intersection length depends on offset).
\item A scaling obstruction: if \(A_{Q,j}\sim h^{k}\) with \(k=2n-2p\) and \(M_Q\sim m h^{2n}\), then \(M_Q/A_{Q,j}\sim m h^{2p}\to 0\) as \(h\to 0\) for fixed \(m\). Thus integer matching fails on sufficiently fine meshes unless one lets \(m\to\infty\), contradicting the fixed-\(m\) SYR definition.
\end{itemize}

\subsection*{Correction applied}
The matching is now routed through the manuscript’s \emph{corner-exit template} mechanism:
\begin{itemize}
\item introduce a tunable footprint scale \(s\ll h:=\mathrm{side}(Q)\);
\item use the corner-exit template lemmas to choose, for each direction label \((Q,j)\), a family of sheet pieces in \(Q\) with \emph{identical} corner-exit footprints (hence equal \(\psi\)-mass within the family, up to a common small-slope distortion factor);
\item denote this common per-piece mass by \(A_{Q,j}\), with scaling \(A_{Q,j}\asymp s^{k}\) where \(k=2n-2p\);
\item choose \(s\) small enough so that \(M_Q/A_{Q,j}\gg 1\), enabling integer rounding with fixed \(m\) (no “\(m\to\infty\)” quantifier swap).
\end{itemize}
The same idea was reflected in the local precursor lemma \texttt{lem:local-bary}, which was strengthened to explicitly include the cube budget \(M_Q:=m\int_Q\beta\wedge\psi\) and a quantitative error target \(|\Mass(S_Q)-M_Q|\le \delta M_Q\).

\subsection*{Why this resolves the blocker}
The equal-mass property is no longer a false geometric claim about generic translates; it becomes a \emph{design feature} of the template box. The scaling obstruction is addressed by shrinking the per-piece mass scale \(A_{Q,j}\asymp s^k\) with \(s\ll h\), allowing many pieces per cube while keeping the total budget \(M_Q\) fixed (so fixed-\(m\) SYR quantifiers are not violated).

\section{Editorial hygiene (referee readability)}
\subsection*{Duplicate proof blocks removed}
Two locations contained \emph{duplicated} proof environments back-to-back (a common artifact when patching draft text with conditional blocks):
\begin{itemize}
\item a second proof following Lemma \texttt{lem:radial-min};
\item a second proof following the distance-to-cone proposition in the “calibrated cone” preliminaries.
\end{itemize}
These duplicates were deleted, keeping a single clean proof in each case.
Additionally, the extra duplicated proof blocks previously flagged around \texttt{lem:limit\_is\_calibrated} and \texttt{prop:almost-calibration} were removed so each result has exactly one proof in the compiled manuscript.

\subsection*{Clarification in the transport $\Rightarrow$ flat norm step}
In Proposition \texttt{prop:transport-flat-glue}, the Step~1 Lipschitz bound implicitly requires working on an \emph{interior} face patch after the standard edge-trimming/localization (so that each face slice is a cycle, i.e.\ has zero boundary as a current on the interior face).
This hypothesis was made explicit in item~(b) and the proof was adjusted accordingly (dropping the spurious boundary-slice term in the homotopy/Lipschitz estimate).

\subsection*{Notation cleanup: cohomology multiplier \(m\) vs.\ Bergman/holomorphic power}
In the weighted-scaling discussion, the symbol \(m\) was being used in two different senses (cohomology multiplier \(m\) in \(\mathrm{PD}(m[\gamma])\) and tensor power in Bergman-scale holomorphic manufacturing). This led to misleading phrases such as “\(h\downarrow 0\) (equivalently \(m\to\infty\))” and a dimensional typo involving \(\sqrt m\,h\).
The manuscript now separates these roles explicitly (using a distinct holomorphic scale parameter \(N\) where both appear simultaneously), corrects the Bergman-scale relation to \(h\sim c\,N^{-1/2}\), and removes the misleading “\(h\downarrow 0\iff m\to\infty\)” equivalence in the fixed-\(m\) SYR regime.

\subsection*{Additional notation hygiene: Carath\'eodory bound and \(\omega^p\) lemma}
Two further low-level notation collisions were removed:
\begin{itemize}
\item The Carath\'eodory bound (formerly denoted \(N(n,p)\) in the local decomposition/quantization step) was renamed to \(N_{\mathrm{Car}}=N_{\mathrm{Car}}(n,p)\) to avoid collision with the holomorphic tensor power \(N\).
\item In Lemma \texttt{lem:gamma-minus-alg} (\(\omega^p\) is algebraic), the line-bundle tensor power used to produce a complete intersection was renamed from \(m\) to \(q\) to avoid collision with the global cohomology multiplier \(m\) used throughout the SYR/microstructure closure chain.
\end{itemize}

\subsection*{Packaging clarity: global coherence and template displacement}
Two small “reader-facing” clarifications were added to reduce hidden-quantifier concerns in the gluing hinge:
\begin{itemize}
\item In Proposition \texttt{prop:global-coherence-all-labels}, the proof now explicitly defines the integer counts \(N_{Q,v,i}=\lfloor M_{Q,v,i}/\mu_{Q,v,i}\rceil\) (vertex-split budgets and per-piece mass scales) and clarifies that \(m\) in the period-locking step is the \emph{fixed} cohomology multiplier from the global parameter schedule.
\item In Lemma \texttt{lem:template-displacement}, the proof now explicitly bounds the summed slice-mass contribution by \(h^{-1}M_F\) (available in either the rounded-cell or corner-exit regimes), justifying the stated \(C_{\angle}\,\varepsilon\,M_F\) term from the small-angle/model-error estimate.
\end{itemize}

\subsection*{Uniformly convex slice-boundary inequality: proof hygiene}
In Lemma \texttt{lem:uniformly-convex-slice-boundary}, we added two short referee-facing clarifications:
\begin{itemize}
\item an explicit bound for the large-volume case \(v(t)\ge v_0\), using a parallel-body/Steiner argument to bound \(\mathcal H^{k-1}(\partial K_t)\) uniformly when the slice \(K_t\) lies in a fixed-radius ball;
\item an explicit choice of the normal direction \(u\) for the small-volume case, taking \(t_0\) as a nearest boundary point in the projection \(D=\pi(Q)\) and setting \(u=(t_0-t)/\|t_0-t\|\), so that the parameterization \(t=t_0-su\) is fully justified.
\end{itemize}

\subsection*{Borderline case \(p=n/2\): consistent endgame packaging}
The TeX now treats the middle-dimensional case \(p=n/2\) consistently across the parameter schedule and the H2-package summary:
\begin{itemize}
\item Lemma \texttt{lem:borderline-p-half} is used as the explicit closure mechanism in the borderline regime, showing
\(\mathcal F(\partial T^{\mathrm{raw}})\to 0\) under the refined displacement schedule \(\varrho=o(\varepsilon)\) (together with the footprint-scale packing bound).
\item Consequently, the same implication
\(\mathcal F(\partial T^{\mathrm{raw}}_j)\to 0 \Rightarrow \Mass(U_\epsilon)\to 0\) via Proposition \texttt{prop:glue-gap}
feeds into the exact-class rounding step \texttt{prop:cohomology-match} without any “extra closure input” placeholder.
\end{itemize}

\subsection*{Step 5 (period locking): making the “\(R\) bounds integrally” point explicit}
In the Step~5 boundary-correction subsection (used inside Proposition \texttt{prop:cohomology-match}), the TeX now explicitly notes that
in an integral flat-norm decomposition \(\partial S=R+\partial Q\) one has \(\partial R=0\) and \(R=\partial(S-Q)\).
This makes it completely formal that \(R\) bounds in \(X\) by an \emph{integral} current, so Lemma \texttt{lem:FF-filling-X}
applies without any hidden homological step.

\subsection*{Final closure chain: indexing and compactness minimality}
Two final referee-facing hygiene improvements were made in the SYR endgame:
\begin{itemize}
\item In Proposition \texttt{prop:almost-calibration} and Theorem \texttt{thm:syr-realization}, the sheet current is now written as \(S_\epsilon\) (rather than a fixed \(S\)),
so the family \(T_\epsilon:=S_\epsilon-U_\epsilon\) is unambiguous as \(\epsilon\downarrow 0\).
\item In Theorem \texttt{thm:syr-realization}, the compactness step is now phrased purely in the Federer--Fleming framework for integral currents; varifold language was removed since it was not used in the argument.
\end{itemize}

\subsection*{Classical SYR under LICD: gluing corrected to use flat norm (not boundary mass)}
In the auxiliary theorem \texttt{thm:classical-syr-licd}, the original Step~3 asserted a global estimate of the form
\(\Mass(\partial \sum_Q S_Q)\lesssim C\varepsilon\) at mesh scale \(\varepsilon\).
As a global mass statement this is not robust (and is generally false in mesh-refinement regimes).
The TeX now correctly phrases the closure step in the \emph{flat norm}:
\(\mathcal F(\partial S^{\mathrm{raw}}_\varepsilon)\to 0\) by duality plus Stokes (using that \(d\beta=0\)),
then applies a standard filling/isoperimetric inequality to produce a filling current \(R_\varepsilon\) with \(\Mass(R_\varepsilon)\to 0\).
This yields a closed cycle \(T_\varepsilon=S^{\mathrm{raw}}_\varepsilon+R_\varepsilon\) with quantitative almost-calibration
\(0\le \Def_{\mathrm{cal}}(T_\varepsilon)\le 2\,\Mass(R_\varepsilon)\to 0\).

\subsection*{Bergman kernel jet-control lemma: scaling typo corrected}
In Lemma \texttt{lem:bergman-control}, the proof constructs basis sections by differentiating the Bergman kernel and rescaling.
For the resulting \(1\)-jets \(ds_{a,N}(0)\) to be \(O(1)\) (hence uniformly invertible) on Bergman balls of radius \(\asymp N^{-1/2}\),
the normalization factor must be \(N^{-(n+1/2)}\) (not \(N^{-(n+1)/2}\)).
This scaling was corrected in the TeX, and the downstream derivative comparison in the proof was rewritten in the stable form
\(\sup_{|Z|\le\sigma}\|ds_{a,N}(Z)-ds_{a,N}(0)\|\le \varepsilon\).

\subsection*{Local multi-sheet construction: translate-mass dependence made explicit}
In Theorem \texttt{thm:local-sheets}, the proof previously treated the \(\psi\)-mass of a flat model translate \((\widetilde\Pi_j+t)\cap Q\) as independent of \(t\), which is false for generic cube geometry.
The proof now treats the translate mass \(A_j(t)\) as a continuous function of \(t\) and chooses translations within a small radius so that all masses in a family agree up to \(O(\delta)\), which is sufficient for the subsequent Diophantine rounding step.
The internal rounding parameter was also renamed (to \(M\)) to avoid collision with global uses of \(m\).

\subsection*{Corner-exit disjointness scale: separation at the footprint diameter (important for \(p=n/2\))}
The corner-exit/sliver program uses small footprint simplices of diameter \(D_Q\asymp s\asymp \varrho h\) inside a mesh cube of diameter \(h\). The natural “graph thickness” of a sheet on its footprint is \(\asymp \varepsilon D_Q\), not \(\varepsilon h\).
To avoid an implicit contradiction (especially in the borderline schedule \(p=n/2\) where one requires \(\varrho=o(\varepsilon)\)), the manuscript’s disjointness bookkeeping was tightened:
\begin{itemize}
\item Lemma \texttt{lem:sliver-stability} now states disjointness persistence in terms of the \emph{footprint diameter} \(D_i=\mathrm{diam}((P+t_i)\cap Q)\), giving the correct separation scale \(\|t_1-t_2\|\gtrsim \varepsilon\max\{D_1,D_2\}\).
\item Proposition \texttt{prop:finite-template} now phrases the separation hypothesis using \(D_Q:=\max_a \mathrm{diam}((P+t_a)\cap Q)\), which matches the corner-exit regime \(D_Q\asymp s\).
\item The parameter-schedule / weighted-scaling remarks were updated to explicitly treat separation at the footprint scale \(\delta_{\mathrm{sep}}\asymp \varepsilon s\), and the borderline lemma \texttt{lem:borderline-p-half} now explicitly cites the footprint-scale packing interpretation.
\end{itemize}
These are “quantifier/scale” fixes (not changes in mathematical strategy) that remove a potential referee objection about simultaneously having \(\varrho=o(\varepsilon)\) and still manufacturing many disjoint sliver pieces.

\subsection*{Borderline regime $p=n/2$ (middle dimension)}
Several places in the TeX already flagged that the middle-dimensional regime requires extra care. We made one additional clarification:
\begin{itemize}
\item Corollary \texttt{cor:raw-boundary-flat-small} now explicitly notes that in the borderline case \(p=n/2\), the conclusion
\(\mathcal F(\partial T^{\mathrm{raw}})=o(m)\) is to be read under the refined displacement schedule of Lemma \texttt{lem:borderline-p-half}
(e.g.\ \(\varrho=o(\varepsilon)\)).
\end{itemize}

\subsection*{Period locking proof hygiene}
The “period locking” proposition \texttt{prop:cohomology-match} contains an internal Step~5 (construction of the tiny-mass boundary correction) and is immediately followed by a dedicated subsection “Step~5: Boundary correction with vanishing mass” that carries the same construction.
To avoid two near-identical constructions living in the manuscript simultaneously, the internal Step~5 construction was removed from the proposition proof; the proof now points to the following subsection for the existence and estimates for \(U_\epsilon\).

\subsection*{SYR realization proof: varifold tangent-plane bookkeeping made explicitly optional}
In the “SYR Realization” theorem (\texttt{thm:syr-realization}), an intermediate tangent-plane/varifold viewpoint was written out in a way that could be misread as an additional input (or as requiring a specific \emph{oriented} Grassmann-bundle convention, since \(\langle T,\psi\rangle\) depends on current orientation).
Because this tangent-plane bookkeeping is \emph{not needed} for the closure step, the detailed formulas were marked as optional and disabled in the TeX.
The proof’s core remains purely current-theoretic: Federer–Fleming flat compactness, mass lower semicontinuity, and the comass inequality imply the subsequential limit is \(\psi\)-calibrated.

\subsection*{Filling lemma: ensure the tubular projection is actually defined}
In Lemma \texttt{lem:FF-filling-X}, the original Nash-embedding proof implicitly pushed forward a Euclidean filling current \(Q\) by the nearest-point projection \(\pi:U\to X\) from a tubular neighborhood \(U\) of \(\iota(X)\).
This requires \(Q\) to be supported in \(U\), which is not automatic for an arbitrary Euclidean filling.
The proof now explicitly invokes a \emph{relative} filling inequality in the fixed tubular neighborhood \(U\) (a standard GMT fact for integral currents in domains), so that \(Q\subset U\) and \(\pi_\#Q\) is well-defined.

\subsection*{Holomorphic corner-exit footprint geometry: duplicate proofs removed}
The corner-exit geometry proposition \texttt{prop:holomorphic-corner-exit-g1g2} had accumulated multiple back-to-back proof environments (draft variants of the same argument).
These were removed, leaving a single concise proof so that the compiled manuscript presents one unambiguous argument.

\subsection*{Corner-exit direction-net packaging: missing uniform constant added}
In the finite direction-net packaging proposition \texttt{prop:corner-exit-template-net}, the proof introduced per-direction constants
\(\alpha_*(i),\alpha^*(i),A_*(i)\) but only explicitly took a minimum/maximum for \(\alpha_*\) and \(A_*\) before applying the open-template lemma.
This was tightened to also define \(\alpha^*:=\max_i\alpha^*(i)\) (hence a finite \(\Lambda=\alpha^*/\alpha_*\)) before invoking Lemma \texttt{lem:corner-exit-template-open}.
This is a small but important bookkeeping correction for referee readability.

\subsection*{Template packing language: “arbitrarily long \(\delta\)-separated lists”}
Several template statements used shorthand like “an ordered \(\delta\)-separated list \((t_a)_{a\ge 1}\)” inside a bounded parameter box (or “arbitrarily long \(\delta\)-separated list for any \(\delta>0\)”). Literally, a bounded set cannot contain an infinite \(\delta\)-separated family for fixed \(\delta>0\).
These statements were corrected to the accurate formulation: for each fixed \(\delta>0\) one gets a \emph{finite} \(\delta\)-separated list of length \(N(\delta)\), and \(N(\delta)\to\infty\) as \(\delta\downarrow 0\) (with the footprint scale fixed).
This does not change any downstream argument (where \(\delta_{\perp}\) is chosen as part of the mesh schedule), but removes a potential referee objection.

\subsection*{Grid master-template quantifiers (finite prefixes on a finite lattice set)}
In Proposition \texttt{prop:integer-transport}, the master transverse atoms were written as an “ordered template \((y_a)_{a\ge 1}\subset B_{C_0\varrho h}(0)\cap \delta_\perp\mathbb Z^{2p}\)”. For fixed \(h\) and fixed \(\delta_\perp>0\), the grid intersection is a \emph{finite} set, so the intended meaning is: pick \emph{as many grid atoms as needed} for the finite prefixes used at that mesh scale.
The TeX now states this correctly by choosing a finite ordered list \((y_a)_{a=1}^{N_*}\) of grid atoms and requiring the working prefix length \(N_F\le N_*\).

\subsection*{Convention note: “algebraic class”}
A short remark was added near the main theorem stating the manuscript’s convention: an \emph{algebraic class} in \(H^{2p}(X,\mathbb Q)\) means a class in the \(\mathbb Q\)-span of cycle classes (equivalently, the image of the cycle class map with \(\mathbb Q\)-coefficients), hence a \(\mathbb Q\)-vector subspace.

\section*{What remains to be checked}
These edits remove two hard referee blockers but do not, by themselves, certify the full proof. The remaining highest-risk verification areas include:
\begin{itemize}
\item global coherence across all labels and meshes (integer data satisfying face constraints and period constraints simultaneously);
\item transport \(\Rightarrow\) flat-norm gluing estimates and their parameter dependence;
\item exact-class enforcement (\texttt{prop:cohomology-match}) under the full parameter schedule;
\item ensuring the schedule can satisfy all asymptotic requirements simultaneously (including the new explicit footprint scale \(s_j\ll h_j\)).
\end{itemize}

\end{document}


