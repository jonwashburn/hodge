\documentclass[11pt]{article}

\usepackage[utf8]{inputenc}
\usepackage[T1]{fontenc}
\usepackage{amsmath,amssymb,amsthm}
\usepackage{geometry}
\geometry{margin=1in}

\title{CPM--Hodge Bridge and Calibrated Microstructure:\\
From Global Cones to Flat-Space Obstructions}
\author{Jonathan Washburn\\
Recognition Science, Recognition Physics Institute\\
Austin, Texas, USA\\
\texttt{jon@recognitionphysics.org}}
\date{}

\newtheorem{theorem}{Theorem}[section]
\newtheorem{conjecture}[theorem]{Conjecture}
\newtheorem{assumption}[theorem]{Assumption}
\newtheorem{lemma}[theorem]{Lemma}
\newtheorem{fact}[theorem]{Fact}
\theoremstyle{definition}
\newtheorem{definition}[theorem]{Definition}

\newcommand{\R}{\mathbb{R}}
\newcommand{\C}{\mathbb{C}}
\newcommand{\T}{\mathbb{T}}
\newcommand{\Z}{\mathbb{Z}}
\newcommand{\Gr}{\mathrm{Gr}}
\newcommand{\Mass}{\mathrm{Mass}}
\newcommand{\Defcone}{\mathrm{Def}_{\mathrm{cone}}}

\begin{document}
\maketitle

\begin{abstract}
We formalize a CPM--Hodge bridging conjecture that connects cone-valued harmonic representatives to calibrated currents, identify analytic gaps (regularization and microstructure), and then pass to a flat Kähler model to probe the microstructure problem locally. In the simplest nontrivial case $\C^3\cong\R^6$ and its torus quotient, we exhibit a sharp obstruction to finite-direction calibrated laminates realizing nonconstant cone-valued forms, showing that any such approach must use either infinitely many calibrated directions or genuinely curved complex surfaces. We also formulate a precise stationarity (PDE-type) condition at the level of Young measures which any realizable microstructure must satisfy.
\end{abstract}

\section{Global CPM--Hodge Setting}

Let $X$ be a smooth projective complex variety of dimension $n$, equipped with a Kähler form $\omega$. Fix $1\le p\le n$. The Kähler calibration is
\[
  \varphi := \frac{\omega^p}{p!},
\]
a closed real $(p,p)$--form on $X$.

At each $x\in X$, the space of real $(p,p)$--forms
\[
  V_x := \Lambda^{p,p}(T_x^*X)_\R
\]
carries a natural inner product induced by the metric. The pointwise \emph{calibration cone} is defined as the closed convex cone
\[
  K_p(x) := \left\{ \beta\in V_x \,\middle|\,
    \beta = \sum_{i=1}^k a_i\,\xi_{P_i},\ a_i\ge 0,\ 
    P_i\subset T_xX \text{ a complex $p$--plane calibrated by }\varphi_x
  \right\},
\]
i.e.\ the convex cone generated by simple $(p,p)$--forms associated to $\varphi_x$--calibrated $p$--planes.

\begin{definition}[Cone-defect for forms]
For a smooth real $(p,p)$--form $\alpha$ on $X$, define the cone-distance defect
\[
  \Defcone(\alpha)
  := \int_X \operatorname{dist}(\alpha(x),K_p(x))^2\,d\mathrm{vol}_\omega(x),
\]
where the distance is taken with respect to the fixed inner product on $V_x$.
\end{definition}

Let
\[
  [\gamma]\in H^{2p}(X,\Q)\cap H^{p,p}(X,\C)
\]
be a rational Hodge class, and denote by $\gamma_{\mathrm{harm}}$ its $\omega$--harmonic representative. The $L^2$ energy of a form is
\[
  E(\alpha) := \int_X \|\alpha(x)\|^2\, d\mathrm{vol}_\omega(x) = \|\alpha\|_{L^2}^2.
\]

\section{The CPM--Hodge Bridging Conjecture}

We now state a CPM--Hodge bridge that formalizes ``existence = zero defect under a coercive projection'' in this geometric context.

\begin{conjecture}[CPM--Hodge bridge]\label{conj:CPM-Hodge}
Let $(X,\omega)$ be a smooth projective Kähler manifold of complex dimension $n$, and fix $1\le p\le n$. Let $\varphi=\omega^p/p!$ and the pointwise cones $K_p(x)$ be as above.

Assume that $[\gamma]\in H^{2p}(X,\Q)\cap H^{p,p}(X,\C)$ has harmonic representative $\gamma_{\mathrm{harm}}$ (for $\omega$) satisfying
\[
  \gamma_{\mathrm{harm}}(x)\in K_p(x)\quad\text{for all }x\in X.
\]

Then there exists a positive integral $(p,p)$--current $T$ calibrated by $\varphi$ in the sense of Harvey--Lawson, and some integer $m\ge 1$, such that
\[
  [T] = m[\gamma] \in H^{2p}(X,\Z)\cap H^{p,p}(X,\C).
\]
In particular, $T$ is a finite integral combination
\[
  T = \sum_i m_i [Z_i],\quad m_i\in\N,
\]
of integration currents over irreducible complex subvarieties $Z_i\subset X$ of codimension $p$, and
\[
  [\gamma] = \sum_i \frac{m_i}{m}\,[Z_i] \in H^{2p}(X,\Q).
\]
Thus $[\gamma]$ is a rational linear combination of algebraic cycles of codimension $p$.
\end{conjecture}

In CPM language, the \emph{structured set} is the cone of calibrated forms/currents; the \emph{defect} is the cone-distance for forms and the difference $\Mass(T)-\int T\wedge\varphi$ for currents; the \emph{energy} is Dirichlet/$L^2$ for forms and mass for currents. Conjecture~\ref{conj:CPM-Hodge} states that when the unique energy minimizer in a rational class lies in the structured set (zero defect), the class is realized by an actual calibrated current.

\section{Forms-Level Coercivity and Gap A/B}

\subsection{Forms-level coercivity}

Under the hypothesis $\gamma_{\mathrm{harm}}(x)\in K_p(x)$, the forms-level CPM coercivity inequality can be proven.

\begin{lemma}[Forms-level coercivity]\label{lem:forms-coercivity}
Let $[\gamma]\in H^{2p}(X,\Q)\cap H^{p,p}(X,\C)$ have harmonic representative $\gamma_{\mathrm{harm}}$ with $\gamma_{\mathrm{harm}}(x)\in K_p(x)$ for all $x\in X$. Then there exists $c>0$ such that for every smooth closed $(p,p)$--form $\alpha$ with $[\alpha]=[\gamma]$,
\[
  E(\alpha) - E(\gamma_{\mathrm{harm}})
  \;\ge\; c\,\Defcone(\alpha).
\]
\end{lemma}

\begin{proof}
By Hodge theory, writing $\alpha = \gamma_{\mathrm{harm}} + d\beta + d^\ast\eta$ and restricting to closed $\alpha$ gives
\[
  E(\alpha) - E(\gamma_{\mathrm{harm}}) = \|\alpha-\gamma_{\mathrm{harm}}\|_{L^2}^2.
\]

Fix $x\in X$ and let $u_0 = \gamma_{\mathrm{harm}}(x)\in K_p(x)$, which lies in the interior of the closed convex cone $K_p(x)$ by strong positivity. The metric projection $P_{K_p(x)}$ onto $K_p(x)$ is $1$--Lipschitz, and in a neighborhood of $u_0$ there exists a constant $C(x)$ such that
\[
  \operatorname{dist}(u,K_p(x)) \le C(x)\,\|u-u_0\|
\]
for all $u$ in that neighborhood. By compactness of $X$ and continuity of $\gamma_{\mathrm{harm}}$, there is a global constant $C>0$ such that
\[
  \operatorname{dist}(\alpha(x),K_p(x))
  \le C\,\|\alpha(x)-\gamma_{\mathrm{harm}}(x)\|
\]
for all $x\in X$ and all $\alpha$ in a bounded $C^0$--neighborhood of $\gamma_{\mathrm{harm}}$. Squaring and integrating yields
\[
  \Defcone(\alpha)
  = \int_X \operatorname{dist}(\alpha(x),K_p(x))^2\,d\mathrm{vol}_\omega
  \le C^2 \|\alpha-\gamma_{\mathrm{harm}}\|_{L^2}^2.
\]
Thus
\[
  E(\alpha)-E(\gamma_{\mathrm{harm}})
  = \|\alpha-\gamma_{\mathrm{harm}}\|_{L^2}^2
  \;\ge\; \frac{1}{C^2}\,\Defcone(\alpha),
\]
and the lemma holds with $c = 1/C^2$.
\end{proof}\fi

\subsection{Gap A: cone-compatible regularization}

To bridge from forms to currents, one would like to approximate a positive closed $(p,p)$--current by smooth forms with small cone-defect.

\begin{assumption}[Gap A -- cone-compatible regularization]\label{ass:gapA}
Let $T$ be a positive closed $(p,p)$--current on $X$ with $[T]=[\gamma]$. There exists a sequence of smooth closed $(p,p)$--forms $\alpha_k$ with $[\alpha_k]=[\gamma]$ such that
\begin{itemize}
  \item $\alpha_k \to T$ weakly as currents,
  \item $\Defcone(\alpha_k)\to 0$ as $k\to\infty$.
\end{itemize}
\end{assumption}

Demailly's regularization guarantees approximants $\alpha_k$ with $\alpha_k\ge -\varepsilon_k\omega^p$ and $\varepsilon_k\to 0$, but does not directly control $\Defcone(\alpha_k)$.

Under Assumption~\ref{ass:gapA}, Lemma~\ref{lem:forms-coercivity} implies $\alpha_k\to\gamma_{\mathrm{harm}}$ in $L^2$, so any ``positive realization'' collapses to the harmonic cone-valued form itself.

\subsection{Gap B: microstructure and calibration}

Passing from cone-valued harmonic forms to calibrated currents is the microstructure problem. In homology, let $A = \mathrm{PD}(m[\gamma])\in H_{2n-2p}(X,\Z)$ for some $m\ge 1$, and let $\mathcal{C}_A$ be the set of integral currents in class $A$. Calibration theory gives
\[
  \int_X T\wedge\varphi \le \Mass(T)
\]
for any $T\in\mathcal{C}_A$, with equality if and only if $T$ is calibrated. Let
\[
  c_0 := \int_X T\wedge\varphi
\]
(be independent of $T$ in the class), and let $T_{\min}\in\mathcal{C}_A$ be a mass minimizer. Then $\Mass(T_{\min})\ge c_0$, and achieving equality would give a calibrated current in the class.

A natural microstructure conjecture formalizing Gap B is:

\begin{conjecture}[Microstructure realization (abstract form)]\label{conj:micro-abstract}
Let $(X,\omega)$, $\varphi$, $A$ and $[\gamma]$ be as above. Suppose there exists a smooth closed $(p,p)$--form $\beta$ with $[\beta]=[\gamma]$ and $\beta(x)\in K_p(x)$ for all $x\in X$. Then for every $\varepsilon>0$ there exists an integral current $T_\varepsilon\in\mathcal{C}_A$ such that
\[
  \Mass(T_\varepsilon) \le c_0 + \varepsilon.
\]
Equivalently, one can find integral cycles whose masses approach the calibration lower bound.
\end{conjecture}

If Conjectures~\ref{conj:CPM-Hodge} and \ref{conj:micro-abstract} (plus Assumption~\ref{ass:gapA}) hold, one recovers a calibrated current in $A$ and hence the algebraicity of $[\gamma]$.

\begin{lemma}[Flat-norm boundary control suffices for gluing]\label{lem:hodge-blocker-flat-glue}
Let $X$ be a compact oriented Riemannian manifold and let $k\in\{1,\dots,\dim X-1\}$.
Let $T^{\mathrm{raw}}$ be an integral $k$--current on $X$ (not necessarily closed) and set $S:=\partial T^{\mathrm{raw}}$.
Assume $\mathcal F(S)\le \varepsilon$ for some $\varepsilon>0$, where $\mathcal F$ is the flat norm.
Then there exists an integral $k$--current $R_{\mathrm{glue}}$ with $\partial R_{\mathrm{glue}}=-S$ and
\[
\Mass(R_{\mathrm{glue}})\le C\bigl(\varepsilon+\varepsilon^{\frac{k}{k-1}}\bigr),
\]
where $C=C(X,k)$ depends only on the geometry of $X$ and the isoperimetric constant.
In particular, $T^{\mathrm{raw}}+R_{\mathrm{glue}}$ is a closed integral $k$--current and
\[
\Mass(T^{\mathrm{raw}}+R_{\mathrm{glue}})\le \Mass(T^{\mathrm{raw}})+C\bigl(\varepsilon+\varepsilon^{\frac{k}{k-1}}\bigr).
\]
\end{lemma}

\begin{proof}[Sketch]
Write $S=R+\partial Q$ with $\Mass(R)+\Mass(Q)\le 2\mathcal F(S)\le 2\varepsilon$.
Since $S$ is a boundary, $R=S-\partial Q$ is also a boundary, hence null-homologous.
By Federer--Fleming isoperimetry, $R=\partial Q_R$ for some integral $k$--current $Q_R$ with
$\Mass(Q_R)\le C\,\Mass(R)^{k/(k-1)}$.
Then $R_{\mathrm{glue}}:=-(Q+Q_R)$ satisfies $\partial R_{\mathrm{glue}}=-S$ and the stated mass bound.
\end{proof}

\begin{lemma}[Transport control of face mismatch in a flat chart]\label{lem:hodge-blocker-w1}
In the flat model $\R^{k+q}=\R^k_x\times\R^q_y$, consider two adjacent unit cubes $Q^\pm$ sharing a codimension-$1$ interface face $F$.
Fix the oriented $k$-plane $P=\R^k\times\{0\}$ and build two sheet-stacks on $Q^\pm$ from parallel translates of $P$ parameterized by discrete measures
$\mu=\sum_a m_a\delta_{y_a}$ and $\mu'=\sum_b m'_b\delta_{y'_b}$ on $[0,1]^q$ (integer weights).
Let $B_F$ be the induced mismatch $(k-1)$-current on $F$ (difference of the two face-slice boundaries).
Then for every smooth $(k-1)$-form $\eta$ with $\|\eta\|_{\mathrm{comass}},\|d\eta\|_{\mathrm{comass}}\le 1$ one has
\[
|B_F(\eta)|\ \le\ C\,W_1(\mu,\mu'),
\]
and hence $\mathcal F(B_F)\le C\,W_1(\mu,\mu')$, where $C=C(k,q)$ and $W_1$ is the $1$-Wasserstein distance.
\end{lemma}

\begin{proof}[Sketch]
For each transverse parameter $y$, the face slice is a translate $\Sigma_y\subset F$.
The map $y\mapsto \Sigma_y(\eta)$ is Lipschitz with constant $\lesssim \|d\eta\|_{\mathrm{comass}}$ by Stokes on the cylinder between $\Sigma_y$ and $\Sigma_{y'}$.
Then $B_F(\eta)$ is the difference of $\int \Sigma_y(\eta)\,d\mu$ and $\int \Sigma_y(\eta)\,d\mu'$, and Kantorovich--Rubinstein duality yields the bound by $W_1(\mu,\mu')$.
\end{proof}

\begin{lemma}[Stability of $W_1$ under small linear perturbations]\label{lem:w1-linear-stability}
Let $\mu$ be a finite Borel measure on $\R^q$ with finite first moment $\int_{\R^q}\|y\|\,d\mu(y)<\infty$.
Let $L,L':\R^q\to\R^q$ be linear maps. Then
\[
W_1(L_\#\mu,\,L'_\#\mu)\ \le\ \|L-L'\|_{\mathrm{op}}\int_{\R^q}\|y\|\,d\mu(y).
\]
\end{lemma}

\begin{proof}
Couple $L_\#\mu$ and $L'_\#\mu$ by pushing forward $\mu$ under the map $y\mapsto (Ly,L'y)$.
The transport cost is
\[
\int \|Ly-L'y\|\,d\mu(y)\le \|L-L'\|_{\mathrm{op}}\int \|y\|\,d\mu(y),
\]
and taking the infimum over couplings yields the claim.
\end{proof}

\begin{corollary}[Small-angle change $\Rightarrow$ small face mismatch (model estimate)]\label{cor:angle-to-face}
In the flat chart setting of Lemma~\ref{lem:hodge-blocker-w1}, suppose the two stacks on $Q^\pm$ are parameterized by the
\emph{same} discrete transverse measure $\mu$ in some reference coordinates, but the geometric identification of transverse parameters
with the face depends on a linear map $L$ on one side and $L'$ on the other (coming from two nearby plane directions).
Then
\[
\mathcal F(B_F)\ \le\ C\,W_1(L_\#\mu,\,L'_\#\mu)
\ \le\ C\,\|L-L'\|_{\mathrm{op}}\int \|y\|\,d\mu(y).
\]
In particular, if $\|L-L'\|_{\mathrm{op}}=O(\angle(P,P'))$ and $\mu$ is supported in a ball of radius $O(h)$, then
\[
\mathcal F(B_F)\ \lesssim\ \angle(P,P')\cdot h\cdot \mu(\Omega),
\]
which is the expected scaling used in transport-based gluing bounds.
\end{corollary}

\begin{lemma}[Face-slice functionals vary Lipschitzly with plane angle (flat cube)]\label{lem:slice-angle-lip}
Let $Q\subset\R^{k+q}$ be a cube of side length $h$ and let $F\subset\partial Q$ be a codimension-$1$ face.
Let $P,P'\in \Gr_k(\R^{k+q})$ be two oriented $k$-planes with $\angle(P,P')\le \alpha\ll 1$.
For $t$ in a bounded set of translations, consider the slice currents
\[
\Sigma_F^{P}(t):=\partial\big([P+t]\llcorner Q\big)\llcorner F,
\qquad
\Sigma_F^{P'}(t):=\partial\big([P'+t]\llcorner Q\big)\llcorner F.
\]
Then for every smooth $(k-1)$-form $\eta$ with $\|\eta\|_{\mathrm{comass}}\le 1$ and $\|d\eta\|_{\mathrm{comass}}\le 1$,
\[
\bigl|\Sigma_F^{P}(t)(\eta)-\Sigma_F^{P'}(t)(\eta)\bigr|
\ \le\ C(k,q)\,\alpha\,h^{k-1}.
\]
\end{lemma}

\begin{proof}[Sketch]
On $Q$, both planes can be parameterized as graphs over a fixed reference $k$-plane (after a rigid motion) with $C^1$ norm $O(\alpha)$.
The slice on $F$ is obtained by restricting these graph parameterizations to the corresponding boundary hyperplane.
The difference of integrals against $\eta$ is controlled by the $C^0$ variation of the induced tangent $(k-1)$-vectors
times the $(k-1)$-dimensional area of the slice, which is $O(h^{k-1})$.
\end{proof}

\begin{lemma}[Intersection mass of a translated plane is continuous]\label{lem:plane-section-continuity}
Let $P\subset\R^{d}$ be an oriented $k$-plane and let $Q\subset\R^d$ be a compact set (e.g.\ a cube or ball).
Define
\[
f(t):=\mathcal H^{k}\bigl((P+t)\cap Q\bigr)
\qquad (t\in\R^d).
\]
Then $f$ is continuous in $t$. In particular, the set of attainable masses $\{f(t):t\in\R^d\}$ is an interval $[0,f_{\max}]$,
where $f_{\max}=\sup_t f(t)$ and $f(t)=0$ whenever $(P+t)\cap Q=\emptyset$.
\end{lemma}

\begin{proof}
Write
\[
f(t)=\int_{P}\mathbf 1_Q(x+t)\,d\mathcal H^{k}(x).
\]
If $t_n\to t$, then $\mathbf 1_Q(x+t_n)\to \mathbf 1_Q(x+t)$ pointwise for $\mathcal H^{k}$-a.e.\ $x\in P$ (the exceptional set
is contained in $P\cap(\partial Q-t)$, which is $\mathcal H^{k}$-null for a.e.\ translate and in particular harmless for continuity along a fixed sequence after a standard approximation argument).
Moreover $\mathbf 1_Q\le 1$ is integrably dominated on $P\cap(Q-t_n)$, so dominated convergence yields $f(t_n)\to f(t)$.
The interval statement follows because $f$ is continuous and vanishes outside a bounded region of translations.
\end{proof}

\begin{remark}[Why this matters for “sliver microstructure’’]
Lemma~\ref{lem:plane-section-continuity} shows that even in the flat model, one can tune the mass of a single plane-slice piece
continuously from $0$ up to its maximal value by translating the plane relative to the cell.
Thus, in principle, one can split a prescribed total mass $M_Q$ into many very small “sliver’’ pieces (many translations with tiny intersections)
while keeping the total mass fixed. This is the geometric degree of freedom that would be needed to keep \emph{many} sheets per cube/face
even when $m$ is fixed, potentially resolving the dense-vs-gluing tension highlighted in Remark~\ref{rem:param-tension} of the manuscript.
\end{remark}

\begin{lemma}[Toy scaling for sliver pieces in a ball]\label{lem:sliver-ball-scaling}
Let $B_h\subset\R^d$ be the Euclidean ball of radius $h$ centered at $0$ and let $P$ be a $k$-plane.
For $s\in[0,h)$, let $P_s$ be an affine translate of $P$ at distance $s$ from the origin (orthogonal shift).
Then
\[
\mathcal H^{k}(P_s\cap B_h)=\omega_k\,(h^2-s^2)^{k/2},
\]
so in particular as $s\uparrow h$ (write $h-s=\delta$) one has
\[
\mathcal H^{k}(P_s\cap B_h)\asymp h^{k/2}\,\delta^{k/2}.
\]
\end{lemma}

\begin{proof}
By symmetry, $P_s\cap B_h$ is a $k$-ball of radius $\sqrt{h^2-s^2}$ inside $P_s$, hence has volume $\omega_k (h^2-s^2)^{k/2}$.
The asymptotic as $s\uparrow h$ follows from $h^2-s^2=(h-s)(h+s)\sim 2h\,\delta$.
\end{proof}

\begin{lemma}[Boundary size vs.\ volume for a ball slice]\label{lem:sliver-ball-boundary}
In the setting of Lemma~\ref{lem:sliver-ball-scaling}, the $(k-1)$--dimensional boundary size of the slice is
\[
\mathcal H^{k-1}\bigl(P_s\cap \partial B_h\bigr)=\omega_{k-1}\,(h^2-s^2)^{(k-1)/2}.
\]
Equivalently, writing $v(s):=\mathcal H^{k}(P_s\cap B_h)$, one has the exact relation
\[
\mathcal H^{k-1}\bigl(P_s\cap \partial B_h\bigr)=
\omega_{k-1}\,\omega_k^{-\frac{k-1}{k}}\;\bigl(v(s)\bigr)^{\frac{k-1}{k}}.
\]
\end{lemma}

\begin{proof}
The intersection $P_s\cap\partial B_h$ is a $(k-1)$-sphere of radius $\sqrt{h^2-s^2}$ inside $P_s$, hence has area $\omega_{k-1}(h^2-s^2)^{(k-1)/2}$.
The power-law relation follows by eliminating $(h^2-s^2)$ using Lemma~\ref{lem:sliver-ball-scaling}.
\end{proof}

\begin{remark}[Why smooth convexity matters]
Lemma~\ref{lem:sliver-ball-boundary} is the clean mechanism that makes “small mass $\Rightarrow$ small boundary slice’’ true for a ball.
For sharp cubical cells, this implication can fail: one can have plane sections with very small $k$-volume but boundary on a face still $\asymp h^{k-1}$
(thin-long slices).  Any rigorous “sliver’’ bookkeeping therefore needs either smooth convex cells (rounded cubes) or additional geometric restrictions.
\end{remark}

\begin{lemma}[Flat-norm stability under translation]\label{lem:flat-translate}
Let $S$ be an integral $(\ell)$-cycle in $\R^d$ (so $\partial S=0$) with finite mass.
For any translation vector $v\in\R^d$, let $\tau_v(x):=x+v$ and write $(\tau_v)_\# S$ for the pushforward.
Then
\[
\mathcal F\!\bigl((\tau_v)_\# S - S\bigr)\ \le\ \|v\|\,\Mass(S).
\]
\end{lemma}

\begin{proof}[Sketch]
Let $H:[0,1]\times\R^d\to\R^d$ be the homotopy $H(t,x)=x+t v$.
The pushforward $Q:=H_\#([0,1]\times S)$ is an integral $(\ell+1)$-current with
$\partial Q = (\tau_v)_\# S - S$.
Moreover $\Mass(Q)\le \|v\|\Mass(S)$ by the area formula for products.
Taking $R=0$ in the flat norm definition gives the bound.
\end{proof}

\begin{remark}[Application to sliver boundary mismatch]
In a smooth convex cell model, each sliver piece has a boundary slice on the cell boundary (or on a smooth interface hypersurface).
If adjacent cells induce a small displacement $\|v\|=O(h^2)$ of these boundary slices (e.g.\ due to $O(h)$ face-map variation applied to $O(h)$ transverse parameters),
Lemma~\ref{lem:flat-translate} bounds the per-piece flat mismatch by $O(h^2)\times$ (boundary mass).
Combined with Lemma~\ref{lem:sliver-ball-boundary} (ball model) this yields the heuristic $N^{1/k}$ boundary-growth factor recorded in
Remark~\ref{rem:sliver-bergman-scaling} of the manuscript.
\end{remark}

\begin{proposition}[Uniformly convex cells force boundary shrinkage for plane slices]\label{conj:uniformly-convex-slice-boundary}
Let $Q\subset\R^d$ be a bounded $C^2$ \emph{uniformly convex} domain of diameter $\asymp h$.
Assume the principal curvatures of $\partial Q$ satisfy
\[
\frac{c}{h}\ \le\ \kappa_i\ \le\ \frac{C}{h}
\qquad\text{everywhere on }\partial Q,
\]
for fixed constants $0<c\le C$.
Fix $1\le k<d$ and a $k$-plane $P$.
Then there exists $C_*=C_*(d,k,c,C)$ such that for every translate $P+t$ with nonempty intersection,
writing
\[
v(t):=\mathcal H^{k}\bigl((P+t)\cap Q\bigr),
\qquad
a(t):=\mathcal H^{k-1}\bigl((P+t)\cap \partial Q\bigr),
\]
one has the upper bound
\[
a(t)\ \le\ C_*\,\bigl(v(t)\bigr)^{\frac{k-1}{k}}.
\]
\end{proposition}

\begin{proof}
Rescale so that $h\asymp 1$.  Set $K_t:=(P+t)\cap Q\subset P+t\cong\R^k$.
Then $K_t$ is a convex body with $v(t)=\mathcal H^{k}(K_t)$ and $a(t)=\mathcal H^{k-1}(\partial K_t)$.

\smallskip\noindent
\textbf{Step 1 (non-tiny sections).}
Fix any $v_0>0$.  Since $K_t\subset Q$ and $\mathrm{diam}(Q)\asymp 1$, each $K_t$ is contained in a $k$--ball of radius $O(1)$,
hence $\partial K_t$ has uniformly bounded $(k-1)$--measure:
\[
a(t)\ \le\ A_0(d,k)\qquad\text{for all }t.
\]
Therefore, on $\{v(t)\ge v_0\}$ one has $a(t)\le A_0(d,k)\,v_0^{-(k-1)/k}\,v(t)^{(k-1)/k}$.
It remains to treat the regime $v(t)\le v_0$, where $v_0$ is chosen small.

\smallskip\noindent
\textbf{Step 2 (uniform rolling ball condition).}
The curvature pinching implies an interior/exterior ball condition: there exist radii
$r_{\mathrm{in}},r_{\mathrm{out}}\asymp 1$ (depending only on $c,C$) such that for each $x\in\partial Q$ with outward unit normal $n(x)$,
\[
B(x-r_{\mathrm{in}}n(x),r_{\mathrm{in}})\ \subset\ Q\ \subset\ B(x-r_{\mathrm{out}}n(x),r_{\mathrm{out}}),
\]
and both balls are tangent to $\partial Q$ at $x$.

\smallskip\noindent
\textbf{Step 3 (ball sandwich for small sections).}
Let $\pi:\R^d\to P^\perp$ be orthogonal projection and set $D:=\pi(Q)\subset P^\perp$.
If $v(t)\le v_0$ with $v_0$ small, then $t$ lies within a small distance $s$ of $\partial D$.
Choose a nearest point $t_0\in\partial D$ and let $u\in P^\perp$ be the unit normal of some supporting hyperplane of $D$ at $t_0$.
Write $t=t_0-s u$ with $s=\|t-t_0\|$.
Uniform convexity implies the supporting point of $Q$ with outward normal $u$ is unique; denote it by $x_0\in\partial Q$.
Then $\pi(x_0)=t_0$ and $n(x_0)=u$.

Intersect the tangent balls at $x_0$ (Step 2) with the affine plane $P+t$.
Since $u\perp P$, these intersections are $k$-balls of radii
\[
\rho_{\mathrm{in}}(s)=\sqrt{2 r_{\mathrm{in}}s - s^2},
\qquad
\rho_{\mathrm{out}}(s)=\sqrt{2 r_{\mathrm{out}}s - s^2}.
\]
The inclusions of balls imply
\[
\omega_k\,\rho_{\mathrm{in}}(s)^k\ \le\ v(t)\ \le\ \omega_k\,\rho_{\mathrm{out}}(s)^k
\qquad\text{and}\qquad
a(t)\ \le\ \omega_{k-1}\,\rho_{\mathrm{out}}(s)^{k-1}.
\]
For $s$ small, $\rho_{\mathrm{in}}(s)\gtrsim \sqrt{s}$ and $\rho_{\mathrm{out}}(s)\lesssim \sqrt{s}$, hence
$v(t)\gtrsim s^{k/2}$ and $a(t)\lesssim s^{(k-1)/2}$.
Thus $s\lesssim v(t)^{2/k}$ and
\[
a(t)\ \lesssim\ s^{(k-1)/2}\ \lesssim\ v(t)^{(k-1)/k}.
\]
Undoing the rescaling gives the claim, with a constant depending only on $(d,k,c,C)$.
\end{proof}
\iffalse
\begin{proof}[Proof sketch]
By scaling invariance, we may rescale so that $\mathrm{diam}(Q)\asymp 1$; the curvature bounds then become
uniform positive constants.
For each $t$ with $v(t)>0$, the section $K_t:=(P+t)\cap Q$ is a compact convex body in the affine $k$--plane $P+t$ and
$\partial K_t=(P+t)\cap\partial Q$ is a smooth convex hypersurface in $P+t$.

\smallskip\noindent
\textbf{Step 1 (compactness away from degeneration).}
Fix any threshold $v_0>0$. Since $\partial Q$ is compact, $a(t)$ is uniformly bounded above on $\{t:\ v(t)\ge v_0\}$,
while $v(t)^{(k-1)/k}\ge v_0^{(k-1)/k}$ there. Thus the desired inequality holds on $\{v\ge v_0\}$ with a constant depending on $v_0$ and $Q$.

\smallskip\noindent
\textbf{Step 2 (blow-up near $v(t)\to 0$).}
It remains to control the ratio as $v(t)\downarrow 0$, i.e.\ as the translate approaches the boundary of the projection set
$\pi_{P^\perp}(Q)$.
Pick a sequence $t_\nu$ with $v_\nu:=v(t_\nu)\to 0$ and choose a boundary point
$x_\nu\in (P+t_\nu)\cap\partial Q$.
Set the rescaling radius $r_\nu:=v_\nu^{1/k}$ and consider the rescaled domains and sections
\[
Q_\nu := r_\nu^{-1}(Q-x_\nu),\qquad
K_\nu := r_\nu^{-1}\bigl((P+t_\nu-x_\nu)\cap Q\bigr)\subset r_\nu^{-1}(P+t_\nu-x_\nu).
\]
By construction, $\mathcal H^k(K_\nu)=1$. Uniform convexity and smoothness of $\partial Q$ imply that, after passing to a subsequence,
$Q_\nu$ converges in $C^2_{\mathrm{loc}}$ near $0$ to the epigraph of a strictly convex quadratic form (a paraboloid),
and $K_\nu$ converges to a $k$--dimensional ellipsoid given by a quadratic sublevel set in the limit plane.
For such a quadratic limit, the boundary measure satisfies
$\mathcal H^{k-1}(\partial K_\infty)\le C_*$ with $C_*$ depending only on the Hessian bounds (hence only on $Q$).
By $C^2$ convergence, $\sup_\nu \mathcal H^{k-1}(\partial K_\nu)<\infty$.
Undoing the scaling gives
\[
a(t_\nu)=\mathcal H^{k-1}\bigl(\partial K_{t_\nu}\bigr)
= r_\nu^{k-1}\,\mathcal H^{k-1}(\partial K_\nu)
\ \lesssim\ r_\nu^{k-1}
= v_\nu^{\frac{k-1}{k}},
\]
which yields the desired bound as $v_\nu\to 0$.

\smallskip\noindent
Combining Steps 1--2 gives a global constant $C=C(d,k,Q)$.
\end{proof}

\begin{remark}
Lemma~\ref{lem:sliver-ball-boundary} proves Proposition~\ref{conj:uniformly-convex-slice-boundary} for $Q=B_h$ with an explicit sharp constant.
For rounded-cube cells with curvature pinched at scale $h$ (rounding radius $\asymp h$), Proposition~\ref{conj:uniformly-convex-slice-boundary}
provides exactly the “small mass $\Rightarrow$ small boundary slices’’ bookkeeping needed to make the sliver route compatible with flat-norm gluing.
\end{remark}

\begin{remark}[Heuristic implication]
Lemma~\ref{lem:sliver-ball-scaling} shows that by placing many parallel planes very near the boundary of a ball cell,
one can create many disjoint “sliver’’ pieces whose individual $k$-masses are extremely small (polynomial in the boundary gap $\delta$),
while keeping the total mass fixed.  This is the kind of geometric freedom one would need to keep the number of sheet pieces per cell/face
large even at small mesh in the fixed-$m$ regime.
\end{remark}

\begin{lemma}[Equal-mass sliver sampling on a ball]\label{lem:sliver-sampling}
Let $B_h\subset\R^{k+2p}$ be the Euclidean ball of radius $h$ centered at $0$, and let $P\subset\R^{k+2p}$ be an oriented $k$-plane.
Fix $M>0$ and $N\in\N$ with $0< M/N < \omega_k h^k$ (so the desired per-piece mass is below the maximal section mass).
Then there exists a radius $r\in(0,h)$ such that for every translation $t\in P^\perp$ with $\|t\|=r$ one has
\[
\Mass([P+t]\llcorner B_h)=\mathcal H^k\bigl((P+t)\cap B_h\bigr)=\frac{M}{N}.
\]
Moreover, one can choose $t_1,\dots,t_N\in P^\perp$ with $\|t_a\|=r$ such that
\begin{enumerate}
\item[\textnormal{(i)}] the plane pieces $(P+t_a)\cap B_h$ are pairwise disjoint (hence define disjoint calibrated “sliver’’ pieces);
\item[\textnormal{(ii)}] letting $\mu_N:=\sum_{a=1}^N \frac{M}{N}\,\delta_{t_a}$ and letting $\sigma_r$ denote the normalized surface measure on the sphere $S^{2p-1}(r)\subset P^\perp$,
\[
W_1\!\left(\mu_N,\ M\,\sigma_r\right)\ \le\ C(p)\,M\,r\,N^{-\frac{1}{2p-1}}.
\]
\end{enumerate}
\end{lemma}

\begin{proof}[Sketch]
By Lemma~\ref{lem:sliver-ball-scaling}, the section mass depends only on $s=\|t\|$ and equals
$\omega_k(h^2-s^2)^{k/2}$, which varies continuously from $\omega_k h^k$ at $s=0$ to $0$ at $s=h$.
Thus there exists $r\in(0,h)$ with $\omega_k(h^2-r^2)^{k/2}=M/N$, proving the first claim.

To choose $t_a$, partition the sphere $S^{2p-1}(r)$ into $N$ measurable pieces of diameter $\le C(p)\,r\,N^{-1/(2p-1)}$
(e.g.\ using a maximal separated set and Voronoi cells), and pick one point $t_a$ in each piece.
Parallel affine planes are disjoint in $\R^{k+2p}$, so (i) holds.
For (ii), couple $M\,\sigma_r$ to $\mu_N$ by transporting the mass of each cell to its chosen representative point; the transport cost
is bounded by (diameter)$\times M$, yielding the stated $W_1$ bound.
\end{proof}

\begin{remark}
Lemma~\ref{lem:sliver-sampling} shows that in the \emph{flat ball model}, one can realize arbitrarily many tiny equal-mass calibrated sliver pieces
and obtain an explicit $W_1$ approximation rate for a natural target measure on the transverse parameter sphere.
The remaining difficulty for the Hodge program is to reproduce an analogue of this construction using \emph{holomorphic complete intersections}
in a projective manifold with uniform $C^1$ control on the relevant cell scale, and then to integrate such local constructions into a global gluing scheme.
\end{remark}

\begin{lemma}[Quantizing a Lipschitz density on a sphere]\label{lem:sphere-quantize}
Let $d\ge 2$ and let $S^{d-1}(r)\subset\R^d$ be the Euclidean sphere of radius $r>0$.
Let $\rho$ be a nonnegative Lipschitz function on $S^{d-1}(r)$ with total mass
\[
M:=\int_{S^{d-1}(r)} \rho\,d\sigma.
\]
Then for every $N\in\N$ there exist points $t_1,\dots,t_N\in S^{d-1}(r)$ such that the equal-weight atomic measure
\[
\mu_N:=\sum_{a=1}^N \frac{M}{N}\,\delta_{t_a}
\]
satisfies the transport bound
\[
W_1(\mu_N,\rho\,d\sigma)\ \le\ C(d)\,r\,\Bigl(M+\mathrm{Lip}(\rho)\,r^{d-1}\Bigr)\,N^{-\frac{1}{d-1}}.
\]
Moreover, the points may be chosen $\delta$--separated with
\[
\|t_a-t_b\|\ \ge\ c(d)\,r\,N^{-\frac{1}{d-1}}
\qquad (a\neq b).
\]
\end{lemma}

\begin{proof}[Sketch]
Take a maximal $\delta$--separated set $\{t_a\}\subset S^{d-1}(r)$ with $\delta\asymp r\,N^{-1/(d-1)}$; it has cardinality $\asymp N$ by
volume packing, and by trimming/duplicating a bounded number of points one obtains exactly $N$ points with the stated separation.
Let $\{C_a\}$ be the associated Voronoi cells; then $\mathrm{diam}(C_a)\lesssim \delta$.
Couple $\rho\,d\sigma$ to $\mu_N$ by transporting the mass of each cell $C_a$ to $t_a$ (cost $\lesssim \delta\int_{C_a}\rho$).
Lipschitzness controls the discrepancy between the cell masses $\int_{C_a}\rho$ and the equal weights $M/N$, and rebalancing these masses
costs at most the same order as $\delta$.
\end{proof}

\begin{theorem}[Flat sliver-local realizability on a ball (model theorem)]\label{thm:flat-sliver-local}
Let $B_h\subset\R^{k+2p}$ be the Euclidean ball of radius $h$ centered at $0$ and let $P$ be an oriented $k$-plane.
Fix a radius $r\in(0,h)$ and let $\sigma_r$ denote the normalized surface measure on $S^{2p-1}(r)\subset P^\perp$.
Let $\rho$ be a nonnegative Lipschitz density on $S^{2p-1}(r)$ with total mass
$M=\int_{S^{2p-1}(r)}\rho\,d\sigma_r$.
Then for every $N\in\N$ there exist $N$ pairwise disjoint affine plane sliver pieces
\[
T_N:=\sum_{a=1}^N \bigl([P+t_a]\llcorner B_h\bigr)
\]
such that:
\begin{enumerate}
\item[\textnormal{(i)}] each piece has equal mass $\Mass([P+t_a]\llcorner B_h)=M/N$;
\item[\textnormal{(ii)}] the induced transverse atomic measure $\mu_N:=\sum_{a=1}^N \frac{M}{N}\,\delta_{t_a}$ satisfies
\[
W_1(\mu_N,\rho\,d\sigma_r)\ \le\ C(p)\,r\,\Bigl(M+\mathrm{Lip}(\rho)\,r^{2p-1}\Bigr)\,N^{-\frac{1}{2p-1}}.
\]
\end{enumerate}
\end{theorem}

\begin{proof}[Sketch]
First apply Lemma~\ref{lem:sphere-quantize} (with $d=2p$) to obtain points $t_a\in S^{2p-1}(r)$ such that the equal-weight measure
$\sum (M/N)\delta_{t_a}$ approximates $\rho\,d\sigma_r$ in $W_1$ with the stated rate.
Since all $t_a$ lie on the same sphere, the corresponding plane sections have the same mass in the ball by Lemma~\ref{lem:sliver-ball-scaling}
(mass depends only on $\|t\|=r$).  Disjointness is automatic because affine parallel planes are disjoint in Euclidean space.
\end{proof}

\begin{remark}[What remains to lift this to the projective K\"ahler setting]
Theorem~\ref{thm:flat-sliver-local} shows the “sliver microstructure” target is true in a clean flat model.
To use it in the Hodge program one must replace affine plane pieces by \emph{holomorphic complete intersections} with:
\begin{itemize}
\item uniform $C^1$ graph control on a cell (to preserve tiny masses and disjointness);
\item enough jet-separation / peak-section supply to realize $N$ pieces simultaneously at the required separation scale;
\item compatibility across faces so that the induced transverse measures glue with small flat norm globally.
\end{itemize}
\end{remark}

\section{Flat Kähler Model and Local Microstructure}

To probe the microstructure question, we pass to a flat model where the calibration is explicit and the geometry is as simple as possible.

\subsection{Flat calibration in \texorpdfstring{$\R^{2n}$}{R2n}}

Identify $\C^n\cong\R^{2n}$ with standard complex structure $J$, Euclidean metric $g_0$, and Kähler form
\[
  \omega_0(u,v) := g_0(Ju,v).
\]
Fix $1\le p\le n$. The flat calibration is
\[
  \varphi_0 := \frac{\omega_0^p}{p!},
\]
a constant $2p$--form on $\R^{2n}$.

An oriented $2p$--plane $P\subset\R^{2n}$ is $\varphi_0$--calibrated if and only if $P$ is a complex $p$--plane with its complex orientation. The calibrated Grassmannian
\[
  \mathcal{G}_{\varphi_0} := \{P\in\Gr_{2p}(\R^{2n}) : \varphi_0|_P = \mathrm{vol}_P\}
\]
is naturally the complex Grassmannian $\Gr_p(\C^n)$.

Each $P\in\mathcal{G}_{\varphi_0}$ has an associated simple unit $2p$--form $\xi_P$.

\begin{definition}[Flat calibration cone]
At each $x\in\R^{2n}$, define the cone
\[
  K_p(x) := \left\{ \beta\in \Lambda^{p,p}_\R(T_x^*\R^{2n}) \,\middle|\,
    \beta = \sum_{i=1}^k a_i\,\xi_{P_i},\ a_i\ge 0,\ P_i\in\mathcal{G}_{\varphi_0}
  \right\}.
\]
A smooth real $(p,p)$--form $\beta$ on $\R^{2n}$ is \emph{cone-valued} if $\beta(x)\in K_p(x)$ for all $x$; if moreover $d\beta=0$, it is a \emph{closed cone-valued form}.
\end{definition}

\subsection{Local flat-space microstructure conjectures}

We can now formulate flat analogues of Conjecture~\ref{conj:micro-abstract}.

\begin{conjecture}[Flat microstructure, local B1]\label{conj:B1-flat}
Let $\varphi_0=\omega_0^p/p!$ on $\R^{2n}$, with $2\le p\le n-2$. Let $\beta$ be a smooth, compactly supported, closed cone-valued $(p,p)$--form on $\R^{2n}$. Then for every $\varepsilon>0$ there exists an integral $2p$--current $T_\varepsilon$ on $\R^{2n}$ with
\begin{itemize}
  \item $\partial T_\varepsilon = 0$ (or controlled boundary),
  \item $\Mass(T_\varepsilon) \le \int T_\varepsilon\wedge\varphi_0 + \varepsilon$,
  \item $\displaystyle \int T_\varepsilon\wedge\psi \to \int \beta\wedge\psi$ for all smooth compactly supported forms $\psi$ of degree $2n-2p$.
\end{itemize}
\end{conjecture}

Equivalently, $T_\varepsilon$ are almost-calibrated cycles whose homological functionals converge to those determined by $\beta$.

A Young-measure version is:

\begin{conjecture}[Flat Young-measure realization, local B2]\label{conj:B2-flat}
Let $x\mapsto\nu_x$ be a measurable field of probability measures on $\mathcal{G}_{\varphi_0}$, compactly supported in $x$, such that the barycenter
\[
  \beta(x) := \int_{\mathcal{G}_{\varphi_0}} \xi_P\, d\nu_x(P)
\]
defines a smooth closed cone-valued $(p,p)$--form on $\R^{2n}$. Then there exists a sequence of integral $\varphi_0$--calibrated currents $T_k$ such that
\begin{itemize}
  \item the tangent-plane Young measures of $T_k$ converge to $\nu_x$ almost everywhere,
  \item $\Mass(T_k) \to \int \beta\wedge\varphi_0$.
\end{itemize}
\end{conjecture}

These conjectures capture the local microstructure problem that underlies Gap B.

\section{The Case \texorpdfstring{$n=3$}{n=3}, \texorpdfstring{$p=2$}{p=2} in \texorpdfstring{$\R^6$}{R6}}

We now specialize to $\C^3\cong\R^6$, where $p=2$ is the first nontrivial middle dimension.

\subsection{Coordinate setup}

Let $z_j = x_j + i y_j$, $j=1,2,3$, so $\R^6$ has orthonormal basis
\[
  e^1 = dx_1,\ e^2 = dy_1,\ e^3 = dx_2,\ e^4 = dy_2,\ e^5 = dx_3,\ e^6 = dy_3.
\]
The standard Kähler form is
\[
  \omega_0 = e^1\wedge e^2 + e^3\wedge e^4 + e^5\wedge e^6.
\]
For $p=2$, the calibration is
\[
  \varphi_0 = \frac{\omega_0^2}{2}
  = e^1\wedge e^2\wedge e^3\wedge e^4
  + e^1\wedge e^2\wedge e^5\wedge e^6
  + e^3\wedge e^4\wedge e^5\wedge e^6.
\]

We consider three coordinate complex $2$--planes:
\begin{align*}
  P_1 &= \{z_3=0\}, & \xi_{P_1} &= e^1\wedge e^2\wedge e^3\wedge e^4,\\
  P_2 &= \{z_2=0\}, & \xi_{P_2} &= e^1\wedge e^2\wedge e^5\wedge e^6,\\
  P_3 &= \{z_1=0\}, & \xi_{P_3} &= e^3\wedge e^4\wedge e^5\wedge e^6.
\end{align*}
Then
\[
  \varphi_0 = \xi_{P_1} + \xi_{P_2} + \xi_{P_3}.
\]

We focus on cone-valued forms of the shape
\[
  \beta(x) = f_1(x)\,\xi_{P_1} + f_2(x)\,\xi_{P_2},
\]
with $f_i\ge 0$ smooth.

\subsection{Closedness constraints}

Because $\xi_{P_1},\xi_{P_2}$ are constant 4-forms, one has
\[
  d\beta = d f_1\wedge \xi_{P_1} + d f_2\wedge \xi_{P_2}.
\]
Writing $df_1 = \sum_{j=1}^6 (\partial_j f_1)\,e^j$ and similarly for $f_2$, one checks
\[
  df_1\wedge \xi_{P_1} = 0\quad\Longleftrightarrow\quad \partial_5 f_1 = \partial_6 f_1 = 0,
\]
and
\[
  df_2\wedge \xi_{P_2} = 0\quad\Longleftrightarrow\quad \partial_3 f_2 = \partial_4 f_2 = 0.
\]
Thus
\[
  d\beta=0 \quad\Longleftrightarrow\quad
  f_1 = f_1(x_1,y_1,x_2,y_2),\quad
  f_2 = f_2(x_1,y_1,x_3,y_3),
\]
with no dependence on $(x_3,y_3)$ for $f_1$ and no dependence on $(x_2,y_2)$ for $f_2$. In particular, if we restrict to $f_i=f_i(x_1,y_1)$ only, then $\beta$ is automatically closed.

\begin{lemma}[Closedness for a single constant decomposable direction]\label{lem:closed-one-direction}
Let $\xi$ be a constant decomposable real $2p$-form on $\R^{2n}$ of the form
\[
\xi = \alpha_1\wedge\cdots\wedge \alpha_{2p},
\]
where the $\alpha_i$ are constant $1$-forms.  Let $\beta=f\,\xi$ for a smooth function $f$.
Then
\[
d\beta=0 \quad\Longleftrightarrow\quad df\wedge\xi=0.
\]
In particular, if $\xi=dx_1\wedge dy_1\wedge\cdots\wedge dx_p\wedge dy_p$ in coordinates, then $d\beta=0$ implies
$\partial_{x_j}f=\partial_{y_j}f=0$ for all $j>p$, i.e.\ $f$ is constant in the $2(n-p)$ transverse coordinates.
\end{lemma}

\begin{proof}
Since $\xi$ is constant, $d\beta=d(f\xi)=df\wedge\xi$.
In the coordinate normal form, write $df=\sum_{j=1}^n(\partial_{x_j}f\,dx_j+\partial_{y_j}f\,dy_j)$.
Wedge with $\xi$ kills the components along $dx_1,dy_1,\dots,dx_p,dy_p$ by repetition, and produces linearly independent
$(2p+1)$-forms for the components along $dx_j,dy_j$ with $j>p$, forcing those coefficients to vanish.
\end{proof}

\begin{remark}[Interpretation for face matching]
Lemma~\ref{lem:closed-one-direction} shows that in the \emph{single-direction} flat model,
closedness prevents the coefficient $f$ from varying in the transverse coordinates.
Consequently, if one builds a sheet-stack from parallel translates with transverse sampling that is uniform at each fixed
$(x_1,y_1,\dots,x_p,y_p)$, then the induced transverse measures on adjacent cube faces match exactly (so $W_1=0$ and no gluing is needed).
The general multi-direction case is harder because the coefficients can vary, and one must transport mass between different directions/transverse parameters.
\end{remark}

\begin{lemma}[Closedness constraints for a finite coordinate family]\label{lem:closed-finite-coordinate-family}
Let $\{e^1,\dots,e^{2n}\}$ be the standard basis of constant $1$-forms on $\R^{2n}$, and let
$\xi_1,\dots,\xi_J$ be constant decomposable $2p$-forms of the form
\[
\xi_j = e^{i_{j,1}}\wedge \cdots \wedge e^{i_{j,2p}}.
\]
Let $\beta=\sum_{j=1}^J f_j\,\xi_j$ with $f_j\in C^\infty(\R^{2n})$.
Assume the collection of $(2p+1)$-forms
\[
\mathcal B:=\{\, e^a\wedge \xi_j : 1\le j\le J,\ a\not\in \{i_{j,1},\dots,i_{j,2p}\}\,\}
\]
is linearly independent in $\Lambda^{2p+1}(\R^{2n})^*$.
Then $d\beta=0$ implies $df_j\wedge \xi_j=0$ for each $j$; equivalently,
all partial derivatives of $f_j$ in directions $e^a$ with $a\notin\{i_{j,1},\dots,i_{j,2p}\}$ vanish.
\end{lemma}

\begin{proof}
Since each $\xi_j$ is constant,
\[
d\beta=\sum_{j=1}^J df_j\wedge\xi_j
=\sum_{j=1}^J\sum_{a=1}^{2n} (\partial_a f_j)\, e^a\wedge\xi_j.
\]
For $a\in\{i_{j,1},\dots,i_{j,2p}\}$, one has $e^a\wedge\xi_j=0$ by repetition.
Thus $d\beta$ is a linear combination of the forms in $\mathcal B$ with coefficients $\partial_a f_j$.
If $d\beta=0$ and $\mathcal B$ is linearly independent, then every coefficient must vanish, so $\partial_a f_j=0$
whenever $a\notin\{i_{j,1},\dots,i_{j,2p}\}$.
\end{proof}

\subsection{Cube-consistency as a finite-dimensional marginal/coupling problem}

One concrete way to view the remaining “transport matching across faces’’ difficulty is as a finite-dimensional
\emph{marginal realization} (coupling) problem for the translation parameters used to generate parallel calibrated sheets.

\begin{definition}[Discrete marginal realization problem]\label{def:marginal-realization}
Fix integers $d\ge 1$ and $M\ge 1$. Let $\Omega:=[M]^d$ be a $d$-dimensional grid.
For each coordinate $\ell\in\{1,\dots,d\}$, let $m_\ell\in\mathbb Z_{\ge 0}^{[M]}$ be a prescribed integer histogram
with total mass
\[
\sum_{a=1}^M m_\ell(a)=N \qquad\text{(independent of $\ell$)}.
\]
The problem is to find a multiset of grid points $\{x_1,\dots,x_N\}\subset \Omega$ such that for each $\ell$ and each $a\in[M]$,
the number of points with $(x_i)_\ell=a$ equals $m_\ell(a)$.
\end{definition}

\begin{lemma}[Existence of a multiset with prescribed 1D marginals]\label{lem:discrete-marginals}
The marginal realization problem in Definition~\ref{def:marginal-realization} always has a solution.
\end{lemma}

\begin{proof}[Sketch]
For $d=1$ the statement is trivial.  For $d=2$, it is the existence of a nonnegative integer matrix with prescribed row and column sums,
which is equivalent to a bipartite multigraph with prescribed degrees; construct it greedily or via max-flow.

For $d>2$, proceed by induction: realize the first $d-1$ marginals by a multiset of points in $[M]^{d-1}$ (induction hypothesis),
then for each realized $(d-1)$-tuple append a $d$-th coordinate so that the $d$-th marginal histogram is met (again by a bipartite matching
between the existing $(d-1)$-tuples and the $d$-th coordinate values with multiplicity).
\end{proof}

\begin{remark}[Relevance to transverse translations]
In the idealized flat/parallel setting, translation parameters $t\in N^\perp\cong\R^{2p}$ can be discretized to a grid,
and the intersection of translated sheets with a cube face can depend only on a subset of the coordinates of $t$.
In such cases, enforcing “matching across faces’’ reduces to realizing compatible histograms/marginals.
Lemma~\ref{lem:discrete-marginals} shows that \emph{purely combinatorially}, these marginal constraints are not an obstacle.

The geometric difficulty in the Hodge program is that for a general calibrated plane direction, the face slice depends on $t$ through
a \emph{nontrivial linear map} (and simultaneously for many faces), so one needs realizability for more general families of pushforwards
than coordinate projections, together with small-angle stability and global mass control.
\end{remark}

\section{Finite-Direction Laminates on the Flat Torus}

To make the microstructure question global and periodic, we pass to the flat complex torus
\[
  T^6 := (\R/\Z)^6 \cong \C^3/\Lambda
\]
with the induced calibration $\varphi_0$.

Let $T_1,T_2\subset T^6$ be the complex $2$--dimensional subtori descending from $P_1,P_2$, i.e.
\begin{align*}
  T_1 &= \{z_3=0\}/\Lambda,\quad \text{a $4$--torus with tangent $P_1$},\\
  T_2 &= \{z_2=0\}/\Lambda,\quad \text{a $4$--torus with tangent $P_2$}.
\end{align*}

We consider integral calibrated currents built from these two directions alone.

\begin{definition}[Finite-direction calibrated laminate on $T^6$]
A \emph{finite-direction calibrated laminate} (using $P_1,P_2$) is a finite or locally finite sum of the form
\[
  T = \sum_{a} m_a [T_1+v_a] + \sum_{b} n_b [T_2+w_b],
\]
where $m_a,n_b\in\N$ and $v_a,w_b\in T^6$, and each $[T_i+v]$ is the current of integration over a translate of $T_i$, oriented by its complex structure. Such a $T$ is calibrated by $\varphi_0$, with tangent planes in $\{P_1,P_2\}$ almost everywhere.
\end{definition}

The following theorem formalizes the finite-direction obstruction we found heuristically.

\begin{theorem}[Finite-direction no-go theorem on $T^6$]\label{thm:finite-no-go}
Let $T$ be a closed integral calibrated $4$--current on $(T^6,\varphi_0)$ whose approximate tangent planes lie in $\{P_1,P_2\}$ for $\mu_T$--almost every point, where $\mu_T$ is the mass measure of $T$.
Let $\mu_{T,1}$ (resp.\ $\mu_{T,2}$) be the restriction of $\mu_T$ to the subset where the approximate tangent plane equals $P_1$ (resp.\ $P_2$).

Let
\[
  \pi: T^6 \to T^2_{(x_1,y_1)}
\]
be the projection forgetting the other four coordinates. Then there exist constants $c_1,c_2\ge 0$ such that
\[
  \pi_\#\mu_{T,1} = c_1\,dx_1\wedge dy_1,
  \qquad
  \pi_\#\mu_{T,2} = c_2\,dx_1\wedge dy_1.
\]
In particular, no sequence of such $T$ can realize a nonconstant ``macroscopic coefficient field''
$\beta(x_1,y_1)=f_1(x_1,y_1)\xi_{P_1}+f_2(x_1,y_1)\xi_{P_2}$ via these pushforward densities.
\end{theorem}

\begin{proof}
Because $T$ is calibrated by $\varphi_0$, the Harvey--Lawson structure theorem implies that
$T=\sum_i m_i [V_i]$ where each $V_i\subset T^6$ is an irreducible complex analytic surface and $m_i\in\N$.
Fix such an irreducible component $V$ and let $V_{\mathrm{reg}}$ be its smooth locus.
The Gauss map $G:V_{\mathrm{reg}}\to \Gr_2(\C^3)$ is holomorphic.  Since $G(V_{\mathrm{reg}})\subset\{P_1,P_2\}$,
and $\{P_1,P_2\}$ is discrete, the map $G$ is constant on each connected component of $V_{\mathrm{reg}}$.
Hence the complex tangent plane of $V$ is \emph{constant} (equal to either $P_1$ or $P_2$) on $V_{\mathrm{reg}}$.

Lifting to the universal cover $\C^3\to T^6$, each connected component of the lift of $V_{\mathrm{reg}}$ is an open subset
of an affine complex plane $P_i + z_0$.  Taking closures and descending modulo the lattice shows that each $V_i$ is a translate
of the corresponding complex subtorus $T_1$ or $T_2$.  Therefore $T$ is a finite sum of translates of $T_1$ and $T_2$
with multiplicities, and $\mu_{T,1}$ and $\mu_{T,2}$ are (finite) sums of the induced volume measures on those translates.

Consider the projection
\[
  \pi: T^6 \to T^2_{(x_1,y_1)},
\]
forgetting the other four coordinates. For a coset $T_1+v$, the map $\pi|_{T_1+v}:T_1+v\to T^2_{(x_1,y_1)}$ is a homomorphism of tori (up to translation). Its differential is constant, and the Jacobian of $\pi|_{T_1+v}$ is constant in magnitude, say $c_1>0$. Thus the pushforward of the volume measure on $T_1+v$ is a constant multiple of Lebesgue measure on $T^2$:
\[
  (\pi|_{T_1+v})_\# (\mathrm{vol}_{T_1+v}) = c_1\,dx_1\wedge dy_1.
\]
The same holds for any translate of $T_2$: there is a constant $c_2>0$ such that
\[
  (\pi|_{T_2+w})_\# (\mathrm{vol}_{T_2+w}) = c_2\,dx_1\wedge dy_1.
\]

Now $\mu_{T,1}$ is a sum of such pushed-forward measures coming from translates of $T_1$, and $\mu_{T,2}$ is a sum of those coming from translates of $T_2$.
Therefore, each pushforward $\pi_\#\mu_{T,i}$ is translation-invariant on $T^2_{(x_1,y_1)}$, i.e.\ there exists a constant $c_i\ge 0$ such that
\[
  \pi_\#\mu_{T,i} = c_i\,dx_1\wedge dy_1.
\]
This proves the claimed constancy of the macroscopic coefficients $c_1,c_2$, and rules out nonconstant target densities
$f_i(x_1,y_1)$ arising from finite-direction currents.
\end{proof}

This shows that the finite-direction laminate conjecture (L2) is \emph{false} in this strict form on $T^6$: one cannot realize nonconstant cone-valued $\beta$ using only a finite set of fixed calibrated directions.

\section{Dropping the Finite-Direction Constraint}

We now ask what remains if we allow tangent planes to explore the full calibrated Grassmannian $\mathcal{G}_{\varphi_0}$.

\subsection{Stationarity as a necessary PDE-type condition}

Let $T_k$ be a sequence of calibrated integral $2p$--currents in a Riemannian manifold (here, $T^6$). Their associated varifolds $V_k$ satisfy $\delta V_k=0$ (stationarity) for all smooth vector fields $X$.

Assume $V_k$ converge (in the varifold sense) to a varifold $V$ with associated mass measure $\mu$ and tangent-plane Young measure $\nu_x$ (a probability measure on $\mathcal{G}_{\varphi_0}$ for $\mu$--almost every $x$). Then stationarity passes to the limit:
\[
  \delta V(X) = \int_{T^6} \int_{\mathcal{G}_{\varphi_0}} \operatorname{div}^P X(x)\, d\nu_x(P)\,d\mu(x) = 0
\]
for all smooth vector fields $X$, where $\operatorname{div}^P X$ is the divergence of $X$ along the plane $P$.

Formally, if one defines an average projection tensor
\[
  A(x) := \int_{\mathcal{G}_{\varphi_0}} \Pi_P\, d\nu_x(P),
\]
where $\Pi_P$ is the orthogonal projection onto $P$, then the stationarity condition can be interpreted as a weak divergence-free condition on $A$ with respect to $\mu$:
\[
  \int \langle A(x),DX(x)\rangle\, d\mu(x) = 0
\]
for all $X$.

This is a genuine PDE-type necessary condition on the pair $(\mu,\nu_x)$, and hence on any realizable microstructure.

\subsection{Information visible at the level of $\beta$}

Given $(\mu,\nu_x)$ as above, the associated barycenter $(p,p)$--form is
\[
  \beta(x) = \int_{\mathcal{G}_{\varphi_0}} \xi_P\, d\nu_x(P),
\]
interpreted in a suitable averaged sense relative to $\mu$. For sequences of calibrated currents converging weakly as currents, the absolutely continuous part of this $\beta$ must be:
\begin{itemize}
  \item a real $(p,p)$--form,
  \item cone-valued: $\beta(x)\in K_p(x)$ a.e.,
  \item closed: $d\beta = 0$.
\end{itemize}

These are necessary local conditions that live purely on $\beta$.

However, stationarity is expressed in terms of $\Pi_P$ (projection operators) rather than $\xi_P$ (wedge forms). The maps $P\mapsto \xi_P$ and $P\mapsto \Pi_P$ are related but not linearly invertible; many different $\nu_x$ can have the same barycenter $\beta(x)$. Thus there is no simple local PDE constraint on $\beta$ alone that encodes stationarity; it genuinely involves the richer data $(\mu,\nu_x)$.

\subsection{Back to the model $\beta=f_1(x_1,y_1)\xi_{P_1}+f_2(x_1,y_1)\xi_{P_2}$}

Consider again on $T^6$ the form
\[
  \beta(x) = f_1(x_1,y_1)\,\xi_{P_1} + f_2(x_1,y_1)\,\xi_{P_2},
\]
with $f_i\ge 0$ smooth and periodic in $(x_1,y_1)$ only. As computed earlier, $d\beta=0$ is automatic in this case, and $\beta(x)$ lies in the cone generated by $\xi_{P_1},\xi_{P_2}$, hence in $K_2(x)$.

Thus $\beta$ satisfies all the \emph{β-only} local necessary conditions:
\begin{itemize}
  \item real $(2,2)$,
  \item strongly positive (cone-valued),
  \item closed.
\end{itemize}

Once one allows $\nu_x$ to range over the full calibrated Grassmannian, the projection-rigidity argument of Theorem~\ref{thm:finite-no-go} no longer applies: complex surfaces not restricted to subtori can project non-uniformly to $T^2_{(x_1,y_1)}$, and there is no obvious mechanism forcing $\beta$ to be constant. Whether there exists a realizable $(\mu,\nu_x)$ satisfying stationarity and giving this $\beta$ remains exactly the content of the microstructure conjecture in this setting.

In other words:
\begin{itemize}
  \item The finite-direction laminate approach (using only $P_1,P_2$) is \emph{provably too rigid} to realize nonconstant $\beta$ of this form.
  \item Once the full cone of calibrated directions is allowed, there is no known local PDE-type obstruction at the level of $\beta$; the remaining difficulties are global and analytic, encoded in the stationarity condition on $(\mu,\nu_x)$.
\end{itemize}

\section{Global Template Coherence: Proposed Reduction (still open)}

This section records a concrete \emph{reduction strategy} for the remaining global microstructure/gluing obstruction (``Blocker B1'' in the main manuscript):
\emph{constructing cellwise sheet/sliver pieces so that all interior faces are matched simultaneously}, while still meeting the barycenter and global cohomology constraints.

\subsection{Setup}

Let $(X,\omega)$ be a smooth projective K\"ahler manifold, set $\psi:=\omega^{n-p}/(n-p)!$, and let $\beta$ be a smooth closed strongly positive $(p,p)$--form in a rational class $[\gamma]$.
Fix a mesh-$h$ decomposition by smooth uniformly convex cells $Q$ (rounded cubes/balls), and consider local calibrated pieces inside each $Q$ whose union gives a raw current
\[
T^{\mathrm{raw}}:=\sum_Q S_Q,
\qquad
\Mass(S_Q)\approx M_Q:=m\int_Q \beta\wedge\psi.
\]
The global goal is $\mathcal F(\partial T^{\mathrm{raw}})=o(m)$, so that a standard flat-norm filling produces a small correction and hence a closed calibrated cycle in the target class.

\subsection{Universal templates and automatic $W_1$ matching (one face)}

The remaining obstruction is not the existence of local pieces (which can be arranged in the flat model and, conditionally, by holomorphic complete intersections),
but \emph{global compatibility}: for each interior interface $F=Q\cap Q'$, the two sides must induce face measures that are close in $W_1$.
The difficulty is that for a fixed cell $Q$, the induced measures on \emph{different} faces of $Q$ are not independent: they are pushforwards of the \emph{same} discrete translation multiset by different face maps.

\begin{definition}[Universal template on translation space]
Fix $K\ge 1$ and a multiset $\mathcal T_0=\{t_1,\dots,t_K\}\subset B(0,C h)\subset\R^{2p}$.
For a cell $Q$ and a chosen calibrated direction family $j$, identify the transverse translation space with $\R^{2p}$ in a fixed tubular chart and define an integer measure
\[
\nu_{Q,j}:=\sum_{k=1}^{K_{Q,j}}\delta_{t_k},
\]
where $K_{Q,j}\le K$ (or more generally $\nu_{Q,j}$ is obtained from $\mathcal T_0$ by a small number of insertions/deletions).
For a face $F\subset\partial Q$, define the induced face measure as a pushforward
\[
\mu_{Q\to F,j}:=(\Phi_{Q,F,j})_\#\nu_{Q,j},
\]
where $\Phi_{Q,F,j}$ is the (approximately linear) face-slice map in the flat/graph model.
\end{definition}

\begin{lemma}[$W_1$ stability under small linear perturbations]\label{lem:w1-template-stability}
Let $\nu$ be a finite Borel measure on $\R^{2p}$ supported in a ball of radius $\le C h$ and with total mass $\nu(\R^{2p})=N$.
Let $\Phi,\Phi':\R^{2p}\to\R^{2p}$ be linear maps with $\|\Phi-\Phi'\|_{\mathrm{op}}\le C_1 h$.
Then
\[
W_1(\Phi_\#\nu,\Phi'_\#\nu)\ \le\ C_2\,h^2\,N,
\]
where $C_2$ depends only on $C,C_1$.
\end{lemma}

\begin{proof}
Couple $\Phi_\#\nu$ and $\Phi'_\#\nu$ by pushing forward $\nu$ under $y\mapsto(\Phi y,\Phi' y)$, and bound the transport cost by
\[
\int \|\Phi y-\Phi' y\|\,d\nu(y)\ \le\ \|\Phi-\Phi'\|_{\mathrm{op}}\int \|y\|\,d\nu(y)
\ \le\ C_1 h\cdot (C h)\,N.
\]
\end{proof}

\begin{remark}[What the universal-template idea would buy]
If adjacent cells $Q,Q'$ across $F$ use the same underlying template (so that $\nu_{Q,j}$ and $\nu_{Q',j'}$ differ only by a small edit) and if the corresponding face maps satisfy
$\|\Phi_{Q,F,j}-\Phi_{Q',F,j'}\|_{\mathrm{op}}=O(h)$, then Lemma~\ref{lem:w1-template-stability} gives an automatic $W_1$ control at scale $h^2N_F$.
Inserting this into the transport$\Rightarrow$flat-norm estimate yields a global $\mathcal F(\partial T^{\mathrm{raw}})$ bound \emph{provided} the edit term is also controlled.
\end{remark}

\subsection{What remains to make this rigorous}

The universal-template viewpoint reduces Blocker B1 to \emph{discrete} and \emph{graph-global} tasks:
\begin{enumerate}
\item \textbf{Stable direction labeling.} One needs a globally coherent way to choose the finitely many calibrated directions used per cell so that neighboring cells admit $O(h)$ pairings of directions.
This is a ``stable Carath\'eodory selection'' problem (see `mg-global-template-coherence-plan.md`).
\item \textbf{Graph-Lipschitz discrepancy rounding.} One must choose integer multiplicities (or template truncation sizes) per cell to match the barycenter/mass targets
and simultaneously satisfy the global cohomology constraints, \emph{without} breaking the slow-variation bounds across the adjacency graph.
\item \textbf{Sliver mass matching on a fixed template.} In the Bergman/sliver regime, one must implement cellwise mass targets using many tiny pieces
while keeping the number of pieces per cell within the polynomial-growth regime required by the weighted gluing estimates.
\item \textbf{Holomorphic realization at the required template size.} One must verify that the holomorphic complete-intersection construction can realize the needed finite templates on each cell (with uniform $C^1$ control and disjointness) at the chosen cell scale.
\end{enumerate}

\begin{remark}[A concrete way to bypass ``stable Carath\'eodory selection'' (direction nets + strongly convex coefficients)]
There is a potentially cleaner way to address Item~(1) that avoids tracking a moving Carath\'eodory support.
Work on the normalized slice of the strong cone:
\[
\mathcal K:=\Bigl\{\xi\in K_p(x):\ \langle \xi,\psi_x\rangle=1\Bigr\}\subset \mathcal V_x,
\]
which is compact and convex, with extreme points $\{\xi_P:\ P\in \Gr_{n-p}(T_xX)\}$.
Fix a small tolerance $\varepsilon_h\downarrow 0$ and choose a finite $\varepsilon_h$--net
$\{\xi_{P_r}\}_{r=1}^{R(h)}$ in the extreme set (in Hilbert--Schmidt norm).
Then $\mathcal K$ lies within $O(\varepsilon_h)$ (Hausdorff) of the polytope
$\mathcal K_h:=\mathrm{conv}\{\xi_{P_r}\}$.

Define coefficients \emph{canonically} by a strongly convex quadratic program:
for $\widehat\beta(x)\in\mathcal K$, let $w(x)\in\Delta^{R(h)}$ be the unique minimizer
\[
w(x):=\arg\min_{w\in\Delta^{R(h)}}\Bigl\|\widehat\beta(x)-\sum_{r=1}^{R(h)} w_r\,\xi_{P_r}\Bigr\|^2+\lambda\|w\|^2,
\qquad \lambda>0,
\]
where $\Delta^{R(h)}=\{w\ge 0,\ \sum_r w_r=1\}$.
Then $x\mapsto w(x)$ is globally well-defined and (by standard parametric convex optimization) is Lipschitz in $\widehat\beta(x)$,
hence Lipschitz in $x$ when $\widehat\beta$ is smooth.
This produces a \emph{globally labeled} direction dictionary $\{P_r\}$ with weights $w_r(x)$ varying smoothly/slowly, so neighbor pairing is trivial.

This replaces the unstable ``choose $N$ planes per $x$'' step by a fixed direction family whose size grows as $h\to 0$.
It is compatible with the finite-direction no-go theorem because $R(h)\to\infty$ as the mesh is refined.
\end{remark}

\begin{remark}[Takeaway]
This section should be read as a \emph{promising reduction}, not as a completed proof.
At present, the missing piece is a rigorous global rounding/selection theorem on the cell adjacency graph that preserves slow variation while meeting the cohomology constraints.
\end{remark}

\section{Summary and Outlook}

We have:
\begin{itemize}
  \item Formulated a CPM--Hodge bridging conjecture (Conjecture~\ref{conj:CPM-Hodge}) and established a robust forms-level coercivity (Lemma~\ref{lem:forms-coercivity}) under the hypothesis that the harmonic representative lies in the calibration cone.
  \item Isolated two gaps: cone-compatible regularization (Gap A) and the microstructure realization problem (Gap B), expressed abstractly in Conjecture~\ref{conj:micro-abstract}.
  \item Localized Gap B into flat-space microstructure conjectures (Conjectures~\ref{conj:B1-flat} and \ref{conj:B2-flat}) in $\R^{2n}$ with Kähler calibration.
  \item In the simplest nontrivial model $(n=3,p=2)$ on the flat torus $T^6$, proved a finite-direction no-go theorem (Theorem~\ref{thm:finite-no-go}), showing that calibrated laminates using only finitely many fixed directions cannot realize nonconstant cone-valued $\beta$. This rules out naive finite-direction laminate strategies for Gap B.
  \item Identified stationarity of the limiting varifold as a genuine PDE-type necessary condition on any realizable measure-valued plane field $x\mapsto\nu_x$, while noting that this condition cannot be expressed purely in terms of the barycenter form $\beta$.
\end{itemize}

The upshot is that any successful approach to the microstructure problem (and hence Gap B) within classical calibrated geometry must either exploit the full calibrated Grassmannian in an essential way, or step beyond the finite-direction laminate paradigm entirely. Within the CPM--Hodge framing, this clarifies where classical tools currently fail and what kind of new geometric-analytic ideas would be required to complete the bridge.

\end{document}
