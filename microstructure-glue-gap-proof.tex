\documentclass[11pt]{article}

\usepackage[T1]{fontenc}
\usepackage{amsmath,amssymb,amsthm}
\usepackage{mathtools}
\usepackage[margin=1in]{geometry}
\usepackage[colorlinks=true,linkcolor=blue,citecolor=blue,urlcolor=blue]{hyperref}

% --------------------
% Basic notation
% --------------------
\newcommand{\R}{\mathbb{R}}
\newcommand{\C}{\mathbb{C}}
\newcommand{\N}{\mathbb{N}}
\newcommand{\F}{\mathcal{F}}
\DeclareMathOperator{\Mass}{Mass}
\DeclareMathOperator{\dist}{dist}

\theoremstyle{plain}
\newtheorem{theorem}{Theorem}
\newtheorem{lemma}[theorem]{Lemma}
\newtheorem{proposition}[theorem]{Proposition}
\newtheorem{corollary}[theorem]{Corollary}

\theoremstyle{remark}
\newtheorem{remark}[theorem]{Remark}

\title{\bfseries A standalone proof of the microstructure/gluing estimate\\(\texorpdfstring{$\F(\partial T^{\mathrm{raw}})=o(m)$}{F(dTraw)=o(m)})}
\author{}
\date{\today}

\begin{document}
\maketitle

\section{What this note proves (and why)}

In the main manuscript \texttt{hodge-SAVE-dec-12-handoff.tex}, the microstructure/gluing
checkpoint is recorded as the quantitative estimate
\[
\F\!\bigl(\partial T^{\mathrm{raw}}\bigr)\ =\ o(m),
\]
for the raw current $T^{\mathrm{raw}}$ built by assembling many local calibrated holomorphic
pieces across a mesh of size $h$.
This is the input needed to produce a correction current $U_\varepsilon$ with
$\partial U_\varepsilon=\partial T^{\mathrm{raw}}$ and $\Mass(U_\varepsilon)\to 0$,
which in turn isolates the global mass convergence
\[
0\le \Mass(T_\varepsilon)-\langle T_\varepsilon,\psi\rangle \to 0
\qquad (T_\varepsilon:=T^{\mathrm{raw}}-U_\varepsilon).
\]

\medskip
\noindent
The purpose of this note is to present the gluing bound as a single
referee-facing argument: \emph{face-level flat-norm control} $\Rightarrow$
\emph{global summation} $\Rightarrow$ \emph{scaling/parameter choice} $\Rightarrow$
$o(m)$.

\section{Set-up}

Fix integers $n\ge 2$ and $1\le p\le n$ and set
\[
d:=2n,\qquad k:=2n-2p \quad (1\le k<d).
\]
Let $X$ be a compact smooth manifold equipped with a Riemannian metric; the argument below is
local and may be carried out in coordinate charts.

\medskip
\noindent
\textbf{Cells.}
Fix a mesh scale $h\in(0,1)$ and a partition of $X$ into finitely many
smooth \emph{uniformly convex} cells $\{Q\}$ (``rounded cubes'') such that:
\begin{itemize}
\item each $Q\subset \R^d$ in local coordinates has diameter $\asymp h$;
\item $\partial Q$ is $C^2$ and its principal curvatures satisfy
$\frac{c}{h}\le \kappa_i\le \frac{C}{h}$ for fixed constants $0<c\le C$.
\end{itemize}

\medskip
\noindent
\textbf{Pieces and the raw current.}
In each cell $Q$, we are given finitely many disjoint calibrated pieces
$Y^{Q,a}\cap Q$ (coming from holomorphic complete intersections in the main paper),
and define the integral current
\[
S_Q\ :=\ \sum_{a\in\mathcal S(Q)} [Y^{Q,a}]\llcorner Q,\qquad
T^{\mathrm{raw}}\ :=\ \sum_Q S_Q.
\]
Each $[Y^{Q,a}]\llcorner Q$ has finite mass; write
\[
m_{Q,a}\ :=\ \Mass\!\bigl([Y^{Q,a}]\llcorner Q\bigr),\qquad
M_Q\ :=\ \sum_{a\in\mathcal S(Q)} m_{Q,a}.
\]
The boundary $\partial T^{\mathrm{raw}}$ is supported on inter-cell interfaces.
For an interior interface $F=Q\cap Q'$ we write the face mismatch current
\[
B_F\ :=\ \bigl(\partial S_Q\bigr)\llcorner F\ -\ \bigl(\partial S_{Q'}\bigr)\llcorner F,
\]
so that
\(
\partial T^{\mathrm{raw}}=\sum_{F} B_F
\)
(sum over interior faces).

\section{The three local lemmas}

\subsection{Flat norm stability under translation}

\begin{lemma}[Flat-norm stability under translation]\label{lem:flat-translate-standalone}
Let $S$ be an integral $\ell$-cycle in $\R^d$ (so $\partial S=0$) with finite mass.
For any translation vector $v\in\R^d$, writing $\tau_v(x):=x+v$ and $(\tau_v)_\#S$ for pushforward,
one has
\[
\F\!\bigl((\tau_v)_\#S-S\bigr)\ \le\ \|v\|\,\Mass(S).
\]
\end{lemma}

\begin{proof}
Consider the homotopy $H:[0,1]\times \R^d\to \R^d$, $H(t,x)=x+t v$.
Then $Q:=H_\#([0,1]\times S)$ satisfies $\partial Q=(\tau_v)_\#S-S$ and
$\Mass(Q)\le \|v\|\,\Mass(S)$.
Taking $R=0$ in the definition of $\F$ yields the claim.
\end{proof}

\subsection{Weighted transport bound for a single face}

\begin{proposition}[Weighted transport $\Rightarrow$ flat-norm face control]\label{prop:transport-flat-weighted-standalone}
Work in a face chart for an interior interface $F=Q\cap Q'$.
Assume each piece meeting $F$ contributes an integral cycle slice current $\Sigma(u)$ on $F$
depending on a transverse parameter $u\in\Omega_F\subset \R^{2p}$, and that
$\Sigma(u)$ is obtained from $\Sigma(0)$ by translation in the face chart.
Let the two adjacent cells induce two multisets of parameters
$\{u_a\}_{a=1}^N$ and $\{u'_a\}_{a=1}^N$ (same cardinality) and define
\[
S_{Q\to F}:=\sum_{a=1}^N \Sigma(u_a),\qquad
S_{Q'\to F}:=\sum_{a=1}^N \Sigma(u'_a),\qquad
B_F:=S_{Q\to F}-S_{Q'\to F}.
\]
Then
\[
\F(B_F)\ \le\ \inf_{\sigma\in S_N}\ \sum_{a=1}^N \|u_a-u'_{\sigma(a)}\|\,\Mass(\Sigma(u_a)).
\]
In particular, if $\|u_a-u'_{\sigma(a)}\|\le \Delta_F$ for all $a$ under some matching $\sigma$,
then
\[
\F(B_F)\ \le\ \Delta_F\sum_{a=1}^N \Mass(\Sigma(u_a)).
\]
\end{proposition}

\begin{proof}
Fix a permutation $\sigma$.
For each $a$, the difference $\Sigma(u_a)-\Sigma(u'_{\sigma(a)})$ is a translated-cycle difference.
By Lemma~\ref{lem:flat-translate-standalone}, there exists an integral filling $Q_a$
with $\partial Q_a=\Sigma(u_a)-\Sigma(u'_{\sigma(a)})$ and
\(
\Mass(Q_a)\le \|u_a-u'_{\sigma(a)}\|\,\Mass(\Sigma(u_a)).
\)
Summing $Q:=\sum_a Q_a$ gives $\partial Q=B_F$ and
$
\Mass(Q)\le \sum_a \|u_a-u'_{\sigma(a)}\|\,\Mass(\Sigma(u_a)).
$
Taking the infimum over $\sigma$ yields the first inequality; the second follows by the uniform bound $\|u_a-u'_{\sigma(a)}\|\le \Delta_F$.
\end{proof}

\subsection{Slice boundary shrinkage in uniformly convex cells}

\begin{lemma}[Boundary shrinkage for plane slices]\label{lem:uniformly-convex-slice-standalone}
Let $Q\subset\R^d$ be a bounded $C^2$ uniformly convex domain of diameter $\asymp h$
whose principal curvatures satisfy $\frac{c}{h}\le \kappa_i\le \frac{C}{h}$ on $\partial Q$.
Fix $1\le k<d$ and a $k$-plane $P$.
For each translate $P+t$ with nonempty intersection set
\[
v(t):=\mathcal H^{k}\bigl((P+t)\cap Q\bigr),\qquad
a(t):=\mathcal H^{k-1}\bigl((P+t)\cap \partial Q\bigr).
\]
Then there exists $C_*=C_*(d,k,c,C)$ such that
\[
a(t)\ \le\ C_*\,\bigl(v(t)\bigr)^{\frac{k-1}{k}}
\qquad\text{for all such }t.
\]
\end{lemma}

\begin{proof}
The estimate is scale invariant; rescale so $h\asymp 1$.
Write $K_t:=(P+t)\cap Q\subset P+t\cong \R^k$ so that $v(t)=\mathcal H^k(K_t)$ and $a(t)=\mathcal H^{k-1}(\partial K_t)$.
If $v(t)\ge v_0>0$, then $K_t$ is a convex body contained in a fixed $k$-ball of radius $O(1)$, hence $a(t)\le A_0(d,k)$, and the bound follows after enlarging $C_*$.
Assume $v(t)\le v_0$ with $v_0$ small.
The curvature pinching implies an interior/exterior rolling-ball condition with radii $r_{\mathrm{in}},r_{\mathrm{out}}\asymp 1$ at every boundary point.
Let $\pi:\R^d\to P^\perp$ be orthogonal projection and set $D:=\pi(Q)\subset P^\perp$.
Choose $t_0\in\partial D$ nearest to $t$ and let $u\in P^\perp$ be an outward normal of a supporting hyperplane at $t_0$, writing $t=t_0-su$.
Let $x_0\in \partial Q$ be the supporting point with outward normal $u$.
Intersect the tangent balls at $x_0$ with $P+t$.
Since $u\perp P$, these intersections are $k$-balls of radii
$
\rho_{\mathrm{in}}(s)=\sqrt{2r_{\mathrm{in}}s-s^2}
$
and
$
\rho_{\mathrm{out}}(s)=\sqrt{2r_{\mathrm{out}}s-s^2}.
$
Thus
\[
\omega_k\,\rho_{\mathrm{in}}(s)^k\ \le\ v(t)\ \le\ \omega_k\,\rho_{\mathrm{out}}(s)^k,
\qquad
a(t)\ \le\ \omega_{k-1}\,\rho_{\mathrm{out}}(s)^{k-1}.
\]
For $s$ small one has $\rho_{\mathrm{in}}(s)\gtrsim \sqrt{s}$ and $\rho_{\mathrm{out}}(s)\lesssim \sqrt{s}$,
so $v(t)\gtrsim s^{k/2}$ and $a(t)\lesssim s^{(k-1)/2}$, hence $s\lesssim v(t)^{2/k}$ and $a(t)\lesssim v(t)^{(k-1)/k}$.
\end{proof}

\section{Global flat-norm bound and the scaling that yields \texorpdfstring{$o(m)$}{o(m)}}

\subsection{Global bound from face control}

\begin{corollary}[Global flat-norm bound from weighted face control]\label{cor:global-flat-weighted-standalone}
Assume that on each interior interface $F=Q\cap Q'$ the face mismatch current $B_F$
fits the setting of Proposition~\ref{prop:transport-flat-weighted-standalone}, and that there exists a matching
with a uniform displacement bound
\[
\|u_a-u'_{\sigma(a)}\|\ \le\ \Delta_F\qquad\text{for all }a.
\]
Assume moreover that each slice $\Sigma_F(u_a)$ arises as the face boundary slice of a piece
$Y^{Q,a}\cap Q$ of interior mass $m_{Q,a}$, and that the cell geometry is uniformly convex as above so that
\[
\Mass(\Sigma_F(u_a))\ \lesssim\ m_{Q,a}^{\frac{k-1}{k}}.
\]
Then
\[
\F\!\bigl(\partial T^{\mathrm{raw}}\bigr)
\ \lesssim\ \sum_F \Delta_F\sum_{a\in\mathcal S(F)} m_{Q,a}^{\frac{k-1}{k}}.
\]
In particular, if $\Delta_F\lesssim h^2$ for all faces, then
\[
\F\!\bigl(\partial T^{\mathrm{raw}}\bigr)
\ \lesssim\ h^2\sum_Q\sum_{a\in\mathcal S(Q)} m_{Q,a}^{\frac{k-1}{k}}.
\]
\end{corollary}

\begin{proof}
By Proposition~\ref{prop:transport-flat-weighted-standalone},
\(
\F(B_F)\le \Delta_F\sum_a \Mass(\Sigma_F(u_a)).
\)
Summing over faces yields the first inequality.  The second follows by inserting $\Delta_F\lesssim h^2$ and the uniform slice bound.
\end{proof}

\subsection{Two elementary geometric bounds}

\begin{lemma}[Pointwise displacement bound under nearby face maps]\label{lem:face-displacement-standalone}
Let $y_1,\dots,y_N\in\R^{2p}$ satisfy $\|y_a\|\le C_0 h$ and let $\Phi,\Phi':\R^{2p}\to\R^{2p}$ be linear maps with
$\|\Phi-\Phi'\|_{\mathrm{op}}\le C_1 h$.  Define $u_a:=\Phi y_a$ and $u'_a:=\Phi' y_a$.
Then the index-wise matching satisfies
\[
\|u_a-u'_a\|\ \le\ C_0C_1\,h^2\qquad\text{for all }a.
\]
\end{lemma}

\begin{proof}
$\|u_a-u'_a\|=\|(\Phi-\Phi')y_a\|\le \|\Phi-\Phi'\|_{\mathrm{op}}\|y_a\|\le (C_1h)(C_0h)=C_0C_1h^2$.
\end{proof}

\begin{lemma}[Packing bound for disjoint sliver graphs]\label{lem:sliver-packing-standalone}
Let $Q\subset\R^{2n}$ be a bounded domain of diameter $h$ and fix an affine $(2n-2p)$-plane $P$ with transverse space $P^\perp\cong\R^{2p}$.
Assume we have affine translates $P+t_1,\dots,P+t_N$ such that each $(P+t_a)\cap Q\neq\emptyset$ and
\[
\|t_a-t_b\|\ \ge\ 10\,\varepsilon\,h\qquad (a\neq b).
\]
Then $N\le C(n,p)\,\varepsilon^{-2p}$.
\end{lemma}

\begin{proof}
Since $(P+t_a)\cap Q\neq\emptyset$ and $\mathrm{diam}(Q)=h$, the translation parameters $t_a$ all lie in a transverse ball $B_{Ch}(0)\subset P^\perp$.
The balls $B(t_a,5\varepsilon h)\subset P^\perp$ are pairwise disjoint and contained in $B_{(C+5\varepsilon)h}(0)$.
Comparing Euclidean volumes in $\R^{2p}$ gives $N\,(5\varepsilon h)^{2p}\lesssim (Ch)^{2p}$, hence $N\lesssim \varepsilon^{-2p}$.
\end{proof}

\subsection{The gluing estimate \texorpdfstring{$\F(\partial T^{\mathrm{raw}})=o(m)$}{F(dTraw)=o(m)}}

\begin{theorem}[Microstructure/gluing estimate (flat-norm form)]\label{thm:glue-gap-standalone}
Assume the hypotheses of Corollary~\ref{cor:global-flat-weighted-standalone}, and suppose additionally:
\begin{itemize}
\item[\textnormal{(A)}] (\textbf{Displacement}) On each interior face $F=Q\cap Q'$, the two face parameterizations arise from applying two linear face maps
$\Phi,\Phi'$ to the same transverse template $\{y_a\}$ with $\|\Phi-\Phi'\|_{\mathrm{op}}=O(h)$ and $\|y_a\|=O(h)$, so that
$\Delta_F=O(h^2)$ by Lemma~\ref{lem:face-displacement-standalone}.
\item[\textnormal{(B)}] (\textbf{Piece count per cell}) For each direction family, disjointness of the pieces in each cell is achieved via transverse separation
$\gtrsim \varepsilon h$ and therefore $|\mathcal S(Q)|\lesssim \varepsilon^{-2p}$ by Lemma~\ref{lem:sliver-packing-standalone}.
\item[\textnormal{(C)}] (\textbf{Total mass scale}) The total mass per cell satisfies $M_Q\asymp m\,h^{2n}$ (uniformly up to bounded factors),
so the total mass is $\sum_Q M_Q\asymp m$.
\end{itemize}
Then there exists a function $\varepsilon_{\mathrm{glue}}=\varepsilon_{\mathrm{glue}}(m,h,\varepsilon)$ with
$\varepsilon_{\mathrm{glue}}\to 0$ in the regime $m\to\infty$, $h\sim m^{-1/2}$, and $\varepsilon=\varepsilon(m)\to 0$ sufficiently slowly,
such that
\[
\F\!\bigl(\partial T^{\mathrm{raw}}\bigr)\ \le\ \varepsilon_{\mathrm{glue}}\,m.
\]
In particular, one may take $h=m^{-1/2}$ and $\varepsilon(m):=(\log m)^{-1}$, in which case $\F(\partial T^{\mathrm{raw}})=o(m)$ whenever
$k=2n-2p>n-1$ (equivalently $p<\tfrac{n+1}{2}$).
\end{theorem}

\begin{proof}
By Corollary~\ref{cor:global-flat-weighted-standalone} and (A),
\[
\F(\partial T^{\mathrm{raw}})\ \lesssim\ h^2\sum_Q\sum_{a\in\mathcal S(Q)} m_{Q,a}^{\frac{k-1}{k}}.
\]
For each cell $Q$, H\"older/concavity gives
\[
\sum_{a\in\mathcal S(Q)} m_{Q,a}^{\frac{k-1}{k}}
\ \le\ M_Q^{\frac{k-1}{k}}\ |\mathcal S(Q)|^{\frac1k}.
\]
Using (B), $|\mathcal S(Q)|^{1/k}\lesssim \varepsilon^{-2p/k}$, hence
\[
\F(\partial T^{\mathrm{raw}})
\ \lesssim\ h^2\,\varepsilon^{-\frac{2p}{k}}\sum_Q M_Q^{\frac{k-1}{k}}.
\]
Using (C), $M_Q\asymp m h^{2n}$ and the number of cells is $\asymp h^{-2n}$, we obtain the scaling
\[
\sum_Q M_Q^{\frac{k-1}{k}}
\ \asymp\ h^{-2n}\,(m h^{2n})^{\frac{k-1}{k}}
\ =\ m^{\frac{k-1}{k}}\,h^{-\frac{2n}{k}}.
\]
Therefore
\[
\F(\partial T^{\mathrm{raw}})
\ \lesssim\ m^{\frac{k-1}{k}}\,h^{\,2-\frac{2n}{k}}\;\varepsilon^{-\frac{2p}{k}}.
\]
At the Bergman cell size $h=m^{-1/2}$, this becomes
\[
\frac{\F(\partial T^{\mathrm{raw}})}{m}
\ \lesssim\ m^{-1+\frac{n-1}{k}}\;\varepsilon^{-\frac{2p}{k}}.
\]
If $k>n-1$ (equivalently $p<\tfrac{n+1}{2}$) then the exponent $-1+\frac{n-1}{k}$ is negative.
Taking $\varepsilon(m)=(\log m)^{-1}$ gives $\varepsilon^{-\frac{2p}{k}}=(\log m)^{2p/k}$, which is dominated by the decaying power of $m$,
so $\F(\partial T^{\mathrm{raw}})/m\to 0$.
\end{proof}

\section{From \texorpdfstring{$\F(\partial T^{\mathrm{raw}})=o(m)$}{F(dTraw)=o(m)} to a vanishing-mass correction}

\begin{corollary}[Existence of a small-mass gluing correction]\label{cor:glue-correction-standalone}
Assume $\F(\partial T^{\mathrm{raw}})=o(m)$ for the raw current above.
Then there exist integral currents $U$ with
\[
\partial U=\partial T^{\mathrm{raw}}
\qquad\text{and}\qquad
\Mass(U)=o(m).
\]
\end{corollary}

\begin{proof}[Proof sketch (standard flat-norm decomposition + isoperimetric filling)]
By definition of $\F$, there exist integral currents $R,Q$ with
$\partial T^{\mathrm{raw}}=R+\partial Q$ and $\Mass(R)+\Mass(Q)\le 2\F(\partial T^{\mathrm{raw}})$.
Since $\partial(\partial T^{\mathrm{raw}})=0$ we have $\partial R=0$, so $R$ is a cycle.
By the Federer--Fleming isoperimetric inequality in dimension $(k-1)$ there exists an integral filling $Q_R$ with
$\partial Q_R=R$ and $\Mass(Q_R)\le C\,\Mass(R)^{\frac{k}{k-1}}$.
Set $U:=-(Q+Q_R)$, so $\partial U=-\partial T^{\mathrm{raw}}$ and
\[
\Mass(U)\ \le\ \Mass(Q)+\Mass(Q_R)\ \le\ 2\F(\partial T^{\mathrm{raw}})
\ +\ C\,(2\F(\partial T^{\mathrm{raw}}))^{\frac{k}{k-1}}.
\]
If $\F(\partial T^{\mathrm{raw}})=o(m)$ then the right-hand side is $o(m)$.
\end{proof}

\section{Verification of assumptions (A)--(C) from the holomorphic corner-exit construction}

This section records where the assumptions \textnormal{(A)}--\textnormal{(C)} in
Theorem~\ref{thm:glue-gap-standalone} are supplied in the main manuscript
\texttt{hodge-SAVE-dec-12-handoff.tex}.

\medskip\noindent
\textbf{(A) Displacement $\Delta_F\lesssim h^2$.}
In the manuscript, this is exactly the content of the pointwise displacement lemma
\texttt{lem:face-displacement}:
if two neighboring cubes use the \emph{same} ordered transverse template
$\{y_a\}\subset B_{C_0h}(0)\subset\R^{2p}$ and their face parameterizations differ by
$O(h)$ in operator norm, then the induced parameters satisfy the index-wise bound
$\|u_a-u'_a\|\le C\,h^2$.
This is Lemma \texttt{lem:face-displacement} in \texttt{hodge-SAVE-dec-12-handoff.tex}.
The fact that adjacent cubes use the \emph{same} ordered template (so the matching is index-wise/prefix-wise) is
part of the global prefix-template organization in \texttt{thm:sliver-mass-matching-on-template} and is packaged
across all direction labels by \texttt{prop:global-coherence-all-labels} (see also \texttt{rem:vertex-star-coherence} for
the vertex-star holomorphic realization that keeps one template coherent across all cubes incident to a vertex).
Combined with the weighted face inequality \texttt{prop:transport-flat-glue-weighted} and
\texttt{cor:global-flat-weighted}, it yields the uniform face displacement hypothesis used here.

\medskip\noindent
\textbf{(B) Piece count per cell: $|\mathcal S(Q)|\lesssim \varepsilon^{-2p}$.}
In the manuscript, disjointness of slivers in a fixed direction family is enforced by
transverse separation $\gtrsim \varepsilon h$ (see the disjointness persistence statement
\texttt{lem:sliver-stability(ii)}), and the resulting packing bound is recorded as
\texttt{lem:sliver-packing} in \texttt{hodge-SAVE-dec-12-handoff.tex},
which gives $N_Q\le C(n,p)\,\varepsilon^{-2p}$ disjoint translates (hence sliver graphs) in a
cell of diameter $h$.
This is the source of the bound $|\mathcal S(Q)|\lesssim \varepsilon^{-2p}$ used in the global
H\"older step.

\medskip\noindent
\textbf{(C) Total mass scale: $M_Q\asymp m h^{2n}$ and $\sum_Q M_Q\asymp m$.}
In the manuscript, the local sliver manufacturing + mass-budget matching is packaged as:
\begin{itemize}
\item \texttt{prop:holomorphic-corner-exit-L1} (local existence of holomorphic corner-exit slivers with controlled geometry),
\item \texttt{prop:vertex-template-mass-matching} (cellwise mass-budget matching, i.e.\ $\sum_{a\le N_Q}\Mass([Y^{Q,a}]\llcorner Q)=M_Q+o(M_Q)$), and
\item \texttt{thm:sliver-mass-matching-on-template} together with \texttt{prop:global-coherence-all-labels} (global organization across all direction labels and the
prefix activation scheme).
\end{itemize}
The target mass budget $M_Q$ is defined from the smooth density
$m\,\beta\wedge\psi$ on each cell (so $M_Q\sim m\int_Q \beta\wedge\psi$), hence for a mesh of size $h$ one has
$M_Q\asymp m h^{2n}$ (up to $O(h)$ variation of the smooth density) and summing over all $\asymp h^{-2n}$ cells gives $\sum_Q M_Q\asymp m$.
The construction of the raw cycle from these local budgets is the content of \texttt{thm:local-sheets} (local multi-sheet manufacturing) and
\texttt{thm:global-cohom} (global cohomology quantization/gluing set-up).
This is the mass-scaling input used in the final summation in Theorem~\ref{thm:glue-gap-standalone}.

\medskip\noindent
\textbf{End-to-end conclusion in the manuscript.}
With \textnormal{(A)}--\textnormal{(C)} verified as above, the manuscript’s weighted summation estimate
\texttt{cor:global-flat-weighted} plus the scaling computation \texttt{rem:weighted-scaling} yield
\(
\F(\partial T^{\mathrm{raw}})=o(m)
\),
which is exactly the microstructure/gluing bound recorded in \texttt{rem:glue-gap}.

\end{document}


